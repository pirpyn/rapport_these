chapter intro

$1 probleme exterieur + interieur

$2 probleme exterieur + calderon

$3 cas simples : l'operateur devient un multiplcateur de Fourier kx ky/ serie de bessel n, multiplicateur de Fourier kz / serie de Mie n m

$4 probleme exterieur + cioe
3.1 ci0
3.2 ...
"Il est bien entendue que pour chaque CIOE, le problème avec une incidente, conduit s'il y a une unicté a une unique solution differente, e tdonc donc dessolutions different, dont on espere qu'elles sont une approximation du probleme complet / calderon"

Tout le pb est de choisir les coeffs complexes, obtenues grâce à des approximations issues de critères physiques mais non issus de cette derniere. 

Il est inutile de considerer des coefficients qui ne donnerait pas d'unique solution des champs

Donc interet immediat des CSU

$5 CSU <= RE(X)>=0 (1.1.3). CIOE + CSU
    Reilich
5.1 1.1.3 => 1.1.1 1.1.2 a  unique solution = 0
5.2 CSU J => 1.1.3 => PJ (1.1.1 1.1.2 + CIOE) a unique solution 0

/!\ Un operateur / Des CIOE

