\chapter{Expression de l'opérateur pseudo-différentiel d'impédance}
%\citationChap{L'infini c'est long, surtout vers la fin}{Inconnu}
\minitoc
\begin{TODO}
    On explicite le symbole sur les coordonnées tangentielles. Or on a des champs 3D. Est-ce bien équivalent ?
\end{TODO}
\section{Analyse de Fourier de l'opérateur d'impédance}

Des résultat sur l'analyse spectrale de l'opérateur d'impédance ont déjà été présenté par \cite{hoppe_impedance_1995} pour des conducteurs plans et des cylindriques recouvert d'une couche de matériau.
Nous les rappellerons ainsi la méthode pour les obtenir puis nous expliciterons ces derniers à des matériaux multi-couches sur les même objets

% \TODO{
%     Généraliser ce qui suit à des matériaux non borné à l'aide de fonctions rapidement décroissantes
% }

Soit \(\OO\) un domaine fermé, bornée, de frontière régulière. Supposons que ces champs soient \(L^2\) en espace et en temps: \((\vE(\vect{x},t),\vH(\vect{x},t)) \in L^2(\OO \times \RR_+) \cap L^2(\OO^c\times\RR_+)\) et vérifient les équations de Maxwell:
\begin{equation}
    \left\lbrace 
    \begin{matrix}
    \vrot \vE(\vect{x},t) &=& -\mu(\vx) \ddr{t}{\vH}(\vx,t) \\
    \vrot \vH(\vect{x},t) &=& \eps(\vx) \ddr{t}{\vE}(\vx,t)
    \end{matrix}
    \right.
\end{equation}

On suppose aussi que \(\eps\) et \(\mu\) sont constant par morceaux.

Puisque ces champs \(\vE(\vect{x},t),\vH(\vect{x},t)\) sont \(L^2\), on peut définir leurs transformées de Fourier \(\hat{\vE}(\vect{k},\w), \hat{\vH}(\vect{k},\w)\) (\cite[Théorème de Plancherel, p.~153]{yosida_functional_1995}) telle que

\begin{equation}
    \hat{\vE} (\vect{k},\w) = \frac{1}{\sqrt{2\pi}^3}\int_{\RR^3\times\RR_+} e^{-i(\vect{k} \cdot \vect{x}+\w t)}\vE(\vect{x},t) \dd{x}\dd{t}\,, \quad \dd x = \prod\limits_{i=1}^3 \dd{x}_i
\end{equation}
\begin{equation}
    \vE(\vect{x},t) = \frac{1}{\sqrt{2\pi}^3}\int_{\RR^3\times\RR_+} e^{i(\vect{k} \cdot \vect{x}+\w t)}\hat{\vE} (\vect{k},\w) \dd{k} \dd{\w}\,, \quad \dd k = \prod\limits_{i=1}^3 \dd{k}_i
\end{equation}
La méthode pour trouver une expression de l'opérateur d'impédance est la suivante.
\begin{itemize}
\item Faire une transformée de Fourier partielle des champs, dépendante de la géométrie.
\item Réécrire le système d'équation de Maxwell simplifié.
\item Obtenir des EDO simple à résoudre sur les composantes des champs.
\item Utiliser les conditions limites pour obtenir les solutions particulières de cette EDO. 
\item En déduire l'opérateur d'impédance en Fourier.
\end{itemize}

Dans ce qui suit, on réalise au moins une transformée partielle en temps. La variable de Fourier associée est \(\w\), et donc l'opérateur \(\ddr{t}{~}\) est remplacé par \(i\w\).

% \begin{tcolorbox}
% On ne différencie pas les champs \(\vE,\vH\) de leurs transformées de Fourier \(\hat \vE, \hat \vH\).
% \end{tcolorbox}

On va donc utiliser le système d'équations de Maxwell harmonique:
\begin{equation}
    \left\lbrace 
    \begin{matrix}
    \vrot \hat \vE(\vx,\w)  &=& -i \omega \mu \hat \vH(\vx,\w)  \\
    \vrot \hat \vH(\vx,\w)  &=& i \omega \eps \hat \vE(\vx,\w) 
    \end{matrix}
    \right.
    \label{eq:imp_fourier:intro:maxwell_harmonique}
\end{equation}

A partir de maintenant, la dépendance en \(\w\) sera implicite: \(\hat \vE(\vx) \equiv \hat \vE (\vx, \w)\).
\section{Cas d'un objet plan infini}
    % Ce cas est très bien documenté (\cite{senior_approximate_x995},\cite{hoppe_impedance_x995}) et pose la méthodologie à adopter pour les objets courbes.

\begin{TODO}
    Corriger problème de signe
\end{TODO}

    \begin{figure}[!h]
        \begin{center}
            \begin{tikzpicture}
                \tikzmath{
    \largeur = 6;
    \hauteur = 1;
    \milieu = 1.3;
    \xC = \largeur;
    \xA = 0;
}

%% 1ere couche
\tikzmath{
    \yC = \hauteur;
    \yA = 0;
}

\coordinate (A) at (\xA,\yA);
\coordinate (B) at (\xA,\yC);
\coordinate (C) at (\xC,\yC);

\draw ($(B)!0.5!(C)$) node [above] {vide};


\fill [lightgray] (A) rectangle (C);
\draw ($(A)!0.5!(C)$) node {$\peps,\pmu,d$};
\draw (B) -- (C) node [right] {$\z = 0$};

%% Le repère
\tikzmath{
    \xD = \xC + 1.5;
}

\coordinate (n) at (\xD,\yA);

\draw [->] (n) -- ++(0,1) node [at end, right] {$\v{\z}$};
\draw [->] (n) -- ++(1,0) node [at end, right] {$\v{\x}$};

\draw (n) circle(0.1cm) node [below=0.1cm] {$\v{\y}$};
\draw (n) +(135:0.1cm) -- +(315:0.1cm);
\draw (n) +(45:0.1cm) -- +(225:0.1cm);

%% Le conducteur
\tikzmath{
    \yC = \yC - \hauteur;
    \yA = \yA - 0.5*\hauteur;
}

\coordinate (A) at (\xA,\yA);
\coordinate (B) at (\xA,\yC);
\coordinate (C) at (\xC,\yC);
\draw (B) -- (C);

\fill [pattern=north east lines] (A) rectangle (C);



            \end{tikzpicture}
        \end{center}
    \end{figure}

    Dans un premier temps, on peut sans perte de généralités faire une rotation du repère pour avoir le plan orthogonal à \(\vect{z}\). Comme il est infini dans les directions \(\vect{e_x} \vect{e_y}\) et que le matériau est homogène isotrope, on utilise la transformée partielle en \(x, y\) seulement.
    \begin{equation}
        \vE(x,y,z) = \frac{1}{2\pi}\iint_{\RR^2} e^{i(k_x x + k_y y)}\hat{\vE} (k_x,k_y,z) \dd{k_x}\dd{k_y}
    \end{equation}

    \begin{prop}
        Soient \(\vect{X}(k_x,k_z,z) =
        \begin{bmatrix}
        \hat{E_x} &
        \hat{E_y} &
        \hat{H_x} &
        \hat{H_y}
        \end{bmatrix}^t\),
        où \((\hat \vE,\hat \vH)\) sont des solutions du problème \eqref{eq:imp_fourier:intro:maxwell_harmonique}, alors il existe une matrice \(\mat{M}\) ne dépendant pas de \(z\) telle que
        \begin{equation}
            \ddr{z}{}\vect{X}(k_x,k_z,z) = \mat{M}(k_x,k_z) \vect{X}(k_x,k_z,z)
            \label{eq:imp_fourier:plan:edo}
        \end{equation}
    \end{prop}

    \begin{proof}
        En utilisant les multiplicateur de Fourier associés aux opérateurs différentiels, le problème \eqref{eq:imp_fourier:intro:maxwell_harmonique} s'écrit
        \begin{align*}
            \left\lbrace
            \begin{matrix}
            ik_y \hat{E_z}  - \ddr{z}{\hat{E_y}} = -i \w \mu \hat{H_x} \\
            \ddr{z}{\hat{E_x}} - ik_x \hat{E_z} = -i\w \mu \hat{H_y} \\
            ik_x \hat{E_y} - ik_y \hat{E_x} = -i\w \mu \hat{H_z} \\
            \end{matrix}
            \right. \quad
            \left\lbrace
            \begin{matrix}
            ik_y \hat{H_z}  - \ddr{z}{\hat{H_y}} = i \w \eps \hat{E_x} \\
            \ddr{z}{\hat{H_x}} - ik_x \hat{H_z} = i\w \eps \hat{E_y} \\
            ik_x \hat{H_y} - ik_y \hat{H_x} = i\w \eps \hat{E_z} \\
            \end{matrix}
            \right.
        \end{align*}

        Les composantes normales se déduisant des composantes tangentielles, on résout l'équation différentielle  matricielle à coefficients constants
        suivante \(\ddr{z}{}\vect{X} = \mat{M} \vect{X}\) où

        \begin{equation}
            \mat{M} = \begin{bmatrix}
            0 & 0 & -i\frac{k_xk_y}{\w\eps} & -i\left(\w\mu - \frac{k_x^2}{\w\eps}\right)\\
            0 & 0 & i\left(\w\mu - \frac{k_y^2}{\w\eps}\right) & i\frac{k_xk_y}{\w\eps}\\
            i\frac{k_xk_y}{\w\mu} & i\left(\w\eps - \frac{k_x^2}{\w\mu}\right) & 0 & 0 \\
            -i\left(\w\eps - \frac{k_y^2}{\w\mu}\right) & -i\frac{k_xk_y}{\w\mu} & 0 & 0 \\
            \end{bmatrix}
        \end{equation}
    \end{proof}

    \begin{prop}
        On définit
        \begin{equation}
            k_3=\sqrt{\w^2\eps\mu - k_x^2 -k_y^2} \quad \lambda_\pm=\pm i k_3
        \end{equation}
        \begin{equation}
            \vect{V_\pm} =
            \begin{bmatrix}
            \lambda_\pm \\
                0 \\
                -i\frac{k_xk_y}{\w\mu} \\
                i\left(\w\eps - \frac{k_y^2}{\w\mu}\right) \\
            \end{bmatrix}
            \quad
            \vect{W_\pm} =
                \begin{bmatrix}
                0 \\
                \lambda_\pm \\
                -i\left(\w\eps - \frac{k_x^2}{\w\mu}\right) \\
                i\frac{k_xk_y}{\w\mu} \\
            \end{bmatrix}
        \end{equation}
        alors
        \begin{subequations}
            \begin{align}
                \Ker(\mat{M}-\lambda_+\mI)=\Vect{\vect{V_+};\vect{W_+}}\\
                \Ker(\mat{M}-\lambda_-\mI)=\Vect{\vect{V_-};\vect{W_-}}
            \end{align}
        \end{subequations}
    \end{prop}

    \begin{proof}
        % Pour résoudre cette EDO, nous allons chercher les vecteurs propres \(V_i\) et les valeurs propres \(\lambda_i\) associées de ce système. En effet, une solution générale de ce système s'écrit
        % \begin{equation}
        %     \vect{X}(z)= \sum\limits_{i=1}^{4}c_i e^{\lambda_i z} \vect{V}_i \, c_i \in \CC
        % \end{equation}
        On pose
        \begin{equation*}
            \mat{A} = -\begin{bmatrix}
                i\frac{k_xk_y}{i\w\eps} & i\left(\w\mu - \frac{k_x^2}{\w\eps}\right) \\
                -i\left(\w\mu - \frac{k_y^2}{\w\eps}\right) & -i\frac{k_xk_y}{i\w\eps} \\
            \end{bmatrix}
            \quad
            \mat{B} = -\begin{bmatrix}
                -i\frac{k_xk_y}{\w\mu} & -i\left(\w\eps - \frac{k_x^2}{\w\mu}\right) \\
                i\left(\w\eps - \frac{k_y^2}{\w\mu}\right) & i\frac{k_xk_y}{\w\mu} \\
            \end{bmatrix}
        \end{equation*}
        Le déterminant de \(\mat{M}-\lambda \mat{I}\) est
        \begin{align*}
            \det(\mat{M}-\lambda \mat{I}) &=
            \begin{vmatrix}
                -\lambda \mI & \mA \\
                \mB & -\lambda \mI
            \end{vmatrix}
                = \frac{\det(- \lambda \mI - \mB(-\lambda \mI)^{-1} \mA)}{\det((-\lambda \mI)^{-1})} \\
                &= \det(\lambda^2 \mI - \mB\mA) \\
                &= (\lambda^2 + (\w^2\eps\mu - k_x^2 -k_y^2))^2 \\
                &= (\lambda^2 - \lambda_\pm^2)^2
        \end{align*}
        % On note alors
        % \begin{equation}
        % k_3=\sqrt{\w^2\eps\mu - k_x^2 -k_y^2}
        % \end{equation}

        % Les valeurs propres sont alors
        % \begin{equation}
        %     \lambda_\pm = \pm i k_3
        % \end{equation}
        % Les espaces propres associés sont de dimension 2, on a

        Par un calcul immédiat, on vérifie que \(\mat{M}\vect{V}_\pm = \lambda_\pm\vect{V}_\pm\) et \(\mat{M}\vect{W}_\pm = \lambda_\pm\vect{W}_\pm\).
    \end{proof}

    \begin{prop}
        On pose
        \begin{align}
            \mC &=
            \begin{bmatrix}
                \left(\w\eps-\frac{k_y^2}{\w\mu}\right) & \frac{k_xk_y}{\w\mu}\\
                \frac{k_xk_y}{\w\mu} & \left(\w\eps-\frac{k_x^2}{\w\mu}\right)
            \end{bmatrix}
        \end{align}
        alors
        \begin{subequations}
            \label{eq:imp_fourier:plan:champs}
            \begin{align}
                \begin{bmatrix}
                    \hat{E_x}(k_x,k_y,z)\\
                    \hat{E_y}(k_x,k_y,z)\\
                \end{bmatrix}
                &=ik_3\left( e^{ik_3 z}
                \begin{bmatrix}
                    c_1 \\
                    c_2
                \end{bmatrix}
                -e^{-ik_3 z}
                \begin{bmatrix}
                    c_3 \\
                    c_4
                \end{bmatrix}
                \right)
                \label{eq:imp_fourier:plan:champs:E}
                \\
                \begin{bmatrix}
                    -\hat{H_y}(k_x,k_y,z)\\
                    \hat{H_x}(k_x,k_y,z)\\
                \end{bmatrix}
                &=-i
                \mC
                \left(
                    e^{ik_3 z}
                    \begin{bmatrix}
                        c_1 \\
                        c_2
                    \end{bmatrix}
                    +e^{-ik_3 z}
                    \begin{bmatrix}
                        c_3 \\
                        c_4
                    \end{bmatrix}
                \right)
                \label{eq:imp_fourier:plan:champs:H}
            \end{align}
        \end{subequations}
    \end{prop}

    \begin{proof}
        On déduit des vecteur propres une solution générale de \eqref{eq:imp_fourier:plan:edo}.

        Soient \((c_i)_{i} \in \CC^4\)
        \begin{equation}
            \vect{X}(k_x,k_y,z) = c_1e^{\lambda_+ z}\vect{V_+}  + c_2e^{\lambda_+ z}\vect{W_+} + c_3e^{\lambda_- z}\vect{V_-} +c_4e^{\lambda_- z}\vect{W_-}
        \end{equation}

        On exprime \(\hat \vE_t(k_x,k_y,z)\) et \(\vect{e_z} \pvect \hat \vH_t(k_x,k_y,z)\)
        \begin{align}
            \begin{bmatrix}
                \hat{E_x}(k_x,k_y,z)\\
                \hat{E_y}(k_x,k_y,z)\\
            \end{bmatrix}
            &=
            \begin{bmatrix}
                c_1 e^{\lambda_+ z} \lambda_{+} + c_3 e^{\lambda_- z} \lambda_{-} \\
                c_2 e^{\lambda_+ z} \lambda_{+} + c_4 e^{\lambda_- z} \lambda_{-}
            \end{bmatrix}\\
            &=ik_3\left( e^{ik_3 z}
            \begin{bmatrix}
                c_1 \\
                c_2
            \end{bmatrix}
            -e^{-ik_3 z}
            \begin{bmatrix}
                c_3 \\
                c_4
            \end{bmatrix}
            \right)
        \end{align}

        \begin{align}
            \begin{bmatrix}
                -\hat{H_y}(k_x,k_y,z)\\
                \hat{H_x}(k_x,k_y,z)\\
            \end{bmatrix}
            &=
            \begin{bmatrix}
                -i\left(\w\eps - \frac{k_y^2}{\w\mu}\right) \left( c_1 e^{ik_3 z} + c_3 e^{-ik_3 z} \right) - i\frac{k_xk_y}{\w\mu} \left( c_2 e^{ik_3 z} + c_4 e^{-ik_3 z} \right)
                \\
                -i\frac{k_xk_y}{\w\mu} \left( c_1 e^{ik_3 z} + c_3 e^{-ik_3 z} \right) - i\left(\w\eps - \frac{k_x^2}{\w\mu}\right)\left( c_2 e^{ik_3 z} + c_4 e^{-ik_3 z} \right)
            \end{bmatrix} \\
            &=-i
            \mC
            \left(
                e^{ik_3 z}
                \begin{bmatrix}
                    c_1 \\
                    c_2
                \end{bmatrix}
                +e^{-ik_3 z}
                \begin{bmatrix}
                    c_3 \\
                    c_4
                \end{bmatrix}
            \right)
        \end{align}

        Comme \(\det(\mC) = k_3^2\frac{\eps}{\mu}=\frac{k_3^2}{\eta^2}\) alors une condition nécessaire pour trouver l'opérateur d'impédance est que \(k_3\) soit non nul ce qui aussi une hypothèse à vérifier car les valeurs propres doivent être non nulles.\footnote{\(k_3\) peut s'annuler pour des \(\eps,\mu\) réels.}.

    \end{proof}
    % On peut noter d'après \cite[eq.~(6)]{stupfel_2011}

    % \begin{equation}
    %     \mC^{-1}= \frac{\eta^2}{k_3^2}\left(k^2\mI - \mat{L_R}\right)
    %     \label{eq:imp_fourier:plan:C}
    % \end{equation}


    %%%%%%%%%%%%%%%%%%%%%%%%%%%%%%%%%%%%%%%%%%%%%%%%%%%%%%%%%%%%%%%%%%%%%%%
    %%%%%%%%%%%%%%%%%%%%%%%%%%%%%%%%%%%%%%%%%%%%%%%%%%%%%%%%%%%%%%%%%%%%%%%
    %%%%%%%%%%%%%%%%%%%%%%%%%%%%%%%%%%%%%%%%%%%%%%%%%%%%%%%%%%%%%%%%%%%%%%%


    \subsection{Opérateur d'impédance pour une couche}

        \begin{defn}
            On définit le symbole de l'opérateur d'impédance, la matrice \(\hat \mZ(k_x,k_y)\) tel que
            \begin{equation}
                \hat \vE_t(k_x,k_y,0) = \hat \mZ(k_x,k_y) \left(\vect{e_z} \pvect \hat \vH_t(k_x,k_y,0)\right)
            \end{equation}
        \end{defn}

        \begin{thm}
            Supposons que
            \begin{subequations}
                \label{eq:imp_fourier:plan:hyp_1_c}
                \begin{align}
                    k_3     & \not =0 \\
                    k_3d    & \not = \frac{\pi}{2}+n\pi\,, \forall n \in \NN
                \end{align}
            \end{subequations}

            Alors
            \begin{align}
            \label{eq:imp_plan:symb_z:1c}
            \hat \mZ(k_x,k_y) &= -i\eta\frac{\tan\left(k_3d\right)}{kk_3}
                \begin{bmatrix}
                   k^2-k_x^2  & -k_xk_y\\
                    -k_xk_y & k^2-k_y^2\\
                \end{bmatrix}
            \end{align}
        \end{thm}

        \begin{proof}
            Nous utilisons la condition limite
            \begin{equation}
                \begin{bmatrix}
                    \hat{E_x}(k_x,k_y,-d)\\
                    \hat{E_y}(k_x,k_y,-d)\\
                \end{bmatrix}
                =
                \begin{bmatrix}
                    0\\
                    0\\
                \end{bmatrix}
            \end{equation}

            De \eqref{eq:imp_fourier:plan:champs:E}, on déduit
            \begin{align}
                \begin{bmatrix}
                    c_1 \\
                    c_2
                \end{bmatrix}
                = e^{2ik_3 d}
                \begin{bmatrix}
                    c_3 \\
                    c_4
                \end{bmatrix}
            \end{align}

            En injectant ce qui précède dans \eqref{eq:imp_fourier:plan:champs}, on déduit que
            \begin{align}
                \begin{bmatrix}
                    \hat{E_x}(k_x,k_y,0)\\
                    \hat{E_y}(k_x,k_y,0)\\
                \end{bmatrix}
                &=ik_3\left( e^{i2k_3 d} -1 \right)
                \begin{bmatrix}
                    c_3 \\
                    c_4
                \end{bmatrix} \\
                \begin{bmatrix}
                    -\hat{H_y}(k_x,k_y,0)\\
                    \hat{H_x}(k_x,k_y,0)\\
                \end{bmatrix}
                & = - i\left(e^{i2k_3 d} +1 \right)
                \mC
                \begin{bmatrix}
                c_3 \\
                c_4
                \end{bmatrix}
            \end{align}

            En supposant \(k_3d \not = \frac{\pi}{2} + n\pi\), on déduit donc que
            \begin{align}
                \hat \mZ(k_x,k_y) &=  - k_3 \frac{e^{i2k_3d} -1}{e^{i2k_3d} +1} \mC^{-1}
                \\
                &= -\frac{\eta^2}{k_3} \frac{e^{i2k_3d} -1}{e^{i2k_3d} +1}
                    \begin{bmatrix}
                       \left(\w\eps-\frac{k_x^2}{\w\mu}\right)  & -\frac{k_xk_y}{\w\mu}\\
                        -\frac{k_xk_y}{\w\mu} &  \left(\w\eps-\frac{k_y^2}{\w\mu}\right)
                    \end{bmatrix}
                \\
                &= -i\eta\frac{\tan\left(k_3d\right)}{kk_3}
                    \begin{bmatrix}
                       k^2-k_x^2  & -k_xk_y\\
                        -k_xk_y & k^2-k_y^2\\
                    \end{bmatrix}
            \end{align}

        \end{proof}
        %On remarque que \(\det(\mat{Z}) = i\frac{\eta^2}{k_3}\eta\tan(k_3d)\) et donc pour un matériau \((\eps,\mu,d)\) donné, l'opérateur d'impédance n'est pas inversible pour tous  \((k_x,k_y) \in \RR^2, n \in \NN\), \(k_x^2+k_y^2 =  \w^2\eps\mu - \frac{1}{d^2}\left(\frac{\pi}{2} + n\pi\right)^2\), qui ne peut être vérifié que si \(\eps\mu\) est réel\footnote{Comme \(\eps, \mu\) sont à partie réelle (resp. imaginaire) strictement positive (resp. négative), alors ce n'est vrai pour les matériaux à partie imaginaire nulle.}.

        En pratique, on néglige toute les dépendance en \(y\) : \(k_y = 0\) ce qui revient à une propagation dans le plan \(xz\). Grâce à cette hypothèse, on trouve que \(\mC, \hat \mZ\) sont des matrices diagonales.

        De plus, on exprime souvent l'impédance selon la polarisation.
        Dans le cas plan, le champ \(\vE\)-TE correspond à \({E_y} \vect{e_y}\), le champ \(\vE\)-TM à \({E_x} \vect{e_x}\), tandis que le champ \(\vH\)-TM correspond à \({H_x} \vect{e_x}\) et le champ \(\vH\)-TE correspond à \({H_y} \vect{e_y}\).
        Dans ce cas, le symbole \(\hat \mZ\) peut se réécrire comme
        \begin{equation}
            \hat \mZ =
            \begin{bmatrix}
                \hat Z_{TM} & 0
                \\
                0 & \hat Z_{TE}
            \end{bmatrix}
        \end{equation}

    %%%%%%%%%%%%%%%%%%%%%%%%%%%%%%%%%%%%%%%%%%%%%%%%%%%%%%%%%%%%%%%%%%%%%%%
    %%%%%%%%%%%%%%%%%%%%%%%%%%%%%%%%%%%%%%%%%%%%%%%%%%%%%%%%%%%%%%%%%%%%%%%
    %%%%%%%%%%%%%%%%%%%%%%%%%%%%%%%%%%%%%%%%%%%%%%%%%%%%%%%%%%%%%%%%%%%%%%%

    \subsection{Opérateur d'impédance pour plusieurs couches}
        On suppose que l'on a \(n\) couches de matériaux :

        \begin{figure}[h!btp]
            \centering
            \begin{tikzpicture}
                \tikzmath{
    \largeur = 6;
    \hauteur = 0.5;
    \milieu = 1.3;
    \xC = \largeur;
    \xA = 0;
}

%% 1ere couche
\tikzmath{
    \yC = \hauteur;
    \yA = 0;
}

\coordinate (A) at (\xA,\yA);
\coordinate (B) at (\xA,\yC);
\coordinate (C) at (\xC,\yC);

\draw ($(B)!0.5!(C)$) node [above] {vide};


\fill [lightgray] (A) rectangle (C);
\draw ($(A)!0.5!(C)$) node {$\eps_n,\mu_n,d_n$};
\draw (B) -- (C) node [right] {$e_3 = 0$};

%% Des couches
\tikzmath{
    \yC = \yC - \hauteur;
    \yA = \yA - \milieu*\hauteur;
}

\coordinate (A) at (\xA,\yA);
\coordinate (B) at (\xA,\yC);
\coordinate (C) at (\xC,\yC);

\fill [lightgray]    (A) rectangle (C);
\fill [pattern=dots] (A) rectangle (C);
\draw (B) -- (C);

%% N ieme couche
\tikzmath{
    \yC = \yC - \milieu*\hauteur;
    \yA = \yA - \hauteur;
}

\coordinate (A) at (\xA,\yA);
\coordinate (B) at (\xA,\yC);
\coordinate (C) at (\xC,\yC);
\fill [lightgray] (A) rectangle (C);
\draw ($(A)!0.5!(C)$) node {$\eps_1,\mu_1,d_1$};
\draw (B) -- (C);

%% Le repère
\tikzmath{
    \xD = \xC + 0.5;
}

\coordinate (n) at (\xD,\yA);
\draw [->] (n) -- ++(1,0) node [at end, right] {$\v{e_1}$};
\draw [->] (n) -- ++(0,1) node [at end, right] {$\v{e_3}$};

\draw (n) circle(0.1cm) node [below=0.1cm] {$\v{e_2}$};
\draw (n) +(135:0.1cm) -- +(315:0.1cm);
\draw (n) +(45:0.1cm) -- +(225:0.1cm);

%% Le conducteur
\tikzmath{
    \yC = \yC - \hauteur;
    \yA = \yA - 0.5*\hauteur;
}

\coordinate (A) at (\xA,\yA);
\coordinate (B) at (\xA,\yC);
\coordinate (C) at (\xC,\yC);
\draw (B) -- (C);

\fill [pattern=north east lines] (A) rectangle (C);



            \end{tikzpicture}
        \end{figure}

        Pour chaque couche caractérisée par \((\eps_m,\mu_m,d_m)\), on définit:
        \begin{align}
        k_{3m} &= \sqrt{w^2\eps_m\mu_m - k_y^2 - k_x^2}
        \\
        \mC_m &=
            \begin{bmatrix}
                \left(\w\eps_m-\frac{k_y^2}{\w\mu_m}\right) & \frac{k_xk_y}{\w\mu_m}\\
                \frac{k_xk_y}{\w\mu_m} & \left(\w\eps_m-\frac{k_x^2}{\w\mu_m}\right)
            \end{bmatrix}
        \end{align}

        On rappelle que \(\det{\mC_m} = \frac{k_{3m}\eps_m^2}{\mu_m^2}\)

        On définit aussi la profondeur de la couche \(m\), \(l_m = -\sum_{i=0}^{n-m} d_{n-i} \).

        \begin{defn}
            On définit pour chaque interface, le symbole \(\hat \mZ_m\) tel que
            \begin{equation}
                \hat \vE_t(k_x,k_y,l_m) = \hat \mZ_m(k_x,k_y) \left(\vect{e_z} \pvect \hat \vH_t(k_x,k_y,l_m)\right)
            \end{equation}
        \end{defn}

        \begin{thm}
            Soit \(\hat \mZ_0(k_x,k_y) = \mat{0}_{\mathcal{M}_2(\CC)}\).

            Si pour tout \(0<m < n\)
            \begin{align}
                k_{3m} &\not = 0 \\
                \det\left(k_{3m}\mI \pm \hat \mZ_{m-1}\mC_m \right) &\not = 0 \\
                k_{3m}d_m &\not = \frac{\pi}{2}+n\pi\,, \forall n \in \NN \\
                \det\left(k_{3m}\mI - i\tan(k_{3m}d_m)\mZ_{m-1}\mC_m\right) &\not = 0
            \end{align}
            Alors le symbole \(\hat \mZ_n\) est défini par la relation de récurrence :
            \begin{multline}
                \hat \mZ_m = -k_{3m}
                \left(ik_{3m}\tan\left(k_{3m}d_m\right)\mI - \hat \mZ_{m-1}\mC_m\right) \\
                \left(k_{3m}\mI - i\tan\left(k_{3m}d_m\right)\hat \mZ_{m-1}\mC_m\right)^{-1}
                \mC_m^{-1}
            \end{multline}
        \end{thm}

        \begin{proof}
            À l'initialisation, la condition limite sur le conducteur impose \(\hat \mZ_0 = \mat{0}_{\mathcal{M}_2(\CC)}\) et on retrouve le résultat pour une couche.

            Par récurrence, un empilement à \(n\) couches se ramène à un empilement à une couche avec la condition:
            \begin{equation}
                \begin{bmatrix}
                    \hat{E_x}(k_x,k_y,-d)\\
                    \hat{E_y}(k_x,k_y,-d)\\
                \end{bmatrix}
                =
                \mZ_m
                \begin{bmatrix}
                    -\hat{H_y}(k_x,k_y,-d)\\
                    \hat{H_x}(k_x,k_y,-d)\\
                \end{bmatrix}
            \end{equation}

            On reprend donc tous les résultats de la partie précédente. Notamment, de \eqref{eq:imp_fourier:plan:champs:E} et \eqref{eq:imp_fourier:plan:champs:H}, on déduit que

            \begin{equation}
                \begin{bmatrix}
                    \hat{E_x}(k_x,k_y,-d)\\
                    \hat{E_y}(k_x,k_y,-d)\\
                \end{bmatrix}
                = ik_3\left( e^{-ik_3 d}
                \begin{bmatrix}
                    c_1 \\
                    c_2
                \end{bmatrix}
                -e^{ik_3 d}
                \begin{bmatrix}
                    c_3 \\
                    c_4
                \end{bmatrix}
                \right)
            \end{equation}

            \begin{equation}
                \begin{bmatrix}
                    -\hat{H_y}(k_x,k_y,-d)\\
                    \hat{H_x}(k_x,k_y,-d)\\
                \end{bmatrix}
                =-i
                \mC
                \left(
                    e^{-ik_3 d}
                    \begin{bmatrix}
                        c_1 \\
                        c_2
                    \end{bmatrix}
                    +e^{ik_3 d}
                    \begin{bmatrix}
                        c_3 \\
                        c_4
                    \end{bmatrix}
                \right)
            \end{equation}

            \begin{equation}
                ik_3\left( e^{-ik_3 d}
                \begin{bmatrix}
                    c_1 \\
                    c_2
                \end{bmatrix}
                -e^{ik_3 d}
                \begin{bmatrix}
                    c_3 \\
                    c_4
                \end{bmatrix}
                \right)
                =-i\hat\mZ_m\mC
                \left(
                    e^{-ik_3 d}
                    \begin{bmatrix}
                        c_1 \\
                        c_2
                    \end{bmatrix}
                    +e^{ik_3 d}
                    \begin{bmatrix}
                        c_3 \\
                        c_4
                    \end{bmatrix}
                \right)
            \end{equation}

            \begin{equation}
                \left(k_3\mI + \hat\mZ_m\mC\right)
                \begin{bmatrix}
                    c_1 \\
                    c_2
                \end{bmatrix}
                = e^{i2k_3 d} \left(k_3\mI - \hat\mZ_m\mC\right)
                \begin{bmatrix}
                    c_3 \\
                    c_4
                \end{bmatrix}
            \end{equation}

            On pose
            \begin{align}
                \mA_\pm &= k_3\mI \pm \hat\mZ_m\mC
            \end{align}

            On remarque que par définition, \(\mA_+\) et \(\mA_-\) commutent.

            Pour continuer il faut exprimer un vecteur en fonction de l'autre. On suppose donc \(\pm k_3\) ne sont pas des valeurs propres de \(\hat\mZ_m\mC\) et l'on déduit que

            \begin{align}
                \begin{bmatrix}
                    c_1 \\
                    c_2
                \end{bmatrix}
                &= e^{i2 k_3 d} \mA_+^{-1}\mA_-
                \begin{bmatrix}
                    c_3 \\
                    c_4
                \end{bmatrix}
                \\
                & = \mat{F}
                \begin{bmatrix}
                    c_3 \\
                    c_4
                \end{bmatrix}
            \end{align}

            \begin{align}
                \begin{bmatrix}
                    \hat{E_x}(k_x,k_y,0)\\
                    \hat{E_y}(k_x,k_y,0)\\
                \end{bmatrix}
                &=ik_3\left(\mat{F} - \mI \right)
                \begin{bmatrix}
                    c_3 \\
                    c_4
                \end{bmatrix}\\
                \begin{bmatrix}
                    -\hat{H_y}(k_x,k_y,0)\\
                    \hat{H_x}(k_x,k_y,0)\\
                \end{bmatrix}
                &=-i\mC \left(\mat{F} + \mI \right)
                \begin{bmatrix}
                        c_3 \\
                        c_4
                \end{bmatrix}
            \end{align}

            On suppose qu'en plus de \(\mA_+\) et \(\mA_-\), \(\mat{F} + \mI\) est inversible, on va utiliser la commutativité de \(\mA_+\) et \(\mA_-\).

            Alors le symbole \(\hat \mZ\) s'exprime

            \begin{align}
                \hat{\mat{Z}}_{m+1}
                &=-k_3\left(\mat{F} - \mI \right)\left(\mat{F}+ \mI \right)^{-1}\mC^{-1}
                \\
                &=-k_3\mA_+^{-1}\left(e^{i2 k_3 d}\mA_- - \mA_+ \right)\left(e^{i2 k_3 d}\mA_- + \mA_+ \right)^{-1}\mA_+\mC^{-1}
                \\
                &= -k_3\left( e^{i2 k_3 d} \mA_- -  \mA_+\right)
                \left( e^{i2 k_3 d} \mA_- + \mA_+ \right)^{-1}\mC^{-1}
                \\
                &= -k_3\left(\left( e^{i2 k_3 d} - 1 \right)\mI - \left( e^{i2 k_3 d} + 1 \right) \hat\mZ_m\mC \right)
                \left( \left( e^{i2 k_3 d} + 1 \right)\mI - \left( e^{i2 k_3 d} - 1 \right)\hat\mZ_m\mC \right)^{-1}\mC^{-1}
            \end{align}

            En supposant que \(\forall n \in \NN \,, k_3d\not = \frac{\pi}{2}+n\pi\), on a

            \begin{equation}
                \hat\mZ_{m+1} = -k_3\left(ik_3\tan(k_3 d)\mI - \hat\mZ_m\mC \right)
                    \left( k_3\mI - i\tan(k_3 d)\hat\mZ_m\mC \right)^{-1}\mC^{-1}
            \end{equation}

            % à condition que
            % \begin{align}
            %     \det\left(k_3\mI \pm \hat\mZ_m\mC \right) \not = 0 \\
            %     k_3d\not = \frac{\pi}{2}+n\pi\,, \forall n \in \NN \\
            %     \det\left(k_3\mI - i\tan(k_3d)\hat\mZ_m\mC\right) \not = 0
            % \end{align}

        \end{proof}

    %%%%%%%%%%%%%%%%%%%%%%%%%%%%%%%%%%%%%%%%%%%%%%%%%%%%%%%%%%%%%%%%%%%%%%%
    %%%%%%%%%%%%%%%%%%%%%%%%%%%%%%%%%%%%%%%%%%%%%%%%%%%%%%%%%%%%%%%%%%%%%%%
    %%%%%%%%%%%%%%%%%%%%%%%%%%%%%%%%%%%%%%%%%%%%%%%%%%%%%%%%%%%%%%%%%%%%%%%

    \subsection{Applications numériques}

        La figure \ref{fig:imp_fourier:plan:hoppe} permet de vérifier les résultats de \cite[p.~33]{hoppe_impedance_1995} pour une couche de matériau sans perte. On remarque que pour \(s=2\), \(k_3 = 0\) et donc \(\mC\) n'est pas inversible.

        \begin{figure}[!hbt]
            \centering
            \begin{tikzpicture}[scale=1]
                \begin{axis}[
                        title={},
                        ylabel={\(\Im(\hat{\mZ}(k_x,0)\)},
                        xlabel={\(k_x\slash k_0\)},
                        width=0.8\textwidth,
                        xmin=0,
                        xmax=2,
                        mark repeat=20,
                        legend pos=outer north east
                    ]
                    \legend{TM,TE}
                    \addplot [black] table [col sep=semicolon, x={s}, y={imag(z.tm)}] {tikz/csv/impedance/plan/HOPPE_33/hoppe_p33.csv};
                    \addplot [black,dashed] table [col sep=semicolon, x={s}, y={imag(z.te)}] {tikz/csv/impedance/plan/HOPPE_33/hoppe_p33.csv};
                \end{axis}
            \end{tikzpicture}
            \caption[Reproduction résultat Hoppe & Rahmat-Samii p.~33]{\(\eps = 4, \mu = 1, d=0.015\text{m}, f=1\text{GHz}\)}
            \label{fig:imp_fourier:plan:hoppe}
        \end{figure}

        La figure \ref{fig:imp_fourier:plan:soudais} permet de vérifier les résultats de \cite{soudais_3d_2017} pour une couche de matériau sans perte où \(k_3d = \frac{\pi}{2}\) pour \(k_x \simeq 0.9 k_0\).

        \begin{figure}[!hbt]
            \centering
            \begin{tikzpicture}[scale=1]
                \begin{axis}[
                        title={},
                        width=0.4\textwidth,
                        xmin=0,
                        xmax=2,
                        ylabel={\(\Im(\hat{\mZ}(k_x,0))\)},
                        xlabel={\(k_x\slash k_0\)},
                        mark repeat=20,
                        legend pos=outer north east
                    ]
                    \addplot [black] table [x={s}, y={imag(z.tm)},col sep=semicolon] {tikz/csv/impedance/plan/soudais.csv};
                    \addplot [black,dashed] table [x={s}, y={imag(z.te)},col sep=semicolon] {tikz/csv/impedance/plan/soudais.csv};
                \end{axis}
            \end{tikzpicture}
            \begin{tikzpicture}[scale=1]
                \begin{axis}[
                        title={},
                        width=0.4\textwidth,
                        ymin=-100,
                        ymax=100,
                        xmin=0.8,
                        xmax=1,
                        ylabel={},
                        xlabel={\(k_x\slash k_0\)},
                        mark repeat=20,
                        legend pos=outer north east
                    ]
                    \legend{TM,TE}
                    \addplot [black] table [x={s}, y={imag(z.tm)},col sep=semicolon] {tikz/csv/impedance/plan/soudais_zoom.csv};
                    \addplot [black,dashed] table [x={s}, y={imag(z.te)},col sep=semicolon] {tikz/csv/impedance/plan/soudais_zoom.csv};
                \end{axis}
            \end{tikzpicture}
            \caption[Reproduction résultat P. Soudais]{\(\eps = 4, \mu = 1, d=0.035\text{m}, f=12\text{GHz}\)}
            \label{fig:imp_fourier:plan:soudais}
        \end{figure}

        % \begin{figure}[!hbt]
        %     \begin{tikzpicture}
        %         \begin{axis}
        %             \addplot[%
        %                 contour prepared,%
        %                 contour prepared format=matlab%
        %             ] table {
        %                 % (0.2,6) ==> contour `0.2' (x), 5 points follow (y):
        %                 0.3000000e+00 6.0000000e+00
        %                 3.0000000e+00 4.0000000e-01
        %                 2.0000000e+00 2.8571429e-01
        %                 1.0000000e+00 3.3333333e-01
        %                 3.3333333e-01 1.0000000e+00
        %                 2.8571429e-01 2.0000000e+00
        %                 3.3333333e-01 3.0000000e+00
        %                 % (0.4,5) ==> contour `0.4', consists of 5 points
        %                 4.0000000e-01 5.0000000e+00
        %                 3.0000000e+00 8.0000000e-01
        %                 2.0000000e+00 5.7142857e-01
        %                 1.0000000e+00 6.6666667e-01
        %                 6.6666667e-01 1.0000000e+00
        %                 5.7142857e-01 2.0000000e+00
        %                 % (0.6,6) ==> contour `0.6', has 6 points
        %                 6.0000000e-01 6.0000000e+00
        %                 2.6666667e+00 2.0000000e+00
        %                 2.5000000e+00 1.0000000e+00
        %                 2.0000000e+00 8.5714286e-01
        %                 1.0000000e+00 1.0000000e+00
        %                 1.0000000e+00 1.0000000e+00
        %                 8.5714286e-01 2.0000000e+00
        %             };
        %         \end{axis}
        %     \end{tikzpicture}
        % \end{figure}

        % \def\tmpfile{graphes.txt}
        % \immediate\write18{rm -rf "\tmpfile"}
        % \immediate\write18{touch "\tmpfile"}
        % \immediate\write18{date >> "\tmpfile"}
        % \immediate\write18{gnuplot --version >> "\tmpfile"}
        % \begin{figure}[!hbt]
        %     \begin{tikzpicture}
        %         \begin{axis}
        %             \addplot3[
        %                 mesh/rows=4,
        %                 mesh/num points=12,
        %                 contour gnuplot={
        %                     number=5,
        %                     % cdata should not be affected by z filter:
        %                     output point meta=rawz,
        %                     labels=false,
        %                 }
        %             ] table [x index={1}, y index={2}, z index={3},col sep=comma] {
        %             1,-6.2832,-6.2832,0.74038,
        %             2,-6.2832,-6.1563,0.74997,
        %             3,-6.2832,-6.0293,0.75982,
        %             4,-6.2832,-5.9024,0.76986,
        %             5,-16.2832,-16.2832,10.74038,
        %             6,-16.2832,-16.1563,10.74997,
        %             7,-16.2832,-16.0293,10.75982,
        %             8,-16.2832,-15.9024,10.76986,
        %             9,-26.2832,-26.2832,20.74038,
        %             10,-26.2832,-26.1563,20.74997,
        %             11,-26.2832,-26.0293,20.75982,
        %             12,-26.2832,-25.9024,20.76986,
        %             };
        %         \end{axis}
        %     \end{tikzpicture}
        % \end{figure}
\section{Cas d'un objet cylindrique}

    % On rappelle les formules des opérateurs $\vdiv, \vrot$ en coordonnée cylindrique $(r,\theta,z)$.
    % \begin{align}
    %     \vrot \v{V} &= \left(\frac{1}{r}\ddr{\theta}{V_z} - \ddr{z}{V_\theta}\right)\v{e_r} + 
    %     \left(\ddr{z}{V_r} - \ddr{r}{V_z}\right)\v{e_\theta} +
    %     \frac{1}{r}\left(\ddr{r}{(rV_\theta)}-\ddr{\theta}{V_r}\right)\v{e_z} 
    %     \\
    %     \vdiv \v{V} &= \frac{1}{r}\ddr{r}{(rV_r)}+\frac{1}{r}\ddr{\theta}{V_\theta}+\ddr{z}{V_z}
    %     \\
    %     \vgrad f &= \ddr{r}{f}\v{e_r}
    %     +\frac{1}{r}\ddr{\theta}{f}\v{e_\theta} + \ddr{z}{f}\v{e_z}
    % \end{align}

    \begin{figure}[!hbt]
        \centering
        \begin{tikzpicture}
            \coordinate (mat) at (0,-1.5);
\coordinate (vide) at (0,-2);
\coordinate (c) at (0,0);

\fill [lightgray] (c) circle (2);
\fill [white] (c) circle (1.5);
\fill [pattern=north east lines] (c) circle (1.5);

\draw (c) circle (2);
\draw (c) circle (1.5);


\coordinate (n) at (0,2);

%\draw (vide) node [below] {$\eps_0,\mu_0$};
\draw (mat) node [below] {$\peps,\pmu$};

% Axess
\draw [->] (n) -- ++(0,1) node [at end, right] {$\v{\mr}$};
\draw [->] (n) -- ++(1,0) node [at end, right] {$\v{\mt}$};

\draw (n) ++(0.2,0.2) circle(0.1cm) node [above=0.1cm] {$\v{\mz}$};
\draw (n) ++(0.2,0.2) +(135:0.1cm) -- +(315:0.1cm);
\draw (n) ++(0.2,0.2) +(45:0.1cm) -- +(225:0.1cm);

%\draw [->>,thick] (lt) ++ (1,1) -- (mt) ;


        \end{tikzpicture}
    \end{figure}

    On exprime les équations de Maxwell dans le matériau dans la base cylindrique et sans pertes de généralité, on peut réaliser une transformée de Fourier en $z$ par invariance en translation et en $\theta$ par invariance en rotation. Cependant, le multiplicateur de Fourier associé à la coordonnée $\theta$ doit être un entier pour assurer la périodicité. On le note $n$.

    \begin{equation}
        \vE(r,\theta,z) = \frac{1}{2\pi}\sum_{i=-\infty}^{\infty}\int_{\RR} e^{i(n \theta + k_z z )}\hat{\vE} (r,n,k_z) \dd{k_z}
    \end{equation}

    \begin{prop}
        Soit
        \begin{equation}
            k_3 = \sqrt{\w^2\eps\mu - k_z^2}
        \end{equation}
        et $J_n$ et $H_n^{(2)}$ des solutions de l'équation de Bessel d'ordre $n$.
        
        Alors $\exists (c_1,c_2,c_3,c_4) \in \CC^4$ tels que la 3\ieme composante des champs soit
        \begin{subequations}
            \begin{align}
                \hat{E_z}(r,n,k_z) &= c_1 J_n\left(k_3r\right) + c_2 H_n^{(2)}\left(k_3r\right)
                \\
                \hat{H_z}(r,n,k_z) &= c_3 J_n\left(k_3r\right) + c_4 H_n^{(2)}\left(k_3r\right)
            \end{align}
        \end{subequations}
    \end{prop}

    \begin{proof}

        On peut simplifier les opérateurs différentiels:

        \begin{align}
            \vrot \hat \vE(r,n,k_z) &= i\left(\frac{n}{r}\hat{E_z} - k_z\hat{E_\theta}\right)\v{e_r} + 
            \left(ik_z\hat{E_r} - \ddr{r}{\hat{E_z}}\right)\v{e_\theta} +
            \frac{1}{r}\left(\ddr{r}{(r\hat{E_\theta})}-in\hat{E_r}\right)\v{e_z}
            \\
            &=i\w\mu \hat \vH(r,n,k_z)
        \end{align}

        On remarque que la méthode utilisée pour le plan aboutie à une équation différentielle à coefficients non constant de type $r\ddr{r}{X}(r,n,k_z) = M(r,n,k_z)X(r,n,k_z)$. On ne peut pas exprimer la solution avec les valeurs et vecteurs propres de la matrice. 
        %Nous allons donc trouver une équation de Bessel en développant le système de Maxwell.
        On développe le système de Maxwell:

        \begin{align}
            \vrot \vrot \hat \vE &= \w^2\eps\mu \hat \vE
            \\
            \vdiv \hat \vE &= 0
        \end{align}

        \begin{multline}
            \vrot \vrot \hat \vE = \dots\\
            i\left(\frac{n}{r^2}\left(\ddr{r}{(r\hat{E_\theta})} - in\hat{E_r}\right) - k_z\left(ik_z\hat{E_r} - \ddr{r}{\hat{E_z}}\right)\right)    \v{e_r} \dots\\ 
            + \left(-k_z\left(\frac{n}{r}\hat{E_z} - k_z\hat{E_\theta}\right) -\ddr{r}{}\left(\frac{1}{r}\left(\ddr{r}{(r\hat{E_\theta})}-in\hat{E_r}\right)\right)\right)    \v{e_\theta} \dots\\
            + \frac{1}{r}\left(\ddr{r}{} \left(r\left(ik_z\hat{E_r} - \ddr{r}{\hat{E_z}}\right)\right) + n \left(\frac{n}{r}\hat{E_z} - k_z\hat{E_\theta}\right)\right) \v{e_z}
        \end{multline}

        On aboutit au système suivant
        \begin{equation}
            \left\lbrace
            \begin{array}{ccc}
                -\left(\w^2\eps\mu -\frac{n^2}{r^2}  - k_z^2\right)\hat{E_r}  +i\frac{n}{r^2}\ddr{r}{(r\hat{E_\theta})}  +k_z\ddr{r}{\hat{E_z}} & = & 0\\
                in\ddr{r}{}\left(\frac{\hat{E_r}}{r}\right) -\left(\w^2\eps\mu - k_z^2\right)\hat{E_\theta} + \ddr{r}{}\left(\frac{1}{r}\ddr{r}{(r\hat{E_\theta})}\right)  - n\frac{k_z}{r}\hat{E_z} & = & 0\\
                i\frac{k_z}{r}\ddr{r}{(r\hat{E_r})}  - n\frac{k_z}{r}\hat{E_\theta}  -\left(\w^2\eps\mu - \frac{n^2}{r^2} \right)\hat{E_z} - \frac{1}{r}\ddr{r}{}\left(r\ddr{r}{\hat{E_z}}\right) & = & 0
            \end{array}
            \right.
        \end{equation}

        Comme l'on cherche $\hat \vE_t, \hat \vH_t$, on remarque que les 2\ieme composantes des équations de Maxwell permettent de déduire $\hat\vE_t, \hat\vH_t$ de $ \hat E_z, \hat H_z$

        De la troisième  équation, on trouve pour $r\not=0$
        \begin{equation}
        r^2 \ddr[2]{r}{\hat{E_z}} + r\ddr{r}{\hat{E_z}} + \left(r^2\w^2\eps\mu - n^2\right)\hat{E_z} =ik_zr\ddr{r}{(r\hat{E_r})} -  nk_zr\hat{E_\theta}
        \end{equation}

        Or comme $\vdiv \hat \vE = i\w\mu\vdiv\vrot \hat \vH = 0$, on a
        \begin{align}
            \vdiv\hat \vE &= \frac{1}{r}\ddr{r}{(r\hat{E_r})} + \frac{in}{r}\hat{E_\theta} + ik_z\hat{E_z}
            \\
            &=0
            \\
            k_z^2r^2 \hat{E_z} &= ik_zr\ddr{r}{(r\hat{E_r})} - nk_zr\hat{E_\theta}
        \end{align}

        On obtient donc sur la composante $\hat{E_z}$:
        \begin{equation}
            r^2 \ddr[2]{r}{\hat{E_z}} + r\ddr{r}{\hat{E_z}} + \left(r^2\left(\w^2\eps\mu - k_z^2\right) - n^2\right)\hat{E_z} = 0
        \end{equation}

        C'est une équation de Bessel (cf \cite[eq (6.80)]{bowman_introduction_1958}), dont des solutions générales sont: soient $(c_1,c_2) \in \CC^2$:
       \begin{equation}
            \hat{E_z}(r,n,k_z) = c_1 J_n\left(k_3r\right) + c_2 H_n^{(2)}\left(k_3r\right)
        \end{equation}
        où $J_n$ est la fonction de Bessel du premier type, $H_n^{(2)}$ la fonction de Hankel de deuxième type. On sait que l'on peut prendre n'importe quel couple de fonctions de Bessel (cf \eqref{eq:annex:bessel:equiv_bessel}), on choisit ce dernier car les $J_n$ sont régulières et les $H_n$ évoluent en $\frac{1}{\sqrt{r}}$ à l'infini, donc ce choix est adapté à une décomposition en une onde incidente partout définie et une onde réfléchie décroissante à l'infini.

        De plus, d'après \cite[p.~358]{abramowitz_handbook_1964}, on sait qu'une fonction de Bessel d'ordre $n$ est linéairement dépendante de celle d'ordre $-n$. On peut donc se restreindre à $n$ entier naturel

        On trouve exactement le même résultat pour $\hat{H_z}$: soient $(c_3,c_4) \in \CC^2$
        \begin{equation}
            \hat{H_z}(r,n,k_z) = c_3 J_n\left(k_3r\right) + c_4 H_n^{(2)}\left(k_3r\right)
        \end{equation}
    \end{proof}


    \newcommand{\mJ}{\mat{J}}
    \newcommand{\mH}{\mat{H}}

    \begin{defn}
        On définit les matrices $\mJ_{E}(r,n,k_z),\mH_{E}(r,n,k_z),\mJ_{H}(r,n,k_z),\mH_{H}(r,n,k_z)$
        \begin{align}
            \mJ_{E}(r,n,k_z) &= 
            \begin{bmatrix}
                -\frac{nk_z}{rk_3^2}J_n(k_3r) & -\frac{i\w\mu}{k_3}J_n'(k_3r)
                \\
                J_n(k_3r) & 0
            \end{bmatrix}
            \\
            \mH_{E}(r,n,k_z) &= 
            \begin{bmatrix}
                -\frac{nk_z}{rk_3^2}H_n^{(2)}(k_3r) & -\frac{i\w\mu}{k_3}H_n^{(2)}{}'(k_3r)
                \\
                H_n^{(2)}(k_3r) & 0
            \end{bmatrix}
            \\
            \mJ_{H}(r,n,k_z) &= 
            \begin{bmatrix}
                0 & -J_n(k_3r)
                \\
                \frac{i\w\eps}{k_3}J_n'(k_3r) & -\frac{nk_z}{rk_3^2}J_n(k_3r)
            \end{bmatrix}
            \\
            \mH_{H}(r,n,k_z) &= 
            \begin{bmatrix}
                0 & -H_n^{(2)}(k_3r)
                \\
                \frac{i\w\eps}{k_3}H_n^{(2)}{}'(k_3r) & -\frac{nk_z}{rk_3^2}H_n^{(2)}(k_3r)
            \end{bmatrix}
        \end{align}
    \end{defn}

    \begin{prop}
        Alors les champs tangentiels s'écrivent
        \begin{subequations}
            \begin{align}
                \hat \vE_t(r,n,k_z) &= \mJ_{E}(r,n,k_z)
                \begin{bmatrix}
                    c_1 \\
                    c_3
                \end{bmatrix}
                +
                \mH_{E}(r,n,k_z)
                \begin{bmatrix}
                    c_2 \\
                    c_4
                \end{bmatrix}
                \label{eq:imp_fourier:cylindre:Et}\\
                \v{e_r}\times\hat \vH_t(r,n,k_z) &= 
                \mJ_{H}(r,n,k_z)
                \begin{bmatrix}
                    c_1 \\
                    c_3
                \end{bmatrix}
                +
                \mH_{H}(r,n,k_z)
                \begin{bmatrix}
                    c_2 \\
                    c_4
                \end{bmatrix}
                \label{eq:imp_fourier:cylindre:Ht}
            \end{align}
        \end{subequations}
    \end{prop}


    \begin{proof}
        À partir des équations de Maxwell restantes, on peut déterminer $\hat{E_r},\hat{E_\theta},\hat{H_r},\hat{H_\theta}$.
        \begin{equation}
            \left\lbrace
            \begin{matrix}
                -ik_z\hat{E_\theta} - i\w\mu \hat{H_r} = -\frac{in}{r}\hat{E_z}
                \\
                ik_z\hat{E_r} - i\w\mu \hat{H_\theta} = \ddr{r}{\hat{E_z}}
                \\
                -i\w\eps \hat{E_r} + ik_z \hat{H_\theta} = \frac{in}{r}\hat{H_z}
                \\
                -i\w\eps \hat{E_\theta} - ik_z \hat{H_r} = -\ddr{r}{\hat{H_z}}
            \end{matrix}
            \right.
        \end{equation}

        Cela revient à résoudre $\v{Y} = \mat{M}\v{X}$ où la matrice $\mat{M}$ et les vecteurs $\v{X}, \v{Y}$ sont définis tels que
        \begin{equation}
            \mat{M} =
            \begin{bmatrix}
            0 & -ik_z & -i\w\mu & 0 
            \\
            ik_z & 0 & 0 & -i\w\mu
            \\
            -i\w\eps & 0 & 0 & ik_z
            \\
            0 & -i\w\eps & -ik_z & 0
            \end{bmatrix}
            \,
            \v{X} = 
            \begin{bmatrix}
                \hat{E_r}\\
                \hat{E_\theta}\\
                \hat{H_r}\\
                \hat{H_\theta}
            \end{bmatrix}
            \,
            \v{Y} = 
            \begin{bmatrix}
                -\frac{in}{r}\hat{E_z}\\
                \ddr{r}{\hat{E_z}}\\
                \frac{in}{r}\hat{H_z}\\
                -\ddr{r}{\hat{H_z}}
            \end{bmatrix}
        \end{equation}

        À condition que $\det(\mat{M}) = (\w^2\eps\mu-k_z^2)^2$ soit non nul, on peut déduire $\v{X}$:

        \begin{equation}
            \begin{bmatrix}
                \hat{E_r}\
                \hat{E_\theta}\\
                \hat{H_r}\\
                \hat{H_\theta}
            \end{bmatrix} =
            \frac{1}{k_z^2 - \w^2\eps\mu}
            \begin{bmatrix}
            0 & -ik_z & -i\w\mu & 0 
            \\
            ik_z & 0 & 0 & -i\w\mu
            \\
            -i\w\eps & 0 & 0 & ik_z
            \\
            0 & -i\w\eps & -ik_z & 0
            \end{bmatrix}
            \begin{bmatrix}
                -\frac{in}{r}\hat{E_z}\\
                \ddr{r}{\hat{E_z}}\\
                \frac{in}{r}\hat{H_z}\\
                -\ddr{r}{\hat{H_z}}
            \end{bmatrix}
        \end{equation}

        On extrait alors $\hat{E_\theta}, \hat{H_\theta}$ pour obtenir les champs tangentielles à $\v{e_r}$ en tout point, sachant déjà $\hat{E_z}, \hat{H_z}$.

        \begin{align}
            \hat{E_\theta} &= -\frac{1}{k_3^2}\left(\frac{nk_z}{r}\hat{E_z} + i\w\mu\ddr{r}{\hat{H_z}}\right)
            \\
            \hat{E_z} &= c_1 J_n(k_3 r) + c_2 H_n^{(2)}(k_3 r)
            \\
            -\hat{H_z} &= -c_3 J_n(k_3 r) - c_4 H_n^{(2)}(k_3 r)
            \\
            \hat{H_\theta} &= \frac{1}{k_3^2}\left(i\w\eps\ddr{r}{\hat{E_z}} - \frac{nk_z}{r}\hat{H_z}\right)
        \end{align}

        On dérive les fonctions de Bessel:

       \begin{align}
            \hat{E_\theta} &= -\frac{nk_z}{rk_3^2}\left(c_1J_n(k_3r) + c_2 H_n^{(2)}(k_3r)\right) - \frac{i\w\mu}{k_3}\left(c_3J_n'(k_3r) + c_4 H_n^{(2)}{}'(k_3r)\right)
            \\
            \hat{E_z} &= c_1 J_n(k_3 r) + c_2 H_n^{(2)}(k_3 r)
            \\
            -\hat{H_z} &= -c_3 J_n(k_3 r) - c_4 H_n^{(2)}(k_3 r)
            \\
            \hat{H_\theta} &= \frac{i\w\eps}{k_3}\left(c_1J_n'(k_3r) + c_2 H_n^{(2)}{}'(k_3r)\right) - \frac{nk_z}{rk_3^2}\left(c_3J_n(k_3r) + c_4 H_n^{(2)}(k_3r)\right)
        \end{align}

        Et on obtient

        \begin{subequations}
            \label{eq:imp_fourier:cylindre:champs}
            \begin{align}
                \label{eq:imp_fourier:cylindre:champs:E}
                \hat \vE_t(r,n,k_z) &= \mJ_{E}(r)
                \begin{bmatrix}
                    c_1 \\
                    c_3
                \end{bmatrix}
                +
                \mH_{E}(r)
                \begin{bmatrix}
                    c_2 \\
                    c_4
                \end{bmatrix}
                \\
                \label{eq:imp_fourier:cylindre:champs:H}
                \v{e_r}\times\hat \vH_t(r,n,k_z) &= 
                \mJ_{H}(r)
                \begin{bmatrix}
                    c_1 \\
                    c_3
                \end{bmatrix}
                +
                \mH_{H}(r)
                \begin{bmatrix}
                    c_2 \\
                    c_4
                \end{bmatrix}
            \end{align}
        \end{subequations}

    \end{proof}

    %%%%%%%%%%%%%%%%%%%%%%%%%%%%%%%%%%%%%%%%%%%%%%%%%%%%%%%%%%%%%%%%%%%%%%%%%%%%%%%%%%%%%%%%%%%%%%%%%%%%%%%%
    %%%%%%%%%%%%%%%%%%%%%%%%%%%%%%%%%%%%%%%%%%%%%%%%%%%%%%%%%%%%%%%%%%%%%%%%%%%%%%%%%%%%%%%%%%%%%%%%%%%%%%%%
    %%%%%%%%%%%%%%%%%%%%%%%%%%%%%%%%%%%%%%%%%%%%%%%%%%%%%%%%%%%%%%%%%%%%%%%%%%%%%%%%%%%%%%%%%%%%%%%%%%%%%%%%


    \subsection{Opérateur d'impédance pour une couche}

        Soit $r_1 = r_0 + d$
        \begin{defn}
            On définit le symbole $\hat \mZ(n,k_z)$ de l'opérateur d'impédance la matrice telle que
            \begin{equation}
                \hat \vE_t(r_1,n,k_z) = \hat \mZ(n,k_z) \left(\v{e_r}\pvect \hat \vH_t(r_1,n,k_z)\right)
            \end{equation}
        \end{defn}

        \begin{thm}
            Si on suppose que les fonctions de Bessel et leurs dérivées ne s’annulent pas en $k_3r_0$ et que 
            la matrice $\mH_{H}(r_1) - \mJ_{H}(r_1)\mJ_{E}(r_0)^{-1}\mH_{E}(r_0)$ est inversible

            Alors le symbole $\hat \mZ(n,k_z)$ de l'opérateur d'impédance est
            \begin{multline}
                \hat \mZ(n,k_z) = 
                \left(\mH_{E}(r_1)\mH_{E}(r_0)^{-1} - \mJ_{E}(r_1)\mJ_{E}(r_0)^{-1}\right)\\
                \left(\mH_{H}(r_1)\mH_{E}(r_0)^{-1} - \mJ_{H}(r_1)\mJ_{E}(r_0)^{-1}\right)^{-1}
            \end{multline}
        \end{thm}

        \begin{proof}

            On injecte la relation $\vE_t(r_0,\theta,z) = 0$ équivalente à $\hat \vE(r_0,n,k_z) = 0$ dans \eqref{eq:imp_fourier:cylindre:Et}.
            \begin{equation}
                \mJ_{E}(r_0)
                \begin{bmatrix}
                    c_1 \\
                    c_3
                \end{bmatrix}
                =-\mH_{E}(r_0)
                \begin{bmatrix}
                    c_2 \\
                    c_4
                \end{bmatrix}
            \end{equation}

            Or par définition des matrices,
            \begin{align}
                \det(\mJ_E(r_0)) &= -\frac{i\w\mu}{k_3}J_n(k_3r_0)J_n'(k_3r_0)
                \\
                \det(\mH_E(r_0)) &= -\frac{i\w\mu}{k_3}H_n^{(2)}(k_3r_0)H_n^{(2)}{}'(k_3r_0)
            \end{align}

            D’après \cite[p.~370]{abramowitz_handbook_1964}, les zéros des fonctions de Bessel d'ordre réel $>-1$ sont tous réels. Donc à condition d'avoir $k_3$ complexe, comme l'ordre est entier et que l'on se restreint au entiers naturels, ces matrices sont inversibles\footnote{Là encore, il faut étudier le cas des matériaux sans pertes où $k_3$ est réel pour $k_z < w\sqrt{\mu\eps}$}.
            
            À condition de l'inversibilité de ces deux matrices, on peut donc exprimer uniquement grâce à la moitié des constantes
            \begin{align}
                \hat \vE_t(r_1,n,k_z) &= 
                \left(\mH_{E}(r_1) - \mJ_{E}(r_1)\mJ_{E}(r_0)^{-1}\mH_{E}(r_0)\right)
                \begin{bmatrix}
                    c_2 \\
                    c_4
                \end{bmatrix}
                \\
                \v{e_r}\pvect \hat \vH_t(r_1,n,k_z) &= 
                \left(\mH_{H}(r_1) - \mJ_{H}(r_1)\mJ_{E}(r_0)^{-1}\mH_{E}(r_0) \right)
                \begin{bmatrix}
                    c_2 \\
                    c_4
                \end{bmatrix}
            \end{align}

            Et à condition que $\mH_{H}(r_1) - \mJ_{H}(r_1)\mJ_{E}(r_0)^{-1}\mH_{E}(r_0)$ soit inversible, le symbole de l'opérateur d'impédance est:
            \TODO{Inversibilité de $\mH_{H}(r_1) - \mJ_{H}(r_1)\mJ_{E}(r_0)^{-1}\mH_{E}(r_0)$}
            \begin{align}
                \hat \mZ &= 
                \left(\mH_{E}(r_1) - \mJ_{E}(r_1)\mJ_{E}(r_0)^{-1}\mH_{E}(r_0)\right)
                \left(\mH_{H}(r_1) - \mJ_{H}(r_1)\mJ_{E}(r_0)^{-1}\mH_{E}(r_0)\right)^{-1}
                \\
                &=
                \left(\mH_{E}(r_1)\mH_{E}(r_0)^{-1} - \mJ_{E}(r_1)\mJ_{E}(r_0)^{-1}\right)
                \left(\mH_{H}(r_1)\mH_{E}(r_0)^{-1} - \mJ_{H}(r_1)\mJ_{E}(r_0)^{-1}\right)^{-1}
            \end{align}

            Contrairement au plan, les matrices ne commutent pas et on ne peut pas simplifier le résultat.

        \end{proof}

    %%%%%%%%%%%%%%%%%%%%%%%%%%%%%%%%%%%%%%%%%%%%%%%%%%%%%%%%%%%%%%%%%%%%%%%%%%%%%%%%%%%%%%%%%%%%%%%%%%%%%%%%
    %%%%%%%%%%%%%%%%%%%%%%%%%%%%%%%%%%%%%%%%%%%%%%%%%%%%%%%%%%%%%%%%%%%%%%%%%%%%%%%%%%%%%%%%%%%%%%%%%%%%%%%%
    %%%%%%%%%%%%%%%%%%%%%%%%%%%%%%%%%%%%%%%%%%%%%%%%%%%%%%%%%%%%%%%%%%%%%%%%%%%%%%%%%%%%%%%%%%%%%%%%%%%%%%%%


    \subsection{Opérateur d'impédance pour plusieurs couches}

        \begin{figure}[!hbt]
            \centering
            \begin{tikzpicture}
                \tikzmath{
    \a = 83;
    \b = 97;
    \d = 0.5;
    \ri = 30;
    \re = \ri;
}

% Le conducteur
\tikzmath{
    \ri = \re;
    \re = \ri + 0.5*\d;
    \xa = cos(\a)*\re;
    \ya = sin(\a)*\re;
    \xb = cos(\b)*\ri;
    \yb = sin(\b)*\ri;
}

\coordinate (a) at (\xa,\ya);
\coordinate (b) at (\xb,\yb);

\fill [pattern=north east lines] (a) arc (\a:\b:\re) -- (b) arc (\b:\a:\ri) -- cycle;
\draw (a) arc (\a:\b:\re);
\draw (a) node [right] {$r_0$};

% Le repère
\coordinate (n) at ($(a)+(0.5,-1)$);
%
%
%\draw [->] (n) -- ++(0,1) node [at end, right] {$\v{\pr}$};
%\draw [->] (n) -- ++(1,0) node [at end, right] {$\v{\pt}$};
%
\draw (n) ++(0.2,0.2) circle(0.1cm) node [above=0.1cm] {$\vect{e_z}$};
\draw (n) ++(0.2,0.2) +(135:0.1cm) -- +(315:0.1cm);
\draw (n) ++(0.2,0.2) +(45:0.1cm) -- +(225:0.1cm);

% 1 ere couche

\tikzmath{
    \ri = \re;
    \re = \ri + \d;
    \xa = cos(\a)*\re;
    \ya = sin(\a)*\re;
    \xb = cos(\b)*\ri;
    \yb = sin(\b)*\ri;
    \xc = cos(0.5*(\b+\a))*(\ri+0.5*\d);
    \yc = sin(0.5*(\b+\a))*(\ri+0.5*\d);
}

\coordinate (a) at (\xa,\ya);
\coordinate (b) at (\xb,\yb);
\coordinate (c) at (\xc,\yc);

\fill [lightgray] (a) arc (\a:\b:\re) -- (b) arc (\b:\a:\ri) -- cycle;
\draw (a) arc (\a:\b:\re);
\draw (c) node {$\eps_1,\mu_1,d_1$};


% Des couches

\tikzmath{
    \ri = \re;
    \re = \ri + 2*\d;
    \xa = cos(\a)*\re;
    \ya = sin(\a)*\re;
    \xb = cos(\b)*\ri;
    \yb = sin(\b)*\ri;
    \xc = cos(0.5*(\b+\a))*(\ri+0.5*\d);
    \yc = sin(0.5*(\b+\a))*(\ri+0.5*\d);
}

\coordinate (a) at (\xa,\ya);
\coordinate (b) at (\xb,\yb);
\coordinate (c) at (\xc,\yc);

\fill [lightgray]    (a) arc (\a:\b:\re) -- (b) arc (\b:\a:\ri) -- cycle;
\fill [pattern=dots] (a) arc (\a:\b:\re) -- (b) arc (\b:\a:\ri) -- cycle;
\draw (a) arc (\a:\b:\re);

% n eme couche

\tikzmath{
    \ri = \re;
    \re = \ri + \d;
    \xa = cos(\a)*\re;
    \ya = sin(\a)*\re;
    \xb = cos(\b)*\ri;
    \yb = sin(\b)*\ri;
    \xc = cos(0.5*(\b+\a))*(\ri+0.5*\d);
    \yc = sin(0.5*(\b+\a))*(\ri+0.5*\d);
}

\coordinate (a) at (\xa,\ya);
\coordinate (b) at (\xb,\yb);
\coordinate (c) at (\xc,\yc);

\fill [lightgray] (a) arc (\a:\b:\re) -- (b) arc (\b:\a:\ri) -- cycle;
\draw (a) arc (\a:\b:\re);
\draw (c) node {$\eps_{Nc},\mu_{Nc},d_{Nc}$};

% Le vide
\tikzmath{
    \xc = cos(0.5*(\b+\a))*(\re);
    \yc = sin(0.5*(\b+\a))*(\re);
}

\draw (\xc,\yc) node [above] {vide};


            \end{tikzpicture}
        \end{figure}

        Soit $r_m$ le rayon de la couche $m$, $r_m = r_0 +\sum_{i=1}^{m} d_{i}$. 

        \begin{defn}
            Pour chaque couche caractérisée par $(\eps_m,\mu_m,d_m)$, définissons
            \begin{subequations}
                \begin{align}
                    k_{3m} &= \sqrt{w^2\eps_m\mu_m - k_z^2}
                    \\
                    \mJ_{Em}(r) &= 
                        \begin{bmatrix}
                            -\frac{nk_z}{rk_{3m}^2}J_n(k_{3m}r) & -\frac{i\w\mu_m}{k_{3m}}J_n'(k_{3m}r)
                            \\
                            J_n(k_{3m}r) & 0
                        \end{bmatrix}
                    \\
                    \mH_{Em}(r) &= 
                        \begin{bmatrix}
                            -\frac{nk_z}{rk_{3m}^2}H_n^{(2)}(k_{3m}r) & -\frac{i\w\mu_m}{k_{3m}}H_n^{(2)}{}'(k_{3m}r)
                            \\
                            H_n^{(2)}(k_{3m}r) & 0
                        \end{bmatrix}
                    \\
                    \mJ_{Hm}(r) &= 
                        \begin{bmatrix}
                            0 & -J_n(k_{3m}r)
                            \\
                            \frac{i\w\eps_m}{k_{3m}}J_n'(k_{3m}r) & -\frac{nk_z}{rk_{3m}^2}J_n(k_{3m}r)
                        \end{bmatrix}
                    \\
                    \mH_{Hm}(r) &= 
                        \begin{bmatrix}
                            0 & -H_n^{(2)}(k_{3m}r)
                            \\
                            \frac{i\w\eps_m}{k_{m3}}H_n^{(2)}{}'(k_{3m}r) & -\frac{nk_z}{rk_{3m}^2}H_n^{(2)}(k_{3m}r)
                        \end{bmatrix}
                    \\
                    \mA_{Jm}(r) &= \mJ_{Em}(r) -  \mZ_{m-1} \mJ_{Hm}(r)
                    \\
                    \mA_{Hm}(r) &= \mH_{Em}(r) -  \mZ_{m-1} \mH_{Hm}(r)
                \end{align}
            \end{subequations}

            On définit pour chaque interface, le symbole $\hat \mZ_m$ tel que 
            \begin{equation}
                \hat \vE_t(r_m,n,k_z) = \hat \mZ_m(n,k_z) \left(\v{e_r} \pvect \hat \vH_t(r_m,n,k_z)\right)
            \end{equation}
        \end{defn}

        \begin{thm}
            Soit $\hat \mZ_0(n,k_z) = \mat{0}_{\mathcal{M}_2(\CC)}$.

            Si pour tout $0 < m < n$

            \begin{subequations}
                \begin{align}
                    k_{3m} & \not = 0 \\
                    \det\left(\mA_{Jm}(r_{m-1})\right) & \not = 0 \\
                    \det\left(\mA_{Hm}(r_{m-1})\right) & \not = 0 \\
                    \det\left(\mH_{Hm}(r_{m})\mA_{Hm}(r_{m-1})^{-1} - \mJ_{Hm}(r_{m})(\mA_{Jm}(r_{m-1}))^{-1}\right) &\not = 0
                \end{align}
            \end{subequations}

            Alors le symbole $\hat \mZ_n$ est défini par la relation de récurrence : 
            \begin{multline}
                \mZ_m = \left(\mH_{Em}(r_m)\mA_{Hm}(r_{m-1})^{-1} - \mJ_{Em}(r_m)\mA_{Jm}(r_{m-1})^{-1}\right) \\
                        \left(\mH_{Hm}(r_m)\mA_{Hm}(r_{m-1})^{-1} - \mJ_{Hm}(r_m)\mA_{Jm}(r_{m-1})^{-1}\right)^{-1}
            \end{multline}
        \end{thm}

        \begin{proof}
            À l'initialisation, on retrouve le résultat pour une couche.

            On résonne par récursivité: 

            On se situe dans la couche $m$ et l'on sait que les champs vérifient
            \begin{equation}
                \begin{bmatrix}
                    \hat{E_\theta}(r_{m-1},n,k_z)\\
                    \hat{E_z}(r_{m-1},n,k_z)\\
                \end{bmatrix}
                =
                \hat \mZ_{m-1}(n,k_z)
                \begin{bmatrix}
                    -\hat{H_z}(r_{m-1},n,k_z)\\
                    \hat{H_\theta}(r_{m-1},n,k_z)\\
                \end{bmatrix}
            \end{equation}

            En injectant ce qui précède dans \eqref{eq:imp_fourier:cylindre:champs} en $r = r_{m-1}$
            \begin{align}
                \mJ_{Em}(r_{m-1})
                \begin{bmatrix}
                    c_1 \\
                    c_3
                \end{bmatrix}
                +
                \mH_{Em}(r_{m-1})
                \begin{bmatrix}
                    c_2 \\
                    c_4
                \end{bmatrix}
                &=
                \hat \mZ_{m-1}
                \left(
                    \mJ_{Hm}(r_{m-1})
                    \begin{bmatrix}
                        c_1 \\
                        c_3
                    \end{bmatrix}
                    +
                    \mH_{Hm}(r_{m-1})
                    \begin{bmatrix}
                        c_2 \\
                        c_4
                    \end{bmatrix}
                \right)
                \\
                \mA_{Jm}(r_{m-1})
                \begin{bmatrix}
                    c_1 \\
                    c_3
                \end{bmatrix}
                &=
                -\mA_{Hm}(r_{m-1})
                \begin{bmatrix}
                    c_2 \\
                    c_4
                \end{bmatrix}
            \end{align}

            \TODO{Inversibilité de $\mA_{Jm}(r_m), \mA_{Hm}(r_m)$.}

            On injecte ce qui précède dans \eqref{eq:imp_fourier:cylindre:champs} en $r = r_{m}$
            \begin{align}
                \vE_t &= 
                \left(\mH_{Em}(r_{m}) - \mJ_{Em}(r_{m})\mA_{Jm}(r_{m-1})^{-1}\mA_{Hm}(r_{m-1})\right)
                \begin{bmatrix}
                    c_2 \\
                    c_4
                \end{bmatrix}
                \\
                \v{e_r}\times\vH_t &= 
                \left(\mH_{Hm}(r_{m}) - \mJ_{Hm}(r_{m})\mA_{Jm}(r_{m-1})^{-1}\mA_{Hm}(r_{m-1}) \right)
                \begin{bmatrix}
                    c_2 \\
                    c_4
                \end{bmatrix}
            \end{align}

            \TODO{Inversibilité de $\mH_{Hm}(r_m) - \mJ_{Hm}(r_m)(\mA_{Jm}(r_{m-1}))^{-1}\mA_{Hm}(r_{m-1})$.}

            On peut alors conclure sur le symbole

            \begin{multline}
                \hat \mZ_{m} = 
                    \left(\mH_{Em}(r_m) - \mJ_{Em}(r_m)\mA_{Jm}(r_{m-1})^{-1}\mA_{Hm}(r_{m-1})\right) \\
                    \left(\mH_{Hm}(r_m) - \mJ_{Hm}(r_m)\mA_{Jm}(r_{m-1})^{-1}\mA_{Hm}(r_{m-1})\right)^{-1}
            \end{multline}

            \begin{multline}
                \hat \mZ_{m} =
                    \left(\mH_{Em}(r_m)\mA_{Hm}(r_{m-1})^{-1} - \mJ_{Em}(r_m)\mA_{Jm}(r_{m-1})^{-1}\right) \\
                    \left(\mH_{Hm}(r_m)\mA_{Hm}(r_{m-1})^{-1} - \mJ_{Hm}(r_m)\mA_{Jm}(r_{m-1})^{-1}\right)^{-1}
            \end{multline}

        \end{proof}

    \subsection{Applications numérique}

        La figure \ref{fig:imp_fourier:cylindre:hoppe_p62} permet de vérifier les résultats de \cite[p.~62]{hoppe_impedance_1995} pour une couche de matériau sans perte.

        \TODO{Expliquer pourquoi on prendre des valeurs continues de $n$ et non discrètes}

        \begin{figure}[!hbt]
            \centering
            \begin{tikzpicture}[scale=1]
                \begin{axis}[
                        title={},
                        ylabel={$\Im(\hat{\mZ}(k_t r_1,0))$},
                        xlabel={$k_t\slash k_0$},
                        width=0.8\textwidth,
                        xmin=0,
                        xmax=1.5,
                        mark repeat=20,
                        legend pos=outer north east
                    ]
                    \legend{TM,TE}
                    \addplot table [x={s}, y={imag(z.tm)},col sep=semicolon] {tikz/csv/impedance/cylindre/hoppe_p62_1_r.csv};
                    \addplot table [x={s}, y={imag(z.te)},col sep=semicolon] {tikz/csv/impedance/cylindre/hoppe_p62_1_r.csv};
                \end{axis}
            \end{tikzpicture}
            \caption{$\eps = 6, \mu = 1, r_0 = 0.0300\text{m}, d=0.0225\text{m}, f=1\text{GHz}$}
            \label{fig:imp_fourier:cylindre:hoppe_p62}
        \end{figure}
        
        La figure \ref{fig:imp_fourier:cylindre:hoppe_p62:converge_rayon} montre la converge du symbole de l'impédance d'un cylindre vers le symbole du plan en fonction du rayon du cylindre.

        \begin{figure}[!hbt]
            \centering
            \begin{tikzpicture}[scale=1]
                \begin{axis}[
                        title={},
                        ylabel={$\Im(\hat{\mZ}(\cdot,0))$},
                        xlabel={$k_t \slash k_0; k_x \slash k_0$},
                        width=0.6\textwidth,
                        xmin=0,
                        xmax=1.5,
                        mark repeat=20,
                        legend pos=outer north east
                    ]
                    \addplot table [col sep=semicolon, x={s}, y={imag(z.tm)}] {tikz/csv/impedance/plan/hoppe_p62.csv};
                    \addplot table [col sep=semicolon, x={s}, y={imag(z.te)}] {tikz/csv/impedance/plan/hoppe_p62.csv};
                    \addplot table [col sep=semicolon, x={s}, y={imag(z.tm)}] {tikz/csv/impedance/cylindre/hoppe_p62_1_r.csv};
                    \addplot table [col sep=semicolon, x={s}, y={imag(z.te)}] {tikz/csv/impedance/cylindre/hoppe_p62_1_r.csv};
                    \addplot table [col sep=semicolon, x={s}, y={imag(z.tm)}] {tikz/csv/impedance/cylindre/hoppe_p62_100_r.csv};
                    \addplot table [col sep=semicolon, x={s}, y={imag(z.te)}] {tikz/csv/impedance/cylindre/hoppe_p62_100_r.csv};
                    \legend{TM plan,TE plan,TM cylindre $r_0=r_i$,TE cylindre $r_0=r_i$,TM cylindre $r_0=100r_i$,TE cylindre $r_0=100r_i$}
                \end{axis}
            \end{tikzpicture}
            \caption{$\eps = 6, \mu = 1, r_i = 0.0300m, d=0.0225\text{m}, f=1\text{GHz}$}
            \label{fig:imp_fourier:cylindre:hoppe_p62:converge_rayon}
        \end{figure}
        
        \begin{figure}[!hbt]
            \centering
            \begin{tikzpicture}[scale=1]
                \begin{semilogxaxis}[
                        title={},
                        ylabel={$||\hat{\mZ}_{plan} - \hat{\mZ}_{cyl}||_2$},
                        xlabel={$r_0/r_i$},
                        width=0.8\textwidth,
                        xmin=0,
                        xmax=100,
                        % mark repeat=20,
                        legend pos=outer north east
                    ]
                    \legend{TM,TE}
                    \addplot table [x={r0/ri}, y={tm},col sep=semicolon] {tikz/csv/impedance/cylindre/hoppe_p62_error.csv};
                    \addplot table [x={r0/ri}, y={te},col sep=semicolon] {tikz/csv/impedance/cylindre/hoppe_p62_error.csv};
                \end{semilogxaxis}
            \end{tikzpicture}
            \caption{$\eps = 6, \mu = 1, r_i = 0.0300\text{m}, d=0.0225\text{m}, f=1\text{GHz}$}
            \label{fig:imp_fourier:cylindre:hoppe_p62:converge_rayon:error}
        \end{figure}
\section{Cas d'un objet sphérique}

    Les champs solutions de Maxwell dans le cas d'un repère sphérique sont décomposables en harmoniques sphériques. Nous rappelons d'abord l’expression de ces dernières puis nous donnerons l'expression du symbole de l'opérateur de d'impédance de la même manière que \cite{cheng_spectral_1993}.

    \subsection{Les harmoniques sphériques}

        \begin{TODO}
          Mettre ici la démonstration des harmoniques sphériques? Ou une référence vers annexe ?
        \end{TODO}

        On définit les harmoniques sphériques les solutions de \(\Delta U + k^2 U = 0 \). Ce sont les fonctions \(Y_{m,n} = C(m,n) e^{im\phi}\PP^m_n(\cos \theta) \) avec \(C(m,n)\) tel que,
        \[
         \ds\int_S Y_{m,n} \conj{Y_{p,q}} ds = \delta_m^p \delta_n^q
        \]

        D’après \cite[p.~24]{nedelec_acoustic_2001}, \( C(m,n) = (-1)^m\sqrt{\frac{2n+1}{4\pi}\frac{(n-m)!}{(n+m)!}}\)

        On définit les vecteurs harmonique sphériques\(\gls{phy-Mmn} ,\gls{phy-Nmn}\) solution de \(\vrot \vrot \vect{U} - k^2 \vect{U} = 0 \):
        \begin{align}
         \label{eq:defMmn}
          \Mmn[z_n](\rtp) &:= \vrot \left( \vect{r} z_n(kr) Y_{m,n} \right)\\
          &= z_n(kr)
          \begin{bmatrix}
            0 
            \\
            \frac{im}{\sin\theta}Y_{mn}
            \\
            - \ddr{\theta}{Y_{mn}}
          \end{bmatrix}
        \end{align}

        \begin{align}
        \label{eq:defNnn}
          \Nmn[z_n](\rtp) &:= \frac{\vrot \Mmn[z_n]}{k} \\
          &= \frac{1}{kr}\begin{bmatrix}
            \frac{z_n(kr)}{\sin\theta}\ddr{\theta}{}\left(\sin\theta\ddr{\theta}{Y_{mn}}\right)
            \\
            \ddr{r}{z_n}(kr)\ddr{\theta}{Y_{mn}}
            \\
            \ddr{r}{z_n}(kr)\frac{im}{\sin\theta}Y_{mn}
          \end{bmatrix}
        \end{align}

        % Ces vecteurs harmoniques sphériques possèdent les propriétés suivantes

        % \begin{align}
        % \int_{S(0,R)} \vect{M_{m,n}^{z_n}} \cdot \conj{\vect{N_{p,q}^{z_n}}} ds &= 0
        % \\
        % \int_{S(0,R)} \vect{M_{m,n}^{z_n}} \cdot \conj{\vect{M_{p,q}^{z_n}}} ds &= \gamma_{m,n}R^2 \delta_{mp}\delta_{nq}
        % \\
        % \int_{S(0,R)} \vect{N_{m,n}^{z_n}} \cdot \conj{\vect{N_{p,q}^{z_n}}} ds &= \frac{\gamma_{m,n}}{k^2} \delta_{mp}\delta_{nq}
        % \end{align}

        On définit les parties tangentielles de ces vecteurs par séparation des variables \(r\) et \((\theta,\phi)\):
     
        \begin{align}
          \Mmn[z_n]_t(\rtp) &= z_n(kr)\hat{\Mmn}(\tp)
          \\
          \Nmn[z_n]_t(\rtp) &= \frac{1}{kr}\ddr{r}{z_n}(kr)\hat{\Nmn}(\tp))
        \end{align}

        On a alors les propriétés supplémentaires

        \begin{align}
          \vect{e_r} \pvect \Mmn[z_n]_t(\rtp) &= krz_n(kr)\hat{\Nmn}(\tp)
          \\
          \vect{e_r} \pvect \Nmn[z_n]_t(\rtp) &= -\frac{1}{kr}\ddr{r}{z_n}(kr)\hat{\Mmn}(\tp)
        \end{align}


        On a ( \cite{cheng_spectral_1993})
        \begin{multline}
            \vE(\rtp) = \sum_{n\in\ZZ}\sum_{m\in\ZZ} a_{mn} \Mmn[j_n](\rtp) + b_{mn} \Nmn[j_n](\rtp)
            \\
            + c_{mn} \Mmn[h_n](\rtp) + d_{mn} \Nmn[h_n](\rtp)
        \end{multline}

        D'après les équations de Maxwell, \(\vH = i\frac{\vrot \vE}{k\eta}\)

        \begin{multline}
            \vH(\rtp) = \frac{i}{\eta}\sum_{n\in\ZZ}\sum_{m\in\ZZ} a_{mn} \Nmn[j_n](\rtp) + b_{mn} \Mmn[j_n](\rtp)
            \\
            + c_{mn} \Nmn[h_n](\rtp) + d_{mn} \Mmn[h_n](\rtp)
        \end{multline}

        Donc \(\vJ = \vect{e_r} \pvect \vH\) s'écrit en série de Mie

        \begin{multline}
            \vJ(\rtp) = \frac{i}{\eta}\sum_{n\in\ZZ}\sum_{m\in\ZZ} - a_{mn} \frac{1}{kr}\ddr{r}{j_n}\hat{\Mmn}(\tp) + b_{mn} k r j_n \hat{\Mmn}(\tp)
            \\
            -  \frac{1}{kr}\ddr{r}{h_n} c_{mn} \hat{\Mmn}(\tp) + k r h_n d_{mn} \hat{\Nmn}(\tp)
        \end{multline}

        Dans la suite, pour simplifier les écritures, on utilisera la notation tilde \gls{mat-tild}, telle que \( \tilde{z_n}(k_r) = \ddr{r}{z_n}(kr) \). On omettra aussi les dépendances en \(kr\) lorsqu'il n'y a pas d’ambiguïtés.

        On réécrit alors matriciellement les expressions de \(\vE_t,\vJ\).

        \begin{equation}
            \vE_t(\rtp) = \sum_{n\in\ZZ}\sum_{m\in\ZZ}
            \begin{bmatrix}
              \hat{\Nmn} & \hat{\Mmn}
            \end{bmatrix}
            \left( 
              \begin{bmatrix}
                  0 & \tilde{j_n}
                  \\
                  j_n & 0
              \end{bmatrix}
              \begin{bmatrix}
                  a_{mn}
                  \\
                  b_{mn}
              \end{bmatrix}
              + 
              \begin{bmatrix}
                  0 & \tilde{h_n}
                  \\
                  h_n & 0
              \end{bmatrix}
              \begin{bmatrix}
                  c_{mn}
                  \\
                  d_{mn}
              \end{bmatrix}
            \right)
        \end{equation}


        \begin{equation}
            \vJ(\rtp) = \frac{i}{\eta kr}\sum_{n\in\ZZ}\sum_{m\in\ZZ}
            \begin{bmatrix}
                \hat{\Nmn} & \hat{\Mmn}
            \end{bmatrix}
            \left( 
                \begin{bmatrix}
                    0 & (kr)^2 j_n
                    \\
                    -\tilde{j_n} & 0
                \end{bmatrix}
                \begin{bmatrix}
                    a_{mn}
                    \\
                    b_{mn}
                \end{bmatrix}
                + 
                \begin{bmatrix}
                    0 & (kr)^2 h_n
                    \\
                    -\tilde{h_n} & 0
                \end{bmatrix}
                \begin{bmatrix}
                    c_{mn}
                    \\
                    d_{mn}
                \end{bmatrix}
            \right)
        \end{equation}

        \begin{defn}
            On définit les matrices \(\mJ_{E}(r,n),\mH_{E}(r,n),\mJ_{H}(r,n),\mH_{H}(r,n)\)
            \begin{align}
                \mJ_{E}(r,n) &=
                \begin{bmatrix}
                    0 & \tilde{j_n}(kr)
                    \\
                    j_n(kr) & 0
                \end{bmatrix}
                \\
                \mH_{E}(r,n) &=
                \begin{bmatrix}
                    0 & \tilde{h_n}(kr)
                    \\
                    h_n(kr) & 0
                \end{bmatrix}
                \\
                \mJ_{H}(r,n) &=
                \begin{bmatrix}
                    0 & (kr)^2 j_n(kr)
                    \\
                    -\tilde{j_n}(kr) & 0
                \end{bmatrix}
                \\
                \mH_{H}(r,n) &=
                \begin{bmatrix}
                    0 & (kr)^2 h_n(kr)
                    \\
                    -\tilde{h_n}(kr) & 0
                \end{bmatrix}
            \end{align}
        \end{defn}

        \begin{equation}
            \hat{\vE_t}(r,m,n) = 
            \mJ_{E}(r,n)
            \begin{bmatrix}
                a_{mn}
                \\
                b_{mn}
            \end{bmatrix}
            + 
            \mH_{E}(r,n)
            \begin{bmatrix}
                c_{mn}
                \\
                d_{mn}
            \end{bmatrix}
        \end{equation}

        \begin{equation}
            \hat{\vJ}(r,m,n) = \frac{i}{\eta k r}
            \left(
            \mJ_{H}(r,n)
            \begin{bmatrix}
                a_{mn}
                \\
                b_{mn}
            \end{bmatrix}
            + 
            \mH_{H}(r,n)
            \begin{bmatrix}
                c_{mn}
                \\
                d_{mn}
            \end{bmatrix}
            \right)
        \end{equation}

    \subsection{Symbole de l'opérateur d'impédance pour une couche}

        \begin{figure}[!hbt]
          \centering
          \begin{tikzpicture}
            \tikzmath{
    \a = 80;
    \b = 100;
    \d = 0.5;
    \ri = 20;
    \re = \ri;
}

% Le conducteur
\tikzmath{
    \ri = \re;
    \re = \ri + 0.5*\d;
    \xa = cos(\a)*\re;
    \ya = sin(\a)*\re;
    \xb = cos(\b)*\ri;
    \yb = sin(\b)*\ri;
}

\coordinate (a) at (\xa,\ya);
\coordinate (b) at (\xb,\yb);

\fill [pattern=north east lines] (a) arc (\a:\b:\re) -- (b) arc (\b:\a:\ri) -- cycle;
\draw (a) arc (\a:\b:\re);
\draw (a) node [right] {$r_0$};


% Le repère
\coordinate (n) at ($(a)+(0.5,-1)$);
%
%
%\draw [->] (n) -- ++(0,1) node [at end, right] {$\v{\pr}$};
%\draw [->] (n) -- ++(1,0) node [at end, right] {$\v{\pt}$};
%
\draw (n) ++(0.2,0.2) circle(0.1cm) node [above=0.1cm] {\(\vect{e_\phi}\)};
\draw (n) ++(0.2,0.2) +(135:0.1cm) -- +(315:0.1cm);
\draw (n) ++(0.2,0.2) +(45:0.1cm) -- +(225:0.1cm);


% 1ere couche
\tikzmath{
    \ri = \re;
    \re = \ri + \d;
    \xa = cos(\a)*\re;
    \ya = sin(\a)*\re;
    \xb = cos(\b)*\ri;
    \yb = sin(\b)*\ri;
    \xc = cos(0.5*(\b+\a))*(\ri+0.5*\d);
    \yc = sin(0.5*(\b+\a))*(\ri+0.5*\d);
}

\coordinate (a) at (\xa,\ya);
\coordinate (b) at (\xb,\yb);
\coordinate (c) at (\xc,\yc);

\fill [lightgray] (a) arc (\a:\b:\re) -- (b) arc (\b:\a:\ri) -- cycle;
\draw (a) arc (\a:\b:\re);
\draw (c) node {$\nu,\eta,d$};

% Le vide
\tikzmath{
    \xc = cos(0.5*(\b+\a))*(\re);
    \yc = sin(0.5*(\b+\a))*(\re);
}

\draw (\xc,\yc) node [above] {vide};
          \end{tikzpicture}
        \end{figure}

        \begin{defn}
          On définit le symbole de l'opérateur d'impédance \(\hat{\mZ}(m,n)\) tel que 
          \[
              \hat{\vE_t}(r_1,m,n) = \hat{\mZ}(m,n)\hat{\vJ}(r_1,m,n)
          \]
        \end{defn}

        En \(r=r_0\), on a la relation \(\vE_t(\rtp) = 0\) donc \(\hat{\vE_t}(r_0,m,n) = 0 \)

        \begin{equation}
            \mJ_{E}(r_0,n)
            \begin{bmatrix}
                a_{mn}
                \\
                b_{mn}
            \end{bmatrix}
            = - 
            \mH_{E}(r_0,n)
            \begin{bmatrix}
                c_{mn}
                \\
                d_{mn}
            \end{bmatrix}
        \end{equation}

        On suppose que les matrices \(\mJ_{E}(r_0,n)\) et \(\mH_{E}(r_0,n)\) soient inversibles.

        \begin{TODO}
          Inversibilité \(\mJ_{E}(r,n), \mH_{E}(r,n)\)
        \end{TODO}

        \begin{equation}
            \hat{\vE_t}(r,m,n) = 
            \left(
                \mH_{E}(r,n)
                -
                \mJ_{E}(r,n)
                \mJ_{E}(r_0,n)^{-1}
                \mH_{E}(r_0,n)
            \right)
            \begin{bmatrix}
                c_{mn}
                \\
                d_{mn}
            \end{bmatrix}
        \end{equation}


        \begin{equation}
            \hat{\vJ}(r,m,n) = \frac{i}{\eta}
            \left(
                \mH_{H}(r,n)
                -
                \mJ_{H}(r,n)
                \mJ_{E}(r_0,n)^{-1}
                \mH_{E}(r_0,n)
            \right)
            \begin{bmatrix}
                c_{mn}
                \\
                d_{mn}
            \end{bmatrix}
        \end{equation}

        De la même manière que pour le plan et le cylindre, on en déduit le symbole de l'opérateur d'impédance

        \begin{multline}
            \hat{\mZ}(m,n) = -i\eta
            \left(
                \mH_{E}(r_1,n)
                \mH_{E}(r_0,n)^{-1}
                -
                \mJ_{E}(r_1,n)
                \mJ_{E}(r_0,n)^{-1}
            \right)
            \\
            \left(
                \mH_{H}(r_1,n)
                \mH_{E}(r_0,n)^{-1}
                -
                \mJ_{H}(r_1,n)
                \mJ_{E}(r_0,n)^{-1}
            \right)^{-1}
        \end{multline}

        Par définition des matrices \(\mJ_E,\mH_E,\mJ_H,\mH_H\), elle sont anti-diagonale. Donc leur inverse l'est aussi. Donc le produit de l'une avec l'inverse d'une autre est une matrice diagonale. Donc le symbole de l'opérateur d'impédance est une matrice diagonale.

        \begin{equation}
            \hat{\mZ}(m,n) = -i\eta
            \begin{bmatrix}
                \frac
                {\tilde{h_n}(kr_1)\tilde{j_n}(kr_0)-\tilde{j_n}(kr_1)\tilde{h_n}(kr_0)}
                {{h_n}(kr_1)\tilde{j_n}(kr_0)-{j_n}(kr_1)\tilde{h_n}(kr_0)} & 0
                \\
                0 & \frac
                {{j_n}(kr_1){h_n}(kr_0)-{h_n}(kr_1){j_n}(kr_0)}
                {\tilde{h_n}(kr_1){j_n}(kr_0)-\tilde{j_n}(kr_1){h_n}(kr_0)}
            \end{bmatrix}
        \end{equation}

    \subsection{Symbole de l'opérateur d'impédance pour plusieurs couche}

        \begin{TODO}
            Les matrices J,H dépendent de k donc de nu qui depend de la couche. Corriger.
        \end{TODO}

        \begin{figure}[!hbt]
          \centering
          \begin{tikzpicture}
            \tikzmath{
    \a = 83;
    \b = 97;
    \d = 0.5;
    \ri = 30;
    \re = \ri;
}

% Le conducteur
\tikzmath{
    \ri = \re;
    \re = \ri + 0.5*\d;
    \xa = cos(\a)*\re;
    \ya = sin(\a)*\re;
    \xb = cos(\b)*\ri;
    \yb = sin(\b)*\ri;
}

\coordinate (a) at (\xa,\ya);
\coordinate (b) at (\xb,\yb);

\fill [pattern=north east lines] (a) arc (\a:\b:\re) -- (b) arc (\b:\a:\ri) -- cycle;
\draw (a) arc (\a:\b:\re);
\draw (a) node [right] {$r_0$};

% Le repère
\coordinate (n) at ($(a)+(0.5,-1)$);
%
%
%\draw [->] (n) -- ++(0,1) node [at end, right] {$\v{\pr}$};
%\draw [->] (n) -- ++(1,0) node [at end, right] {$\v{\pt}$};
%
\draw (n) ++(0.2,0.2) circle(0.1cm) node [above=0.1cm] {$\vect{e_\phi}$};
\draw (n) ++(0.2,0.2) +(135:0.1cm) -- +(315:0.1cm);
\draw (n) ++(0.2,0.2) +(45:0.1cm) -- +(225:0.1cm);

% 1 ere couche

\tikzmath{
    \ri = \re;
    \re = \ri + \d;
    \xa = cos(\a)*\re;
    \ya = sin(\a)*\re;
    \xb = cos(\b)*\ri;
    \yb = sin(\b)*\ri;
    \xc = cos(0.5*(\b+\a))*(\ri+0.5*\d);
    \yc = sin(0.5*(\b+\a))*(\ri+0.5*\d);
}

\coordinate (a) at (\xa,\ya);
\coordinate (b) at (\xb,\yb);
\coordinate (c) at (\xc,\yc);

\fill [lightgray] (a) arc (\a:\b:\re) -- (b) arc (\b:\a:\ri) -- cycle;
\draw (a) arc (\a:\b:\re);
\draw (c) node {$\nu_1,\eta_1,d_1$};


% Des couches

\tikzmath{
    \ri = \re;
    \re = \ri + 2*\d;
    \xa = cos(\a)*\re;
    \ya = sin(\a)*\re;
    \xb = cos(\b)*\ri;
    \yb = sin(\b)*\ri;
    \xc = cos(0.5*(\b+\a))*(\ri+0.5*\d);
    \yc = sin(0.5*(\b+\a))*(\ri+0.5*\d);
}

\coordinate (a) at (\xa,\ya);
\coordinate (b) at (\xb,\yb);
\coordinate (c) at (\xc,\yc);

\fill [lightgray]    (a) arc (\a:\b:\re) -- (b) arc (\b:\a:\ri) -- cycle;
\fill [pattern=dots] (a) arc (\a:\b:\re) -- (b) arc (\b:\a:\ri) -- cycle;
\draw (a) arc (\a:\b:\re);

% n eme couche

\tikzmath{
    \ri = \re;
    \re = \ri + \d;
    \xa = cos(\a)*\re;
    \ya = sin(\a)*\re;
    \xb = cos(\b)*\ri;
    \yb = sin(\b)*\ri;
    \xc = cos(0.5*(\b+\a))*(\ri+0.5*\d);
    \yc = sin(0.5*(\b+\a))*(\ri+0.5*\d);
}

\coordinate (a) at (\xa,\ya);
\coordinate (b) at (\xb,\yb);
\coordinate (c) at (\xc,\yc);

\fill [lightgray] (a) arc (\a:\b:\re) -- (b) arc (\b:\a:\ri) -- cycle;
\draw (a) arc (\a:\b:\re);
\draw (c) node {$\nu_p,\eta_p,d_p$};

% Le vide
\tikzmath{
    \xc = cos(0.5*(\b+\a))*(\re);
    \yc = sin(0.5*(\b+\a))*(\re);
}

\draw (\xc,\yc) node [above] {vide};


          \end{tikzpicture}
        \end{figure}


        \begin{defn}
          Pour chaque couche \(p\), on définit le symbole de l'opérateur d'impédance \(\hat{\mZ}_p(m,n)\) tel que 
          \[
              \hat{\vE_t}(r_p,m,n) = \hat{\mZ}_p(m,n)\hat{\vJ}(r_p,m,n)
          \]
        \end{defn}

        On résonne par récurrence: on suppose connu le symbole de l'opérateur d'impédance de la couche \(p\) et on cherche le suivant

        En \(r=r_{p}=r_0+\sum_{i=1}^p d_p\), on a la relation \( \hat{\vE_t}(r_p,m,n) = \hat{\mZ}_p(m,n)\hat{\vJ}(r_p,m,n)\) où \(\hat{\mZ}_p(m,n)\) est un matrice diagonale

        \begin{equation}
            \left(\mJ_{E}(r_p,n) - \frac{i}{\eta_p}\hat{\mZ}_p(m,n)\mJ_{H}(r_p,n) \right)
            \begin{bmatrix}
                a_{mn}
                \\
                b_{mn}
            \end{bmatrix}
            = -
            \left(\mH_{E}(r_p,n) - \frac{i}{\eta_p}\hat{\mZ}_p(m,n)\mH_{H}(r_p,n) \right)
            \begin{bmatrix}
                c_{mn}
                \\
                d_{mn}
            \end{bmatrix}
        \end{equation}

        On définit les matrices \(\mA_{J}(r,n)\) et \(\mA_{H}(r,n)\) telle que

        \begin{align}
            \mA_{J}(r,n) &= \mJ_{E}(r,n) - \frac{i}{\eta_p}\hat{\mZ}_p(m,n)\mJ_{H}(r,n)
            \\
            \mA_{H}(r,n) &= \mH_{E}(r,n) - \frac{i}{\eta_p}\hat{\mZ}_p(m,n)\mH_{H}(r,n)
        \end{align}

        On suppose que les matrices \(\mA_{E}(r,n)\) et \(\mA_{H}(r,n)\) soient inversibles.

        \begin{TODO}
          Inversibilité \(\mA_{E}(r_p,n)\) et \(\mA_{H}(r_p,n)\)
        \end{TODO}

        Par hypothèse sur \(\hat{\mZ_p}(m,n)\), ces matrices sont anti-diagonale.

        On en déduit

        \begin{equation}
            \hat{\vE_t}(r_{p+1},m,n) = 
            \left(
                \mH_{E}(r_{p+1},n)
                -
                \mJ_{E}(r_{p+1},n)
                \mA_{J}(r_p,n)^{-1}
                \mA_{H}(r_p,n)
            \right)
            \begin{bmatrix}
                c_{mn}
                \\
                d_{mn}
            \end{bmatrix}
        \end{equation}


        \begin{equation}
            \hat{\vJ}(r_{p+1},m,n) = \frac{i}{\eta_p}
            \left(
                \mH_{H}(r_{p+1},n)
                -
                \mJ_{H}(r_{p+1},n)
                \mA_{J}(r_{p+1},n)^{-1}
                \mA_{H}(r_{p+1},n)
            \right)
            \begin{bmatrix}
                c_{mn}
                \\
                d_{mn}
            \end{bmatrix}
        \end{equation}

        On en déduit aisément le symbole de la couche \(p+1\)

        \begin{multline}
            \hat{\mZ}_{p+1}(m,n) = -i\eta_p
            \left(
                \mH_{E}(r_{p+1},n)
                \mA_{H}(r_p,n)^{-1}
                -
                \mJ_{E}(r_{p+1},n)
                \mA_{E}(r_p,n)^{-1}
            \right)
            \\
            \left(
                \mH_{H}(r_{p+1},n)
                \mA_{H}(r_p,n)^{-1}
                -
                \mJ_{H}(r_{p+1},n)
                \mA_{E}(r_p,n)^{-1}
            \right)^{-1}
        \end{multline}

        Pour les mêmes raisons que dans le cas d'une couche, ce symbole est diagonal. Cependant son expression n'est plus aussi simple (voir annexe \ref{sec:annex:imp_sphere} ).

  \subsection{Applications numérique}

    \begin{figure}[!hbt]
      \centering
      \begin{tikzpicture}[scale=1]
        \begin{loglogaxis}[
            title={},
            ylabel={\(||\hat{\mZ}_{plan} - \hat{\mZ}_{sphere}||_2\)},
            xlabel={\(r_0/d\)},
            width=0.8\textwidth,
            xmin=0.1,
            xmax=100,
            % mark repeat=20,
            legend pos=outer north east
          ]
          \legend{TM,TE}
          \addplot table [x={r0/d}, y={tm},col sep=semicolon] {tikz/csv/impedance/sphere/hoppe_p62_error.csv};
          \addplot table [x={r0/d}, y={te},col sep=semicolon] {tikz/csv/impedance/sphere/hoppe_p62_error.csv};
        \end{loglogaxis}
      \end{tikzpicture}
      \caption{\(\eps = 6, \mu = 1, d=0.0225\text{m}, f=1\text{GHz}\)}
      \label{fig:imp_fourier:sphere:hoppe_p62:converge_rayon:error}
    \end{figure}