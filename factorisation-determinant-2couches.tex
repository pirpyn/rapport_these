\documentclass{amsart}

\usepackage[utf8]{inputenc}
\usepackage{ amssymb, nccmath}
\usepackage{amsmath}
%\usepackage{showkeys}
\let\proof\relax 
\let\endproof\relax
%\usepackage{amsthm} %theorem environment option
\usepackage{enumerate}
\usepackage{latexsym,amscd,amsfonts}
\usepackage{graphicx}
\usepackage{graphics}
\usepackage{tikz,tkz-tab}
\usepackage{float}

%-------------------------------------------------------------%
\def \Z {\mathbb Z}
\def \R {\mathbb R}
\def \C {\mathbb C}
\def \Q {\mathbb Q}
\def \N {\mathbb N}
\def \T {\mathbb T}
\def \P {\mathbb P}
\def \E {\mathbb E}

\newcommand{\ex}{\mathrm{e}}
\newcommand{\ii}{\mathrm{i}}
\newcommand{\dd}{\mathrm{d}}
\newcommand{\eps}{\varepsilon}
%--------------------------------------------------------------%

\newtheorem{lem}{Lemma}[section]
\newtheorem{defi}{Definition}[section]
\newtheorem{theo}{Theorem}[section]
\newtheorem{prop}{Proposition}[section]
\newtheorem{rmk}{Remark}[section]
%\newtheorem{cor}{Corollary}[section]
%\newtheorem{rem}{Remark}[section]
%\newenvironment{demo}{\noindent \textit{Proof:}}{\begin{flushright} \(\Box\) \end{flushright}}
%--------------------------------------------------------------%
%%%% debut macro %%%%
\makeatletter
\renewcommand\theequation{\thesection.\arabic{equation}}
\@addtoreset{equation}{section}
\makeatother
%%%% fin macro %%%%


\begin{document}


%% Title of the article. 
% The optional argument [] is the short version of the title,
% and the mandatory argument {} the title itself
\title{Factorisation du déterminant dans le cas de deux couches de diélectrique}


%% Authors, addresses and supports.
% The optional argument is for shortened appearing in the headings. Please
% distinguish between first, middle and last names with the appropriate commands.
\author{Olivier Lafitte}
\address{LAGA, Institut Galil\'ee\\
Universit\'e Paris 13\\
Sorbonne Paris Cit\'e
99, avenue J.B. Cl\'ement\\
93430 Villetaneuse \\
France}
% No support for the second author
%\thanks{This work was completed while the author was invited professor at IU Mathematics Department, Bloomington (Spring 2017)}
%\thanks{The author wants to thank the anonymous referee for all his comments for improving the present paper}
%% Keywords
%\keywords{Example, Applied mathematics, Journal}
 
%% Mathematical classification   (2000)
%\subjclass{00X99}
% Abstract.

Tu as raison, je ne t'ai pas transmis mes notes où j'avais réussi à factoriser. Je te les écris ci-dessous que tu puisse recopier directement si tu veux.

\newcommand{\kk}{\tilde{k}}
\renewcommand{\frac}{\dfrac}

J'étais parti, en notant \(\kk_1\) et \(\kk_2\) associées aux couches \(1\) et \(2\), \(z_0<z_1<z_2\):

\[E_x=a_2\sin(\kk_2(z-z_1))+b\cos(\kk_2(z-z_1)), z_1<z<z_2\]
\[E_x=a_1\sin(\kk_1(z-z_1))+b\cos(\kk_1(z-z_1)), z_0<z<z_1\]
\[E_y=c_2\sin(\kk_2(z-z_1))+d\cos(\kk_2(z-z_1)), z_1<z<z_2\]
\[E_y=c_1\sin(\kk_1(z-z_1))+d\cos(\kk_1(z-z_1)), z_0<z<z_1\]
où j'ai déjà utilisé la condition de continuité tangentielle du champ électrique à l'interface \(z=z_1\).

Je définis \(A_2,C_2, A_1, C_1, B, D\) par
% \[
% \left\{
% \begin{array}{ll}
% (a_2,c_2)&=A_2(k_x,k_y)+C_2(-k_y,k_x)
% \cr
% (a_1,c_1)&=A_1(k_x,k_y)+C_1(-k_y,k_x)
% \cr
% (b,d)&=B(k_x,k_y)+D(-k_y,k_x)
% \end{array}
% \right.
% \]
\begin{align*}
a_2 ={}& A_2k_x - C_2k_y, & c_2 ={}& A_2k_y + C_2 k_x,
\\
a_1 ={}& A_1k_x - C_1k_y, & c_1 ={}& A_1k_y + C_1 k_x,
\\
b ={}& Bk_x - Dk_y, & d ={}& Bk_y + D k_x,
\end{align*}
\begin{align*}
A_2 ={}& \frac{a_2 k_x + c_2 k_y}{k_x^2 + k_y^2}, & C_2 ={}& \frac{c_2 k_y - a_2 k_x}{k_x^2 + k_y^2},
\\
A_1 ={}& \frac{a_1 k_x + c_1 k_y}{k_x^2 + k_y^2}, & C_1 ={}& \frac{c_1 k_y - a_1 k_x}{k_x^2 + k_y^2},
\\
B ={}& \frac{b k_x + d k_y}{k_x^2 + k_y^2}, & D ={}& \frac{d k_y - b k_x}{k_x^2 + k_y^2}.
\end{align*}

Les divergences nulle des champs permettent de déduire de \(a_2,c_2,a_1,c_1,b,d\) la décomposition sur \((\sin(\kk_i(z-z_i)),\cos(\kk_i(z-z_i)))\) de \(E_z\):

\[
E_z=i\frac{k_x^2+k_y^2}{\kk_2}A_2\cos(\kk_2(z-z_1))-i\frac{k_x^2+k_y^2}{\kk_2}B\sin(\kk_2(z-z_1)), z_1<z<z_2,
\]
Dans \(z_0<z<z_1\), il faut remplacer \(A_2\) par \(A_1\), \(C_2\) par \(C_1\) et \(\kk_2\) par \(\kk_1\).

Lorsque je cherche les résonances, il suffit que j'écrive toutes les conditions de champ nul sur les bord \(z=z_0\), \(z=z_2\).
Je note \(\alpha_1=\kk_1(z_0-z_1)\), \(\alpha_2=\kk_2(z_2-z_1)\).

Les conditions sur \(E_x\) et sur \(E_y\) donnent, après décomposition sur \((k_x,k_y)\) et sur \((-k_y,k_x)\), en \(z=z_0\) et en \(z=z_2\)
\[
\left\{
\begin{aligned}
B\cos \alpha_1={}& A_1\sin \alpha_1,
\\
D\cos \alpha_1={}& C_1\sin \alpha_1,
\\
B\cos \alpha_2={}& A_2\sin \alpha_2,
\\
D\cos \alpha_2={}& C_2\sin \alpha_2.
\end{aligned}
\right.
\]

D'autre part, on écrit la condition de saut du champ magnétique à l'interface \(z=z_1\).

Je calcule les coordonnées tangentielles de \(F = rot E\) dans \(z_1<z<z_2\):
\begin{multline*}
F_x = -k_y\left(\frac{k_x^2+k_y^2}{\kk_2}A_2\cos(\kk_2(z-z_1))-\frac{k_x^2+k_y^2}{\kk_2}B\sin(\kk_2(z-z_1))\right)
\\
-\kk_2\left((k_yA_2+k_xC_2)\cos(\kk_2(z-z_1))-(Bk_y+Dk_x)\sin(\kk_2(z-z_1))\right)
\end{multline*}
\begin{multline*}
F_y = \kk_1\left((k_xA_2-k_yC_2)\cos(\kk_2(z-z_1))-(k_xB-k_yD)\sin(\kk_2(z-z_1))\right)
\\
+k_x\left(\frac{k_x^2+k_y^2}{\kk_2}A_2\cos(\kk_2(z-z_1))-\frac{k_x^2+k_y^2}{\kk_2}B\sin(\kk_2(z-z_1))\right)
\end{multline*}

D'où l'on déduit les coordonnées tangentielles du champ magnétique dans \(z_1<z<z_2\):
\begin{multline*}
H_x = -\frac{k_y}{-i\omega\mu_2}\left(\frac{k_x^2+k_y^2}{\kk_2}A_2\cos(\kk_2(z-z_1))-\frac{k_x^2+k_y^2}{\kk_2}B\sin(\kk_2(z-z_1))\right)
\\
-\frac{\kk_2}{-i\omega\mu_2}\left((k_yA_2+k_xC_2)\cos(\kk_2(z-z_1))-(Bk_y+Dk_x)\sin(\kk_2(z-z_1))\right)
\end{multline*}
\begin{multline*}
H_y = \frac{\kk_2}{-i\omega\mu_2}\left((k_xA_2-k_yC_2)\cos(\kk_2(z-z_1))-(k_xB-k_yD)\sin(\kk_2(z-z_1))\right)
\\
+\frac{k_x}{-i\omega\mu_2}\left(\frac{k_x^2+k_y^2}{\kk_2}A_2\cos(\kk_2(z-z_1))-\frac{k_x^2+k_y^2}{\kk_2}B\sin(\kk_2(z-z_1))\right)
\end{multline*}
Dans \(z_0<z<z_1\), il faut remplacer \(A_2\) par \(A_1\), \(C_2\) par \(C_1\), \(\mu_2\) par \(\mu_1\) et \(\kk_2\) par \(\kk_1\).


J'en déduis les conditions à vérifier pour satisfaire à la continuité tangentielle du champ magnétique
\[
\left\{
\begin{aligned}
-\frac{k_y}{\mu_1}\frac{k_x^2+k_y^2}{\kk_1}A_1-\frac{\kk_1}{\mu_1}(k_yA_1+k_xC_1)={}&-\frac{k_y}{\mu_2}\frac{k_x^2+k_y^2}{\kk_2}A_2-\frac{\kk_2}{\mu_2}(k_yA_2+k_xC_2),
\\
-\frac{\kk_1}{\mu_1}(k_xA_1-k_yC_1)+\frac{k_x}{\mu_1}\frac{k_x^2+k_y^2}{\kk_1}A_1={}&-\frac{\kk_2}{\mu_2}(k_xA_2-k_yC_2)+\frac{k_x}{\mu_2}\frac{k_x^2+k_y^2}{\kk_2^2}A_2.
\end{aligned}
\right.
\]

On a un système de 6 équations à 6 inconnues.

On introduit
\begin{align*}
M_1={}&
\begin{pmatrix}
-\kk_1k_y-k_y\frac{k_x^2+k_y^2}{\kk_1}  &   -\kk_1k_x
\cr
\kk_1k_x+k_x\frac{k_x^2+k_y^2}{\kk_1}   &   -\kk_1k_y
\end{pmatrix},
&
M_2={}&
\begin{pmatrix}
-\kk_2k_y-k_y\frac{k_x^2+k_y^2}{\kk_2}  &   -\kk_2k_x
\cr
\kk_2k_x+k_x\frac{k_x^2+k_y^2}{\kk_2}   &   -\kk_2k_y
\end{pmatrix}.
\end{align*}

La continuité tangentielle du champ magnétique se résume à 
\[
\frac{1}{\mu_1}M_1\begin{pmatrix}A_1\\C_1\end{pmatrix}=\frac{1}{\mu_2}M_2\begin{pmatrix}A_2\\C_2\end{pmatrix}.
\]
J'aboutis, en multipliant par \(\sin \alpha_1\sin \alpha_2\) puis en utilisant les relations en fonction de \(B\) et \(D\) :

\[
\mu_2\sin\alpha_2M_1\begin{pmatrix}A_1\sin \alpha_1\\C_1\sin \alpha_1\end{pmatrix}
=
\mu_1\sin \alpha_1M_2\begin{pmatrix}A_2\sin \alpha_2\\C_2\sin \alpha_2\end{pmatrix},
\]

\[
\mu_2\sin \alpha_2\cos \alpha_1 M_1\begin{pmatrix}B\\D\end{pmatrix}
=
\mu_1\sin \alpha_1\cos \alpha_2M_2\begin{pmatrix}B\\D\end{pmatrix},
\]
et le problème de résonance se ramène à 

\[
\operatorname{det}(K_1M_1-K_2M_2)=0
\]
avec \(K_1=\mu_2\cos \alpha_1\sin \alpha_2\), \(K_2=\mu_1\cos \alpha_2\sin \alpha_1\).

J'observe alors que
\[
M_1=
\begin{pmatrix}
-\kk_1k_y-k_y\frac{k_x^2+k_y^2}{\kk_1}  &   -\kk_1k_x
\\
\kk_1k_x+k_x\frac{k_x^2+k_y^2}{\kk_1}   &   -\kk_1k_y
\end{pmatrix}
=
\begin{pmatrix}
-k_y\frac{k_1^2}{\kk_1}   &   -\kk_1k_x
\\
k_x\frac{k_1^2}{\kk_1}    &   -\kk_1k_y
\end{pmatrix},
\]
\(M_2\) s'en déduit.

En calculant directement le déterminant de \(K_1M_1+K_2M_2\), on remarque que \(k_x^2+k_y^2\) est en facteur. Si ce coefficient est non nul, j'obtiens l'équation
\[
\epsilon_1\mu_1K_1^2+(\epsilon_1\mu_1\frac{\kk_2}{\kk_1}+\epsilon_2\mu_2\frac{\kk_1}{\kk_2})K_1K_2+\epsilon_2\mu_2K_2^2=0.
\]
J'avais alors noté \(X=\sqrt{\epsilon_1\mu_1}\kk_1\), \(Y=\sqrt{\epsilon_2\mu_2}\kk_2\), et je trouve
\[
X^2+[\sqrt{\frac{\epsilon_2\mu_2}{\epsilon_1\mu_1}\frac{\kk_1}{\kk_2}}+\sqrt{\frac{\epsilon_1\mu_1}{\epsilon_2\mu_2}\frac{\kk_2}{\kk_1}}]XY+Y^2
=
0
=
(X+[\sqrt{\frac{\epsilon_2\mu_2}{\epsilon_1\mu_1}\frac{\kk_1}{\kk_2}}Y)(X+\sqrt{\frac{\epsilon_1\mu_1}{\epsilon_2\mu_2}\frac{\kk_2}{\kk_1}}Y).
\]
J'ai donc obtenu les deux égalités totalement factorisées résumées ci-dessus.

\end{document}










