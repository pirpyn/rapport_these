\chapter{Calcul des coefficients pour une sphère}
\label{sec:sphere}
\minitoc
\newpage
\sectionstar{Introduction}
Après le cylindre, nous présentons la sphère, qui possède une courbure dans deux directions. En réalité, cette géométrie est plus simple que la précédente et on s'attend à avoir de meilleurs résultats. 

\section{Cas d'un objet sphérique}

    Les champs solutions de Maxwell dans le cas d'un repère sphérique sont décomposables en harmoniques sphériques. Nous rappelons d'abord l’expression de ces dernières puis nous donnerons l'expression du symbole de l'opérateur de d'impédance de la même manière que \cite{cheng_spectral_1993}.

    \subsection{Les harmoniques sphériques}

        \begin{TODO}
          Mettre ici la démonstration des harmoniques sphériques? Ou une référence vers annexe ?
        \end{TODO}

        On définit les harmoniques sphériques les solutions de \(\Delta U + k^2 U = 0 \). Ce sont les fonctions \(Y_{m,n} = C(m,n) e^{im\phi}\PP^m_n(\cos \theta) \) avec \(C(m,n)\)\footnote{D’après \cite[p.~24]{nedelec_acoustic_2001}, \( C(m,n) = (-1)^m\sqrt{\frac{2n+1}{4\pi}\frac{(n-m)!}{(n+m)!}}\)} tel que
        \[
         \ds\int_S Y_{m,n} \conj{Y_{p,q}} ds = \delta_m^p \delta_n^q
        \]


        On définit les vecteurs harmonique sphériques\(\gls{phy-Mmn} ,\gls{phy-Nmn}\) solution dans la base des coordonnées sphériques de
        \[
            \left\lbrace
                \begin{aligned}
                    \vrot \vrot \vect{U} - k^2 \vect{U} = 0\\
                    \vdiv \vect{U} = 0
                \end{aligned}
            \right.
        \]
        \begin{align}
            \label{eq:defMmn}
            \Mmn[z_n](\rtp) &:= \vrot \left( \vect{r} z_n(kr) Y_{m,n}(\tp) \right)\\
            &= z_n(kr)
            \begin{bmatrix}
                0
                \\
                \frac{im}{\sin\theta}Y_{mn}(\tp)
                \\
                - \ddr{\theta}{Y_{mn}}(\tp)
            \end{bmatrix}
        \end{align}

        \begin{align}
        \label{eq:defNnn}
          \Nmn[z_n](\rtp) &:= \frac{\vrot \Mmn[z_n]}{k}(\rtp) \\
          &= \frac{1}{kr}\begin{bmatrix}
            z_n(kr)n(n+1)Y_{mn}(\tp)
            \\
            \ddr{r}{z_n}(kr)\ddr{\theta}{Y_{mn}}(\tp)
            \\
            \ddr{r}{z_n}(kr)\frac{im}{\sin\theta}Y_{mn}(\tp)
          \end{bmatrix}
        \end{align}

        L'obtention de ces vecteurs est disponible en annexes \ref{sec:annex:harmoniques_spheriques}.

        Par définition de ces vecteurs, on a les propriétés suivantes
        \begin{prop}
            \label{prop:Mmn_Nmn_rot}
            \begin{align}
                \vrot \Mmn[z_n](\rtp) &= k\Nmn[z_n](\rtp)
                \\
                \vrot \Nmn[z_n](\rtp) &= k\Mmn[z_n](\rtp)
            \end{align}
        \end{prop}

        % Ces vecteurs harmoniques sphériques possèdent les propriétés suivantes

        % \begin{align}
        % \int_{S(0,R)} \vect{M_{m,n}^{z_n}} \cdot \conj{\vect{N_{p,q}^{z_n}}} ds &= 0
        % \\
        % \int_{S(0,R)} \vect{M_{m,n}^{z_n}} \cdot \conj{\vect{M_{p,q}^{z_n}}} ds &= \gamma_{m,n}R^2 \delta_{mp}\delta_{nq}
        % \\
        % \int_{S(0,R)} \vect{N_{m,n}^{z_n}} \cdot \conj{\vect{N_{p,q}^{z_n}}} ds &= \frac{\gamma_{m,n}}{k^2} \delta_{mp}\delta_{nq}
        % \end{align}

        On a (\cite{cheng_spectral_1993})
        \begin{multline}
            \vE(\rtp) = \sum_{n\in\ZZ}\sum_{m\in\ZZ} a_{mn} \Mmn[j_n](\rtp) + b_{mn} \Nmn[j_n](\rtp)
            \\
            + c_{mn} \Mmn[h_n](\rtp) + d_{mn} \Nmn[h_n](\rtp)
        \end{multline}

        D'après les équations de Maxwell, \(\vH = i\frac{\vrot \vE}{k\eta}\)

        \begin{multline}
            \vH(\rtp) = \frac{i}{\eta}\sum_{n\in\ZZ}\sum_{m\in\ZZ} a_{mn} \Nmn[j_n](\rtp) + b_{mn} \Mmn[j_n](\rtp)
            \\
            + c_{mn} \Nmn[h_n](\rtp) + d_{mn} \Mmn[h_n](\rtp)
        \end{multline}

        On définit alors le vecteur de \(\RR^2\) suivant ( resp. son orthogonal ), la réduction du vecteur harmonique sphérique \gls{phy-Mmn} ( resp. \gls{phy-Nmn} ) aux composantes tangentielles et indépendant du rayon:

        \begin{align}
            \label{eq:defUmn_tgt}
            \Umn(\tp) &=
            \begin{bmatrix}
                \frac{im}{\sin\theta}Y_{mn}(\tp)
                \\
                - \ddr{\theta}{Y_{mn}}(\tp)
            \end{bmatrix}
        \end{align}

        \begin{align}
        \label{eq:defNmn_tgt}
          \Umn^\perp(\tp) &=
          \begin{bmatrix}
            \ddr{\theta}{Y_{mn}}(\tp)
            \\
            \frac{im}{\sin\theta}Y_{mn}(\tp)
          \end{bmatrix}
        \end{align}


        On remarque alors que les parties tangentielles des vecteurs harmoniques sphériques peuvent s'écrire:
        \begin{align}
          \Mmn[z_n]_t(\rtp) &= z_n(kr)\Umn(\tp)
          \\
          \Nmn[z_n]_t(\rtp) &= \frac{1}{kr}\ddr{r}{z_n}(kr)\Umn^\perp(\tp)
        \end{align}

        On a alors les propriétés supplémentaires
        \begin{prop}
            \label{prop:Mmn_Nmn_vect}
            \begin{align}
              \vect{e_r} \pvect \Mmn[z_n]_t(\rtp) &= krz_n(kr)\Umn^\perp(\tp)
              \\
              \vect{e_r} \pvect \Nmn[z_n]_t(\rtp) &= -\frac{1}{kr}\ddr{r}{z_n}(kr)\Umn(\tp)
            \end{align}
        \end{prop}
        Sachant donc que les composantes tangentielles du champs \(\vE\) s'écrivent

        \begin{multline}
            \vE_t(\rtp) = \sum_{n\in\ZZ}\sum_{m\in\ZZ} a_{mn} j_n(kr)\Umn(\tp) + b_{mn} \frac{1}{kr}\ddr{r}{j_n}(kr)\Umn^\perp(\tp)
            \\
            + c_{mn} h_n(kr)\Umn(\tp) + d_{mn} \frac{1}{kr}\ddr{r}{j_n}(kr)\Umn^\perp(\tp)
        \end{multline}


        Donc \(\vJ = \vect{e_r} \pvect \vH\) s'écrit

        \begin{multline}
            \vJ(\rtp) = \frac{i}{\eta}\sum_{n\in\ZZ}\sum_{m\in\ZZ} - a_{mn} \frac{1}{kr}\ddr{r}{j_n}\Umn(\tp) + b_{mn} k r j_n \Umn^\perp(\tp)
            \\
            -  \frac{1}{kr}\ddr{r}{h_n} c_{mn} \Umn(\tp) + k r h_n d_{mn} \Umn^\perp(\tp)
        \end{multline}

        Dans la suite, pour simplifier les écritures, on utilisera la notation tilde \gls{mat-tild}, telle que \( \tilde{z_n}(k_r) = \ddr{r}{z_n}(kr) \). On omettra aussi les dépendances en \(kr\) lorsqu'il n'y a pas d’ambiguïtés.

        On réécrit alors matriciellement les expressions de \(\vE_t,\vJ\).

        \begin{equation}
            \vE_t(\rtp) = \sum_{n\in\ZZ}\sum_{m\in\ZZ}
            \begin{bmatrix}
              \Umn^\perp & \Umn
            \end{bmatrix}
            \left(
              \begin{bmatrix}
                  0 & \tilde{j_n}
                  \\
                  j_n & 0
              \end{bmatrix}
              \begin{bmatrix}
                  a_{mn}
                  \\
                  b_{mn}
              \end{bmatrix}
              +
              \begin{bmatrix}
                  0 & \tilde{h_n}
                  \\
                  h_n & 0
              \end{bmatrix}
              \begin{bmatrix}
                  c_{mn}
                  \\
                  d_{mn}
              \end{bmatrix}
            \right)
        \end{equation}


        \begin{equation}
            \vJ(\rtp) = \frac{i}{\eta kr}\sum_{n\in\ZZ}\sum_{m\in\ZZ}
            \begin{bmatrix}
                \Umn^\perp & \Umn
            \end{bmatrix}
            \left(
                \begin{bmatrix}
                    0 & (kr)^2 j_n
                    \\
                    -\tilde{j_n} & 0
                \end{bmatrix}
                \begin{bmatrix}
                    a_{mn}
                    \\
                    b_{mn}
                \end{bmatrix}
                +
                \begin{bmatrix}
                    0 & (kr)^2 h_n
                    \\
                    -\tilde{h_n} & 0
                \end{bmatrix}
                \begin{bmatrix}
                    c_{mn}
                    \\
                    d_{mn}
                \end{bmatrix}
            \right)
        \end{equation}

        \begin{defn}
            On définit les vecteurs de \(\RR^2\) \(\hat{\vE_t}(r,m,n)\) et \(\hat{\vJ}(r,m,n)\) tels que
            \begin{align}
                \vE_t(\rtp) &= \sum_{n\in\ZZ}\sum_{m\in\ZZ}
                \begin{bmatrix}
                  \Umn^\perp & \Umn
                \end{bmatrix}\hat{\vE_t}(r,m,n)
                \\
                \vJ(\rtp) &= \sum_{n\in\ZZ}\sum_{m\in\ZZ}
                \begin{bmatrix}
                  \Umn^\perp & \Umn
                \end{bmatrix}\hat{\vJ}(r,m,n)
            \end{align}
        \end{defn}

        \begin{defn}
            On définit les matrices \(\mJ_{E}(r,n),\mH_{E}(r,n),\mJ_{H}(r,n),\mH_{H}(r,n)\)
            \begin{align}
                \mJ_{E}(r,n) &=
                \begin{bmatrix}
                    0 & \tilde{j_n}(kr)
                    \\
                    j_n(kr) & 0
                \end{bmatrix}
                \\
                \mH_{E}(r,n) &=
                \begin{bmatrix}
                    0 & \tilde{h_n}(kr)
                    \\
                    h_n(kr) & 0
                \end{bmatrix}
                \\
                \mJ_{H}(r,n) &=
                \begin{bmatrix}
                    0 & (kr)^2 j_n(kr)
                    \\
                    -\tilde{j_n}(kr) & 0
                \end{bmatrix}
                \\
                \mH_{H}(r,n) &=
                \begin{bmatrix}
                    0 & (kr)^2 h_n(kr)
                    \\
                    -\tilde{h_n}(kr) & 0
                \end{bmatrix}
            \end{align}
        \end{defn}

        On peut donc expliciter les vecteurs précédemment introduits

        \begin{equation}
            \hat{\vE_t}(r,m,n) =
            \mJ_{E}(r,n)
            \begin{bmatrix}
                a_{mn}
                \\
                b_{mn}
            \end{bmatrix}
            +
            \mH_{E}(r,n)
            \begin{bmatrix}
                c_{mn}
                \\
                d_{mn}
            \end{bmatrix}
        \end{equation}

        \begin{equation}
            \hat{\vJ}(r,m,n) = \frac{i}{\eta k r}
            \left(
            \mJ_{H}(r,n)
            \begin{bmatrix}
                a_{mn}
                \\
                b_{mn}
            \end{bmatrix}
            +
            \mH_{H}(r,n)
            \begin{bmatrix}
                c_{mn}
                \\
                d_{mn}
            \end{bmatrix}
            \right)
        \end{equation}

    \subsection{Symbole de l'opérateur d'impédance pour une couche}

        \begin{figure}[!hbt]
          \centering
            \tikzsetnextfilename{sphere_1_couche}
          \begin{tikzpicture}
            \tikzmath{
    \a = 80;
    \b = 100;
    \d = 0.5;
    \ri = 20;
    \re = \ri;
}

% Le conducteur
\tikzmath{
    \ri = \re;
    \re = \ri + 0.5*\d;
    \xa = cos(\a)*\re;
    \ya = sin(\a)*\re;
    \xb = cos(\b)*\ri;
    \yb = sin(\b)*\ri;
}

\coordinate (a) at (\xa,\ya);
\coordinate (b) at (\xb,\yb);

\fill [pattern=north east lines] (a) arc (\a:\b:\re) -- (b) arc (\b:\a:\ri) -- cycle;
\draw (a) arc (\a:\b:\re);
\draw (a) node [right] {$r_0$};


% Le repère
\coordinate (n) at ($(a)+(0.5,-1)$);
%
%
%\draw [->] (n) -- ++(0,1) node [at end, right] {$\v{\pr}$};
%\draw [->] (n) -- ++(1,0) node [at end, right] {$\v{\pt}$};
%
\draw (n) ++(0.2,0.2) circle(0.1cm) node [above=0.1cm] {\(\vect{e_\phi}\)};
\draw (n) ++(0.2,0.2) +(135:0.1cm) -- +(315:0.1cm);
\draw (n) ++(0.2,0.2) +(45:0.1cm) -- +(225:0.1cm);


% 1ere couche
\tikzmath{
    \ri = \re;
    \re = \ri + \d;
    \xa = cos(\a)*\re;
    \ya = sin(\a)*\re;
    \xb = cos(\b)*\ri;
    \yb = sin(\b)*\ri;
    \xc = cos(0.5*(\b+\a))*(\ri+0.5*\d);
    \yc = sin(0.5*(\b+\a))*(\ri+0.5*\d);
}

\coordinate (a) at (\xa,\ya);
\coordinate (b) at (\xb,\yb);
\coordinate (c) at (\xc,\yc);

\fill [lightgray] (a) arc (\a:\b:\re) -- (b) arc (\b:\a:\ri) -- cycle;
\draw (a) arc (\a:\b:\re);
\draw (c) node {$\nu,\eta,d$};

% Le vide
\tikzmath{
    \xc = cos(0.5*(\b+\a))*(\re);
    \yc = sin(0.5*(\b+\a))*(\re);
}

\draw (\xc,\yc) node [above] {vide};
          \end{tikzpicture}
        \end{figure}

        \begin{defn}
          On définit le symbole de l'opérateur d'impédance \(\hat{\mZ}(m,n)\) tel que
          \[
              \hat{\vE_t}(r_1,m,n) = \hat{\mZ}(m,n)\hat{\vJ}(r_1,m,n)
          \]
        \end{defn}

        En \(r=r_0\), on a la relation \(\vE_t(\rtp) = 0\) donc \(\hat{\vE_t}(r_0,m,n) = 0 \)

        \begin{equation}
            \mJ_{E}(r_0,n)
            \begin{bmatrix}
                a_{mn}
                \\
                b_{mn}
            \end{bmatrix}
            = -
            \mH_{E}(r_0,n)
            \begin{bmatrix}
                c_{mn}
                \\
                d_{mn}
            \end{bmatrix}
        \end{equation}

        On suppose que les matrices \(\mJ_{E}(r_0,n)\) et \(\mH_{E}(r_0,n)\) soient inversibles.

        \begin{TODO}
          Inversibilité \(\mJ_{E}(r,n), \mH_{E}(r,n)\)
        \end{TODO}

        \begin{equation}
            \hat{\vE_t}(r,m,n) =
            \left(
                \mH_{E}(r,n)
                -
                \mJ_{E}(r,n)
                \mJ_{E}(r_0,n)^{-1}
                \mH_{E}(r_0,n)
            \right)
            \begin{bmatrix}
                c_{mn}
                \\
                d_{mn}
            \end{bmatrix}
        \end{equation}


        \begin{equation}
            \hat{\vJ}(r,m,n) = \frac{i}{\eta}
            \left(
                \mH_{H}(r,n)
                -
                \mJ_{H}(r,n)
                \mJ_{E}(r_0,n)^{-1}
                \mH_{E}(r_0,n)
            \right)
            \begin{bmatrix}
                c_{mn}
                \\
                d_{mn}
            \end{bmatrix}
        \end{equation}

        De la même manière que pour le plan et le cylindre, on en déduit le symbole de l'opérateur d'impédance

        \begin{multline}
            \hat{\mZ}(m,n) = -i\eta
            \left(
                \mH_{E}(r_1,n)
                \mH_{E}(r_0,n)^{-1}
                -
                \mJ_{E}(r_1,n)
                \mJ_{E}(r_0,n)^{-1}
            \right)
            \\
            \left(
                \mH_{H}(r_1,n)
                \mH_{E}(r_0,n)^{-1}
                -
                \mJ_{H}(r_1,n)
                \mJ_{E}(r_0,n)^{-1}
            \right)^{-1}
        \end{multline}

        Par définition des matrices \(\mJ_E,\mH_E,\mJ_H,\mH_H\), elle sont anti-diagonale. Donc leur inverse l'est aussi. Donc le produit de l'une avec l'inverse d'une autre est une matrice diagonale. Donc le symbole de l'opérateur d'impédance est une matrice diagonale.

        \begin{equation}
            \hat{\mZ}(m,n) = -i\eta
            \begin{bmatrix}
                \frac
                {\tilde{h_n}(kr_1)\tilde{j_n}(kr_0)-\tilde{j_n}(kr_1)\tilde{h_n}(kr_0)}
                {{h_n}(kr_1)\tilde{j_n}(kr_0)-{j_n}(kr_1)\tilde{h_n}(kr_0)} & 0
                \\
                0 & \frac
                {{j_n}(kr_1){h_n}(kr_0)-{h_n}(kr_1){j_n}(kr_0)}
                {\tilde{h_n}(kr_1){j_n}(kr_0)-\tilde{j_n}(kr_1){h_n}(kr_0)}
            \end{bmatrix}
        \end{equation}

    \subsection{Symbole de l'opérateur d'impédance pour plusieurs couche}

        \begin{TODO}
            Les matrices J,H dépendent de k donc de nu qui depend de la couche. Corriger.
        \end{TODO}

        \begin{figure}[!hbt]
          \centering
            \tikzsetnextfilename{sphere_n_couche}
          \begin{tikzpicture}
            \tikzmath{
    \a = 83;
    \b = 97;
    \d = 0.5;
    \ri = 30;
    \re = \ri;
}

% Le conducteur
\tikzmath{
    \ri = \re;
    \re = \ri + 0.5*\d;
    \xa = cos(\a)*\re;
    \ya = sin(\a)*\re;
    \xb = cos(\b)*\ri;
    \yb = sin(\b)*\ri;
}

\coordinate (a) at (\xa,\ya);
\coordinate (b) at (\xb,\yb);

\fill [pattern=north east lines] (a) arc (\a:\b:\re) -- (b) arc (\b:\a:\ri) -- cycle;
\draw (a) arc (\a:\b:\re);
\draw (a) node [right] {$r_0$};

% Le repère
\coordinate (n) at ($(a)+(0.5,-1)$);
%
%
%\draw [->] (n) -- ++(0,1) node [at end, right] {$\v{\pr}$};
%\draw [->] (n) -- ++(1,0) node [at end, right] {$\v{\pt}$};
%
\draw (n) ++(0.2,0.2) circle(0.1cm) node [above=0.1cm] {$\vect{e_\phi}$};
\draw (n) ++(0.2,0.2) +(135:0.1cm) -- +(315:0.1cm);
\draw (n) ++(0.2,0.2) +(45:0.1cm) -- +(225:0.1cm);

% 1 ere couche

\tikzmath{
    \ri = \re;
    \re = \ri + \d;
    \xa = cos(\a)*\re;
    \ya = sin(\a)*\re;
    \xb = cos(\b)*\ri;
    \yb = sin(\b)*\ri;
    \xc = cos(0.5*(\b+\a))*(\ri+0.5*\d);
    \yc = sin(0.5*(\b+\a))*(\ri+0.5*\d);
}

\coordinate (a) at (\xa,\ya);
\coordinate (b) at (\xb,\yb);
\coordinate (c) at (\xc,\yc);

\fill [lightgray] (a) arc (\a:\b:\re) -- (b) arc (\b:\a:\ri) -- cycle;
\draw (a) arc (\a:\b:\re);
\draw (c) node {$\nu_1,\eta_1,d_1$};


% Des couches

\tikzmath{
    \ri = \re;
    \re = \ri + 2*\d;
    \xa = cos(\a)*\re;
    \ya = sin(\a)*\re;
    \xb = cos(\b)*\ri;
    \yb = sin(\b)*\ri;
    \xc = cos(0.5*(\b+\a))*(\ri+0.5*\d);
    \yc = sin(0.5*(\b+\a))*(\ri+0.5*\d);
}

\coordinate (a) at (\xa,\ya);
\coordinate (b) at (\xb,\yb);
\coordinate (c) at (\xc,\yc);

\fill [lightgray]    (a) arc (\a:\b:\re) -- (b) arc (\b:\a:\ri) -- cycle;
\fill [pattern=dots] (a) arc (\a:\b:\re) -- (b) arc (\b:\a:\ri) -- cycle;
\draw (a) arc (\a:\b:\re);

% n eme couche

\tikzmath{
    \ri = \re;
    \re = \ri + \d;
    \xa = cos(\a)*\re;
    \ya = sin(\a)*\re;
    \xb = cos(\b)*\ri;
    \yb = sin(\b)*\ri;
    \xc = cos(0.5*(\b+\a))*(\ri+0.5*\d);
    \yc = sin(0.5*(\b+\a))*(\ri+0.5*\d);
}

\coordinate (a) at (\xa,\ya);
\coordinate (b) at (\xb,\yb);
\coordinate (c) at (\xc,\yc);

\fill [lightgray] (a) arc (\a:\b:\re) -- (b) arc (\b:\a:\ri) -- cycle;
\draw (a) arc (\a:\b:\re);
\draw (c) node {$\nu_p,\eta_p,d_p$};

% Le vide
\tikzmath{
    \xc = cos(0.5*(\b+\a))*(\re);
    \yc = sin(0.5*(\b+\a))*(\re);
}

\draw (\xc,\yc) node [above] {vide};


          \end{tikzpicture}
        \end{figure}


        \begin{defn}
          Pour chaque couche \(p\), on définit le symbole de l'opérateur d'impédance \(\hat{\mZ}_p(m,n)\) tel que
          \[
              \hat{\vE_t}(r_p,m,n) = \hat{\mZ}_p(m,n)\hat{\vJ}(r_p,m,n)
          \]
        \end{defn}

        On résonne par récurrence: on suppose connu le symbole de l'opérateur d'impédance de la couche \(p\) et on cherche le suivant

        En \(r=r_{p}=r_0+\sum_{i=1}^p d_p\), on a la relation \( \hat{\vE_t}(r_p,m,n) = \hat{\mZ}_p(m,n)\hat{\vJ}(r_p,m,n)\) où \(\hat{\mZ}_p(m,n)\) est un matrice diagonale

        \begin{equation}
            \left(\mJ_{E}(r_p,n) - \frac{i}{\eta_p}\hat{\mZ}_p(m,n)\mJ_{H}(r_p,n) \right)
            \begin{bmatrix}
                a_{mn}
                \\
                b_{mn}
            \end{bmatrix}
            = -
            \left(\mH_{E}(r_p,n) - \frac{i}{\eta_p}\hat{\mZ}_p(m,n)\mH_{H}(r_p,n) \right)
            \begin{bmatrix}
                c_{mn}
                \\
                d_{mn}
            \end{bmatrix}
        \end{equation}

        On définit les matrices \(\mA_{J}(r,n)\) et \(\mA_{H}(r,n)\) telle que

        \begin{align}
            \mA_{J}(r,n) &= \mJ_{E}(r,n) - \frac{i}{\eta_p}\hat{\mZ}_p(m,n)\mJ_{H}(r,n)
            \\
            \mA_{H}(r,n) &= \mH_{E}(r,n) - \frac{i}{\eta_p}\hat{\mZ}_p(m,n)\mH_{H}(r,n)
        \end{align}

        On suppose que les matrices \(\mA_{E}(r,n)\) et \(\mA_{H}(r,n)\) soient inversibles.

        \begin{TODO}
          Inversibilité \(\mA_{E}(r_p,n)\) et \(\mA_{H}(r_p,n)\)
        \end{TODO}

        Par hypothèse sur \(\hat{\mZ_p}(m,n)\), ces matrices sont anti-diagonale.

        On en déduit

        \begin{equation}
            \hat{\vE_t}(r_{p+1},m,n) =
            \left(
                \mH_{E}(r_{p+1},n)
                -
                \mJ_{E}(r_{p+1},n)
                \mA_{J}(r_p,n)^{-1}
                \mA_{H}(r_p,n)
            \right)
            \begin{bmatrix}
                c_{mn}
                \\
                d_{mn}
            \end{bmatrix}
        \end{equation}


        \begin{equation}
            \hat{\vJ}(r_{p+1},m,n) = \frac{i}{\eta_p}
            \left(
                \mH_{H}(r_{p+1},n)
                -
                \mJ_{H}(r_{p+1},n)
                \mA_{J}(r_{p+1},n)^{-1}
                \mA_{H}(r_{p+1},n)
            \right)
            \begin{bmatrix}
                c_{mn}
                \\
                d_{mn}
            \end{bmatrix}
        \end{equation}

        On en déduit aisément le symbole de la couche \(p+1\)

        \begin{multline}
            \hat{\mZ}_{p+1}(m,n) = -i\eta_p
            \left(
                \mH_{E}(r_{p+1},n)
                \mA_{H}(r_p,n)^{-1}
                -
                \mJ_{E}(r_{p+1},n)
                \mA_{E}(r_p,n)^{-1}
            \right)
            \\
            \left(
                \mH_{H}(r_{p+1},n)
                \mA_{H}(r_p,n)^{-1}
                -
                \mJ_{H}(r_{p+1},n)
                \mA_{E}(r_p,n)^{-1}
            \right)^{-1}
        \end{multline}

        Pour les mêmes raisons que dans le cas d'une couche, ce symbole est diagonal. Cependant son expression n'est plus aussi simple (voir annexe \ref{sec:annex:imp_sphere} ).

  \subsection{Applications numérique}

    \begin{figure}[!hbt]
        \centering
        \tikzsetnextfilename{Z_HOPPE_62_sphere_erreur}
\begin{tikzpicture}[scale=1]
\begin{loglogaxis}[
    title={},
    ylabel={\(||\hat{\mZ}_{plan} - \hat{\mZ}_{sphere}||_2\)},
    xlabel={\(r_0/d\)},
    width=0.8\textwidth,
    xmin=0.1,
    xmax=100,
    % mark repeat=20,
    legend pos=outer north east
  ]
  \legend{TM,TE}
  \addplot [black] table [x={r0/d}, y={tm},col sep=semicolon] {csv/sphere/hoppe_p62_error.csv};
  \addplot [black,dashed] table [x={r0/d}, y={te},col sep=semicolon] {csv/sphere/hoppe_p62_error.csv};
\end{loglogaxis}
\end{tikzpicture}
        \caption{\(\eps = 6, \mu = 1, d=0.0225\text{m}, f=1\text{GHz}\)}
        \label{fig:imp_fourier:sphere:hoppe_p62:converge_rayon:error}
    \end{figure}
\section{Approximation de la matrice d'impédance pour une sphère par une CIOE}

  \subsection[Expression des opérateurs LD,LR en Fourier]{Expression des opérateurs \(\LD,\LR\) en Fourier}

    Par définition de \(\LD\), on a
    \begin{align}
      \LD \vE_t & = \vgrads{} \vdivs{} \vE_t
    \end{align}
    Or 
    \begin{align*}
      \vE_t(\rtp) &= \sum_{n\in\ZZ}\sum_{m\in\ZZ} (a_{mn} j_n + c_{mn}h_n) \Umn(\rtp) + (b_{mn} \tilde{j_n}+d_{mn} \tilde{h_n}) \Umn^\perp(\rtp)
      %\intertext{donc}
      %\LD\vE_t(\rtp) &= \sum_{n\in\ZZ}\sum_{m\in\ZZ} (a_{mn} j_n + c_{mn}h_n) \LD\Umn(\rtp) + (b_{mn} \tilde{j_n}+d_{mn} \tilde{h_n}) \LD\Umn^\perp(\rtp)
    \end{align*}

    On rappelle  les expressions des vecteurs (voir \eqref{eq:defUmn_tgt}, \eqref{eq:defNmn_tgt}) dans la base sphérique (\(\vect{e_r},\vect{e_\theta},\vect{e_\phi}\))
    \begin{align*}
      \Umn(\tp) =
      \begin{bmatrix}
          0
          \\
          \frac{im}{\sin\theta}\Pmn(\cos(\theta))e^{im\phi}
          \\
          - \ddr{\theta}{\Pmn(\cos(\theta))}e^{im\phi}
      \end{bmatrix}
      &&
      \Umn^\perp(\tp) =
      \begin{bmatrix}
        0
        \\
        \ddr{\theta}{\Pmn(\cos(\theta))}e^{im\phi}
        \\
        \frac{im}{\sin\theta}\Pmn(\cos(\theta))e^{im\phi}
      \end{bmatrix}
    \end{align*}

    On commence par calculer le divergent surfacique (cf annexe \ref{sec:annexe:div_grad_rot}) en sphérique
    \begin{align*}
      \vdivs{}\Umn(\rtp) &= \frac{1}{r\sin(\theta)} \ddr{\theta}{(\sin(\theta)\Umn_\theta)} + \frac{1}{r\sin(\theta)}\ddr{\phi}{(\Umn_\phi)}
      \\
      &=\frac{ime^{im\phi}}{r\sin(\theta)}\left( \ddr{\theta}{\Pmn(\cos(\theta))} - \ddr{\theta}{\Pmn(\cos(\theta))} \right)
      \\
      &= 0
    \end{align*}
    Donc 
    \begin{align*}
      \LD\Umn(\rtp) = 0
    \end{align*}

    Calculons maintenant l'action de \(\LD\) sur \(\Umn^\perp\)
    \begin{align*}
      \vdivs{}\Umn^\perp(\rtp) &= \frac{1}{r\sin(\theta)} \ddr{\theta}{(\sin(\theta)\Umn^\perp_\theta)} + \frac{1}{r\sin(\theta)}\ddr{\phi}{(\Umn^\perp_\phi)}
      \\
      &= \frac{e^{im\phi}}{r\sin(\theta)}
      \left(
        \ddr{\theta}{}\left(\sin(\theta)\ddr{\theta}{\Pmn(\cos(\theta))}\right) - \frac{m^2}{\sin\theta}\Pmn(\cos(\theta))
      \right)
    \end{align*}
    Or d’après \cite[\href{https://dlmf.nist.gov/14.10}{sec.~14.10}]{dlmf_nist_2019} les fonctions de Legendre sont récurrentes
    \begin{align}
      \sin(\theta)\ddr{\theta}{\Pmn(\cos(\theta))} &= (n+m)\PP_{n-1}^m(\cos(\theta)) - n\cos(\theta)\Pmn(\cos(\theta))
    \end{align}

    Donc 
    \begin{align*}
      \LD\Umn^\perp(\rtp) = -\frac{n(n+1)}{r^2}\Umn^\perp(\rtp)
    \end{align*}

    On utilise les résultats de \cite{marceaux_high-order_2000}

    \begin{align*}
      \vgrads{}\vdivs{} \Mmn[z_n]_t(r,\theta,\phi) &= 0
      \\
      \vgrads{}\vdivs{} \Nmn[z_n]_t(r,\theta,\phi) &= -\frac{n(n+1)}{r^2}\Nmn[z_n]_t(r,\theta,\phi)
    \end{align*}

    On définit \(\hat{\mLD}\) l'opérateur matriciel tel que
    \begin{align}
      \LD \vE_t (r_{ext},\theta,\phi)
      &= \frac{1}{2\pi}\sum_{n=-\infty}^\infty\sum_{m=-n}^n \hat{\mLD} \hat{\vE_t}(r_{ext},n,m)
    \end{align}

    Son expression est de ce qui précède
    \begin{equation}
      \label{eq:cylindre:fourier:LD}
      \hat{\mLD}(n,m) = -
      \begin{bmatrix}
        0 & 0
        \\
        0 & \frac{n(n+1)}{r_{ext}^2}
      \end{bmatrix}
    \end{equation}

    On reprend exactement la même méthode pour l'opérateur \(\LR\).
    Par définition de \(\LR\), on a
    \begin{align}
      \LR \vE_t & = \vrots{} \vrots{} \vE_t
    \end{align}

    On utilise les résultats de \cite{marceaux_high-order_2000}

    \begin{align*}
      \vrots{}\vrots{} \Mmn[z_n]_t(r,\theta,\phi) &= \frac{n(n+1)}{r^2}\Mmn[z_n]_t(r,\theta,\phi)
      \\
      \vrots{}\vrots{} \Nmn[z_n]_t(r,\theta,\phi) &= 0
    \end{align*}

    On définit \(\hat{\LR}\) l'opérateur matriciel tel que
    \begin{align}
      \LR \vE_t(r_{ext},\theta,\phi)
      &= \frac{1}{2\pi}\sum_{n=-\infty}^\infty\sum_{m=-n}^n \hat{\LR} \hat{\vE_t}(r_{ext},n,m)
    \end{align}

    \begin{equation}
      \hat{\mLR}(n,m) =
      \begin{bmatrix}
        \frac{n(n+1)}{r_{ext}^2} & 0
        \\
        0 & 0
      \end{bmatrix}
    \end{equation}

  \subsection{Expression de la matrice d'impédance approchée par la CI3}

    Tout comme dans le cas du plan infini, on peut donc définir \(\hat{\mZ}_{IBC}\) l’opérateur matriciel associé à la condition d'impédance.

    \begin{multline}
        \hat{\mZ}_{CI3}(n,m) = \left(\mI + b_1 \frac{\hat{\mLD}(n,m)}{k_0^2} - b_2 \frac{\hat{\mLR}(n,m)}{k_0^2} \right)^{-1}\\
        \left(a_0 \mI + a_1 \frac{\hat{\mLD}(n,m)}{k_0^2} - a_2 \frac{\hat{\mLR}(n,m)}{k_0^2}\right)
    \end{multline}

\section{Calcul des coefficients de la CI3 par moindres carrés sur l'impédance}

  \subsection{Expression de la fonctionnelle}

    \begin{defn}
      On définit \(\mH_{CI3}\) la fonction de \(\NN \times \RR \times \mathcal{M}_2(\CC) \rightarrow \mathcal{M}_{4\times5}(\CC)\) telle que
      \begin{equation*}
        \mH_{CI3}(n,\mZ) = \begin{bmatrix}
        1 & \hat{\mLD}(n)_{11} & -\hat{\mLR}(n)_{11} & -\left(\hat{\mLD}(n){\mZ}\right)_{11} & \left(\hat{\mLR}(n){\mZ}\right)_{11}
        \\
        0 & \hat{\mLD}(n)_{12} & -\hat{\mLR}(n)_{12} & -\left(\hat{\mLD}(n){\mZ}\right)_{12} & \left(\hat{\mLR}(n){\mZ}\right)_{12}
        \\
        0 & \hat{\mLD}(n)_{21} & -\hat{\mLR}(n)_{21} & -\left(\hat{\mLD}(n){\mZ}\right)_{21} & \left(\hat{\mLR}(n){\mZ}\right)_{21}
        \\
        1 & \hat{\mLD}(n)_{22} & -\hat{\mLR}(n)_{22} & -\left(\hat{\mLD}(n){\mZ}\right)_{22} & \left(\hat{\mLR}(n){\mZ}\right)_{22}
        \end{bmatrix}
      \end{equation*}
      On définit \(b\) la fonction de \(\mathcal{M}_2(\CC) \rightarrow \mathcal{M}_{4\times1}(\CC)\) telle que
      \begin{equation*}
        b(\mZ) = \begin{bmatrix}
        {\mZ}_{11}
        \\
        {\mZ}_{12}
        \\
        {\mZ}_{21}
        \\
        {\mZ}_{22}
        \end{bmatrix}
      \end{equation*}
    \end{defn}

    \begin{prop}
      Soit \(X = (a_0,a_1,a_2,b_1,b_2)\), \(n\) fixé et \(\hat\mZ_{ex}\) l'opérateur d'impédance exact du cylindre, alors
      \begin{equation*}
        \argmin{X\in\CC^5} \norm{\hat\mZ_{CI3}(n,X) - \hat\mZ_{ex}(n)} = \argmin{X\in\CC^5} \norm{\mH_{CI3}(n,\hat\mZ_{ex}(n))X - b(\hat\mZ_{ex}(n))}^2
      \end{equation*}
    \end{prop}

    \begin{proof}
      C'est la même méthodologie que pour le plan.
      On rappelle que dans la section précédente, on a introduit
      \begin{equation*}
        \hat{\mZ}_{CI3}(n) = \left(\mI + b_1 \hat{\mLD}(n) - b_2 \hat{\mLR}(n) \right)^{-1}\left(a_0 \mI + a_1 {\hat{\mLD}(n)} - a_2 {\hat{\mLR}(n)}\right)
      \end{equation*}
      On pose \(\hat\mZ_D(n) = \mI + b_1 \hat{\mLD}(n) - b_2 \hat{\mLR}(n)\) et \(\hat\mZ_N(n) = a_0 \mI + a_1 {\hat{\mLD}(n)} - a_2 {\hat{\mLR}(n)}\) donc

      \begin{align*}
      &{}~ \argmin{X\in\CC^5} \norm{\hat\mZ_{CI3}(n,X) - \hat\mZ_{ex}(n)}
      \\
      & = \argmin{X\in\CC^5} \norm{\hat\mZ_D(n)^{-1}\hat\mZ_N(n) - \hat\mZ_{ex}(n) }
      \\
      &= \argmin{X\in\CC^5} \norm{\hat\mZ_D(n)^{-1}\left(\hat\mZ_N(n) - \hat\mZ_D(n)\hat\mZ_{ex}(n)\right) }
      \\
      &= \argmin{X\in\CC^5} \norm{\hat\mZ_N(n) - \hat\mZ_D(n)\hat\mZ_{ex}(n)}
      \\
      &= \argmin{X\in\CC^5} \norm{\hat\mZ_N(n) - \left(b_1 \hat{\mLD}(n) - b_2 \hat{\mLR}(n)\right)\hat\mZ_{ex}(n) - \hat\mZ_{ex}(n) }
      \\
      &= \argmin{X\in\CC^5} \norm{\mH_{CI3}(n,\hat\mZ_{ex}(n))X - b(\hat\mZ_{ex}(n))}
      \end{align*}
    \end{proof}

    On tronque la série de Mie à \(N_{n}\) termes.
    \begin{defn}
      On définit \(\mA_{CI3}\) la matrice de \(\mathcal{M}_{4N_{n}\times5}(\CC)\) telle que
      \begin{equation*}
        \mA_{CI3} = 
        \begin{bmatrix}
          \mH_{CI3}(n_1,\hat\mZ_{ex}(n_1))
          \\
          \vdots
          \\
          \mH_{CI3}(n_i,\hat\mZ_{ex}(n_i))
          \\
          \vdots
          \\
          \mH_{CI3}(n_{N_n},k_{z},\hat\mZ_{ex}(n_{N_n},k_{z}))
        \end{bmatrix}
      \end{equation*}
      On définit \(g\) la matrice de \(\mathcal{M}_{4N_{n}\times1}(\CC)\) telle que
      \begin{equation*}
        g = 
        \begin{bmatrix}
          b(\hat\mZ_{ex}(n_1))
          \\
          \vdots
          \\
          b(\hat\mZ_{ex}(n_i))
          \\
          \vdots
          \\
          b(\hat\mZ_{ex}(n_{N_n},k_{z}))
        \end{bmatrix}
      \end{equation*}
    \end{defn}

    On peut alors calculer les coefficients de la CI3
    \begin{defn}
      On définit la fonctionnelle \(J_Z\)
      \begin{equation*}
        J_Z(X) = \norm{{\mA}_{CI3}X - {g}}^2
      \end{equation*}
    \end{defn}
    \begin{thm}[Minimisation sans contraintes pour la CI3]

      Les coefficients de la CIOE sont solutions du problème

      Trouver \(X^* \in \CC^5\) tel que
      \begin{equation*}
        X^* = \argmin{X\in \CC^5}  J_Z(X)
      \end{equation*}
    \end{thm}

    \begin{prop}
      \label{prop:sphere:minimisation:minimum_sans_contraintes}
      Si \(\conj{\mA_{CI3}^t}\mA_{CI3}\) est inversible alors
      \begin{equation*}
        X^* = (\conj{\mA_{CI3}^t}\mA_{CI3})^{-1}\conj{\mA_{CI3}^t}g
      \end{equation*}
    \end{prop}

    \begin{proof}
      Même démonstration que pour le théorème \ref{prop:plan:minimisation:minimum_sans_contraintes}.
    \end{proof}

    Nous n'avons pas réussi à démontrer que cette matrice était définie pour tout empilement et tout incidence.

    \begin{thm}[Minimisation avec contraintes pour la CI3]

      Soit \(\CSU[3]{CI3}\) le sous-espace de \(\CC^5\) issu de la définition \ref{def:csu:ci3-3}.
      Alors les coefficients de la CIOE respectant les CSU sont solutions du problème

      Trouver \(X^* \in \CC^5\) tel que
      \begin{equation*}
        X^* = \argmin{X\in \CSU[3]{CI3}}  J_Z(X)
      \end{equation*}
    \end{thm}

\section{Calcul des coefficients de la CI3 par moindres carrés sur les coefficients de la série de Fourier}

  Soit \(\mM_j\) et \(\mM_h\) les fonctions introduites à la définition \ref{def:sphere:matrices_MJ-MH} et \(\hat\mR\) la fonction définie à la définition \ref{def:sphere:reflexion:impedance}.

  \begin{defn}%[]
    \label{def:sphere:minimisation:matrices_MR}
    On définit les fonctions \(\hat\mR_{ex}, \hat\mR_{CI3}\) de \(\NN\times \rightarrow \mathcal{M}_2(\CC)\) telles que
    \begin{align*}
      \hat\mR_{ex}(n) &= \hat\mR(n, \hat\mZ_{ex}(n))
      \\
      \hat\mR_{CI3}(n) &= \hat\mR(n, \hat\mZ_{CI3}(n))
    \end{align*}
    où \(\hat\mZ_{ex},\hat\mZ_{CI3}\) sont des fonctions définies à la proposition \ref{prop:sphere:synthese:impedance} et à l'équation \eqref{eq:sphere:hoibc:ci3}.
  \end{defn}

  \subsection{Expression de la fonctionnelle}

    On utilise les fonctions \(\mN_E, \mN_H\) introduite à la définition \ref{def:sphere:matrices_NE-NH}.

    \begin{defn}
      On définit \(\mA_0,\mA_1,\mA_2,\mA_2,\mB_1,\mB_2\) les fonctions de \(\NN \times \mathcal{M}_2(\CC) \rightarrow \ \mathcal{M}_2(\CC)\) telles que        
      \begin{align*}
        \mA_0(n,\mR) &= \mN_E(r_c^+,n,\mR)
        \\
        \mA_1(n,\mR) &= \hat{\mLD}(n)\mN_E(r_c^+,n,\mR)
        \\
        \mA_2(n,\mR) &= -\hat{\mLR}(n)\mN_E(r_c^+,n,\mR)
        \\
        \mB_1(n,\mR) &= \hat{\mLD}(n)\mN_H(r_c^+,n,\mR)
        \\
        \mB_2(n,\mR) &= -\hat{\mLR}(n)\mN_H(r_c^+,n,\mR)            
      \end{align*}

      On définit \(\tilde{\mH}_{CI3}\) la fonction de \(\NN \times \mathcal{M}_2(\CC) \rightarrow \mathcal{M}_{4\times5}(\CC)\) telle que
      \begin{align*}
        & \tilde\mH_{CI3}(n,\mR) =  \\ &
        \begin{bmatrix}
          \mA_0(n,\mR)_{11} & \mA_1(n,\mR)_{11} & \mA_2(n,\mR)_{11} & \mB_1(n,\mR)_{11} & \mB_2(n,\mR)_{11}
          \\
          \mA_0(n,\mR)_{12} & \mA_1(n,\mR)_{12} & \mA_2(n,\mR)_{12} & \mB_1(n,\mR)_{12} & \mB_2(n,\mR)_{12}
          \\
          \mA_0(n,\mR)_{21} & \mA_1(n,\mR)_{21} & \mA_2(n,\mR)_{21} & \mB_1(n,\mR)_{21} & \mB_2(n,\mR)_{21}
          \\
          \mA_0(n,\mR)_{22} & \mA_1(n,\mR)_{22} & \mA_2(n,\mR)_{22} & \mB_1(n,\mR)_{22} & \mB_2(n,\mR)_{22}
        \end{bmatrix}
      \end{align*}

      On définit \(\tilde{b}\) la fonction de \(\NN \times \mathcal{M}_2(\CC) \rightarrow \mathcal{M}_{4\times1}(\CC)\) telle que
      \begin{equation*}
        \tilde{b}(n,\mR) = -
        \begin{bmatrix}
          \mN_H(r_c^+,n,\mR)_{11}
          \\
          \mN_H(r_c^+,n,\mR)_{12}
          \\
          \mN_H(r_c^+,n,\mR)_{21}
          \\
          \mN_H(r_c^+,n,\mR)_{22}
        \end{bmatrix}
      \end{equation*}
    \end{defn}

    \begin{prop}
      Soit \(X = (a_0,a_1,a_2,b_1,b_2)\), \((n)\) fixé et \(\hat\mR_{ex}\) la matrice définie en \ref{def:sphere:minimisation:matrices_MR}, alors
      \begin{align*}
        \argmin{X\in\CC^5} \norm{\hat\mR_{CI3}(n,X) - \hat\mR_{ex}(n)} =
        \argmin{X\in\CC^5} \norm{\tilde{\mH}_{CI3}(n,\hat\mR_{ex}(n))X - \tilde{b}(n,\hat\mR_{ex}(n))}
      \end{align*}
    \end{prop}

    \begin{proof}
      C'est la même méthodologie que pour l'impédance.
      On rappelle de la section précédente
      \begin{align*}
        \hat{\mZ}_{CI3}(n) = \left(\mI + b_1 \hat{\mLD}(n) - b_2 \hat{\mLR}(n) \right)^{-1}
        \left(a_0 \mI + a_1 {\hat{\mLD}(n)} - a_2 {\hat{\mLR}(n)}\right)
      \end{align*}
      On pose \(\hat\mZ_D(n) = \mI + b_1 \hat{\mLD}(n) - b_2 \hat{\mLR}(n)\) et \(\hat\mZ_N(n) = a_0 \mI + a_1 {\hat{\mLD}(n)} - a_2 {\hat{\mLR}(n)}\) donc

      \begin{align*}
        &{\hspace{1em}}~ \argmin{X\in\CC^5} \norm{\hat\mR_{CI3}(n,X) - \hat\mR_{ex}(n)}
        \\
        & = \argmin{X\in\CC^5} \norm{ - \mM_h(r_s^+,n,\hat\mZ_{CI3})^{-1}\mM_j(r_s^+,n,\hat\mZ_{CI3})- \hat\mR_{ex}(n) }
        \\
        & = \argmin{X\in\CC^5} \norm{ - \mM_h(r_s^+,n,\hat\mZ_{CI3})^{-1}\left(\mM_j(r_s^+,n,\hat\mZ_{CI3}) +  \mM_h(r_s^+,n,\hat\mZ_{CI3})\hat\mR_{ex}(n)\right) }      
        \\ 
        & = \argmin{X\in\CC^5} \norm{\mM_j(r_s^+,n,\hat\mZ_{CI3}) +\mM_h(r_s^+,n,\hat\mZ_{CI3})\hat\mR_{ex}(n)}
        \intertext{D'après la définition \ref{def:sphere:matrices_MJ-MH} des fonctions \(\mM_j, \mM_h\),}
        & = \argmin{X\in\CC^5} \left\lVert \left(\mJ_E(r_s^+,n)-\hat\mZ_{CI3}(n)\mJ_H(r_s^+,n)\right) \right.
        \\
        & \qquad \qquad \quad + \left.\left(\mH_E(r_s^+,n)-\hat\mZ_{CI3}(n)\mH_H(r_s^+,n)\right)\hat\mR_{ex}(n) \right\lVert
        \intertext{D'après la définition de \(\hat\mZ_{CI3}\),}        
        & = \argmin{X\in\CC^5} \left\lVert \hat\mZ_D(n)^{-1}\left(\hat\mZ_D(n)\mJ_E(r_s^+,n)-\hat\mZ_N(n)\mJ_H(r_s^+,n)\right) \right.
        \\
        & \qquad \qquad \quad + \left.\hat\mZ_D(n)^{-1}\left(\hat\mZ_D(n)\mH_E(r_s^+,n)-\hat\mZ_N(n)\mH_H(r_s^+,n)\right)\hat\mR_{ex}(n) \right\lVert
        \\
        & = \argmin{X\in\CC^5} \left\lVert \left(\hat\mZ_D(n)\mJ_E(r_s^+,n)-\hat\mZ_N(n)\mJ_H(r_s^+,n)\right) \right.
        \\
        & \qquad \qquad \quad + \left.\left(\hat\mZ_D(n)\mH_E(r_s^+,n)-\hat\mZ_N(n)\mH_H(r_s^+,n)\right)\hat\mR_{ex}(n) \right\lVert
        \intertext{D'après la définition \ref{def:sphere:matrices_NE-NH} des fonctions \(\mN_E, \mN_H\),}        
        & = \argmin{X\in\CC^5} \norm{\hat\mZ_N(n)\mN_E(r_s^+,n,\hat\mR_{ex}(n)) + \hat\mZ_D(n)\mN_H(r_s^+,n,\hat\mR_{ex}(n))}
      \end{align*}
      et on conclut d'après la définition des fonctions \(\hat\mZ_D, \hat\mZ_N\).
    \end{proof}

    On tronque la série de Mie à \(N_{n}\) termes.
    \begin{defn}
      On définit \(\tilde{\mA}_{CI3}\) la matrice de \(\mathcal{M}_{4N_{n}\times5}(\CC)\) telle que
      \begin{equation*}
        \tilde{\mA}_{CI3} = 
        \begin{bmatrix}
          \tilde\mH_{CI3}(n_1,\hat\mR_{ex}(n_1))
          \\
          \vdots
          \\
          \tilde\mH_{CI3}(n_i,\hat\mR_{ex}(n_i))
          \\
          \vdots
          \\
          \tilde\mH_{CI3}(n_{N_n},\hat\mR_{ex}(n_{N_n}))
        \end{bmatrix}
      \end{equation*}
      On définit \(\tilde{g}\) le vecteur colonne \(\CC^{4N_{n}}\) telle que
      \begin{equation*}
        \tilde{g} = 
        \begin{bmatrix}
          \tilde{b}(n_1,\hat\mR_{ex}(n_1))
          \\
          \vdots
          \\
          \tilde{b}(n_i,\hat\mR_{ex}(n_i))
          \\
          \vdots
          \\
          \tilde{b}(n_{N_n},\hat\mR_{ex}(n_{N_n}))
        \end{bmatrix}
      \end{equation*}
    \end{defn}

    On peut alors calculer les coefficients de la CI3

    \begin{defn}
      On définit la fonctionnelle \(J_R\)
      \begin{equation*}
        J_R(X) = \norm{\tilde{\mA}_{CI3}X - \tilde{g}}
      \end{equation*}
    \end{defn}

    \begin{thm}[Minimisation sans contraintes pour la CI3]

      Les coefficients de la CIOE sont solutions du problème

      Trouver \(X^* \in \CC^5\) tel que
      \begin{equation*}
        X^* = \argmin{X\in \CC^5} J_R(X)
      \end{equation*}
    \end{thm}

    \begin{prop}
      Si \(\conj{\tilde{\mA}_{CI3}^t}\tilde{\mA}_{CI3}\) est inversible alors
      \begin{equation*}
        X^* = (\conj{\tilde{\mA}_{CI3}^t}\tilde{\mA}_{CI3})^{-1}\conj{\tilde{\mA}_{CI3}^t}\tilde{g}
      \end{equation*}
    \end{prop}
    \begin{proof}
      Même méthode que pour la proposition \ref{prop:sphere:minimisation:minimum_sans_contraintes} sur l'impédance.
    \end{proof}

    Nous n'avons pas réussi à démontrer que cette matrice était définie pour tout empilement et tout incidence.

    \begin{thm}[Minimisation avec contraintes pour la CI3]

      Soit \(\CSU[3]{CI3}\) le sous-espace de \(\CC^5\) issu de la définition \ref{def:csu:ci3-3}.
      Alors les coefficients de la CIOE respectant les CSU sont solutions du problème

      Trouver \(X^* \in \CC^5\) tel que
      \begin{equation*}
        X^* = \argmin{X\in \CSU[3]{CI3}} J_R(X)
      \end{equation*}
    \end{thm}

    
\section{Résultats numériques}

  La figure \ref{fig:imp_fourier:sphere:hoppe:62:hoibc:mode_2} trace les valeurs des termes diagonaux de la matrice \(\hat\mZ\) quand les coefficients sont calculés en minimisant \(J_Z\) sans contraintes.
  \begin{figure}[!hbt]
    \centering
    \tikzsetnextfilename{Z_HOPPE_62_sphere_hoibc_mode_2.TM}
\begin{tikzpicture}[scale=1]
    \begin{axis}[
            title={Polarisation TM},
            ylabel={\(\Im(\hat{Z}(n,0)\)},
            xlabel={\(k_t\slash k_0\)},
            width=0.4\textwidth,
            xmin=0,
            xmax=1.5,
            mark repeat=1,
            legend pos=outer north east
        ]
        \addplot [black,mark=square*] table [col sep=comma, x={s2}, y={Im(z_ex.22)}] {csv/HOPPE_62/HOPPE_62.z_ex.MODE_2_TYPE_S_+3.000E-02m.csv};

        \addplot [blue,mark=x] table [col sep=comma, x={s2}, y={Im(z_ibc0.22)}] {csv/HOPPE_62/HOPPE_62.z_ibc.IBC_ibc0_SUC_F_MODE_2_TYPE_S_+3.000E-02m.csv};

        \addplot [red,mark=diamond*] table [col sep=comma, x={s2}, y={Im(z_ibc3.22)}] {csv/HOPPE_62/HOPPE_62.z_ibc.IBC_ibc3_SUC_F_MODE_2_TYPE_S_+3.000E-02m.csv};
    \end{axis}
\end{tikzpicture}
\tikzsetnextfilename{Z_HOPPE_62_sphere_hoibc_mode_2.TE}
\begin{tikzpicture}[scale=1]
    \begin{axis}[
            title={Polarisation TE},
            ylabel={},
            xlabel={\(k_t\slash k_0\)},
            width=0.4\textwidth,
            xmin=0,
            xmax=1.5,
            mark repeat=1,
            legend pos=outer north east
        ]
        \addplot [black,mark=square*] table [col sep=comma, x={s2}, y={Im(z_ex.11)}] {csv/HOPPE_62/HOPPE_62.z_ex.MODE_2_TYPE_S_+3.000E-02m.csv};
        \addlegendentry{Exact};

        \addplot [blue,mark=x] table [col sep=comma, x={s2}, y={Im(z_ibc0.11)}] {csv/HOPPE_62/HOPPE_62.z_ibc.IBC_ibc0_SUC_F_MODE_2_TYPE_S_+3.000E-02m.csv};
        \addlegendentry{CI0};

        \addplot [red,mark=diamond*] table [col sep=comma, x={s2}, y={Im(z_ibc3.11)}] {csv/HOPPE_62/HOPPE_62.z_ibc.IBC_ibc3_SUC_F_MODE_2_TYPE_S_+3.000E-02m.csv};
        \addlegendentry{CI3};
    \end{axis}
\end{tikzpicture}
    \caption[CIOE sur empilement de Hoppe & Rahmat-Samii p.~62]{Minimisation de \(J_Z\): Partie imaginaire des termes diagonaux des matrices d'impédance pour l'empilement \(\eps = 6\), \(\mu = 1\), \(d=0.0225\text{m}\), \(f=1\text{GHz}\), \(r_0=0.03\text{m}\) en fonction de \(k_t = n / (r_0+d)\).}
    \label{fig:imp_fourier:sphere:hoppe:62:hoibc:mode_2}
  \end{figure}
  On remarque alors que la CI3 semble être une bonne approximation. 

  \begin{table}[!hbt]
    \centering
    % On fait deux tables de même hauteur
    \begin{minipage}[t]{0.49\textwidth}
      \vspace{0pt}
      \centering
      \begin{coefftable}{\hyperlink{ci0}{CI0}}
        \input{csv/HOPPE_62/HOPPE_62.IBC_ibc0_SUC_F_MODE_2_TYPE_S_+3.000E-02m.coeff.txt}
      \end{coefftable}
    \end{minipage}
    \begin{minipage}[t]{0.49\textwidth}
      \vspace{0pt}
      \centering
      \begin{coefftable}{\hyperlink{ci3}{CI3}}
        \input{csv/HOPPE_62/HOPPE_62.IBC_ibc3_SUC_F_MODE_2_TYPE_S_+3.000E-02m.coeff.txt}
      \end{coefftable}
    \end{minipage}
    \caption{Coefficients associés à la figure \ref{fig:imp_fourier:sphere:hoppe:62:hoibc:mode_2}}
    \label{tab:imp_fourier:sphere:hoppe:62:hoibc:mode_2}
  \end{table}

  % La figure \ref{fig:imp_fourier:sphere:hoppe:62:hoibc:mode_1} minimise \(J_R\) sans contraintes.
  % \begin{figure}[!hbt]
  %   \centering
  %   \tikzsetnextfilename{Z_HOPPE_62_sphere_hoibc_mode_1.TM}
\begin{tikzpicture}[scale=1]
    \begin{axis}[
            title={Polarisation TM},
            ylabel={\(\Im(\hat{Z}(n,0))\)},
            xlabel={\(k_t\slash k_0\)},
            width=0.4\textwidth,
            xmin=0,
            xmax=1.5,
            mark repeat=1,
            legend pos=outer north east
        ]
        \addplot [black,mark=square*] table [col sep=comma, x={s2}, y={Im(z_ex.22)}] {csv/HOPPE_62/HOPPE_62.z_ex.MODE_1_TYPE_S_+3.000E-02m.csv};

        \addplot [\ccio,mark=x] table [col sep=comma, x={s2}, y={Im(z_ibc0.22)}] {csv/HOPPE_62/HOPPE_62.z_ibc.IBC_ibc0_SUC_F_MODE_1_TYPE_S_+3.000E-02m.csv};

        \addplot [\ccit,mark=diamond*] table [col sep=comma, x={s2}, y={Im(z_ibc3.22)}] {csv/HOPPE_62/HOPPE_62.z_ibc.IBC_ibc3_SUC_F_MODE_1_TYPE_S_+3.000E-02m.csv};
    \end{axis}
\end{tikzpicture}
\tikzsetnextfilename{Z_HOPPE_62_sphere_hoibc_mode_1.TE}
\begin{tikzpicture}[scale=1]
    \begin{axis}[
            title={Polarisation TE},
            ylabel={},
            xlabel={\(k_t\slash k_0\)},
            width=0.4\textwidth,
            xmin=0,
            xmax=1.5,
            mark repeat=1,
            legend pos=outer north east
        ]
        \addplot [black,mark=square*] table [col sep=comma, x={s2}, y={Im(z_ex.11)}] {csv/HOPPE_62/HOPPE_62.z_ex.MODE_1_TYPE_S_+3.000E-02m.csv};
        \addlegendentry{Exact};

        \addplot [\ccio,mark=x] table [col sep=comma, x={s2}, y={Im(z_ibc0.11)}] {csv/HOPPE_62/HOPPE_62.z_ibc.IBC_ibc0_SUC_F_MODE_1_TYPE_S_+3.000E-02m.csv};
        \addlegendentry{CI0};

        \addplot [\ccit,mark=diamond*] table [col sep=comma, x={s2}, y={Im(z_ibc3.11)}] {csv/HOPPE_62/HOPPE_62.z_ibc.IBC_ibc3_SUC_F_MODE_1_TYPE_S_+3.000E-02m.csv};
        \addlegendentry{CI3};
    \end{axis}
\end{tikzpicture}
  %   \caption[CIOE sur empilement de Hoppe & Rahmat-Samii p.~62]{Minimisation de \(J_R\): Partie imaginaire des termes diagonaux des matrices d'impédance pour l'empilement \(\eps = 6\), \(\mu = 1\), \(d=0.0225\text{m}\), \(f=1\text{GHz}\), \(r_0=0.03\text{m}\) en fonction de \(k_t = n / (r_0+d)\).}
  %   \label{fig:imp_fourier:sphere:hoppe:62:hoibc:mode_1}
  % \end{figure}
  % Nous ne savons pas expliquer pourquoi la valeur calculée pour la condition de Leontovich est si différente.

  % \begin{table}[!hbt]
  %   \centering
  %   % On fait deux tables de même hauteur
  %   \begin{minipage}[t]{0.49\textwidth}
  %     \vspace{0pt}
  %     \centering
  %     \begin{coefftable}{\hyperlink{ci0}{CI0}}
  %       \input{csv/HOPPE_62/HOPPE_62.IBC_ibc0_SUC_F_MODE_1_TYPE_S_+3.000E-02m.coeff.txt}
  %     \end{coefftable}
  %   \end{minipage}
  %   \begin{minipage}[t]{0.49\textwidth}
  %     \vspace{0pt}
  %     \centering
  %     \begin{coefftable}{\hyperlink{ci3}{CI3}}
  %       \input{csv/HOPPE_62/HOPPE_62.IBC_ibc3_SUC_F_MODE_1_TYPE_S_+3.000E-02m.coeff.txt}
  %     \end{coefftable}
  %   \end{minipage}
  %   \caption{Coefficients associés à la figure \ref{fig:imp_fourier:sphere:hoppe:62:hoibc:mode_1}}
  %   \label{tab:imp_fourier:sphere:hoppe:62:hoibc:mode_1}
  % \end{table}



\sectionstar{Conclusion}
Nous avons montré comment calculer les coefficients dans le cas d'un objet sphérique en minimisant au sens des moindres carrés la différence entre les coefficients de Mie exacts et approchés. 
