\chapter{Calcul des coefficients pour une sphère}
\label{sec:sphere}
\minitoc
\newpage
\sectionstar{Introduction}
Après le cylindre, nous présentons la sphère, qui possède une courbure dans deux directions. Cette géométrie présente des résultats similaires au plan et au cylindre, car elle possède une courbure, comme pour le cylindre, et fera donc intervenir des fonctions de Bessel, mais possède une invariance par rotation en plus, comme pour le plan.

Nous rappelons d'abord de \cite{nedelec_acoustic_2001} des résultats connus sur les fonctions spéciales usuelles.

\section{Les harmoniques sphériques vectorielles}

    \begin{defn}[Harmonique sphérique d'ordre \(n\) et de degré \(m\)]
        \label{def:sphere:harmoniques_spheriques}
        Soit \(\PP^m_n\) la fonction associée de Legendre (voir \cite[\href{https://dlmf.nist.gov/14}{Chapitre.~14}]{dlmf_nist_2019}) d'ordre \(n\) et de degré \(m\). 

        On appelle harmoniques sphériques d'ordre \(n\) et de degré \(m\) les fonctions de \(\RR\times\RR \rightarrow \CC\) telles que
        \begin{align*}
            Y_{m,n}(\tp) &= (-1)^m\sqrt{\frac{2n+1}{4\pi}\frac{(n-m)!}{(n+m)!}}e^{im\phi}\PP^m_n(\cos \theta)
        \end{align*}
    \end{defn}

    \begin{prop}
        Les harmoniques sphériques \(Y_{m,n}\) sont solutions de 
        \begin{align*}
            \lapls{} u(\tp) = 0 && \forall (\tp) \in [0,\pi]\times[0,2\pi]
        \end{align*}
    \end{prop}
    \begin{prop}
        Soit \(S\) la sphère unité. Les harmoniques sphériques forment une base orthonormale de \(L^2(S)\).
    \end{prop}
    \begin{proof}
        Voir \cite[p.~14]{nedelec_acoustic_2001}
    \end{proof}

    \begin{defn}
        \label{def:sphere:bessel_spheriques}
        On appelle fonction de Bessel sphérique d'ordre \(n\) les fonctions de \(\CC \rightarrow \CC\) solution de
        \begin{align*}
            z^{2}\ddp[2]{z}{u(z)}+2z\ddp{z}{u(z)}+\left(z^{2}-n(n+1)\right)u(z)=0
        \end{align*}
    \end{defn}
    Ces fonctions sont détaillées dans \cite[\href{https://dlmf.nist.gov/10.47}{section.~10.47}]{dlmf_nist_2019}.
    On note \(j_n\) la fonction de Bessel sphérique de première espèce et \(h^{(2)_n}\) la fonction de Hankel de deuxième espèce.

    \begin{prop}
        Soit \(z_n\) une fonction de Bessel sphérique d'ordre \(n\) et \(Y_{m,n}\) une harmonique sphérique d'ordre \(n\) et de degré \(m\). Les fonctions \((\rtp) \mapsto z_n(k(r)r)Y_{m,n}(\tp)\) sont solutions de 
        \begin{align*}
            \lapl u(\rtp) + k^2u(\rtp) = 0 && \forall (\rtp) \in \RR\times[0,\pi]\times[0,2\pi]
        \end{align*}
    \end{prop}
    \begin{proof}
        Voir \cite[section.~2.6.1]{nedelec_acoustic_2001}
    \end{proof}

    \begin{defn}
        On définit l’opérateur \gls{mat-tild} agissant sur les fonctions dérivables tel que soit \(f \in C^1(\CC)\),
        \begin{align*}
        &\tilde{f}(z) = f(z) + zf'(z)&& \forall z \in \CC
        \end{align*}
    \end{defn}

    \begin{defn}
        \label{def:sphere:harmoniques_spheriques_vect}
        Soit \(z_n\) une fonction de Bessel sphérique d'ordre \(n\) et \(Y_{m,n}\) une harmonique sphérique d'ordre \(n\) et de degré \(m\).
        On appelle harmoniques sphériques vectorielles les fonctions \(\gls{phy-Mmn},\gls{phy-Nmn}\) telles que
        \begin{align}
            \label{eq:defMmn}
            \Mmn[z_n](\rtp) &:= \vrot \left( \vect{r} z_n(k(r)r) Y_{m,n}(\tp) \right)
            \\
            &= z_n(k(r)r)
            \begin{bmatrix}
                0
                \\
                \frac{im}{\sin\theta}Y_{mn}(\tp)
                \\
                - \ddr{\theta}{Y_{mn}}(\tp)
            \end{bmatrix}
            \\
            \label{eq:defNnn}
            \Nmn[z_n](\rtp) &:= \frac{\vrot \Mmn[z_n]}{k}(\rtp) 
            \\
            &= \frac{1}{k(r)r}\begin{bmatrix}
                z_n(k(r)r)n(n+1)Y_{mn}(\tp)
                \\
                \tilde{z_n}(k(r)r)\ddr{\theta}{Y_{mn}}(\tp)
                \\
                \tilde{z_n}(k(r)r)\frac{im}{\sin\theta}Y_{mn}(\tp)
            \end{bmatrix}
        \end{align}
    \end{defn}

    \begin{prop}
        Ces fonctions sont solutions de 
        \begin{equation*}
            \left\lbrace
                \begin{aligned}
                    \vrot \vrot \vect{U}(\rtp) - k^2 \vect{U}(\rtp) &= 0
                    \\
                    \vdiv \vect{U}(\rtp) &= 0
                \end{aligned}
            \right.
        \end{equation*}
    \end{prop}
    \begin{proof}
        Voir \cite[Théorème.~5.3.1]{nedelec_acoustic_2001} et sa démonstration ou l'annexe \ref{sec:annex:harmoniques_spheriques}.
    \end{proof}

    On admet les propriétés suivantes
    \begin{prop}
        \label{prop:Mmn_Nmn_rot}
        \begin{align*}
            \vrot \Mmn[z_n](\rtp) &= k\Nmn[z_n](\rtp)
            \\
            \vrot \Nmn[z_n](\rtp) &= k\Mmn[z_n](\rtp)
        \end{align*}
    \end{prop}

    % Ces vecteurs harmoniques sphériques possèdent les propriétés suivantes

    % \begin{align}
    % \int_{S(0,R)} \vect{M_{m,n}^{z_n}} \cdot \conj{\vect{N_{p,q}^{z_n}}} ds &= 0
    % \\
    % \int_{S(0,R)} \vect{M_{m,n}^{z_n}} \cdot \conj{\vect{M_{p,q}^{z_n}}} ds &= \gamma_{m,n}R^2 \delta_{mp}\delta_{nq}
    % \\
    % \int_{S(0,R)} \vect{N_{m,n}^{z_n}} \cdot \conj{\vect{N_{p,q}^{z_n}}} ds &= \frac{\gamma_{m,n}}{k^2} \delta_{mp}\delta_{nq}
    % \end{align}

    \begin{defn}
        On définit les fonctions de \(\RR\times\RR\mapsto \CC^2\)
        \begin{align}
            \label{eq:defUmn_tgt}
            \Umn(\tp) &=
            \begin{bmatrix}
                \frac{im}{\sin\theta}Y_{mn}(\tp)
                \\
                - \ddr{\theta}{Y_{mn}}(\tp)
            \end{bmatrix}
            \\
            \label{eq:defNmn_tgt}
            \Umn^\perp(\tp) &=
            \begin{bmatrix}
                \ddr{\theta}{Y_{mn}}(\tp)
                \\
                \frac{im}{\sin\theta}Y_{mn}(\tp)
            \end{bmatrix}
        \end{align}
    \end{defn}

    \begin{prop}
        On remarque alors que les parties tangentielles des vecteurs harmoniques sphériques peuvent s'écrire:
        \begin{align*}
          \Mmn[z_n]_t(\rtp) &= z_n(k(r)r)\Umn(\tp)
          \\
          \Nmn[z_n]_t(\rtp) &= \frac{1}{k(r)r}\tilde{z_n}(k(r)r)\Umn^\perp(\tp)
        \end{align*}
    \end{prop}

    On a alors les propriétés supplémentaires
    \begin{prop}
        \label{prop:Mmn_Nmn_vect}
        \begin{align*}
          \vect{e_r} \pvect \Mmn[z_n]_t(\rtp) &= k(r)rz_n(k(r)r)\Umn^\perp(\tp)
          \\
          \vect{e_r} \pvect \Nmn[z_n]_t(\rtp) &= -\frac{1}{k(r)r}\tilde{z_n}(k(r)r)\Umn(\tp)
        \end{align*}
    \end{prop}

\section{Expressions exactes des matrices d'impédance et des matrice de réflexions pour une sphère}

    \begin{figure}[!hbt]
        \centering
        \tikzsetnextfilename{sphere_1_couche}
        \begin{tikzpicture}
              \tikzmath{
    \a = 80;
    \b = 100;
    \d = 0.5;
    \ri = 20;
    \re = \ri;
}

% Le conducteur
\tikzmath{
    \ri = \re;
    \re = \ri + 0.5*\d;
    \xa = cos(\a)*\re;
    \ya = sin(\a)*\re;
    \xb = cos(\b)*\ri;
    \yb = sin(\b)*\ri;
}

\coordinate (a) at (\xa,\ya);
\coordinate (b) at (\xb,\yb);

\fill [pattern=north east lines] (a) arc (\a:\b:\re) -- (b) arc (\b:\a:\ri) -- cycle;
\draw (a) arc (\a:\b:\re);
\draw (a) node [right] {$r_0$};


% Le repère
\coordinate (n) at ($(a)+(0.5,-1)$);
%
%
%\draw [->] (n) -- ++(0,1) node [at end, right] {$\v{\pr}$};
%\draw [->] (n) -- ++(1,0) node [at end, right] {$\v{\pt}$};
%
\draw (n) ++(0.2,0.2) circle(0.1cm) node [above=0.1cm] {\(\vect{e_\phi}\)};
\draw (n) ++(0.2,0.2) +(135:0.1cm) -- +(315:0.1cm);
\draw (n) ++(0.2,0.2) +(45:0.1cm) -- +(225:0.1cm);


% 1ere couche
\tikzmath{
    \ri = \re;
    \re = \ri + \d;
    \xa = cos(\a)*\re;
    \ya = sin(\a)*\re;
    \xb = cos(\b)*\ri;
    \yb = sin(\b)*\ri;
    \xc = cos(0.5*(\b+\a))*(\ri+0.5*\d);
    \yc = sin(0.5*(\b+\a))*(\ri+0.5*\d);
}

\coordinate (a) at (\xa,\ya);
\coordinate (b) at (\xb,\yb);
\coordinate (c) at (\xc,\yc);

\fill [lightgray] (a) arc (\a:\b:\re) -- (b) arc (\b:\a:\ri) -- cycle;
\draw (a) arc (\a:\b:\re);
\draw (c) node {$\nu,\eta,d$};

% Le vide
\tikzmath{
    \xc = cos(0.5*(\b+\a))*(\re);
    \yc = sin(0.5*(\b+\a))*(\re);
}

\draw (\xc,\yc) node [above] {vide};
        \end{tikzpicture}
    \end{figure}

    On exprime les équations de Maxwell dans le matériau en coordonnées sphériques.

    Les solutions des équations de Maxwell sont développables en série de Mie
    \begin{prop}
        Il existe des constantes \((a_{mn},c_{mn},d_{mn},d_{mn})\) constantes par morceaux en \(r\) telles que les champs sont
        \begin{multline*}
            \vE(\rtp) = \sum_{n\in\ZZ}\sum_{m\in\ZZ} a_{mn} \Mmn[j_n](\rtp) + b_{mn} \Nmn[j_n](\rtp)
            \\
            + c_{mn} \Mmn[h^{(2)}_n](\rtp) + d_{mn} \Nmn[h^{(2)}_n](\rtp)
        \end{multline*}
        \begin{multline*}
            \vH(\rtp) = \frac{i}{\eta(r)}\sum_{n\in\ZZ}\sum_{m\in\ZZ} a_{mn} \Nmn[j_n](\rtp) + b_{mn} \Mmn[j_n](\rtp)
            \\
            + c_{mn} \Nmn[h^{(2)}_n](\rtp) + d_{mn} \Mmn[h^{(2)}_n](\rtp)
        \end{multline*}
    \end{prop}
    \begin{proof}
        Admise. On renvoie à \cite{cheng_spectral_1993}.
    \end{proof}

  \subsection{Expressions des champs tangentiels dans chaque couche}


    \begin{defn}
      \label{def:sphere:je-jH-hE-hH}
      On définit les matrices \(\mj_{E}(r,n),\mh_{E}(r,n),\mj_{H}(r,n),\mh_{H}(r,n)\) où \(k,\eta\) sont constants par morceaux en \(r\)
      \begin{align*}
        \mj_{E}(r,n) &=
        \begin{bmatrix}
            j_n(k(r)r) & 0
            \\
            0 & \tilde{j_n}(k(r)r)
        \end{bmatrix}
        \\
        \mh_{E}(r,n) &=
        \begin{bmatrix}
            h^{(2)}_n(k(r)r) & 0
            \\
            0 & \tilde{h^{(2)}_n}(k(r)r)
        \end{bmatrix}
        \\
        \mj_{H}(r,n) &=\frac{i}{k\eta(r)r}
        \begin{bmatrix}
                -\tilde{j_n}(k(r)r) & 0
                \\
                0 & (k(r)r)^2 j_n(k(r)r)
        \end{bmatrix}
        \\
        \mh_{H}(r,n) &=\frac{i}{k\eta(r)r}
        \begin{bmatrix}
                -\tilde{h^{(2)}_n}(k(r)r) & 0
                \\
                0 & (k(r)r)^2 h^{(2)}_n(k(r)r)
        \end{bmatrix}
      \end{align*}
    \end{defn}

    En utilisant les fonctions définies précédemment, on a les expressions des coordonnées tangentielles de \(\vE,\vect{e_r}\times\vH\).
    \begin{prop}
        \label{prop:imp_fourier:sphere:Et}
     \begin{align*}
            \vE_t(\rtp) &= \sum_{n\in\ZZ}\sum_{m\in\ZZ}
            \begin{bmatrix}
              \Umn(\tp) & \Umn^\perp(\tp)
            \end{bmatrix}
            \left(
              \mj_E(r,n)
              \begin{bmatrix}
                  a_{mn}
                  \\
                  b_{mn}
              \end{bmatrix}
              +
              \mh_E(r,n)
              \begin{bmatrix}
                  c_{mn}
                  \\
                  d_{mn}
              \end{bmatrix}
            \right)
            \\
            \vJ(\rtp) &= \sum_{n\in\ZZ}\sum_{m\in\ZZ}
            \begin{bmatrix}
                \Umn(\tp) & \Umn^\perp(\tp)
            \end{bmatrix}
            \left(
                \mj_H(r,n)
                \begin{bmatrix}
                    a_{mn}
                    \\
                    b_{mn}
                \end{bmatrix}
                +
                \mh_H(r,n)
                \begin{bmatrix}
                    c_{mn}
                    \\
                    d_{mn}
                \end{bmatrix}
            \right)
        \end{align*}
    \end{prop}

    \begin{proof}
        On exprime les champs tangentiels
        \begin{multline*}
            \vE_t(\rtp) = \sum_{n\in\ZZ}\sum_{m\in\ZZ} a_{mn} j_n(k(r)r)\Umn(\tp) + b_{mn} \frac{\tilde{j_n}(k(r)r)}{k(r)r}\Umn^\perp(\tp)
            \\
            + c_{mn} h^{(2)}_n(k(r)r)\Umn(\tp) + d_{mn} \frac{\tilde{j_n}(k(r)r)}{k(r)r}\Umn^\perp(\tp)
        \end{multline*}
        \begin{multline*}
            \vJ(\rtp) = \frac{i}{\eta(r)}\sum_{n\in\ZZ}\sum_{m\in\ZZ} - a_{mn} \frac{\tilde{j_n}(k(r)r)}{k(r)r}\Umn(\tp) + b_{mn} k(r)r j_n(k(r)r) \Umn^\perp(\tp)
            \\
            -  \frac{\tilde{h^{(2)}_n}(k(r)r)}{k(r)r} c_{mn} \Umn(\tp) + k(r)r h^{(2)}_n(k(r)r) d_{mn} \Umn^\perp(\tp)
        \end{multline*}
        Puis en réorganisant les termes, on a la propriété
        \begin{multline*}
            \vE_t(\rtp) = \sum_{n\in\ZZ}\sum_{m\in\ZZ}
            \begin{bmatrix}
              \Umn(\tp) & \Umn^\perp(\tp)
            \end{bmatrix}
            \\
            \left(
              \begin{bmatrix}
                j_n(k(r)r) & 0
                \\              
                0 & \tilde{j_n}(k(r)r)
              \end{bmatrix}
              \begin{bmatrix}
                  a_{mn}
                  \\
                  b_{mn}
              \end{bmatrix}
              +
              \begin{bmatrix}
                h^{(2)}_n(k(r)r) & 0
                \\
                0 & \tilde{h^{(2)}_n}(k(r)r)
              \end{bmatrix}
              \begin{bmatrix}
                  c_{mn}
                  \\
                  d_{mn}
              \end{bmatrix}
            \right)
        \end{multline*}
        \begin{multline*}
            \vJ(\rtp) = \frac{i}{k(r)\eta(r)r}\sum_{n\in\ZZ}\sum_{m\in\ZZ}
            \begin{bmatrix}
                \Umn(\tp) & \Umn^\perp(\tp)
            \end{bmatrix}
            \\
            \left(
                \begin{bmatrix}
                    -\tilde{j_n}(k(r)r) & 0
                    \\
                    0 & (k(r)r)^2 j_n(k(r)r)
                \end{bmatrix}
                \begin{bmatrix}
                    a_{mn}
                    \\
                    b_{mn}
                \end{bmatrix}
                +
                \begin{bmatrix}
                    -\tilde{h^{(2)}_n}(k(r)r) & 0
                    \\                
                    0 & (k(r)r)^2 h^{(2)}_n(k(r)r)
                \end{bmatrix}
                \begin{bmatrix}
                    c_{mn}
                    \\
                    d_{mn}
                \end{bmatrix}
            \right)
        \end{multline*}
    \end{proof}

    On remarque que \(\eps,\mu\) sont constantes par morceaux.
    Sur une interface d'équation \(r=r_p\) entre deux matériaux, il y a donc un saut de valeurs pour ces matrices: \(\lim_{\delta\rightarrow 0 } \mj_E(r_p+ \delta,n) - \mj_E(r_p - \delta,n) \not = 0\) .

    \begin{prop}
      \label{lem:cylindre:imp:inv_matrices_JE-HE}
      Si \(\mu(r)\eps(r) \in \CC\backslash\RR\), alors les matrices \(\mj_E(r,n)\), \(\mj_H(r,n)\)  sont inversibles pour tout couple \((n)\).
      % Sinon, pour \(n\) donné, il existe un nombre fini de \(k_z\) tels que ces matrices ne soient pas inversibles. 
    \end{prop}

    \begin{proof}
      Par définition des matrices,
      \begin{align*}
        \det(\mj_E(r,n)) &= j_n(k(r)r)\tilde{j_n}(k(r)r)
        \\
        \det(\mj_H(r,n)) &= j_n(k(r)r)\tilde{j_n}(k(r)r)(k(r)r)^2
      \end{align*}

      D’après \cite[p.~370]{abramowitz_handbook_1964}, les zéros des fonctions de Bessel d'ordre réel \(\nu >-1\) sont tous réels, or les fonctions de Bessel sphériques d'ordre \(n\) sont des fonctions de Bessel d'ordre \(n+1/2\) donc si \(k(r) \in\CC\backslash\RR\) alors ces matrices sont inversibles.
    \end{proof}


    Pour les matrices \(\mh_E(r,n)\), \(\mh_H(r,n)\), on a
    \begin{align*}
        \det(\mh_E(r,n)) &= h^{(2)}_n(k(r)r)\tilde{h^{(2)}_n}(k(r)r)
        \\
        \det(\mh_H(r,n)) &= h^{(2)}_n(k(r)r)\tilde{h^{(2)}_n}(k(r)r)(k(r)r)^2
     \end{align*}
    Et on ne peut rien conclure, car les zéros peuvent être complexes (voir \cite{sandstrom_note_2007}). On va donc supposer ces déterminants non nuls.


    %%%%%%%%%%%%%%%%%%%%%%%%%%%%%%%%%%%%%%%%%%%%%%%%%%%%%%%%%%%%%%%%%%%%%%%%%%%%%%%%%%%%%%%%%%%%%%%%%%%%%%%%
    %%%%%%%%%%%%%%%%%%%%%%%%%%%%%%%%%%%%%%%%%%%%%%%%%%%%%%%%%%%%%%%%%%%%%%%%%%%%%%%%%%%%%%%%%%%%%%%%%%%%%%%%
    %%%%%%%%%%%%%%%%%%%%%%%%%%%%%%%%%%%%%%%%%%%%%%%%%%%%%%%%%%%%%%%%%%%%%%%%%%%%%%%%%%%%%%%%%%%%%%%%%%%%%%%%


  \subsection{Expression de la matrice d'impédance pour une couche}

    Soit \(r_1 = r_0 + d\). On se place dans le matériau donc \(r_0 \le r < r_1\).

    \begin{prop}
      Si on suppose que la matrice \(\mh_{H}(r_1^-,n) - \mj_{H}(r_1^-,n)\mj_{E}(r_0^+,n)^{-1}\mh_{E}(r_0^+,n)\) est inversible pour tout \(n\), alors on a 
      \begin{equation*}
        \hat \vE_t(r_1^-,n) = \hat \mZ(n) \left(\vect{e_r}\pvect \hat \vH_t(r_1^-,n)\right)
      \end{equation*}
      où
      \begin{multline*}
        \hat \mZ(n) =
        \left(\mh_{E}(r_1^-,n)\mh_{E}(r_0^+,n)^{-1} - \mj_{E}(r_1^-,n)\mj_{E}(r_0^+,n)^{-1}\right)\\
        \left(\mh_{H}(r_1^-,n)\mh_{E}(r_0^+,n)^{-1} - \mj_{H}(r_1^-,n)\mj_{E}(r_0^+,n)^{-1}\right)^{-1}
      \end{multline*}
    \end{prop}

    \begin{proof}
      On se place dans la matériau: \(r_0 \le r \le r_ 1 \).

      On applique la condition limite du conducteur parfait \(\hat \vE(r_0,n) = 0\) dans \eqref{prop:imp_fourier:sphere:Et}. Comme les harmoniques sphériques forment un base de \(L^2(S(r_0))\), chaque terme doit être nul.
      \begin{equation*}
        \mj_{E}(r_0,n)
        \begin{bmatrix}
          a_{nm} \\
          b_{nm}
        \end{bmatrix}
        =-\mh_{E}(r_0,n)
        \begin{bmatrix}
          c_{nm} \\
          d_{nm}
        \end{bmatrix}
      \end{equation*}

      On peut donc exprimer les composantes tangentielles%, on omet les dépendances en \((n)\).
      \begin{align*}
        \hat \vE_t(r_1,n) &=
        \left(\mh_{E}(r_1,n) - \mj_{E}(r_1,n)\mj_{E}(r_0,n)^{-1}\mh_{E}(r_0,n)\right)
        \begin{bmatrix}
          c_{nm} \\
          d_{nm}
        \end{bmatrix}
        \\
        \vect{e_r}\pvect \hat \vH_t(r_1,n) &=
        \left(\mh_{H}(r_1,n) - \mj_{H}(r_1,n)\mj_{E}(r_0,n)^{-1}\mh_{E}(r_0,n) \right)
        \begin{bmatrix}
          c_{nm} \\
          d_{nm}
        \end{bmatrix}
      \end{align*}

      On calcule le déterminant de cette matrice 
      \begin{multline*}
        \det(\mh_{H}(r_1,n) - \mj_{H}(r_1,n)\mj_{E}(r_0,n)^{-1}\mh_{E}(r_0,n)) = \\
        i\w\eps\frac{(H_n^{(2)}(r_1k_3)J_n'(r_0k_3)-J_n(r_1k_3)H_n^{(2)}{}'(r_0k_3))(H_n^{(2)}{}'(r_1k_3)J_n(r_0k_3)-J_n'(r_1k_3)H_n^{(2)}(r_0k_3))}{J_n'(r_0^+k_3)J_n(r_0^+k_3)}
      \end{multline*}
      Ce dernier s'annule quand \(\frac{H_n^{(2)}(r_1k_3)}{J_n(r_1k_3)} = \frac{H_n^{(2)}{}'(r_0k_3)}{J_n'(r_0k_3)}\) ou quand \(\frac{H_n^{(2)}{}'(r_1k_3)}{J_n'(r_1k_3)} = \frac{H_n^{(2)}(r_0k_3)}{J_n(r_0k_3)}\).

      Il n'existe pas à notre connaissance de propriétés connues des fonctions de Bessel pour déduire de ces égalités des conditions sur \(\omega^2\eps\mu,k_z,r_1,r_0\).

      On suppose donc simplement \(\mh_{H}(r_1,n) - \mj_{H}(r_1,n)\mj_{E}(r_0,n)^{-1}\mh_{E}(r_0,n)\) inversible, la matrice \(\hat\mZ(n)\) telle que \(\hat\vE_t(r_1,n) = \hat\mZ(n) (\vect{e_r}\pvect\vH(r_1,n))\) est alors

      \begin{multline*}
        \hat \mZ =
        \left(\mh_{E}(r_1,n) - \mj_{E}(r_1,n)\mj_{E}(r_0,n)^{-1}\mh_{E}(r_0,n)\right)
        \\
        \left(\mh_{H}(r_1,n) - \mj_{H}(r_1,n)\mj_{E}(r_0,n)^{-1}\mh_{E}(r_0,n)\right)^{-1}.
      \end{multline*}
      Comme on a supposé la matrice \(\mh_E(r_0,n)\) inversible, on peut factoriser par cette matrice à droite le numérateur et le dénominateur d'où la proposition.

    \end{proof}


  \subsection{Expression de la matrice d'impédance pour plusieurs couches}

    \begin{figure}[!hbt]
      \centering
      \tikzsetnextfilename{cylindre_n_couches}
      \begin{tikzpicture}
        \input{tikz/schema/cylindre_n_couches.tikz}
      \end{tikzpicture}
    \end{figure}

    Soit \(r_p\) le rayon de l'interface \(p\), \(r_p = r_0 +\sum_{i=1}^{p} d_{i}\). On dit que l'on se trouve dans la couche \(p\) si \(r_{p-1} \le r \le r_p \).

    \begin{defn}
      \label{def:cylindre:matrices_MJ-MH}
      On définit les fonctions de \([r_{p-1}, r_p[\times \NN \times \RR \times \mathcal{M}_{2}(\CC) \rightarrow \mathcal{M}_{2}(\CC)\)
      \begin{align*}
        \mM_{\mj}(r,n,\mA) &= \mj_{E}(r,n) -  \mA \mj_{H}(r,n)
        \\
        \mM_{\mh}(r,n,\mA) &= \mh_{E}(r,n) -  \mA \mh_{H}(r,n)
      \end{align*}
    \end{defn}

    \begin{defn}
      \label{def:cylindre:reflexion:impedance}
      On définit la fonction de \([r_{p-1}, r_p[\times \NN \times \RR \times \mathcal{M}_{2}(\CC) \rightarrow \mathcal{M}_{2}(\CC)\)
      \begin{align*}
        \mR(r,n,\mA) &= -\mM_{\mh}(r,n,\mA)^{-1}\mM_{\mj}(r,n,\mA)
      \end{align*}
    \end{defn}
    A priori, pour \(r,n\) donné, \(\mR(r,n,\mA)\) n'est pas défini pour toute matrice \(\mA\).

    On prolonge ces définitions aux autres couches.

    \begin{defn}%[Fonction de transfert]{}~
      \label{def:cylindre:transfert:impedance}

      On définit \(\mT_p\) la fonction de \([r_{p-1}, r_p[^2\times\NN\times\RR\times\mathcal{M}_2(\CC)\rightarrow \mathcal{M}_{2}(\CC)\)
      \begin{multline*}
        \mT_p(r,r',n,\mA) = \\
          \left(\mj_{E}(r,n)\mM_{\mj}(r',n,\mA)^{-1} - \mh_{E}(r,n)\mM_{\mh}(r',n,\mA)^{-1}\right) 
          \\
          \left(\mj_{H}(r,n)\mM_{\mj}(r',n,\mA)^{-1} - \mh_{H}(r,n)\mM_{\mh}(r',n,\mA)^{-1}\right)^{-1}
      \end{multline*}
    \end{defn}
    A priori, pour \(r,r',n\) donné, \(\mT_p(r,r',n,\mA)\) n'est pas défini pour toute matrice \(\mA\).

    \begin{prop}%[Théorème de transfert]~
      \label{prop:cylindre:transfert:impedance}

      Soient \(\hat\vE_t,\hat\vH_t\) tels que \(\vE_t(r_{p}^-,n) = \hat\mZ_{p}(n)(\vn \pvect \hat\vH(r_{p}^-,n)\).

      Si les matrices suivantes sont inversibles
      \begin{align*}
        \mM_{\mj}(r_p^-,n,\hat\mZ_{p}(n)) && \mM_{\mh}(r_p^-,n,\hat\mZ_{p}(n))
      \end{align*}
      \begin{align*}
        \mj_{H}(r_{p-1}^+,n)\mM_{\mj}(r_{p}^-,n,\hat\mZ_{p}(n))^{-1} - \mh_{H}(r_{p-1}^+,n)\mM_{\mh}(r_{p}^-,n,\hat\mZ_{p}(n))^{-1}
      \end{align*}

      alors \(\hat\vE_t(r_{p-1}^+,n) = \mT_p(r_{p-1}^+,r_{p}^-,n,\hat\mZ_{p}(n))(\vn \pvect \hat\vH(r_{p-1}^+,n))\).

      Une condition d'impédance sur le bord supérieur d'une couche détermine la condition limite sur le bord inférieur.
    \end{prop}


    \begin{proof}
      On se situe dans la couche \(p\) (\(r_{p-1}\le r\le r_p\))
      \begin{multline*}
        \mj_{E}(r_{p},n)
        \begin{bmatrix}
          a_{nm} \\
          b_{nm}
        \end{bmatrix}
        +
        \mh_{E}(r_{p},n)
        \begin{bmatrix}
          c_{nm} \\
          d_{nm}
        \end{bmatrix}
        =
        \\
        \hat \mZ_{p}(n)
        \left(
          \mj_{H}(r_{p},n)
          \begin{bmatrix}
            a_{nm} \\
            b_{nm}
          \end{bmatrix}
          +
          \mh_{H}(r_{p},n)
          \begin{bmatrix}
            c_{nm} \\
            d_{nm}
          \end{bmatrix}
        \right)
      \end{multline*}

      Ce qui revient à 
      \begin{equation*}
        \mM_{\mj}(r_{p},n,\hat\mZ_p(n))
        \begin{bmatrix}
          a_{nm} \\
          b_{nm}
        \end{bmatrix}
        =
        -\mM_{\mh}(r_{p},n,\hat\mZ_p(n))
        \begin{bmatrix}
          c_{nm} \\
          d_{nm}
        \end{bmatrix}
      \end{equation*}

      On suppose que les matrices \(\mM_{\mj}(r_p,n,\hat\mZ_p(n)), \mM_{\mh}(r_p,n,\hat\mZ_p(n))\) sont inversibles pour donc
      \begin{equation*}
        \begin{bmatrix}
          c_{nm} \\
          d_{nm}
        \end{bmatrix}
        =
        \mR(r_{p},n,\hat\mZ_p(n))
        \begin{bmatrix}
          a_{nm} \\
          b_{nm}
        \end{bmatrix}
      \end{equation*}

      On injecte ce qui précède en \(r = r_{p-1}\)
      \begin{multline*}
        \hat{\vE}_t(r_{p-1},n) = 
        \\
        \left(\mh_{E}(r_{p-1},n)\mR(r_{p},n,\hat{\mZ}_p(n)) + \mj_{E}(r_{p-1},n)\right)
        \begin{bmatrix}
          a_{nm} \\
          b_{nm}
        \end{bmatrix}
      \end{multline*}        
      \begin{multline*}
        \vect{e_r}\times\hat{\vH}_t(r_{p-1},n) =
        \\
        \left(\mh_{H}(r_{p-1},n)\mR(r_{p},n,\hat{\mZ}_p(n)) + \mj_{H}(r_{p-1},n))\right)
        \begin{bmatrix}
          a_{nm} \\
          b_{nm}
        \end{bmatrix}
      \end{multline*}

      On suppose alors que cette dernière est inversible pour tout \((n)\).

      On obtient
      \begin{multline*}
        \hat{\vE}_t(r_{p-1},n) =
        \\
        \left(\mj_{E}(r_{p-1},n) + \mh_{E}(r_{p-1},n)\mR(r_{p},n,\hat{\mZ}_p(n))\right) \\
        \left(\mj_{H}(r_{p-1},n) + \mh_{H}(r_{p-1},n)\mR(r_{p},n,\hat{\mZ}_p(n))\right)^{-1}
        \\
        \vect{e_r}\times\hat{\vH}_t(r_{p-1},n)
      \end{multline*}

      Comme on a supposé l'inversibilité des deux matrices \(\mM_J\), \(\mM_H\) alors on peut factoriser à droite le numérateur et le dénominateur et on a la propriété.
    \end{proof}

    \begin{prop}%[Théorème de relèvement]~
      \label{prop:cylindre:relevement:impedance}

      Soient \(\hat\vE_t,\hat\vH_t\) tels que \(\vE_t(r_{p-1}^+,n) = \hat\mZ_{p-1}(n)(\vn \pvect \hat\vH(r_{p-1}^+,n)\).

      Si les matrices suivantes sont inversibles
      \begin{align*}
        \mM_{\mj}(r_{p-1}^+,n,\hat\mZ_{p-1}(n)), && \mM_{\mh}(r_{p-1}^+,n,\hat\mZ_{p-1}(n)),
      \end{align*}
      \begin{align*}
        \mh_{H}(r_{p}^-,n)\mM_{\mh}(r_{p-1}^+,n,\hat\mZ_{p-1}(n))^{-1} - \mj_{H}(r_{p}^-,n)\mM_{\mj}(r_{p-1}^+,n,\hat\mZ_{p-1}(n))^{-1},
      \end{align*}

      alors \(\hat\vE_t(r_{p}^-,n) = \mT_p(r_p^-,r_{p-1}^+,n,\hat\mZ_{p-1}(n))(\vn \pvect \hat\vH(r_{p}^-,n))\).

      Une condition d'impédance sur le bord inférieur d'une couche détermine la condition limite sur le bord supérieur.
    \end{prop}

    \begin{proof}
      Même méthodologie que pour la proposition \ref{prop:cylindre:transfert:impedance}.
    \end{proof}

    \begin{prop}%[Corollaire aux théorèmes de transfert et de relèvement.]{}~
      \label{prop:cylindre:synthese:impedance}{}~

      Soient \(\hat\vE_t,\hat\vH_t\) tels que 
      \begin{align*}
      \vE_t(r_{p-1}^+,n) &= \hat\mZ_{p-1}(n)(\vn \pvect \hat\vH(r_{p-1}^+,n))
      \\
      \vE_t(r_{p}^-,n) &= \hat\mZ_{p}(n)(\vn \pvect \hat\vH(r_{p}^-,n))
      \end{align*}

      Si les matrices suivantes sont inversibles
      \begin{align*}
        \mM_{\mj}(r_p^-,n,\hat\mZ_{p}(n)), && \mM_{\mj}(r_{p-1}^+,n,\hat\mZ_{p-1}(n)),
        \\
        \mM_{\mh}(r_p^-,n,\hat\mZ_{p}(n)), && \mM_{\mh}(r_{p-1}^+,n,\hat\mZ_{p-1}(n)),
      \end{align*}
      \begin{align*}
        \mj_{H}(r_{p}^-,n)\mM_{\mj}(r_{p-1}^+,n,\hat\mZ_{p-1}(n))^{-1} - \mh_{H}(r_{p}^-,n)\mM_{\mh}(r_{p-1}^+,n,\hat\mZ_{p-1}(n))^{-1},
        \\
        \mj_{H}(r_{p-1}^+,n)\mM_{\mj}(r_{p}^-,n,\hat\mZ_{p}(n))^{-1} - \mh_{H}(r_{p-1}^+,n)\mM_{\mh}(r_{p}^-,n,\hat\mZ_{p}(n))^{-1},
      \end{align*}

      Alors 
      \begin{align*}
        \hat\mZ_{p-1}(n) &= \mT_p(r_{p-1}^+,r_{p}^-,n,\hat\mZ_{p}(n))
        \\
        \hat\mZ_{p}(n) &= \mT_p(r_{p}^-,r_{p-1}^+,n,\hat\mZ_{p-1}(n))
      \end{align*}

    \end{prop}

    On peut donc déterminer itérativement les matrices d'impédance. Dans notre cadre d'étude, la présence d'un conducteur parfait sur l'interface \(r=r_0^+\) implique \(\hat\mZ_{0}(n) = 0\).

  %%%%%%%%%%%%%%%%%%%%%%%%%%%%%%%%%%%%%%%%%%%%%%%%%%%%%%%%%%%%%%%%%%%%%%%%%%%%%%%%%%%%%%%%%%%%%%%%%%%%%%%%
  %%%%%%%%%%%%%%%%%%%%%%%%%%%%%%%%%%%%%%%%%%%%%%%%%%%%%%%%%%%%%%%%%%%%%%%%%%%%%%%%%%%%%%%%%%%%%%%%%%%%%%%%
  %%%%%%%%%%%%%%%%%%%%%%%%%%%%%%%%%%%%%%%%%%%%%%%%%%%%%%%%%%%%%%%%%%%%%%%%%%%%%%%%%%%%%%%%%%%%%%%%%%%%%%%%

  \subsection{Expression des coefficients de la série de Fourier}

    On se place à l'interface \(p\) donc \(r_{p-1} \le r \le r_{p+1} \).

    \begin{defn}
      \label{def:cylindre:matrices_NE-NH}
      On définit les fonctions de \(\RR\times \NN \times \RR \times \mathcal{M}_{2}(\CC) \rightarrow \mathcal{M}_{2}(\CC)\)
      \begin{align*}
        \mN_{E}(r,n,\mA) &= \mj_{E}(r,n) + \mh_{E}(r,n)\mA
        \\
        \mN_{H}(r,n,\mA) &= \mj_{H}(r,n) + \mh_{H}(r,n)\mA
      \end{align*}
    \end{defn}

    \begin{defn}%[Fonction de transfert]{}~
      \label{def:cylindre:transfert:reflexion}{}~

      On définit \(\mathfrak{T}_p\) la fonction de \([r_{p-1}, r_p]\times[r_p, r_{p+1}]\times\NN\times\RR\times\mathcal{M}_2(\CC)\rightarrow \mathcal{M}_{2}(\CC)\)
      \begin{multline*}
        \mathfrak{T}_p(r,r',n,\mA) = \\
          -\left(\mN_{E}(r',n,\mA)^{-1}\mh_{E}(r,n) - \mN_{H}(r',n,\mA)^{-1}\mh_{H}(r,n)\right)^{-1}
          \\
          \left(\mN_{E}(r',n,\mA)^{-1}\mj_{E}(r,n) - \mN_{H}(r',n,\mA)^{-1}\mj_{H}(r,n)\right)
      \end{multline*}
    \end{defn}
    A priori, pour \(r,r',n\) donné, \(\mathfrak{T}_p(r,r',n,\mA)\) n'est pas défini pour toute matrice \(\mA\).

    \begin{prop}%[Théorème de transfert]~
      \label{prop:cylindre:transfert:reflexion}{}~

      On suppose qu'il existe \(\hat\mR_{p+1}(n)\) telle que 
      \begin{align*}
        \vE_t(r_{p}^+,n) &= \mN_{E}(r_p^+,n,\hat\mR_{p+1}(n))\vect{C}_{p+1}(n)
        \\
        \vect{e_r}\pvect\vH(r_{p}^+,n) &= \mN_{H}(r_p^+,n,\hat\mR_{p+1}(n))\vect{C}_{p+1}(n)
      \end{align*}

      Si les matrices suivantes sont inversibles
      \begin{align*}
        \mN_{E}(r_p^+,n,\hat\mR_{p+1}(n)), && \mN_{H}(r_p^+,n,\hat\mR_{p+1}(n)),
      \end{align*}
      \begin{align*}
        \mN_{E}(r_p^+,n,\hat\mR_{p+1}(n))^{-1}\mh_{E}(r_p^-,n) - \mN_{H}(r_p^+,n,\hat\mR_{p+1}(n))^{-1}\mh_{H}(r_p^-,n),
      \end{align*}
      alors
      \begin{align*}
        \vE_t(r_{p}^-,n) &= \mN_{E}(r_p^-,n,\mathfrak{T}_p(r_p^-,r_p^+,n,\hat\mR_{p+1}(n)))\vect{C}_{p}(n)
        \\
        \vE_t(r_{p}^-,n) &= \mN_{H}(r_p^-,n,\mathfrak{T}_p(r_p^-,r_p^+,n,\hat\mR_{p+1}(n)))\vect{C}_{p}(n)
      \end{align*}
    \end{prop}

    \begin{proof}
      De part et d'autre de \(r=r_p\), on a 
      \begin{align*}
        \vE_t(r_p^+,n) &= \mN_{E}(r_p^+,n,\hat\mR_{p+1}(n))\vect{C}_{1}^+(n)
        \\
        \vE_t(r_p^-,n) &= \mj_E(r_p^-,n)\vect{C}_{1}^-(n) + \mh_E(r_p^-,n)\vect{C}_{2}^-(n)
      \end{align*}
      \begin{align*}
        \vect{e_r}\pvect\vH(r_p^+,n) &= \mN_{H}(r_p^+,n,\hat\mR_{p+1}(n))\vect{C}_{1}^+(n)
        \\
        \vect{e_r}\pvect\vH(r_p^-,n) &= \mj_H(r_p^-,n)\vect{C}_{1}^-(n) + \mh_H(r_p^-,n)\vect{C}_{2}^-(n)
      \end{align*}
      Il y a continuité des champs au travers de l'interface donc
      \begin{align*}
        \mj_E(r_p^-,n)\vect{C}_{1}^-(n) + \mh_E(r_p^-,n)\vect{C}_{2}^-(n) &= \mN_{E}(r_p^+,n,\hat\mR_{p+1}(n))\vect{C}_{1}^+(n)
        \\
        \mj_H(r_p^-,n)\vect{C}_{1}^-(n) + \mh_H(r_p^-,n)\vect{C}_{2}^-(n) &= \mN_{H}(r_p^+,n,\hat\mR_{p+1}(n))\vect{C}_{1}^+(n)
      \end{align*}
      donc si on suppose les matrices \(\mN_E, \mN_H\) inversibles
      \begin{multline*}
        \mN_{E}(r_p^+,n,\hat\mR_{p+1}(n))^{-1}\left(\mj_E(r_p^-,n)\vect{C}_{1}^-(n) + \mh_E(r_p^-,n)\vect{C}_{2}^-(n)\right) =
        \\
        \mN_{H}(r_p^+,n,\hat\mR_{p+1}(n))^{-1}\left(\mj_H(r_p^-,n)\vect{C}_{1}^-(n) + \mh_H(r_p^-,n)\vect{C}_{2}^-(n)\right)
      \end{multline*}

      On factorise les termes
      \begin{multline*}
        \left(\mN_{E}(r_p^+,n,\hat\mR_{p+1}(n))^{-1}\mj_E(r_p^-,n) - \mN_{H}(r_p^+,n,\hat\mR_{p+1}(n))^{-1}\mj_H(r_p^-,n)\right)\vect{C}_{1}^-(n) =
        \\
        -\left(\mN_{E}(r_p^+,n,\hat\mR_{p+1}(n))^{-1}\mh_E(r_p^-,n) + \mN_{H}(r_p^+,n,\hat\mR_{p+1}(n))^{-1}\mh_H(r_p^-,n)\right)\vect{C}_{2}^-(n)
      \end{multline*}

      On suppose l'inversion de cette dernière et alors
      \begin{equation*}
        \vect{C}_{2}^-(n) = \mathfrak{T}_p(r_p^-,r_p^+,n,\hat\mR_{p+1}(n)) \vect{C}_{1}^-(n)
      \end{equation*}
    \end{proof}

    \begin{prop}%[Théorème de relévement]~
      \label{prop:cylindre:relevement:reflexion}{}~

      On suppose qu'il existe \(\hat\mR_{p}(n)\) telle que 
      \begin{align*}
        \vE_t(r_{p}^-,n) &= \mN_{E}(r_p^-,n,\hat\mR_{p}(n))\vect{C}_{p}(n)
        \\
        \vect{e_r}\pvect\vH(r_{p}^-,n) &= \mN_{H}(r_p^-,n,\hat\mR_{p}(n))\vect{C}_{p}(n)
      \end{align*}

      Si les matrices suivantes sont inversibles
      \begin{align*}
        \mN_{E}(r_p^-,n,\mR_{p}(n)), && \mN_{H}(r_p^-,n,\hat\mR_{p}(n)),
      \end{align*}
      \begin{align*}
        \mN_{E}(r_p^-,n,\mR_{p}(n))^{-1}\mh_{E}(r_p^+,n) - \mN_{H}(r_p^-,n,\hat\mR_{p}(n))^{-1}\mh_{H}(r_p^+,n),
      \end{align*}
      alors
      \begin{align*}
        \vE_t(r_{p}^+,n) &= \mN_{E}(r_p^+,n,\mathfrak{T}_p(r_p^+,r_p^-,n,\hat\mR_{p}(n)))\vect{C}_{p+1}(n)
        \\
        \vE_t(r_{p}^+,n) &= \mN_{H}(r_p^+,n,\mathfrak{T}_p(r_p^+,r_p^-,n,\hat\mR_{p}(n)))\vect{C}_{p+1}(n)
      \end{align*}
    \end{prop}

    \begin{proof}
      Même méthodologie que pour la proposition \ref{prop:cylindre:transfert:reflexion}.
    \end{proof}

    \begin{prop}%[Théorème de relévement]~
      \label{prop:cylindre:synthese:reflexion}{}~

      On suppose qu'il existe \(\hat\mR_{p}(n)\) et \(\hat\mR_{p+1}(n)\) telles que 
      \begin{align*}
      &\left\lbrace\begin{aligned}
        \vE_t(r_{p}^-,n) &= \mN_{E}(r_p^-,n,\hat\mR_{p}(n))\vect{C}_{p}(n)
        \\
        \vect{e_r}\pvect\vH(r_{p}^-,n) &= \mN_{H}(r_p^-,n,\hat\mR_{p}(n))\vect{C}_{p}(n)
        \end{aligned}
      \right.
      \\
      &\left\lbrace\begin{aligned}
        \vE_t(r_{p}^+,n) &= \mN_{E}(r_p^+,n,\hat\mR_{p+1}(n))\vect{C}_{p+1}(n)
        \\
        \vect{e_r}\pvect\vH(r_{p}^+,n) &= \mN_{H}(r_p^+,n,\hat\mR_{p+1}(n))\vect{C}_{p+1}(n)
        \end{aligned}
      \right.      
      \end{align*}

      Si les matrices suivantes sont inversibles
      \begin{align*}
        \mN_{E}(r_p^-,n,\mR_{p}(n)), && \mN_{H}(r_p^-,n,\hat\mR_{p}(n)),
        \\
        \mN_{E}(r_p^+,n,\mR_{p+1}(n)), && \mN_{H}(r_p^+,n,\hat\mR_{p+1}(n)),
      \end{align*}
      \begin{align*}
        \mN_{E}(r_p^-,n,\mR_{p}(n))^{-1}\mh_{E}(r_p^+,n) - \mN_{H}(r_p^-,n,\hat\mR_{p}(n))^{-1}\mh_{H}(r_p^+,n),
        \\
        \mN_{E}(r_p^+,n,\mR_{p+1}(n))^{-1}\mh_{E}(r_p^-,n) - \mN_{H}(r_p^+,n,\hat\mR_{p+1}(n))^{-1}\mh_{H}(r_p^-,n),
      \end{align*}
      alors
      \begin{align*}
        \hat\mR_{p+1}(n) &= \mathfrak{T}_p(r_p^+,r_p^-,n,\hat\mR_{p}(n))
        \\
        \hat\mR_{p}(n) &= \mathfrak{T}_p(r_p^-,r_p^+,n,\hat\mR_{p+1}(n))
      \end{align*}
    \end{prop}

    On peut donc déterminer itérativement les matrices de réflexions. Dans notre cadre d'étude, la présence d'un conducteur parfait sur l'interface \(r=r_0^+\) implique \(\mR_{1}(n) = -\mh_E(r_0^+,n)^{-1}\mj_E(r_0^+,n)\).

  \subsection{Applications numérique}

    La figure \ref{fig:imp_fourier:cylindre:hoppe_p62:converge_rayon} permet de vérifier les résultats de \cite[p.~62]{hoppe_impedance_1995} pour une couche de matériau sans perte (voir Figure \ref{fig:annex:hoppe:p62}). 

    Deux graphes sont superposés sur cette figure. D'abord, 3 courbes représentant les valeurs des parties imaginaires des termes diagonaux de la matrice \(\hat\mZ_{cylindre}\) en fonction de \((n)\) où \(k_z=0\) et \(n\) varie de \(0\) à \(\lfloor k_t r_1 \rfloor\) pour 3 rayons intérieur différent. Puis une courbe représentant les valeurs des parties imaginaires des termes diagonaux de la matrice \(\hat\mZ_{plan}\) en fonction de \((k_x,k_y)\) où \(k_y=0\)).
    \begin{figure}[!hbt]
      \centering
      \tikzsetnextfilename{Z_HOPPE_62_cylindre_converge.TM}
\begin{tikzpicture}[scale=1]
  \begin{axis}[
      title={Polarisation TM},
      ylabel={\(\Im(\hat{Z}(n,0));\Im(\hat{Z}(k_x,0))\)},
      xlabel={\(k_t \slash k_0 ; k_x \slash k_0\)},
      width=0.37\textwidth,
      xmin=0,
      xmax=1.5,
      legend pos=outer north east
    ]

    \addplot [black,dotted,mark=diamond] table [col sep=comma, x={s1}, y={Im(z_ex.11)}] {csv/HOPPE_62/HOPPE_62.z_ex.MODE_2_TYPE_C_+3.000E-02m.csv};

    \addplot [black,dotted,mark=*] table [col sep=comma, x={s1}, y={Im(z_ex.11)}] {csv/HOPPE_62/HOPPE_62.z_ex.MODE_2_TYPE_C_+3.000E-01m.csv};

    \addplot [black,dashed] table [col sep=comma, x={s1}, y={Im(z_ex.11)}] {csv/HOPPE_62/HOPPE_62.z_ex.MODE_2_TYPE_C_+3.000E+00m.csv};

    \addplot [black] table [col sep=comma, x={s1}, y={Im(z_ex.11)}] {csv/HOPPE_62/HOPPE_62.z_ex.MODE_2_TYPE_P.csv};
  \end{axis}
\end{tikzpicture}
\tikzsetnextfilename{Z_HOPPE_62_cylindre_converge.TE}
\begin{tikzpicture}[scale=1]
  \begin{axis}[
      title={Polarisation TE},
      ylabel={},
      xlabel={\(k_t \slash k_0 ; k_x \slash k_0\)},
      width=0.37\textwidth,
      xmin=0,
      xmax=1.5,
      legend pos=outer north east
    ]

    \addplot [black,dotted,mark=diamond] table [col sep=comma, x={s1}, y={Im(z_ex.22)}] {csv/HOPPE_62/HOPPE_62.z_ex.MODE_2_TYPE_C_+3.000E-02m.csv};
    \addlegendentry{\(r_0=0.03m\)}

    \addplot [black,dotted,mark=*] table [col sep=comma, x={s1}, y={Im(z_ex.22)}] {csv/HOPPE_62/HOPPE_62.z_ex.MODE_2_TYPE_C_+3.000E-01m.csv};
    \addlegendentry{\(r_0=0.3m\)}

    \addplot [black,dashed] table [col sep=comma, x={s1}, y={Im(z_ex.22)}] {csv/HOPPE_62/HOPPE_62.z_ex.MODE_2_TYPE_C_+3.000E+00m.csv};
    \addlegendentry{\(r_0=3m\)}

    \addplot [black] table [col sep=comma, x={s1}, y={Im(z_ex.22)}] {csv/HOPPE_62/HOPPE_62.z_ex.MODE_2_TYPE_P.csv};
    \addlegendentry{plan}
  \end{axis}
\end{tikzpicture}
      \caption{\(\eps = 6, \mu = 1, d=0.0225\text{m}, f=1\text{GHz}\)}
      \label{fig:imp_fourier:cylindre:hoppe_p62:converge_rayon}
    \end{figure}
    Pour une couche de matériau sans pertes, la matrice \(\hat\mZ\) est imaginaire pure, donc les parties réelles ne sont pas tracées.


    La figure \ref{fig:imp_fourier:cylindre:hoppe_p62:coeff_fourier} affiche le module des coefficients de la série de Fourier, c'est à dire les coefficients diagonaux de la matrice \(\hat\mR\) définie plus haut. Comme on peut le voir, dès que \(n\) dépasse \(k_0r_{1}\), le module des coefficients décroit très rapidement. Ce dépassement arrive pour \(n\) supérieur à 2 (resp. 7, 64) pour \(r_0\) égal à 0.03m (resp. 0.3m, 3m).

    \begin{figure}[!hbt]
      \centering
      % update mar. sept. 24 10:36:01 MET 2019
\tikzsetnextfilename{F_HOPPE_62_cylindre_converge_TM}
\begin{tikzpicture}[scale=1]
  \begin{loglogaxis}[
      title={Polarisation TM},
      ylabel={\(|\hat{R}(n,0)|\)},
      xlabel={\(n\)},
      width=0.36\textwidth,
      xmin=1,
      xmax=103,
      legend pos=outer north east
    ]

    \addplot [black,dotted,mark=diamond] table [col sep=comma, x={n}, y={Abs(f_ex.tm)}] {csv/HOPPE_62/HOPPE_62.f_ex.MODE_2_TYPE_C_+3.000E-02.csv};

    \addplot [black,dotted,mark=*] table [col sep=comma, x={n}, y={Abs(f_ex.tm)}] {csv/HOPPE_62/HOPPE_62.f_ex.MODE_2_TYPE_C_+3.000E-01.csv};

    \addplot [black,dashed] table [col sep=comma, x={n}, y={Abs(f_ex.tm)}] {csv/HOPPE_62/HOPPE_62.f_ex.MODE_2_TYPE_C_+3.000E+00.csv};

  \end{loglogaxis}
\end{tikzpicture}
\tikzsetnextfilename{F_HOPPE_62_cylindre_converge_TE}
\begin{tikzpicture}[scale=1]
  \begin{loglogaxis}[
      title={Polarisation TE},
      ylabel={},
      xlabel={\(n\)},
      width=0.36\textwidth,
      xmin=1,
      xmax=103,
      legend pos=outer north east
    ]

    \addplot [black,dotted,mark=diamond] table [col sep=comma, x={n}, y={Abs(f_ex.te)}] {csv/HOPPE_62/HOPPE_62.f_ex.MODE_2_TYPE_C_+3.000E-02.csv};
    \addlegendentry{\(r_0=0.03m\)}

    \addplot [black,dotted,mark=*] table [col sep=comma, x={n}, y={Abs(f_ex.te)}] {csv/HOPPE_62/HOPPE_62.f_ex.MODE_2_TYPE_C_+3.000E-01.csv};
    \addlegendentry{\(r_0=0.3m\)}

    \addplot [black,dashed] table [col sep=comma, x={n}, y={Abs(f_ex.te)}] {csv/HOPPE_62/HOPPE_62.f_ex.MODE_2_TYPE_C_+3.000E+00.csv};
    \addlegendentry{\(r_0=3m\)}

  \end{loglogaxis}
\end{tikzpicture}
      \caption{\(\eps = 6, \mu = 1, d=0.0225\text{m}, f=1\text{GHz}\)}
      \label{fig:imp_fourier:cylindre:hoppe_p62:coeff_fourier}
    \end{figure}


% !-!-!-!-!!--!-!!-!-!-!--!-!-!-!-!-!-!-!!!!!!!!!!!!!!!!!!!!!!!!!!!!!!!!!!!!!!!!!!!!!!!!!!!!!!!!!!!!!!!!!!
% !-!-!-!-!!--!-!!-!-!-!--!-!-!-!-!-!-!-!!!!!!!!!!!!!!!!!!!!!!!!!!!!!!!!!!!!!!!!!!!!!!!!!!!!!!!!!!!!!!!!!!
% !-!-!-!-!!--!-!!-!-!-!--!-!-!-!-!-!-!-!!!!!!!!!!!!!!!!!!!!!!!!!!!!!!!!!!!!!!!!!!!!!!!!!!!!!!!!!!!!!!!!!!
% !-!-!-!-!!--!-!!-!-!-!--!-!-!-!-!-!-!-!!!!!!!!!!!!!!!!!!!!!!!!!!!!!!!!!!!!!!!!!!!!!!!!!!!!!!!!!!!!!!!!!!
% !-!-!-!-!!--!-!!-!-!-!--!-!-!-!-!-!-!-!!!!!!!!!!!!!!!!!!!!!!!!!!!!!!!!!!!!!!!!!!!!!!!!!!!!!!!!!!!!!!!!!!
% !-!-!-!-!!--!-!!-!-!-!--!-!-!-!-!-!-!-!!!!!!!!!!!!!!!!!!!!!!!!!!!!!!!!!!!!!!!!!!!!!!!!!!!!!!!!!!!!!!!!!!
% !-!-!-!-!!--!-!!-!-!-!--!-!-!-!-!-!-!-!!!!!!!!!!!!!!!!!!!!!!!!!!!!!!!!!!!!!!!!!!!!!!!!!!!!!!!!!!!!!!!!!!
% !-!-!-!-!!--!-!!-!-!-!--!-!-!-!-!-!-!-!!!!!!!!!!!!!!!!!!!!!!!!!!!!!!!!!!!!!!!!!!!!!!!!!!!!!!!!!!!!!!!!!!
% !-!-!-!-!!--!-!!-!-!-!--!-!-!-!-!-!-!-!!!!!!!!!!!!!!!!!!!!!!!!!!!!!!!!!!!!!!!!!!!!!!!!!!!!!!!!!!!!!!!!!!
% !-!-!-!-!!--!-!!-!-!-!--!-!-!-!-!-!-!-!!!!!!!!!!!!!!!!!!!!!!!!!!!!!!!!!!!!!!!!!!!!!!!!!!!!!!!!!!!!!!!!!!
% !-!-!-!-!!--!-!!-!-!-!--!-!-!-!-!-!-!-!!!!!!!!!!!!!!!!!!!!!!!!!!!!!!!!!!!!!!!!!!!!!!!!!!!!!!!!!!!!!!!!!!
% !-!-!-!-!!--!-!!-!-!-!--!-!-!-!-!-!-!-!!!!!!!!!!!!!!!!!!!!!!!!!!!!!!!!!!!!!!!!!!!!!!!!!!!!!!!!!!!!!!!!!!
% !-!-!-!-!!--!-!!-!-!-!--!-!-!-!-!-!-!-!!!!!!!!!!!!!!!!!!!!!!!!!!!!!!!!!!!!!!!!!!!!!!!!!!!!!!!!!!!!!!!!!!
% !-!-!-!-!!--!-!!-!-!-!--!-!-!-!-!-!-!-!!!!!!!!!!!!!!!!!!!!!!!!!!!!!!!!!!!!!!!!!!!!!!!!!!!!!!!!!!!!!!!!!!
% !-!-!-!-!!--!-!!-!-!-!--!-!-!-!-!-!-!-!!!!!!!!!!!!!!!!!!!!!!!!!!!!!!!!!!!!!!!!!!!!!!!!!!!!!!!!!!!!!!!!!!
% !-!-!-!-!!--!-!!-!-!-!--!-!-!-!-!-!-!-!!!!!!!!!!!!!!!!!!!!!!!!!!!!!!!!!!!!!!!!!!!!!!!!!!!!!!!!!!!!!!!!!!
% !-!-!-!-!!--!-!!-!-!-!--!-!-!-!-!-!-!-!!!!!!!!!!!!!!!!!!!!!!!!!!!!!!!!!!!!!!!!!!!!!!!!!!!!!!!!!!!!!!!!!!
% !-!-!-!-!!--!-!!-!-!-!--!-!-!-!-!-!-!-!!!!!!!!!!!!!!!!!!!!!!!!!!!!!!!!!!!!!!!!!!!!!!!!!!!!!!!!!!!!!!!!!!
% !-!-!-!-!!--!-!!-!-!-!--!-!-!-!-!-!-!-!!!!!!!!!!!!!!!!!!!!!!!!!!!!!!!!!!!!!!!!!!!!!!!!!!!!!!!!!!!!!!!!!!
% !-!-!-!-!!--!-!!-!-!-!--!-!-!-!-!-!-!-!!!!!!!!!!!!!!!!!!!!!!!!!!!!!!!!!!!!!!!!!!!!!!!!!!!!!!!!!!!!!!!!!!
% !-!-!-!-!!--!-!!-!-!-!--!-!-!-!-!-!-!-!!!!!!!!!!!!!!!!!!!!!!!!!!!!!!!!!!!!!!!!!!!!!!!!!!!!!!!!!!!!!!!!!!
% !-!-!-!-!!--!-!!-!-!-!--!-!-!-!-!-!-!-!!!!!!!!!!!!!!!!!!!!!!!!!!!!!!!!!!!!!!!!!!!!!!!!!!!!!!!!!!!!!!!!!!
% !-!-!-!-!!--!-!!-!-!-!--!-!-!-!-!-!-!-!!!!!!!!!!!!!!!!!!!!!!!!!!!!!!!!!!!!!!!!!!!!!!!!!!!!!!!!!!!!!!!!!!
% !-!-!-!-!!--!-!!-!-!-!--!-!-!-!-!-!-!-!!!!!!!!!!!!!!!!!!!!!!!!!!!!!!!!!!!!!!!!!!!!!!!!!!!!!!!!!!!!!!!!!!
% !-!-!-!-!!--!-!!-!-!-!--!-!-!-!-!-!-!-!!!!!!!!!!!!!!!!!!!!!!!!!!!!!!!!!!!!!!!!!!!!!!!!!!!!!!!!!!!!!!!!!!
% !-!-!-!-!!--!-!!-!-!-!--!-!-!-!-!-!-!-!!!!!!!!!!!!!!!!!!!!!!!!!!!!!!!!!!!!!!!!!!!!!!!!!!!!!!!!!!!!!!!!!!
% !-!-!-!-!!--!-!!-!-!-!--!-!-!-!-!-!-!-!!!!!!!!!!!!!!!!!!!!!!!!!!!!!!!!!!!!!!!!!!!!!!!!!!!!!!!!!!!!!!!!!!
% !-!-!-!-!!--!-!!-!-!-!--!-!-!-!-!-!-!-!!!!!!!!!!!!!!!!!!!!!!!!!!!!!!!!!!!!!!!!!!!!!!!!!!!!!!!!!!!!!!!!!!
% !-!-!-!-!!--!-!!-!-!-!--!-!-!-!-!-!-!-!!!!!!!!!!!!!!!!!!!!!!!!!!!!!!!!!!!!!!!!!!!!!!!!!!!!!!!!!!!!!!!!!!
% !-!-!-!-!!--!-!!-!-!-!--!-!-!-!-!-!-!-!!!!!!!!!!!!!!!!!!!!!!!!!!!!!!!!!!!!!!!!!!!!!!!!!!!!!!!!!!!!!!!!!!
% !-!-!-!-!!--!-!!-!-!-!--!-!-!-!-!-!-!-!!!!!!!!!!!!!!!!!!!!!!!!!!!!!!!!!!!!!!!!!!!!!!!!!!!!!!!!!!!!!!!!!!
% !-!-!-!-!!--!-!!-!-!-!--!-!-!-!-!-!-!-!!!!!!!!!!!!!!!!!!!!!!!!!!!!!!!!!!!!!!!!!!!!!!!!!!!!!!!!!!!!!!!!!!
% !-!-!-!-!!--!-!!-!-!-!--!-!-!-!-!-!-!-!!!!!!!!!!!!!!!!!!!!!!!!!!!!!!!!!!!!!!!!!!!!!!!!!!!!!!!!!!!!!!!!!!


\section{Cas d'un objet sphérique}

    Les champs solutions de Maxwell dans le cas d'un repère sphérique sont décomposables en harmoniques sphériques. Nous rappelons d'abord l’expression de ces dernières puis nous donnerons l'expression de la matrice d'impédance de la même manière que \cite{cheng_spectral_1993}.

    \subsection{Les harmoniques sphériques}

        % \begin{TODO}
        %   Mettre ici la démonstration des harmoniques sphériques? Ou une référence vers annexe ?
        % \end{TODO}

        On définit les harmoniques sphériques les solutions de \(\Delta U + k^2 U = 0 \). Ce sont les fonctions \(Y_{m,n} = C(m,n) e^{im\phi}\PP^m_n(\cos \theta) \) avec \(C(m,n)\)\footnote{D’après \cite[p.~24]{nedelec_acoustic_2001}, \( C(m,n) = (-1)^m\sqrt{\frac{2n+1}{4\pi}\frac{(n-m)!}{(n+m)!}}\)} tel que
        \[
         \ds\int_S Y_{m,n} \conj{Y_{p,q}} ds = \delta_m^p \delta_{nm}^q
        \]


         On définit les vecteurs harmonique sphériques\(\gls{phy-Mmn} ,\gls{phy-Nmn}\) solution dans la base des coordonnées sphériques de
        \[
            \left\lbrace
                \begin{aligned}
                    \vrot \vrot \vect{U} - k^2 \vect{U} = 0\\
                    \vdiv \vect{U} = 0
                \end{aligned}
            \right.
        \]
        \begin{align}
            \label{eq:defMmn}
            \Mmn[z_n](\rtp) &:= \vrot \left( \vect{r} z_n(k(r)r) Y_{m,n}(\tp) \right)\\
            &= z_n(k(r)r)
            \begin{bmatrix}
                0
                \\
                \frac{im}{\sin\theta}Y_{mn}(\tp)
                \\
                - \ddr{\theta}{Y_{mn}}(\tp)
            \end{bmatrix}
        \end{align}

        \begin{defn}
            On définit l’opérateur \gls{mat-tild} agissant sur les fonctions dérivables tel que 

            Soit \(f \in C^1(\CC)\), on pose
            \begin{align*}
            &\tilde{f}(z)=f(z) + zf'(z)&& \forall z \in \CC
            \end{align*}
        \end{defn}

        \begin{align}
        \label{eq:defNnn}
          \Nmn[z_n](\rtp) &:= \frac{\vrot \Mmn[z_n]}{k}(\rtp) \\
          &= \frac{1}{k(r)r}\begin{bmatrix}
            z_n(k(r)r)n(n+1)Y_{mn}(\tp)
            \\
            \tilde{z_n}(k(r)r)\ddr{\theta}{Y_{mn}}(\tp)
            \\
            \tilde{z_n}(k(r)r)\frac{im}{\sin\theta}Y_{mn}(\tp)
          \end{bmatrix}
        \end{align}

        L'obtention de ces vecteurs est disponible en annexes \ref{sec:annex:harmoniques_spheriques}.

        Par définition de ces vecteurs, on a les propriétés suivantes
        \begin{prop}
            \label{prop:Mmn_Nmn_rot}
            \begin{align*}
                \vrot \Mmn[z_n](\rtp) &= k\Nmn[z_n](\rtp)
                \\
                \vrot \Nmn[z_n](\rtp) &= k\Mmn[z_n](\rtp)
            \end{align*}
        \end{prop}

        % Ces vecteurs harmoniques sphériques possèdent les propriétés suivantes

        % \begin{align}
        % \int_{S(0,R)} \vect{M_{m,n}^{z_n}} \cdot \conj{\vect{N_{p,q}^{z_n}}} ds &= 0
        % \\
        % \int_{S(0,R)} \vect{M_{m,n}^{z_n}} \cdot \conj{\vect{M_{p,q}^{z_n}}} ds &= \gamma_{m,n}R^2 \delta_{mp}\delta_{nq}
        % \\
        % \int_{S(0,R)} \vect{N_{m,n}^{z_n}} \cdot \conj{\vect{N_{p,q}^{z_n}}} ds &= \frac{\gamma_{m,n}}{k^2} \delta_{mp}\delta_{nq}
        % \end{align}

        On a (\cite{cheng_spectral_1993})
        \begin{multline*}
            \vE(\rtp) = \sum_{n\in\ZZ}\sum_{m\in\ZZ} a_{mn} \Mmn[j_n](\rtp) + b_{mn} \Nmn[j_n](\rtp)
            \\
            + c_{mn} \Mmn[h^{(2)}_n](\rtp) + d_{mn} \Nmn[h^{(2)}_n](\rtp)
        \end{multline*}

        D'après les équations de Maxwell, \(\vH = i\frac{\vrot \vE}{\w\mu}\)

        \begin{multline}
            \vH(\rtp) = \frac{ik}{\w\mu}\sum_{n\in\ZZ}\sum_{m\in\ZZ} a_{mn} \Nmn[j_n](\rtp) + b_{mn} \Mmn[j_n](\rtp)
            \\
            + c_{mn} \Nmn[h^{(2)}_n](\rtp) + d_{mn} \Mmn[h^{(2)}_n](\rtp)
        \end{multline}

        On définit alors le vecteur de \(\RR^2\) suivant ( resp. son orthogonal ), la réduction du vecteur harmonique sphérique \gls{phy-Mmn} ( resp. \gls{phy-Nmn} ) aux composantes tangentielles et indépendant du rayon:

        \begin{align}
            \label{eq:defUmn_tgt}
            \Umn(\tp) &=
            \begin{bmatrix}
                \frac{im}{\sin\theta}Y_{mn}(\tp)
                \\
                - \ddr{\theta}{Y_{mn}}(\tp)
            \end{bmatrix}
        \end{align}

        \begin{align}
        \label{eq:defNmn_tgt}
          \Umn^\perp(\tp) &=
          \begin{bmatrix}
            \ddr{\theta}{Y_{mn}}(\tp)
            \\
            \frac{im}{\sin\theta}Y_{mn}(\tp)
          \end{bmatrix}
        \end{align}


        On remarque alors que les parties tangentielles des vecteurs harmoniques sphériques peuvent s'écrire:
        \begin{align*}
          \Mmn[z_n]_t(\rtp) &= z_n(k(r)r)\Umn(\tp)
          \\
          \Nmn[z_n]_t(\rtp) &= \frac{1}{k(r)r}\tilde{z_n}(k(r)r)\Umn^\perp(\tp)
        \end{align*}

        On a alors les propriétés supplémentaires
        \begin{prop}
            \label{prop:Mmn_Nmn_vect}
            \begin{align*}
              \vect{e_r} \pvect \Mmn[z_n]_t(\rtp) &= k(r)rz_n(k(r)r)\Umn^\perp(\tp)
              \\
              \vect{e_r} \pvect \Nmn[z_n]_t(\rtp) &= -\frac{1}{k(r)r}\tilde{z_n}(k(r)r)\Umn(\tp)
            \end{align*}
        \end{prop}
        Sachant donc que les composantes tangentielles du champs \(\vE\) s'écrivent

        \begin{multline*}
            \vE_t(\rtp) = \sum_{n\in\ZZ}\sum_{m\in\ZZ} a_{mn} j_n(k(r)r)\Umn(\tp) + b_{mn} \frac{1}{k(r)r}\tilde{j_n}(k(r)r)\Umn^\perp(\tp)
            \\
            + c_{mn} h^{(2)}_n(k(r)r)\Umn(\tp) + d_{mn} \frac{1}{k(r)r}\tilde{j_n}(k(r)r)\Umn^\perp(\tp)
        \end{multline*}


        Donc \(\vJ = \vect{e_r} \pvect \vH\) s'écrit

        \begin{multline*}
            \vJ(\rtp) = \frac{ik}{\w\mu}\sum_{n\in\ZZ}\sum_{m\in\ZZ} - a_{mn} \frac{1}{k(r)r}\tilde{j_n}(k(r)r)\Umn(\tp) + b_{mn} k(r)r j_n(k(r)r) \Umn^\perp(\tp)
            \\
            -  \frac{1}{k(r)r}\tilde{h^{(2)}_n}(k(r)r) c_{mn} \Umn(\tp) + k(r)r h^{(2)}_n(k(r)r) d_{mn} \Umn^\perp(\tp)
        \end{multline*}

        Alors on a 

        \begin{multline*}
            \vE_t(\rtp) = \sum_{n\in\ZZ}\sum_{m\in\ZZ}\\
            \begin{bmatrix}
              \Umn & \Umn^\perp
            \end{bmatrix}
            \left(
              \begin{bmatrix}
                j_n(k(r)r) & 0
                \\              
                0 & \tilde{j_n}(k(r)r)
              \end{bmatrix}
              \begin{bmatrix}
                  a_{mn}
                  \\
                  b_{mn}
              \end{bmatrix}
              +
              \begin{bmatrix}
                h^{(2)}_n(k(r)r) & 0
                \\
                0 & \tilde{h^{(2)}_n}(k(r)r)
              \end{bmatrix}
              \begin{bmatrix}
                  c_{mn}
                  \\
                  d_{mn}
              \end{bmatrix}
            \right)
        \end{multline*}


        \begin{multline*}
            \vJ(\rtp) = \frac{i}{\w\mu r}\sum_{n\in\ZZ}\sum_{m\in\ZZ}\\
            \begin{bmatrix}
                \Umn & \Umn^\perp
            \end{bmatrix}
            \left(
                \begin{bmatrix}
                    -\tilde{j_n}(k(r)r) & 0
                    \\
                    0 & (k(r)r)^2 j_n(k(r)r)
                \end{bmatrix}
                \begin{bmatrix}
                    a_{mn}
                    \\
                    b_{mn}
                \end{bmatrix}
                +
                \begin{bmatrix}
                    -\tilde{h^{(2)}_n}(k(r)r) & 0
                    \\                
                    0 & (k(r)r)^2 h^{(2)}_n(k(r)r)
                \end{bmatrix}
                \begin{bmatrix}
                    c_{mn}
                    \\
                    d_{mn}
                \end{bmatrix}
            \right)
        \end{multline*}

        \begin{defn}
            On définit les vecteurs de \(\RR^2\) \(\hat{\vE_t}(r,m,n)\) et \(\hat{\vJ}(r,m,n)\) tels que
            \begin{align*}
                \vE_t(\rtp) &= \sum_{n\in\ZZ}\sum_{m\in\ZZ}
                \begin{bmatrix}
                  \Umn & \Umn^\perp
                \end{bmatrix}\hat{\vE_t}(r,m,n)
                \\
                \vJ(\rtp) &= \sum_{n\in\ZZ}\sum_{m\in\ZZ}
                \begin{bmatrix}
                  \Umn & \Umn^\perp
                \end{bmatrix}\hat{\vJ}(r,m,n)
            \end{align*}
        \end{defn}

        \begin{defn}
            On définit les matrices \(\mj_{E}(r,n),\mh_{E}(r,n),\mj_{H}(r,n),\mh_{H}(r,n)\)
            \begin{align*}
                \mj_{E}(r,n) &=
                \begin{bmatrix}
                    j_n(k(r)r) & 0
                    \\                
                    0 & \tilde{j_n}(k(r)r)
                \end{bmatrix}
                \\
                \mh_{E}(r,n) &=
                \begin{bmatrix}
                    h^{(2)}_n(k(r)r) & 0
                    \\                
                    0 & \tilde{h^{(2)}_n}(k(r)r)
                \end{bmatrix}
                \\
                \mj_{H}(r,n) &= \frac{i}{\w\mu r}
                \begin{bmatrix}
                    -\tilde{j_n}(k(r)r) & 0
                    \\                
                    0 & (k(r)r)^2 j_n(k(r)r)
                \end{bmatrix}
                \\
                \mh_{H}(r,n) &= \frac{i}{\w\mu r}
                \begin{bmatrix}
                    -\tilde{h^{(2)}_n}(k(r)r) & 0
                    \\
                    0 & (k(r)r)^2 h^{(2)}_n(k(r)r)
                \end{bmatrix}
            \end{align*}
        \end{defn}

        On peut donc expliciter les vecteurs précédemment introduits

        \begin{equation*}
            \hat{\vE_t}(r,m,n) =
            \mj_{E}(r,n)
            \begin{bmatrix}
                a_{mn}
                \\
                b_{mn}
            \end{bmatrix}
            +
            \mh_{E}(r,n)
            \begin{bmatrix}
                c_{mn}
                \\
                d_{mn}
            \end{bmatrix}
        \end{equation*}

        \begin{equation*}
            \hat{\vJ}(r,m,n) = 
            \mj_{H}(r,n)
            \begin{bmatrix}
                a_{mn}
                \\
                b_{mn}
            \end{bmatrix}
            +
            \mh_{H}(r,n)
            \begin{bmatrix}
                c_{mn}
                \\
                d_{mn}
            \end{bmatrix}
        \end{equation*}

    \subsection{Matrice d'impédance pour une couche}

        \begin{figure}[!hbt]
          \centering
            \tikzsetnextfilename{sphere_1_couche}
          \begin{tikzpicture}
            \tikzmath{
    \a = 80;
    \b = 100;
    \d = 0.5;
    \ri = 20;
    \re = \ri;
}

% Le conducteur
\tikzmath{
    \ri = \re;
    \re = \ri + 0.5*\d;
    \xa = cos(\a)*\re;
    \ya = sin(\a)*\re;
    \xb = cos(\b)*\ri;
    \yb = sin(\b)*\ri;
}

\coordinate (a) at (\xa,\ya);
\coordinate (b) at (\xb,\yb);

\fill [pattern=north east lines] (a) arc (\a:\b:\re) -- (b) arc (\b:\a:\ri) -- cycle;
\draw (a) arc (\a:\b:\re);
\draw (a) node [right] {$r_0$};


% Le repère
\coordinate (n) at ($(a)+(0.5,-1)$);
%
%
%\draw [->] (n) -- ++(0,1) node [at end, right] {$\v{\pr}$};
%\draw [->] (n) -- ++(1,0) node [at end, right] {$\v{\pt}$};
%
\draw (n) ++(0.2,0.2) circle(0.1cm) node [above=0.1cm] {\(\vect{e_\phi}\)};
\draw (n) ++(0.2,0.2) +(135:0.1cm) -- +(315:0.1cm);
\draw (n) ++(0.2,0.2) +(45:0.1cm) -- +(225:0.1cm);


% 1ere couche
\tikzmath{
    \ri = \re;
    \re = \ri + \d;
    \xa = cos(\a)*\re;
    \ya = sin(\a)*\re;
    \xb = cos(\b)*\ri;
    \yb = sin(\b)*\ri;
    \xc = cos(0.5*(\b+\a))*(\ri+0.5*\d);
    \yc = sin(0.5*(\b+\a))*(\ri+0.5*\d);
}

\coordinate (a) at (\xa,\ya);
\coordinate (b) at (\xb,\yb);
\coordinate (c) at (\xc,\yc);

\fill [lightgray] (a) arc (\a:\b:\re) -- (b) arc (\b:\a:\ri) -- cycle;
\draw (a) arc (\a:\b:\re);
\draw (c) node {$\nu,\eta,d$};

% Le vide
\tikzmath{
    \xc = cos(0.5*(\b+\a))*(\re);
    \yc = sin(0.5*(\b+\a))*(\re);
}

\draw (\xc,\yc) node [above] {vide};
          \end{tikzpicture}
        \end{figure}

        \begin{defn}
          On définit la matrice d'impédance \(\hat{\mZ}(m,n)\) tel que
          \[
              \hat{\vE_t}(r_1,m,n) = \hat{\mZ}(m,n)\hat{\vJ}(r_1,m,n)
          \]
        \end{defn}

        En \(r=r_0\), on a la relation \(\vE_t(\rtp) = 0\) donc \(\hat{\vE_t}(r_0,m,n) = 0 \)

        \begin{equation*}
            \mj_{E}(r_0,n)
            \begin{bmatrix}
                a_{mn}
                \\
                b_{mn}
            \end{bmatrix}
            = -
            \mh_{E}(r_0,n)
            \begin{bmatrix}
                c_{mn}
                \\
                d_{mn}
            \end{bmatrix}
        \end{equation*}

        On suppose que les matrices \(\mj_{E}(r_0,n)\) et \(\mh_{E}(r_0,n)\) soient inversibles.

        % \begin{TODO}
        %   Inversibilité \(\mj_{E}(r,n), \mh_{E}(r,n)\)
        % \end{TODO}

        \begin{equation*}
            \hat{\vE_t}(r,m,n) =
            \left(
                \mh_{E}(r,n)
                -
                \mj_{E}(r,n)
                \mj_{E}(r_0,n)^{-1}
                \mh_{E}(r_0,n)
            \right)
            \begin{bmatrix}
                c_{mn}
                \\
                d_{mn}
            \end{bmatrix}
        \end{equation*}


        \begin{equation*}
            \hat{\vJ}(r,m,n) = \frac{i}{\eta}
            \left(
                \mh_{H}(r,n)
                -
                \mj_{H}(r,n)
                \mj_{E}(r_0,n)^{-1}
                \mh_{E}(r_0,n)
            \right)
            \begin{bmatrix}
                c_{mn}
                \\
                d_{mn}
            \end{bmatrix}
        \end{equation*}

        De la même manière que pour le plan et le cylindre, on en déduit la matrice d'impédance

        \begin{multline*}
            \hat{\mZ}(m,n) = -i\eta
            \left(
                \mh_{E}(r_1,n)
                \mh_{E}(r_0,n)^{-1}
                -
                \mj_{E}(r_1,n)
                \mj_{E}(r_0,n)^{-1}
            \right)
            \\
            \left(
                \mh_{H}(r_1,n)
                \mh_{E}(r_0,n)^{-1}
                -
                \mj_{H}(r_1,n)
                \mj_{E}(r_0,n)^{-1}
            \right)^{-1}
        \end{multline*}

        Par définition des matrices \(\mj_E,\mh_E,\mj_H,\mh_H\), elle sont anti-diagonale. Donc leur inverse l'est aussi. Donc le produit de l'une avec l'inverse d'une autre est une matrice diagonale. Donc la matrice d'impédance est une matrice diagonale.

        \begin{equation*}
            \hat{\mZ}(m,n) = -i\eta
            \begin{bmatrix}
                \frac
                {\tilde{h^{(2)}_n}(k(r)r_1)\tilde{j_n}(k(r)r_0)-\tilde{j_n}(k(r)r_1)\tilde{h^{(2)}_n}(k(r)r_0)}
                {{h^{(2)}_n}(k(r)r_1)\tilde{j_n}(k(r)r_0)-{j_n}(k(r)r_1)\tilde{h^{(2)}_n}(k(r)r_0)} & 0
                \\
                0 & \frac
                {{j_n}(k(r)r_1){h^{(2)}_n}(k(r)r_0)-{h^{(2)}_n}(k(r)r_1){j_n}(k(r)r_0)}
                {\tilde{h^{(2)}_n}(k(r)r_1){j_n}(k(r)r_0)-\tilde{j_n}(k(r)r_1){h^{(2)}_n}(k(r)r_0)}
            \end{bmatrix}
        \end{equation*}

    \subsection{Matrice d'impédance pour plusieurs couches}
    
        \begin{figure}[!hbt]
          \centering
            \tikzsetnextfilename{sphere_n_couche}
          \begin{tikzpicture}
            \input{tikz/schema/sphere_n_couche.tikz}
          \end{tikzpicture}
        \end{figure}


        \begin{defn}
          Pour chaque couche \(p\), on définit la matrice d'impédance \(\hat{\mZ}_p(m,n)\) tel que
          \[
              \hat{\vE_t}(r_p,m,n) = \hat{\mZ}_p(m,n)\hat{\vJ}(r_p,m,n)
          \]
        \end{defn}

        On résonne par récurrence: on suppose connu la matrice d'impédance de la couche \(p\) \(\hat{\mZ}_p\) et on cherche \(\hat{\mZ}_{p+1}\).

        En \(r=r_{p}=r_0+\sum_{i=1}^p d_p\), on a la relation \( \hat{\vE_t}(r_p,m,n) = \hat{\mZ}_p(m,n)\hat{\vJ}(r_p,m,n)\) où \(\hat{\mZ}_p(m,n)\) est un matrice diagonale

        \begin{equation*}
            \left(\mj_{E}(r_p,n) - \frac{i}{\eta_p}\hat{\mZ}_p(m,n)\mj_{H}(r_p,n) \right)
            \begin{bmatrix}
                a_{mn}
                \\
                b_{mn}
            \end{bmatrix}
            = -
            \left(\mh_{E}(r_p,n) - \frac{i}{\eta_p}\hat{\mZ}_p(m,n)\mh_{H}(r_p,n) \right)
            \begin{bmatrix}
                c_{mn}
                \\
                d_{mn}
            \end{bmatrix}
        \end{equation*}

        On définit les matrices \(\mA_{J}(r,n)\) et \(\mA_{H}(r,n)\) telle que

        \begin{align*}
            \mA_{J}(r,n) &= \mj_{E}(r,n) - \frac{i}{\eta_p}\hat{\mZ}_p(m,n)\mj_{H}(r,n)
            \\
            \mA_{H}(r,n) &= \mh_{E}(r,n) - \frac{i}{\eta_p}\hat{\mZ}_p(m,n)\mh_{H}(r,n)
        \end{align*}

        On suppose que les matrices \(\mA_{E}(r,n)\) et \(\mA_{H}(r,n)\) soient inversibles.

        % \begin{TODO}
        %   Inversibilité \(\mA_{E}(r_p,n)\) et \(\mA_{H}(r_p,n)\)
        % \end{TODO}

        Par hypothèse sur \(\hat{\mZ_p}(m,n)\), ces matrices sont anti-diagonale.

        On en déduit

        \begin{equation*}
            \hat{\vE_t}(r_{p+1},m,n) =
            \left(
                \mh_{E}(r_{p+1},n)
                -
                \mj_{E}(r_{p+1},n)
                \mA_{J}(r_p,n)^{-1}
                \mA_{H}(r_p,n)
            \right)
            \begin{bmatrix}
                c_{mn}
                \\
                d_{mn}
            \end{bmatrix}
        \end{equation*}


        \begin{equation*}
            \hat{\vJ}(r_{p+1},m,n) = \frac{i}{\eta_p}
            \left(
                \mh_{H}(r_{p+1},n)
                -
                \mj_{H}(r_{p+1},n)
                \mA_{J}(r_{p+1},n)^{-1}
                \mA_{H}(r_{p+1},n)
            \right)
            \begin{bmatrix}
                c_{mn}
                \\
                d_{mn}
            \end{bmatrix}
        \end{equation*}

        On en déduit la matrice de la couche \(p+1\)

        \begin{multline*}
            \hat{\mZ}_{p+1}(m,n) = -i\eta_p
            \left(
                \mh_{E}(r_{p+1},n)
                \mA_{H}(r_p,n)^{-1}
                -
                \mj_{E}(r_{p+1},n)
                \mA_{E}(r_p,n)^{-1}
            \right)
            \\
            \left(
                \mh_{H}(r_{p+1},n)
                \mA_{H}(r_p,n)^{-1}
                -
                \mj_{H}(r_{p+1},n)
                \mA_{E}(r_p,n)^{-1}
            \right)^{-1}
        \end{multline*}

        Pour les mêmes raisons que dans le cas d'une couche, cette matrice est diagonale. Cependant son expression n'est plus aussi simple (voir annexe \ref{sec:annex:imp_sphere} ).

  \subsection{Applications numérique}

    % \begin{figure}[!hbt]
    %     \centering
    %     \tikzsetnextfilename{Z_HOPPE_62_sphere_erreur}
\begin{tikzpicture}[scale=1]
\begin{loglogaxis}[
    title={},
    ylabel={\(||\hat{\mZ}_{plan} - \hat{\mZ}_{sphere}||_2\)},
    xlabel={\(r_0/d\)},
    width=0.8\textwidth,
    xmin=0.1,
    xmax=100,
    % mark repeat=20,
    legend pos=outer north east
  ]
  \legend{TM,TE}
  \addplot [black] table [x={r0/d}, y={tm},col sep=semicolon] {csv/sphere/hoppe_p62_error.csv};
  \addplot [black,dashed] table [x={r0/d}, y={te},col sep=semicolon] {csv/sphere/hoppe_p62_error.csv};
\end{loglogaxis}
\end{tikzpicture}
    %     \caption{\(\eps = 6, \mu = 1, d=0.0225\text{m}, f=1\text{GHz}\)}
    %     \label{fig:imp_fourier:sphere:hoppe_p62:converge_rayon:error}
    % \end{figure}
\section{Approximation de la matrice d'impédance pour une sphère par une CIOE}

  \subsection[Expression des opérateurs LD,LR en Fourier]{Expression des opérateurs \(\LD,\LR\) en Fourier}

    Par définition de \(\LD\), on a
    \begin{align}
      \LD \vE_t & = \vgrads{} \vdivs{} \vE_t
    \end{align}
    Or 
    \begin{align*}
      \vE_t(\rtp) &= \sum_{n\in\ZZ}\sum_{m\in\ZZ} (a_{mn} j_n + c_{mn}h_n) \Umn(\rtp) + (b_{mn} \tilde{j_n}+d_{mn} \tilde{h_n}) \Umn^\perp(\rtp)
      %\intertext{donc}
      %\LD\vE_t(\rtp) &= \sum_{n\in\ZZ}\sum_{m\in\ZZ} (a_{mn} j_n + c_{mn}h_n) \LD\Umn(\rtp) + (b_{mn} \tilde{j_n}+d_{mn} \tilde{h_n}) \LD\Umn^\perp(\rtp)
    \end{align*}

    On rappelle  les expressions des vecteurs (voir \eqref{eq:defUmn_tgt}, \eqref{eq:defNmn_tgt}) dans la base sphérique (\(\vect{e_r},\vect{e_\theta},\vect{e_\phi}\))
    \begin{align*}
      \Umn(\tp) =
      \begin{bmatrix}
          0
          \\
          \frac{im}{\sin\theta}\Pmn(\cos(\theta))e^{im\phi}
          \\
          - \ddr{\theta}{\Pmn(\cos(\theta))}e^{im\phi}
      \end{bmatrix}
      &&
      \Umn^\perp(\tp) =
      \begin{bmatrix}
        0
        \\
        \ddr{\theta}{\Pmn(\cos(\theta))}e^{im\phi}
        \\
        \frac{im}{\sin\theta}\Pmn(\cos(\theta))e^{im\phi}
      \end{bmatrix}
    \end{align*}

    On commence par calculer le divergent surfacique (cf annexe \ref{sec:annexe:div_grad_rot}) en sphérique
    \begin{align*}
      \vdivs{}\Umn(\rtp) &= \frac{1}{r\sin(\theta)} \ddr{\theta}{(\sin(\theta)\Umn_\theta)} + \frac{1}{r\sin(\theta)}\ddr{\phi}{(\Umn_\phi)}
      \\
      &=\frac{ime^{im\phi}}{r\sin(\theta)}\left( \ddr{\theta}{\Pmn(\cos(\theta))} - \ddr{\theta}{\Pmn(\cos(\theta))} \right)
      \\
      &= 0
    \end{align*}
    Donc 
    \begin{align*}
      \LD\Umn(\rtp) = 0
    \end{align*}

    Calculons maintenant l'action de \(\LD\) sur \(\Umn^\perp\)
    \begin{align*}
      \vdivs{}\Umn^\perp(\rtp) &= \frac{1}{r\sin(\theta)} \ddr{\theta}{(\sin(\theta)\Umn^\perp_\theta)} + \frac{1}{r\sin(\theta)}\ddr{\phi}{(\Umn^\perp_\phi)}
      \\
      &= \frac{e^{im\phi}}{r\sin(\theta)}
      \left(
        \ddr{\theta}{}\left(\sin(\theta)\ddr{\theta}{\Pmn(\cos(\theta))}\right) - \frac{m^2}{\sin\theta}\Pmn(\cos(\theta))
      \right)
    \end{align*}
    Or d’après \cite[\href{https://dlmf.nist.gov/14.10}{sec.~14.10}]{dlmf_nist_2019} les fonctions de Legendre sont récurrentes
    \begin{align}
      \sin(\theta)\ddr{\theta}{\Pmn(\cos(\theta))} &= (n+m)\PP_{n-1}^m(\cos(\theta)) - n\cos(\theta)\Pmn(\cos(\theta))
    \end{align}

    Donc 
    \begin{align*}
      \LD\Umn^\perp(\rtp) = -\frac{n(n+1)}{r^2}\Umn^\perp(\rtp)
    \end{align*}

    On utilise les résultats de \cite{marceaux_high-order_2000}

    \begin{align*}
      \vgrads{}\vdivs{} \Mmn[z_n]_t(r,\theta,\phi) &= 0
      \\
      \vgrads{}\vdivs{} \Nmn[z_n]_t(r,\theta,\phi) &= -\frac{n(n+1)}{r^2}\Nmn[z_n]_t(r,\theta,\phi)
    \end{align*}

    On définit \(\hat{\mLD}\) l'opérateur matriciel tel que
    \begin{align}
      \LD \vE_t (r_{ext},\theta,\phi)
      &= \frac{1}{2\pi}\sum_{n=-\infty}^\infty\sum_{m=-n}^n \hat{\mLD} \hat{\vE_t}(r_{ext},n,m)
    \end{align}

    Son expression est de ce qui précède
    \begin{equation}
      \label{eq:cylindre:fourier:LD}
      \hat{\mLD}(n,m) = -
      \begin{bmatrix}
        0 & 0
        \\
        0 & \frac{n(n+1)}{r_{ext}^2}
      \end{bmatrix}
    \end{equation}

    On reprend exactement la même méthode pour l'opérateur \(\LR\).
    Par définition de \(\LR\), on a
    \begin{align}
      \LR \vE_t & = \vrots{} \vrots{} \vE_t
    \end{align}

    On utilise les résultats de \cite{marceaux_high-order_2000}

    \begin{align*}
      \vrots{}\vrots{} \Mmn[z_n]_t(r,\theta,\phi) &= \frac{n(n+1)}{r^2}\Mmn[z_n]_t(r,\theta,\phi)
      \\
      \vrots{}\vrots{} \Nmn[z_n]_t(r,\theta,\phi) &= 0
    \end{align*}

    On définit \(\hat{\LR}\) l'opérateur matriciel tel que
    \begin{align}
      \LR \vE_t(r_{ext},\theta,\phi)
      &= \frac{1}{2\pi}\sum_{n=-\infty}^\infty\sum_{m=-n}^n \hat{\LR} \hat{\vE_t}(r_{ext},n,m)
    \end{align}

    \begin{equation}
      \hat{\mLR}(n,m) =
      \begin{bmatrix}
        \frac{n(n+1)}{r_{ext}^2} & 0
        \\
        0 & 0
      \end{bmatrix}
    \end{equation}

  \subsection{Expression de la matrice d'impédance approchée par la CI3}

    Tout comme dans le cas du plan infini, on peut donc définir \(\hat{\mZ}_{IBC}\) l’opérateur matriciel associé à la condition d'impédance.

    \begin{multline}
        \hat{\mZ}_{CI3}(n,m) = \left(\mI + b_1 \frac{\hat{\mLD}(n,m)}{k_0^2} - b_2 \frac{\hat{\mLR}(n,m)}{k_0^2} \right)^{-1}\\
        \left(a_0 \mI + a_1 \frac{\hat{\mLD}(n,m)}{k_0^2} - a_2 \frac{\hat{\mLR}(n,m)}{k_0^2}\right)
    \end{multline}

\section[Choix 1 du calcul des coefficients de la CI3]{Choix des coefficients de la CI3 par moindres carrés sur l'impédance}

  \subsection{Expression de la fonctionnelle}

    \begin{defn}
      On définit \(\mH_{CI3}\) la fonction de \(\NN \times \RR \times \mathcal{M}_2(\CC) \rightarrow \mathcal{M}_{4\times5}(\CC)\) telle que
      \begin{equation*}
        \mH_{CI3}(n,\mZ) = \begin{bmatrix}
        1 & \hat{\mLD}(n)_{11} & -\hat{\mLR}(n)_{11} & -\left(\hat{\mLD}(n){\mZ}\right)_{11} & \left(\hat{\mLR}(n){\mZ}\right)_{11}
        \\
        0 & \hat{\mLD}(n)_{12} & -\hat{\mLR}(n)_{12} & -\left(\hat{\mLD}(n){\mZ}\right)_{12} & \left(\hat{\mLR}(n){\mZ}\right)_{12}
        \\
        0 & \hat{\mLD}(n)_{21} & -\hat{\mLR}(n)_{21} & -\left(\hat{\mLD}(n){\mZ}\right)_{21} & \left(\hat{\mLR}(n){\mZ}\right)_{21}
        \\
        1 & \hat{\mLD}(n)_{22} & -\hat{\mLR}(n)_{22} & -\left(\hat{\mLD}(n){\mZ}\right)_{22} & \left(\hat{\mLR}(n){\mZ}\right)_{22}
        \end{bmatrix}.
      \end{equation*}
      On définit \(b\) la fonction de \(\mathcal{M}_2(\CC) \rightarrow \mathcal{M}_{4\times1}(\CC)\) telle que
      \begin{equation*}
        b(\mZ) = \begin{bmatrix}
        {\mZ}_{11}
        \\
        {\mZ}_{12}
        \\
        {\mZ}_{21}
        \\
        {\mZ}_{22}
        \end{bmatrix}.
      \end{equation*}
    \end{defn}

    \begin{prop}
      Soit \(X = (a_0,a_1,a_2,b_1,b_2)^t\), \(n\) fixé et \(\hat\mZ_{ex}\) l'opérateur d'impédance exact du cylindre, alors
      \begin{equation*}
        \argmin{X\in\CC^5} \norm{\hat\mZ_{CI3}(n,X) - \hat\mZ_{ex}(n)} = \argmin{X\in\CC^5} \norm{\mH_{CI3}(n,\hat\mZ_{ex}(n))X - b(\hat\mZ_{ex}(n))}^2.
      \end{equation*}
    \end{prop}

    \begin{proof}
      C'est la même méthodologie que pour le plan.
      On rappelle que dans la section précédente, on a introduit
      \begin{equation*}
        \hat{\mZ}_{CI3}(n) = \left(\mI + b_1 \hat{\mLD}(n) - b_2 \hat{\mLR}(n) \right)^{-1}\left(a_0 \mI + a_1 {\hat{\mLD}(n)} - a_2 {\hat{\mLR}(n)}\right).
      \end{equation*}
      On pose \(\hat\mZ_D(n,X) = \mI + b_1 \hat{\mLD}(n) - b_2 \hat{\mLR}(n)\) et \(\hat\mZ_N(n,X) = a_0 \mI + a_1 {\hat{\mLD}(n)} - a_2 {\hat{\mLR}(n)}\) donc \(\argmin{X\in\CC^5} \norm{\hat\mZ_{CI3}(n,X) - \hat\mZ_{ex}(n)}\)
      \begin{align*}
      & = \argmin{X\in\CC^5} \norm{\hat\mZ_D(n,X)^{-1}\hat\mZ_N(n,X) - \hat\mZ_{ex}(n) },
      \\
      &= \argmin{X\in\CC^5} \norm{\hat\mZ_D(n,X)^{-1}\left(\hat\mZ_N(n,X) - \hat\mZ_D(n,X)\hat\mZ_{ex}(n)\right) },
      \\
      &= \argmin{X\in\CC^5} \norm{\hat\mZ_N(n,X) - \hat\mZ_D(n)\hat\mZ_{ex}(n,X)}
      \\
      &= \argmin{X\in\CC^5} \norm{\hat\mZ_N(n,X) - \left(b_1 \hat{\mLD}(n) - b_2 \hat{\mLR}(n)\right)\hat\mZ_{ex}(n) - \hat\mZ_{ex}(n) },
      \\
      &= \argmin{X\in\CC^5} \norm{\mH_{CI3}(n,\hat\mZ_{ex}(n))X - b(\hat\mZ_{ex}(n))}.
      \end{align*}
    \end{proof}

    On tronque la série de Mie à \(N_{n}\) termes.
    \begin{defn}
      On définit \(\mA_{CI3}\) la matrice de \(\mathcal{M}_{4N_{n}\times5}(\CC)\) telle que
      \begin{equation*}
        \mA_{CI3} = 
        \begin{bmatrix}
          \mH_{CI3}(n_1,\hat\mZ_{ex}(n_1))
          \\
          \vdots
          \\
          \mH_{CI3}(n_i,\hat\mZ_{ex}(n_i))
          \\
          \vdots
          \\
          \mH_{CI3}(n_{N_n},k_{z},\hat\mZ_{ex}(n_{N_n},k_{z}))
        \end{bmatrix}.
      \end{equation*}
      On définit \(g\) la matrice de \(\mathcal{M}_{4N_{n}\times1}(\CC)\) telle que
      \begin{equation*}
        g = 
        \begin{bmatrix}
          b(\hat\mZ_{ex}(n_1))
          \\
          \vdots
          \\
          b(\hat\mZ_{ex}(n_i))
          \\
          \vdots
          \\
          b(\hat\mZ_{ex}(n_{N_n},k_{z}))
        \end{bmatrix}.
      \end{equation*}
    \end{defn}

    On peut alors calculer les coefficients de la CI3
    \begin{defn}
      On définit la fonctionnelle \(J_Z\)
      \begin{equation*}
        J_Z(X) = \norm{{\mA}_{CI3}X - {g}}^2.
      \end{equation*}
    \end{defn}
    \begin{thm}[Minimisation sans contraintes pour la CI3]

      Les coefficients de la CIOE sont solutions du problème

      Trouver \(X^* \in \CC^5\) tel que
      \begin{equation*}
        X^* = \argmin{X\in \CC^5}  J_Z(X).
      \end{equation*}
    \end{thm}

    \begin{prop}
      \label{prop:sphere:minimisation:minimum_sans_contraintes}
      Si \(\conj{\mA_{CI3}^t}\mA_{CI3}\) est inversible, alors
      \begin{equation*}
        X^* = (\conj{\mA_{CI3}^t}\mA_{CI3})^{-1}\conj{\mA_{CI3}^t}g.
      \end{equation*}
    \end{prop}

    \begin{proof}
      Même démonstration que pour le théorème \ref{prop:plan:minimisation:minimum_sans_contraintes}.
    \end{proof}

    Nous n'avons pas réussi à démontrer que cette matrice était définie pour tout empilement et toute incidence.

    \begin{thm}[Minimisation avec contraintes pour la CI3]

      Soit \(\CSU[3]{CI3}\) le sous-espace de \(\CC^5\) issu de la définition \ref{def:csu:ci3-3}, alors les coefficients de la CIOE respectant les CSU sont solutions du problème

      Trouver \(X^* \in \CC^5\) tel que
      \begin{equation*}
        X^* = \argmin{X\in \CSU[3]{CI3}}  J_Z(X).
      \end{equation*}
    \end{thm}


\section{Calcul des coefficients de la CI3 par moindres carrés sur les coefficients de la série de Fourier}

  Soit \(\mM_j\) et \(\mM_h\) les fonctions introduites à la définition \ref{def:sphere:matrices_MJ-MH} et \(\hat\mR\) la fonction définie à la définition \ref{def:sphere:reflexion:impedance}.

  \begin{defn}%[]
    \label{def:sphere:minimisation:matrices_MR}
    On définit les fonctions \(\hat\mR_{ex}, \hat\mR_{CI3}\) de \(\NN\times \rightarrow \mathcal{M}_2(\CC)\) telles que
    \begin{align*}
      \hat\mR_{ex}(n) &= \hat\mR(n, \hat\mZ_{ex}(n))
      \\
      \hat\mR_{CI3}(n) &= \hat\mR(n, \hat\mZ_{CI3}(n))
    \end{align*}
    où \(\hat\mZ_{ex},\hat\mZ_{CI3}\) sont des fonctions définies à la proposition \ref{prop:sphere:synthese:impedance} et à l'équation \eqref{eq:sphere:hoibc:ci3}.
  \end{defn}

  \subsection{Expression de la fonctionnelle}

    On utilise les fonctions \(\mN_E, \mN_H\) introduites à la définition \ref{def:sphere:matrices_NE-NH}.

    \begin{defn}
      On définit \(\mA_0,\mA_1,\mA_2,\mA_2,\mB_1,\mB_2\) les fonctions de \(\NN \times \mathcal{M}_2(\CC) \rightarrow \ \mathcal{M}_2(\CC)\) telles que        
      \begin{align*}
        \mA_0(n,\mR) &= \mN_E(r_c^+,n,\mR)
        \\
        \mA_1(n,\mR) &= \hat{\mLD}(n)\mN_E(r_c^+,n,\mR)
        \\
        \mA_2(n,\mR) &= -\hat{\mLR}(n)\mN_E(r_c^+,n,\mR)
        \\
        \mB_1(n,\mR) &= \hat{\mLD}(n)\mN_H(r_c^+,n,\mR)
        \\
        \mB_2(n,\mR) &= -\hat{\mLR}(n)\mN_H(r_c^+,n,\mR)            
      \end{align*}

      On définit \(\tilde{\mH}_{CI3}\) la fonction de \(\NN \times \mathcal{M}_2(\CC) \rightarrow \mathcal{M}_{4\times5}(\CC)\) telle que
      \begin{align*}
        & \tilde\mH_{CI3}(n,\mR) =  \\ &
        \begin{bmatrix}
          \mA_0(n,\mR)_{11} & \mA_1(n,\mR)_{11} & \mA_2(n,\mR)_{11} & \mB_1(n,\mR)_{11} & \mB_2(n,\mR)_{11}
          \\
          \mA_0(n,\mR)_{12} & \mA_1(n,\mR)_{12} & \mA_2(n,\mR)_{12} & \mB_1(n,\mR)_{12} & \mB_2(n,\mR)_{12}
          \\
          \mA_0(n,\mR)_{21} & \mA_1(n,\mR)_{21} & \mA_2(n,\mR)_{21} & \mB_1(n,\mR)_{21} & \mB_2(n,\mR)_{21}
          \\
          \mA_0(n,\mR)_{22} & \mA_1(n,\mR)_{22} & \mA_2(n,\mR)_{22} & \mB_1(n,\mR)_{22} & \mB_2(n,\mR)_{22}
        \end{bmatrix}
      \end{align*}

      On définit \(\tilde{b}\) la fonction de \(\NN \times \mathcal{M}_2(\CC) \rightarrow \mathcal{M}_{4\times1}(\CC)\) telle que
      \begin{equation*}
        \tilde{b}(n,\mR) = -
        \begin{bmatrix}
          \mN_H(r_c^+,n,\mR)_{11}
          \\
          \mN_H(r_c^+,n,\mR)_{12}
          \\
          \mN_H(r_c^+,n,\mR)_{21}
          \\
          \mN_H(r_c^+,n,\mR)_{22}
        \end{bmatrix}
      \end{equation*}
    \end{defn}

    \begin{prop}
      Soit \(X = (a_0,a_1,a_2,b_1,b_2)\), \((n)\) fixé et \(\hat\mR_{ex}\) la matrice définie en \ref{def:sphere:minimisation:matrices_MR}, alors
      \begin{align*}
        \argmin{X\in\CC^5} \norm{\hat\mR_{CI3}(n,X) - \hat\mR_{ex}(n)} =
        \argmin{X\in\CC^5} \norm{\tilde{\mH}_{CI3}(n,\hat\mR_{ex}(n))X - \tilde{b}(n,\hat\mR_{ex}(n))}
      \end{align*}
    \end{prop}

    \begin{proof}
      C'est la même méthodologie que pour l'impédance.
      On rappelle de la section précédente
      \begin{align*}
        \hat{\mZ}_{CI3}(n) = \left(\mI + b_1 \hat{\mLD}(n) - b_2 \hat{\mLR}(n) \right)^{-1}
        \left(a_0 \mI + a_1 {\hat{\mLD}(n)} - a_2 {\hat{\mLR}(n)}\right)
      \end{align*}
      On pose \(\hat\mZ_D(n) = \mI + b_1 \hat{\mLD}(n) - b_2 \hat{\mLR}(n)\) et \(\hat\mZ_N(n) = a_0 \mI + a_1 {\hat{\mLD}(n)} - a_2 {\hat{\mLR}(n)}\) donc

      \begin{align*}
        &{\hspace{1em}}~ \argmin{X\in\CC^5} \norm{\hat\mR_{CI3}(n,X) - \hat\mR_{ex}(n)}
        \\
        & = \argmin{X\in\CC^5} \norm{ - \mM_h(r_s^+,n,\hat\mZ_{CI3})^{-1}\mM_j(r_s^+,n,\hat\mZ_{CI3})- \hat\mR_{ex}(n) }
        \\
        & = \argmin{X\in\CC^5} \norm{ - \mM_h(r_s^+,n,\hat\mZ_{CI3})^{-1}\left(\mM_j(r_s^+,n,\hat\mZ_{CI3}) +  \mM_h(r_s^+,n,\hat\mZ_{CI3})\hat\mR_{ex}(n)\right) }      
        \\ 
        & = \argmin{X\in\CC^5} \norm{\mM_j(r_s^+,n,\hat\mZ_{CI3}) +\mM_h(r_s^+,n,\hat\mZ_{CI3})\hat\mR_{ex}(n)}
        \intertext{D'après la définition \ref{def:sphere:matrices_MJ-MH} des fonctions \(\mM_j, \mM_h\),}
        & = \argmin{X\in\CC^5} \left\lVert \left(\mJ_E(r_s^+,n)-\hat\mZ_{CI3}(n)\mJ_H(r_s^+,n)\right) \right.
        \\
        & \qquad \qquad \quad + \left.\left(\mH_E(r_s^+,n)-\hat\mZ_{CI3}(n)\mH_H(r_s^+,n)\right)\hat\mR_{ex}(n) \right\lVert
        \intertext{D'après la définition de \(\hat\mZ_{CI3}\),}        
        & = \argmin{X\in\CC^5} \left\lVert \hat\mZ_D(n)^{-1}\left(\hat\mZ_D(n)\mJ_E(r_s^+,n)-\hat\mZ_N(n)\mJ_H(r_s^+,n)\right) \right.
        \\
        & \qquad \qquad \quad + \left.\hat\mZ_D(n)^{-1}\left(\hat\mZ_D(n)\mH_E(r_s^+,n)-\hat\mZ_N(n)\mH_H(r_s^+,n)\right)\hat\mR_{ex}(n) \right\lVert
        \\
        & = \argmin{X\in\CC^5} \left\lVert \left(\hat\mZ_D(n)\mJ_E(r_s^+,n)-\hat\mZ_N(n)\mJ_H(r_s^+,n)\right) \right.
        \\
        & \qquad \qquad \quad + \left.\left(\hat\mZ_D(n)\mH_E(r_s^+,n)-\hat\mZ_N(n)\mH_H(r_s^+,n)\right)\hat\mR_{ex}(n) \right\lVert
        \intertext{D'après la définition \ref{def:sphere:matrices_NE-NH} des fonctions \(\mN_E, \mN_H\),}        
        & = \argmin{X\in\CC^5} \norm{\hat\mZ_N(n)\mN_E(r_s^+,n,\hat\mR_{ex}(n)) + \hat\mZ_D(n)\mN_H(r_s^+,n,\hat\mR_{ex}(n))}
      \end{align*}
      et on conclut d'après la définition des fonctions \(\hat\mZ_D, \hat\mZ_N\).
    \end{proof}

    On tronque la série de Mie à \(N_{n}\) termes.
    \begin{defn}
      On définit \(\tilde{\mA}_{CI3}\) la matrice de \(\mathcal{M}_{4N_{n}\times5}(\CC)\) telle que
      \begin{equation*}
        \tilde{\mA}_{CI3} = 
        \begin{bmatrix}
          \tilde\mH_{CI3}(n_1,\hat\mR_{ex}(n_1))
          \\
          \vdots
          \\
          \tilde\mH_{CI3}(n_i,\hat\mR_{ex}(n_i))
          \\
          \vdots
          \\
          \tilde\mH_{CI3}(n_{N_n},\hat\mR_{ex}(n_{N_n}))
        \end{bmatrix}
      \end{equation*}
      On définit \(\tilde{g}\) le vecteur colonne \(\CC^{4N_{n}}\) telle que
      \begin{equation*}
        \tilde{g} = 
        \begin{bmatrix}
          \tilde{b}(n_1,\hat\mR_{ex}(n_1))
          \\
          \vdots
          \\
          \tilde{b}(n_i,\hat\mR_{ex}(n_i))
          \\
          \vdots
          \\
          \tilde{b}(n_{N_n},\hat\mR_{ex}(n_{N_n}))
        \end{bmatrix}
      \end{equation*}
    \end{defn}

    On peut alors calculer les coefficients de la CI3

    \begin{defn}
      On définit la fonctionnelle \(J_R\)
      \begin{equation*}
        J_R(X) = \norm{\tilde{\mA}_{CI3}X - \tilde{g}}
      \end{equation*}
    \end{defn}

    \begin{thm}[Minimisation sans contraintes pour la CI3]

      Les coefficients de la CIOE sont solutions du problème

      Trouver \(X^* \in \CC^5\) tel que
      \begin{equation*}
        X^* = \argmin{X\in \CC^5} J_R(X)
      \end{equation*}
    \end{thm}

    \begin{prop}
      Si \(\conj{\tilde{\mA}_{CI3}^t}\tilde{\mA}_{CI3}\) est inversible alors
      \begin{equation*}
        X^* = (\conj{\tilde{\mA}_{CI3}^t}\tilde{\mA}_{CI3})^{-1}\conj{\tilde{\mA}_{CI3}^t}\tilde{g}
      \end{equation*}
    \end{prop}
    \begin{proof}
      Même méthode que pour la proposition \ref{prop:sphere:minimisation:minimum_sans_contraintes} sur l'impédance.
    \end{proof}

    Nous n'avons pas réussi à démontrer que cette matrice était définie pour tout empilement et toute incidence.

    \begin{thm}[Minimisation avec contraintes pour la CI3]

      Soit \(\CSU[3]{CI3}\) le sous-espace de \(\CC^5\) issu de la définition \ref{def:csu:ci3-3}.
      Alors les coefficients de la CIOE respectant les CSU sont solutions du problème

      Trouver \(X^* \in \CC^5\) tel que
      \begin{equation*}
        X^* = \argmin{X\in \CSU[3]{CI3}} J_R(X)
      \end{equation*}
    \end{thm}

\section{Résultats numériques}

  La figure \ref{fig:imp_fourier:sphere:hoppe:62:hoibc:mode_2} trace les valeurs des termes diagonaux de la matrice \(\hat\mZ\) quand les coefficients sont calculés en minimisant \(J_Z\) sans contraintes.
  \begin{figure}[!hbt]
    \centering
    \input{tikz/plot/HOPPE_62/Z_HOPPE_62_sphere_hoibc_mode_2.tikz}
    \caption[Partie imaginaire de l'opérateur de Calderón, compararé avec les approximations CI0, CI3, CI6 (calculées avec Jz) pour une couche sphérique de matériau de Hoppe \& Rahmat-Samii.]{Minimisation de \(J_Z\): Partie imaginaire des termes diagonaux des matrices d'impédance pour l'empilement \(\eps = 6\), \(\mu = 1\), \(d=0.0225\)m, \(f=1\)GHz, \(r_0=0.03\)m en fonction de \(k_t = n / (r_0+d)\).}
    \label{fig:imp_fourier:sphere:hoppe:62:hoibc:mode_2}
  \end{figure}
  On remarque alors que la CI3 semble être une bonne approximation. 

  \begin{table}[!hbt]
    \centering
    % On fait deux tables de même hauteur
    \begin{minipage}[t]{0.49\textwidth}
      \vspace{0pt}
      \centering
      \begin{coefftable}{\hyperlink{ci0}{CI0}}
        \input{csv/HOPPE_62/HOPPE_62.IBC_ibc0_SUC_F_MODE_2_TYPE_S_+3.000E-02m.coeff.txt}
      \end{coefftable}
    \end{minipage}
    \begin{minipage}[t]{0.49\textwidth}
      \vspace{0pt}
      \centering
      \begin{coefftable}{\hyperlink{ci3}{CI3}}
        \input{csv/HOPPE_62/HOPPE_62.IBC_ibc3_SUC_F_MODE_2_TYPE_S_+3.000E-02m.coeff.txt}
      \end{coefftable}
    \end{minipage}
    \caption{Coefficients associés à la figure \ref{fig:imp_fourier:sphere:hoppe:62:hoibc:mode_2}}
    \label{tab:imp_fourier:sphere:hoppe:62:hoibc:mode_2}
  \end{table}

  % La figure \ref{fig:imp_fourier:sphere:hoppe:62:hoibc:mode_1} minimise \(J_R\) sans contraintes.
  % \begin{figure}[!hbt]
  %   \centering
  %   \input{tikz/plot/HOPPE_62/Z_HOPPE_62_sphere_hoibc_mode_1.tikz}
  %   \caption[CIOE sur empilement de Hoppe Rahmat-Samii p.~62]{Minimisation de \(J_R\): Partie imaginaire des termes diagonaux des matrices d'impédance pour l'empilement \(\eps = 6\), \(\mu = 1\), \(d=0.0225\text{m}\), \(f=1\text{GHz}\), \(r_0=0.03\text{m}\) en fonction de \(k_t = n / (r_0+d)\).}
  %   \label{fig:imp_fourier:sphere:hoppe:62:hoibc:mode_1}
  % \end{figure}
  % Nous ne savons pas expliquer pourquoi la valeur calculée pour la condition de Leontovich est si différente.

  % \begin{table}[!hbt]
  %   \centering
  %   % On fait deux tables de même hauteur
  %   \begin{minipage}[t]{0.49\textwidth}
  %     \vspace{0pt}
  %     \centering
  %     \begin{coefftable}{\hyperlink{ci0}{CI0}}
  %       \input{csv/HOPPE_62/HOPPE_62.IBC_ibc0_SUC_F_MODE_1_TYPE_S_+3.000E-02m.coeff.txt}
  %     \end{coefftable}
  %   \end{minipage}
  %   \begin{minipage}[t]{0.49\textwidth}
  %     \vspace{0pt}
  %     \centering
  %     \begin{coefftable}{\hyperlink{ci3}{CI3}}
  %       \input{csv/HOPPE_62/HOPPE_62.IBC_ibc3_SUC_F_MODE_1_TYPE_S_+3.000E-02m.coeff.txt}
  %     \end{coefftable}
  %   \end{minipage}
  %   \caption{Coefficients associés à la figure \ref{fig:imp_fourier:sphere:hoppe:62:hoibc:mode_1}}
  %   \label{tab:imp_fourier:sphere:hoppe:62:hoibc:mode_1}
  % \end{table}



\sectionstar{Conclusion}
Nous avons montré comment calculer les coefficients dans le cas d'un objet sphérique en minimisant au sens des moindres carrés la différence entre les coefficients de Mie exacts et approchés. 
