\chapter{Calcul des coefficients pour un cylindre infini}
\label{sec:cylindre}
\minitoc
\newpage
\sectionstar{Introduction}
L'introduction de courbure dans les CIOE est d'un atout majeur, car les objets réels sont rarement localement plat, mais plutôt de type sphère-cone. Nous allons présentons dans cette partie l'introduction d'une courbure dans une seule direction.

\section{Cas d'un objet cylindrique}

  % On rappelle les formules des opérateurs \(\vdiv, \vrot\) en coordonnée cylindrique \((r,\theta,z)\).
  % \begin{align}
  %   \vrot \vect{V} &= \left(\frac{1}{r}\ddr{\theta}{V_z} - \ddr{z}{V_\theta}\right)\vect{e_r} +
  %   \left(\ddr{z}{V_r} - \ddr{r}{V_z}\right)\vect{e_\theta} +
  %   \frac{1}{r}\left(\ddr{r}{(rV_\theta)}-\ddr{\theta}{V_r}\right)\vect{e_z}
  %   \\
  %   \vdiv \vect{V} &= \frac{1}{r}\ddr{r}{(rV_r)}+\frac{1}{r}\ddr{\theta}{V_\theta}+\ddr{z}{V_z}
  %   \\
  %   \vgrad f &= \ddr{r}{f}\vect{e_r}
  %   +\frac{1}{r}\ddr{\theta}{f}\vect{e_\theta} + \ddr{z}{f}\vect{e_z}
  % \end{align}

  \begin{figure}[!hbt]
    \centering
    \tikzsetnextfilename{cylindre_1_couche}
    \begin{tikzpicture}
      \coordinate (mat) at (0,-1.5);
\coordinate (vide) at (0,-2);
\coordinate (c) at (0,0);

\fill [lightgray] (c) circle (2);
\fill [white] (c) circle (1.5);
\fill [pattern=north east lines] (c) circle (1.5);

\draw (c) circle (2);
\draw (c) circle (1.5);


\coordinate (n) at (0,2);

%\draw (vide) node [below] {$\eps_0,\mu_0$};
\draw (mat) node [below] {$\peps,\pmu$};

% Axess
\draw [->] (n) -- ++(0,1) node [at end, right] {$\v{\mr}$};
\draw [->] (n) -- ++(1,0) node [at end, right] {$\v{\mt}$};

\draw (n) ++(0.2,0.2) circle(0.1cm) node [above=0.1cm] {$\v{\mz}$};
\draw (n) ++(0.2,0.2) +(135:0.1cm) -- +(315:0.1cm);
\draw (n) ++(0.2,0.2) +(45:0.1cm) -- +(225:0.1cm);

%\draw [->>,thick] (lt) ++ (1,1) -- (mt) ;


    \end{tikzpicture}
  \end{figure}

  On exprime les équations de Maxwell dans le matériau dans la base cylindrique et sans pertes de généralité, on peut réaliser une transformée de Fourier en \(z\) par invariance en translation et en \(\theta\) par invariance en rotation.
  Cependant, le multiplicateur de Fourier associé à la coordonnée \(\theta\) doit être un entier pour assurer la périodicité. On le note \(n\).

  \begin{equation}
    \vE(r,\theta,z) = \frac{1}{2\pi}\sum_{n=-\infty}^{\infty}\int_{\RR} e^{i(n \theta + k_z z )}\hat{\vE} (r,n,k_z) \dd{k_z}
  \end{equation}

  \begin{prop}
    Soit
    \begin{equation}
      k_3 = \sqrt{\w^2\eps\mu - k_z^2}
    \end{equation}
    et \(J_n(z)\) et \(H_n^{(2)}(z)\) des solutions de l'équation de Bessel d'ordre \(n\).

    Alors \(\exists (c_i(n,k_z))_{1<=i<=4} \in \CC(\NN\times\RR)^4\) tels que
    \begin{subequations}
      \begin{align}
        \hat{E_z}(r,n,k_z) &= c_1(n,k_z) J_n\left(k_3r\right) + c_2(n,k_z) H_n^{(2)}\left(k_3r\right)
        \\
        \hat{H_z}(r,n,k_z) &= c_3(n,k_z) J_n\left(k_3r\right) + c_4(n,k_z) H_n^{(2)}\left(k_3r\right)
      \end{align}
    \end{subequations}
  \end{prop}

  \begin{proof}

    On peut simplifier les opérateurs différentiels:

    \begin{align}
      \vrot \hat \vE(r,n,k_z) &= i\left(\frac{n}{r}\hat{E_z} - k_z\hat{E_\theta}\right)\vect{e_r} +
      \left(ik_z\hat{E_r} - \ddr{r}{\hat{E_z}}\right)\vect{e_\theta} +
      \frac{1}{r}\left(\ddr{r}{(r\hat{E_\theta})}-in\hat{E_r}\right)\vect{e_z}
      \\
      &=-i\w\mu \hat \vH(r,n,k_z)
    \end{align}

    On remarque que la méthode utilisée pour le plan aboutie à une équation différentielle à coefficients non constants de type \(r\ddr{r}{\vect{X}}(r,n,k_z) = \mat{M}(r,n,k_z)\vect{X}(r,n,k_z)\).
    On ne peut pas exprimer la solution avec les valeurs et vecteurs propres de la matrice.
    %Nous allons donc trouver une équation de Bessel en développant le système de Maxwell.

    Comme l'on cherche \(\hat \vE_t, \hat \vH_t\), on remarque que les 2\ieme composantes des équations de Maxwell permettent de déduire \(\hat\vE_t, \hat\vH_t\) de \( \hat E_z, \hat H_z\).

    On couple les 2 équations du système de Maxwell pour aboutir à une équation sur \(\hat \vE\) seul:

    % \begin{align}
    %   \vrot \vrot \hat \vE &= \w^2\eps\mu \hat \vE
    %   \\
    %   \vdiv \hat \vE &= 0
    % \end{align}

    % \begin{multline}
    %   \vrot \vrot \hat \vE = \dots\\
    %   i\left(\frac{n}{r^2}\left(\ddr{r}{(r\hat{E_\theta})} - in\hat{E_r}\right) - k_z\left(ik_z\hat{E_r} - \ddr{r}{\hat{E_z}}\right)\right)  \vect{e_r} \dots\\
    %   + \left(-k_z\left(\frac{n}{r}\hat{E_z} - k_z\hat{E_\theta}\right) -\ddr{r}{}\left(\frac{1}{r}\left(\ddr{r}{(r\hat{E_\theta})}-in\hat{E_r}\right)\right)\right)  \vect{e_\theta} \dots\\
    %   + \frac{1}{r}\left(\ddr{r}{} \left(r\left(ik_z\hat{E_r} - \ddr{r}{\hat{E_z}}\right)\right) + n \left(\frac{n}{r}\hat{E_z} - k_z\hat{E_\theta}\right)\right) \vect{e_z}
    % \end{multline}

    % On aboutit au système suivant
    \begin{equation}
      \left\lbrace
      \begin{array}{ccc}
        -\left(\w^2\eps\mu -\frac{n^2}{r^2}  - k_z^2\right)\hat{E_r}  +i\frac{n}{r^2}\ddr{r}{(r\hat{E_\theta})}  +k_z\ddr{r}{\hat{E_z}} & = & 0\\
        in\ddr{r}{}\left(\frac{\hat{E_r}}{r}\right) -\left(\w^2\eps\mu - k_z^2\right)\hat{E_\theta} + \ddr{r}{}\left(\frac{1}{r}\ddr{r}{(r\hat{E_\theta})}\right)  - n\frac{k_z}{r}\hat{E_z} & = & 0\\
        i\frac{k_z}{r}\ddr{r}{(r\hat{E_r})}  - n\frac{k_z}{r}\hat{E_\theta}  -\left(\w^2\eps\mu - \frac{n^2}{r^2} \right)\hat{E_z} - \frac{1}{r}\ddr{r}{}\left(r\ddr{r}{\hat{E_z}}\right) & = & 0
      \end{array}
      \right.
    \end{equation}

    De la troisième  équation, on trouve pour \(r\not=0\)
    \begin{equation}
    r^2 \ddr[2]{r}{\hat{E_z}} + r\ddr{r}{\hat{E_z}} + \left(r^2\w^2\eps\mu - n^2\right)\hat{E_z} =ik_zr\ddr{r}{(r\hat{E_r})} -  nk_zr\hat{E_\theta}
    \end{equation}

    Or comme \(\vdiv \hat \vE = 0\), on a
    \begin{align}
      \vdiv\hat \vE &= \frac{1}{r}\ddr{r}{(r\hat{E_r})} + \frac{in}{r}\hat{E_\theta} + ik_z\hat{E_z}
      \\
      k_z^2r^2 \hat{E_z} &= ik_zr\ddr{r}{(r\hat{E_r})} - nk_zr\hat{E_\theta}
    \end{align}

    On obtient donc sur la composante \(\hat{E_z}\):
    \begin{equation}
      r^2 \ddr[2]{r}{\hat{E_z}} + r\ddr{r}{\hat{E_z}} + \left(r^2\left(\w^2\eps\mu - k_z^2\right) - n^2\right)\hat{E_z} = 0
    \end{equation}

    C'est une équation de Bessel (cf \cite[eq (6.80)]{bowman_introduction_1958}), dont des solutions générales sont: soient \((c_1,c_2) \in \CC^2\):
     \begin{equation}
      \hat{E_z}(r,n,k_z) = c_1 J_n\left(k_3r\right) + c_2 H_n^{(2)}\left(k_3r\right)
    \end{equation}
    où \(J_n\) est la fonction de Bessel du premier type, \(H_n^{(2)}\) la fonction de Hankel de deuxième type.
    On sait que l'on peut prendre n'importe quel couple de fonctions de Bessel (cf \eqref{eq:annex:bessel:equiv_bessel}), on choisit ce dernier car les \(J_n\) sont régulières et les \(H_n\) évoluent en \(\frac{1}{\sqrt{r}}\) à l'infini, donc ce choix est adapté à une décomposition en une onde incidente partout définie et une onde réfléchie décroissante à l'infini.

    De plus, d'après \cite[p.~358]{abramowitz_handbook_1964}, on sait qu'une fonction de Bessel d'ordre \(n\) est linéairement dépendante de celle d'ordre \(-n\).
    On peut donc se restreindre à \(n\) entier naturel

    On trouve exactement le même résultat pour \(\hat{H_z}\): soient \((c_3,c_4) \in \CC^2\)
    \begin{equation}
      \hat{H_z}(r,n,k_z) = c_3 J_n\left(k_3r\right) + c_4 H_n^{(2)}\left(k_3r\right)
    \end{equation}
  \end{proof}


  \begin{defn}
    On définit les matrices \(\mJ_{E}(r,n,k_z),\mH_{E}(r,n,k_z),\mJ_{H}(r,n,k_z),\mH_{H}(r,n,k_z)\)
    \begin{align}
      \mJ_{E}(r,n,k_z) &=
      \begin{bmatrix}
        -\frac{nk_z}{rk_3^2}J_n(k_3r) & \frac{ik\eta}{k_3}J_n'(k_3r)
        \\
        J_n(k_3r) & 0
      \end{bmatrix}
      \\
      \mH_{E}(r,n,k_z) &=
      \begin{bmatrix}
        -\frac{nk_z}{rk_3^2}H_n^{(2)}(k_3r) & \frac{ik\eta}{k_3}H_n^{(2)}{}'(k_3r)
        \\
        H_n^{(2)}(k_3r) & 0
      \end{bmatrix}
      \\
      \mJ_{H}(r,n,k_z) &=
      \begin{bmatrix}
        0 & -J_n(k_3r)
        \\
        -\frac{ik}{\eta k_3}J_n'(k_3r) & -\frac{nk_z}{rk_3^2}J_n(k_3r)
      \end{bmatrix}
      \\
      \mH_{H}(r,n,k_z) &=
      \begin{bmatrix}
        0 & -H_n^{(2)}(k_3r)
        \\
        -\frac{ik}{\eta k_3}H_n^{(2)}{}'(k_3r) & -\frac{nk_z}{rk_3^2}H_n^{(2)}(k_3r)
      \end{bmatrix}
    \end{align}
  \end{defn}

  \begin{prop}
    Alors les champs tangentiels s'écrivent
    \begin{subequations}
      \begin{align}
        \hat \vE_t(r,n,k_z) &= \mJ_{E}(r,n,k_z)
        \begin{bmatrix}
          c_1 \\
          c_3
        \end{bmatrix}
        +
        \mH_{E}(r,n,k_z)
        \begin{bmatrix}
          c_2 \\
          c_4
        \end{bmatrix}
        \label{eq:imp_fourier:cylindre:Et}\\
        \vect{e_r}\times\hat \vH_t(r,n,k_z) &=
        \mJ_{H}(r,n,k_z)
        \begin{bmatrix}
          c_1 \\
          c_3
        \end{bmatrix}
        +
        \mH_{H}(r,n,k_z)
        \begin{bmatrix}
          c_2 \\
          c_4
        \end{bmatrix}
        \label{eq:imp_fourier:cylindre:Ht}
      \end{align}
    \end{subequations}
  \end{prop}


  \begin{proof}
    À partir des équations de Maxwell restantes, on peut déterminer \(\hat{E_r},\hat{E_\theta},\hat{H_r},\hat{H_\theta}\).
    \begin{equation}
      \left\lbrace
      \begin{matrix}
        -ik_z\hat{E_\theta} + i\w\mu \hat{H_r} = -\frac{in}{r}\hat{E_z}
        \\
        ik_z\hat{E_r} + i\w\mu \hat{H_\theta} = \ddr{r}{\hat{E_z}}
        \\
        i\w\eps \hat{E_r} + ik_z \hat{H_\theta} = \frac{in}{r}\hat{H_z}
        \\
        i\w\eps \hat{E_\theta} - ik_z \hat{H_r} = -\ddr{r}{\hat{H_z}}
      \end{matrix}
      \right.
    \end{equation}

    Cela revient à résoudre \(\vect{Y} = \mat{M}\vect{X}\) où la matrice \(\mat{M}\) et les vecteurs \(\vect{X}, \vect{Y}\) sont définis tels que
    \begin{equation}
      \mat{M} =
      \begin{bmatrix}
      0 & -ik_z & i\w\mu & 0
      \\
      ik_z & 0 & 0 & i\w\mu
      \\
      i\w\eps & 0 & 0 & ik_z
      \\
      0 & i\w\eps & -ik_z & 0
      \end{bmatrix}
      \,
      \vect{X} =
      \begin{bmatrix}
        \hat{E_r}\\
        \hat{E_\theta}\\
        \hat{H_r}\\
        \hat{H_\theta}
      \end{bmatrix}
      \,
      \vect{Y} =
      \begin{bmatrix}
        -\frac{in}{r}\hat{E_z}\\
        \ddr{r}{\hat{E_z}}\\
        \frac{in}{r}\hat{H_z}\\
        -\ddr{r}{\hat{H_z}}
      \end{bmatrix}
    \end{equation}

    On remarque que \(\mM\mM = \left(k_z^2 - \omega^2\eps\mu\right)\mI\) et donc que \(\det(\mat{M}) = \left(ik_3\right)^2\).

    On suppose ce dernier non nul, on peut déduire \(\vect{X}\):

    \begin{equation}
      \begin{bmatrix}
        \hat{E_r}\\
        \hat{E_\theta}\\
        \hat{H_r}\\
        \hat{H_\theta}
      \end{bmatrix} =
      \frac{1}{-k_3^2}
      \begin{bmatrix}
      0 & -ik_z & i\w\mu & 0
      \\
      ik_z & 0 & 0 & i\w\mu
      \\
      i\w\eps & 0 & 0 & ik_z
      \\
      0 & i\w\eps & -ik_z & 0
      \end{bmatrix}
      \begin{bmatrix}
        -\frac{in}{r}\hat{E_z}\\
        \ddr{r}{\hat{E_z}}\\
        \frac{in}{r}\hat{H_z}\\
        -\ddr{r}{\hat{H_z}}
      \end{bmatrix}
    \end{equation}

    On extrait alors \(\hat{E_\theta}, \hat{H_\theta}\) pour obtenir les champs tangentielles à \(\vect{e_r}\) en tout point, sachant déjà \(\hat{E_z}, \hat{H_z}\).

    \begin{align}
      \hat{E_r} & = \frac{1}{k_3^2}\left(ik_z\ddr{r}{\hat{E_z}}+\frac{k\eta n}{r}\hat{H_z}\right)
      \\
      \hat{E_\theta} &= -\frac{1}{k_3^2}\left(\frac{nk_z}{r}\hat{E_z} - i\w\mu\ddr{r}{\hat{H_z}}\right)
      \\
      \hat{E_z} &= c_1 J_n(k_3 r) + c_2 H_n^{(2)}(k_3 r)
      \\
      -\hat{H_z} &= -c_3 J_n(k_3 r) - c_4 H_n^{(2)}(k_3 r)
      \\
      \hat{H_\theta} &= -\frac{1}{k_3^2}\left(i\w\eps\ddr{r}{\hat{E_z}} + \frac{nk_z}{r}\hat{H_z}\right)
    \end{align}

    On dérive les fonctions de Bessel:

     \begin{align}
      \hat{E_\theta} &= -\frac{nk_z}{rk_3^2}\left(c_1J_n(k_3r) + c_2 H_n^{(2)}(k_3r)\right) + \frac{ik\eta}{k_3}\left(c_3J_n'(k_3r) + c_4 H_n^{(2)}{}'(k_3r)\right)
      \\
      \hat{E_z} &= c_1 J_n(k_3 r) + c_2 H_n^{(2)}(k_3 r)
      \\
      -\hat{H_z} &= -c_3 J_n(k_3 r) - c_4 H_n^{(2)}(k_3 r)
      \\
      \hat{H_\theta} &= -\frac{ik}{\eta k_3}\left(c_1J_n'(k_3r) + c_2 H_n^{(2)}{}'(k_3r)\right) - \frac{nk_z}{rk_3^2}\left(c_3J_n(k_3r) + c_4 H_n^{(2)}(k_3r)\right)
    \end{align}

    Et on obtient


    \begin{subequations}
      \label{eq:imp_fourier:cylindre:champs}
      \begin{align}
        \label{eq:imp_fourier:cylindre:champs:E}
        \hat \vE_t(r,n,k_z) &= \mJ_{E}(r)
        \begin{bmatrix}
          c_1 \\
          c_3
        \end{bmatrix}
        +
        \mH_{E}(r)
        \begin{bmatrix}
          c_2 \\
          c_4
        \end{bmatrix}
        \\
        \label{eq:imp_fourier:cylindre:champs:H}
        \vect{e_r}\times\hat \vH_t(r,n,k_z) &=
        \mJ_{H}(r)
        \begin{bmatrix}
          c_1 \\
          c_3
        \end{bmatrix}
        +
        \mH_{H}(r)
        \begin{bmatrix}
          c_2 \\
          c_4
        \end{bmatrix}
      \end{align}
    \end{subequations}

  \end{proof}

  %%%%%%%%%%%%%%%%%%%%%%%%%%%%%%%%%%%%%%%%%%%%%%%%%%%%%%%%%%%%%%%%%%%%%%%%%%%%%%%%%%%%%%%%%%%%%%%%%%%%%%%%
  %%%%%%%%%%%%%%%%%%%%%%%%%%%%%%%%%%%%%%%%%%%%%%%%%%%%%%%%%%%%%%%%%%%%%%%%%%%%%%%%%%%%%%%%%%%%%%%%%%%%%%%%
  %%%%%%%%%%%%%%%%%%%%%%%%%%%%%%%%%%%%%%%%%%%%%%%%%%%%%%%%%%%%%%%%%%%%%%%%%%%%%%%%%%%%%%%%%%%%%%%%%%%%%%%%


  \subsection{Opérateur d'impédance pour une couche}

    Soit \(r_1 = r_0 + d\)
    \begin{defn}
      On définit le symbole \(\hat \mZ(n,k_z)\) de l'opérateur d'impédance la matrice telle que
      \begin{equation}
        \hat \vE_t(r_1,n,k_z) = \hat \mZ(n,k_z) \left(\vect{e_r}\pvect \hat \vH_t(r_1,n,k_z)\right)
      \end{equation}
    \end{defn}

    \begin{thm}
      Si on suppose que les fonctions de Bessel et leurs dérivées ne s’annulent pas en \(k_3r_0\) et que
      la matrice \(\mH_{H}(r_1) - \mJ_{H}(r_1)\mJ_{E}(r_0)^{-1}\mH_{E}(r_0)\) est inversible

      Alors le symbole \(\hat \mZ(n,k_z)\) de l'opérateur d'impédance est
      \begin{multline}
        \hat \mZ(n,k_z) =
        \left(\mH_{E}(r_1)\mH_{E}(r_0)^{-1} - \mJ_{E}(r_1)\mJ_{E}(r_0)^{-1}\right)\\
        \left(\mH_{H}(r_1)\mH_{E}(r_0)^{-1} - \mJ_{H}(r_1)\mJ_{E}(r_0)^{-1}\right)^{-1}
      \end{multline}
    \end{thm}

    \begin{proof}

      On injecte la relation \(\vE_t(r_0,\theta,z) = 0\) équivalente à \(\hat \vE(r_0,n,k_z) = 0\) dans \eqref{eq:imp_fourier:cylindre:Et}.
      \begin{equation}
        \mJ_{E}(r_0)
        \begin{bmatrix}
          c_1 \\
          c_3
        \end{bmatrix}
        =-\mH_{E}(r_0)
        \begin{bmatrix}
          c_2 \\
          c_4
        \end{bmatrix}
      \end{equation}

      Or par définition des matrices,
      \begin{align}
        \det(\mJ_E(r_0)) &= -\frac{ik\eta}{k_3}J_n(k_3r_0)J_n'(k_3r_0)
        \\
        \det(\mH_E(r_0)) &= -\frac{ik\eta}{k_3}H_n^{(2)}(k_3r_0)H_n^{(2)}{}'(k_3r_0)
      \end{align}

      D’après \cite[p.~370]{abramowitz_handbook_1964}, les zéros des fonctions de Bessel d'ordre réel \(>-1\) sont tous réels.
      Donc à condition d'avoir \(k_3\) complexe, comme l'ordre est entier et que l'on se restreint au entiers naturels, ces matrices sont inversibles\footnote{Là encore, il faut étudier le cas des matériaux sans pertes où \(k_3\) est réel pour \(k_z < w\sqrt{\mu\eps}\)}.

      À condition de l'inversibilité de ces deux matrices, on peut donc exprimer les composantes tangentielles
      \begin{align}
        \hat \vE_t(r_1,n,k_z) &=
        \left(\mH_{E}(r_1) - \mJ_{E}(r_1)\mJ_{E}(r_0)^{-1}\mH_{E}(r_0)\right)
        \begin{bmatrix}
          c_2 \\
          c_4
        \end{bmatrix}
        \\
        \vect{e_r}\pvect \hat \vH_t(r_1,n,k_z) &=
        \left(\mH_{H}(r_1) - \mJ_{H}(r_1)\mJ_{E}(r_0)^{-1}\mH_{E}(r_0) \right)
        \begin{bmatrix}
          c_2 \\
          c_4
        \end{bmatrix}
      \end{align}

      Et à condition que \(\mH_{H}(r_1) - \mJ_{H}(r_1)\mJ_{E}(r_0)^{-1}\mH_{E}(r_0)\) soit inversible, le symbole de l'opérateur d'impédance est:
      \begin{TODO}
        Inversibilité de \(\mH_{H}(r_1) - \mJ_{H}(r_1)\mJ_{E}(r_0)^{-1}\mH_{E}(r_0)\)
      \end{TODO}
      \begin{align}
        \hat \mZ &=
        \left(\mH_{E}(r_1) - \mJ_{E}(r_1)\mJ_{E}(r_0)^{-1}\mH_{E}(r_0)\right)
        \left(\mH_{H}(r_1) - \mJ_{H}(r_1)\mJ_{E}(r_0)^{-1}\mH_{E}(r_0)\right)^{-1}
        \\
        &=
        \left(\mH_{E}(r_1)\mH_{E}(r_0)^{-1} - \mJ_{E}(r_1)\mJ_{E}(r_0)^{-1}\right)
        \left(\mH_{H}(r_1)\mH_{E}(r_0)^{-1} - \mJ_{H}(r_1)\mJ_{E}(r_0)^{-1}\right)^{-1}
      \end{align}

      Contrairement au plan, les matrices ne commutent pas et on ne peut pas simplifier le résultat.

    \end{proof}

  %%%%%%%%%%%%%%%%%%%%%%%%%%%%%%%%%%%%%%%%%%%%%%%%%%%%%%%%%%%%%%%%%%%%%%%%%%%%%%%%%%%%%%%%%%%%%%%%%%%%%%%%
  %%%%%%%%%%%%%%%%%%%%%%%%%%%%%%%%%%%%%%%%%%%%%%%%%%%%%%%%%%%%%%%%%%%%%%%%%%%%%%%%%%%%%%%%%%%%%%%%%%%%%%%%
  %%%%%%%%%%%%%%%%%%%%%%%%%%%%%%%%%%%%%%%%%%%%%%%%%%%%%%%%%%%%%%%%%%%%%%%%%%%%%%%%%%%%%%%%%%%%%%%%%%%%%%%%


  \subsection{Opérateur d'impédance pour plusieurs couches}

    \begin{figure}[!hbt]
      \centering
      \tikzsetnextfilename{cylindre_n_couches}
      \begin{tikzpicture}
        \tikzmath{
    \a = 83;
    \b = 97;
    \d = 0.5;
    \ri = 30;
    \re = \ri;
}

% Le conducteur
\tikzmath{
    \ri = \re;
    \re = \ri + 0.5*\d;
    \xa = cos(\a)*\re;
    \ya = sin(\a)*\re;
    \xb = cos(\b)*\ri;
    \yb = sin(\b)*\ri;
}

\coordinate (a) at (\xa,\ya);
\coordinate (b) at (\xb,\yb);

\fill [pattern=north east lines] (a) arc (\a:\b:\re) -- (b) arc (\b:\a:\ri) -- cycle;
\draw (a) arc (\a:\b:\re);
\draw (a) node [right] {$r_0$};

% Le repère
\coordinate (n) at ($(a)+(0.5,-1)$);
%
%
%\draw [->] (n) -- ++(0,1) node [at end, right] {$\v{\pr}$};
%\draw [->] (n) -- ++(1,0) node [at end, right] {$\v{\pt}$};
%
\draw (n) ++(0.2,0.2) circle(0.1cm) node [above=0.1cm] {$\vect{e_z}$};
\draw (n) ++(0.2,0.2) +(135:0.1cm) -- +(315:0.1cm);
\draw (n) ++(0.2,0.2) +(45:0.1cm) -- +(225:0.1cm);

% 1 ere couche

\tikzmath{
    \ri = \re;
    \re = \ri + \d;
    \xa = cos(\a)*\re;
    \ya = sin(\a)*\re;
    \xb = cos(\b)*\ri;
    \yb = sin(\b)*\ri;
    \xc = cos(0.5*(\b+\a))*(\ri+0.5*\d);
    \yc = sin(0.5*(\b+\a))*(\ri+0.5*\d);
}

\coordinate (a) at (\xa,\ya);
\coordinate (b) at (\xb,\yb);
\coordinate (c) at (\xc,\yc);

\fill [lightgray] (a) arc (\a:\b:\re) -- (b) arc (\b:\a:\ri) -- cycle;
\draw (a) arc (\a:\b:\re);
\draw (c) node {$\eps_1,\mu_1,d_1$};


% Des couches

\tikzmath{
    \ri = \re;
    \re = \ri + 2*\d;
    \xa = cos(\a)*\re;
    \ya = sin(\a)*\re;
    \xb = cos(\b)*\ri;
    \yb = sin(\b)*\ri;
    \xc = cos(0.5*(\b+\a))*(\ri+0.5*\d);
    \yc = sin(0.5*(\b+\a))*(\ri+0.5*\d);
}

\coordinate (a) at (\xa,\ya);
\coordinate (b) at (\xb,\yb);
\coordinate (c) at (\xc,\yc);

\fill [lightgray]    (a) arc (\a:\b:\re) -- (b) arc (\b:\a:\ri) -- cycle;
\fill [pattern=dots] (a) arc (\a:\b:\re) -- (b) arc (\b:\a:\ri) -- cycle;
\draw (a) arc (\a:\b:\re);

% n eme couche

\tikzmath{
    \ri = \re;
    \re = \ri + \d;
    \xa = cos(\a)*\re;
    \ya = sin(\a)*\re;
    \xb = cos(\b)*\ri;
    \yb = sin(\b)*\ri;
    \xc = cos(0.5*(\b+\a))*(\ri+0.5*\d);
    \yc = sin(0.5*(\b+\a))*(\ri+0.5*\d);
}

\coordinate (a) at (\xa,\ya);
\coordinate (b) at (\xb,\yb);
\coordinate (c) at (\xc,\yc);

\fill [lightgray] (a) arc (\a:\b:\re) -- (b) arc (\b:\a:\ri) -- cycle;
\draw (a) arc (\a:\b:\re);
\draw (c) node {$\eps_{Nc},\mu_{Nc},d_{Nc}$};

% Le vide
\tikzmath{
    \xc = cos(0.5*(\b+\a))*(\re);
    \yc = sin(0.5*(\b+\a))*(\re);
}

\draw (\xc,\yc) node [above] {vide};


      \end{tikzpicture}
    \end{figure}

    Soit \(r_m\) le rayon de la couche \(m\), \(r_m = r_0 +\sum_{i=1}^{m} d_{i}\).

    \begin{defn}
      On définit pour chaque interface, le symbole \(\hat \mZ_m\) tel que
      \begin{equation}
        \hat \vE_t(r_m,n,k_z) = \hat \mZ_m(n,k_z) \left(\vect{e_r} \pvect \hat \vH_t(r_m,n,k_z)\right)
      \end{equation}
    \end{defn}

    Pour chaque couche caractérisée par \((\eps_m,\mu_m,d_m)\), définissons
    \begin{subequations}
      \begin{align}
        k_{3m} &= \sqrt{w^2\eps_m\mu_m - k_z^2}
        \\
        \mJ_{Em}(r) &=
          \begin{bmatrix}
            -\frac{nk_z}{rk_{3m}^2}J_n(k_{3m}r) & \frac{i\w\mu_m}{k_{3m}}J_n'(k_{3m}r)
            \\
            J_n(k_{3m}r) & 0
          \end{bmatrix}
        \\
        \mH_{Em}(r) &=
          \begin{bmatrix}
            -\frac{nk_z}{rk_{3m}^2}H_n^{(2)}(k_{3m}r) & \frac{i\w\mu_m}{k_{3m}}H_n^{(2)}{}'(k_{3m}r)
            \\
            H_n^{(2)}(k_{3m}r) & 0
          \end{bmatrix}
        \\
        \mJ_{Hm}(r) &=
          \begin{bmatrix}
            0 & -J_n(k_{3m}r)
            \\
            -\frac{i\w\eps_m}{k_{3m}}J_n'(k_{3m}r) & -\frac{nk_z}{rk_{3m}^2}J_n(k_{3m}r)
          \end{bmatrix}
        \\
        \mH_{Hm}(r) &=
          \begin{bmatrix}
            0 & -H_n^{(2)}(k_{3m}r)
            \\
            -\frac{i\w\eps_m}{k_{m3}}H_n^{(2)}{}'(k_{3m}r) & -\frac{nk_z}{rk_{3m}^2}H_n^{(2)}(k_{3m}r)
          \end{bmatrix}
        \\
        \mA_{Jm}(r) &= \mJ_{Em}(r) -  \mZ_{m-1} \mJ_{Hm}(r)
        \\
        \mA_{Hm}(r) &= \mH_{Em}(r) -  \mZ_{m-1} \mH_{Hm}(r)
      \end{align}
    \end{subequations}

    \begin{thm}
      Soit \(\hat \mZ_0(n,k_z) = \mat{0}_{\mathcal{M}_2(\CC)}\).

      Si pour tout \(0 < m < n\)

      \begin{equation}
        \begin{aligned}
          k_{3m} & \not = 0 \\
          \det\left(\mA_{Jm}(r_{m-1})\right) & \not = 0
          \\
          \det\left(\mA_{Hm}(r_{m-1})\right) & \not = 0
          \\
          \det\left(\mH_{Hm}(r_{m})\mA_{Hm}(r_{m-1})^{-1} - \mJ_{Hm}(r_{m})(\mA_{Jm}(r_{m-1}))^{-1}\right) &\not = 0
        \end{aligned}
      \end{equation}

      Alors le symbole \(\hat \mZ_n\) est défini par la relation de récurrence :
      \begin{multline}
        \mZ_m = \left(\mH_{Em}(r_m)\mA_{Hm}(r_{m-1})^{-1} - \mJ_{Em}(r_m)\mA_{Jm}(r_{m-1})^{-1}\right) \\
            \left(\mH_{Hm}(r_m)\mA_{Hm}(r_{m-1})^{-1} - \mJ_{Hm}(r_m)\mA_{Jm}(r_{m-1})^{-1}\right)^{-1}
      \end{multline}
    \end{thm}

    \begin{proof}
      À l'initialisation, on retrouve le résultat pour une couche.

      On résonne par récursivité:

      On se situe dans la couche \(m\) et l'on sait que les champs vérifient
      \begin{equation}
        \begin{bmatrix}
          \hat{E_\theta}(r_{m-1},n,k_z)\\
          \hat{E_z}(r_{m-1},n,k_z)\\
        \end{bmatrix}
        =
        \hat \mZ_{m-1}(n,k_z)
        \begin{bmatrix}
          -\hat{H_z}(r_{m-1},n,k_z)\\
          \hat{H_\theta}(r_{m-1},n,k_z)\\
        \end{bmatrix}
      \end{equation}

      En injectant ce qui précède dans \eqref{eq:imp_fourier:cylindre:champs} en \(r = r_{m-1}\)
      \begin{align}
        \mJ_{Em}(r_{m-1})
        \begin{bmatrix}
          c_1 \\
          c_3
        \end{bmatrix}
        +
        \mH_{Em}(r_{m-1})
        \begin{bmatrix}
          c_2 \\
          c_4
        \end{bmatrix}
        &=
        \hat \mZ_{m-1}
        \left(
          \mJ_{Hm}(r_{m-1})
          \begin{bmatrix}
            c_1 \\
            c_3
          \end{bmatrix}
          +
          \mH_{Hm}(r_{m-1})
          \begin{bmatrix}
            c_2 \\
            c_4
          \end{bmatrix}
        \right)
        \\
        \mA_{Jm}(r_{m-1})
        \begin{bmatrix}
          c_1 \\
          c_3
        \end{bmatrix}
        &=
        -\mA_{Hm}(r_{m-1})
        \begin{bmatrix}
          c_2 \\
          c_4
        \end{bmatrix}
      \end{align}

      \begin{TODO}
        Inversibilité de \(\mA_{Jm}(r_m), \mA_{Hm}(r_m)\).
      \end{TODO}

      On injecte ce qui précède dans \eqref{eq:imp_fourier:cylindre:champs} en \(r = r_{m}\)
      \begin{align}
        \vE_t &=
        \left(\mH_{Em}(r_{m}) - \mJ_{Em}(r_{m})\mA_{Jm}(r_{m-1})^{-1}\mA_{Hm}(r_{m-1})\right)
        \begin{bmatrix}
          c_2 \\
          c_4
        \end{bmatrix}
        \\
        \vect{e_r}\times\vH_t &=
        \left(\mH_{Hm}(r_{m}) - \mJ_{Hm}(r_{m})\mA_{Jm}(r_{m-1})^{-1}\mA_{Hm}(r_{m-1}) \right)
        \begin{bmatrix}
          c_2 \\
          c_4
        \end{bmatrix}
      \end{align}

      \begin{TODO}
        Inversibilité de \(\mH_{Hm}(r_m) - \mJ_{Hm}(r_m)(\mA_{Jm}(r_{m-1}))^{-1}\mA_{Hm}(r_{m-1})\).
      \end{TODO}

      On peut alors conclure sur le symbole

      \begin{multline}
        \hat \mZ_{m} =
          \left(\mH_{Em}(r_m) - \mJ_{Em}(r_m)\mA_{Jm}(r_{m-1})^{-1}\mA_{Hm}(r_{m-1})\right) \\
          \left(\mH_{Hm}(r_m) - \mJ_{Hm}(r_m)\mA_{Jm}(r_{m-1})^{-1}\mA_{Hm}(r_{m-1})\right)^{-1}
      \end{multline}

      \begin{multline}
        \hat \mZ_{m} =
          \left(\mH_{Em}(r_m)\mA_{Hm}(r_{m-1})^{-1} - \mJ_{Em}(r_m)\mA_{Jm}(r_{m-1})^{-1}\right) \\
          \left(\mH_{Hm}(r_m)\mA_{Hm}(r_{m-1})^{-1} - \mJ_{Hm}(r_m)\mA_{Jm}(r_{m-1})^{-1}\right)^{-1}
      \end{multline}

    \end{proof}

  \subsection{Applications numérique}

    La figure \ref{fig:imp_fourier:cylindre:hoppe_p62} permet de vérifier les résultats de \cite[p.~62]{hoppe_impedance_1995} pour une couche de matériau sans perte (voir Figure \ref{fig:annex:hoppe:p62}).

    \begin{TODO}
      Expliquer pourquoi on prendre des valeurs continues de \(n\) et non discrètes
    \end{TODO}

    % \begin{figure}[!hbt]
    %   \centering
    %   \begin{tikzpicture}[scale=1]
  \begin{axis}[
      title={},
      ylabel={\(\Im(\hat{Z}(k_t r_1,0))\)},
      xlabel={\(k_t\slash k_0\)},
      width=0.8\textwidth,
      xmin=0,
      xmax=1.5,
      mark repeat=20,
      legend pos=outer north east
    ]
    \addplot [black] table [x={s1}, y={Im(z_ex.tm)},col sep=comma] {csv/HOPPE_62/HOPPE_62.z_ex.C_+3.000E-02.csv};
    \addlegendentry{TM}
    \addplot [black,dashed] table [x={s1}, y={Im(z_ex.te)},col sep=comma]  {csv/HOPPE_62/HOPPE_62.z_ex.C_+3.000E-02.csv};
    \addlegendentry{TE}
  \end{axis}
\end{tikzpicture}
    %   \caption{\(\eps = 6, \mu = 1, r_0 = 0.0300\text{m}, d=0.0225\text{m}, f=1\text{GHz}\)}
    %   \label{fig:imp_fourier:cylindre:hoppe_p62}
    % \end{figure}

    La figure \ref{fig:imp_fourier:cylindre:hoppe_p62:converge_rayon} montre la convergence du symbole de l'impédance d'un cylindre vers le symbole du plan en fonction du rayon du cylindre.

    \begin{figure}[!hbt]
      \centering
      \tikzsetnextfilename{Z_HOPPE_62_cylindre_converge_TM}
\begin{tikzpicture}[scale=1]
  \begin{axis}[
      title={Polarisation TM},
      ylabel={\(\Im(\hat{Z}(k_tr_1,0))\)},
      xlabel={\(k_t \slash k_0\)},
      width=0.37\textwidth,
      xmin=0,
      xmax=1.5,
      legend pos=outer north east
    ]

    \addplot [black,dotted,mark=diamond] table [col sep=comma, x={s2}, y={Im(z_ex.tm)}] {csv/HOPPE_62/HOPPE_62.z_ex.MODE_2_TYPE_C_+3.000E-02.csv};

    \addplot [black,dotted,mark=*] table [col sep=comma, x={s2}, y={Im(z_ex.tm)}] {csv/HOPPE_62/HOPPE_62.z_ex.MODE_2_TYPE_C_+3.000E-01.csv};

    \addplot [black,dashed] table [col sep=comma, x={s2}, y={Im(z_ex.tm)}] {csv/HOPPE_62/HOPPE_62.z_ex.MODE_2_TYPE_C_+3.000E+00.csv};

    \addplot [black] table [col sep=comma, x={s1}, y={Im(z_ex.tm)}] {csv/HOPPE_62/HOPPE_62.z_ex.MODE_2_TYPE_P.csv};
  \end{axis}
\end{tikzpicture}
\tikzsetnextfilename{Z_HOPPE_62_cylindre_converge_TE}
\begin{tikzpicture}[scale=1]
  \begin{axis}[
      title={Polarisation TE},
      ylabel={},
      xlabel={\(k_t \slash k_0\)},
      width=0.37\textwidth,
      xmin=0,
      xmax=1.5,
      legend pos=outer north east
    ]

    \addplot [black,dotted,mark=diamond] table [col sep=comma, x={s2}, y={Im(z_ex.te)}] {csv/HOPPE_62/HOPPE_62.z_ex.MODE_2_TYPE_C_+3.000E-02.csv};
    \addlegendentry{\(r_0=0.03m\)}

    \addplot [black,dotted,mark=*] table [col sep=comma, x={s2}, y={Im(z_ex.te)}] {csv/HOPPE_62/HOPPE_62.z_ex.MODE_2_TYPE_C_+3.000E-01.csv};
    \addlegendentry{\(r_0=0.3m\)}

    \addplot [black,dashed] table [col sep=comma, x={s2}, y={Im(z_ex.te)}] {csv/HOPPE_62/HOPPE_62.z_ex.MODE_2_TYPE_C_+3.000E+00.csv};
    \addlegendentry{\(r_0=3m\)}

    \addplot [black] table [col sep=comma, x={s1}, y={Im(z_ex.te)}] {csv/HOPPE_62/HOPPE_62.z_ex.MODE_2_TYPE_P.csv};
    \addlegendentry{plan}
  \end{axis}
\end{tikzpicture}
      \caption{\(\eps = 6, \mu = 1, d=0.0225\text{m}, f=1\text{GHz}\)}
      \label{fig:imp_fourier:cylindre:hoppe_p62:converge_rayon}
    \end{figure}

\begin{TODO}
  Courbes erreurs plan cylindre
\end{TODO}

    % \begin{figure}[!hbt]
    %   \centering
    %   \begin{tikzpicture}[scale=1]
    %     \begin{loglogaxis}[
    %         title={},
    %         ylabel={\(||\hat{\mZ}_{plan} - \hat{\mZ}_{cyl}||_2\)},
    %         xlabel={\(r_0/d\)},
    %         width=0.8\textwidth,
    %         xmin=0.1,
    %         xmax=100,
    %         % mark repeat=20,
    %         legend pos=outer north east
    %       ]
    %       \legend{TM,TE}
    %       \addplot [black] table [x={r0/d}, y={tm},col sep=semicolon] {csv/cylindre/hoppe_p62_error.csv};
    %       \addplot [black,dashed] table [x={r0/d}, y={te},col sep=semicolon] {csv/cylindre/hoppe_p62_error.csv};
    %     \end{loglogaxis}
    %   \end{tikzpicture}
    %   \caption{\(\eps = 6, \mu = 1, d=0.0225\text{m}, f=1\text{GHz}\)}
    %   \label{fig:imp_fourier:cylindre:hoppe_p62:converge_rayon:error}
    % \end{figure}
\section{Approximation de la matrice d'impédance pour un cylindre infini par une CIOE}

  \subsection[Expression des opérateurs LD,LR en Fourier]{Expression des opérateurs \(\LD,\LR\) en Fourier}
    Soit \(C(0,r_C)\) un cylindre de centre 0, de rayon \(r_C\) et d'axe \(\vect{e_z}\) et \((r,\theta,z)\) les coordonnées cylindre d'un point de l'espace.

    Soit \(V = \left(\mathcal{C}^\infty(C(0,r_C))\right)^2 \cap L^2(C(0,r_C)))\).

    \begin{defn}
      \label{eq:cylindre:fourier:LD}
      On définit \(\LD\) l'endomorphisme de \(V\) tel que
      \begin{align*}
        \LD \vect{U}(r,\theta,z) & = \vgrads{} \vdivs{} \vect{U}(r,\theta,z)
      \end{align*}

      On définit \(\hat{\mLD}\) la fonction de \(\NN\times\RR \rightarrow \mathcal{M}_2(\RR)\) telle que
      \begin{equation*}
        \hat{\mLD}(n,k_z) = -
        \begin{bmatrix}
          \left({n}\slash{r_C}\right)^2 & k_z{n}\slash{r_C}
          \\
          k_z{n}\slash{r_C} & k_z^2
        \end{bmatrix}
      \end{equation*}
    \end{defn}

    \begin{prop}
      Soit \(\vect{U} \in V\)
      Alors
      \begin{equation*}
        \widehat{\LD \vect{U}} (r_C,n,k_z) = \hat{\mLD}(n,k_z) \hat{\vect{U}}(r_C,n,k_z)
      \end{equation*}
    \end{prop}

    \begin{proof}
      Par définition de \(\LD\), on a
      \begin{align*}
        \LD \vect{U} & = \vgrads{} \vdivs{} \vect{U}
      \end{align*}
      On utilise les expression en coordonnées cylindrique des opérateurs différentiels ( voir annexe \ref{sec:annexe:div_grad_rot}).
      \begin{align*}
        \vdivs{\vect{U}}(r,\theta,z) = \frac{1}{r}\ddr{\theta}{U_\theta}(r,\theta,z) + \ddr{z}{U_z}(r,\theta,z)
      \end{align*}
      \begin{align*}
        \vgrads{f}(r,\theta,z) = \frac{1}{r}\ddr{\theta}{f}(r,\theta,z)\vect{e_\theta} + \ddr{z}{f}(r,\theta,z)\vect{e_z}
      \end{align*}
      Or d’après la définition de la transformée de Fourier
      \begin{align*}
        \vect{U}(r,\theta,z) & = \frac{1}{2\pi}\sum_{n=-\infty}^\infty \int_\RR \hat{\vect{U}}(r,n,k_z)e^{in\theta + ik_zz}\dd{k_z}
      \end{align*}
      les opérateurs en Fourier sont
      \begin{align*}
        \widehat{\vdivs{\vect{U}}}(r,n,k_z) = \frac{in}{r}{\hat{U}_\theta}(r,n,k_z) + ik_z{\hat{U}_z}(r,n,k_z)
      \end{align*}
      \begin{align*}
        \widehat{\vgrads{f}}(r,n,k_z) = \frac{in}{r}\hat{f}(r,n,k_z)\vect{e_\theta} + ik_z\hat{f}(r,n,k_z)\vect{e_z}
      \end{align*}
      donc
      \begin{align*}
        \widehat{\vgrads \vdivs{\vect{U}}}(r,n,k_z) =  \left(-\frac{n^2}{r^2}\vect{e_\theta} - \frac{nk_z}{r}\vect{e_z}\right){\hat{U}_\theta}(r,n,k_z) + \left(-\frac{nk_z}{r}\vect{e_\theta} - {k_z^2}\vect{e_z}\right){\hat{U}_z}(r,n,k_z)
      \end{align*}

    \end{proof}


    \begin{defn}
      \label{eq:cylindre:fourier:LR}

      On définit \(\LR\) l'endomorphisme de \(V\) tel que
      \begin{align*}
        \LR \vect{U}(r,\theta,z) & = \vrots{} (\rots{} \vect{U})(r,\theta,z)
      \end{align*}

      On définit \(\hat{\mLR}\) la fonction de \(\NN\times\RR \rightarrow \mathcal{M}_2(\RR)\) telle que
      \begin{equation*}
        \hat{\mLR}(n,k_z) = 
        \begin{bmatrix}
          -k_z^2 & k_z{n}\slash{r_C}
          \\
          k_z{n}\slash{r_C} & -\left({n}\slash{r_C}\right)^2
        \end{bmatrix}
      \end{equation*}
    \end{defn}

    \begin{prop}
      Soit \(\vect{U} \in V\)
      Alors
      \begin{equation*}
        \widehat{\LR \vect{U}} (r_C,n,k_z) = \hat{\mLR}(n,k_z) \hat{\vect{U}}(r_C,n,k_z)
      \end{equation*}
    \end{prop}

    \begin{proof}
      Par définition de \(\LR\), on a
      \begin{align*}
        \LR \vect{U} & = \vrots{} (\rots{} \vect{U})
      \end{align*}
      On utilise les expression en coordonnées cylindrique des opérateurs différentiels ( voir annexe \ref{sec:annexe:div_grad_rot}).
      \begin{align*}
        \rots{\vect{U}}(r,\theta,z) = \frac{1}{r}\ddr{\theta}{U_z}(r,\theta,z) - \ddr{z}{U_\theta}(r,\theta,z)
      \end{align*}
      \begin{align*}
        \vrots{f}(r,\theta,z) = \ddr{z}{f}(r,\theta,z)\vect{e_\theta} - \frac{1}{r}\ddr{\theta}{f}(r,\theta,z)\vect{e_z}
      \end{align*}
      donc comme pour l'opérateur \(\LD\)
      \begin{align*}
        \widehat{\rots{\vect{U}}}(r,n,k_z) = \frac{in}{r}{\hat{U}_z}(r,n,k_z) - ik_z{\hat{U}_\theta}(r,n,k_z)
      \end{align*}
      \begin{align*}
        \widehat{\vrots{f}}(r,n,k_z) =  ik_z\hat{f}(r,n,k_z)\vect{e_\theta} - \frac{in}{r}\hat{f}(r,n,k_z)\vect{e_z}
      \end{align*}
      donc
      \begin{align*}
        \widehat{\vrots (\rots{\vect{U}})}(r,n,k_z) =  \left({k_z^2}\vect{e_\theta} - \frac{nk_z}{r}\vect{e_z}\right){\hat{U}_\theta}(r,n,k_z) + \left(-\frac{nk_z}{r}\vect{e_\theta} + \frac{n^2}{r^2}\vect{e_z}\right){\hat{U}_z}(r,n,k_z)
      \end{align*}

    \end{proof}

  \subsection{Expression de la matrice d'impédance approchée par la CI3}

    Tout comme dans le cas du plan infini, on peut donc définir \(\hat{\mZ}_{IBC}\) l’opérateur matriciel associé à la condition d'impédance.

    \begin{multline}
        \hat{\mZ}_{CI3}(n,k_z) = \left(I + b_1 \frac{\hat{\mLD}(n,k_z)}{k_0^2} - b_2 \frac{\hat{\mLR}(n,k_z)}{k_0^2} \right)^{-1}\\
        \left(a_0 I + a_1 \frac{\hat{\mLD}(n,k_z)}{k_0^2} - a_2 \frac{\hat{\mLR}(n,k_z)}{k_0^2}\right)
    \end{multline}

\section{Calcul des coefficients des CIOE par moindres carrés sur l'impédance}

  \subsection{Expression des moindre carrés dans le cadre de l'approximation cylindre infini pour une incidence}

  \subsection{Expression des moindre carrés dans le cadre de l'approximation cylindre infini avec un balayage en incidence}

  \subsection{Résultats numériques sur l'approximation de la matrice d'impédance}

    La figure \ref{fig:imp_fourier:plan:hoppe:62:hoibc:ibc6} permet de vérifier les résultats de \cite[p.~62]{hoppe_impedance_1995} pour une couche de matériau sans perte.

    % \begin{figure}[!hbt]
    %   \centering
    %   \tikzsetnextfilename{Z_HOPPE_62_cylindre_hoibc_TM}
\begin{tikzpicture}[scale=1]
    \begin{axis}[
            title={Polarisation TM},
            ylabel={\(\Im(\hat{Z}(k_t r_{1},0)\)},
            xlabel={\(k_t\slash k_0\)},
            width=0.4\textwidth,
            xmin=0,
            xmax=2.5,
            mark repeat=20,
            legend pos=outer north east
        ]
        \addplot [black,mark=square*] table [col sep=comma, x={s1}, y={Im(z_ex.tm)}] {csv/HOPPE_62/HOPPE_62.z_ex.MODE_2_TYPE_C_+3.000E-02.csv};

        \addplot [blue,mark=x] table [col sep=comma, x={s1}, y={Im(z_ibc0.tm)}] {csv/HOPPE_62/HOPPE_62.z_ibc.IBC_ibc0_SUC_F_MODE_2_TYPE_C_+3.000E-02.csv};

        \addplot [red,mark=diamond*] table [col sep=comma, x={s1}, y={Im(z_ibc3.tm)}] {csv/HOPPE_62/HOPPE_62.z_ibc.IBC_ibc3_SUC_F_MODE_2_TYPE_C_+3.000E-02.csv};
    \end{axis}
\end{tikzpicture}
\tikzsetnextfilename{Z_HOPPE_62_cylindre_hoibc_TE}
\begin{tikzpicture}[scale=1]
    \begin{axis}[
            title={Polarisation TE},
            ylabel={},
            xlabel={\(k_t\slash k_0\)},
            width=0.4\textwidth,
            xmin=0,
            xmax=2.5,
            mark repeat=20,
            legend pos=outer north east
        ]
        \addplot [black,mark=square*] table [col sep=comma, x={s1}, y={Im(z_ex.te)}] {csv/HOPPE_62/HOPPE_62.z_ex.MODE_2_TYPE_C_+3.000E-02.csv};
        \addlegendentry{Exact};

        \addplot [blue,mark=x] table [col sep=comma, x={s1}, y={Im(z_ibc0.te)},color=] {csv/HOPPE_62/HOPPE_62.z_ibc.IBC_ibc0_SUC_F_MODE_2_TYPE_C_+3.000E-02.csv};
        \addlegendentry{CI0};

        \addplot [red,mark=diamond*] table [col sep=comma, x={s1}, y={Im(z_ibc3.te)}] {csv/HOPPE_62/HOPPE_62.z_ibc.IBC_ibc3_SUC_F_MODE_2_TYPE_C_+3.000E-02.csv};
        \addlegendentry{CI3};
    \end{axis}
\end{tikzpicture}
    %   \caption[CIOE sur empilement de Hoppe & Rahmat-Samii p.~62]{\(\eps = 6, \mu = 1, d=0.0225\text{m}, f=1\text{GHz}, r_0=0.03\text{m}\)}
    %   \label{fig:imp_fourier:plan:hoppe:62:hoibc}
    % \end{figure}
    % \begin{table}[!hbt]
    %   \centering
    %   % On fait deux tables de même hauteur
    %   \begin{coefftable}{\hyperlink{ci0}{CI0}}
    %     \input{csv/HOPPE_62/HOPPE_62.IBC_ibc0_SUC_F_MODE_2_TYPE_C_+3.000E-02.coeff.txt}
    %     \\
    %     \\
    %     \\
    %     \\
    %     \\
    %   \end{coefftable}
    %   \begin{coefftable}{\hyperlink{ci3}{CI3}}
    %     \input{csv/HOPPE_62/HOPPE_62.IBC_ibc3_SUC_F_MODE_2_TYPE_C_+3.000E-02.coeff.txt}
    %   \end{coefftable}
    %   \caption{Coefficients associés à la figure \ref{fig:imp_fourier:plan:hoppe:62:hoibc}}
    %   \label{tab:imp_fourier:plan:hoppe:62:hoibc}
    % \end{table}

    On remarque que la CI3 si performante dans l'approximation plan infini ne donnent pas de bons résultats dans l’approximation cylindre infini. 
    En effet, la matrice d'impédance exacte n'est pas une constante mais une matrice diagonale pour \(n=k_z=0\). 
    Or par définition la CIOE est une constante pour ce couple, c'est la CI0. On subit donc cette erreur dans les résultats. 

    Une CIOE plus intéressante, que l'on nomme CI6 et qui est inspirée de \cite[p.~60]{hoppe_impedance_1995}, serait alors:

    \begin{equation}
      \left(\oI + c_1\frac{\LD}{k_0^2} -c_2\frac{\LR}{k_0^2}\right)\vE_t = \left(\diag{a_1}{a_2} + b_1\frac{\LD}{k_0^2} - b_2 \frac{\LR}{k_0^2} \right)\vJ
    \end{equation}

    \begin{figure}[!hbt]
      \centering
      \tikzsetnextfilename{Z_HOPPE_62_cylindre_hoibc_ibc6_TM}
\begin{tikzpicture}[scale=1]
    \begin{axis}[
            title={Polarisation TM},
            ylabel={\(\Im(\hat{Z}(k_t r_{1},0)\)},
            xlabel={\(k_t\slash k_0\)},
            width=0.4\textwidth,
            xmin=0,
            xmax=2.5,
            mark repeat=1,
            legend pos=outer north east
        ]
        \addplot [black,mark=square*] table [col sep=comma, x={s2}, y={Im(z_ex.tm)}] {csv/HOPPE_62/HOPPE_62.z_ex.MODE_2_TYPE_C_+3.000E-02.csv};

        \addplot [blue,mark=x] table [col sep=comma, x={s2}, y={Im(z_ibc0.tm)}] {csv/HOPPE_62/HOPPE_62.z_ibc.IBC_ibc0_SUC_F_MODE_2_TYPE_C_+3.000E-02.csv};

        \addplot [red,mark=diamond*] table [col sep=comma, x={s2}, y={Im(z_ibc3.tm)}] {csv/HOPPE_62/HOPPE_62.z_ibc.IBC_ibc3_SUC_F_MODE_2_TYPE_C_+3.000E-02.csv};

        \addplot [violet,mark=triangle*] table [col sep=comma, x={s2}, y={Im(z_ibc6.tm)}] {csv/HOPPE_62/HOPPE_62.z_ibc.IBC_ibc6_SUC_F_MODE_2_TYPE_C_+3.000E-02.csv};
    \end{axis}
\end{tikzpicture}
\tikzsetnextfilename{Z_HOPPE_62_cylindre_hoibc_ibc6_TE}
\begin{tikzpicture}[scale=1]
    \begin{axis}[
            title={Polarisation TE},
            ylabel={},
            xlabel={\(k_t\slash k_0\)},
            width=0.4\textwidth,
            xmin=0,
            xmax=2.5,
            mark repeat=1,
            legend pos=outer north east
        ]
        \addplot [black,mark=square*] table [col sep=comma, x={s2}, y={Im(z_ex.te)}] {csv/HOPPE_62/HOPPE_62.z_ex.MODE_2_TYPE_C_+3.000E-02.csv};
        \addlegendentry{Exact};

        \addplot [blue,mark=x] table [col sep=comma, x={s2}, y={Im(z_ibc0.te)},color=] {csv/HOPPE_62/HOPPE_62.z_ibc.IBC_ibc0_SUC_F_MODE_2_TYPE_C_+3.000E-02.csv};
        \addlegendentry{CI0};

        \addplot [red,mark=diamond*] table [col sep=comma, x={s2}, y={Im(z_ibc3.te)}] {csv/HOPPE_62/HOPPE_62.z_ibc.IBC_ibc3_SUC_F_MODE_2_TYPE_C_+3.000E-02.csv};
        \addlegendentry{CI3};

        \addplot [violet,mark=triangle*] table [col sep=comma, x={s2}, y={Im(z_ibc6.te)}] {csv/HOPPE_62/HOPPE_62.z_ibc.IBC_ibc6_SUC_F_MODE_2_TYPE_C_+3.000E-02.csv};
        \addlegendentry{CI6};
    \end{axis}
\end{tikzpicture}
      \caption[CIOE sur empilement de Hoppe & Rahmat-Samii p.~62]{\(\eps = 6, \mu = 1, d=0.0225\text{m}, f=1\text{GHz}, r_0=0.03\text{m}\)}
      \label{fig:imp_fourier:plan:hoppe:62:hoibc:ibc6}
    \end{figure}
    \begin{table}[!hbt]
      \centering
      % On fait deux tables de même hauteur
      \begin{minipage}[t]{0.49\textwidth}
        \vspace{0pt}
        \centering
        \begin{coefftable}{\hyperlink{ci0}{CI0}}
          \input{csv/HOPPE_62/HOPPE_62.IBC_ibc0_SUC_F_MODE_2_TYPE_C_+3.000E-02.coeff.txt}
        \end{coefftable}
        \begin{coefftable}{\hyperlink{ci3}{CI3}}
          \input{csv/HOPPE_62/HOPPE_62.IBC_ibc3_SUC_F_MODE_2_TYPE_C_+3.000E-02.coeff.txt}
        \end{coefftable}
      \end{minipage}
      \begin{minipage}[t]{0.49\textwidth}
        \vspace{0pt}
        \centering
        \begin{coefftable}{\hyperlink{ci6}{CI6}}
          \input{csv/HOPPE_62/HOPPE_62.IBC_ibc6_SUC_F_MODE_2_TYPE_C_+3.000E-02.coeff.txt}
        \end{coefftable}
      \end{minipage}
      \caption{Coefficients associés à la figure \ref{fig:imp_fourier:plan:hoppe:62:hoibc:ibc6}}
      \label{tab:imp_fourier:plan:hoppe:62:hoibc:ibc6}
    \end{table}

    Cependant cette CIOE ne sera pas retenue car son implémentation dans le code équation intégrale nécessite une modification de ce dernier. On présente néanmoins dans la figure \ref{fig:imp_fourier:plan:hoppe:62:hoibc:ibc6} sa performance vis à vis de la CI3 sur le cylindre.



\section{Calcul des coefficients des CIOE par moindres carrés sur les coefficients de la série de Fourier}

  On remarque alors que sans contraintes, la CI3 n'est pas adaptée au cylindre, même en incidence radiale. Cependant, on peut aussi résoudre l'erreur entre les coefficients de Fourier. Pour cela, on adapte au cylindre le lemme permettant de déduire d'une impédance la matrice de réflexion qui contient les coefficients de la série de Fourier.

  \begin{defn}[Matrice de réflexion associée à une impédance]\label{def:cylindre:reflexion_from_impedance}{}~

    Soit \(\mM_\mJ\) et \(\mM_\mH\) les fonctions introduite à la définition \ref{def:cylindre:matrices_MJ-MH}.

    On définit la fonction \(\hat\mR\) de \(\NN\times\RR\times\tilde{\mA}thcal{M}_2(\CC) \rightarrow \tilde{\mA}thcal{M}_2(\CC)\) telle que
    \begin{equation*}
      \hat\mR(n,k_z,\tilde{\mA}) = - \mM_\mH(r_c^+,n,k_z,\tilde{\mA})^{-1}\mM_\mJ(r_c^+,n,k_z,\tilde{\mA})
    \end{equation*}
  \end{defn}
  % \begin{prop}
  %   Soit \(r_c^+\) l'interface cylindre-vide.

  %   Soit \(\hat{\mZ}(n,k_z)\) la matrice d'impédance à cette interface.
  %   \begin{align*}
  %     \hat{\vE}_t(r_c^+,n,k_z) &= \hat{\mZ}(n,k_z) \left(\vect{e_r} \pvect \hat{\vH}_t(r_c^+,n,k_z)\right)
  %   \end{align*}
  %   alors il existe \(\vect{C}(n,k_z) \in \CC^2\) tel que 
  %   \begin{align*}
  %     \hat{\vE}_t(r_c^+,n,k_z) &= \left(\mJ_E(r_c^+,n,k_z) + \mH_E(r_c^+,n,k_z)\hat\mR(n,k_z,\hat\mZ(n,k_z)\right)\vect{C}(n,k_z)
  %     \\
  %     \vect{e_r} \pvect \hat{\vH}_t(r_c^+,n,k_z) &= \left(\mJ_H(r_c^+,n,k_z) + \mH_H(r_c^+,n,k_z)\hat\mR(n,k_z,\hat\mZ(n,k_z)\right)\vect{C}(n,k_z)
  %   \end{align*}
  % \end{prop}
  % \begin{proof}
  %   Immédiat depuis la définition des champs \eqref{eq:imp_fourier:cylindre:Et},\eqref{eq:imp_fourier:cylindre:Ht}.
  % \end{proof}

  \begin{defn}%[]
    \label{def:cylindre:minimisation:matrices_MR}
    On définit les fonctions \(\hat\mR_{ex}, \hat\mR_{CI3}\) de \(\NN\times\RR\times \rightarrow \tilde{\mA}thcal{M}_2(\CC)\) telles que
    \begin{align*}
      \hat\mR_{ex}(n,k_z) &= \hat\mR(n,k_z, \hat\mZ_{ex}(n,k_z))
      \\
      \hat\mR_{CI3}(n,k_z) &= \hat\mR(n,k_z, \hat\mZ_{CI3}(n,k_z))
    \end{align*}
    où \(\hat\mZ_{ex},\hat\mZ_{CI3}\) sont des fonctions définies à la proposition \ref{prop:cylindre:synthese:impedance} et à l'équation \eqref{eq:cylindre:hoibc:ci3}.
  \end{defn}

  \subsection{Expression de la fonctionnelle}

    On utilise les fonctions \(\mN_E, \mN_H\) introduite à la définition \ref{def:cylindre:matrices_NE-NH}.

    \begin{defn}
      On définit \(\tilde{\mH}_{CI3}\) la fonction de \(\NN \times \RR \times \tilde{\mA}thcal{M}_2(\CC) \rightarrow \tilde{\mA}thcal{M}_{4\times5}(\CC)\) telle que
      \begin{equation*}
        \mH_{CI3}(n,k_z,\tilde{\mA}) = 
        \begin{bmatrix}
          A_0(n,k_z,\tilde{\mA})_{11} & A_1(n,k_z,\tilde{\mA})_{11} & A_2(n,k_z,\tilde{\mA})_{11} & B_1(n,k_z,\tilde{\mA})_{11} & B_2(n,k_z,\tilde{\mA})_{11}
          \\
          A_0(n,k_z,\tilde{\mA})_{12} & A_1(n,k_z,\tilde{\mA})_{12} & A_2(n,k_z,\tilde{\mA})_{12} & B_1(n,k_z,\tilde{\mA})_{12} & B_2(n,k_z,\tilde{\mA})_{12}
          \\
          A_0(n,k_z,\tilde{\mA})_{21} & A_1(n,k_z,\tilde{\mA})_{21} & A_2(n,k_z,\tilde{\mA})_{21} & B_1(n,k_z,\tilde{\mA})_{21} & B_2(n,k_z,\tilde{\mA})_{21}
          \\
          A_0(n,k_z,\tilde{\mA})_{22} & A_1(n,k_z,\tilde{\mA})_{22} & A_2(n,k_z,\tilde{\mA})_{22} & B_1(n,k_z,\tilde{\mA})_{22} & B_2(n,k_z,\tilde{\mA})_{22}
        \end{bmatrix}
        \end{equation*}
        où
        \begin{align*}
          A_0(n,k_z,\tilde{\mA}) &= \mN_E(r_c^+,n,k_z,\tilde{\mA})
          \\
          A_1(n,k_z,\tilde{\mA}) &= \hat{\mLD}(n,k_z)\mN_E(r_c^+,n,k_z,\tilde{\mA})
          \\
          A_2(n,k_z,\tilde{\mA}) &= -\hat{\mLR}(n,k_z)\mN_E(r_c^+,n,k_z,\tilde{\mA})
          \\
          B_1(n,k_z,\tilde{\mA}) &= \hat{\mLD}(n,k_z)\mN_H(r_c^+,n,k_z,\tilde{\mA})
          \\
          B_2(n,k_z,\tilde{\mA}) &= -\hat{\mLR}(n,k_z)\mN_H(r_c^+,n,k_z,\tilde{\mA})            
        \end{align*}

        On définit \(\tilde{b}\) la fonction de \(\NN \times \RR \times \tilde{\mA}thcal{M}_2(\CC) \rightarrow \tilde{\mA}thcal{M}_{4\times1}(\CC)\) telle que
        \begin{equation*}
          \tilde{b}(n,k_z,\tilde{\mA}) = -
          \begin{bmatrix}
            \mN_H(r_c^+,n,k_z,\tilde{\mA})_{11}
            \\
            \mN_H(r_c^+,n,k_z,\tilde{\mA})_{12}
            \\
            \mN_H(r_c^+,n,k_z,\tilde{\mA})_{21}
            \\
            \mN_H(r_c^+,n,k_z,\tilde{\mA})_{22}
          \end{bmatrix}
        \end{equation*}
      \end{defn}

    \begin{prop}
      Soit \(X = (a_0,a_1,a_2,b_1,b_2)\), \((n,k_z)\) fixé et \(\hat\mR_{ex}\) la matrice définie en \ref{def:cylindre:minimisation:matrices_MR}, alors
      \begin{multline*}
        \argmin{X\in\CC^5} \norm{\hat\mR_{CI3}(n,k_z,X) - \hat\mR_{ex}(n,k_z)} =
        \\
        \argmin{X\in\CC^5} \norm{\tilde{\mH}_{CI3}(n,k_z,\hat\mR_{ex}(n,k_z))X - \tilde{b}(n,k_z,\hat\mR_{ex}(n,k_z))}
      \end{multline*}
    \end{prop}

    \begin{proof}
      C'est la même méthodologie que pour l'impédance.
      On rappelle de la section précédente
      \begin{multline*}
        \hat{\mZ}_{CI3}(n,k_z) = \left(\mI + b_1 \hat{\mLD}(n,k_z) - b_2 \hat{\mLR}(n,k_z) \right)^{-1}
        \\
        \left(a_0 \mI + a_1 {\hat{\mLD}(n,k_z)} - a_2 {\hat{\mLR}(n,k_z)}\right)
      \end{multline*}
      On pose \(\hat\mZ_D(n,k_z) = \mI + b_1 \hat{\mLD}(n,k_z) - b_2 \hat{\mLR}(n,k_z)\) et \(\hat\mZ_N(n,k_z) = a_0 \mI + a_1 {\hat{\mLD}(n,k_z)} - a_2 {\hat{\mLR}(n,k_z)}\) donc

      \begin{align*}
        &{\hspace{1em}}~ \argmin{X\in\CC^5} \norm{\hat\mR_{CI3}(n,k_z,X) - \hat\mR_{ex}(n,k_z)}
        \\
        & = \argmin{X\in\CC^5} \norm{ - \mM_\mH(r_c^+,n,k_z,\hat\mZ_{CI3})^{-1}\mM_\mJ(r_c^+,n,k_z,\hat\mZ_{CI3})- \hat\mR_{ex}(n,k_z) }
        \\
        & = \argmin{X\in\CC^5} \norm{ - \mM_\mH(r_c^+,n,k_z,\hat\mZ_{CI3})^{-1}\left(\mM_\mJ(r_c^+,n,k_z,\hat\mZ_{CI3}) +  \mM_\mH(r_c^+,n,k_z,\hat\mZ_{CI3})\hat\mR_{ex}(n,k_z)\right) }      
        \\ 
        & = \argmin{X\in\CC^5} \norm{\mM_\mJ(r_c^+,n,k_z,\hat\mZ_{CI3}) +\mM_\mH(r_c^+,n,k_z,\hat\mZ_{CI3})\hat\mR_{ex}(n,k_z)}
        \intertext{D'après la définition \ref{def:cylindre:matrices_MJ-MH} des fonctions \(\mM_\mJ, \mM_\mH\),}
        & = \argmin{X\in\CC^5} \left\lVert \left(\mJ_E(r_c^+,n,k_z)-\hat\mZ_{CI3}(n,k_z)\mJ_H(r_c^+,n,k_z)\right) \right.
        \\
        & \qquad \qquad \quad + \left.\left(\mH_E(r_c^+,n,k_z)-\hat\mZ_{CI3}(n,k_z)\mH_H(r_c^+,n,k_z)\right)\hat\mR_{ex}(n,k_z) \right\lVert
        \intertext{D'après la définition de \(\hat\mZ_{CI3}\),}        
        & = \argmin{X\in\CC^5} \left\lVert \hat\mZ_D(n,k_z)^{-1}\left(\hat\mZ_D(n,k_z)\mJ_E(r_c^+,n,k_z)-\hat\mZ_N(n,k_z)\mJ_H(r_c^+,n,k_z)\right) \right.
        \\
        & \qquad \qquad \quad + \left.\hat\mZ_D(n,k_z)^{-1}\left(\hat\mZ_D(n,k_z)\mH_E(r_c^+,n,k_z)-\hat\mZ_N(n,k_z)\mH_H(r_c^+,n,k_z)\right)\hat\mR_{ex}(n,k_z) \right\lVert
        \\
        & = \argmin{X\in\CC^5} \left\lVert \left(\hat\mZ_D(n,k_z)\mJ_E(r_c^+,n,k_z)-\hat\mZ_N(n,k_z)\mJ_H(r_c^+,n,k_z)\right) \right.
        \\
        & \qquad \qquad \quad + \left.\left(\hat\mZ_D(n,k_z)\mH_E(r_c^+,n,k_z)-\hat\mZ_N(n,k_z)\mH_H(r_c^+,n,k_z)\right)\hat\mR_{ex}(n,k_z) \right\lVert
        \intertext{D'après la définition \ref{def:cylindre:matrices_NE-NH} des fonctions \(\mN_E, \mN_H\),}        
        & = \argmin{X\in\CC^5} \norm{\hat\mZ_N(n,k_z)\mN_E(r_c^+,n,k_z,\hat\mR_{ex}(n,k_z)) + \hat\mZ_D(n,k_z)\mN_H(r_c^+,n,k_z,\hat\mR_{ex}(n,k_z))}
      \end{align*}
      et on conclut d'après la définition des fonctions \(\hat\mZ_D, \hat\mZ_N\).
    \end{proof}

    On tronque la série de Fourier à \(N_{n}\) termes et on se dote de \(N_{k_z}\) \(k_z\). Il existe donc \(N_{n}N_{k_z}\) couples tels que \((n_i,k_{zj}) = (n,k_z)_{(j-1)N_{n}+i}\).
    \begin{defn}
      On définit \(\tilde{\mA}_{CI3}\) la matrice de \(\mathcal{M}_{4N_{n}N_{k_z}\times5}(\CC)\) telle que
      \begin{equation*}
        \tilde{\mA}_{CI3} = 
        \begin{bmatrix}
          \tilde\mH_{CI3}(n_1,k_{z1},\hat\mR_{ex}(n_1,k_{z1}))
          \\
          \vdots
          \\
          \tilde\mH_{CI3}(n_i,k_{zj},\hat\mR_{ex}(n_i,k_{zj}))
          \\
          \vdots
          \\
          \tilde\mH_{CI3}(n_{N_n},k_{zN_{k_z}},\hat\mR_{ex}(n_{N_n},k_{zN_{k_z}}))
        \end{bmatrix}
      \end{equation*}
      On définit \(\tilde{g}\) le vecteur colonne \(\CC^{4N_{n}N_{k_z}}\) telle que
      \begin{equation*}
        \tilde{g} = 
        \begin{bmatrix}
          \tilde{b}(n_1,k_{z1},\hat\mR_{ex}(n_1,k_{z1}))
          \\
          \vdots
          \\
          \tilde{b}(n_i,k_{zj},\hat\mR_{ex}(n_i,k_{zj}))
          \\
          \vdots
          \\
          \tilde{b}(n_{N_n},k_{zN_{k_z}},\hat\mR_{ex}(n_{N_n},k_{zN_{k_z}}))
        \end{bmatrix}
      \end{equation*}
    \end{defn}

    On peut alors calculer les coefficients

    \begin{thm}[Minimisation sans contraintes pour la CI3]

      Les coefficients de la CIOE sont solutions du problème

      Trouver \(X^* \in \CC^5\) tel que
      \begin{equation*}
        X^* = \argmin{X\in \CC^5} \norm{\tilde{\mA}_{CI3}X - \tilde{g}}
      \end{equation*}
    \end{thm}

    \begin{prop}
      Si \(\conj{\tilde{\mA}_{CI3}^t}\tilde{\mA}_{CI3}\) est inversible alors
      \begin{equation*}
        X^* = (\conj{\tilde{\mA}_{CI3}^t}\tilde{\mA}_{CI3})^{-1}\conj{\tilde{\mA}_{CI3}^t}\tilde{g}
      \end{equation*}
    \end{prop}
    \begin{proof}
      Même méthode que pour la proposition \ref{prop:cylindre:minimisation:minimum_sans_contraintes} sur l'impédance.
    \end{proof}

    Nous n'avons pas réussi à démontrer que cette matrice était définie pour tout empilement et tout incidence.

    \begin{thm}[Minimisation avec contraintes pour la CI3]

      Soit \(\CSU[3]{CI3}\) le sous-espace de \(\CC^5\) issu de la définition \ref{def:csu:ci3-3}.
      Alors les coefficients de la CIOE respectant les CSU sont solutions du problème

      Trouver \(X^* \in \CC^5\) tel que
      \begin{equation*}
        X^* = \argmin{X\in \CSU[3]{CI3}} \norm{\tilde{\mA}_{CI3}X - \tilde{g}}
      \end{equation*}
    \end{thm}

    \subsection{Résultats numériques sur l'approximation des coefficients de la série de Fourier}


\sectionstar{Conclusion}
Nous avons montré comment calculer les coefficients dans le cas d'un objet cylindre infini en minimisant au sens des moindres carrés la différence entre les coefficients de Fourier exact et approchés. 

Cette géométrie a pu montrer la limite de la CI3 pour approcher l'impédance exacte, mais qu'elle approchait bien les coefficients de la série de Fourier.
