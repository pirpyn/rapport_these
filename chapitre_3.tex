\chapter{Conditions suffisantes d'unicité}
\minitoc
\newpage
\sectionstar{Introduction}
Nous avons montrés dans les parties précédentes comment calculer les coefficients des \glspl{acr-cioe} par les moindres carrés. Dans cette partie, nous allons rappeler et étendre les travaux de \cite{stupfel_sufficient_2011} à la \hyperref[ci3]{CI3}. Nous rappellerons donc la \gls{acr-cgu} et les \glspl{acr-csu} qu'elle implique sur les coefficients des CIOE. Nous introduirons le problème de minimisation sous contraintes et la méthode numérique choisie pour le résoudre. Nous verrons que certains empilements de matériaux nécessiterons une réduction du problème pour obtenir de bon coefficients. Enfin nous résoudrons le problème de minimisation exactement dans le cadre de la \hyperref[ci4]{CI4}.
\section{Condition suffisante d'unicité générale}

  %Considérons le problème énoncé dans \cite{stupfel_sufficient_2011} (convention \(e^{i\omega t}\)).
  Soit une \(\Gamma\) une surface régulière fermée dans \(\RR^3\).
  À l'extérieur de cette surface, on définit \(k_0 \in \RR_+^*\), le nombre d'onde dans le vide et \(\eta_0 \in \RR^*\) l'impédance du vide.
  À l'intérieur, on définit les constantes relatives \(\eps,\mu \in \CC\), invariantes par translation.

  \begin{tcolorbox}
    \centering
    La dépendance temporelle est en \(e^{i\w t}\), et l'on identifie \(\vH \equiv \eta _0 \vH\).
  \end{tcolorbox}

  Soit \((\vE,\vH)\) dans \((\Hrot(\OO)) \times (\Hrot(\OO))\).

  Alors \((\vE,\vH)\) sont solutions de Maxwell si
  \begin{align}
    \label{eq:unicite:form_var:maxwell-E}
    \trot \vE + ik_0 \mu \vH &= 0 && \text{dans \(\OO\)}
    \\\label{eq:unicite:form_var:maxwell-H}
    \trot \vH - ik_0\eps \vE &= 0 && \text{dans \(\OO\)}
    % \\\label{eq:unicite:form_var:TR}
    % \Tr(\vE_t) &= - \vn_{Y_R} \pvect \vH && \text{sur \(\Gamma(0,R)\)}
  \end{align}
  % Où \(\Tr\) est l'opérateur de capacité \cite[p.~200]{nedelec_acoustic_2001}, \(\vn_{Y_R}\) la normale unitaire sortante à \(\Gamma(0,R)\).\\

  Les relations \eqref{eq:unicite:form_var:maxwell-E} et \eqref{eq:unicite:form_var:maxwell-H} permettent grâce aux formules de Green d'obtenir la forme variationnelle suivante :
  Trouver \(\vE \in \left(\Hrot(\OO)\right)\), \(\forall \vect \phi \in \left(\Hrot(\OO)\right)\)
  \[
    a(\vE,\vect\phi) = 0
  \]
  où a est une forme sesquilinéaire telle que, soit \(\vn\) la normale unitaire sortante à \(\Gamma\).
  \begin{align*}
    a(\vE,\vect\phi) &:=  \frac{1}{-ik_0\mu} \int_\OO \trot \vE \cdot \trot \conj{\vect{\phi}} dx + -ik_0\eps\int_\OO\vE\cdot\conj{\vect{\phi}} dx
     %+ \int_{Y_R} \conj{ \vect \phi } \cdot \Tr(\vE_t)ds
     - \int_\Gamma \left(\vn \pvect \frac{\trot \vE}{-ik_0\mu}\right) \cdot \conj{\vect \phi} ds \\
   \end{align*}

  \begin{defn}[Coercivité]
    Une forme sesquilinéaire \(a(\vu,\vv)\) est coercive dans \(\Hrot(\OO)\) si \(\exists \alpha > 0\) tel que
    \[
      |\Re(a(\vu,\vu))| \ge \alpha ||\vu||_{\Hrot(\OO)}^2 = \alpha \left( || \trot \vu ||_{L^2}^2 + || \vu ||_{L^2}^2\right) \, \forall \vu \in \Hrot(\OO)
    \]
   \end{defn}

  %La forme bilinéaire \(a\) est coercive \Gamma'il existe une constante réel positive \(\mathcal{C}\) telle que \(|a(\vE,\vE)|^2 \ge \mathcal{C}  || \vE ||_{\Hrot}^4 = \mathcal{C}\left( || \trot \vE ||_{L^2}^2  + || \vE ||_{L^2}^2 \right)^2 \).

  %Supposons \(\eps, \mu\) constants et
  Posons les définitions suivantes:
  \begin{align*}
    % X&:= \int_\OO | \trot \vE | ^2 dx  =|| \trot \vE ||_{L^2}^2
    % \\B&:= \int_\OO | \vE | ^2 dx  = || \vE ||_{L^2}^2
    %\\CC&:= \int_{Y_R} \conj{E_t}\cdot T_R \vE_t ds
    \vJ &:=  \vn \pvect \frac{\trot \vE}{-ik_0\mu} && \text{sur \(\Gamma\)}
    \\
    X&:= \int_\Gamma \vJ \cdot \conj{\vE_t} ds
  \end{align*}
  La partie réelle de la forme bilinéaire \(a\) s'écrit donc
  \begin{equation}
    \label{eq:unicite:form_var:decomp_form_bilin_1}
    |\Re(a(\vE,\vE))| = \left|\frac{\Im(\mu)}{k_0} || \trot \vE ||_{L^2}^2  + k_0 \Im(\eps) || \vE ||_{L^2}^2
    %+ \Re(C)
    - \Re(X)\right|
  \end{equation}

  \begin{hyp}[Hypothèses de coercivité en convention \(i\omega t\)]\label{hyp:unicite:form_var:hyp_coercivite}
    ~{}

    \begin{enumerate}
      \item \(\Im(\eps)\) et \(\Im(\mu)\) sont de même signe.
      \item \(\Im(\eps)\) et \(\Re(X)\) sont de signes opposés\footnote{En convention \(-i\omega t\), il faut que le signe soit le même, et donc \(\Re(X) \le 0\).}.
      %\item \(\Re(C)\) et \(\Re(X)\) sont de même signe.
    \end{enumerate}
  \end{hyp}

\subsection{Cas des matériaux avec pertes}

  Si l'hypothèse \ref{hyp:unicite:form_var:hyp_coercivite} est vérifiée alors \(a\) est coercive ( \(\alpha = \min(-\Im(\mu)k_0^{-1},-\Im(\eps)k_0)\)) donc il y a unicité des solutions du problème de Maxwell avec conditions d'impédance.

  Comme en convention \(e^{i\omega t}\), le signe de \(\Im(\eps)\) et \(\Im(\mu)\) est négatif
  %, sachant que d'après \cite[p.~97]{nedelec_acoustic_2001} \(\Re(C)\ge 0\)
  alors l'unicité est assurée par la
  \begin{defn}[\gls{acr-cgu}]~\\
    \begin{equation}\label{eq:unicite:form_var:cgu}
      \Re(X) = \Re\left(\int_\Gamma \vJ \cdot \conj{\vE_t} ds\right) \ge 0
    \end{equation}
  \end{defn}

\subsection{Cas des matériaux sans pertes}

  Si \(\Im(\mu) = \Im(\eps) = 0\), le résultat précédent n'est plus valable car \(\alpha = 0\).

  \begin{REF}
      Fredholm
  \end{REF}

\section{CSU pour les CIOE de \cite{stupfel_sufficient_2011}}

  Maintenant que nous connaissant une condition suffisante, nous allons montrer comment les CIOE peuvent la garantir. Ce qu'il faut retenir dans cette démarche est que par nature, le caractère suffisant permet d'obtenir pour une CIOE plusieurs jeux de CSU. Cependant, il est intéressant que si l'on ait un jeu de CSU pour une CIOE, alors on y retrouve des CSU d'ordre moins élevés.

  Les CIOE de \cite{stupfel_sufficient_2011} font intervenir l'opérateur de Hodge \(\mathcal{L}\), commençons par rappeler son expression et quelques propriétés propriétés.

  \begin{defn}
    Pour tous \((\vu) \in (\mathcal C^\infty(\CC,\Gamma))^3\)
    \begin{equation}
      \LL(\vu) = \tgrads{\tdivs \vu} - \trots{\trots \vu}
    \end{equation}
  \end{defn}

  \begin{prop}
    Par définition, l’opérateur \(\LL\) est antisymétrique négatif.

    Pour tous \(\vu,\vv \in (\mathcal C^\infty(\CC,\Gamma))^3\)
    \begin{align}
      \int_\Gamma \vu\cdot \LL(\conj{\vv}) &= \int_\Gamma \conj{\vv}\cdot \LL(\vu)
      \\
      \int_\Gamma \vu\cdot \LL(\conj{\vu}) &\le 0
    \end{align}
  \end{prop}

  De plus, on rappelle la notation de la section précédente
  \begin{equation}
    X = \int_\Gamma \vJ \cdot \conj{\vE_t} ds(\vx)
  \end{equation}
  On rappelle que l'on veut trouver des conditions permettant de garantir \eqref{eq:unicite:form_var:cgu} qui est \(\Re(X)\ge0\).

  %%%%%%%%%%%%%%%%%%%%%%%%%%%%%%%%%%%%%%%%%%%%%%%%%%
  \subsection{CSU de la CI0}
    Utilisons la condition d’impédance de Leontovich, la \hyperlink{ci0}{CI0}:

    Soit \(a_0 \in \CC\) tel que
    \[
      \vE_t = a_0 \vJ
    \]

    On a alors
    \begin{equation*}
    X = \conj{a_0}||\vJ||_{L_2(\Gamma)}^2
    \end{equation*}

    De \eqref{eq:unicite:form_var:cgu}, on déduit une \gls{acr-csu}:
    \begin{equation}
    \Re\left(a_0\right) \ge 0
    \end{equation}

  %%%%%%%%%%%%%%%%%%%%%%%%%%%%%%%%%%%%%%%%%%%%%%%%
  \subsection{CSU de la CI01}
    Utilisons la condition d’impédance \hyperlink{ci01}{CI01}:

    Soit \((a_0, a_1) \in \CC\times\CC\) tel que

    \[
      \vE_t = (a_0 +a_1 \LL)\vJ
    \]


    \begin{prop}
      Les CSU qui impliquent la \gls{acr-cgu} sont
      \begin{align}
        \Re\left(a_0\right) \ge 0\\
        \Re\left(a_1\right) \le 0
      \end{align}
    \end{prop}

    \begin{proof}
      On pose:
      \begin{align*}
        F&:= || \vJ|| ^2 \ge 0  & G&:= -\int_\Gamma \vJ\cdot \LL\conj{\vJ} ds \ge 0
      \end{align*}

      On a alors
      \begin{equation*}
        X = \conj{a_0}F - \conj{a_1}G
      \end{equation*}

      De \eqref{eq:unicite:form_var:cgu}, on déduit des CSU suivantes:
      \begin{align}
        \Re\left(a_0\right) \ge 0\\
        \Re\left(a_1\right) \le 0
      \end{align}
    \end{proof}

    Il n'y a pas d'autre CSU qui soient plus évidentes que celles ci et elle permettent de retrouver la CSU de la CI0 quand \(a_1=0\) donc ce sont des CSU satisfaisante pour cette CIOE.

  %%%%%%%%%%%%%%%%%%%%%%%%%%%%%%%%%%%%%%%%%%%%%%%%
  \subsection{CSU de la CI1}

    Utilisons la condition d’impédance \hyperlink{ci1}{CI1}:

    Soit \((a_0, a_1,b) \in \CC^3\) tel que
    \[
      \vE_t = (1 + b \LL)^{-1} (a_0 + a_1 \LL) \vJ
    \]

    Pour cette CIOE, il existe déjà plusieurs jeux de CSU possible.

    \subsubsection{CSU de \cite{stupfel_sufficient_2011}}

      \begin{prop}
        Soit \(\Delta = a_1 - a_0\conj{b}\), alors les CSU sont
        \begin{align}
          &\Re(\Delta) &= 0\\
          &\Im(\Delta)\Im(b) &\ge 0\\
          &\Im(\Delta)\Im(a_1\conj{a_0})&\ge 0
        \end{align}
      \end{prop}

      \begin{proof}
        On utilise l'identité \((a_1-a_0\conj{b}) = (a_1(1+\conj{b}\LL) - \conj{b}(a_0+a_1\LL))\):

        \begin{align*}
          (a_1-a_0\conj{b})X &= \int_\Gamma \left(a_1(1+\conj{b}\LL) \vJ\right)\cdot\conj{\vE_t} - \left(\conj{b}(a_0+a_1 \LL)\vJ\right)\cdot\conj{\vE_t} ds\\
          &= \int_\Gamma \left(a_1(1+\conj{b}\LL) \conj{\vE_t}\right)\cdot\vJ ds - \int_\Gamma \left(\conj{b}(a_0+a_1 \LL)\vJ\right)\cdot\conj{\vE_t} ds\\
          &= \int_\Gamma \left(a_1(\conj{a_0}+\conj{a_1}\LL) \conj{\vJ}\right)\cdot\vJ ds  - \int_\Gamma \left(\conj{b}(1+b \LL)\vE_t\right)\cdot\conj{\vE_t} ds\\
          &= a_1\conj{a_0} ||\vJ||^2 + |a_1|^2 \int_\Gamma \vJ \LL \conj{\vJ} ds - \conj{b} ||\vE_t||^2 - |b|^2 \int_\Gamma \vE_t \LL \conj{\vE_t} ds
        \end{align*}

        On pose \(F = -\int_\Gamma \vJ \LL \conj{\vJ} ds \ge 0 \), \(G = -\int_\Gamma \vE_t \LL \conj{\vE_t} ds \ge 0 \) .

        Si on décompose les parties réelles et imaginaires de cette expression, on a
        \begin{align*}
          \Re(\Delta)\Re(X) - \Im(\Delta)\Im(X) &= \Re(a_1\conj{a_0}) ||\vJ||^2 - \Re(\conj{b})||\vE_t||^2 -|a_1|^2 F + |b|^2 G \\
          \Im(\Delta)\Re(X) + \Re(\Delta)\Im(X) &= \Im(a_1\conj{a_0}) ||\vJ||^2 - \Im(\conj{b})||\vE_t||^2
        \end{align*}
        La première relation nous empêche de conclure sur le signe de \(\Re(X)\) car il y des signes différents entre les deux derniers termes, sauf si nous imposons \(\Re(\Delta)= 0\) auquel cas, nous pouvons conclure grâce à la deuxième relation. Les CSU sont alors
        \begin{align}
          &\Re(\Delta) &= 0\\
          &\Im(\Delta)\Im(b) &\ge 0\\
          &\Im(\Delta)\Im(a_1\conj{a_0})&\ge 0
        \end{align}
      \end{proof}

      Nous pouvons noter que ces CSU fixent par exemple la partie réelle de \(a_1\) en fonction des autres coefficients, ce qui est plus contraignant qu'une condition d'inégalité. De plus, on remarque qu'on ne retombe pas sur les CSU de la CI01 si on annule \(b\). Ce sont donc des CSU trop contraignantes.
    
    \subsubsection{CSU de \cite{stupfel_implementation_2015}}

      Les CSU présentées dans cette article contiennent une erreur. Nous présentons ici une correction qui.

      \begin{prop}
        Un autre jeu de CSU est
        \begin{align}
          a_1 = a_0 \conj{b} \\
          \Re\left(\frac{\conj{a_0}}{1 + \conj{b}^2}\right) &\ge 0 \\
          \Re\left(\frac{\conj{a_1}}{1 + \conj{b}^2}\right) &\le 0
        \end{align}
      \end{prop}

      \begin{proof}
        On cherche à résoudre le problème suivante:
        \[
          \begin{bmatrix}
            1 & \conj{b} \\
            a_0 & a_1
          \end{bmatrix}
          \begin{bmatrix}
            \int_\Gamma \vJ \cdot \conj{\vE_t} \\
            \int_\Gamma \vJ\cdot \LL\conj{\vE_t}
          \end{bmatrix}
          =
          \begin{bmatrix}
            \conj{a_0}||\vJ||_2^2 + \conj{a_1}\int_\Gamma \vJ \cdot \LL\conj{\vJ} \\
            ||\vE_t||_2^2 + b\int_\Gamma \conj{\vE_t} \cdot \LL\vE_t
          \end{bmatrix}
        \]

        Si la matrice est inversible, on pose \(\Delta = a_1 - a_0\conj{b}\) et alors
        \[
        \begin{bmatrix}
          \int_\Gamma \vJ \cdot \conj{\vE_t} \\
          \int_\Gamma \vJ\cdot \LL\conj{\vE_t}
        \end{bmatrix}
        =\frac{1}{\Delta}
        \begin{bmatrix}
          a_1 & -\conj{b} \\
          -a_0 & 1
        \end{bmatrix}
        \begin{bmatrix}
          \conj{a_0}||\vJ||_2^2 + \conj{a_1}\int_\Gamma \vJ \cdot \LL\conj{\vJ} \\
          ||\vE_t||_2^2 + b\int_\Gamma \conj{\vE_t} \cdot \LL\vE_t
        \end{bmatrix}
        \]
        Donc
        \begin{align*}
          X &=  \frac{a_1\conj{a_0}}{\Delta}||\vJ^2||_2^2 - \frac{\conj{b}}{\Delta}||\vE_t||_2^2 \\
          &~+\frac{|a_1|^2}{\Delta}\int_\Gamma\vJ\cdot \LL \conj{\vJ} - \frac{|b|^2}{\Delta}\int_\Gamma\conj{\vE_t}\cdot \LL\vE
        \end{align*}
        Les CSU sont alors bien celles que l'on a déjà trouvées

          \begin{align}
          \Re\left(a_0\conj{a_1}\Delta\right) &\ge 0\\
          \Re\left(b\Delta\right) &\le 0\\
          \Re\left(|a_1|^2\Delta\right) &\le 0\\
          \Re\left(|b|^2\Delta\right) &\ge 0
        \end{align}

        Si la matrice n'est pas inversible, alors on cherche à résoudre
        \[
          \begin{bmatrix}
            1 & \conj{b} \\
            a_0 & a_0\conj{b}
          \end{bmatrix}
          \begin{bmatrix}
            \int_\Gamma \vJ \cdot \conj{\vE_t} \\
            \int_\Gamma \vJ \cdot \LL\conj{\vE_t}
          \end{bmatrix}
          =
          \begin{bmatrix}
            \conj{a_0}||\vJ||_2^2 + \conj{a_1}\int_\Gamma \vJ \cdot \LL\conj{\vJ} \\
            ||\vE_t||_2^2 + b\int_\Gamma \conj{\vE_t} \cdot \LL\vE_t
          \end{bmatrix}
        \]

        Le noyau de la matrice est alors \(\Vect{\begin{bmatrix}\conj{b}\\-1\end{bmatrix}}\) dont l'orthogonal est  \(\Vect{\begin{bmatrix}1\\\conj{b}\end{bmatrix}}\).
        Pour tout \(\int_\Gamma \vJ\cdot \LL\conj{\vE_t} = \conj{b} \int_\Gamma \vJ \cdot \conj{\vE_t} \), on a unicité des solutions. On déduit alors que

        \[
          (1 + \conj{b}^2) X = \conj{a_0} ||\vJ||_2^2 + \conj{a_1}\int_\Gamma \vJ \cdot \LL\conj{\vJ}
        \]

        Les CSU sont alors

        \begin{align}
          \Delta = 0 \\
          \Re\left(\frac{\conj{a_0}}{1 + \conj{b}^2}\right) &\ge 0 \\
          \Re\left(\frac{\conj{a_1}}{1 + \conj{b}^2}\right) &\le 0
        \end{align}

      \end{proof}

      Mais dans ce cas, on ne retombe pas non plus sur les CSU de la CI01 quand \(b\) s'annule.

    \subsubsection{CSU de 2017}
      ~
      Nous avons trouvé qu'il est possible de relâcher un peu les contraintes, si l'on fait une distinction sur la valeur de la constante \(a_1\).
      \paragraph{Cas \(a_1\not=0\)}
        ~
        \begin{prop}
          Les CSU sont
          \begin{align}
            \Re\left(\frac{b}{a_1}\right) \ge 0 \\
            \Re\left(a_0\right) \ge 0 \\
            \Re\left(\left(a_1-a_0 b\right)\frac{\conj{a_1}}{a_1}\right) \le 0
          \end{align}
        \end{prop}
        \begin{proof}
          En supposant \(a_1 \not=0\), on utilise l'identité \((a_0 + a_1 \LL)^{-1}(1 + b \LL)  = \frac{b}{a_1} I_d + \left(1-b\frac{a_0}{a_1}\right)(a_0+a_1 \LL)^{-1}\):
          \[
            X = \int_\Gamma \left(\left(\frac{b}{a_1} I_d + \left(1-b\frac{a_0}{a_1}\right)(a_0+a_1 \LL)^{-1}\right)\vE_t\right) \cdot \conj{\vE_t} ds
          \]

          On pose:
          \begin{align*}
            \vect D &:= (a_0 + a_1 \LL)^{-1}\vE_t & F&:= || \vect D || ^2 \ge 0  \\
            G&:= -\int_\Gamma \vect D \cdot \LL\conj{\v{D}} ds \ge 0 & H &:= || \vE_t || ^2 \ge 0
          \end{align*}
          Comme \(\conj{E_t} = (\conj{a_0} + \conj{a_1}\LL)D\) alors \(\ds\int_\Gamma (a_0 +a_1 \LL) ^{-1}\vE_t\cdot \conj{\vE_t} ds = \conj{a_0} F - \conj{a_1} G\) et l'on peut alors écrire

          \begin{equation}
            \label{eq:unicite:form_var:decomp_cgu_ci1_a1}
            X = \frac{b}{a_1}H   + \left(1-b\frac{a_0}{a_1}\right)\left(\conj{a_0} F - \conj{a_1} G\right)
          \end{equation}
          De \eqref{eq:unicite:form_var:cgu}, on déduit des CSU suivantes:
          \begin{align}
            \Re\left(\frac{b}{a_1}\right) \ge 0 \\
            \Re\left(a_0\right) \ge 0 \\
            \Re\left(\left(a_1-a_0 b\right)\frac{\conj{a_1}}{a_1}\right) \le 0
          \end{align}
        \end{proof}

        Ces CSU permettent de retomber sur les CSU de la CI01 quand \(b\) vaut zéro. De plus, une condition de non-nullité peut n'être vérifiée qu'à posteriori et donc ce jeu est un meilleur choix dans le cadre d'un code numérique.

      \paragraph{Cas \(a_1=0\)}
        ~
        On complète les CSU précédentes
        \begin{prop}
          Les CSU sont
          \begin{align}
            a_1 = 0 \\
            a_0 \not= 0\\
            \Re\left(a_0\right) \ge 0\\
            \Re\left(b\conj{a_0}\right) \le 0
          \end{align}
        \end{prop}
        \begin{proof}
          \[
            X = \int_\Gamma \left( \frac{1}{a_0}\left(1+b\LL\right)\vE_t\right) \cdot \conj{\vE_t} ds
          \]

          On pose:
          \begin{align*}
            F&:= \int_\Gamma | \vE_t | ^2 ds \ge 0 & G &:= -\int_\Gamma \vE_t \cdot \LL\conj{\vE_t} ds \ge 0
          \end{align*}

          On a alors
          \begin{equation}
            \label{eq:unicite:form_var:decomp_cgu_ci1_a1_nul}
            X = \frac{1}{a_0}F - \frac{b}{a_0}G
          \end{equation}

          De \eqref{eq:unicite:form_var:cgu}, on déduit des CSU suivantes:
          \begin{align}
            a_1 = 0 \\
            a_0 \not= 0\\
            \Re\left(a_0\right) \ge 0\\
            \Re\left(b\conj{a_0}\right) \le 0
          \end{align}
        \end{proof}

        On remarque que si \(b\) s’annule, on retombe bien sur la CSU de la CI0. Ce jeu, qui fait la distinction entre deux cas, à malgré tout cet avantage sur le premier.
      %%%%%%%%%%%%%%%%%%%%%%%%%%%%%%%%%%%%%%%%%%%%%%%%%%%%%%%%%%%%%%%%%%%%%%%%%%%%%%%%%%%%%%%%%%%%%%%%%%%%%%%%%

\subsection{CSU pour la CI4}
  Soit la CIOE que l'on nomme \hyperlink{ci4}{CI4} :
  \begin{equation}
    \label{eq:unicite:ci4:ci4}
    \vE_t = (a_0\oI + a_1 \LD - a_2 \LR ) \vJ
  \end{equation}

  \begin{prop}
    Des CSU sont
    \begin{align}
      \Re(a_0) \ge 0
      \\
      \Re(a_1) \le 0
      \\
      \Re(a_2) \le 0
    \end{align}
  \end{prop}

  \begin{prop}
    On pose:
    \begin{align*}
      F &:= ||\vJ|| ^2 \ge 0  & G_1 &:= \int_\Gamma \vJ\cdot \LD\conj{\vJ} ds \le 0 & G_2 &:= \int_\Gamma \vJ\cdot \LR\conj{\vJ} ds \ge 0
    \end{align*}

    On a alors
    \begin{equation*}
      X = \conj{a_0}F + \conj{a_1}G - \conj{a_2}G
    \end{equation*}

    De \eqref{eq:unicite:form_var:cgu}, on déduit des CSU suivantes:
    \begin{align}
      \Re\left(a_0\right) \ge 0
      \\
      \Re\left(a_1\right) \le 0
      \\
      \Re\left(a_2\right) \le 0
    \end{align}
  \end{prop}

  On remarque que ces CSU redonnent les CSU de la CI01 quand \(a_1=a_2\).

\section{CSU pour la CIOE CI3 de \cite{aubakirov_electromagnetic_2014}}

  Soit la CIOE énoncé dans \cite{aubakirov_electromagnetic_2014} que l'on nomme \hyperlink{ci3}{CI3} :
  \begin{equation}
    \label{eq:unicite:ci3:ci3}
    ( \oI + b_1 \LD - b_2 \LR)\vE_t = (a_0\oI + a_1 \LD - a_2 \LR ) \vJ
  \end{equation}

  \begin{defn}
    On rappelle les expressions des opérateurs \gls{ope-LD} et \gls{ope-LR} pour des vecteurs tangents \(\vect U,\vect V \in (\mathcal{C}^\infty(\CC,\Gamma))^3\): 
    \begin{align*}
      \LD(\vect U) &= \tgrads \tdivs \vect U\\
      \LR(\vect V) &= \trots( \vn ( \vn \cdot \trots \vect V))
    \end{align*}
  \end{defn}

  \begin{prop}
    Par définition, \(\LD\) est antisymétrique négatif et \(\LR\) antisymétrique positif.
  \end{prop}

  \begin{prop}
    Soit \(\OO\) un domaine borné de \(\RR^3\) , de surface \(\Gamma\) fermée et régulière, où \(\vect n\) y est la normale unitaire
    sortante
    \begin{equation}
      \begin{matrix}
        \forall \vect U \in (\mathcal{C}^\infty(\CC,\Gamma))^3 ,& \LR(\LD(\vect U)) = \LD(\LR(\vect U)) = 0
      \end{matrix}
    \end{equation}
  \end{prop}
  \begin{proof}

    Soient un vecteur \textbf{tangent} \(\vect U \in (\mathcal{C}^\infty(\CC,\Gamma)^3)\). 

    Montrons que \(\LR\LD = 0\).
    D’après \cite[p.~1029, A3.42]{bladel_electromagnetic_2007}, \(\vn \cdot \trots\tgrads f = 0\)
    \begin{align*}
      \LR(\LD \vect U)  &= \trots \left(\vn \left(\vn \cdot \trots \left( \tgrads \left(\tdivs \vect U\right)\right)\right)\right) \\
      &= 0
    \end{align*}
    Montrons que \(\LD\LR = 0\).
    D’après \cite[p.~1029, A3.43]{bladel_electromagnetic_2007}, \(\tdivs \trots (f\vn) = 0\).
    \begin{align*}
      \LD(\LR \vect U) &= \tgrads \tdivs \trots (\vn (\vn \cdot \trots \vect U)) \\
      &= 0
    \end{align*}
  \end{proof}
  % Une relation importante qui découle des propriétés des opérateurs différentiels surfacique \secref{eq:op-LD-LR:prop:LDLR0} est :

  % \begin{equation}
  % \int_\Gamma \LD(\vect U) \cdot \LR(\vect V) ds = 0 , \forall \vect U, \vect V \in (H^1(\OO))^3
  % \end{equation}

  % Cette relation \Gamma'exprime sous forme forte par \(\LD\LR\equiv0\). Elle est là aussi symétrique entre les deux opérateurs.

\subsection{CSU de Stupfel}

  \begin{prop}
    Soit \(\Delta_i = a_i-\conj{b_i}a_0\), \(i=1,2\). Des CSU sont
    \begin{align}
      \Re\left(a_0\conj{a_1}\Delta_1\right) \ge 0 \\
      \Re\left(\frac{\conj{b_1}}{\Delta_1}\right) \le 0 \\
      \Re\left(\conj{a_0}a_2\left(\frac{\conj{b_2}}{\Delta_2}-\frac{\conj{b_2}}{\Delta_2}\right) + \frac{\conj{a_2}a_1}{\Delta_1} \right)\le 0\\
      \Re\left(2\Re(b_2)\frac{\conj{b_1}}{\Delta_1}-\frac{\conj{b_2}^2}{\Delta_2}\right) \ge 0\\
      \Re\left(a_0\conj{a_2}\Delta_2\right) \ge 0 \\
      \Re\left(\frac{\conj{b_2}}{\Delta_2}\right) \le 0 \\
      \Re\left(\conj{a_0}a_1\left(\frac{\conj{b_1}}{\Delta_1}-\frac{\conj{b_2}}{\Delta_2}\right) + \frac{\conj{a_1}a_2}{\Delta_2} \right)\le 0\\
      \Re\left(2\Re(b_1)\frac{\conj{b_2}}{\Delta_2}-\frac{\conj{b_1}^2}{\Delta_1}\right) \ge 0\\
      \Re\left(\Delta_1\right) = 0 \\
      \Re\left(\Delta_2\right) = 0 \\
      \Re\left(\frac{\conj{b_2}}{\Delta_2}-\frac{\conj{b_1}}{\Delta_1}\right) = 0
    \end{align}
  \end{prop}
  
  \begin{proof}
    On prend l'expression de la CIOE \eqref{eq:unicite:ci3:ci3} et on l’intègre avec des produits scalaires judicieusement choisis.

    \begin{multline}
      \label{eq:unicite:ci3:csu3-1}
      \int_\Gamma \vJ\cdot\conj{\eqref{eq:unicite:ci3:ci3}}ds \Rightarrow
      \int_\Gamma \vJ \cdot \conj{\vE_t} ds  + \conj{b_1} \int_\Gamma \vJ\cdot \LD\conj{\vE_t} ds - \conj{b_2} \int_\Gamma \vJ \LR\conj{\vE_t} ds \\
      = \conj{a_0} \int_\Gamma |\vJ|^2ds - \conj{a_1} \int_\Gamma |\tdivs \vJ|^2 ds - \conj{a_2} \int_\Gamma |\vn \cdot \trots \vJ|^2 ds
    \end{multline}
    \begin{multline}
      \label{eq:unicite:ci3:csu3-2}
      \int_\Gamma \eqref{eq:unicite:ci3:ci3} \cdot \conj{\vE_t} ds \Rightarrow
      \int_\Gamma |\vE_t|^2 ds  - b_1 \int_\Gamma | \tdivs \vE |^2 ds - b_2 \int_\Gamma | \vn \cdot \trots \vE_t|^2 ds \\
      = a_0 \int_\Gamma \vJ\cdot \conj{\vE_t}ds + a_1 \int_\Gamma \conj{\vE_t} \LD \vJ ds - a_2 \int_\Gamma \conj{\vE_t} \cdot \LR \vJ ds
    \end{multline}
    \begin{multline}
      \label{eq:unicite:ci3:csu3-3}
      \int_\Gamma \vJ \cdot \LR ( \conj{\eqref{eq:unicite:ci3:ci3}} ) ds \Rightarrow
      \int_\Gamma \vJ \cdot \LR \conj{\vE_t} ds  - \conj{b_2} \int_\Gamma \LR \vJ \cdot \LR \conj{\vE_t} ds \\
      =  \conj{a_0} \int_\Gamma |\vn \cdot \trots \vJ|^2ds - \conj{a_2} \int_\Gamma | \LR \vJ|^2 ds
    \end{multline}
    \begin{multline}
      \label{eq:unicite:ci3:csu3-4}
      \int_\Gamma  \LR ( \eqref{eq:unicite:ci3:ci3} ) \cdot \conj{\vE_t} ds \Rightarrow
      \int_\Gamma | \vn \cdot \trots \vE_t |^2 ds  - \conj{b_2} \int_\Gamma | \LR \vE_t|^2 ds \\
      = a_0 \int_\Gamma \conj{\vE_t} \LR \vJ ds - a_2 \int_\Gamma \LR \conj{\vE_t} \cdot \LR \vJ ds
    \end{multline}
      \begin{multline}
      \label{eq:unicite:ci3:csu3-5}
      \int_\Gamma \vJ \cdot \LD ( \conj{\eqref{eq:unicite:ci3:ci3}} ) ds \Rightarrow
      \int_\Gamma \vJ \cdot \LD \conj{\vE_t} ds  + \conj{b_1} \int_\Gamma \LD \vJ \cdot \LD \conj{\vE_t} ds \\
      = - \conj{a_0} \int_\Gamma |\tdivs \vJ|^2ds + \conj{a_1} \int_\Gamma | \LD \vJ|^2 ds
    \end{multline}
    \begin{multline}
      \label{eq:unicite:ci3:csu3-6}
      \int_\Gamma  \LD ( \eqref{eq:unicite:ci3:ci3} ) \cdot \conj{\vE_t} ds \Rightarrow
      -\int_\Gamma | \tdivs \vE_t |^2 ds  + \conj{b_1} \int_\Gamma | \LD \vE_t|^2 ds \\
      = a_0 \int_\Gamma \conj{\vE_t} \LD \vJ ds + a_1 \int_\Gamma \LD \conj{\vE_t} \cdot \LD \vJ ds
    \end{multline}
    On pose alors les définitions suivantes :
    \begin{align*}
      X&:= \int_\Gamma \vJ \cdot \conj{\vE_t} ds\\
      Y_D&:= \int_\Gamma \vJ \cdot \LD \conj{\vE_t} ds
      &Y_R&:= \int_\Gamma \vJ \cdot \LR \conj{\vE_t} ds\\
      Z_D&:= \int_\Gamma \LD \vJ \cdot \LD \conj{\vE_t} ds
      &Z_R&:= \int_\Gamma \LR \vJ \cdot \LR \conj{\vE_t} ds
    \end{align*}

    Les équations \eqref{eq:unicite:ci3:csu3-1} à \eqref{eq:unicite:ci3:csu3-4} sont équivalentes au système \(M_R X_R = F_R\) où

    \begin{align*}
      M_R&:=
      \begin{bmatrix}
        1&\conj{b_1}&-\conj{b_2}&0\\
        a_0&a_1&-a_2&0\\
        0&0&1&-\conj{b_2}\\
        0&0&a_0&-a_2\\
      \end{bmatrix},\;
      X_R =
      \begin{bmatrix}
        X\\
        Y_D\\
        Y_R\\
        Z_R
      \end{bmatrix}\\
      F_R &=
      \begin{bmatrix}
        \conj{a_0} \int_\Gamma |\vJ|^2ds - \conj{a_1} \int_\Gamma |\tdivs \vJ|^2 ds - \conj{a_2} \int_\Gamma |\vn \cdot \trots \vJ|^2 ds \\
        \int_\Gamma |\vE_t|^2 ds  - b_1 \int_\Gamma | \tdivs \vE |^2 ds - b_2 \int_\Gamma | \vn \cdot \trots \vE_t|^2 ds \\
        \conj{a_0} \int_\Gamma |\vn \cdot \trots \vJ|^2ds - \conj{a_2} \int_\Gamma | \LR \vJ|^2 ds \\
        \int_\Gamma | \vn \cdot \trots \vE_t |^2 ds  - \conj{b_2} \int_\Gamma | \LR \vE_t|^2 ds
      \end{bmatrix}
    \end{align*}

    Tandis que les équations \eqref{eq:unicite:ci3:csu3-1},\eqref{eq:unicite:ci3:csu3-2},\eqref{eq:unicite:ci3:csu3-5},\eqref{eq:unicite:ci3:csu3-6} sont équivalentes au système \(M_D X_D= F_D\) où

    \begin{align*}
      M_D&:=
      \begin{bmatrix}
        1&-\conj{b_2}&\conj{b_1}&0\\
        a_0&-a_2&a_1&0\\
        0&0&1&\conj{b_1}\\
        0&0&a_0&a_1\\
      \end{bmatrix},\;
      X_D =
      \begin{bmatrix}
        X\\
        Y_R\\
        Y_D\\
        Z_D
      \end{bmatrix}\\
      F_D &=
      \begin{bmatrix}
        \conj{a_0} \int_\Gamma |\vJ|^2ds - \conj{a_1} \int_\Gamma |\tdivs \vJ|^2 ds - \conj{a_2} \int_\Gamma |\vn \cdot \trots \vJ|^2 ds \\
        \int_\Gamma |\vE_t|^2 ds  - b_1 \int_\Gamma | \tdivs \vE |^2 ds - b_2 \int_\Gamma | \vn \cdot \trots \vE_t|^2 ds \\
        -\conj{a_0} \int_\Gamma |\tdivs \vJ|^2ds + \conj{a_1} \int_\Gamma | \LR \vJ|^2 ds \\
        -\int_\Gamma | \tdivs \vE_t |^2 ds  + \conj{b_1} \int_\Gamma | \LR \vE_t|^2 ds
      \end{bmatrix},\;
    \end{align*}

    On note dans la suite \(\Delta_i = a_i-\conj{b_i}a_0\), \(i=1,2\). On suppose que ces système aient une unique solution. Alors on obtient la première condition suffisante:

    \begin{equation}
      \label{eq:unicite:ci3:csu3-cn-det}
      \Delta_1\Delta_2 \not = 0
    \end{equation}

    \begin{minipage}{0.49\textwidth}
      \textbf{Cas LR}:
      \begin{align}
        \label{eq:unicite:ci3:csu3r-j2}&\Re\left(a_0\conj{a_2}\Delta_2\right) \ge 0 \\
        \label{eq:unicite:ci3:csu3r-e2}&\Re\left(\frac{\conj{b_2}}{\Delta_2}\right) \le 0 \\
        \label{eq:unicite:ci3:csu3r-jdj}&\Re\left(\conj{a_0}a_1\left(\frac{\conj{b_1}}{\Delta_1}-\frac{\conj{b_2}}{\Delta_2}\right) + \frac{\conj{a_1}a_2}{\Delta_2} \right)\le 0\\
        \label{eq:unicite:ci3:csu3r-ede}&\Re\left(2\Re(b_1)\frac{\conj{b_2}}{\Delta_2}-\frac{\conj{b_1}^2}{\Delta_1}\right) \ge 0\\
        \label{eq:unicite:ci3:csu3r-jrj}&\Re\left(|a_2|^2\Delta_2\right) \le 0 \\
        \label{eq:unicite:ci3:csu3r-ere}&\Re\left(|b_2|^2\Delta_2\right) \ge 0 \\
        \label{eq:unicite:ci3:csu3r-rj2}&\Re\left(|a_1|^2\left(\frac{\conj{b_1}}{\Delta_1}-\frac{\conj{b_2}}{\Delta_2}\right)\right)\ge 0\\
        \label{eq:unicite:ci3:csu3r-re2}&\Re\left(|b_1|^2\left(\frac{\conj{b_1}}{\Delta_1}-\frac{\conj{b_2}}{\Delta_2}\right)\right)\le 0
      \end{align}
      \eqref{eq:unicite:ci3:csu3r-jrj} et \eqref{eq:unicite:ci3:csu3r-ere} impliquent :
      \begin{equation}
        \Re\left(\Delta_2\right) = 0\\\
      \end{equation}
      \eqref{eq:unicite:ci3:csu3r-rj2} et \eqref{eq:unicite:ci3:csu3r-re2} impliquent :
      \begin{equation}
        \Re\left(\frac{\conj{b_1}}{\Delta_1}-\frac{\conj{b_2}}{\Delta_2}\right) = 0
      \end{equation}
    \end{minipage}
    \begin{minipage}{0.49\textwidth}
      \textbf{Cas LD}:
      \begin{align}
        \label{eq:unicite:ci3:csu3d-j2}&\Re\left(a_0\conj{a_1}\Delta_1\right) \ge 0 \\
        \label{eq:unicite:ci3:csu3d-e2}&\Re\left(\frac{\conj{b_1}}{\Delta_1}\right) \le 0 \\
        \label{eq:unicite:ci3:csu3d-jrj}&\Re\left(\conj{a_0}a_2\left(\frac{\conj{b_2}}{\Delta_2}-\frac{\conj{b_2}}{\Delta_2}\right) + \frac{\conj{a_2}a_1}{\Delta_1} \right)\le 0\\
        \label{eq:unicite:ci3:csu3d-ere}&\Re\left(2\Re(b_2)\frac{\conj{b_1}}{\Delta_1}-\frac{\conj{b_2}^2}{\Delta_2}\right) \ge 0\\
        \label{eq:unicite:ci3:csu3d-jdj}&\Re\left(|a_1|^2\Delta_1\right) \le 0 \\
        \label{eq:unicite:ci3:csu3d-ede}&\Re\left(|b_1|^2\Delta_1\right) \ge 0 \\
        \label{eq:unicite:ci3:csu3d-dj2}&\Re\left(|a_2|^2\left(\frac{\conj{b_2}}{\Delta_2}-\frac{\conj{b_1}}{\Delta_1}\right)\right)\ge 0\\
        \label{eq:unicite:ci3:csu3d-de2}&\Re\left(|b_2|^2\left(\frac{\conj{b_2}}{\Delta_2}-\frac{\conj{b_1}}{\Delta_1}\right)\right)\le 0
      \end{align}
      \eqref{eq:unicite:ci3:csu3d-jdj} et \eqref{eq:unicite:ci3:csu3d-ede} impliquent :
      \begin{equation}
        \Re\left(\Delta_1\right) = 0\\\
      \end{equation}
      \eqref{eq:unicite:ci3:csu3d-dj2} et \eqref{eq:unicite:ci3:csu3d-de2} impliquent :
      \begin{equation}
        \Re\left(\frac{\conj{b_1}}{\Delta_1}-\frac{\conj{b_2}}{\Delta_2}\right) = 0\\\
      \end{equation}
    \end{minipage}
  \end{proof}
  %Pour le système \(M_D X_D = F_D\), les conditions sont identiques à une permutation des indices 1 et 2 près.
  De part leur nombre, ces CSU sont très contraignantes et ne permettent pas de retrouver des CSU des CIOE d'ordres inférieurs lorsque l'on annule les coefficients \(b_1, b_2\). 


\subsection{CSU de Payen}

  \begin{prop}
    D'autre CSU sont
    \begin{align}
      \Re\left(a_0\right)\ge 0 \\
      \Re\left(a_1 - \frac{\conj{b_1a_0}a_1}{\Delta_1}\right) \le 0 \\
      \Re\left(a_2 - \frac{\conj{b_2a_0}a_2}{\Delta_2}\right) \le 0 \\
      \Re\left(b_1\Delta_1\right) = 0 \\
      \Re\left(b_2\Delta_2\right) = 0 \\
      \Im\left(b_1\Delta_1\right)\Im(b_1)\ge 0\\
      \Im\left(b_2\Delta_2\right)\Im(b_2)\ge 0
    \end{align}
  \end{prop}

  \begin{proof}
    En se basant sur la méthode précédente, on remarque que l'on peut déterminer les inconnus \((Y_R,Z_R)\) (resp. \((Y_D,Z_R)\)) uniquement en fonction des équations \eqref{eq:unicite:ci3:csu3-3} et \eqref{eq:unicite:ci3:csu3-4} (resp. \eqref{eq:unicite:ci3:csu3-5} et \eqref{eq:unicite:ci3:csu3-6}).

    On déduit donc que si \(\Delta_1 \not = 0\) et \(\Delta_2 \not = 0\) alors

    \begin{align}
      Y_R &= \frac{1}{\Delta_2}\left(a_2\left[\conj{a_0}\int_\Gamma \vJ\cdot\LR\conj{\vJ} - \conj{a_2}||\LR J||^2\right]  -\conj{b_2}\left[\int_\Gamma \conj{\vE}\LR{\vE} - b_2 ||\LR \vE ||^2\right]\right) \\
      Y_D &= \frac{1}{\Delta_1}\left(a_1\left[\conj{a_0}\int_\Gamma \vJ\cdot\LD\conj{\vJ} + \conj{a_1}||\LD J||^2\right]  -\conj{b_1}\left[\int_\Gamma \conj{\vE}\LD{\vE} + b_1 ||\LD \vE ||^2\right]\right)
    \end{align}

    Il reste alors à utiliser l'équation \eqref{eq:unicite:ci3:csu3-1} pour obtenir
    \begin{equation}
      X = -\conj{b_1} Y_D + \conj{b_2} Y_R + \conj{a_0} || \vJ ||^2 + \conj{a_1} \int_\Gamma \vJ \cdot \LD \conj{\vJ} - \conj{a_2} \int_\Gamma \vJ \cdot \LR \conj{\vJ}
    \end{equation}

    \begin{multline}
      X = \conj{a_0} || \vJ ||^2 - \conj{a_1} || \vdivs \vJ ||^2 - \conj{a_2} || \vrots \vJ ||^2
      \\
      + \frac{\conj{b_2}}{\Delta_2}\left(a_2\left(\conj{a_0}||\vrots \vJ||^2 - \conj{a_2}||\LR J||^2\right)  -\conj{b_2}\left(||\vrots\vE||^2 - b_2 ||\LR \vE ||^2\right)\right)
      \\
      - \frac{\conj{b_1}}{\Delta_1}\left(a_1\left(-\conj{a_0}||\vdivs\vJ||^2 + \conj{a_1}||\LD J||^2\right)  -\conj{b_1}\left(-||\vdivs\vE||^2 + b_1 ||\LD \vE ||^2\right)\right)
    \end{multline}

    On factorise les termes en \(\vJ\)

    \begin{multline}
      X = \conj{a_0} || \vJ ||^2 - \left(a_1 - \frac{\conj{b_1a_0}a_1}{\Delta_1}\right) || \vdivs \vJ ||^2 - \left(a_2 - \frac{\conj{b_2a_0}a_2}{\Delta_2}\right) || \vrots \vJ ||^2
      \\
      + \frac{\conj{b_2}}{\Delta_2}\left( - |a_2|^2||\LR \vJ||^2  - \conj{b_2}\left(||\vrots\vE||^2 - b_2 ||\LR \vE ||^2\right)\right) 
      \\
      - \frac{\conj{b_1}}{\Delta_1}\left( |a_1|^2||\LD \vJ||^2  - \conj{b_1}\left(-||\vdivs\vE||^2 + b_1 ||\LD \vE ||^2\right)\right)
    \end{multline}

    On développe tous les termes
    \begin{multline}
      X = \conj{a_0} || \vJ ||^2 - \left(a_1 - \frac{\conj{b_1a_0}a_1}{\Delta_1}\right) || \vdivs \vJ ||^2 - \left(a_2 - \frac{\conj{b_2a_0}a_2}{\Delta_2}\right) || \vrots \vJ ||^2
      \\
      - \frac{\conj{b_2}|a_2|^2}{\Delta_2}||\LR \vJ||^2  -  \frac{\conj{b_2}^2}{\Delta_2}||\vrots\vE||^2 +  \frac{|b_2|}{\Delta_2} ||\LR \vE ||^2
      \\
      - \frac{\conj{b_1}|a_1|^2}{\Delta_1}||\LD \vJ||^2  - \frac{\conj{b_1}^2}{\Delta_1}||\vdivs\vE||^2 + \frac{|b_1|^2}{\Delta_1} ||\LD \vE ||^2
    \end{multline}

    On impose alors à la partie réelle de chaque terme d'être positive, et on obtient les CSU suivantes :

    \begin{equation}
      \Re\left(a_0\right)\ge 0
    \end{equation}
    \begin{minipage}{0.5\textwidth}
      \begin{align}
        \Re\left(a_1 - \frac{\conj{b_1a_0}a_1}{\Delta_1}\right) \le 0 \\
        \Re\left(a_2 - \frac{\conj{b_2a_0}a_2}{\Delta_2}\right) \le 0 \\
        \Re\left(\frac{|a_1|^2\conj{b_1}}{\Delta_1}\right) \le 0 \\
        \Re\left(\frac{|a_2|^2\conj{b_2}}{\Delta_2}\right) \le 0
      \end{align}
    \end{minipage}
    \begin{minipage}{0.5\textwidth}
      \begin{align}
        \Re\left(\frac{\conj{b_1}^2}{\Delta_1}\right) \le 0 \\
        \Re\left(\frac{\conj{b_2}^2}{\Delta_2}\right) \le 0 \\
        \Re\left(\frac{|b_1|^2\conj{b_1}}{\Delta_1}\right) \ge 0 \\
        \Re\left(\frac{|b_2|^2\conj{b_2}}{\Delta_2}\right) \ge 0
      \end{align}
    \end{minipage}

    On remarque alors que certaines CSU peuvent se combiner et imposent que les parties réelles de \(b_1\Delta_1,b_2\Delta_2\) soient nulles.

    \begin{align}
      \Re\left(a_0\right)\ge 0 \\
      \Re\left(a_1 - \frac{\conj{b_1a_0}a_1}{\Delta_1}\right) \le 0 \\
      \Re\left(a_2 - \frac{\conj{b_2a_0}a_2}{\Delta_2}\right) \le 0 \\
      \Re\left(b_1\Delta_1\right) = 0 \\
      \Re\left(b_2\Delta_2\right) = 0 \\
      \Im\left(b_1\Delta_1\right)\Im(b_1)\ge 0\\
      \Im\left(b_2\Delta_2\right)\Im(b_2)\ge 0
    \end{align}
  \end{proof}

  On a réussi à réduire le nombre de CSU et en plus, fixer \(b_1=b_2=0\) permet de retomber sur les CSU de la CI4.

\subsection{CSU de Lafitte-Stupfel}

  \begin{prop}
    Soit \(z = \left(1 - \frac{b_1a_0}{a_1} - \frac{b_2a_0}{a_2}\right) \). Des CSU qui assurent la \gls{acr-cgu} sont
    \begin{align}
      \Re\left(\conj{a_0}z\right) \ge 0
      \\
      \Re\left(\conj{a_1}z\right) \le 0
      \\
      \Re\left(\conj{a_2}z\right) \le 0
      \\
      \Re\left(\frac{b_1}{a_1}\right) \ge 0
      \\
      \Re\left(\frac{b_2}{a_2}\right) \ge 0
      \\
      \Re\left(a_0\right) \ge 0
      \\
      \Re\left(a_1\right) \le 0
      \\
      \Re\left(a_2\right) \le 0
      \\
      \Re\left(\frac{b_1\conj{a_2}}{a_1\conj{a_0}}\right) \le 0
      \\
      \Re\left(\frac{b_2\conj{a_1}}{a_2\conj{a_0}}\right) \le 0
    \end{align}
  \end{prop}
  \begin{proof}
    Par définition de la CIOE, on a

    \begin{align}
      X &= \int_\Gamma \left(a_0\oI + a_1 \LD - a_2 \LR \right)^{-1}\left(\oI + b_1 \LD - b_2 \LR \right) \vE_t\cdot \conj{\vE_t}
    \end{align}

    On développe simplement chaque terme

    \begin{multline}
      X = \int_\Gamma \left(a_0\oI + a_1 \LD - a_2 \LR \right)^{-1}
      \\
      + b_1 \left(a_0\oI + a_1 \LD - a_2 \LR \right)^{-1}\LD
      \\
      \left.
      - b_2 \left(a_0\oI + a_1 \LD - a_2 \LR \right)^{-1}\LR \right) \vE_t\cdot \conj{\vE_t}
    \end{multline}

    L'astuce pour obtenir réside dans les égalités suivantes, valables si \(a_1\) et \(a_2\) sont non-nuls.
    \begin{align}
      \LD & = \frac{a_0 + a_1 \LD - a_0}{a_1}
      \\
      \LR & = -\frac{a_0 - a_2 \LD - a_0}{a_2}
    \end{align}

    On pose
    \begin{equation}
      z = \left(1 - \frac{b_1a_0}{a_1} - \frac{b_2a_0}{a_2}\right)
    \end{equation}

    On déduit de ce qui précède que

    \begin{multline}
      X = \int_\Gamma z\left(a_0\oI + a_1 \LD - a_2 \LR \right)^{-1}
      \\
      + \frac{b_1}{a_1} \left(a_0\oI + a_1 \LD - a_2 \LR \right)^{-1}\left(a_0+a_1\LD\right)
      \\
      - \frac{b_2}{a_2} \left(a_0\oI + a_1 \LD - a_2 \LR \right)^{-1}\left(a_0-a_2\LR\right) \vE_t\cdot \conj{\vE_t}
    \end{multline}

    On définit

    \newcommand{\vD}{\vect{D}}
    \newcommand{\vF}{\vect{F}}

    \begin{align}
      \vD & = \left(a_0 \oI + a_1 \LD - a_2\LR \right)^{-1} \vE_t
      \\
      \vF_1 & = \left(\oI - a_2 \left( a_0 + a_1\LD\right)^{-1}\LR\right)^{-1} \vE_t
      \\
      \vF_2 & = \left(\oI + a_1 \left( a_0 - a_2\LR\right)^{-1}\LD\right)^{-1} \vE_t
    \end{align}

    Alors immédiatement, on a

    \begin{multline}
      X = \int_\Gamma z \vD \cdot \left(\conj{a_0} \oI + \conj{a_1} \LD - \conj{a_2}\LR\right)\conj{\vD}
      \\
      + \frac{b_1}{a_1} \left(\oI - \conj{a_2} \left( \conj{a_0} + \conj{a_1}\LD\right)^{-1}\LR\right)\conj{\vF_1}\cdot\vF_1
      \\
      + \frac{b_2}{a_2} \left(\oI + \conj{a_1} \left( \conj{a_0} - \conj{a_2}\LR\right)^{-1}\LD\right)\conj{\vF_2}\cdot\vF_2
    \end{multline}

    Finalement posons

    \newcommand{\vG}{\vect{G}}

    \begin{align}
      \vG_1 & = \left(\conj{a_0} \oI + \conj{a_1} \LD \right)^{-1}\LR \conj{\vF_1}
      \\
      \vG_2 & = \left(\conj{a_0} \oI - \conj{a_2} \LR \right)^{-1}\LD \conj{\vF_2}
    \end{align}

    Puisque \(\LD\) (resp. \(\LR\)) commute avec lui-même, on a les égalités suivantes

    \begin{align}
      \LD\left(\conj{a_0} \oI + \conj{a_1} \LD \right)&=\left(\conj{a_0} \oI + \conj{a_1} \LD \right)\LD
      \\
      \LR\left(\conj{a_0} \oI - \conj{a_2} \LR \right)&=\left(\conj{a_0} \oI - \conj{a_2} \LR \right)\LR
    \end{align}

    Or on a démontré que \(\LD\LR=\LR\LD=0\), et ainsi

    \begin{align}
      \LD\LR\conj{\vF_1} &= \LD\left(\conj{a_0} \oI + \conj{a_1} \LD \right)\vG_1
      \\
      0 & =\left(\conj{a_0} \oI + \conj{a_1} \LD \right)\LD\vG_1
    \end{align}

    Si l'on suppose alors que \(\Re(a_0) \ge 0 \) et \(\Re(a_1) \le 0\) (resp. \(\Re(a_2)\le0\)), alors \(\left(\conj{a_0} \oI + \conj{a_1} \LD \right)\) (resp. \(\left(\conj{a_0} \oI - \conj{a_2} \LR \right)\)) est injectif et donc on déduit que

    \begin{align}
      \LD\vG_1 = 0
      \\
      \LR\vG_2 = 0
    \end{align}

  % \begin{TODO}
  %   En fait plus largement, il faut que \(\Re(a_0)\) et \(\Re(a_{1/2})\) soit de signes opposées. Mais il suffit d'avoir une des deux.
  % \end{TODO}

    Or par définition \(\LR\conj{\vF_1} = \left(\conj{a_0} \oI + \conj{a_1} \LD \right)\vG_1\) (resp. \(\LR\conj{\vF_2} = \left(\conj{a_0} \oI - \conj{a_2} \LR \right)\vG_2\)) donc
    \begin{align}
      \LR\conj{\vF_1} &= \conj{a_0}\vG_1
      \\
      \LD\conj{\vF_2} &= \conj{a_0}\vG_2
    \end{align}

    On réinjecte ce résultat dans la définition de \(X\)

    \begin{multline}
      X = \int_\Gamma z \vD \cdot \left(\conj{a_0} \oI + \conj{a_1} \LD - \conj{a_2}\LR\right)\conj{\vD}
      \\
      + \frac{b_1}{a_1} ||\vF_1||^2 - \frac{b_1\conj{a_2}}{a_1\conj{a_0}} \LR\conj{\vF_1}\cdot\vF_1
      \\
      + \frac{b_2}{a_2} ||\vF_2||^2 + \frac{b_2\conj{a_1}}{a_2\conj{a_0}} \LD\conj{\vF_2}\cdot\vF_2
    \end{multline}

    Les CSU sont alors

    \begin{minipage}{0.5\textwidth}
    \begin{align}
      \Re\left(\conj{a_0}z\right) \ge 0
      \\
      \Re\left(\conj{a_1}z\right) \le 0
      \\
      \Re\left(\conj{a_2}z\right) \le 0
      \\
      \Re\left(\frac{b_1}{a_1}\right) \ge 0
      \\
      \Re\left(\frac{b_2}{a_2}\right) \ge 0
    \end{align}
    \end{minipage}
    \begin{minipage}{0.49\textwidth}
    \begin{align}
      \Re\left(a_0\right) \ge 0
      \\
      \Re\left(a_1\right) \le 0
      \\
      \Re\left(a_2\right) \le 0
      \\
      \Re\left(\frac{b_1\conj{a_2}}{a_1\conj{a_0}}\right) \le 0
      \\
      \Re\left(\frac{b_2\conj{a_1}}{a_2\conj{a_0}}\right) \le 0
    \end{align}
    \end{minipage}
  \end{proof}
\subsection{Problème de la singularité de l'impédance dans le cadre du plan infini pour une couche de matériaux}

Dans la partie précédente nous avons introduit la fonctionnelle que l'on cherche à minimiser qui est \(F(X) = \left\lVert\tilde{\mH} X - b(\mZ_{ex})\right\rVert_{\RR^N}\).

On se place dans le cadre du plan infini de \cite{aubakirov_electromagnetic_2014}, où \(\eps=4,\mu=1,f=12\) Ghz, et \(d=3.5\) mm. %Ce cas non physique possède un onde guidée pour \((k_x,k_y) = (k_0s^\star,0.)\) où \(\mR(k_0s^\star,0.) = \infty\).

Il existe un \(s_z\) tel que \(\hat\mZ_{ex}(k_0s_z,0.) = \infty\). En effet, d'après la formule pour une couche de matériau \eqref{eq:imp_plan:symb_z:1c}, 

\begin{equation}
  \hat{\mZ}_{ex}(k_x,0.) = i\frac{\eta}{k\sqrt{k^2 - k_x^2}}\tan(\sqrt{k^2 - k_x^2}d)\left(k^2\mI - \hat{\mLR}\right)
\end{equation}
Donc il est facile de voir que l'on a une asymptote à cause de la tangente et donc pour cet empilement
\begin{equation}
  s_z = \sqrt{\eps \mu - \left(\frac{\pi/2}{k_0 d}\right)^2}
\end{equation}

Le problème est donc que si nous balayons en incidence et que l'on passe par ce point, alors la matrice \(\hat\mZ\) n'est pas défini en ce point. Or comme le gradient de la fonctionnelle est fonction de cette matrice, le gradient n'est pas défini pour tout \(X\). Si l'on utilise une méthode basée sur le gradient de type Newton, ce que nous avons fait, on comprend pourquoi la méthode numérique échoue à calculer des coefficients.

\subsubsection{Réduction du nombre de variables de la minimisation}

On décompose alors nos matrices et vecteurs en séparant les parties contentant cette asymptote.

On suppose donc qu'il existe \(\tilde{\mH}_\infty, \tilde{b}_\infty, X_\infty,\) tels que
\begin{align*}
  \tilde{\mH} &= \tilde{\mH}_\infty + \tilde{\mH}_r
  \\
  \tilde{b} &= \tilde{b}_\infty + \tilde{b}_r
  \\
  X &= X_\infty + X_r
\end{align*}

Ces matrices et vecteurs sont reliés par les relations
\begin{align}
  \tilde{\mH}_\infty X_\infty &= \tilde{b}_\infty
  \\
  \tilde{\mH}_\infty X_r &= 0
\end{align}

Il faut vraiment voir cette décomposition comme deux partie, où l'une est nulle quasiment partout sauf pour le \(s_z\) problématique et l'autre est définie normalement sauf aux termes correspondant au \(s_z\) où elle est nulle.

Schématiquement on définit \(i_z\) l'indice d'une ligne telle que \(\tilde{b}(\hat{\mZ})_{i_z} = \infty\)

\begin{equation*}
  \begin{matrix}
    \tilde{\mH} &=& \tilde{\mH}_\infty &+& \tilde{\mH}_r
    \\
    \begin{bmatrix}
      \tilde{\mH}_{1,1} & \cdots & \tilde{\mH}_{1,N_{CI}}
      \\
      \vdots & \ddots & \vdots
      \\
      \tilde{\mH}_{i_z-1,1} & \cdots & \tilde{\mH}_{i_z-1,N_{CI}}
      \\
      \tilde{\mH}_{i_z,1} & \cdots & \tilde{\mH}_{i_z,N_{CI}}
      \\
      \tilde{\mH}_{i_z+1,1} & \cdots & \tilde{\mH}_{i_z+1,N_{CI}}
      \\
      \vdots & \ddots & \vdots
      \\
      \tilde{\mH}_{4N_i,1} & \cdots & \tilde{\mH}_{4N_i,N_{CI}}
    \end{bmatrix}
    & = &
    \begin{bmatrix}
      0 & \cdots & 0
      \\
      \vdots & \ddots & \vdots
      \\
      0 & \cdots & 0
      \\
      \tilde{\mH}_{i_z,1} & \cdots & \tilde{\mH}_{i_z,N_{CI}}
      \\
      0 & \cdots & 0
      \\
      \vdots & \ddots & \vdots
      \\
      0 & \cdots & 0
    \end{bmatrix}
    & + & 
    \begin{bmatrix}
      \tilde{\mH}_{1,1} & \cdots & \tilde{\mH}_{1,N_{CI}}
      \\
      \vdots & \ddots & \vdots
      \\
      \tilde{\mH}_{i_z-1,1} & \cdots & \tilde{\mH}_{i_z-1,N_{CI}}
      \\
      0 & \cdots & 0
      \\
      \tilde{\mH}_{i_z+1,1} & \cdots & \tilde{\mH}_{i_z+1,N_{CI}}
      \\
      \vdots & \ddots & \vdots
      \\
      \tilde{\mH}_{4N_i,1} & \cdots & \tilde{\mH}_{4N_i,N_{CI}}
    \end{bmatrix}
  \end{matrix}
\end{equation*}
Dans les fait, \(\tilde{\mH}_\infty\) à \(4\) lignes non-nulles et \(\tilde{\mH}_r\) en a autant de nulles au même endroits.


On développe donc la fonctionnelle suivant cette décomposition.
\begin{align*}
\argmin{X}\left\rVert \tilde{\mH} X - \tilde{b} \right \rVert &= \argmin{X_r}\left\rVert \left(\tilde{\mH}_\infty + \tilde{\mH}_r\right)\left( X_\infty + X_r \right) - \tilde{b}_\infty - \tilde{b}_r \right \rVert
\\
\intertext{On utilise la relation entre \(\tilde{\mH}_\infty X_\infty\) et \(\tilde{b}_\infty\)}
&=  \argmin {X_r}\left\rVert \tilde{\mH}_\infty X_r + \tilde{\mH}_r X_\infty + \tilde{\mH}_r X_r - \tilde{b}_r \right \rVert
\\
\intertext{Enfin par définition de \(\tilde{\mH}_\infty\) et \(X_r\), leur produit est nul}
&= \argmin{X_r} \left\rVert \tilde{\mH}_r ( X_r + X_\infty)- \tilde{b}_r \right \rVert
\end{align*}

On voit alors que l'on peut résoudre le problème si l'on minimise uniquement sur les \(X_r\), les autres étant fixés et que l'on enlève du système les lignes où l'impédance n'est pas définie.

\subsubsection{Application de la réduction à la CI3}

On rappelle l'expression de la CI3
\begin{equation}
  \hat{\mZ}_{ap}(k_x,0.) = \left(\mI + b_1 \hat{\mLD} - b_2 \hat{\mLR} \right)^{-1}\left(a_0\mI + a_1 \hat{\mLD} - a_2 \hat{\mLR} \right)
\end{equation}

On voit donc que pour faire apparaître une asymptote, il faut que la matrice de gauche ne soit pas inversible en \(s_z\).

La matrice \(\tilde{\mH}_\infty\) est donc nulle partout sauf en 8 termes, placés sur les deux dernières colonnes et les 4 lignes correspondantes à \((k_x,k_t)=(k_0 s_z,0)\).

Connaissant les expressions des matrices \(\hat\mLD,\hat\mLR\) introduites dans la partie précédente alors on déduit que
\begin{align}
  X_\infty = \begin{bmatrix}
    0\\
    0\\
    0\\
    (k_0 s_z)^{-2}\\
    (k_0 s_z)^{-2}\\
  \end{bmatrix}
  & &
  X_r = \begin{bmatrix}
  a_0\\
  a_1\\
  a_2\\
  0\\
  0\\
  \end{bmatrix}
\end{align}

\section{Étude de la CI4 par optimisation sous contraintes}

%Soit la condition d'impédance
%\begin{equation}\label{eq:ci4}
%\E_t = ( a_0 + a_1 \LD - a_2 \LR )(\J)
%\end{equation}
\newcommand{\st}{\ds\sum_{i=1}^n t_i}
\newcommand{\stc}{\ds\sum_{i=1}^n t_i^2 }
\newcommand{\mma}{\frac{\left(\st\right)^2 - 2n \stc}{\stc}}
\newcommand{\mmb}{\frac{\left(\st\right)^2}{\stc}}
Soit \(n>1\) un entier.
Soit \(t\) un vecteur réel négatif de taille \(n\), strictement décroissant. Soit \(D\) un réel.
Soit \(Z_{TE}, Z_{TM}\) deux vecteurs de complexe de taille \(n\) dont une composante représente l'impédance du plan infini pour l'onde d'incidence \(t_i\) pour la polar TE, respectivement. TM.
Soit \(M\) la matrice réelle \(3\times3\) suivante:
\[
M = 2\begin{bmatrix}
2n & \st & \st \\
\st & \stc & 0 \\
\st & 0 & \stc
\end{bmatrix}
\]
Soit \(Q\) la matrice réelle de taille \(6 \times 6\) suivante :
\[
Q = \begin{bmatrix}
M & 0_{3 \times 3} \\
0_{3 \times 3} & M
\end{bmatrix}
\]

Soit \(F = \begin{bmatrix}
-1 & 0 & 0 \\
0 & 1 & 0 \\
0 & 0 & 1 \\
0 & 0 & 0 \\
0 & 0 & 0 \\
0 & 0 & 0
\end{bmatrix}\) et soit \(P\) un vecteur de \(\RR^6\) tel que \( F^tP  \in (\RR_-)^3\).

On cherche \(X=(a_0', a_1', a_2',a_0'',a_1'',a_2'') \in \RR^6\) tel que
\begin{align}
X = \argmin{X\in \RR^6} \frac{1}{2}\left<QX,X\right> - \left<P,X\right> + D \text{ s.c } F^tX \in (\RR_-)^3
\end{align}

Cette fonctionnelle possède un unique minimum global car Q est inversible (démontré ci-après) et un unique minimum local à l'espace des contrainte car ce dernier est convexe et  \(Q\) est quadratique ( car correspondante à \(|| Z_{ap}(X)||_2^2\) ).

Pour résoudre ce problème nous allons appliquer la théorie des multiplicateurs de Karush-Khun-Lagrange-Tucker.

Le Lagrangien de ce problème est le suivant :
Soient \(\lambda \in (\RR_+)^3\)
\begin{equation}
\mathcal{L}(X,\lambda) = \frac{1}{2}\left<Q X, X\right>  - \left<P, X\right> + D + \left<\lambda, F^tX\right>
\end{equation}

Pour trouver le point-selle de ce lagrangien, nous allons appliquons le principe du min-max: on minimise par rapport à \(X\), puis on maximise par rapport à \(\lambda\).

On cherche la condition de Karush-Khun-Tucker de stationnarité \(\nabla_X \mathcal L (X,\lambda) = 0\).

\begin{equation}
\nabla_X \mathcal{L}(X,\lambda) = Q X - (P - F\lambda)
\end{equation}

% {\color{red}
% Soit \(F\) le vecteur réel de \(\RR^3\) tel que \(F = \begin{bmatrix} 1 \\ -1 \\ -1 \end{bmatrix}\) et \(\lambda\) le vecteur réel de \(\RR^3\) tel que \(\lambda = \begin{bmatrix} \lambda_1 \\ \lambda_2 \\ \lambda_3 \end{bmatrix}\).
% \begin{equation}
% \nabla_X \mathcal{L}(X,\lambda) = 2Q X - 2 P + \begin{bmatrix}\left<\lambda,F\right> \\ 0_{3\times1} \end{bmatrix}.
% \end{equation}
% }

\begin{prop}[Inversibilité de \(Q\)]~\\

Si \(t\) possède au moins deux composantes différentes alors
\begin{equation}
\det(Q) \not = 0
\end{equation}
\end{prop}

\begin{proof}
Par définition \( \det(Q) = \det(M)^2\). Or
  % \[
  % M = \begin{bmatrix}
  % 2n & \st & \st \\
  % \st & \stc & 0 \\
  % \st & 0 & \stc
  % \end{bmatrix}
  % \]
  \begin{align}
  \det(M) &= 2\left(\sum_{i=1}^n t_i^2\right)\left( n \sum_{i=1}^n t_i^2 - \left(\sum_{i=1}^n t_i\right)^2\right) \\
  &= 2\left<t,t\right>\left( \left<1,1\right> \left<t,t\right> - \left<t,1\right>^2\right)
  \end{align}
  et d'après Cauchy–Schwarz, le terme de droite est non-nul pour tout \(t\) non colinéaire au vecteur dont toutes les composantes valent 1, c'est à dire n'importe quel vecteur ayant au moins  deux composantes différentes.
% De plus, on note que

% \begin{equation}
%   M^{-1} = \frac{1}{\ddet(M)}
%   \begin{bmatrix}
%     \stc & -\st & -\st \\
%     -\st & \mma & \mmb \\
%     -\st & \mmb & \mma
%   \end{bmatrix}
% \end{equation}

et ainsi
\begin{equation}
Q^{-1} =
\begin{bmatrix}
M^{-1} & 0_{3 \times 3} \\
0_{3 \times 3} & M^{-1}
\end{bmatrix}
\end{equation}

\end{proof}
Puisque \(Q\) est inversible, on a la
\begin{prop}[Condition de stationnarité de K.K.T]
\begin{equation}
\nabla_X\mathcal{L}(X,\lambda) = 0 \Leftrightarrow X = Q^{-1}\left(P - F\lambda \right)
\end{equation}
\end{prop}


La fonction duale \(\mathcal D\) est alors le lagrangien \(\mathcal L\) en ce point \(X\)
\begin{align}
\mathcal D (\lambda) &= -\frac{1}{2} \left<P - F\lambda , Q^{-1}\left(P - F\lambda \right) \right > + D\\
&= -\frac{1}{2}\left<F^tQ^{-1}F\lambda,\lambda\right> + \left<F^tQ^{-1}P,\lambda\right> + D
\end{align}


On cherche alors à maximiser la fonction duale.
\begin{equation}
\nabla \mathcal D (\lambda) = -F^tQ^{-1}F\lambda + F^tQ^{-1}P
\end{equation}

\begin{prop}[Inversibilité de \(F^tQ^{-1}F\)]
  \(F^tQ^{-1}F\) est inversible.
\end{prop}
\begin{proof}
\(F^tQ^{-1}F = \tilde FM^{-1}\tilde F\) où \(\tilde F = \begin{bmatrix}
-1 & 0 & 0 \\
0 & 1 & 0 \\
0 & 0 & 1
\end{bmatrix}\).  Comme \(\tilde F\) et \(M\) sont inversibles, alors \(\tilde FM^{-1}\tilde F\) est inversible et \(F^tQ^{-1}F\) l'est aussi.
\end{proof}

Directement on peut trouver le point extremum de la fonction duale.
\begin{align}
\nabla \mathcal D (\lambda) = 0 \Leftrightarrow \lambda &= \left(F^tQ^{-1}F\right)^{-1}\left(F^tQ^{-1}P\right)\\
\intertext{Soit \(\tilde P\in (\RR)^3\) dont les composantes sont les 3 premières de \(P\).}
\lambda &  = \left(\tilde FM^{-1}\tilde F\right)^{-1}\left(\tilde FM^{-1}\tilde P\right) \\
& = \tilde F\tilde P\\
\intertext{Ce qui par définition de \(\tilde F\) et \(\tilde P\) vaut}
& = F^tP
\end{align}

La solution optimale est alors
\begin{align}
X^\star &= Q^{-1}\left(P- F\lambda\right)
\end{align}

On peut vérifier \(X\) vérifie les contraintes : \( F^t X = F^t Q^{-1} P - F^tQ^{-1}F\lambda = 0 \in (\RR_-)^3\).

\subsection{Problème des multiplicateurs nuls}


Le lecteur remarquera que par définition les multiplicateurs se doivent par définition être positif, ce dont le précédent résulat ne tient pas compte. La théorie des multiplicateurs de K.K.T énonce alors une condition supplémentaire.
Soit \(\lambda_i\) est positif et la contrainte associée est active, c'est à dire \(\left<F_{i,:},X\right> = 0\), soit la contrainte est inactive et alors le multiplicateur associé est nul.

Cependant comme \(Q\) n'est pas diagonale, en annulant un multiplicateur, les autres sont modifiés.  Il faut donc reprendre l'ensemble des calculs en supprimant la contrainte correspondante de \(F\). On a alors au plus 8 cas à traiter, où l'on considère toutes les combinaisons possibles de contraintes actives. La solution optimale est alors celle où les multiplicateurs sont tous positifs.

\begin{TODO}
  Montrer ces combinaisons
\end{TODO}


%  Il faut donc savoir quelles sont les contraintes qui seront actives avant de le calculer. On peut alors résoudre géometriquement cette question : l'ensemble des contraintes est une pyramide infini de sommet \((0)\) et d'axes \(F_1,F_2,F_3\). Un point appartient à cette pyramide si les composantes \((a,b,c)\) de ce point respectivement à ces 3 vecteurs sont positives, composantes qui doivent résoudre

% \[
% \begin{bmatrix}
%   |F_1|^2 & F_1 \cdot F_2 & F_1\cdot F_3 \\
%   F_1\cdot F_2 & |F_2|^2 & F_2\cdot F_3 \\
%   F_1\cdot F_3 & F_2\cdot F_3 & |F_3|^2
% \end{bmatrix}
% \begin{bmatrix}
% a\\b\\c
% \end{bmatrix} =
% \begin{bmatrix}
% X\cdot F_1\\
% X\cdot F_2\\
% X\cdot F_3\\
% \end{bmatrix}
% \]

% Si l'une des ces composantes est négative alors le point appartient à l'une des pyramides duales, c'est dire de sommet \((0)\)
\sectionstar{Conclusion}
Nous nous sommes dotés d'une méthode numérique pour calculer les coefficients des CIOE qui permet de garantir l'unicité des solutions des équations de Maxwell. Ce calcul n'est fait que pour un empilement de matériaux et une fréquence donnée mais son coût numérique faible et la rapidité de ce dernier nous permettent de le valider pour l'intégration dans CIOE dans un code de résolution des équations de Maxwell.