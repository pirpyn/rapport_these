\chapter{Conditions suffisantes d'unicité}
\label{sec:chap:CSU}
\minitoc
\newpage
\sectionstar{Introduction}
Nous avons montrés dans les parties précédentes comment calculer les coefficients des \glspl{acr-cioe} par les moindres carrés. Dans cette partie, nous allons rappeler et étendre les travaux de \cite{stupfel_sufficient_2011} à la \hyperlink{ci3}{CI3}. Nous rappellerons donc la \gls{acr-cgu} et les \glspl{acr-csu} qu'elle implique sur les coefficients des CIOE. Nous introduirons le problème de minimisation sous contraintes et la méthode numérique choisie pour le résoudre. Nous verrons que certains empilements de matériaux nécessiterons une réduction du problème pour obtenir de bon coefficients.
\section{Condition suffisante d'unicité générale}

  On rappelle les résultats d’unicité démontrés par \cite[chapitre 2, section 11]{cessenat_mathematical_1996}.

  Soit \(\OO\) un domaine fermé borné de \(\RR^3\) et \(\Gamma\) sa frontière.
  À l'extérieur de cette surface, on définit \(k_0 \in \RR_+^*\), le nombre d'onde dans le vide et \(\eta_0 \in \RR^*\) l'impédance du vide.
  À l'intérieur, on définit les constantes relatives \(\eps,\mu \in \CC\), invariantes par translation dans ce domaine, dont on déduit le nombre d'onde intérieur \(k = k_0 \sqrt{\mu\eps}\).

  Soit \((\vE,\vH)\) dans \((\Hrot(\OO)) \times (\Hrot(\OO))\) tels que:
  \begin{align}
  \left\lbrace
    \begin{matrix}
      \trot \vE + ik \mu \vH &= 0
      \\
      \trot \vH - ik \eps \vE &= 0
      % \\\label{eq:unicite:form_var:TR}
      % \Tr(\vE_t) &= - \vn_{Y_R} \pvect \vH && \text{sur \(\Gamma(0,R)\)}
    \end{matrix}
    \right. && \text{dans \(\OO\)}
  \end{align}
  % Où \(\Tr\) est l'opérateur de capacité \cite[p.~200]{nedelec_acoustic_2001}, \(\vn_{Y_R}\) la normale unitaire sortante à \(\Gamma(0,R)\).\\

  Soit \(\vect{\phi}\) dans \(\Hrot(\OO)\) une fonction test. Ces équations aboutissent à la forme variationnelle suivante :
  \begin{prop}
    Trouver \(\vE \in \left(\Hrot(\OO)\right)\), \(\forall \vect \phi \in \left(\Hrot(\OO)\right)\)
    \[
      a(\vE,\vect\phi) = 0
    \]
  \end{prop}

  où a est une forme sesquilinéaire telle que, soit \(\vn\) la normale unitaire sortante à \(\Gamma\).
  \begin{align*}
    a(\vE,\vect\phi) &:=  \frac{1}{ik\mu} \int_\OO \trot \vE \cdot \trot \conj{\vect{\phi}} dx + ik\eps\int_\OO\vE\cdot\conj{\vect{\phi}} dx
     %+ \int_{Y_R} \conj{ \vect \phi } \cdot \Tr(\vE_t)ds
      - \int_\Gamma \left(\vn \pvect \frac{\trot \vE}{ik\mu}\right) \cdot \conj{\vect \phi} ds(\vx) \\
   \end{align*}

  \begin{proof}
    Partons de la première équation.
    \begin{align}
          0 & = \frac{\trot \vE}{ik\mu} + \vH
          \\ \intertext{Appliquons lui le rotationnel.}
          0 & = \trot \frac{\trot \vE}{ik\mu} + \trot \vH
          \\ \intertext{On peut utiliser la seconde équation pour n'avoir plus que \(\vE\).}
          0 & = \trot \frac{\trot \vE}{ik\mu} + ik\eps \vE
          \\ \intertext{Réalisons le produit scalaire avec la fonction test.}
          0 & = \int_\OO{ \trot \frac{\trot \vE}{ik\mu}\cdot \conj{\vect{\phi}}} +  \int_\OO ik\eps \vE \cdot \conj{\vect{\phi}}
          \\ \intertext{Utilisons la formule de Green du rotationnel (voir \cite[eq.~(A1.32)]{bladel_electromagnetic_2007}).}
          0 & = \int_\OO{ \frac{\trot \vE}{ik\mu}\cdot \trot \conj{\vect{\phi}}} - \int_\Gamma \left( \conj{\vect \phi} \pvect \frac{\trot \vE}{ik\mu}\right)  \cdot \vn ds(\vx) + \int_\OO ik\eps \vE \cdot \conj{\vect{\phi}}
          \\ \intertext{On arrange les termes.}
          0 & = \int_\OO \frac{\trot \vE}{ik\mu}\cdot \trot \conj{\vect{\phi}} - \int_\Gamma \left(\vn \pvect \frac{\trot \vE}{ik\mu}\right) \cdot \conj{\vect \phi} ds(\vx) +  \int_\OO ik\eps \vE \cdot \conj{\vect{\phi}}
      \end{align}
  \end{proof}

  On rappelle la définition de \cite[p.~59]{cessenat_mathematical_1996}.
  \begin{defn}[Coercivité d'une forme sesquilinéaire]
    Une forme sesquilinéaire \(a(\vu,\vv)\) est coercive dans \(\Hrot(\OO)\) si \(\exists \alpha > 0\) tel que
    \[
      |\Re(a(\vu,\vu))| \ge \alpha ||\vu||_{\Hrot(\OO)}^2 = \alpha \left( || \trot \vu ||_{L^2}^2 + || \vu ||_{L^2}^2\right) \, \forall \vu \in \Hrot(\OO)
    \]
   \end{defn}

  %La forme bilinéaire \(a\) est coercive \Gamma'il existe une constante réel positive \(\mathcal{C}\) telle que \(|a(\vE,\vE)|^2 \ge \mathcal{C}  || \vE ||_{\Hrot}^4 = \mathcal{C}\left( || \trot \vE ||_{L^2}^2  + || \vE ||_{L^2}^2 \right)^2 \).

  %Supposons \(\eps, \mu\) constants et
  On définit en tout point de \(\Gamma\) la trace tangentielle de \(\vH\) que l'on note \(\vJ = \vn \pvect \vH\). On définit alors la quantité suivante
  \begin{align}
    X &= \int_\Gamma \vJ \cdot \conj{\vect \phi} ds(\vx)
  \end{align}

  La partie réelle de la forme bilinéaire \(a\) s'écrit donc
  \begin{equation}
    \label{eq:unicite:form_var:decomp_form_bilin_1}
    |\Re(a(\vE,\vE))| = \left|\frac{\Im(\mu)}{k} || \trot \vE ||_{L^2}^2  + k \Im(\eps) || \vE ||_{L^2}^2
    %+ \Re(C)
    - \Re(X)\right|
  \end{equation}

\subsection{Conditions suffisantes d'unicités}

  \begin{hyp}[Hypothèses de coercivité en convention \(i\omega t\)]\label{hyp:unicite:form_var:hyp_coercivite}
    ~{}

    \begin{enumerate}
      \item \(\Im(\eps)\) et \(\Im(\mu)\) sont de même signe.
      \item \(\Im(\eps)\) et \(\Re(X)\) sont de signes opposés\footnote{En convention \(-i\omega t\), il faut que le signe soit le même}.
      %\item \(\Re(C)\) et \(\Re(X)\) sont de même signe.
    \end{enumerate}
  \end{hyp}

\subsection{Cas des matériaux avec pertes}

  Les matériaux avec pertes sont tels que au moins l'une des deux constantes a une partie imaginaire non-nulle.

  Si l'hypothèse \ref{hyp:unicite:form_var:hyp_coercivite} est vérifiée alors \(a\) est coercive ( \(\alpha = \min(-\Im(\mu)k^{-1},-\Im(\eps)k)\)) donc il y a unicité des solutions du problème de Maxwell d'après Lax-Milgram (voir \cite{cessenat_mathematical_1996}).

  Comme nous sommes en convention \(e^{i\omega t}\) et que nous imposons que le signe de \(\Im(\eps)\) et \(\Im(\mu)\) soit négatif
  %, sachant que d'après \cite[p.~97]{nedelec_acoustic_2001} \(\Re(C)\ge 0\)
  alors l'unicité est assurée par la
  \begin{defn}[\glsentrydesc{acr-cgu}]~\\
    \begin{equation}\label{eq:unicite:form_var:cgu}
      \Re\left(\int_\Gamma \vJ \cdot \conj{\vE_t} ds\right) \ge 0
    \end{equation}
  \end{defn}
  Dans la suite de la thèse, nous noterons souvent cette quantité par \gls{mat-q}. C'est par définition la partie réelle du flux sortant du vecteur de Poynting à travers la surface de l'objet. 

\subsection{Cas des matériaux sans pertes}

  Si \(\Im(\mu) = \Im(\eps) = 0\), le résultat précédent n'est plus valable car \(\alpha = 0\). Cependant, l'unicité est assuré par l’alternative de Fredholm (voir \cite[Théorème 8]{cessenat_mathematical_1996}) dont la conclusion est aussi que \ref{eq:unicite:form_var:cgu} doit être vérifiée pour assurée l'unicité.

  Maintenant que nous connaissant la condition suffisante à vérifier, nous allons montrer quelles conditions sur les coefficients des CIOE permettent la vérifier.
\section[Des CSU pour les CIOE de Stupfel et Poget 2011]{Des conditions suffisantes pour les CIOE de \cite{stupfel_sufficient_2011}}

  Nous définissons les \glspl{acr-cioe} comme une condition limite liant \(\vE_t\) et \(\vn\pvect\vH\) sur \(\Gamma\). L'existence des CIOE est en dehors du cadre de cette thèse, nous ne ferons donc qu'utiliser des CIOE existantes.

  Grâce à ces CIOE, allons établir des conditions suffisantes qui impliquent la \gls{acr-cgu} \eqref{eq:unicite:form_var:cgu}. Par leur nature suffisante, il n'y pas un unique jeu de CSU pour une CIOE donnée. Une difficulté est d'être capable de juger si ce jeu est satisfaisant, et si ce n'est pas le cas, être capable de proposer un autre jeu.

  Les CIOE de \cite{stupfel_sufficient_2011} font intervenir l'opérateur de Hodge \(\mathcal{L}\), commençons par rappeler son expression et quelques propriétés.

  Soit les opérateurs différentiels surfaciques (voir annexe \ref{sec:annexe:div_grad_rot})
  \begin{align*}
      \vgrads{f}(\vx) &= \vgrad{f}(\vx) - \vn(\vx) (\vn(\vx) \cdot \vgrad{f}(\vx))
      \\
      \vdivs{\vu} &= \vdiv\left(\vu(\vx) - \vn(\vx) (\vn(\vx) \cdot \vu(\vx)\right)
      \\
      \vrots{\vu}(\vx) &= \vn(\vx)\left(\vn(\vx) \cdot \vrot{\vu}(\vx)\right)
  \end{align*}


  Enfin par abus de notations, nous omettrons tous les dépendances en \(\vx\) dans les intégrales \(\int_\Gamma f(\vx)\vE(\vx)\dd{\Gamma(\vx)}\equiv \int_\Gamma f\vE\).
  \begin{defn}
    \label{def:operator:L}
    % Pour tout \(\vu \in (\mathcal{C}^\infty(\Gamma))^2\)
    \begin{equation*}
        \fonction{\LL}{(\mathcal{C}^\infty(\Gamma))^3}{(\mathcal{C}^\infty(\Gamma))^3}%
          {\vu}{\vgrads{\vdivs \vu} - \vrots{\vrots \vu}}
    \end{equation*}
  \end{defn}

  \begin{prop}
    \label{eq:hodge:negatif}
    L’opérateur hermitien \(\LL\) est symétrique négatif.
  \end{prop}

  \begin{proof}
    Pour tous \(\vu,\vv \in (\mathcal C^\infty(\Gamma))^3\)
    \begin{align*}
      \int_\Gamma \vu\cdot \LL(\conj{\vv}) &= \int_\Gamma \conj{\vv}\cdot \LL(\vu)
      \\
      \int_\Gamma \vu\cdot \LL(\conj{\vu}) &= -\norm{\vdivs{\vu}}^2 - \norm{\vrots\vu}^2 \le 0
    \end{align*}
  \end{proof}

  \begin{prop}
    \label{prop:unicite:injectif:operateur:L}
    Soit \(\vu \in V=(\mathcal C^\infty(\Gamma))^3, a_0 \in \CC, a_1 \in \CC\).
    
    Soit \(\mathcal{P}\) l'opérateur tel que \(\mathcal{P}\vu = (a_0\oI + a_1 \LL)\vu\).

    Si \(\Re(a_0)\ge 0\) et \(\Re(a_1)\le 0\) alors l'opérateur \(\mathcal{P}\) est injectif sur \(V\).
  \end{prop}
  \begin{proof}
    Soit \(\vu \in \Ker{\mathcal{P}}\). Donc \(\mathcal{P}\vu  = 0\) ce qui implique,
    \begin{align*}
      \int_\Gamma \mathcal{P}\vu\cdot\conj{\vu}  &= 0
      \\
      & = \int_\Gamma (a_0\oI + a_1 \LL)\vu\cdot\conj{\vu}
      \\
      & = a_0 \norm{{\vu}}^2 - a_1 \left(\norm{\vdivs{\vu}}^2 + \norm{\vrots{\vu}}^2\right)
      \intertext{Donc}
      \Re\left(\int_\Gamma \mathcal{P}\vu\cdot\conj{\vu}\right) &= \Re(a_0) \norm{{\vu}}^2 - \Re(a_1) \left(\norm{\vdivs{\vu}}^2 + \norm{\vrots{\vu}}^2\right)
    \end{align*}
    Or \(\Re(a_0)\ge 0\) et \(\Re(a_1)\le 0\), donc tous les termes de l'expression précédente sont nuls donc \(\norm{\vu}=0\) donc \(\vu=0\).
    Donc l'opérateur \(\mathcal{P}\) est injectif.
  \end{proof}
  
  \begin{prop}
    \label{prop:unicite:inversible:operateur:L}
    Soit \(\vu \in V=\Sobolev[div]{(\Gamma)}\cap\Sobolev[rot]{(\Gamma)}\).
    
    Soit \(\mathcal{P}\) l'opérateur tel que \(\mathcal{P}\vu = (a_0\oI + a_1 \LL)\vu\).

    Si \(\Re(a_0)\ge 0\) et \(\Re(a_1)\le 0\) alors l'opérateur \(\mathcal{P}\) est bijectif sur \(V\).
  \end{prop}
  \begin{proof}
    D'après l'alternative de Fredholm (voir \cite[Théorème~VI.6, p.~92]{brezis_analyse_1996}), si l'opérateur \(\LL\) est compact, alors \(\mathcal{P}\) est soit non-injectif soit bijectif. D'après la proposition \ref{prop:unicite:injectif:operateur:L}, comme \(\Re(a_0)\ge 0\) et \(\Re(a_1)\le 0\) alors il est injectif.
    Montrons que \(\LL\) est compact.
  \end{proof}


  On rappelle que l'on veut trouver des conditions permettant de garantir \eqref{eq:unicite:form_var:cgu} soit \(\Re(X)\ge0\) où \(X = \int_\Gamma \vJ \cdot \conj{\vE_t}\) sachant que \gls{phy-J} est la trace tangentielle sur \(\Gamma\) de l’excitation magnétique (\(\vJ = \vn \pvect \vH\)).

  Ces conditions garantissent que \((\vE,\vH) = (0,0)\) est l'unique solution de 
  \begin{equation}
    \label{eq:unicite:probleme_sans_ci}
    \begin{aligned}
      \left\lbrace
      \begin{aligned}
        \vrot{\vE} + ik_0\vH &= 0
        \\
        \vrot{\vH} - ik_0\vE &= 0
      \end{aligned}
      \right. && \text{dans \(\OO^c_R\)},
      \\
      \Tr(\vE_t) = - \vn_{S_R} \pvect \vH && \text{sur \(S_R\),}
      \\
      \text{une condition limite définit ci-après }  && \text{sur \(\Gamma\).}
    \end{aligned}
  \end{equation}

  %%%%%%%%%%%%%%%%%%%%%%%%%%%%%%%%%%%%%%%%%%%%%%%%%%
  \subsection{CSU de la CI0}
    Considérons la condition d’impédance de Leontovich, la \hyperlink{ci0}{CI0} caractérisée par
    \begin{align}
      \label{eq:unicite:ci0}
      \vE_t = a_0 \vJ &&  a_0 \in \CC
    \end{align}

    \begin{defn}
      \label{def:csu:ci0}
      On définit le sous-espace fermé de \(\CC\)
      \begin{equation*}
        \CSU{CI0} = \lbrace a_0 \in \CC, \Re(a_0) \ge 0 \rbrace
      \end{equation*}
    \end{defn}

    \begin{prop}[Une CSU pour la CI0]
      \label{prop:csu:ci0}
      Si
      \begin{equation*}
        a_0 \in \CSU{CI0}
      \end{equation*}
      alors le problème \eqref{eq:unicite:probleme_sans_ci} avec la CI0 a une unique solution.
    \end{prop}
    \begin{proof}
      Cela découle de \( X = \conj{a_0}\norm{\vJ}^2\) donc \(\Re(X) = \Re(a_0)\norm{\vJ}^2 \).
    \end{proof}
  %%%%%%%%%%%%%%%%%%%%%%%%%%%%%%%%%%%%%%%%%%%%%%%%
  \subsection{CSU de la CI01}
    Considérons la condition d’impédance \hyperlink{ci01}{CI01}:
    \begin{align}
      \label{eq:unicite:ci01}
      \vE_t = (a_0\oI + a_1 \LL)\vJ && \forall (a_0, a_1) \in \CC^2
    \end{align}

    \begin{defn}
      \label{def:csu:ci01}
      On définit le sous-espace fermé de \(\CC^2\)
      \begin{equation*}
        \CSU{CI01} = \left\lbrace (a_0,a_1) \in \CC^2,
        \begin{matrix}
        \Re\left(a_0\right) \ge 0
        \\
        \Re\left(a_1\right) \le 0
        \end{matrix}
        \right\rbrace
      \end{equation*}
    \end{defn}

    \begin{prop}[Une CSU pour la CI01]
      \label{prop:csu:ci01}
      Si
      \begin{equation*}
        (a_0,a_1) \in \CSU{CI01}
      \end{equation*}
      alors le problème \eqref{eq:unicite:probleme_sans_ci} avec la CI01 a une unique solution.
    \end{prop}
    \begin{proof}
      On a
      \begin{align*}
        X & = \conj{a_0} \norm{\vJ} ^2 - \conj{a_1} \left(\norm{\vdivs{\vJ}}^2 + \norm{\vrots{\vJ}}^2\right)
        \intertext{donc}
        \Re(X) & = \Re{(a_0)} \norm{\vJ} ^2 - \Re{(a_1)}\left(\norm{\vdivs{\vJ}}^2 + \norm{\vrots{\vJ}}^2\right)
      \end{align*}
      Si l’on suppose \((a_0,a_1) \in \CSU{CI01}\), tous les termes sont positifs donc \(\Re(X)\ge 0\).
    \end{proof}

  %%%%%%%%%%%%%%%%%%%%%%%%%%%%%%%%%%%%%%%%%%%%%%%%
  \subsection{CSU de la CI1}

    Considérons la condition d’impédance \hyperlink{ci1}{CI1}:
    \begin{align}
    \label{eq:unicite:ci1}
      (\oI + b \LL) \vE_t = (a_0\oI + a_1 \LL) \vJ && \forall (a_0, a_1,b) \in \CC^3
    \end{align}

    Pour cette CIOE, nous observons qu'il y plusieurs sous-espaces différents, inclus dans \(\CC^3\) qui conduisent à l'unicité de la solution du problème.

    %\subsubsection{CSU de \cite{stupfel_sufficient_2011}}

    On définit 
    \begin{equation}
      \label{eq:unicite:delta}
      \begin{matrix}
        \Delta: & \CC^3 &\rightarrow& \CC
        \\
        & (a_0,a_1,b) & \mapsto & a_1 - a_0\conj{b}.
      \end{matrix}
    \end{equation}
    Par abus de notation, on omet les variables \((a_0,a_1,b)\)
    \begin{equation}
       \Delta(a_0,a_1,b) \equiv \Delta.
    \end{equation}

    On rappelle la CSU énoncée dans \cite{stupfel_sufficient_2011}.

    \begin{defn}
      \label{def:csu:ci1-1}

      On définit le sous-espace ouvert de \(\CC^3\)
      \begin{equation*}
        \CSU[1]{CI1} = \left\lbrace 
        (a_0,a_1,b) \in \CC^3,
        \begin{matrix}
        \Re(\Delta) = 0
        \\
        \Im(\Delta) \not = 0
        \\
        \Im(\Delta)\Im(b) \ge 0
        \\
        \Im(\Delta )\Im(a_1\conj{a_0})\ge 0
        \end{matrix}
        \right\rbrace
      \end{equation*}
    \end{defn}

    \begin{prop}[Une première CSU pour la CI1]
      \label{prop:csu:ci1-1}
      On a 
      \begin{equation*}
        (a_0,a_1,b) \in \CSU[1]{CI1} \Rightarrow \Re(X)\ge 0. 
      \end{equation*}
      C'est à dire \((a_0,a_1,b) \in \CSU[1]{CI1}\) entraîne l'unicité de la solution du problème \eqref{eq:unicite:probleme_sans_ci} avec la CI1.
    \end{prop}

    \begin{proof}
      On utilise l'identité \(\Delta\oI = (a_1(\oI +\conj{b}\LL) - \conj{b}(a_0\oI + a_1\LL))\):
      \begin{align*}
        \Delta X &= \int_\Gamma \left(a_1(\oI +\conj{b}\LL) \vJ\right)\cdot\conj{\vE_t} - \left(\conj{b}(a_0\oI + a_1 \LL)\vJ\right)\cdot\conj{\vE_t}
        \intertext{Comme l'opérateur \(\LL\) est symétrique}
        \Delta X &= \int_\Gamma \left(a_1(\oI +\conj{b}\LL) \conj{\vE_t}\right)\cdot\vJ - \int_\Gamma \left(\conj{b}(a_0\oI + a_1 \LL)\vJ\right)\cdot\conj{\vE_t}
        \intertext{En utilisant la CI1}
        \Delta X &= \int_\Gamma \left(a_1(\conj{a_0}+\conj{a_1}\LL) \conj{\vJ}\right)\cdot\vJ - \int_\Gamma \left(\conj{b}(\oI +b \LL)\vE_t\right)\cdot\conj{\vE_t} \\
        \Delta X &= a_1\conj{a_0} \norm{ \vJ }^2 - |a_1|^2 \left(\norm{\vdivs{\vJ}}^2 + \norm{\vrots{\vJ}}^2\right) - \conj{b} \norm{ \vE_t }^2 + |b|^2\left(\norm{\vdivs{\vE_t}}^2 + \norm{\vrots{\vE_t}}^2\right)
      \end{align*}

      % On pose 
      % \begin{align*}
      %   F &= -\int_\Gamma \vJ \LL \conj{\vJ} \ge 0
      %   \\
      %   G &= -\int_\Gamma \vE_t \LL \conj{\vE_t} \ge 0
      % \end{align*}

      La partie imaginaire de \(\Delta X\) est
      \begin{align*}
        % \Re(\Delta)\Re(X) - \Im(\Delta)\Im(X) &= \Re(a_1\conj{a_0}) \norm{\vJ}^2 - \Re(\conj{b})\norm{\vE_t}^2 -|a_1|^2 F + |b|^2 G \\
        \Im(\Delta X)&= \Im(a_1\conj{a_0}) \norm{\vJ}^2 - \Im(\conj{b})\norm{\vE_t}^2
        \intertext{\(\Im(\Delta X) = \Im(\Delta)\Re(X)\) ce qui entraine si \((a_0,a_1,b)\in\CSU[1]{CI1}\) alors \(\Re(\Delta)=0, \Im(\Delta)\not=0\) d'où}
        \Im(\Delta)^2\Re(X) &= \Im(\Delta)\Im(a_1\conj{a_0}) \norm{\vJ}^2 + \Im(\Delta)\Im({b})\norm{\vE_t}^2
      \end{align*}
      Les deux autres conditions de la CSU imposent que les termes du membre de droite sont positifs donc \(\Re(X)\ge0\).
    \end{proof}

    On remarque que
    \begin{align}
      \CSU[1]{CI1} &\subset \CSU{CI01}\times\CC
      \\
      \CSU[1]{CI1}\cap(\CC^2 \times \lbrace0\rbrace) &\subsetneq (\CSU{CI01}\times\lbrace0\rbrace)
      \intertext{Comme La CIOE CI1 avec \(b=0\) est équivalente à la CI01, cela veut dire que cette CSU n'est pas idéale. 
      Plus précisément, soit \(S = \lbrace (a_0,a_1) \in \CC^2; \Re(a_1)=0 \rbrace\), on a}
      \CSU[1]{CI1}\cap(\CC^2 \times \lbrace0\rbrace) &= ((\CSU{CI01}\cap S) \times\lbrace0\rbrace) 
    \end{align}

    % \begin{lemme}
    % \label{lem:coercivite:operateur-l}
    %   Soit \(z\in \CC\), \(\vu,\vv \in \Sobolev[1]{(\Gamma)}\) et \(a\) la forme bilinéaire
    %   \begin{equation*}
    %     a(\vect{u},\vect{v}) = \int_\Gamma(\oI + z\LL)\vect{u}\cdot\conj{\vect{v}}.
    %   \end{equation*}
    %   Si \(z\in \CC\backslash \RR_+^*\) alors \(a\) est coercive sur \(\Sobolev[1]{(\Gamma)}\).
    % \end{lemme}
    % \begin{proof}
    %   On cherche à montrer que \(\exists \delta \in \RR^+, \forall \vu \in \Sobolev[1]{(\Gamma)}, |a(\vu,\vu)|^2\ge \delta\left(\norm{\vu}^2+\norm{\vdivs{\vu}}^2+\norm{\vrots{\vu}}^2\right)^2\).
    %   \begin{align*}
    %     |a(\vect{u},\vect{u})|^2 &= \left(\norm{\vu}^2-\Re(z)\left(\norm{\vdivs{\vu}}^2+\norm{\vrots{\vu}}^2\right)\right)^2 + \left(\Im(z)\left(\norm{\vdivs{\vu}}^2+\norm{\vrots{\vu}}^2\right)\right)^2
    %     \\
    %     &= \begin{bmatrix}
    %       \norm{\vu}^2
    %       &
    %       \norm{\vdivs{\vu}}^2+\norm{\vrots{\vu}}^2
    %     \end{bmatrix}
    %     \begin{bmatrix}
    %       1 & - \Re(z)
    %       \\
    %       -\Re(z) & |z|^2
    %     \end{bmatrix}
    %     \begin{bmatrix}
    %       \norm{\vu}^2
    %       \\
    %       \norm{\vdivs{\vu}}^2+\norm{\vrots{\vu}}^2
    %     \end{bmatrix}
    %     \\
    %   \end{align*}
    %   Cette matrice est positive. Mais elle n'est définie que si \((\Im(z))^2\not=0\). Dans ce cas, on a la coercivité \(\Sobolev[1]{(\Gamma})\).

    %   Si \(\Im(z)=0\) alors
    %   \begin{align*}
    %     a(\vect{u},\vect{u}) &= \norm{\vu}^2-\Re(z)\left(\norm{\vdivs{\vu}}^2+\norm{\vrots{\vu}}^2\right)
    %     \\
    %     &\ge \min(1,-\Re(z))\left(\norm{\vu}^2+\norm{\vdivs{\vu}}^2+\norm{\vrots{\vu}}^2\right).
    %   \end{align*}
    %   Il suffit que \(\Re(z) < 0 \) pour avoir la coercivité
    % \end{proof}

    \begin{defn}
      \label{def:csu:ci1-2}

      On définit le sous-espace fermé de \(\CC^3\)
      \begin{equation*}
        \CSU[2]{CI1} = \left\lbrace
        (a_0,a_1,b) \in \CC^3,
        \begin{matrix}
        \Re(b) \le 0
        \\
        \Re\left(a_0\right) \ge 0
        \\
        \Re\left(b\conj{a_0}+\conj{a_1}\right) \le 0
        \\
        \Re\left(b\conj{a_1}\right) \ge 0
        \end{matrix}
        \right\rbrace
      \end{equation*}
    \end{defn}

    \begin{prop}[Une deuxième CSU pour la CI1]
      \label{prop:csu:ci1-2}
      On a 
      \begin{equation*}
        (a_0,a_1,b) \in \CSU[2]{CI1} \Rightarrow \Re(X)\ge 0. 
      \end{equation*}
      C'est à dire \((a_0,a_1,b) \in \CSU[2]{CI1}\) entraîne l'unicité de la solution du problème \eqref{eq:unicite:probleme_sans_ci} avec la CI1.
    \end{prop}

    \begin{proof}
      Comme on suppose \(\Re(b)\le 0\) donc l'opérateur \(\oI + b\LL\) est injectif d'après la propriété \ref{prop:unicite:injectif:operateur:L}.

      Il existe donc \(\vect{D}\) tel que
      \begin{align*}
        (\oI + b \LL)^{-1}\vJ &= \vect{D}.
      \end{align*}
      Donc 
      \begin{align*}
        (\oI + b \LL)\vE_t &= (a_0\oI + a_1\LL)\vJ,
        \\
        (\oI + b \LL)\vE_t &= (a_0\oI + a_1\LL)(\oI + b \LL)\vect{D},
        \\
        0 &= (\oI + b \LL)(\vE_t -  (a_0\oI + a_1\LL)\vect{D}).
        \intertext{L'opérateur \(\oI + b \LL\) est injectif donc}
        \vE_t &= (a_0\oI + a_1\LL)\vect{D},
        \\
        \int_\Gamma \vJ \cdot \conj{\vE_t} &= \int_\Gamma \vJ \cdot (\conj{a_0}\oI + \conj{a_1}\LL)\conj{\vect{D}}
        \intertext{On rappelle \(\vJ = (\oI + b \LL)\vect{D}.\)},
        \\
        X &= \int_\Gamma (\oI + b \LL)\vect{D} \cdot (\conj{a_0}\oI + \conj{a_1}\LL)\conj{\vect{D}}.
        \\
        \intertext{Or on sait d'après la définition de l'opérateur \(\LL\) }
        X &= \conj{a_0}\norm{\vect{D}}^2 - (b\conj{a_0}+\conj{a_1})\left(\norm{\vdivs{\vect{D}}}^2+\norm{\vrots{\vect{D}}}^2\right) + b\conj{a_1} \norm{\LL\vect{D}}^2.
      \end{align*}
    \end{proof}

    On remarque que
    \begin{align}
      \CSU[2]{CI1} & \subset \CSU{CI01}\times\lbrace0\rbrace
      \\
      \CSU[2]{CI1}\cap(\CC^2 \times \lbrace0\rbrace) &= (\CSU{CI01}\times\lbrace0\rbrace)
    \end{align}

    Pour des fonctions infiniment régulières, la CIOE CI1 avec \(b=0\) est équivalente à la CI01 et la \CSU[2]{CI1} est donc meilleure que la  \CSU[1]{CI1}.

\subsection{CSU pour la CI4}
  Soit la CIOE que l'on nomme \hyperlink{ci4}{CI4} :
  \begin{equation}
    \label{eq:unicite:ci4:ci4}
    \vE_t = (a_0\oI + a_1 \LD - a_2 \LR ) \vJ
  \end{equation}

  \begin{prop}
    Des CSU sont
    \begin{align}
      \Re(a_0) \ge 0
      \\
      \Re(a_1) \le 0
      \\
      \Re(a_2) \le 0
    \end{align}
  \end{prop}

  \begin{prop}
    On pose:
    \begin{align*}
      F &:= ||\vJ|| ^2 \ge 0  & G_1 &:= \int_\Gamma \vJ\cdot \LD\conj{\vJ} ds \le 0 & G_2 &:= \int_\Gamma \vJ\cdot \LR\conj{\vJ} ds \ge 0
    \end{align*}

    On a alors
    \begin{equation*}
      X = \conj{a_0}F + \conj{a_1}G - \conj{a_2}G
    \end{equation*}

    De \eqref{eq:unicite:form_var:cgu}, on déduit des CSU suivantes:
    \begin{align}
      \Re\left(a_0\right) \ge 0
      \\
      \Re\left(a_1\right) \le 0
      \\
      \Re\left(a_2\right) \le 0
    \end{align}
  \end{prop}

  On remarque que ces CSU redonnent les CSU de la CI01 quand \(a_1=a_2\). Ce jeu semble être le plus naturel, sans être trop contraignant.

\section{CSU pour la CIOE CI3 de \cite{aubakirov_electromagnetic_2014}}

  Soit la CIOE énoncé dans \cite{aubakirov_electromagnetic_2014} que l'on nomme \hyperlink{ci3}{CI3} :
  \begin{equation}
    \label{eq:unicite:ci3:ci3}
    ( \oI + b_1 \LD - b_2 \LR)\vE_t = (a_0\oI + a_1 \LD - a_2 \LR ) \vJ
  \end{equation}

  \begin{defn}
    On rappelle les expressions des opérateurs \gls{ope-LD} et \gls{ope-LR} pour des vecteurs tangents \(\vect U,\vect V \in (\mathcal{C}^\infty(\CC,\Gamma))^3\): 
    \begin{align*}
      \LD(\vect U) &= \tgrads \tdivs \vect U\\
      \LR(\vect V) &= \trots( \vn ( \vn \cdot \trots \vect V))\\
    \end{align*}
  \end{defn}

  \begin{prop}
    Par définition, \(\LD\) est antisymétrique négatif et \(\LR\) antisymétrique positif.
  \end{prop}

  \begin{prop}
    Soit \(\OO\) un domaine borné de \(\RR^3\) , de surface \(\Gamma\) fermée et régulière, où \(\vect n\) y est la normale unitaire
    sortante
    \begin{equation}
      \begin{matrix}
        \forall \vect U \in (\mathcal{C}^\infty(\CC,\Gamma))^3 ,& \LR(\LD(\vect U)) = \LD(\LR(\vect U)) = 0
      \end{matrix}
    \end{equation}
  \end{prop}
  \begin{proof}

    Soient un vecteur \textbf{tangent} \(\vect U \in (\mathcal{C}^\infty(\CC,\Gamma)^3)\). 

    Montrons que \(\LR\LD = 0\).
    D’après \cite[p.~1029, A3.42]{bladel_electromagnetic_2007}, \(\vn \cdot \trots\tgrads f = 0\)
    \begin{align*}
      \LR(\LD \vect U)  &= \trots \left(\vn \left(\vn \cdot \trots \left( \tgrads \left(\tdivs \vect U\right)\right)\right)\right) \\
      &= 0
    \end{align*}
    Montrons que \(\LD\LR = 0\).
    D’après \cite[p.~1029, A3.43]{bladel_electromagnetic_2007}, \(\tdivs \trots (f\vn) = 0\).
    \begin{align*}
      \LD(\LR \vect U) &= \tgrads \tdivs \trots (\vn (\vn \cdot \trots \vect U)) \\
      &= 0
    \end{align*}
  \end{proof}
  % Une relation importante qui découle des propriétés des opérateurs différentiels surfacique \secref{eq:op-LD-LR:prop:LDLR0} est :

  % \begin{equation}
  % \int_\Gamma \LD(\vect U) \cdot \LR(\vect V) ds = 0 , \forall \vect U, \vect V \in (H^1(\OO))^3
  % \end{equation}

  % Cette relation \Gamma'exprime sous forme forte par \(\LD\LR\equiv0\). Elle est là aussi symétrique entre les deux opérateurs.

\subsection{CSU de Stupfel}

  \begin{prop}
    Soit \(\Delta_i = a_i-\conj{b_i}a_0\), \(i=1,2\). Des CSU sont
    \begin{align}
      \Re\left(a_0\conj{a_1}\Delta_1\right) \ge 0 \\
      \Re\left(\frac{\conj{b_1}}{\Delta_1}\right) \le 0 \\
      \Re\left(\conj{a_0}a_2\left(\frac{\conj{b_2}}{\Delta_2}-\frac{\conj{b_2}}{\Delta_2}\right) + \frac{\conj{a_2}a_1}{\Delta_1} \right)\le 0\\
      \Re\left(2\Re(b_2)\frac{\conj{b_1}}{\Delta_1}-\frac{\conj{b_2}^2}{\Delta_2}\right) \ge 0\\
      \Re\left(a_0\conj{a_2}\Delta_2\right) \ge 0 \\
      \Re\left(\frac{\conj{b_2}}{\Delta_2}\right) \le 0 \\
      \Re\left(\conj{a_0}a_1\left(\frac{\conj{b_1}}{\Delta_1}-\frac{\conj{b_2}}{\Delta_2}\right) + \frac{\conj{a_1}a_2}{\Delta_2} \right)\le 0\\
      \Re\left(2\Re(b_1)\frac{\conj{b_2}}{\Delta_2}-\frac{\conj{b_1}^2}{\Delta_1}\right) \ge 0\\
      \Re\left(\Delta_1\right) = 0 \\
      \Re\left(\Delta_2\right) = 0 \\
      \Re\left(\frac{\conj{b_2}}{\Delta_2}-\frac{\conj{b_1}}{\Delta_1}\right) = 0\\
    \end{align}
  \end{prop}
  
  \begin{proof}
    On prend l'expression de la CIOE \eqref{eq:unicite:ci3:ci3} et on l’intègre avec des produits scalaires judicieusement choisis.

    \begin{multline}
      \label{eq:unicite:ci3:csu3-1}
      \int_\Gamma \vJ\cdot\conj{\eqref{eq:unicite:ci3:ci3}}ds \Rightarrow
      \int_\Gamma \vJ \cdot \conj{\vE_t} ds  + \conj{b_1} \int_\Gamma \vJ\cdot \LD\conj{\vE_t} ds - \conj{b_2} \int_\Gamma \vJ \LR\conj{\vE_t} ds \\
      = \conj{a_0} \int_\Gamma |\vJ|^2ds - \conj{a_1} \int_\Gamma |\tdivs \vJ|^2 ds - \conj{a_2} \int_\Gamma |\vn \cdot \trots \vJ|^2 ds
    \end{multline}
    \begin{multline}
      \label{eq:unicite:ci3:csu3-2}
      \int_\Gamma \eqref{eq:unicite:ci3:ci3} \cdot \conj{\vE_t} ds \Rightarrow
      \int_\Gamma |\vE_t|^2 ds  - b_1 \int_\Gamma | \tdivs \vE |^2 ds - b_2 \int_\Gamma | \vn \cdot \trots \vE_t|^2 ds \\
      = a_0 \int_\Gamma \vJ\cdot \conj{\vE_t}ds + a_1 \int_\Gamma \conj{\vE_t} \LD \vJ ds - a_2 \int_\Gamma \conj{\vE_t} \cdot \LR \vJ ds
    \end{multline}
    \begin{multline}
      \label{eq:unicite:ci3:csu3-3}
      \int_\Gamma \vJ \cdot \LR ( \conj{\eqref{eq:unicite:ci3:ci3}} ) ds \Rightarrow
      \int_\Gamma \vJ \cdot \LR \conj{\vE_t} ds  - \conj{b_2} \int_\Gamma \LR \vJ \cdot \LR \conj{\vE_t} ds \\
      =  \conj{a_0} \int_\Gamma |\vn \cdot \trots \vJ|^2ds - \conj{a_2} \int_\Gamma | \LR \vJ|^2 ds
    \end{multline}
    \begin{multline}
      \label{eq:unicite:ci3:csu3-4}
      \int_\Gamma  \LR ( \eqref{eq:unicite:ci3:ci3} ) \cdot \conj{\vE_t} ds \Rightarrow
      \int_\Gamma | \vn \cdot \trots \vE_t |^2 ds  - \conj{b_2} \int_\Gamma | \LR \vE_t|^2 ds \\
      = a_0 \int_\Gamma \conj{\vE_t} \LR \vJ ds - a_2 \int_\Gamma \LR \conj{\vE_t} \cdot \LR \vJ ds
    \end{multline}
      \begin{multline}
      \label{eq:unicite:ci3:csu3-5}
      \int_\Gamma \vJ \cdot \LD ( \conj{\eqref{eq:unicite:ci3:ci3}} ) ds \Rightarrow
      \int_\Gamma \vJ \cdot \LD \conj{\vE_t} ds  + \conj{b_1} \int_\Gamma \LD \vJ \cdot \LD \conj{\vE_t} ds \\
      = - \conj{a_0} \int_\Gamma |\tdivs \vJ|^2ds + \conj{a_1} \int_\Gamma | \LD \vJ|^2 ds
    \end{multline}
    \begin{multline}
      \label{eq:unicite:ci3:csu3-6}
      \int_\Gamma  \LD ( \eqref{eq:unicite:ci3:ci3} ) \cdot \conj{\vE_t} ds \Rightarrow
      -\int_\Gamma | \tdivs \vE_t |^2 ds  + \conj{b_1} \int_\Gamma | \LD \vE_t|^2 ds \\
      = a_0 \int_\Gamma \conj{\vE_t} \LD \vJ ds + a_1 \int_\Gamma \LD \conj{\vE_t} \cdot \LD \vJ ds
    \end{multline}
    On pose alors les définitions suivantes :
    \begin{align*}
      X&:= \int_\Gamma \vJ \cdot \conj{\vE_t} ds\\
      Y_D&:= \int_\Gamma \vJ \cdot \LD \conj{\vE_t} ds
      &Y_R&:= \int_\Gamma \vJ \cdot \LR \conj{\vE_t} ds\\
      Z_D&:= \int_\Gamma \LD \vJ \cdot \LD \conj{\vE_t} ds
      &Z_R&:= \int_\Gamma \LR \vJ \cdot \LR \conj{\vE_t} ds
    \end{align*}

    Les équations \eqref{eq:unicite:ci3:csu3-1} à \eqref{eq:unicite:ci3:csu3-4} sont équivalentes au système \(M_R X_R = F_R\) où

    \begin{align*}
      M_R&:=
      \begin{bmatrix}
        1&\conj{b_1}&-\conj{b_2}&0\\
        a_0&a_1&-a_2&0\\
        0&0&1&-\conj{b_2}\\
        0&0&a_0&-a_2\\
      \end{bmatrix},\;
      X_R =
      \begin{bmatrix}
        X\\
        Y_D\\
        Y_R\\
        Z_R
      \end{bmatrix}\\
      F_R &=
      \begin{bmatrix}
        \conj{a_0} \int_\Gamma |\vJ|^2ds - \conj{a_1} \int_\Gamma |\tdivs \vJ|^2 ds - \conj{a_2} \int_\Gamma |\vn \cdot \trots \vJ|^2 ds \\
        \int_\Gamma |\vE_t|^2 ds  - b_1 \int_\Gamma | \tdivs \vE |^2 ds - b_2 \int_\Gamma | \vn \cdot \trots \vE_t|^2 ds \\
        \conj{a_0} \int_\Gamma |\vn \cdot \trots \vJ|^2ds - \conj{a_2} \int_\Gamma | \LR \vJ|^2 ds \\
        \int_\Gamma | \vn \cdot \trots \vE_t |^2 ds  - \conj{b_2} \int_\Gamma | \LR \vE_t|^2 ds
      \end{bmatrix}
    \end{align*}

    Tandis que les équations \eqref{eq:unicite:ci3:csu3-1},\eqref{eq:unicite:ci3:csu3-2},\eqref{eq:unicite:ci3:csu3-5},\eqref{eq:unicite:ci3:csu3-6} sont équivalentes au système \(M_D X_D= F_D\) où

    \begin{align*}
      M_D&:=
      \begin{bmatrix}
        1&-\conj{b_2}&\conj{b_1}&0\\
        a_0&-a_2&a_1&0\\
        0&0&1&\conj{b_1}\\
        0&0&a_0&a_1\\
      \end{bmatrix},\;
      X_D =
      \begin{bmatrix}
        X\\
        Y_R\\
        Y_D\\
        Z_D
      \end{bmatrix}\\
      F_D &=
      \begin{bmatrix}
        \conj{a_0} \int_\Gamma |\vJ|^2ds - \conj{a_1} \int_\Gamma |\tdivs \vJ|^2 ds - \conj{a_2} \int_\Gamma |\vn \cdot \trots \vJ|^2 ds \\
        \int_\Gamma |\vE_t|^2 ds  - b_1 \int_\Gamma | \tdivs \vE |^2 ds - b_2 \int_\Gamma | \vn \cdot \trots \vE_t|^2 ds \\
        -\conj{a_0} \int_\Gamma |\tdivs \vJ|^2ds + \conj{a_1} \int_\Gamma | \LR \vJ|^2 ds \\
        -\int_\Gamma | \tdivs \vE_t |^2 ds  + \conj{b_1} \int_\Gamma | \LR \vE_t|^2 ds
      \end{bmatrix},\;
    \end{align*}

    On note dans la suite \(\Delta_i = a_i-\conj{b_i}a_0\), \(i=1,2\). On suppose que ces système aient une unique solution. Alors on obtient la première condition suffisante:

    \begin{equation}
      \label{eq:unicite:ci3:csu3-cn-det}
      \Delta_1\Delta_2 \not = 0
    \end{equation}

    \begin{minipage}{0.49\textwidth}
      \textbf{Cas LR}:
      \begin{align}
        \label{eq:unicite:ci3:csu3r-j2}&\Re\left(a_0\conj{a_2}\Delta_2\right) \ge 0 \\
        \label{eq:unicite:ci3:csu3r-e2}&\Re\left(\frac{\conj{b_2}}{\Delta_2}\right) \le 0 \\
        \label{eq:unicite:ci3:csu3r-jdj}&\Re\left(\conj{a_0}a_1\left(\frac{\conj{b_1}}{\Delta_1}-\frac{\conj{b_2}}{\Delta_2}\right) + \frac{\conj{a_1}a_2}{\Delta_2} \right)\le 0\\
        \label{eq:unicite:ci3:csu3r-ede}&\Re\left(2\Re(b_1)\frac{\conj{b_2}}{\Delta_2}-\frac{\conj{b_1}^2}{\Delta_1}\right) \ge 0\\
        \label{eq:unicite:ci3:csu3r-jrj}&\Re\left(|a_2|^2\Delta_2\right) \le 0 \\
        \label{eq:unicite:ci3:csu3r-ere}&\Re\left(|b_2|^2\Delta_2\right) \ge 0 \\
        \label{eq:unicite:ci3:csu3r-rj2}&\Re\left(|a_1|^2\left(\frac{\conj{b_1}}{\Delta_1}-\frac{\conj{b_2}}{\Delta_2}\right)\right)\ge 0\\
        \label{eq:unicite:ci3:csu3r-re2}&\Re\left(|b_1|^2\left(\frac{\conj{b_1}}{\Delta_1}-\frac{\conj{b_2}}{\Delta_2}\right)\right)\le 0
      \end{align}
      \eqref{eq:unicite:ci3:csu3r-jrj} et \eqref{eq:unicite:ci3:csu3r-ere} impliquent :
      \begin{equation}
        \Re\left(\Delta_2\right) = 0\\\
      \end{equation}
      \eqref{eq:unicite:ci3:csu3r-rj2} et \eqref{eq:unicite:ci3:csu3r-re2} impliquent :
      \begin{equation}
        \Re\left(\frac{\conj{b_1}}{\Delta_1}-\frac{\conj{b_2}}{\Delta_2}\right) = 0\\\
      \end{equation}
    \end{minipage}
    \begin{minipage}{0.49\textwidth}
      \textbf{Cas LD}:
      \begin{align}
        \label{eq:unicite:ci3:csu3d-j2}&\Re\left(a_0\conj{a_1}\Delta_1\right) \ge 0 \\
        \label{eq:unicite:ci3:csu3d-e2}&\Re\left(\frac{\conj{b_1}}{\Delta_1}\right) \le 0 \\
        \label{eq:unicite:ci3:csu3d-jrj}&\Re\left(\conj{a_0}a_2\left(\frac{\conj{b_2}}{\Delta_2}-\frac{\conj{b_2}}{\Delta_2}\right) + \frac{\conj{a_2}a_1}{\Delta_1} \right)\le 0\\
        \label{eq:unicite:ci3:csu3d-ere}&\Re\left(2\Re(b_2)\frac{\conj{b_1}}{\Delta_1}-\frac{\conj{b_2}^2}{\Delta_2}\right) \ge 0\\
        \label{eq:unicite:ci3:csu3d-jdj}&\Re\left(|a_1|^2\Delta_1\right) \le 0 \\
        \label{eq:unicite:ci3:csu3d-ede}&\Re\left(|b_1|^2\Delta_1\right) \ge 0 \\
        \label{eq:unicite:ci3:csu3d-dj2}&\Re\left(|a_2|^2\left(\frac{\conj{b_2}}{\Delta_2}-\frac{\conj{b_1}}{\Delta_1}\right)\right)\ge 0\\
        \label{eq:unicite:ci3:csu3d-de2}&\Re\left(|b_2|^2\left(\frac{\conj{b_2}}{\Delta_2}-\frac{\conj{b_1}}{\Delta_1}\right)\right)\le 0
      \end{align}
      \eqref{eq:unicite:ci3:csu3d-jdj} et \eqref{eq:unicite:ci3:csu3d-ede} impliquent :
      \begin{equation}
        \Re\left(\Delta_1\right) = 0\\\
      \end{equation}
      \eqref{eq:unicite:ci3:csu3d-dj2} et \eqref{eq:unicite:ci3:csu3d-de2} impliquent :
      \begin{equation}
        \Re\left(\frac{\conj{b_1}}{\Delta_1}-\frac{\conj{b_2}}{\Delta_2}\right) = 0\\\
      \end{equation}
    \end{minipage}
  \end{proof}
  %Pour le système \(M_D X_D = F_D\), les conditions sont identiques à une permutation des indices 1 et 2 près.
  De part leur nombre, ces CSU sont très contraignantes et ne permettent pas de retrouver des CSU des CIOE d'ordres inférieurs lorsque l'on annule les coefficients \(b_1, b_2\). 


\subsection{CSU de Payen}

  \begin{prop}
    D'autre CSU sont
    \begin{align}
      \Re\left(a_0\right)\ge 0 \\
      \Re\left(a_1 - \frac{\conj{b_1a_0}a_1}{\Delta_1}\right) \le 0 \\
      \Re\left(a_2 - \frac{\conj{b_2a_0}a_2}{\Delta_2}\right) \le 0 \\
      \Re\left(b_1\Delta_1\right) = 0 \\
      \Re\left(b_2\Delta_2\right) = 0 \\
      \Im\left(b_1\Delta_1\right)\Im(b_1)\ge 0\\
      \Im\left(b_2\Delta_2\right)\Im(b_2)\ge 0
    \end{align}
  \end{prop}

  \begin{proof}
    En se basant sur la méthode précédente, on remarque que l'on peut déterminer les inconnus \((Y_R,Z_R)\) (resp. \((Y_D,Z_R)\)) uniquement en fonction des équations \eqref{eq:unicite:ci3:csu3-3} et \eqref{eq:unicite:ci3:csu3-4} (resp. \eqref{eq:unicite:ci3:csu3-5} et \eqref{eq:unicite:ci3:csu3-6}).

    On déduit donc que si \(\Delta_1 \not = 0\) et \(\Delta_2 \not = 0\) alors

    \begin{align}
      Y_R &= \frac{1}{\Delta_2}\left(a_2\left[\conj{a_0}\int_\Gamma \vJ\cdot\LR\conj{\vJ} - \conj{a_2}||\LR J||^2\right]  -\conj{b_2}\left[\int_\Gamma \conj{\vE}\LR{\vE} - b_2 ||\LR \vE ||^2\right]\right) \\
      Y_D &= \frac{1}{\Delta_1}\left(a_1\left[\conj{a_0}\int_\Gamma \vJ\cdot\LD\conj{\vJ} + \conj{a_1}||\LD J||^2\right]  -\conj{b_1}\left[\int_\Gamma \conj{\vE}\LD{\vE} + b_1 ||\LD \vE ||^2\right]\right)
    \end{align}

    Il reste alors à utiliser l'équation \eqref{eq:unicite:ci3:csu3-1} pour obtenir
    \begin{equation}
      X = -\conj{b_1} Y_D + \conj{b_2} Y_R + \conj{a_0} || \vJ ||^2 + \conj{a_1} \int_\Gamma \vJ \cdot \LD \conj{\vJ} - \conj{a_2} \int_\Gamma \vJ \cdot \LR \conj{\vJ}
    \end{equation}

    \begin{multline}
      X = \conj{a_0} || \vJ ||^2 - \conj{a_1} || \vdivs \vJ ||^2 - \conj{a_2} || \vrots \vJ ||^2
      \\
      + \frac{\conj{b_2}}{\Delta_2}\left(a_2\left(\conj{a_0}||\vrots \vJ||^2 - \conj{a_2}||\LR J||^2\right)  -\conj{b_2}\left(||\vrots\vE||^2 - b_2 ||\LR \vE ||^2\right)\right)
      \\
      - \frac{\conj{b_1}}{\Delta_1}\left(a_1\left(-\conj{a_0}||\vdivs\vJ||^2 + \conj{a_1}||\LD J||^2\right)  -\conj{b_1}\left(-||\vdivs\vE||^2 + b_1 ||\LD \vE ||^2\right)\right)
    \end{multline}

    On factorise les termes en \(\vJ\)

    \begin{multline}
      X = \conj{a_0} || \vJ ||^2 - \left(a_1 - \frac{\conj{b_1a_0}a_1}{\Delta_1}\right) || \vdivs \vJ ||^2 - \left(a_2 - \frac{\conj{b_2a_0}a_2}{\Delta_2}\right) || \vrots \vJ ||^2
      \\
      + \frac{\conj{b_2}}{\Delta_2}\left( - |a_2|^2||\LR \vJ||^2  - \conj{b_2}\left(||\vrots\vE||^2 - b_2 ||\LR \vE ||^2\right)\right) 
      \\
      - \frac{\conj{b_1}}{\Delta_1}\left( |a_1|^2||\LD \vJ||^2  - \conj{b_1}\left(-||\vdivs\vE||^2 + b_1 ||\LD \vE ||^2\right)\right)
    \end{multline}

    On développe tous les termes
    \begin{multline}
      X = \conj{a_0} || \vJ ||^2 - \left(a_1 - \frac{\conj{b_1a_0}a_1}{\Delta_1}\right) || \vdivs \vJ ||^2 - \left(a_2 - \frac{\conj{b_2a_0}a_2}{\Delta_2}\right) || \vrots \vJ ||^2
      \\
      - \frac{\conj{b_2}|a_2|^2}{\Delta_2}||\LR \vJ||^2  -  \frac{\conj{b_2}^2}{\Delta_2}||\vrots\vE||^2 +  \frac{|b_2|}{\Delta_2} ||\LR \vE ||^2
      \\
      - \frac{\conj{b_1}|a_1|^2}{\Delta_1}||\LD \vJ||^2  - \frac{\conj{b_1}^2}{\Delta_1}||\vdivs\vE||^2 + \frac{|b_1|^2}{\Delta_1} ||\LD \vE ||^2
    \end{multline}

    On impose alors à la partie réelle de chaque terme d'être positive, et on obtient les CSU suivantes :

    \begin{equation}
      \Re\left(a_0\right)\ge 0
    \end{equation}
    \begin{minipage}{0.5\textwidth}
      \begin{align}
        \Re\left(a_1 - \frac{\conj{b_1a_0}a_1}{\Delta_1}\right) \le 0 \\
        \Re\left(a_2 - \frac{\conj{b_2a_0}a_2}{\Delta_2}\right) \le 0 \\
        \Re\left(\frac{|a_1|^2\conj{b_1}}{\Delta_1}\right) \le 0 \\
        \Re\left(\frac{|a_2|^2\conj{b_2}}{\Delta_2}\right) \le 0
      \end{align}
    \end{minipage}
    \begin{minipage}{0.5\textwidth}
      \begin{align}
        \Re\left(\frac{\conj{b_1}^2}{\Delta_1}\right) \le 0 \\
        \Re\left(\frac{\conj{b_2}^2}{\Delta_2}\right) \le 0 \\
        \Re\left(\frac{|b_1|^2\conj{b_1}}{\Delta_1}\right) \ge 0 \\
        \Re\left(\frac{|b_2|^2\conj{b_2}}{\Delta_2}\right) \ge 0
      \end{align}
    \end{minipage}

    On remarque alors que certaines CSU peuvent se combiner et imposent que les parties réelles de \(b_1\Delta_1,b_2\Delta_2\) soient nulles.

    \begin{align}
      \Re\left(a_0\right)\ge 0 \\
      \Re\left(a_1 - \frac{\conj{b_1a_0}a_1}{\Delta_1}\right) \le 0 \\
      \Re\left(a_2 - \frac{\conj{b_2a_0}a_2}{\Delta_2}\right) \le 0 \\
      \Re\left(b_1\Delta_1\right) = 0 \\
      \Re\left(b_2\Delta_2\right) = 0 \\
      \Im\left(b_1\Delta_1\right)\Im(b_1)\ge 0\\
      \Im\left(b_2\Delta_2\right)\Im(b_2)\ge 0
    \end{align}
  \end{proof}

  On a réussi à réduire le nombre de CSU et en plus, fixer \(b_1=b_2=0\) permet de retomber sur les CSU de la CI4.

\subsection{CSU de Lafitte-Stupfel}

  \begin{prop}
    Soit \(z = \left(1 - \frac{b_1a_0}{a_1} - \frac{b_2a_0}{a_2}\right) \). Des CSU qui assurent la \gls{acr-cgu} sont
    \begin{align}
      \Re\left(\conj{a_0}z\right) \ge 0
      \\
      \Re\left(\conj{a_1}z\right) \le 0
      \\
      \Re\left(\conj{a_2}z\right) \le 0
      \\
      \Re\left(\frac{b_1}{a_1}\right) \ge 0
      \\
      \Re\left(\frac{b_2}{a_2}\right) \ge 0
      \\
      \Re\left(a_0\right) \ge 0
      \\
      \Re\left(a_1\right) \le 0
      \\
      \Re\left(a_2\right) \le 0
      \\
      \Re\left(\frac{b_1\conj{a_2}}{a_1\conj{a_0}}\right) \le 0
      \\
      \Re\left(\frac{b_2\conj{a_1}}{a_2\conj{a_0}}\right) \le 0
    \end{align}
  \end{prop}
  \begin{proof}
    Par définition de la CIOE, on a

    \begin{align}
      X &= \int_\Gamma \left(a_0\oI + a_1 \LD - a_2 \LR \right)^{-1}\left(\oI + b_1 \LD - b_2 \LR \right) \vE_t\cdot \conj{\vE_t}
    \end{align}

    On développe simplement chaque terme

    \begin{multline}
      X = \int_\Gamma \left(a_0\oI + a_1 \LD - a_2 \LR \right)^{-1}
      \\
      + b_1 \left(a_0\oI + a_1 \LD - a_2 \LR \right)^{-1}\LD
      \\
      \left.
      - b_2 \left(a_0\oI + a_1 \LD - a_2 \LR \right)^{-1}\LR \right) \vE_t\cdot \conj{\vE_t}
    \end{multline}

    L'astuce pour obtenir réside dans les égalités suivantes, valables si \(a_1\) et \(a_2\) sont non-nuls.
    \begin{align}
      \LD & = \frac{a_0 + a_1 \LD - a_0}{a_1}
      \\
      \LR & = -\frac{a_0 - a_2 \LD - a_0}{a_2}
    \end{align}

    On pose
    \begin{equation}
      z = \left(1 - \frac{b_1a_0}{a_1} - \frac{b_2a_0}{a_2}\right)
    \end{equation}

    On déduit de ce qui précède que

    \begin{multline}
      X = \int_\Gamma z\left(a_0\oI + a_1 \LD - a_2 \LR \right)^{-1}
      \\
      + \frac{b_1}{a_1} \left(a_0\oI + a_1 \LD - a_2 \LR \right)^{-1}\left(a_0+a_1\LD\right)
      \\
      - \frac{b_2}{a_2} \left(a_0\oI + a_1 \LD - a_2 \LR \right)^{-1}\left(a_0-a_2\LR\right) \vE_t\cdot \conj{\vE_t}
    \end{multline}

    On définit

    \newcommand{\vD}{\vect{D}}
    \newcommand{\vF}{\vect{F}}

    \begin{align}
      \vD & = \left(a_0 \oI + a_1 \LD - a_2\LR \right)^{-1} \vE_t
      \\
      \vF_1 & = \left(\oI - a_2 \left( a_0 + a_1\LD\right)^{-1}\LR\right)^{-1} \vE_t
      \\
      \vF_2 & = \left(\oI + a_1 \left( a_0 - a_2\LR\right)^{-1}\LD\right)^{-1} \vE_t
    \end{align}

    Alors immédiatement, on a

    \begin{multline}
      X = \int_\Gamma z \vD \cdot \left(\conj{a_0} \oI + \conj{a_1} \LD - \conj{a_2}\LR\right)\conj{\vD}
      \\
      + \frac{b_1}{a_1} \left(\oI - \conj{a_2} \left( \conj{a_0} + \conj{a_1}\LD\right)^{-1}\LR\right)\conj{\vF_1}\cdot\vF_1
      \\
      + \frac{b_2}{a_2} \left(\oI + \conj{a_1} \left( \conj{a_0} - \conj{a_2}\LR\right)^{-1}\LD\right)\conj{\vF_2}\cdot\vF_2
    \end{multline}

    Finalement posons

    \newcommand{\vG}{\vect{G}}

    \begin{align}
      \vG_1 & = \left(\conj{a_0} \oI + \conj{a_1} \LD \right)^{-1}\LR \conj{\vF_1}
      \\
      \vG_2 & = \left(\conj{a_0} \oI - \conj{a_2} \LR \right)^{-1}\LD \conj{\vF_2}
    \end{align}

    Puisque \(\LD\) (resp. \(\LR\)) commute avec lui-même, on a les égalités suivantes

    \begin{align}
      \LD\left(\conj{a_0} \oI + \conj{a_1} \LD \right)&=\left(\conj{a_0} \oI + \conj{a_1} \LD \right)\LD
      \\
      \LR\left(\conj{a_0} \oI - \conj{a_2} \LR \right)&=\left(\conj{a_0} \oI - \conj{a_2} \LR \right)\LR
    \end{align}

    Or on a démontré que \(\LD\LR=\LR\LD=0\), et ainsi

    \begin{align}
      \LD\LR\conj{\vF_1} &= \LD\left(\conj{a_0} \oI + \conj{a_1} \LD \right)\vG_1
      \\
      0 & =\left(\conj{a_0} \oI + \conj{a_1} \LD \right)\LD\vG_1
    \end{align}

    Si l'on suppose alors que \(\Re(a_0) \ge 0 \) et \(\Re(a_1) \le 0\) (resp. \(\Re(a_2)\le0\)), alors \(\left(\conj{a_0} \oI + \conj{a_1} \LD \right)\) (resp. \(\left(\conj{a_0} \oI - \conj{a_2} \LR \right)\)) est injectif et donc on déduit que

    \begin{align}
      \LD\vG_1 = 0
      \\
      \LR\vG_2 = 0
    \end{align}

  % \begin{TODO}
  %   En fait plus largement, il faut que \(\Re(a_0)\) et \(\Re(a_{1/2})\) soit de signes opposées. Mais il suffit d'avoir une des deux.
  % \end{TODO}

    Or par définition \(\LR\conj{\vF_1} = \left(\conj{a_0} \oI + \conj{a_1} \LD \right)\vG_1\) (resp. \(\LR\conj{\vF_2} = \left(\conj{a_0} \oI - \conj{a_2} \LR \right)\vG_2\)) donc
    \begin{align}
      \LR\conj{\vF_1} &= \conj{a_0}\vG_1
      \\
      \LD\conj{\vF_2} &= \conj{a_0}\vG_2
    \end{align}

    On réinjecte ce résultat dans la définition de \(X\)

    \begin{multline}
      X = \int_\Gamma z \vD \cdot \left(\conj{a_0} \oI + \conj{a_1} \LD - \conj{a_2}\LR\right)\conj{\vD}
      \\
      + \frac{b_1}{a_1} ||\vF_1||^2 - \frac{b_1\conj{a_2}}{a_1\conj{a_0}} \LR\conj{\vF_1}\cdot\vF_1
      \\
      + \frac{b_2}{a_2} ||\vF_2||^2 + \frac{b_2\conj{a_1}}{a_2\conj{a_0}} \LD\conj{\vF_2}\cdot\vF_2
    \end{multline}

    Les CSU sont alors

    \begin{minipage}{0.5\textwidth}
    \begin{align}
      \Re\left(\conj{a_0}z\right) \ge 0
      \\
      \Re\left(\conj{a_1}z\right) \le 0
      \\
      \Re\left(\conj{a_2}z\right) \le 0
      \\
      \Re\left(\frac{b_1}{a_1}\right) \ge 0
      \\
      \Re\left(\frac{b_2}{a_2}\right) \ge 0
    \end{align}
    \end{minipage}
    \begin{minipage}{0.49\textwidth}
    \begin{align}
      \Re\left(a_0\right) \ge 0
      \\
      \Re\left(a_1\right) \le 0
      \\
      \Re\left(a_2\right) \le 0
      \\
      \Re\left(\frac{b_1\conj{a_2}}{a_1\conj{a_0}}\right) \le 0
      \\
      \Re\left(\frac{b_2\conj{a_1}}{a_2\conj{a_0}}\right) \le 0
    \end{align}
    \end{minipage}
  \end{proof}
\section{Calcul des coefficient des CIOE par moindres carrés sous contraintes}

\subsection{Expression des moindre carrés dans le cadre de l'approximation plan infini}
  Pour toutes les CIOE, on cherche à approcher le symbole de l'opérateur d'impédance \(\hat\mZ_{ex}(k_x,k_y)\) par une matrice \(\hat\mZ_{ap}(k_x,k_y)\)

  \begin{prop}
    SOit \(k_x,k_y\) fixés.
    Pour nos CIOE, il existe une matrice \(\mH_{k_x,k_y}(CI,\hat\mZ_{ex}(k_x,k_y))\) et un vecteur \(\vect{X}(CI)\) où CI représente un vecteur de \(\CC^n\) composés des coefficient de la CI telles que minimiser 
    \[
      ||\hat\mZ_{ap}(k_x,k_y)-\hat\mZ_{ex}(k_x,k_y)||^2
    \] 
    revient à minimiser 
    \[ 
      || \mH_{k_x,k_y}(CI,\hat\mZ_{ex}(k_x,k_y)) \vect{X}(CI) - b(\hat\mZ_{ex}(k_x,k_y)) ||^2
    \]
    Les dimensions de \(\mH\) sont de 4 lignes et le nombre de coefficients de la CI colonnes et donc le vecteur colonne b à 4 coefficients.
  \end{prop}

  \begin{proof}
    Nous démontrons ce résultat pour la \hyperlink{ci3}{CI3}, les matrices des autres CIOE s'en déduisent.

    Soient les matrices symétriques \(\hat\mLD, \hat\mLR\) telles que

    \begin{align}
      \hat\mLD(k_x,k_y) & = - \begin{bmatrix} k_x^2 & k_x k_y \\ k_x k_y & k_y^2 \end{bmatrix}
      \\
      \hat\mLR(k_x,k_y) & =  \begin{bmatrix} k_y^2 & -k_x k_y \\ -k_x k_y &  k_x^2 \end{bmatrix}
    \end{align}

    Par définition, on a
    \begin{align}
    ||\hat\mZ_{ap}-\hat\mZ_{ex}||^2 &= ||\left(\mI + b_1 \hat{\mLD} - b_2 \hat{\mLR}\right)^{-1}\left(a_0\mI + a_1 \hat{\mLD} - a_2 \hat{\mLR}\right)-\hat\mZ_{ex} ||^2
    \\
    &= || a_0\mI + a_1 \hat{\mLD} - a_2 \hat{\mLR} ||^2 + ||\hat\mZ_{ex}||^2 - 2 \left<\left(\mI + b_1 \hat{\mLD} - b_2 \hat{\mLR}\right)\hat\mZ_{ex},a_0\mI + a_1 \hat{\mLD} - a_2 \hat{\mLR} \right>
    \\
    &= ||\hat\mZ_{ex}||^2 + \left<a_0\mI + a_1 \hat{\mLD} - a_2 \hat{\mLR} -2\left(\mI + b_1 \hat{\mLD} - b_2 \hat{\mLR}\right)\hat\mZ_{ex},a_0\mI + a_1 \hat{\mLD} - a_2 \hat{\mLR} \right> 
    \end{align}
  \end{proof}
\subsection{Problème de la singularité de l'impédance dans le cadre du plan infini pour une couche de matériaux}
On se place dans le cadre du plan infini de \cite{aubakirov_electromagnetic_2014}, où \(\eps=4,\mu=1,f=12\) Ghz, et \(d=3.5\) mm. %Ce cas non physique possède un onde guidée pour \((k_x,k_y) = (k_0s^\star,0.)\) où \(\mR(k_0s^\star,0.) = \infty\).

Il existe aussi \(s_z\) tel que \(\hat\mZ_{ex}(k_0s_z,0.) = \infty\). En effet, d'après la formule pour une couche de matériau \eqref{eq:imp_plan:symb_z:1c}, 

\begin{equation}
  \hat{\mZ}_{ex}(k_x,0.) = i\frac{\eta}{k\sqrt{k^2 - k_x^2}}\tan(\sqrt{k^2 - k_x^2}d)\left(k^2\mI - \hat{\mLR}\right)
\end{equation}
Donc il est facile de voir que l'on a une asymptote à cause de la tangente et donc pour cet empilement
\begin{equation}
  s_z = \sqrt{\eps \mu - \left(\frac{\pi/2}{k_0 d}\right)^2}
\end{equation}

Dans la partie précédente nous avons introduit la fonctionnelle que l'on cherche à minimiser qui est \(F(X) = \left\lVert\tilde{\mM} X - b(\mZ_{ex})\right\rVert_{\RR^N}\).

Le problème est donc que si nous balayons en incidence et que l'on passe par ce point, il existera une valeur non défini dans la matrice. Or comme le gradient de la fonctionnelle est fonction de cette matrice, le gradient n'est pas défini pour tout \(X\). Si l'on utilise une méthode basée sur le gradient de type Newton (SLSQP), ce que nous avons fait, on comprend pourquoi la méthode numérique échoue à calculer des coefficients.

On décompose alors nos matrices et vecteurs en séparant les parties contentant cette asymptote.

On suppose donc qu'il existe \(\mZ_\infty, \tilde{b}_\infty, X_\infty,\) tels que
\begin{align*}
  \tilde{\mM} &= \tilde{\mM}_\infty + \tilde{\mM}_r
  \\
  \tilde{b} &= \tilde{b}_\infty + \tilde{b}_r
  \\
  X &= X_\infty + X_r
  \\
  \tilde{\mM}_\infty X_\infty &= \tilde{b}_\infty
\end{align*}

Il faut vraiment voir cette décomposition comme deux partie, où l'une est nulle quasiment partout sauf pour le \(s_z\) problématique et l'autre est défini normalement sauf aux termes correspondant au \(s_z\) où elle est nulle.

On rappelle l'expression de la CI3
\begin{equation}
  \hat{\mZ}_{ap}(k_x,0.) = \left(\mI + b_1 \hat{\mLD} - b_2 \hat{\mLR} \right)^{-1}\left(a_0\mI + a_1 \hat{\mLD} - a_2 \hat{\mLR} \right)
\end{equation}

Et on remarque que vu la forme de la CIOE alors
\begin{equation}
  X_\infty = \begin{bmatrix}
    0\\
    0\\
    0\\
    s_z^{-2}\\
    s_z^{-2}\\
  \end{bmatrix}
\end{equation}

Cela veut dire qu'à moins de fixer \(b_1\) et \(b_2\), on se trouvera dans le noyau de \(\tilde{\mM}_\infty\) et donc la minimisation sera fausse.

On pose donc 
\begin{equation}
  X_r = \begin{bmatrix}
  a_0\\
  a_1\\
  a_2\\
  0\\
  0\\
  \end{bmatrix}
\end{equation}

On développe donc la fonctionnelle suivant cette décomposition.
\begin{align*}
\min\left\rVert \tilde{\mM} X - \tilde{b} \right \rVert &= \min\left\rVert \left(\tilde{\mM}_\infty + \tilde{\mM}_r\right)\left( X_\infty + X_r \right) - \tilde{b}_\infty - \tilde{b}_r \right \rVert
\\
\intertext{On utilise la relation \(  \tilde{\mM}_\infty X_\infty = \tilde{b}_\infty\)}
&=  \min \left\rVert \tilde{\mM}_\infty X_r + \tilde{\mM}_r X_\infty + \tilde{\mM}_r X_r - \tilde{b}_r \right \rVert
\\
\intertext{Comme \(X_\infty\) est par définition fixé, il n'influe pas sur le minimum}
&= \min \left\rVert \tilde{\mM}_\infty X_r + \tilde{\mM}_r X_r - \tilde{b}_r \right \rVert
\\
\intertext{Enfin par définition de \(X_r\) et \(\tilde{\mM}_\infty\), leur produit est nulle}
&= \min \left\rVert \tilde{\mM}_r X_r - \tilde{b}_r \right \rVert
\end{align*}

On voit alors que l'on peut résoudre le problème si l'on minimise uniquement sur les 3 premiers coefficients, les deux derniers étant fixés et que l'on enlève du système les lignes où l'impédance n'est pas définie.

\sectionstar{Conclusion}
Nous nous sommes dotés d'une méthode numérique pour calculer les coefficients des CIOE qui permet de garantir l'unicité des solutions des équations de Maxwell. Ce calcul n'est fait que pour un empilement de matériaux et une fréquence donnée mais son coût numérique faible et la rapidité de ce dernier nous permettent de le valider pour l'intégration dans CIOE dans un code de résolution des équations de Maxwell.