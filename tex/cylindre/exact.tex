\section[Opérateur de Calderón pour un cylindre infini]{Expressions exactes des matrices d'impédance et des matrices de réflexion pour un cylindre infini}

  % On rappelle les formules des opérateurs \(\vdiv, \vrot\) en coordonnée cylindrique \((r,\theta,z)\).
  % \begin{align}
  %   \vrot \vect{V} &= \left(\frac{1}{r}\ddp{\theta}{V_z} - \ddp{z}{V_\theta}\right)\vect{e_r} +
  %   \left(\ddp{z}{V_r} - \ddp{r}{V_z}\right)\vect{e_\theta} +
  %   \frac{1}{r}\left(\ddp{r}{(rV_\theta)}-\ddp{\theta}{V_r}\right)\vect{e_z}
  %   \\
  %   \vdiv \vect{V} &= \frac{1}{r}\ddp{r}{(rV_r)}+\frac{1}{r}\ddp{\theta}{V_\theta}+\ddp{z}{V_z}
  %   \\
  %   \vgrad f &= \ddp{r}{f}\vect{e_r}
  %   +\frac{1}{r}\ddp{\theta}{f}\vect{e_\theta} + \ddp{z}{f}\vect{e_z}
  % \end{align}

  \begin{figure}[!hbt]
    \centering
    \tikzsetnextfilename{cylindre_1_couche}
    \begin{tikzpicture}
      \coordinate (mat) at (0,-1.5);
\coordinate (vide) at (0,-2);
\coordinate (c) at (0,0);

\fill [lightgray] (c) circle (2);
\fill [white] (c) circle (1.5);
\fill [pattern=north east lines] (c) circle (1.5);

\draw (c) circle (2);
\draw (c) circle (1.5);


\coordinate (n) at (0,2);

%\draw (vide) node [below] {$\eps_0,\mu_0$};
\draw (mat) node [below] {$\peps,\pmu$};

% Axess
\draw [->] (n) -- ++(0,1) node [at end, right] {$\v{\mr}$};
\draw [->] (n) -- ++(1,0) node [at end, right] {$\v{\mt}$};

\draw (n) ++(0.2,0.2) circle(0.1cm) node [above=0.1cm] {$\v{\mz}$};
\draw (n) ++(0.2,0.2) +(135:0.1cm) -- +(315:0.1cm);
\draw (n) ++(0.2,0.2) +(45:0.1cm) -- +(225:0.1cm);

%\draw [->>,thick] (lt) ++ (1,1) -- (mt) ;


    \end{tikzpicture}
  \end{figure}

  On exprime les équations de Maxwell dans le matériau en coordonnées cylindriques et sans perte de généralité, on effectue une transformée de Fourier en \(z\) par invariance en translation.
  La solution étant périodique en \(\theta\), elle est développable en série de Fourier sur cette coordonnée.

  En utilisant l'inversion de Fourier, il existe \(\hat\vE\) telle que

  \begin{equation*}
    \vE\gls{mat-xcyl} = \frac{1}{2\pi}\sum_{n=-\infty}^{\infty}\int_{\RR} e^{i(n \theta + k_z z )}\hat{\vE} (r,n,k_z) \dd{k_z}.
  \end{equation*}

  Nous rappelons que les coefficients de Fourier \(k_z\) doivent satisfaire la propriété \ref{prop:unicite:interieur:postulat:multi-couche} pour assurer l'unicité des solutions et que le champ a bien une transformée de Fourier.

  La fonction \(\hat{\vE} (r,n,k_z)\) est solution d'une \gls{acr-edo} en \(r\) que nous allons résoudre pour chaque paramètre \(n\), \(k_z\).

  \begin{defn}
    Soit \(k_3\) la fonction constante par morceaux en \(r\) telle que
    \begin{equation*}
      \fonction{k_3}{\RR_+^*}{\CC}
      {r}{\sqrt{k^2 - k_z^2}.}
    \end{equation*}
    Soit \(J_n\) la fonction de Bessel d'ordre \(n\) et \(H_n^{(2)}\) la fonction de Hankel d'ordre \(n\) de 2\ieme espèce.
  \end{defn}

  \begin{prop}
    On suppose que le matériau est homogène isotrope pour \(r_{m-1}\le r \le r_m\), donc \(\eps(r)\equiv \eps, \mu(r)\equiv \mu, k_3(r)\equiv k_3\).

    On suppose qu'il n'existe pas de paramètres \(k_z\) tels que \(k_3(r) = 0\) pour tout \(r_{m-1}\le r \le r_m\).

    Alors pour \((n,k_z)\) donné, pour tout \(r_{m-1}\le r \le r_m\) il existe \((c_i(n,k_z))_{1\le i\le 4} \in (\CC(\NN\times\RR))^4\) telles que 
    \begin{subequations}
      \begin{align*}
        \hat{E_z}(r,n,k_z) &= c_1(n,k_z) J_n\left(k_3r\right) + c_2(n,k_z) H_n^{(2)}\left(k_3r\right),
        \\
        \hat{\cH_z}(r,n,k_z) &= c_3(n,k_z) J_n\left(k_3r\right) + c_4(n,k_z) H_n^{(2)}\left(k_3r\right).
      \end{align*}
    \end{subequations}
  \end{prop}
  Pour toute autre couche, ces constantes sont différentes.
  \begin{proof}

    Les équations de Maxwell harmoniques dans la couche sont

    % \begin{equation*}
    % \left\lbrace \begin{aligned}
    %   \vrot \hat \vE(r,n,k_z) &=-ik(r)\eta(r) \hat \vH(r,n,k_z)
    %   \\
    %   \vrot \hat \vH(r,n,k_z) &= i\frac{k(r)}{\eta(r)} \hat \vE(r,n,k_z)
    % \end{aligned}\right.
    % \end{equation*}
    \begin{equation*}
    % \left\lbrace 
    \begin{aligned}
      i\left(\frac{n}{r}\hat{E_z} - k_z\hat{E_\theta}\right)\vect{e_r} +
      \left(ik_z\hat{E_r} - \ddp{r}{\hat{E_z}}\right)\vect{e_\theta} +
      \frac{1}{r}\left(\ddp{r}{(r\hat{E_\theta})}-in\hat{E_r}\right)\vect{e_z}
      &=-ik \eta \hat \vH,
      \\
      i\left(\frac{n}{r}\hat{\cH_z} - k_z\hat{\cH_\theta}\right)\vect{e_r} +
      \left(ik_z\hat{\cH_r} - \ddp{r}{\hat{\cH_z}}\right)\vect{e_\theta} +
      \frac{1}{r}\left(\ddp{r}{(r\hat{\cH_\theta})}-in\hat{\cH_r}\right)\vect{e_z}
      &=i k\eta^{-1} \hat \vE.
    \end{aligned}
    % \right.
    \end{equation*}
    % On remarque que la méthode utilisée pour le plan aboutie à une équation différentielle à coefficients non constants de type \(r\ddp{r}{\vect{X}}(r,n,k_z) = \mat{M}(r,n,k_z)\vect{X}(r,n,k_z)\).
    % On ne peut pas exprimer la solution avec les valeurs et vecteurs propres de la matrice.
    %Nous allons donc trouver une équation de Bessel en développant le système de Maxwell.

    On remarque que les 2 premières composantes de ces deux égalités permettent de déduire \(\hat E_r,\hat E_\theta, \hat \cH_r,\hat \cH_\theta\) de \(\hat E_z,\hat \cH_z\).

    On couple les 2 équations du système de Maxwell:
    % \begin{align}
    %   \vrot \vrot \hat \vE &= \w^2\eps\mu \hat \vE
    %   \\
    %   \vdiv \hat \vE &= 0
    % \end{align}
    % \begin{multline}
    %   \vrot \vrot \hat \vE = \dots\\
    %   i\left(\frac{n}{r^2}\left(\ddp{r}{(r\hat{E_\theta})} - in\hat{E_r}\right) - k_z\left(ik_z\hat{E_r} - \ddp{r}{\hat{E_z}}\right)\right)  \vect{e_r} \dots\\
    %   + \left(-k_z\left(\frac{n}{r}\hat{E_z} - k_z\hat{E_\theta}\right) -\ddp{r}{}\left(\frac{1}{r}\left(\ddp{r}{(r\hat{E_\theta})}-in\hat{E_r}\right)\right)\right)  \vect{e_\theta} \dots\\
    %   + \frac{1}{r}\left(\ddp{r}{} \left(r\left(ik_z\hat{E_r} - \ddp{r}{\hat{E_z}}\right)\right) + n \left(\frac{n}{r}\hat{E_z} - k_z\hat{E_\theta}\right)\right) \vect{e_z}
    % \end{multline}
    % On aboutit au système suivant
    \begin{equation*}
      \left\lbrace
      \begin{array}{ccc}
        -\left(k^2 -\frac{n^2}{r^2}  - k_z^2\right)\hat{E_r}  +i\frac{n}{r^2}\ddp{r}{(r\hat{E_\theta})}  +i k_z\ddp{r}{\hat{E_z}} & = & 0,
        \\
        in\ddp{r}{}\left(\frac{\hat{E_r}}{r}\right) -\left(k^2 - k_z^2\right)\hat{E_\theta} - \ddp{r}{}\left(\frac{1}{r}\ddp{r}{(r\hat{E_\theta})}\right)  - n\frac{k_z}{r}\hat{E_z} & = & 0,
        \\
        i\frac{k_z}{r}\ddp{r}{(r\hat{E_r})}  - n\frac{k_z}{r}\hat{E_\theta}  -\left(k^2 - \frac{n^2}{r^2} \right)\hat{E_z} - \frac{1}{r}\ddp{r}{}\left(r\ddp{r}{\hat{E_z}}\right) & = & 0.
      \end{array}
      \right.
    \end{equation*}

    De la troisième égalité, on a
    \begin{equation*}
    r^2 \ddp[2]{r}{\hat{E_z}} + r\ddp{r}{\hat{E_z}} + \left(r^2k^2 - n^2\right)\hat{E_z} =ik_zr\ddp{r}{(r\hat{E_r})} -  nk_zr\hat{E_\theta}.
    \end{equation*}

    Or \(\vdiv \hat \vE = 0\) et l'on a
    \begin{align*}
      \vdiv\hat \vE &= \frac{1}{r}\ddp{r}{(r\hat{E_r})} + \frac{in}{r}\hat{E_\theta} + ik_z\hat{E_z} = 0,
      \intertext{donc}
      k_z^2r^2 \hat{E_z} &= ik_zr\ddp{r}{(r\hat{E_r})} - nk_zr\hat{E_\theta}.
    \end{align*}
    Si l'on injecte cette égalité dans la 3\ieme composante du système de Maxwell couplé ci-dessus, on obtient sur \(\hat{E}_z\) (et donc aussi sur \(\hat{\cH}_z\)).
    \begin{equation*}
      r^2 \ddp[2]{r}{\hat{E_z}} + r\ddp{r}{\hat{E_z}} + \left(r^2\left(k^2 - k_z^2\right) - n^2\right)\hat{E_z} = 0.
    \end{equation*}

    On obtient l'équation différentielle en \(r\) vérifiée par \(\hat{E_z}\) et \(\hat{\cH_z}\):
    \begin{equation*}
      x^2 f''(x) + xf'(x) + \left(x^2a^2 - n^2\right)f(x) = 0.
    \end{equation*}

    Si \(a = k_3\) est non nul, c'est une équation de Bessel (voir \cite[eq (6.80)]{bowman_introduction_1958}) sinon une équation d'Euler et nous renvoyons à l'annexe \ref{sec:annexe:cylindre:k3_nul}.

    Si \(k_3\) est non nul, il existe donc \((c_1(n,k_z),c_2(n,k_z)) \in (\CC(\NN\times\RR))^2\) telles que le champ s'exprime dans la couche
    \begin{equation*}
          \hat{E_z}(r,n,k_z) = c_1(n,k_z) J_n\left(k_3r\right) + c_2(n,k_z) H_n^{(2)}\left(k_3r\right)
    \end{equation*}
    où \(J_n\) est la fonction de Bessel d'ordre \(n\), \(H_n^{(2)}\) la fonction de Hankel de deuxième espèce d'ordre \(n\). Il existe d'autres notations pour désigner ces fonctions,  voir \cite[p.~358]{abramowitz_handbook_1964}.
    On sait que l'on peut prendre n'importe quel couple de fonctions de Bessel, voir annexe \eqref{eq:annex:bessel:equiv_bessel}.
    % , on choisit ce dernier car les \(J_n\) sont définies pour tout \(r\) et les \(H_n^{2}\) décroissent plus vite que \(r^{1 \slash 2}\) quand \(r\) tend vers l'infini, donc ce choix est adapté à une décomposition en une onde incidente partout définie et une onde réfléchie décroissante à l'infini.

    De plus, d'après \cite[p.~358]{abramowitz_handbook_1964} ( \cite[\url{https://dlmf.nist.gov/10.4}]{dlmf_nist_2019} ), une fonction solution de l'équation de Bessel d'ordre \(n\) est linéairement dépendante de la même fonction d'ordre \(-n\), donc
    \begin{align*}
       \forall n \in \NN, \forall r \in \RR^+,& J_n(r) = (-1)^nJ_{-n}(r),
       \\
       \forall n \in \NN, \forall r \in \RR_+^*,& H_n^{(2)}(r) = (-1)^nH_{-n}^{(2)}(r).
    \end{align*}
    Donc  \(\hat{E_z}(r,n,k_z)\) est colinéaire à  \(\hat{E_z}(r,-n,k_z)\).
    On peut donc se restreindre à \(n\) entier naturel.

    On trouve exactement les mêmes résultats pour \(\hat{\cH_z}\) dans la couche de matériau.

    Il existe \((c_3(n,k_z),c_4(n,k_z)) \in (\CC(\NN_+\times\RR))^2\)
    \begin{equation*}
      \hat{\cH_z}(r,n,k_z) = c_3(n,k_z) J_n\left(k_3r\right) + c_4(n,k_z) H_n^{(2)}\left(k_3r\right).
    \end{equation*}

    %À l'extérieur, ce sont d'autres constantes et les paramètres \(k,\eta\) sont ceux du vide. 
  \end{proof}

  \subsection{Expressions des champs tangentiels dans chaque couche}


    \begin{defn}
      \label{def:cylindre:JE-JH-HE-HH}
      On définit les matrices \(\mJ_{E}(r,n,k_z),\mH_{E}(r,n,k_z),\mJ_{\cH}(r,n,k_z),\mH_{\cH}(r,n,k_z)\) où \(k(r),\eta(r)\) sont constants par morceaux en \(r\)
      \begin{align*}
        \mJ_{E}(r,n,k_z) &=
        \begin{bmatrix}
          -\frac{nk_z}{rk_3(r)^2}J_n(k_3(r)r) & \frac{ik(r)\eta(r)}{k_3(r)}J_n'(k_3(r)r)
          \\
          J_n(k_3(r)r) & 0
        \end{bmatrix},
        \\
        \mH_{E}(r,n,k_z) &=
        \begin{bmatrix}
          -\frac{nk_z}{rk_3(r)^2}H_n^{(2)}(k_3(r)r) & \frac{ik(r)\eta(r)}{k_3(r)}H_n^{(2)}{}'(k_3(r)r)
          \\
          H_n^{(2)}(k_3(r)r) & 0
        \end{bmatrix},
        \\
        \mJ_{\cH}(r,n,k_z) &=
        \begin{bmatrix}
          0 & -J_n(k_3(r)r)
          \\
          -\frac{ik(r)}{\eta(r)k_3(r)}J_n'(k_3(r)r) & -\frac{nk_z}{rk_3(r)^2}J_n(k_3(r)r)
        \end{bmatrix},
        \\
        \mH_{\cH}(r,n,k_z) &=
        \begin{bmatrix}
          0 & -H_n^{(2)}(k_3(r)r)
          \\
          -\frac{ik(r)}{\eta(r)k_3(r)}H_n^{(2)}{}'(k_3(r)r) & -\frac{nk_z}{rk_3(r)^2}H_n^{(2)}(k_3(r)r)
        \end{bmatrix}.
      \end{align*}
    \end{defn}

    Sur une interface d'équation \(r=r_p\) entre deux matériaux, il y a donc un saut de valeurs pour ces matrices: \(\lim_{\delta\rightarrow 0 } \mJ_E(r_p+ \delta,n,k_z) - \mJ_E(r_p - \delta,n,k_z) \not = 0\) .

    \begin{prop}
      Dans chaque couche \(r_{m-1}\le r < r_m\), il existe \((c_i(n,k_z))_{1\le i\le 4}\) telles que les composantes en \(\vect{e_\theta},\vect{e_z}\) des champs (\(\gls{phy-e},\gls{phy-J}=\vect{e_r}\pvect\gls{phy-h2}\)), dites composantes tangentielles par abus de langage, sont
      \begin{subequations}
        \begin{align}
          \hat \vE_t(r,n,k_z) &= \mJ_{E}(r,n,k_z)
          \begin{bmatrix}
            c_1(n,k_z) \\
            c_3(n,k_z)
          \end{bmatrix}
          +
          \mH_{E}(r,n,k_z)
          \begin{bmatrix}
            c_2(n,k_z) \\
            c_4(n,k_z)
          \end{bmatrix},
          \label{eq:imp_fourier:cylindre:Et}
          \\
          \hat\vJ_t(r,n,k_z) &=
          \mJ_{\cH}(r,n,k_z)
          \begin{bmatrix}
            c_1(n,k_z) \\
            c_3(n,k_z)
          \end{bmatrix}
          +
          \mH_{\cH}(r,n,k_z)
          \begin{bmatrix}
            c_2(n,k_z) \\
            c_4(n,k_z)
          \end{bmatrix}.
          \label{eq:imp_fourier:cylindre:Ht}
        \end{align}
      \end{subequations}
    \end{prop}


    \begin{proof}
      À partir des composantes en \(\vect{e_r},\vect{e_\theta}\) des équations de Maxwell, on peut déterminer \(\hat{E_r},\hat{E_\theta},\hat{\cH_r},\hat{\cH_\theta}\).
      \begin{align*}
        \left\lbrace
        \begin{matrix}
          -ik_z\hat{E_\theta} + ik(r)\eta(r) \hat{\cH_r} = -\frac{in}{r}\hat{E_z},
          \\
          ik_z\hat{E_r} + ik(r)\eta(r) \hat{\cH_\theta} = \ddp{r}{\hat{E_z}},
        \end{matrix}
        \right.
        &&
        \left\lbrace
        \begin{matrix}
          i\frac{k(r)}{\eta(r)} \hat{E_r} + ik_z \hat{\cH_\theta} = \frac{in}{r}\hat{\cH_z},
          \\
          i\frac{k(r)}{\eta(r)} \hat{E_\theta} - ik_z \hat{\cH_r} = -\ddp{r}{\hat{\cH_z}}.
        \end{matrix}
        \right.
      \end{align*}

      Ce système est équivalent à \(\vect{Y} = \mat{M}\vect{X}\) où la matrice \(\mat{M}\) et les champs de vecteurs \(\vect{X}, \vect{Y}\) sont définis tels que
      \begin{align*}
        \mat{M} =
        \begin{bmatrix}
        0 & -ik_z & ik(r)\eta(r) & 0
        \\
        ik_z & 0 & 0 & ik(r)\eta(r)
        \\
        i\frac{k(r)}{\eta(r)} & 0 & 0 & ik_z
        \\
        0 & i\frac{k(r)}{\eta(r)} & -ik_z & 0
        \end{bmatrix}
        ,&&
        \vect{X} =
        \begin{bmatrix}
          \hat{E_r}\\
          \hat{E_\theta}\\
          \hat{\cH_r}\\
          \hat{\cH_\theta}
        \end{bmatrix}
        ,&&
        \vect{Y} =
        \begin{bmatrix}
          -\frac{in}{r}\hat{E_z}\\
          \ddp{r}{\hat{E_z}}\\
          \frac{in}{r}\hat{\cH_z}\\
          -\ddp{r}{\hat{\cH_z}}
        \end{bmatrix}.
      \end{align*}

      On remarque que \(\mM^2 = \left(k_z^2 - k(r)^2\right)\mI=-k_3(r)^2\mI\) et donc que \(\det(\mat{M}) = k_3(r)^4\).

      % \subsection{Cas \(k_3\not=0\)}

      On suppose ce dernier non nul, on peut déduire \(\vect{X}\):

      \begin{equation*}
        \begin{bmatrix}
          \hat{E_r}\\
          \hat{E_\theta}\\
          \hat{\cH_r}\\
          \hat{\cH_\theta}
        \end{bmatrix} =
        \frac{1}{-k_3(r)^2}
        \begin{bmatrix}
        0 & -ik_z & ik(r)\eta(r) & 0
        \\
        ik_z & 0 & 0 & ik(r)\eta(r)
        \\
        i\frac{k(r)}{\eta(r)} & 0 & 0 & ik_z
        \\
        0 & i\frac{k(r)}{\eta(r)} & -ik_z & 0
        \end{bmatrix}
        \begin{bmatrix}
          -\frac{in}{r}\hat{E_z}\\
          \ddp{r}{\hat{E_z}}\\
          \frac{in}{r}\hat{\cH_z}\\
          -\ddp{r}{\hat{\cH_z}}
        \end{bmatrix}.
      \end{equation*}

      On extrait alors \(\hat{E_\theta}, \hat{\cH_\theta}\) pour obtenir les champs orthogonaux à \(\vect{e_r}\) en tout point, connaissant déjà \(\hat{E_z}, \hat{\cH_z}\).

      \begin{align*}
        % \hat{E_r} & = \frac{1}{k_3^2}\left(ik_z\ddp{r}{\hat{E_z}}+\frac{k\eta n}{r}\hat{\cH_z}\right)
        % \\
        \hat{E_\theta} &= -\frac{1}{k_3(r)^2}\left(\frac{nk_z}{r}\hat{E_z} - ik(r)\eta(r)\ddp{r}{\hat{\cH_z}}\right),
        \\
        \hat{E_z} &= c_1 J_n(k_3(r) r) + c_2 H_n^{(2)}(k_3(r) r),
        \\
        -\hat{\cH_z} &= -c_3 J_n(k_3(r) r) - c_4 H_n^{(2)}(k_3(r) r),
        \\
        \hat{\cH_\theta} &= -\frac{1}{k_3(r)^2}\left(i\frac{k(r)}{\eta(r)}\ddp{r}{\hat{E_z}} + \frac{nk_z}{r}\hat{\cH_z}\right).
      \end{align*}

      ce qui donne
       \begin{align*}
        \hat{E_\theta} =& -\frac{nk_z}{rk_3(r)^2}\left(c_1J_n(k_3(r)r) + c_2 H_n^{(2)}(k_3(r)r)\right) 
        \\
        & + \frac{ik(r)\eta(r)}{k_3(r)}\left(c_3J_n'(k_3(r)r) + c_4 H_n^{(2)}{}'(k_3(r)r)\right),
        \\
        \hat{E_z} =&{~} c_1 J_n(k_3(r) r) + c_2 H_n^{(2)}(k_3(r) r),
        \\
        -\hat{\cH_z} =& -c_3 J_n(k_3(r) r) - c_4 H_n^{(2)}(k_3(r) r),
        \\
        \hat{\cH_\theta} =& -{i\frac{k(r)}{k_3(r)\eta(r)}}\left(c_1J_n'(k_3(r)r) + c_2 H_n^{(2)}{}'(k_3(r)r)\right) 
        \\
        &- \frac{nk_z}{rk_3(r)^2}\left(c_3J_n(k_3(r)r) + c_4 H_n^{(2)}(k_3(r)r)\right).
      \end{align*}
    \end{proof}

    \begin{prop}
      \label{lem:cylindre:imp:inv_matrices_JE-HE}
      Si \(\mu(r)\eps(r) \in \CC\backslash\RR\), alors les matrices \(\mJ_E(r,n,k_z)\), \(\mJ_\cH(r,n,k_z)\)  sont inversibles pour tout couple \((n,k_z)\).
      % Sinon, pour \(n\) donné, il existe un nombre fini de \(k_z\) tels que ces matrices ne soient pas inversibles. 
    \end{prop}

    \begin{proof}
      Par définition des matrices,
      \begin{align*}
        \det(\mJ_E(r,n,k_z)) &= -\frac{ik(r)\eta(r)}{k_3(r)}J_n(k_3(r)r)J_n'(k_3(r)r),
        \\
        \det(\mJ_\cH(r,n,k_z)) &= -\frac{ik(r)}{\eta(r)k_3(r)}J_n(k_3(r)r)J_n'(k_3(r)r).
      \end{align*}

      D’après \cite[p.~370]{abramowitz_handbook_1964} ( \cite[\url{https://dlmf.nist.gov/10.21\#i}]{dlmf_nist_2019} ), les zéros des fonctions de Bessel d'ordre réel \(\nu >-1\) sont tous réels donc si \(k_3 \in\CC\backslash\RR\) alors ces matrices sont inversibles.
    \end{proof}


    Pour les matrices \(\mH_E(r,n,k_z)\), \(\mH_\cH(r,n,k_z)\), on a
    \begin{align*}
      \det(\mH_E(r,n,k_z)) &= -\frac{ik(r)\eta(r)}{k_3(r)}H_n^{(2)}(k_3(r)r)H_n^{(2)}{}'(k_3(r)r),
      \\
      \det(\mH_\cH(r,n,k_z)) &= -\frac{ik(r)}{\eta(r)k_3(r)}H_n^{(2)}(k_3(r)r)H_n^{(2)}{}'(k_3(r)r).
     \end{align*}
    Et on ne peut rien conclure, car les zéros peuvent être complexes ( voir \cite{sandstrom_note_2007} ). On va donc supposer ces déterminants non nuls.


    %%%%%%%%%%%%%%%%%%%%%%%%%%%%%%%%%%%%%%%%%%%%%%%%%%%%%%%%%%%%%%%%%%%%%%%%%%%%%%%%%%%%%%%%%%%%%%%%%%%%%%%%
    %%%%%%%%%%%%%%%%%%%%%%%%%%%%%%%%%%%%%%%%%%%%%%%%%%%%%%%%%%%%%%%%%%%%%%%%%%%%%%%%%%%%%%%%%%%%%%%%%%%%%%%%
    %%%%%%%%%%%%%%%%%%%%%%%%%%%%%%%%%%%%%%%%%%%%%%%%%%%%%%%%%%%%%%%%%%%%%%%%%%%%%%%%%%%%%%%%%%%%%%%%%%%%%%%%


  \subsection{Expression de la matrice d'impédance pour une couche}

    Soit \(r_1 = r_0 + d\). On se place dans le matériau donc \(r_0 \le r < r_1\).

    \begin{prop}
      Si on suppose que la matrice 
      \newline
      \(\mH_{\cH}(r_1^-,n,k_z) - \mJ_{\cH}(r_1^-,n,k_z)\mJ_{E}(r_0^+,n,k_z)^{-1}\mH_{E}(r_0^+,n,k_z)\) est inversible pour tout \((n,k_z)\), alors on a 
      \begin{equation*}
        \hat \vE_t(r_1^-,n,k_z) = \hat \mZ(n,k_z) \left(\hat\vJ_t(r_1^-,n,k_z)\right)
      \end{equation*}
      où
      \begin{multline*}
        \hat \mZ(n,k_z) =
        \left(\mH_{E}(r_1^-,n,k_z)\mH_{E}(r_0^+,n,k_z)^{-1} - \mJ_{E}(r_1^-,n,k_z)\mJ_{E}(r_0^+,n,k_z)^{-1}\right)\\
        \left(\mH_{\cH}(r_1^-,n,k_z)\mH_{E}(r_0^+,n,k_z)^{-1} - \mJ_{\cH}(r_1^-,n,k_z)\mJ_{E}(r_0^+,n,k_z)^{-1}\right)^{-1}.
      \end{multline*}
    \end{prop}

    \begin{proof}
      On se place dans le matériau: \(r_0 \le r \le r_ 1 \). Dans cette couche, \(\eps(r)\equiv \eps, \mu(r)\equiv \mu\).

      On applique la condition limite du conducteur parfait \(\hat \vE(r_0,n,k_z) = 0\) dans \eqref{eq:imp_fourier:cylindre:Et}, donc
      \begin{equation*}
        \mJ_{E}(r_0,n,k_z)
        \begin{bmatrix}
          c_1(n,k_z) \\
          c_3(n,k_z)
        \end{bmatrix}
        =-\mH_{E}(r_0,n,k_z)
        \begin{bmatrix}
          c_2(n,k_z) \\
          c_4(n,k_z)
        \end{bmatrix}.
      \end{equation*}

      On peut donc exprimer les composantes tangentielles%, on omet les dépendances en \((n,k_z)\).
      \begin{align*}
        \hat \vE_t(r_1,n,k_z) &=
        \left(\mH_{E}(r_1,n,k_z) - \mJ_{E}(r_1,n,k_z)\mJ_{E}(r_0,n,k_z)^{-1}\mH_{E}(r_0,n,k_z)\right)
        \begin{bmatrix}
          c_2(n,k_z) \\
          c_4(n,k_z)
        \end{bmatrix},
        \\
        \hat \vJ_t(r_1,n,k_z) &=
        \left(\mH_{\cH}(r_1,n,k_z) - \mJ_{\cH}(r_1,n,k_z)\mJ_{E}(r_0,n,k_z)^{-1}\mH_{E}(r_0,n,k_z) \right)
        \begin{bmatrix}
          c_2(n,k_z) \\
          c_4(n,k_z)
        \end{bmatrix}.
      \end{align*}

      On calcule le déterminant de cette matrice 
      \begin{multline*}
        \det(\mH_{\cH}(r_1,n,k_z) - \mJ_{\cH}(r_1,n,k_z)\mJ_{E}(r_0,n,k_z)^{-1}\mH_{E}(r_0,n,k_z)) = \\
        -\frac{k(r)^2}{\eta(r)^2}\frac{(H_n^{(2)}(r_1k_3)J_n'(r_0k_3)-J_n(r_1k_3)H_n^{(2)}{}'(r_0k_3))(H_n^{(2)}{}'(r_1k_3)J_n(r_0k_3)-J_n'(r_1k_3)H_n^{(2)}(r_0k_3))}{J_n'(r_0^+k_3)J_n(r_0^+k_3)}
      \end{multline*}
      Ce dernier s'annule quand \(\frac{\cH_n^{(2)}(r_1k_3)}{J_n(r_1k_3)} = \frac{\cH_n^{(2)}{}'(r_0k_3)}{J_n'(r_0k_3)}\) ou quand \(\frac{\cH_n^{(2)}{}'(r_1k_3)}{J_n'(r_1k_3)} = \frac{\cH_n^{(2)}(r_0k_3)}{J_n(r_0k_3)}\).

      Il n'existe pas à notre connaissance de propriétés connues des fonctions de Bessel pour déduire de ces égalités des conditions sur \(k,k_z,r_1,r_0\).

      On suppose donc simplement \(\mH_{\cH}(r_1,n,k_z) - \mJ_{\cH}(r_1,n,k_z)\mJ_{E}(r_0,n,k_z)^{-1}\mH_{E}(r_0,n,k_z)\) inversible, la matrice \(\hat\mZ(n,k_z)\) telle que \(\hat\vE_t(r_1,n,k_z) = \hat\mZ(n,k_z) (\vect{e_r}\pvect\vH(r_1,n,k_z))\) est alors

      \begin{multline*}
        \hat \mZ =
        \left(\mH_{E}(r_1,n,k_z) - \mJ_{E}(r_1,n,k_z)\mJ_{E}(r_0,n,k_z)^{-1}\mH_{E}(r_0,n,k_z)\right)
        \\
        \left(\mH_{\cH}(r_1,n,k_z) - \mJ_{\cH}(r_1,n,k_z)\mJ_{E}(r_0,n,k_z)^{-1}\mH_{E}(r_0,n,k_z)\right)^{-1}.
      \end{multline*}
      Comme on a supposé la matrice \(\mH_E(r_0,n,k_z)\) inversible, on peut simplifer par cette matrice d'où la proposition.

    \end{proof}


  \subsection{Expression de la matrice d'impédance pour plusieurs couches}

    \begin{figure}[!hbt]
      \centering
      \tikzsetnextfilename{cylindre_n_couches}
      \begin{tikzpicture}
        \tikzmath{
    \a = 83;
    \b = 97;
    \d = 0.5;
    \ri = 30;
    \re = \ri;
}

% Le conducteur
\tikzmath{
    \ri = \re;
    \re = \ri + 0.5*\d;
    \xa = cos(\a)*\re;
    \ya = sin(\a)*\re;
    \xb = cos(\b)*\ri;
    \yb = sin(\b)*\ri;
}

\coordinate (a) at (\xa,\ya);
\coordinate (b) at (\xb,\yb);

\fill [pattern=north east lines] (a) arc (\a:\b:\re) -- (b) arc (\b:\a:\ri) -- cycle;
\draw (a) arc (\a:\b:\re);
\draw (a) node [right] {$r_0$};

% Le repère
\coordinate (n) at ($(a)+(0.5,-1)$);
%
%
%\draw [->] (n) -- ++(0,1) node [at end, right] {$\v{\pr}$};
%\draw [->] (n) -- ++(1,0) node [at end, right] {$\v{\pt}$};
%
\draw (n) ++(0.2,0.2) circle(0.1cm) node [above=0.1cm] {$\vect{e_z}$};
\draw (n) ++(0.2,0.2) +(135:0.1cm) -- +(315:0.1cm);
\draw (n) ++(0.2,0.2) +(45:0.1cm) -- +(225:0.1cm);

% 1 ere couche

\tikzmath{
    \ri = \re;
    \re = \ri + \d;
    \xa = cos(\a)*\re;
    \ya = sin(\a)*\re;
    \xb = cos(\b)*\ri;
    \yb = sin(\b)*\ri;
    \xc = cos(0.5*(\b+\a))*(\ri+0.5*\d);
    \yc = sin(0.5*(\b+\a))*(\ri+0.5*\d);
}

\coordinate (a) at (\xa,\ya);
\coordinate (b) at (\xb,\yb);
\coordinate (c) at (\xc,\yc);

\fill [lightgray] (a) arc (\a:\b:\re) -- (b) arc (\b:\a:\ri) -- cycle;
\draw (a) arc (\a:\b:\re);
\draw (c) node {$\eps_1,\mu_1,d_1$};


% Des couches

\tikzmath{
    \ri = \re;
    \re = \ri + 2*\d;
    \xa = cos(\a)*\re;
    \ya = sin(\a)*\re;
    \xb = cos(\b)*\ri;
    \yb = sin(\b)*\ri;
    \xc = cos(0.5*(\b+\a))*(\ri+0.5*\d);
    \yc = sin(0.5*(\b+\a))*(\ri+0.5*\d);
}

\coordinate (a) at (\xa,\ya);
\coordinate (b) at (\xb,\yb);
\coordinate (c) at (\xc,\yc);

\fill [lightgray]    (a) arc (\a:\b:\re) -- (b) arc (\b:\a:\ri) -- cycle;
\fill [pattern=dots] (a) arc (\a:\b:\re) -- (b) arc (\b:\a:\ri) -- cycle;
\draw (a) arc (\a:\b:\re);

% n eme couche

\tikzmath{
    \ri = \re;
    \re = \ri + \d;
    \xa = cos(\a)*\re;
    \ya = sin(\a)*\re;
    \xb = cos(\b)*\ri;
    \yb = sin(\b)*\ri;
    \xc = cos(0.5*(\b+\a))*(\ri+0.5*\d);
    \yc = sin(0.5*(\b+\a))*(\ri+0.5*\d);
}

\coordinate (a) at (\xa,\ya);
\coordinate (b) at (\xb,\yb);
\coordinate (c) at (\xc,\yc);

\fill [lightgray] (a) arc (\a:\b:\re) -- (b) arc (\b:\a:\ri) -- cycle;
\draw (a) arc (\a:\b:\re);
\draw (c) node {$\eps_{Nc},\mu_{Nc},d_{Nc}$};

% Le vide
\tikzmath{
    \xc = cos(0.5*(\b+\a))*(\re);
    \yc = sin(0.5*(\b+\a))*(\re);
}

\draw (\xc,\yc) node [above] {vide};


      \end{tikzpicture}
    \end{figure}

    Soit \(r_p\) le rayon de l'interface \(p\), \(r_p = r_0 +\sum_{i=1}^{p} d_{i}\). On dit que l'on se trouve dans la couche \(p\) si \(r_{p-1} \le r < r_p \).

    \begin{defn}
      \label{def:cylindre:matrices_MJ-MH}
      On définit les fonctions de \([r_{p-1}, r_p[\times \NN \times \RR \times \mathcal{M}_{2}(\CC) \rightarrow \mathcal{M}_{2}(\CC)\)
      \begin{align*}
        \mM_{\mJ}(r,n,k_z,\mA) &= \mJ_{E}(r,n,k_z) -  \mA \mJ_{\cH}(r,n,k_z),
        \\
        \mM_{\mH}(r,n,k_z,\mA) &= \mH_{E}(r,n,k_z) -  \mA \mH_{\cH}(r,n,k_z).
      \end{align*}
    \end{defn}

    \begin{defn}
      \label{def:cylindre:reflexion:impedance}
      On définit la fonction de \([r_{p-1}, r_p[\times \NN \times \RR \times \mathcal{M}_{2}(\CC) \rightarrow \mathcal{M}_{2}(\CC)\)
      \begin{align*}
        \mR(r,n,k_z,\mA) &= -\mM_{\mH}(r,n,k_z,\mA)^{-1}\mM_{\mJ}(r,n,k_z,\mA).
      \end{align*}
    \end{defn}
    A priori, pour \((r,n,k_z)\) donné, \(\mR(r,n,k_z,\mA)\) n'est pas défini pour toute matrice \(\mA\).

    On prolonge ces définitions aux autres couches.

    \begin{defn}%[Fonction de transfert]{}~
      \label{def:cylindre:transfert:impedance}

      On définit \(\mT_p\) la fonction de \([r_{p-1}, r_p[^2\times\NN\times\RR\times\mathcal{M}_2(\CC)\rightarrow \mathcal{M}_{2}(\CC)\)
      \begin{multline*}
        \mT_p(r,r',n,k_z,\mA) = \\
          \left(\mJ_{E}(r,n,k_z)\mM_{\mJ}(r',n,k_z,\mA)^{-1} - \mH_{E}(r,n,k_z)\mM_{\mH}(r',n,k_z,\mA)^{-1}\right) 
          \\
          \left(\mJ_{\cH}(r,n,k_z)\mM_{\mJ}(r',n,k_z,\mA)^{-1} - \mH_{\cH}(r,n,k_z)\mM_{\mH}(r',n,k_z,\mA)^{-1}\right)^{-1}
      \end{multline*}
    \end{defn}
    A priori, pour \((r,r',n,k_z)\) donné, \(\mT_p(r,r',n,k_z,\mA)\) n'est pas défini pour toute matrice \(\mA\).

    \begin{prop}%[Théorème de transfert]~
      \label{prop:cylindre:transfert:impedance}

      Soient \(\hat\vE_t,\hat\vH_t\) tels que \(\vE_t(r_{p}^-,n,k_z) = \hat\mZ_{p}(n,k_z)(\hat\vJ_t(r_{p}^-,n,k_z)\).

      Si les matrices suivantes sont inversibles
      \begin{align*}
        \mM_{\mJ}(r_p^-,n,k_z,\hat\mZ_{p}(n,k_z)), && \mM_{\mH}(r_p^-,n,k_z,\hat\mZ_{p}(n,k_z)),
      \end{align*}
      \begin{align*}
        \mJ_{\cH}(r_{p-1}^+,n,k_z)\mM_{\mJ}(r_{p}^-,n,k_z,\hat\mZ_{p}(n,k_z))^{-1} - \mH_{\cH}(r_{p-1}^+,n,k_z)\mM_{\mH}(r_{p}^-,n,k_z,\hat\mZ_{p}(n,k_z))^{-1},
      \end{align*}

      alors \(\hat\vE_t(r_{p-1}^+,n,k_z) = \mT_p(r_{p-1}^+,r_{p}^-,n,k_z,\hat\mZ_{p}(n,k_z))(\hat\vJ_t(r_{p-1}^+,n,k_z))\).

      Une condition d'impédance sur le bord supérieur d'une couche détermine la condition limite sur le bord inférieur.
    \end{prop}


    \begin{proof}
      On se situe dans la couche \(p\) (\(r_{p-1}\le r < r_p\)) et l'on sait qu'il existe dans cette couche des constantes \(c_i(n,k_z)\) telles que les champs vérifient
      \begin{multline*}
        \mJ_{E}(r_{p},n,k_z)
        \begin{bmatrix}
          c_1(n,k_z) \\
          c_3(n,k_z)
        \end{bmatrix}
        +
        \mH_{E}(r_{p},n,k_z)
        \begin{bmatrix}
          c_2(n,k_z) \\
          c_4(n,k_z)
        \end{bmatrix}
        =
        \\
        \hat \mZ_{p}(n,k_z)
        \left(
          \mJ_{\cH}(r_{p},n,k_z)
          \begin{bmatrix}
            c_1(n,k_z) \\
            c_3(n,k_z)
          \end{bmatrix}
          +
          \mH_{\cH}(r_{p},n,k_z)
          \begin{bmatrix}
            c_2(n,k_z) \\
            c_4(n,k_z)
          \end{bmatrix}
        \right).
      \end{multline*}

      Ce qui revient à 
      \begin{equation*}
        \mM_{\mJ}(r_{p},n,k_z,\hat\mZ_p(n,k_z))
        \begin{bmatrix}
          c_1(n,k_z) \\
          c_3(n,k_z)
        \end{bmatrix}
        =
        -\mM_{\mH}(r_{p},n,k_z,\hat\mZ_p(n,k_z))
        \begin{bmatrix}
          c_2(n,k_z) \\
          c_4(n,k_z)
        \end{bmatrix}.
      \end{equation*}

      On suppose que les matrices \(\mM_{\mJ}(r_p,n,k_z,\hat\mZ_p(n,k_z)), \mM_{\mH}(r_p,n,k_z,\hat\mZ_p(n,k_z))\) sont inversibles donc
      \begin{equation*}
        \begin{bmatrix}
          c_2(n,k_z) \\
          c_4(n,k_z)
        \end{bmatrix}
        =
        \mR(r_{p},n,k_z,\hat\mZ_p(n,k_z))
        \begin{bmatrix}
          c_1(n,k_z) \\
          c_3(n,k_z)
        \end{bmatrix}.
      \end{equation*}

      On injecte ce qui précède en \(r = r_{p-1}\)
      \begin{multline*}
        \hat{\vE}_t(r_{p-1},n,k_z) = 
        \\
        \left(\mH_{E}(r_{p-1},n,k_z)\mR(r_{p},n,k_z,\hat{\mZ}_p(n,k_z)) + \mJ_{E}(r_{p-1},n,k_z)\right)
        \begin{bmatrix}
          c_1(n,k_z) \\
          c_3(n,k_z)
        \end{bmatrix},
      \end{multline*}        
      \begin{multline*}
        \vect{e_r}\times\hat{\vH}_t(r_{p-1},n,k_z) =
        \\
        \left(\mH_{\cH}(r_{p-1},n,k_z)\mR(r_{p},n,k_z,\hat{\mZ}_p(n,k_z)) + \mJ_{\cH}(r_{p-1},n,k_z))\right)
        \begin{bmatrix}
          c_1(n,k_z) \\
          c_3(n,k_z)
        \end{bmatrix}.
      \end{multline*}

      On suppose alors que cette dernière est inversible pour tout \((n,k_z)\).

      On obtient
      \begin{multline*}
        \hat{\vE}_t(r_{p-1},n,k_z) =
        \\
        \left(\mJ_{E}(r_{p-1},n,k_z) + \mH_{E}(r_{p-1},n,k_z)\mR(r_{p},n,k_z,\hat{\mZ}_p(n,k_z))\right) \\
        \left(\mJ_{\cH}(r_{p-1},n,k_z) + \mH_{\cH}(r_{p-1},n,k_z)\mR(r_{p},n,k_z,\hat{\mZ}_p(n,k_z))\right)^{-1}
        \\
        \vect{e_r}\times\hat{\vH}_t(r_{p-1},n,k_z).
      \end{multline*}

      Comme on a supposé l'inversibilité des deux matrices \(\mM_J\), \(\mM_H\) alors on peut factoriser à droite le numérateur et le dénominateur et on a la propriété.
    \end{proof}

    \begin{prop}%[Théorème de relèvement]~
      \label{prop:cylindre:relevement:impedance}

      Soient \(\hat\vE_t,\hat\vH_t\) tels que \(\vE_t(r_{p-1}^+,n,k_z) = \hat\mZ_{p-1}(n,k_z)(\hat\vJ_t(r_{p-1}^+,n,k_z)\).

      Si les matrices suivantes sont inversibles
      \begin{align*}
        \mM_{\mJ}(r_{p-1}^+,n,k_z,\hat\mZ_{p-1}(n,k_z)), && \mM_{\mH}(r_{p-1}^+,n,k_z,\hat\mZ_{p-1}(n,k_z)),
      \end{align*}
      \begin{align*}
        \mH_{\cH}(r_{p}^-,n,k_z)\mM_{\mH}(r_{p-1}^+,n,k_z,\hat\mZ_{p-1}(n,k_z))^{-1} - \mJ_{\cH}(r_{p}^-,n,k_z)\mM_{\mJ}(r_{p-1}^+,n,k_z,\hat\mZ_{p-1}(n,k_z))^{-1},
      \end{align*}

      alors \(\hat\vE_t(r_{p}^-,n,k_z) = \mT_p(r_p^-,r_{p-1}^+,n,k_z,\hat\mZ_{p-1}(n,k_z))(\hat\vJ_t(r_{p}^-,n,k_z))\).

      Une condition d'impédance sur le bord inférieur d'une couche détermine la condition limite sur le bord supérieur.
    \end{prop}

    \begin{proof}
      Même méthodologie que pour la proposition \ref{prop:cylindre:transfert:impedance}.
    \end{proof}

    \begin{prop}%[Corollaire aux théorèmes de transfert et de relèvement.]{}~
      \label{prop:cylindre:synthese:impedance}
      Soient \(\hat\vE_t,\hat\vH_t\) tels que 
      \begin{align*}
      \hat{\vE}_t(r_{p-1}^+,n,k_z) &= \hat\mZ_{p-1}(n,k_z)(\hat\vJ_t(r_{p-1}^+,n,k_z)),
      \\
      \hat{\vE}_t(r_{p}^-,n,k_z) &= \hat\mZ_{p}(n,k_z)(\hat\vJ_t(r_{p}^-,n,k_z)).
      \end{align*}

      Si les matrices suivantes sont inversibles
      \begin{align*}
        \mM_{\mJ}(r_p^-,n,k_z,\hat\mZ_{p}(n,k_z)), && \mM_{\mJ}(r_{p-1}^+,n,k_z,\hat\mZ_{p-1}(n,k_z)),
        \\
        \mM_{\mH}(r_p^-,n,k_z,\hat\mZ_{p}(n,k_z)), && \mM_{\mH}(r_{p-1}^+,n,k_z,\hat\mZ_{p-1}(n,k_z)),
      \end{align*}
      \begin{align*}
        \mJ_{\cH}(r_{p}^-,n,k_z)\mM_{\mJ}(r_{p-1}^+,n,k_z,\hat\mZ_{p-1}(n,k_z))^{-1} - \mH_{\cH}(r_{p}^-,n,k_z)\mM_{\mH}(r_{p-1}^+,n,k_z,\hat\mZ_{p-1}(n,k_z))^{-1},
        \\
        \mJ_{\cH}(r_{p-1}^+,n,k_z)\mM_{\mJ}(r_{p}^-,n,k_z,\hat\mZ_{p}(n,k_z))^{-1} - \mH_{\cH}(r_{p-1}^+,n,k_z)\mM_{\mH}(r_{p}^-,n,k_z,\hat\mZ_{p}(n,k_z))^{-1},
      \end{align*}
      alors 
      \begin{align*}
        \hat\mZ_{p-1}(n,k_z) &= \mT_p(r_{p-1}^+,r_{p}^-,n,k_z,\hat\mZ_{p}(n,k_z)),
        \\
        \hat\mZ_{p}(n,k_z) &= \mT_p(r_{p}^-,r_{p-1}^+,n,k_z,\hat\mZ_{p-1}(n,k_z)).
      \end{align*}

    \end{prop}

    On peut donc déterminer itérativement les matrices d'impédance. Dans notre cadre d'étude, la présence d'un conducteur parfait sur l'interface \(r=r_0^+\) implique \(\hat\mZ_{0}(n,k_z) = 0\).

  %%%%%%%%%%%%%%%%%%%%%%%%%%%%%%%%%%%%%%%%%%%%%%%%%%%%%%%%%%%%%%%%%%%%%%%%%%%%%%%%%%%%%%%%%%%%%%%%%%%%%%%%
  %%%%%%%%%%%%%%%%%%%%%%%%%%%%%%%%%%%%%%%%%%%%%%%%%%%%%%%%%%%%%%%%%%%%%%%%%%%%%%%%%%%%%%%%%%%%%%%%%%%%%%%%
  %%%%%%%%%%%%%%%%%%%%%%%%%%%%%%%%%%%%%%%%%%%%%%%%%%%%%%%%%%%%%%%%%%%%%%%%%%%%%%%%%%%%%%%%%%%%%%%%%%%%%%%%

  \subsection{Expression des coefficients de la série de Fourier}

    On se place à l'interface \(p\) donc \(r_{p-1} \le r < r_{p+1} \).

    \begin{defn}
      \label{def:cylindre:matrices_NE-NH}
      On définit les fonctions de \(\RR\times \NN \times \RR \times \mathcal{M}_{2}(\CC) \rightarrow \mathcal{M}_{2}(\CC)\)
      \begin{align*}
        \mN_{E}(r,n,k_z,\mA) &= \mJ_{E}(r,n,k_z) + \mH_{E}(r,n,k_z)\mA,
        \\
        \mN_{\cH}(r,n,k_z,\mA) &= \mJ_{\cH}(r,n,k_z) + \mH_{\cH}(r,n,k_z)\mA.
      \end{align*}
    \end{defn}

    \begin{defn}%[Fonction de transfert]{}~
      \label{def:cylindre:transfert:reflexion}{}~

      On définit \(\mathfrak{T}_p\) la fonction de \([r_{p-1}, r_p]\times[r_p, r_{p+1}]\times\NN\times\RR\times\mathcal{M}_2(\CC)\rightarrow \mathcal{M}_{2}(\CC)\)
      \begin{multline*}
        \mathfrak{T}_p(r,r',n,k_z,\mA) = \\
          -\left(\mN_{E}(r',n,k_z,\mA)^{-1}\mH_{E}(r,n,k_z) - \mN_{\cH}(r',n,k_z,\mA)^{-1}\mH_{\cH}(r,n,k_z)\right)^{-1}
          \\
          \left(\mN_{E}(r',n,k_z,\mA)^{-1}\mJ_{E}(r,n,k_z) - \mN_{\cH}(r',n,k_z,\mA)^{-1}\mJ_{\cH}(r,n,k_z)\right).
      \end{multline*}
    \end{defn}
    A priori, pour \(r,r',n,k_z\) donné, \(\mathfrak{T}_p(r,r',n,k_z,\mA)\) n'est pas défini pour toute matrice \(\mA\).

    \begin{prop}%[Théorème de transfert]~
      \label{prop:cylindre:transfert:reflexion}{}~

      On suppose qu'il existe \(\hat\mR_{p+1}(n,k_z)\) telle que 
      \begin{align*}
        \hat{\vE}_t(r_{p}^+,n,k_z) &= \mN_{E}(r_p^+,n,k_z,\hat\mR_{p+1}(n,k_z))\vect{C}_{p+1}(n,k_z),
        \\
        \vect{e_r}\pvect\vH(r_{p}^+,n,k_z) &= \mN_{\cH}(r_p^+,n,k_z,\hat\mR_{p+1}(n,k_z))\vect{C}_{p+1}(n,k_z).
      \end{align*}

      Si les matrices suivantes sont inversibles
      \begin{align*}
        \mN_{E}(r_p^+,n,k_z,\hat\mR_{p+1}(n,k_z)), && \mN_{\cH}(r_p^+,n,k_z,\hat\mR_{p+1}(n,k_z)),
      \end{align*}
      \begin{align*}
        \mN_{E}(r_p^+,n,k_z,\hat\mR_{p+1}(n,k_z))^{-1}\mH_{E}(r_p^-,n,k_z) - \mN_{\cH}(r_p^+,n,k_z,\hat\mR_{p+1}(n,k_z))^{-1}\mH_{\cH}(r_p^-,n,k_z),
      \end{align*}
      alors
      \begin{align*}
        \hat\vE_t(r_{p}^-,n,k_z) &= \mN_{E}(r_p^-,n,k_z,\mathfrak{T}_p(r_p^-,r_p^+,n,k_z,\hat\mR_{p+1}(n,k_z)))\vect{C}_{p}(n,k_z),
        \\
        \hat\vJ_t(r_{p}^-,n,k_z) &= \mN_{\cH}(r_p^-,n,k_z,\mathfrak{T}_p(r_p^-,r_p^+,n,k_z,\hat\mR_{p+1}(n,k_z)))\vect{C}_{p}(n,k_z).
      \end{align*}
    \end{prop}

    \begin{proof}
      De part et d'autre de \(r=r_p\), on a 
      \begin{align*}
        \hat{\vE}_t(r_p^+,n,k_z) &= \mN_{E}(r_p^+,n,k_z,\hat\mR_{p+1}(n,k_z))\vect{C}_{1}^+(n,k_z),
        \\
        \hat{\vE}_t(r_p^-,n,k_z) &= \mJ_E(r_p^-,n,k_z)\vect{C}_{1}^-(n,k_z) + \mH_E(r_p^-,n,k_z)\vect{C}_{2}^-(n,k_z),
      \end{align*}
      \begin{align*}
        \vect{e_r}\pvect\vH(r_p^+,n,k_z) &= \mN_{\cH}(r_p^+,n,k_z,\hat\mR_{p+1}(n,k_z))\vect{C}_{1}^+(n,k_z),
        \\
        \vect{e_r}\pvect\vH(r_p^-,n,k_z) &= \mJ_\cH(r_p^-,n,k_z)\vect{C}_{1}^-(n,k_z) + \mH_\cH(r_p^-,n,k_z)\vect{C}_{2}^-(n,k_z).
      \end{align*}
      Il y a continuité des champs au travers de l'interface donc
      \begin{align*}
        \mJ_E(r_p^-,n,k_z)\vect{C}_{1}^-(n,k_z) + \mH_E(r_p^-,n,k_z)\vect{C}_{2}^-(n,k_z) &= \mN_{E}(r_p^+,n,k_z,\hat\mR_{p+1}(n,k_z))\vect{C}_{1}^+(n,k_z),
        \\
        \mJ_\cH(r_p^-,n,k_z)\vect{C}_{1}^-(n,k_z) + \mH_\cH(r_p^-,n,k_z)\vect{C}_{2}^-(n,k_z) &= \mN_{\cH}(r_p^+,n,k_z,\hat\mR_{p+1}(n,k_z))\vect{C}_{1}^+(n,k_z),
      \end{align*}
      donc si on suppose les matrices \(\mN_E, \mN_H\) inversibles
      \begin{multline*}
        \mN_{E}(r_p^+,n,k_z,\hat\mR_{p+1}(n,k_z))^{-1}\left(\mJ_E(r_p^-,n,k_z)\vect{C}_{1}^-(n,k_z) + \mH_E(r_p^-,n,k_z)\vect{C}_{2}^-(n,k_z)\right) =
        \\
        \mN_{\cH}(r_p^+,n,k_z,\hat\mR_{p+1}(n,k_z))^{-1}\left(\mJ_\cH(r_p^-,n,k_z)\vect{C}_{1}^-(n,k_z) + \mH_\cH(r_p^-,n,k_z)\vect{C}_{2}^-(n,k_z)\right).
      \end{multline*}

      On regroupe les termes pour obtenir une relation entre les deux vecteurs,
      \begin{multline*}
        \left(\mN_{E}(r_p^+,n,k_z,\hat\mR_{p+1}(n,k_z))^{-1}\mJ_E(r_p^-,n,k_z)
        \right.
        \\
        \left.
        - \mN_{\cH}(r_p^+,n,k_z,\hat\mR_{p+1}(n,k_z))^{-1}\mJ_\cH(r_p^-,n,k_z)\right)\vect{C}_{1}^-(n,k_z) =
        \\
        -\left(\mN_{E}(r_p^+,n,k_z,\hat\mR_{p+1}(n,k_z))^{-1}\mH_E(r_p^-,n,k_z)
        \right. 
        \\
        \left.
        + \mN_{\cH}(r_p^+,n,k_z,\hat\mR_{p+1}(n,k_z))^{-1}\mH_\cH(r_p^-,n,k_z)\right)\vect{C}_{2}^-(n,k_z).
      \end{multline*}

      On suppose l'inversibilité de la matrice devant \(\vect{C}_{2}^-(n,k_z)\), et alors
      \begin{equation*}
        \vect{C}_{2}^-(n,k_z) = \mathfrak{T}_p(r_p^-,r_p^+,n,k_z,\hat\mR_{p+1}(n,k_z)) \vect{C}_{1}^-(n,k_z).
      \end{equation*}
    \end{proof}

    \begin{prop}%[Théorème de relévement]~
      \label{prop:cylindre:relevement:reflexion}{}~

      On suppose qu'il existe \(\hat\mR_{p}(n,k_z)\) telle que 
      \begin{align*}
        \hat{\vE}_t(r_{p}^-,n,k_z) &= \mN_{E}(r_p^-,n,k_z,\hat\mR_{p}(n,k_z))\vect{C}_{p}(n,k_z),
        \\
        \hat\vJ_t(r_{p}^-,n,k_z) &= \mN_{\cH}(r_p^-,n,k_z,\hat\mR_{p}(n,k_z))\vect{C}_{p}(n,k_z).
      \end{align*}

      Si les matrices suivantes sont inversibles
      \begin{align*}
        \mN_{E}(r_p^-,n,k_z,\mR_{p}(n,k_z)), && \mN_{\cH}(r_p^-,n,k_z,\hat\mR_{p}(n,k_z)),
      \end{align*}
      \begin{align*}
        \mN_{E}(r_p^-,n,k_z,\mR_{p}(n,k_z))^{-1}\mH_{E}(r_p^+,n,k_z) - \mN_{\cH}(r_p^-,n,k_z,\hat\mR_{p}(n,k_z))^{-1}\mH_{\cH}(r_p^+,n,k_z),
      \end{align*}
      alors
      \begin{align*}
        \hat{\vE}_t(r_{p}^+,n,k_z) &= \mN_{E}(r_p^+,n,k_z,\mathfrak{T}_p(r_p^+,r_p^-,n,k_z,\hat\mR_{p}(n,k_z)))\vect{C}_{p+1}(n,k_z),
        \\
        \hat{\vE}_t(r_{p}^+,n,k_z) &= \mN_{\cH}(r_p^+,n,k_z,\mathfrak{T}_p(r_p^+,r_p^-,n,k_z,\hat\mR_{p}(n,k_z)))\vect{C}_{p+1}(n,k_z).
      \end{align*}
    \end{prop}

    \begin{proof}
      Même méthodologie que pour la proposition \ref{prop:cylindre:transfert:reflexion}.
    \end{proof}

    \begin{prop}%[Théorème de relévement]~
      \label{prop:cylindre:synthese:reflexion}{}~

      On suppose qu'il existe \(\hat\mR_{p}(n,k_z)\) et \(\hat\mR_{p+1}(n,k_z)\) telles que 
      \begin{align*}
      &\left\lbrace\begin{aligned}
        \hat{\vE}_t(r_{p}^-,n,k_z) &= \mN_{E}(r_p^-,n,k_z,\hat\mR_{p}(n,k_z))\vect{C}_{p}(n,k_z),
        \\
        \hat\vJ_t(r_{p}^-,n,k_z) &= \mN_{\cH}(r_p^-,n,k_z,\hat\mR_{p}(n,k_z))\vect{C}_{p}(n,k_z),
        \end{aligned}
      \right.
      \\
      &\left\lbrace\begin{aligned}
        \vE_t(r_{p}^+,n,k_z) &= \mN_{E}(r_p^+,n,k_z,\hat\mR_{p+1}(n,k_z))\vect{C}_{p+1}(n,k_z),
        \\
        \hat\vJ_t(r_{p}^+,n,k_z) &= \mN_{\cH}(r_p^+,n,k_z,\hat\mR_{p+1}(n,k_z))\vect{C}_{p+1}(n,k_z).
        \end{aligned}
      \right.      
      \end{align*}

      Si les matrices suivantes sont inversibles
      \begin{align*}
        \mN_{E}(r_p^-,n,k_z,\mR_{p}(n,k_z)), && \mN_{\cH}(r_p^-,n,k_z,\hat\mR_{p}(n,k_z)),
        \\
        \mN_{E}(r_p^+,n,k_z,\mR_{p+1}(n,k_z)), && \mN_{\cH}(r_p^+,n,k_z,\hat\mR_{p+1}(n,k_z)),
      \end{align*}
      \begin{align*}
        \mN_{E}(r_p^-,n,k_z,\mR_{p}(n,k_z))^{-1}\mH_{E}(r_p^+,n,k_z) - \mN_{\cH}(r_p^-,n,k_z,\hat\mR_{p}(n,k_z))^{-1}\mH_{\cH}(r_p^+,n,k_z),
        \\
        \mN_{E}(r_p^+,n,k_z,\mR_{p+1}(n,k_z))^{-1}\mH_{E}(r_p^-,n,k_z) - \mN_{\cH}(r_p^+,n,k_z,\hat\mR_{p+1}(n,k_z))^{-1}\mH_{\cH}(r_p^-,n,k_z),
      \end{align*}
      alors
      \begin{align*}
        \hat\mR_{p+1}(n,k_z) &= \mathfrak{T}_p(r_p^+,r_p^-,n,k_z,\hat\mR_{p}(n,k_z)),
        \\
        \hat\mR_{p}(n,k_z) &= \mathfrak{T}_p(r_p^-,r_p^+,n,k_z,\hat\mR_{p+1}(n,k_z)).
      \end{align*}
    \end{prop}

    On peut donc déterminer itérativement les matrices de réflexion. Dans notre cadre d'étude, la présence d'un conducteur parfait sur l'interface \(r=r_0^+\) implique \(\mR_{1}(n,k_z) = -\mH_E(r_0^+,n,k_z)^{-1}\mJ_E(r_0^+,n,k_z)\).

  \subsection{Applications numériques}

    On pose \((k_x,k_y) = (k_0 s, 0)\) pour le plan.
    On compare l'impédance du plan \(\hat{\mZ}(k_x,0)\) à l'impédance du cylindre \(\hat{\mZ}(n,0)\) quand \(n\) est de l'ordre de \(k_0r_1s\) pour de faibles courbures.
    On reprend la notation de \cite[p.~62]{hoppe_impedance_1995} qui définit \(k_t= n/r_1\).

    Pour une couche de matériau sans pertes, la matrice \(\hat\mZ\) est imaginaire pure, donc les parties réelles ne sont pas tracées.

    La figure \ref{fig:imp_fourier:cylindre:hoppe_p62:converge_rayon} permet de vérifier les résultats de \cite[p.~62]{hoppe_impedance_1995} (voir Figure \ref{fig:annex:hoppe:p62}).

    \begin{figure}[!hbt]
      \centering
      \tikzsetnextfilename{Z_HOPPE_62_cylindre_converge.TM}
\begin{tikzpicture}[scale=1]
  \begin{axis}[
      title={Polarisation TM},
      ylabel={\(\Im(\hat{Z}(n,0));\Im(\hat{Z}(k_x,0))\)},
      xlabel={\(k_t \slash k_0 ; k_x \slash k_0\)},
      width=0.37\textwidth,
      xmin=0,
      xmax=1.5,
      legend pos=outer north east
    ]

    \addplot [black,dotted,mark=diamond] table [col sep=comma, x={s1}, y={Im(z_ex.tm)}] {csv/HOPPE_62/HOPPE_62.z_ex.MODE_2_TYPE_C_+3.000E-02.csv};

    \addplot [black,dotted,mark=*] table [col sep=comma, x={s1}, y={Im(z_ex.tm)}] {csv/HOPPE_62/HOPPE_62.z_ex.MODE_2_TYPE_C_+3.000E-01.csv};

    \addplot [black,dashed] table [col sep=comma, x={s1}, y={Im(z_ex.tm)}] {csv/HOPPE_62/HOPPE_62.z_ex.MODE_2_TYPE_C_+3.000E+00.csv};

    \addplot [black] table [col sep=comma, x={s1}, y={Im(z_ex.tm)}] {csv/HOPPE_62/HOPPE_62.z_ex.MODE_2_TYPE_P.csv};
  \end{axis}
\end{tikzpicture}
\tikzsetnextfilename{Z_HOPPE_62_cylindre_converge.TE}
\begin{tikzpicture}[scale=1]
  \begin{axis}[
      title={Polarisation TE},
      ylabel={},
      xlabel={\(k_t \slash k_0 ; k_x \slash k_0\)},
      width=0.37\textwidth,
      xmin=0,
      xmax=1.5,
      legend pos=outer north east
    ]

    \addplot [black,dotted,mark=diamond] table [col sep=comma, x={s1}, y={Im(z_ex.te)}] {csv/HOPPE_62/HOPPE_62.z_ex.MODE_2_TYPE_C_+3.000E-02.csv};
    \addlegendentry{\(r_0=0.03m\)}

    \addplot [black,dotted,mark=*] table [col sep=comma, x={s1}, y={Im(z_ex.te)}] {csv/HOPPE_62/HOPPE_62.z_ex.MODE_2_TYPE_C_+3.000E-01.csv};
    \addlegendentry{\(r_0=0.3m\)}

    \addplot [black,dashed] table [col sep=comma, x={s1}, y={Im(z_ex.te)}] {csv/HOPPE_62/HOPPE_62.z_ex.MODE_2_TYPE_C_+3.000E+00.csv};
    \addlegendentry{\(r_0=3m\)}

    \addplot [black] table [col sep=comma, x={s1}, y={Im(z_ex.te)}] {csv/HOPPE_62/HOPPE_62.z_ex.MODE_2_TYPE_P.csv};
    \addlegendentry{plan}
  \end{axis}
\end{tikzpicture}
      \caption{\(\eps = 6, \mu = 1, d=0.0225\text{m}, f=1\text{GHz}\)}
      \label{fig:imp_fourier:cylindre:hoppe_p62:converge_rayon}
    \end{figure}
    On voit donc que l'impédance du cylindre tend vers celle du plan quand \(r\) augmente.


    % La figure \ref{fig:imp_fourier:cylindre:hoppe_p62:coeff_fourier} affiche le module des coefficients de la série de Fourier, c'est à dire les coefficients diagonaux de la matrice \(\hat\mR\) définie plus haut. On vérifie que dès que \(n\) dépasse \(k_0r_{1}\), le module des coefficients décroit très rapidement. Ce dépassement arrive pour \(n\) supérieur à 2 (resp. 7, 64) pour \(r_0\) égal à 0.03m (resp. 0.3m, 3m).

    % \begin{figure}[!hbt]
    %   \centering
    %   \tikzsetnextfilename{F_HOPPE_62_cylindre_converge.TM}
\begin{tikzpicture}[scale=1]
  \begin{loglogaxis}[
      title={Polarisation TM},
      ylabel={\(|\hat{R}(n,0)|\)},
      xlabel={\(n\)},
      width=0.36\textwidth,
      xmin=1,
      xmax=103,
      legend pos=outer north east
    ]

    \addplot [black,dotted,mark=diamond] table [col sep=comma, x={n}, y={Abs(f_ex.11)}] {csv/HOPPE_62/HOPPE_62.f_ex.MODE_2_TYPE_C_+3.000E-02.csv};

    \addplot [black,dotted,mark=*] table [col sep=comma, x={n}, y={Abs(f_ex.11)}] {csv/HOPPE_62/HOPPE_62.f_ex.MODE_2_TYPE_C_+3.000E-01.csv};

    \addplot [black,dashed] table [col sep=comma, x={n}, y={Abs(f_ex.11)}] {csv/HOPPE_62/HOPPE_62.f_ex.MODE_2_TYPE_C_+3.000E+00.csv};

  \end{loglogaxis}
\end{tikzpicture}
\tikzsetnextfilename{F_HOPPE_62_cylindre_converge.TE}
\begin{tikzpicture}[scale=1]
  \begin{loglogaxis}[
      title={Polarisation TE},
      ylabel={},
      xlabel={\(n\)},
      width=0.36\textwidth,
      xmin=1,
      xmax=103,
      legend pos=outer north east
    ]

    \addplot [black,dotted,mark=diamond] table [col sep=comma, x={n}, y={Abs(f_ex.22)}] {csv/HOPPE_62/HOPPE_62.f_ex.MODE_2_TYPE_C_+3.000E-02.csv};
    \addlegendentry{\(r_0=0.03m\)}

    \addplot [black,dotted,mark=*] table [col sep=comma, x={n}, y={Abs(f_ex.22)}] {csv/HOPPE_62/HOPPE_62.f_ex.MODE_2_TYPE_C_+3.000E-01.csv};
    \addlegendentry{\(r_0=0.3m\)}

    \addplot [black,dashed] table [col sep=comma, x={n}, y={Abs(f_ex.22)}] {csv/HOPPE_62/HOPPE_62.f_ex.MODE_2_TYPE_C_+3.000E+00.csv};
    \addlegendentry{\(r_0=3m\)}

  \end{loglogaxis}
\end{tikzpicture}
    %   \caption{\(\eps = 6, \mu = 1, d=0.0225\text{m}, f=1\text{GHz}\)}
    %   \label{fig:imp_fourier:cylindre:hoppe_p62:coeff_fourier}
    % \end{figure}
