\section{Résultats numériques}

  Lors des calculs numériques, les opérateurs différentiels sont divisés par \(k_0^2\) pour respecter les conventions des codes internes existants.

  Ainsi la CI3 s'exprime  
  \begin{equation*}
    \left(\oI + b_1\frac{\LD}{k_0^2} -b_2\frac{\LR}{k_0^2}\right)\vE_t = \left(a_0\oI + a_1\frac{\LD}{k_0^2} - a_2 \frac{\LR}{k_0^2} \right)\vJ
  \end{equation*}
  et donc la matrice d'impédance associée est
  \begin{multline*}
    \hat{\mZ}_{CI3}(n,k_z) = \left(\mI + \frac{b_1}{k_0^2}\hat{\mLD}(n,k_z)- \frac{b_2}{k_0^2}\hat{\mLR}(n,k_z)\right)^{-1}
    \\
    \left(a_0\mI + \frac{a_1}{k_0^2}\hat{\mLD}(n,k_z)- \frac{a_2}{k_0^2}\hat{\mLR}(n,k_z)\right).
  \end{multline*}
  On remarque que cette dernière en \((0,0)\) vaut \(a_0\mI\).
  Or la matrice d'impédance exacte \(\hat\mZ(0,0)\) n'est pas multiple de l'identité.
\begin{REM}
  waouh
\end{REM}
\begin{REP}
  Commentaire non constructif
\end{REP}
  La CI3 n'est pas une bonne CIOE
\begin{REM}
  surligné: lourd
\end{REM} 
   pour approcher l'impédance exacte du cylindre.
  Une CIOE plus intéressante, que l'on nomme CI6 et qui est inspirée de \cite[p.~60]{hoppe_impedance_1995}, est alors
  \begin{equation*}
    \left(\oI + c_1\frac{\LD}{k_0^2} -c_2\frac{\LR}{k_0^2}\right)\vE_t = \left(\diag{a_1}{a_2} + b_1\frac{\LD}{k_0^2} - b_2 \frac{\LR}{k_0^2} \right)\vJ
  \end{equation*}

  La figure \ref{fig:imp_fourier:cylindre:hoppe:62:hoibc:mode_2} trace les valeurs des termes diagonaux de la matrice \(\hat\mZ\) quand les coefficients sont calculés en minimisant \(J_Z\).
  La figure \ref{fig:imp_fourier:cylindre:hoppe:62:hoibc:mode_1} trace les valeurs des termes diagonaux de la matrice \(\hat\mZ\) quand les coefficients sont calculés en minimisant \(J_R\).

  On prend une incidence normale au cylindre (\(k_z=0\)), donc les matrices d'impédance exactes et approchées sont diagonales.
  On fait varier \(n\) de 0 à \(\lceil k_0 (r_0+d) \rceil\). On trace sur ces figures les parties imaginaires des termes diagonaux pour chaque \(n\), les parties réelles étant nulles.
  Les coefficients sont calculés sans contraintes.

  \begin{figure}[!hbt]
    \centering
    \tikzsetnextfilename{Z_HOPPE_62_cylindre_hoibc_mode_2.TM}
\begin{tikzpicture}[scale=1]
    \begin{axis}[
            title={Polarisation TM},
            ylabel={\(\Im(\hat{Z}(n,0)\)},
            xlabel={\(k_t\slash k_0\)},
            width=0.4\textwidth,
            xmin=0,
            xmax=1.5,
            mark repeat=1,
            legend pos=outer north east
        ]
        \addplot [black,mark=square*] table [col sep=comma, x={s1}, y={Im(z_ex.tm)}] {csv/HOPPE_62/HOPPE_62.z_ex.MODE_2_TYPE_C_+3.000E-02.csv};

        \addplot [blue,mark=x] table [col sep=comma, x={s1}, y={Im(z_ibc0.tm)}] {csv/HOPPE_62/HOPPE_62.z_ibc.IBC_ibc0_SUC_F_MODE_2_TYPE_C_+3.000E-02.csv};

        \addplot [red,mark=diamond*] table [col sep=comma, x={s1}, y={Im(z_ibc3.tm)}] {csv/HOPPE_62/HOPPE_62.z_ibc.IBC_ibc3_SUC_F_MODE_2_TYPE_C_+3.000E-02.csv};

        \addplot [violet,mark=triangle*] table [col sep=comma, x={s1}, y={Im(z_ibc6.tm)}] {csv/HOPPE_62/HOPPE_62.z_ibc.IBC_ibc6_SUC_F_MODE_2_TYPE_C_+3.000E-02.csv};
    \end{axis}
\end{tikzpicture}
\tikzsetnextfilename{Z_HOPPE_62_cylindre_hoibc_mode_2.TE}
\begin{tikzpicture}[scale=1]
    \begin{axis}[
            title={Polarisation TE},
            ylabel={},
            xlabel={\(k_t\slash k_0\)},
            width=0.4\textwidth,
            xmin=0,
            xmax=1.5,
            mark repeat=1,
            legend pos=outer north east
        ]
        \addplot [black,mark=square*] table [col sep=comma, x={s1}, y={Im(z_ex.te)}] {csv/HOPPE_62/HOPPE_62.z_ex.MODE_2_TYPE_C_+3.000E-02.csv};
        \addlegendentry{Exact};

        \addplot [blue,mark=x] table [col sep=comma, x={s1}, y={Im(z_ibc0.te)},color=] {csv/HOPPE_62/HOPPE_62.z_ibc.IBC_ibc0_SUC_F_MODE_2_TYPE_C_+3.000E-02.csv};
        \addlegendentry{CI0};

        \addplot [red,mark=diamond*] table [col sep=comma, x={s1}, y={Im(z_ibc3.te)}] {csv/HOPPE_62/HOPPE_62.z_ibc.IBC_ibc3_SUC_F_MODE_2_TYPE_C_+3.000E-02.csv};
        \addlegendentry{CI3};

        \addplot [violet,mark=triangle*] table [col sep=comma, x={s1}, y={Im(z_ibc6.te)}] {csv/HOPPE_62/HOPPE_62.z_ibc.IBC_ibc6_SUC_F_MODE_2_TYPE_C_+3.000E-02.csv};
        \addlegendentry{CI6};
    \end{axis}
\end{tikzpicture}
    \caption[Partie imaginaire de l'opérateur de Calderón, compararé avec les approximations CI0, CI3, CI6 (calculées avec Jz) pour une couche cylindrique de matériau de Hoppe \& Rahmat-Samii.]{Minimisation de \(J_Z\): Partie imaginaire des termes diagonaux des matrices d'impédance pour l'empilement \(\eps = 6\), \(\mu = 1\), \(d=0.0225\)m, \(f=1\)GHz, \(r_0=0.03\)m en fonction de \(k_t = n / (r_0+d)\).}
    \label{fig:imp_fourier:cylindre:hoppe:62:hoibc:mode_2}
  \end{figure}

  \begin{figure}[!hbt]
    \centering
    \tikzsetnextfilename{Z_HOPPE_62_cylindre_hoibc_mode_1.TM}
\begin{tikzpicture}[scale=1]
    \begin{axis}[
            title={Polarisation TM},
            ylabel={\(\Im(\hat{Z}(n,0)\)},
            xlabel={\(k_t\slash k_0\)},
            width=0.4\textwidth,
            xmin=0,
            xmax=1.5,
            mark repeat=1,
            legend pos=outer north east
        ]
        \addplot [black,mark=square*] table [col sep=comma, x={s1}, y={Im(z_ex.11)}] {csv/HOPPE_62/HOPPE_62.z_ex.MODE_1_TYPE_C_+3.000E-02m.csv};

        \addplot [blue,mark=x] table [col sep=comma, x={s1}, y={Im(z_ibc0.11)}] {csv/HOPPE_62/HOPPE_62.z_ibc.IBC_ibc0_SUC_F_MODE_1_TYPE_C_+3.000E-02m.csv};

        \addplot [red,mark=diamond*] table [col sep=comma, x={s1}, y={Im(z_ibc3.11)}] {csv/HOPPE_62/HOPPE_62.z_ibc.IBC_ibc3_SUC_F_MODE_1_TYPE_C_+3.000E-02m.csv};

        \addplot [purple,mark=triangle*] table [col sep=comma, x={s1}, y={Im(z_ibc6.11)}] {csv/HOPPE_62/HOPPE_62.z_ibc.IBC_ibc6_SUC_F_MODE_1_TYPE_C_+3.000E-02m.csv};
    \end{axis}
\end{tikzpicture}
\tikzsetnextfilename{Z_HOPPE_62_cylindre_hoibc_mode_1.TE}
\begin{tikzpicture}[scale=1]
    \begin{axis}[
            title={Polarisation TE},
            ylabel={},
            xlabel={\(k_t\slash k_0\)},
            width=0.4\textwidth,
            xmin=0,
            xmax=1.5,
            mark repeat=1,
            legend pos=outer north east
        ]
        \addplot [black,mark=square*] table [col sep=comma, x={s1}, y={Im(z_ex.22)}] {csv/HOPPE_62/HOPPE_62.z_ex.MODE_1_TYPE_C_+3.000E-02m.csv};
        \addlegendentry{Exact};

        \addplot [blue,mark=x] table [col sep=comma, x={s1}, y={Im(z_ibc0.22)}] {csv/HOPPE_62/HOPPE_62.z_ibc.IBC_ibc0_SUC_F_MODE_1_TYPE_C_+3.000E-02m.csv};
        \addlegendentry{CI0};

        \addplot [red,mark=diamond*] table [col sep=comma, x={s1}, y={Im(z_ibc3.22)}] {csv/HOPPE_62/HOPPE_62.z_ibc.IBC_ibc3_SUC_F_MODE_1_TYPE_C_+3.000E-02m.csv};
        \addlegendentry{CI3};

        \addplot [purple,mark=triangle*] table [col sep=comma, x={s1}, y={Im(z_ibc6.22)}] {csv/HOPPE_62/HOPPE_62.z_ibc.IBC_ibc6_SUC_F_MODE_1_TYPE_C_+3.000E-02m.csv};
        \addlegendentry{CI6};
    \end{axis}
\end{tikzpicture}
    \caption[Partie imaginaire de l'opérateur de Calderón, compararé avec les approximations CI0, CI3, CI6 (calculées avec Jr)  pour une couche cylindrique de matériau de Hoppe \& Rahmat-Samii.]{Minimisation de \(J_R\): Partie imaginaire des termes diagonaux des matrices d'impédance pour l'empilement \(\eps = 6\), \(\mu = 1\), \(d=0.0225\)m, \(f=1\)GHz, \(r_0=0.03\)m en fonction de \(k_t = n / (r_0+d)\).}
    \label{fig:imp_fourier:cylindre:hoppe:62:hoibc:mode_1}
  \end{figure}

  \FloatBarrier

  On voit donc bien que les deux méthodes donnent des résultats différents, mais dans les deux cas, la CI3 est une mauvaise approximation tandis que le calcul des coefficients de la CI6 par la minimisation de \(J_R\) donne la meilleure approximation de l'impédance.

  Cependant cette CIOE ne sera pas retenue car son implémentation dans le code équation intégrale nécessite une modification de ce dernier et nous n'avons pas de CSU.
  \begin{table}[!hbt]
    \centering
    % On fait deux tables de même hauteur
    \begin{minipage}[t]{0.49\textwidth}
      \vspace{0pt}
      \centering
      \begin{coefftable}{\hyperlink{ci0}{CI0}}
        \input{csv/HOPPE_62/HOPPE_62.IBC_ibc0_SUC_F_MODE_2_TYPE_C_+3.000E-02m.coeff.txt}
      \end{coefftable}
      \begin{coefftable}{\hyperlink{ci3}{CI3}}
        \input{csv/HOPPE_62/HOPPE_62.IBC_ibc3_SUC_F_MODE_2_TYPE_C_+3.000E-02m.coeff.txt}
      \end{coefftable}
    \end{minipage}
    \begin{minipage}[t]{0.49\textwidth}
      \vspace{0pt}
      \centering
      \begin{coefftable}{\hyperlink{ci6}{CI6}}
        \input{csv/HOPPE_62/HOPPE_62.IBC_ibc6_SUC_F_MODE_2_TYPE_C_+3.000E-02m.coeff.txt}
      \end{coefftable}
    \end{minipage}
    \caption{Coefficients associés à la figure \ref{fig:imp_fourier:cylindre:hoppe:62:hoibc:mode_2}}
    \label{tab:imp_fourier:cylindre:hoppe:62:hoibc:mode_2}
  \end{table}
  \begin{table}[!hbt]
    \centering
    % On fait deux tables de même hauteur
    \begin{minipage}[t]{0.49\textwidth}
      \vspace{0pt}
      \centering
      \begin{coefftable}{\hyperlink{ci0}{CI0}}
        \input{csv/HOPPE_62/HOPPE_62.IBC_ibc0_SUC_F_MODE_1_TYPE_C_+3.000E-02m.coeff.txt}
      \end{coefftable}
      \begin{coefftable}{\hyperlink{ci3}{CI3}}
        \input{csv/HOPPE_62/HOPPE_62.IBC_ibc3_SUC_F_MODE_1_TYPE_C_+3.000E-02m.coeff.txt}
      \end{coefftable}
    \end{minipage}
    \begin{minipage}[t]{0.49\textwidth}
      \vspace{0pt}
      \centering
      \begin{coefftable}{\hyperlink{ci6}{CI6}}
        \input{csv/HOPPE_62/HOPPE_62.IBC_ibc6_SUC_F_MODE_1_TYPE_C_+3.000E-02m.coeff.txt}
      \end{coefftable}
    \end{minipage}
    \caption{Coefficients associés à la figure \ref{fig:imp_fourier:cylindre:hoppe:62:hoibc:mode_1}}
    \label{tab:imp_fourier:cylindre:hoppe:62:hoibc:mode_1}
  \end{table}

  \pagebreak
  Les tableaux \ref{tab:cylindre:hoppe:62:erreurs_Z} contient les valeurs des erreurs relatives au carré de \(\hat\mZ\). 
  Plus précisément, l'expression de ces erreurs est 
  \begin{align*}
    \text{TM} &: \frac{\sum\limits_{i=1}^{N_n}\sum\limits_{j=1}^{N_{k_z}}|\hat\mZ_{ex}(n_i,k_{zj})_{11}-\hat\mZ_{ap}(n_i,k_{zj})_{11}|^2}{\sum\limits_{i=1}^{N_n}\sum\limits_{j=1}^{N_{k_z}}|\hat\mZ_{ex}(n_i,k_{zj})_{11}|^2},
    \\
    \text{TE} &: \frac{\sum\limits_{i=1}^{N_n}\sum\limits_{j=1}^{N_{k_z}}|\hat\mZ_{ex}(n_i,k_{zj})_{22}-\hat\mZ_{ap}(n_i,k_{zj})_{22}|^2}{\sum\limits_{i=1}^{N_n}\sum\limits_{j=1}^{N_{k_z}}|\hat\mZ_{ex}(n_i,k_{zj})_{22}|^2},
    \\
    \text{complète} &: \frac{\sum\limits_{p=1}^2\sum\limits_{q=1}^2\sum\limits_{i=1}^{N_n}\sum\limits_{j=1}^{N_{k_z}}|\hat\mZ_{ex}(n_i,k_{zj})_{pq}-\hat\mZ_{ap}(n_i,k_{zj})_{pq}|^2}{\sum\limits_{p=1}^2\sum\limits_{q=1}^2\sum\limits_{i=1}^{N_n}\sum\limits_{j=1}^{N_{k_z}}|\hat\mZ_{ex}(n_i,k_{zj})_{pq}|^2}.
  \end{align*}

  \begin{table}[!hbt]
    \centering
    \begin{tabular}{l|ccc|ccc}
      CIOE & \multicolumn{3}{c}{Minimisation \(J_R\)} & \multicolumn{3}{c}{Minimisation \(J_Z\)}\\
      \hline
      \hline
          & {TM} & {TE} & {complète} & {TM} & {TE} & {complète}\\
      \hline
      CI6 & \verb|1.62E-01| & \verb|3.38E-04| & \verb|8.23E-02| & \verb|1.17E-02| & \verb|5.60E-04| & \verb|6.23E-03|\\
      CI3 & \verb|4.30E-01| & \verb|3.99E-01| & \verb|4.15E-01| & \verb|6.82E-02| & \verb|1.71E-01| & \verb|1.19E-01|\\
      CI0 & \verb|5.15E+00| & \verb|1.16E+00| & \verb|3.18E+00| & \verb|5.15E+00| & \verb|1.16E+00| & \verb|3.18E+00|
    \end{tabular}
    \caption{Erreur relative sur les matrices d'impédance \(\hat\mZ\).}
    \label{tab:cylindre:hoppe:62:erreurs_Z}
  \end{table}

  Les tableaux \ref{tab:cylindre:hoppe:62:erreurs_R} contiennent les valeurs des erreurs relatives au carré sur \(\hat\mR\) dont l'expression est 
  \begin{align*}
    \text{TM} &: \sqrt{\frac{\sum\limits_{i=1}^{N_n}\sum\limits_{j=1}^{N_{k_z}}|\hat\mR_{ex}(n_i,k_{zj})_{11}-\hat\mR_{ap}(n_i,k_{zj})_{11}|^2}{\sum\limits_{i=1}^{N_n}\sum\limits_{j=1}^{N_{k_z}}|\hat\mR_{ex}(n_i,k_{zj})_{11}|^2}},
    \\
    \text{TE} &: \sqrt{\frac{\sum\limits_{i=1}^{N_n}\sum\limits_{j=1}^{N_{k_z}}|\hat\mR_{ex}(n_i,k_{zj})_{22}-\hat\mR_{ap}(n_i,k_{zj})_{22}|^2}{\sum\limits_{i=1}^{N_n}\sum\limits_{j=1}^{N_{k_z}}|\hat\mR_{ex}(n_i,k_{zj})_{22}|^2}},
    \\
    \text{somme}  &: \text{TE} + \text{TM}.
  \end{align*}

  \begin{table}[!hbt]
    \centering
    \begin{tabular}{l|ccc|ccc}
      CIOE & \multicolumn{3}{c}{Minimisation \(J_R\)} & \multicolumn{3}{c}{Minimisation \(J_Z\)}\\
      \hline
      \hline
          & {TM} & {TE} & {somme} & {TM} & {TE} & {somme}\\
      \hline 
      CI6 & \verb|1.50E-04| & \verb|2.56E-03| & \verb|2.71E-03| & \verb|1.03E-02| & \verb|2.69E-01| & \verb|2.79E-01|\\
      CI3 & \verb|3.31E-01| & \verb|2.33E-01| & \verb|5.64E-01| & \verb|3.00E-01| & \verb|7.14E-01| & \verb|1.01E+00|\\
      CI0 & \verb|1.56E-01| & \verb|1.73E+00| & \verb|1.88E+00| & \verb|1.56E-01| & \verb|1.73E+00| & \verb|1.88E+00|  
    \end{tabular}
    \caption{Erreur relative sur les matrices de réflexion \(\hat\mR\).}
    \label{tab:cylindre:hoppe:62:erreurs_R}
  \end{table}
