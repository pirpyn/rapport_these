\section{Résultats numériques}

  Dans cette partie, toutes les CIOE faisant intervenir des opérateurs différentielles normalisent ces derniers par \(k_0^2\). Ainsi la CI3 s'exprime
    
  \begin{equation*}
    \left(\oI + c_1\frac{\LD}{k_0^2} -c_2\frac{\LR}{k_0^2}\right)\vE_t = \left(a_0\oI + b_1\frac{\LD}{k_0^2} - b_2 \frac{\LR}{k_0^2} \right)\vJ
  \end{equation*}



  La figure \ref{fig:imp_fourier:cylindre:hoppe:62:hoibc:mode_2} trace les valeurs des termes diagonaux de la matrice \(\hat\mZ\) quand les coefficients sont calculés en minimisant \(J_Z\).
  La figure \ref{fig:imp_fourier:cylindre:hoppe:62:hoibc:mode_1} trace les valeurs des termes diagonaux de la matrice \(\hat\mZ\) quand les coefficients sont calculés en minimisant \(J_R\).

  On prend un incidence normale au cylindre, donc les matrices d'impédance exactes et approchées sont diagonales. On fait varier \(n\) de 0 à \(\lceil k_0 (r_0+d) \rceil\). On trace sur cette figure les parties imaginaires des termes diagonaux pour chaque \(n\) de cette matrice, les parties réelles étant nulles. Les coefficients sont calculés sans contraintes.

  \begin{figure}[!hbt]
    \centering
    \tikzsetnextfilename{Z_HOPPE_62_cylindre_hoibc_mode_2.TM}
\begin{tikzpicture}[scale=1]
    \begin{axis}[
            title={Polarisation TM},
            ylabel={\(\Im(\hat{Z}(n,0)\)},
            xlabel={\(k_t\slash k_0\)},
            width=0.4\textwidth,
            xmin=0,
            xmax=1.5,
            mark repeat=1,
            legend pos=outer north east
        ]
        \addplot [black,mark=square*] table [col sep=comma, x={s1}, y={Im(z_ex.tm)}] {csv/HOPPE_62/HOPPE_62.z_ex.MODE_2_TYPE_C_+3.000E-02.csv};

        \addplot [blue,mark=x] table [col sep=comma, x={s1}, y={Im(z_ibc0.tm)}] {csv/HOPPE_62/HOPPE_62.z_ibc.IBC_ibc0_SUC_F_MODE_2_TYPE_C_+3.000E-02.csv};

        \addplot [red,mark=diamond*] table [col sep=comma, x={s1}, y={Im(z_ibc3.tm)}] {csv/HOPPE_62/HOPPE_62.z_ibc.IBC_ibc3_SUC_F_MODE_2_TYPE_C_+3.000E-02.csv};

        \addplot [violet,mark=triangle*] table [col sep=comma, x={s1}, y={Im(z_ibc6.tm)}] {csv/HOPPE_62/HOPPE_62.z_ibc.IBC_ibc6_SUC_F_MODE_2_TYPE_C_+3.000E-02.csv};
    \end{axis}
\end{tikzpicture}
\tikzsetnextfilename{Z_HOPPE_62_cylindre_hoibc_mode_2.TE}
\begin{tikzpicture}[scale=1]
    \begin{axis}[
            title={Polarisation TE},
            ylabel={},
            xlabel={\(k_t\slash k_0\)},
            width=0.4\textwidth,
            xmin=0,
            xmax=1.5,
            mark repeat=1,
            legend pos=outer north east
        ]
        \addplot [black,mark=square*] table [col sep=comma, x={s1}, y={Im(z_ex.te)}] {csv/HOPPE_62/HOPPE_62.z_ex.MODE_2_TYPE_C_+3.000E-02.csv};
        \addlegendentry{Exact};

        \addplot [blue,mark=x] table [col sep=comma, x={s1}, y={Im(z_ibc0.te)},color=] {csv/HOPPE_62/HOPPE_62.z_ibc.IBC_ibc0_SUC_F_MODE_2_TYPE_C_+3.000E-02.csv};
        \addlegendentry{CI0};

        \addplot [red,mark=diamond*] table [col sep=comma, x={s1}, y={Im(z_ibc3.te)}] {csv/HOPPE_62/HOPPE_62.z_ibc.IBC_ibc3_SUC_F_MODE_2_TYPE_C_+3.000E-02.csv};
        \addlegendentry{CI3};

        \addplot [violet,mark=triangle*] table [col sep=comma, x={s1}, y={Im(z_ibc6.te)}] {csv/HOPPE_62/HOPPE_62.z_ibc.IBC_ibc6_SUC_F_MODE_2_TYPE_C_+3.000E-02.csv};
        \addlegendentry{CI6};
    \end{axis}
\end{tikzpicture}
    \caption[CIOE sur empilement de Hoppe & Rahmat-Samii p.~62]{Partie imaginaire des termes diagonaux des matrices d'impédance pour l'empilement \(\eps = 6\), \(\mu = 1\), \(d=0.0225\text{m}\), \(f=1\text{GHz}\), \(r_0=0.03\text{m}\) en fonction de \(k_t = n / (r_0+d)\).}
    \label{fig:imp_fourier:cylindre:hoppe:62:hoibc:mode_2}
  \end{figure}

  \begin{figure}[!hbt]
    \centering
    \tikzsetnextfilename{Z_HOPPE_62_cylindre_hoibc_mode_1.TM}
\begin{tikzpicture}[scale=1]
    \begin{axis}[
            title={Polarisation TM},
            ylabel={\(\Im(\hat{Z}(n,0)\)},
            xlabel={\(k_t\slash k_0\)},
            width=0.4\textwidth,
            xmin=0,
            xmax=1.5,
            mark repeat=1,
            legend pos=outer north east
        ]
        \addplot [black,mark=square*] table [col sep=comma, x={s1}, y={Im(z_ex.11)}] {csv/HOPPE_62/HOPPE_62.z_ex.MODE_1_TYPE_C_+3.000E-02m.csv};

        \addplot [blue,mark=x] table [col sep=comma, x={s1}, y={Im(z_ibc0.11)}] {csv/HOPPE_62/HOPPE_62.z_ibc.IBC_ibc0_SUC_F_MODE_1_TYPE_C_+3.000E-02m.csv};

        \addplot [red,mark=diamond*] table [col sep=comma, x={s1}, y={Im(z_ibc3.11)}] {csv/HOPPE_62/HOPPE_62.z_ibc.IBC_ibc3_SUC_F_MODE_1_TYPE_C_+3.000E-02m.csv};

        \addplot [purple,mark=triangle*] table [col sep=comma, x={s1}, y={Im(z_ibc6.11)}] {csv/HOPPE_62/HOPPE_62.z_ibc.IBC_ibc6_SUC_F_MODE_1_TYPE_C_+3.000E-02m.csv};
    \end{axis}
\end{tikzpicture}
\tikzsetnextfilename{Z_HOPPE_62_cylindre_hoibc_mode_1.TE}
\begin{tikzpicture}[scale=1]
    \begin{axis}[
            title={Polarisation TE},
            ylabel={},
            xlabel={\(k_t\slash k_0\)},
            width=0.4\textwidth,
            xmin=0,
            xmax=1.5,
            mark repeat=1,
            legend pos=outer north east
        ]
        \addplot [black,mark=square*] table [col sep=comma, x={s1}, y={Im(z_ex.22)}] {csv/HOPPE_62/HOPPE_62.z_ex.MODE_1_TYPE_C_+3.000E-02m.csv};
        \addlegendentry{Exact};

        \addplot [blue,mark=x] table [col sep=comma, x={s1}, y={Im(z_ibc0.22)}] {csv/HOPPE_62/HOPPE_62.z_ibc.IBC_ibc0_SUC_F_MODE_1_TYPE_C_+3.000E-02m.csv};
        \addlegendentry{CI0};

        \addplot [red,mark=diamond*] table [col sep=comma, x={s1}, y={Im(z_ibc3.22)}] {csv/HOPPE_62/HOPPE_62.z_ibc.IBC_ibc3_SUC_F_MODE_1_TYPE_C_+3.000E-02m.csv};
        \addlegendentry{CI3};

        \addplot [purple,mark=triangle*] table [col sep=comma, x={s1}, y={Im(z_ibc6.22)}] {csv/HOPPE_62/HOPPE_62.z_ibc.IBC_ibc6_SUC_F_MODE_1_TYPE_C_+3.000E-02m.csv};
        \addlegendentry{CI6};
    \end{axis}
\end{tikzpicture}
    \caption[CIOE sur empilement de Hoppe & Rahmat-Samii p.~62]{Partie imaginaire des termes diagonaux des matrices d'impédance pour l'empilement \(\eps = 6\), \(\mu = 1\), \(d=0.0225\text{m}\), \(f=1\text{GHz}\), \(r_0=0.03\text{m}\) en fonction de \(k_t = n / (r_0+d)\).}
    \label{fig:imp_fourier:cylindre:hoppe:62:hoibc:mode_1}
  \end{figure}

  On remarque que la CI3 si performante dans l'approximation plan infini ne donnent pas de bons résultats dans l’approximation cylindre infini. 
  En effet, pour \(n=k_z=0\) la matrice d'impédance exacte n'est pas une constante mais une matrice diagonale. 
  Or par définition la CIOE est une constante pour ce couple, c'est la CI0. On subit donc cette erreur dans les résultats. 

  Une CIOE plus intéressante, que l'on nomme CI6 et qui est inspirée de \cite[p.~60]{hoppe_impedance_1995}, serait alors:

  \begin{equation*}
    \left(\oI + c_1\frac{\LD}{k_0^2} -c_2\frac{\LR}{k_0^2}\right)\vE_t = \left(\diag{a_1}{a_2} + b_1\frac{\LD}{k_0^2} - b_2 \frac{\LR}{k_0^2} \right)\vJ
  \end{equation*}

  Cependant cette CIOE ne sera pas retenue car son implémentation dans le code équation intégrale nécessite une modification de ce dernier.

  \begin{table}[!hbt]
    \centering
    % On fait deux tables de même hauteur
    \begin{minipage}[t]{0.49\textwidth}
      \vspace{0pt}
      \centering
      \begin{coefftable}{\hyperlink{ci0}{CI0}}
        \input{csv/HOPPE_62/HOPPE_62.IBC_ibc0_SUC_F_MODE_2_TYPE_C_+3.000E-02.coeff.txt}
      \end{coefftable}
      \begin{coefftable}{\hyperlink{ci3}{CI3}}
        \input{csv/HOPPE_62/HOPPE_62.IBC_ibc3_SUC_F_MODE_2_TYPE_C_+3.000E-02.coeff.txt}
      \end{coefftable}
    \end{minipage}
    \begin{minipage}[t]{0.49\textwidth}
      \vspace{0pt}
      \centering
      \begin{coefftable}{\hyperlink{ci6}{CI6}}
        \input{csv/HOPPE_62/HOPPE_62.IBC_ibc6_SUC_F_MODE_2_TYPE_C_+3.000E-02.coeff.txt}
      \end{coefftable}
    \end{minipage}
    \caption{Coefficients associés à la figure \ref{fig:imp_fourier:cylindre:hoppe:62:hoibc:mode_2}}
    \label{tab:imp_fourier:cylindre:hoppe:62:hoibc:mode_2}
  \end{table}


  \begin{table}[!hbt]
    \centering
    % On fait deux tables de même hauteur
    \begin{minipage}[t]{0.49\textwidth}
      \vspace{0pt}
      \centering
      \begin{coefftable}{\hyperlink{ci0}{CI0}}
        \input{csv/HOPPE_62/HOPPE_62.IBC_ibc0_SUC_F_MODE_1_TYPE_C_+3.000E-02.coeff.txt}
      \end{coefftable}
      \begin{coefftable}{\hyperlink{ci3}{CI3}}
        \input{csv/HOPPE_62/HOPPE_62.IBC_ibc3_SUC_F_MODE_1_TYPE_C_+3.000E-02.coeff.txt}
      \end{coefftable}
    \end{minipage}
    \begin{minipage}[t]{0.49\textwidth}
      \vspace{0pt}
      \centering
      \begin{coefftable}{\hyperlink{ci6}{CI6}}
        \input{csv/HOPPE_62/HOPPE_62.IBC_ibc6_SUC_F_MODE_1_TYPE_C_+3.000E-02.coeff.txt}
      \end{coefftable}
    \end{minipage}
    \caption{Coefficients associés à la figure \ref{fig:imp_fourier:cylindre:hoppe:62:hoibc:mode_1}}
    \label{tab:imp_fourier:cylindre:hoppe:62:hoibc:mode_1}
  \end{table}

  Les tableaux \ref{tab:cylindre:hoppe:62:erreurs_Z} contient les valeurs des erreurs relatives au carré \(\hat\mZ\). 
  Plus précisément, l'expression de ces erreurs est 
  \begin{align*}
    \text{TM} &= \frac{\sum\limits_{i=1}^{N_n}\sum\limits_{j=1}^{N_{k_z}}|\hat\mZ_{ex}(n_i,k_{zj})_{11}-\hat\mZ_{ap}(n_i,k_{zj})_{11}|^2}{\sum\limits_{i=1}^{N_n}\sum\limits_{j=1}^{N_{k_z}}|\hat\mZ_{ex}(n_i,k_{zj})_{11}|^2}
  \end{align*}
  \begin{align*}
    \text{TE} &= \frac{\sum\limits_{i=1}^{N_n}\sum\limits_{j=1}^{N_{k_z}}|\hat\mZ_{ex}(n_i,k_{zj})_{22}-\hat\mZ_{ap}(n_i,k_{zj})_{22}|^2}{\sum\limits_{i=1}^{N_n}\sum\limits_{j=1}^{N_{k_z}}|\hat\mZ_{ex}(n_i,k_{zj})_{22}|^2}
  \end{align*}
  \begin{align*}
    \text{complète}  &= \frac{\sum\limits_{p=1}^2\sum\limits_{q=1}^2\sum\limits_{i=1}^{N_n}\sum\limits_{j=1}^{N_{k_z}}|\hat\mZ_{ex}(n_i,k_{zj})_{pq}-\hat\mZ_{ap}(n_i,k_{zj})_{pq}|^2}{\sum\limits_{p=1}^2\sum\limits_{q=1}^2\sum\limits_{i=1}^{N_n}\sum\limits_{j=1}^{N_{k_z}}|\hat\mZ_{ex}(n_i,k_{zj})_{pq}|^2}
  \end{align*}

  \begin{table}[!hbt]
    \centering
    \begin{tabular}{l|ccc|ccc}
      CIOE & \multicolumn{3}{c}{Minimisation \(J_R\)} & \multicolumn{3}{c}{Minimisation \(J_Z\)}\\
      \hline
      \hline
          & {TM} & {TE} & {complète} & {TM} & {TE} & {complète}\\
      \hline
      CI6 & \verb|1.62E-01| & \verb|3.38E-04| & \verb|8.23E-02| & \verb|1.17E-02| & \verb|5.60E-04| & \verb|6.23E-03|\\
      CI3 & \verb|4.30E-01| & \verb|3.99E-01| & \verb|4.15E-01| & \verb|6.82E-02| & \verb|1.71E-01| & \verb|1.19E-01|\\
      CI0 & \verb|5.15E+00| & \verb|1.16E+00| & \verb|3.18E+00| & \verb|5.15E+00| & \verb|1.16E+00| & \verb|3.18E+00|
    \end{tabular}
    \caption{Erreur relative sur les matrices d'impédance \(\hat\mZ\).}
    \label{tab:cylindre:hoppe:62:erreurs_Z}
  \end{table}

  Les tableaux \ref{tab:cylindre:hoppe:62:erreurs_R} contient les valeurs des erreurs relatives au carré sur \(\hat\mR\). 

  Plus précisément, l'expression de ces erreurs est 
  \begin{align*}
    \text{TM} &= \sqrt{\frac{\sum\limits_{i=1}^{N_n}\sum\limits_{j=1}^{N_{k_z}}|\hat\mR_{ex}(n_i,k_{zj})_{11}-\hat\mR_{ap}(n_i,k_{zj})_{11}|^2}{\sum\limits_{i=1}^{N_n}\sum\limits_{j=1}^{N_{k_z}}|\hat\mR_{ex}(n_i,k_{zj})_{11}|^2}}
  \end{align*}
  \begin{align*}
    \text{TE} &= \sqrt{\frac{\sum\limits_{i=1}^{N_n}\sum\limits_{j=1}^{N_{k_z}}|\hat\mR_{ex}(n_i,k_{zj})_{22}-\hat\mR_{ap}(n_i,k_{zj})_{22}|^2}{\sum\limits_{i=1}^{N_n}\sum\limits_{j=1}^{N_{k_z}}|\hat\mR_{ex}(n_i,k_{zj})_{22}|^2}}
  \end{align*}
  \begin{align*}
    \text{somme}  &= \text{TE} + \text{TM}
  \end{align*}

  \begin{table}[!hbt]
    \centering
    \begin{tabular}{l|ccc|ccc}
      CIOE & \multicolumn{3}{c}{Minimisation \(J_R\)} & \multicolumn{3}{c}{Minimisation \(J_Z\)}\\
      \hline
      \hline
          & {TM} & {TE} & {somme} & {TM} & {TE} & {somme}\\
      \hline 
      CI6 & \verb|1.50E-04| & \verb|2.56E-03| & \verb|2.71E-03| & \verb|1.03E-02| & \verb|2.69E-01| & \verb|2.79E-01|\\
      CI3 & \verb|3.31E-01| & \verb|2.33E-01| & \verb|5.64E-01| & \verb|3.00E-01| & \verb|7.14E-01| & \verb|1.01E+00|\\
      CI0 & \verb|1.56E-01| & \verb|1.73E+00| & \verb|1.88E+00| & \verb|1.56E-01| & \verb|1.73E+00| & \verb|1.88E+00|  
    \end{tabular}
    \caption{Erreur relative sur les matrices de réflexion \(\hat\mR\).}
    \label{tab:cylindre:hoppe:62:erreurs_R}
  \end{table}
