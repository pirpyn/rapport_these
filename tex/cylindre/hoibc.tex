\section{Approximation de la matrice d'impédance pour un cylindre infini par une CIOE}

  \subsection[Expression des opérateurs LD,LR en Fourier]{Expression des opérateurs \(\LD,\LR\) en Fourier}

    Par définition de \(\LD\), on a
    \begin{align}
      \LD \vE_t & = \vgrads{} \vdivs{} \vE_t
    \end{align}

    Or d’après la définition de la transformée de Fourier

    \begin{align}
      \vE_t(r,\theta,z) & = \frac{1}{2\pi}\sum_{i=-\infty}^\infty \int_\RR \hat{\vE_t}(r,n,k_z)e^{in\theta + ik_zz}\dd{k_z}
    \end{align}

    On applique l'opérateur à ce vecteur

    \begin{align}
      \LD \vE_t
      & = \vgrads{} \vdivs{} \vE_t
      \\
      &=\frac{1}{2\pi}\vgrads{} \sum_{i=-\infty}^\infty\int_\RR \hat{\vE_t}(r,n,k_z) \cdot \vgrads{} e^{in\theta + ik_zz}\dd{k_z}
      \\
      &=\frac{1}{2\pi}\sum_{i=-\infty}^\infty \int_\RR \vhesss{}\left(\left( e^{in\theta + ik_zz} \right) \hat{\vE_t}\right)(r,n,k_z)\dd{k_z}
    \end{align}

    On utilise les expression en coordonnées cylindrique des opérateurs différentielles ( voir annexe \ref{sec:annexe:div_grad_rot} ):

    On définit \(\hat{\mLD}\) l'opérateur matriciel tel que
    \begin{align}
      \LD \vE_t
      &= \frac{1}{2\pi}\sum_{i=-\infty}^\infty\int_\RR \hat{\mLD} \hat{\vE_t}(r,n,k_z)\dd{k_z}
    \end{align}

    Son expression est de ce qui précède
    \begin{equation}
      \label{eq:cylindre:fourier:LD}
      \hat{\LD}(n,k_z) = -
      \begin{bmatrix}
        \left({n}\slash{r_{ext}}\right)^2 & k_z{n}\slash{r_{ext}}
        \\
        k_z{n}\slash{r_{ext}} & k_z^2
      \end{bmatrix}
    \end{equation}

    On reprend exactement la même méthode pour l'opérateur \(\LR\):
    \begin{align}
      \LR \vE_t & = \vrots{} \vrots{} \vE_t
      \\
      &=\frac{1}{2\pi}\vrots{}\sum_{i=-\infty}^\infty\int_\RR \vgrads{}\left(e^{in\theta + ik_zz}\right) \pvect \hat{\vE_t}(r,n,k_z)\dd{k_z}
      \\
      &= \frac{1}{2\pi}\sum_{i=-\infty}^\infty \int_\RR \left(\vhesss - \vlapls\right) \left(\left(e^{in\theta + ik_zz}\right) \hat{\vE_t}\right)(r,n,k_z)\dd{k_z}
    \end{align}

    On définit \(\hat{\LR}\) l'opérateur matriciel tel que
    \begin{align}
      \LR \vE_t
      &= \frac{1}{2\pi}\sum_{i=-\infty}^\infty\int_\RR \hat{\LR} \hat{\vE_t}(r,n,k_z)\dd{k_z}
    \end{align}

    \begin{equation}
      \hat{\LR}(n,k_z) =
      \begin{bmatrix}
        k_z^2 & -k_z{n}\slash{r_{ext}}
        \\
        -k_z{n}\slash{r_{ext}} & \left({n}\slash{r_{ext}} \right)^2
      \end{bmatrix}
    \end{equation}

  \subsection{Expression de la matrice d'impédance approchée par la CI3}

    Tout comme dans le cas du plan infini, on peut donc définir \(\hat{\mZ}_{IBC}\) l’opérateur matriciel associé à la condition d'impédance.

    \begin{multline}
        \hat{\mZ}_{IBC}(n,k_z) = \left(I + b_1 \frac{\hat{\LD}(n,k_z)}{k_0^2} - b_2 \frac{\hat{\LR}(n,k_z)}{k_0^2} \right)^{-1}\\
        \left(a_0 I + a_1 \frac{\hat{\LD}(n,k_z)}{k_0^2} - a_2 \frac{\hat{\LR}(n,k_z)}{k_0^2}\right)
    \end{multline}
