\section{Approximation de la matrice d'impédance pour un cylindre infini par une CIOE}

  \subsection[Expression des opérateurs LD,LR en Fourier]{Expression des opérateurs \(\LD,\LR\) en Fourier}
    Soit \(C(0,r_C)\) un cylindre de centre 0, de rayon \(r_C\) et d'axe \(\vect{e_z}\) et \((r,\theta,z)\) les coordonnées cylindre d'un point de l'espace.

    Soit \(V = \left(\mathcal{C}^\infty(C(0,r_C))\right)^2 \cap L^2(C(0,r_C)))\) l'espace des fonctions infiniment dérivables et de carré intégrable.

    \begin{defn}
      \label{eq:cylindre:fourier:LD}
      On définit \(\LD\) l'endomorphisme de \(V\) tel que
      \begin{align*}
        \LD \vect{U}(r,\theta,z) & = \vgrads{} \vdivs{} \vect{U}(r,\theta,z)
      \end{align*}

      On définit \(\hat{\mLD}\) la fonction de \(\NN\times\RR \rightarrow \mathcal{M}_2(\RR)\) telle que
      \begin{equation*}
        \hat{\mLD}(n,k_z) = -
        \begin{bmatrix}
          \left({n}\slash{r_C}\right)^2 & k_z{n}\slash{r_C}
          \\
          k_z{n}\slash{r_C} & k_z^2
        \end{bmatrix}
      \end{equation*}
    \end{defn}

    \begin{prop}
      Soit \(\vect{U} \in V\)
      Alors
      \begin{equation*}
        \widehat{\LD \vect{U}} (r_C,n,k_z) = \hat{\mLD}(n,k_z) \hat{\vect{U}}(r_C,n,k_z)
      \end{equation*}
    \end{prop}

    \begin{proof}
      Par définition de \(\LD\), on a
      \begin{align*}
        \LD \vect{U} & = \vgrads{} \vdivs{} \vect{U}
      \end{align*}
      On utilise les expression en coordonnées cylindrique des opérateurs différentiels ( voir annexe \ref{sec:annexe:div_grad_rot}).
      \begin{align*}
        \vdivs{\vect{U}}(r,\theta,z) = \frac{1}{r}\ddr{\theta}{U_\theta}(r,\theta,z) + \ddr{z}{U_z}(r,\theta,z)
      \end{align*}
      \begin{align*}
        \vgrads{f}(r,\theta,z) = \frac{1}{r}\ddr{\theta}{f}(r,\theta,z)\vect{e_\theta} + \ddr{z}{f}(r,\theta,z)\vect{e_z}
      \end{align*}
      Or d’après la définition de la transformée de Fourier
      \begin{align*}
        \vect{U}(r,\theta,z) & = \frac{1}{2\pi}\sum_{n=-\infty}^\infty \int_\RR \hat{\vect{U}}(r,n,k_z)e^{in\theta + ik_zz}\dd{k_z}
      \end{align*}
      les opérateurs en Fourier sont
      \begin{align*}
        \widehat{\vdivs{\vect{U}}}(r,n,k_z) = \frac{in}{r}{\hat{U}_\theta}(r,n,k_z) + ik_z{\hat{U}_z}(r,n,k_z)
      \end{align*}
      \begin{align*}
        \widehat{\vgrads{f}}(r,n,k_z) = \frac{in}{r}\hat{f}(r,n,k_z)\vect{e_\theta} + ik_z\hat{f}(r,n,k_z)\vect{e_z}
      \end{align*}
      donc
      \begin{align*}
        \widehat{\vgrads \vdivs{\vect{U}}}(r,n,k_z) =  \left(-\frac{n^2}{r^2}\vect{e_\theta} - \frac{nk_z}{r}\vect{e_z}\right){\hat{U}_\theta}(r,n,k_z) + \left(-\frac{nk_z}{r}\vect{e_\theta} - {k_z^2}\vect{e_z}\right){\hat{U}_z}(r,n,k_z)
      \end{align*}

    \end{proof}


    \begin{defn}
      \label{eq:cylindre:fourier:LR}

      On définit \(\LR\) l'endomorphisme de \(V\) tel que
      \begin{align*}
        \LR \vect{U}(r,\theta,z) & = \vrots{} (\rots{} \vect{U})(r,\theta,z)
      \end{align*}

      On définit \(\hat{\mLR}\) la fonction de \(\NN\times\RR \rightarrow \mathcal{M}_2(\RR)\) telle que
      \begin{equation*}
        \hat{\mLR}(n,k_z) = 
        \begin{bmatrix}
          -k_z^2 & k_z{n}\slash{r_C}
          \\
          k_z{n}\slash{r_C} & -\left({n}\slash{r_C}\right)^2
        \end{bmatrix}
      \end{equation*}
    \end{defn}

    \begin{prop}
      Soit \(\vect{U} \in V\)
      Alors
      \begin{equation*}
        \widehat{\LR \vect{U}} (r_C,n,k_z) = \hat{\mLR}(n,k_z) \hat{\vect{U}}(r_C,n,k_z)
      \end{equation*}
    \end{prop}

    \begin{proof}
      Par définition de \(\LR\), on a
      \begin{align*}
        \LR \vect{U} & = \vrots{} (\rots{} \vect{U})
      \end{align*}
      On utilise les expression en coordonnées cylindrique des opérateurs différentiels ( voir annexe \ref{sec:annexe:div_grad_rot}).
      \begin{align*}
        \rots{\vect{U}}(r,\theta,z) = \frac{1}{r}\ddr{\theta}{U_z}(r,\theta,z) - \ddr{z}{U_\theta}(r,\theta,z)
      \end{align*}
      \begin{align*}
        \vrots{f}(r,\theta,z) = \ddr{z}{f}(r,\theta,z)\vect{e_\theta} - \frac{1}{r}\ddr{\theta}{f}(r,\theta,z)\vect{e_z}
      \end{align*}
      donc comme pour l'opérateur \(\LD\)
      \begin{align*}
        \widehat{\rots{\vect{U}}}(r,n,k_z) = \frac{in}{r}{\hat{U}_z}(r,n,k_z) - ik_z{\hat{U}_\theta}(r,n,k_z)
      \end{align*}
      \begin{align*}
        \widehat{\vrots{f}}(r,n,k_z) =  ik_z\hat{f}(r,n,k_z)\vect{e_\theta} - \frac{in}{r}\hat{f}(r,n,k_z)\vect{e_z}
      \end{align*}
      donc
      \begin{align*}
        \widehat{\vrots (\rots{\vect{U}})}(r,n,k_z) =  \left({k_z^2}\vect{e_\theta} - \frac{nk_z}{r}\vect{e_z}\right){\hat{U}_\theta}(r,n,k_z) + \left(-\frac{nk_z}{r}\vect{e_\theta} + \frac{n^2}{r^2}\vect{e_z}\right){\hat{U}_z}(r,n,k_z)
      \end{align*}

    \end{proof}

  \subsection{Expression de la matrice d'impédance approchée par la CI3}

    Tout comme dans le cas du plan infini, on peut donc définir \(\hat{\mZ}_{IBC}\) l’opérateur matriciel associé à la condition d'impédance d'ordre élevée. Par exemple,

    \begin{multline}
        \hat{\mZ}_{CI3}(n,k_z) = \left(\mI + b_1 \frac{\hat{\mLD}(n,k_z)}{k_0^2} - b_2 \frac{\hat{\mLR}(n,k_z)}{k_0^2} \right)^{-1}\\
        \left(a_0 \mI + a_1 \frac{\hat{\mLD}(n,k_z)}{k_0^2} - a_2 \frac{\hat{\mLR}(n,k_z)}{k_0^2}\right)
    \end{multline}
