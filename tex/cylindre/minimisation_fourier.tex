\section[Choix 2 du calcul des coefficients de la CI3]{Choix des coefficients de la CI3 par moindres carrés sur les coefficients de la série de Fourier}

  Soient \(\mM_\mJ\) et \(\mM_\mH\) les fonctions introduites à la définition \ref{def:cylindre:matrices_MJ-MH} et \(\hat\mR\) la fonction définie à la définition \ref{def:cylindre:reflexion:impedance}.

  \begin{defn}%[]
    \label{def:cylindre:minimisation:matrices_MR}
    On définit les fonctions \(\hat\mR_{ex}, \hat\mR_{CI3}\) de \(\NN\times\RR  \rightarrow \mathcal{M}_2(\CC)\) telles que
    \begin{align*}
      \hat\mR_{ex}(n,k_z) &= \hat\mR(n,k_z, \hat\mZ_{ex}(n,k_z)),
      \\
      \hat\mR_{CI3}(n,k_z) &= \hat\mR(n,k_z, \hat\mZ_{CI3}(n,k_z)),
    \end{align*}
    où \(\hat\mZ_{ex},\hat\mZ_{CI3}\) sont des fonctions définies à la proposition \ref{prop:cylindre:synthese:impedance} et à l'équation \eqref{eq:cylindre:hoibc:ci3}.
  \end{defn}

  \subsection[Choix de la fonction JR]{Choix de la fonction \(J_R\)}

    On utilise les fonctions \(\mN_E, \mN_H\) introduites à la définition \ref{def:cylindre:matrices_NE-NH}.

    \begin{defn}
      On définit \(\mA_0,\mA_1,\mA_2,\mA_2,\mB_1,\mB_2\) les fonctions de \(\NN \times \RR \times \mathcal{M}_2(\CC) \rightarrow \ \mathcal{M}_2(\CC)\) telles que        
      \begin{align*}
        \mA_0(n,k_z,\mR) &= \mN_E(r_c^+,n,k_z,\mR),
        \\
        \mA_1(n,k_z,\mR) &= \hat{\mLD}(n,k_z)\mN_E(r_c^+,n,k_z,\mR),
        \\
        \mA_2(n,k_z,\mR) &= -\hat{\mLR}(n,k_z)\mN_E(r_c^+,n,k_z,\mR),
        \\
        \mB_1(n,k_z,\mR) &= \hat{\mLD}(n,k_z)\mN_H(r_c^+,n,k_z,\mR),
        \\
        \mB_2(n,k_z,\mR) &= -\hat{\mLR}(n,k_z)\mN_H(r_c^+,n,k_z,\mR)  .          
      \end{align*}

      On définit \(\tilde{\mH}_{CI3}\) la fonction de \(\NN \times \RR \times \mathcal{M}_2(\CC) \rightarrow \mathcal{M}_{4\times5}(\CC)\) telle que
      \begin{align*}
        & \tilde\mH_{CI3}(n,k_z,\mR) =  \\ &
        \begin{bmatrix}
          \mA_0(n,k_z,\mR)_{11} & \mA_1(n,k_z,\mR)_{11} & \mA_2(n,k_z,\mR)_{11} & \mB_1(n,k_z,\mR)_{11} & \mB_2(n,k_z,\mR)_{11}
          \\
          \mA_0(n,k_z,\mR)_{12} & \mA_1(n,k_z,\mR)_{12} & \mA_2(n,k_z,\mR)_{12} & \mB_1(n,k_z,\mR)_{12} & \mB_2(n,k_z,\mR)_{12}
          \\
          \mA_0(n,k_z,\mR)_{21} & \mA_1(n,k_z,\mR)_{21} & \mA_2(n,k_z,\mR)_{21} & \mB_1(n,k_z,\mR)_{21} & \mB_2(n,k_z,\mR)_{21}
          \\
          \mA_0(n,k_z,\mR)_{22} & \mA_1(n,k_z,\mR)_{22} & \mA_2(n,k_z,\mR)_{22} & \mB_1(n,k_z,\mR)_{22} & \mB_2(n,k_z,\mR)_{22}
        \end{bmatrix}.
      \end{align*}

      On définit \(\tilde{b}\) la fonction de \(\NN \times \RR \times \mathcal{M}_2(\CC) \rightarrow \CC^4\) telle que
      \begin{equation*}
        \tilde{b}(n,k_z,\mR) = -
        \begin{bmatrix}
          \mN_H(r_c^+,n,k_z,\mR)_{11}
          \\
          \mN_H(r_c^+,n,k_z,\mR)_{12}
          \\
          \mN_H(r_c^+,n,k_z,\mR)_{21}
          \\
          \mN_H(r_c^+,n,k_z,\mR)_{22}
        \end{bmatrix}.
      \end{equation*}
    \end{defn}

    \begin{prop}
      Soit \(X = (a_0,a_1,a_2,b_1,b_2)^t \in \CC^5\), \((n,k_z)\) fixé et \(\hat\mR_{ex}\) la matrice définie en \ref{def:cylindre:minimisation:matrices_MR}, alors
      \begin{multline*}
        \argmin{X\in\CC^5} \norm{\hat\mR_{CI3}(n,k_z,X) - \hat\mR_{ex}(n,k_z)} =
        \\
        \argmin{X\in\CC^5} \norm{\tilde{\mH}_{CI3}(n,k_z,\hat\mR_{ex}(n,k_z))X - \tilde{b}(n,k_z,\hat\mR_{ex}(n,k_z))}.
      \end{multline*}
    \end{prop}

    \begin{proof}
      C'est la même méthodologie que pour l'impédance.
      \begin{align*}
        &{\hspace{1em}}~ \argmin{X\in\CC^5} \norm{\hat\mR_{CI3}(n,k_z,X) - \hat\mR_{ex}(n,k_z)}
        \\
        & = \argmin{X\in\CC^5} \norm{ - \mM_\mH(r_c^+,n,k_z,\hat\mZ_{CI3})^{-1}\mM_\mJ(r_c^+,n,k_z,\hat\mZ_{CI3})- \hat\mR_{ex}(n,k_z) },
        \\
        & = \argmin{X\in\CC^5} \norm{ - \mM_\mH(r_c^+,n,k_z,\hat\mZ_{CI3})^{-1}\left(\mM_\mJ(r_c^+,n,k_z,\hat\mZ_{CI3}) +  \mM_\mH(r_c^+,n,k_z,\hat\mZ_{CI3})\hat\mR_{ex}(n,k_z)\right) },
        \\ 
        & = \argmin{X\in\CC^5} \norm{\mM_\mJ(r_c^+,n,k_z,\hat\mZ_{CI3}) +\mM_\mH(r_c^+,n,k_z,\hat\mZ_{CI3})\hat\mR_{ex}(n,k_z)}.
        \intertext{D'après la définition \ref{def:cylindre:matrices_MJ-MH} des fonctions \(\mM_\mJ, \mM_\mH\),}
        & = \argmin{X\in\CC^5} \left\lVert \left(\mJ_E(r_c^+,n,k_z)-\hat\mZ_{CI3}(n,k_z)\mJ_H(r_c^+,n,k_z)\right) \right.
        \\
        & \qquad \qquad \quad + \left.\left(\mH_E(r_c^+,n,k_z)-\hat\mZ_{CI3}(n,k_z)\mH_H(r_c^+,n,k_z)\right)\hat\mR_{ex}(n,k_z) \right\lVert.
        \intertext{D'après la définition de \(\hat\mZ_{CI3}\),}        
        & = \argmin{X\in\CC^5} \left\lVert \hat\mZ_D(n,k_z,X)^{-1}\left(\hat\mZ_D(n,k_z,X)\mJ_E(r_c^+,n,k_z)-\hat\mZ_N(n,k_z,X)\mJ_H(r_c^+,n,k_z)\right) \right.
        \\
        & \qquad \qquad \quad + \left.\hat\mZ_D(n,k_z,X)^{-1}\left(\hat\mZ_D(n,k_z,X)\mH_E(r_c^+,n,k_z)-\hat\mZ_N(n,k_z,X)\mH_H(r_c^+,n,k_z)\right)\hat\mR_{ex}(n,k_z) \right\lVert,
        \\
        & = \argmin{X\in\CC^5} \left\lVert \left(\hat\mZ_D(n,k_z,X)\mJ_E(r_c^+,n,k_z)-\hat\mZ_N(n,k_z,X)\mJ_H(r_c^+,n,k_z)\right) \right.
        \\
        & \qquad \qquad \quad + \left.\left(\hat\mZ_D(n,k_z,X)\mH_E(r_c^+,n,k_z)-\hat\mZ_N(n,k_z,X)\mH_H(r_c^+,n,k_z)\right)\hat\mR_{ex}(n,k_z) \right\lVert.
        \intertext{D'après la définition \ref{def:cylindre:matrices_NE-NH} des fonctions \(\mN_E, \mN_H\),}        
        & = \argmin{X\in\CC^5} \norm{\hat\mZ_N(n,k_z,X)\mN_E(r_c^+,n,k_z,\hat\mR_{ex}(n,k_z)) + \hat\mZ_D(n,k_z,X)\mN_H(r_c^+,n,k_z,\hat\mR_{ex}(n,k_z))}
      \end{align*}
      et l’on conclut d'après la définition des fonctions \(\hat\mZ_D, \hat\mZ_N\).
    \end{proof}

    On tronque la série de Fourier à \(N_{n}\) termes et on se dote de \(N_{k_z}\) \(k_z\). Il existe donc \(N_{n}N_{k_z}\) couples tels que \((n_i,k_{zj}) = (n,k_z)_{(j-1)N_{n}+i}\).
    \begin{defn}
      On définit \(\tilde{\mA}_{CI3}\) la matrice de \(\mathcal{M}_{4N_{n}N_{k_z}\times5}(\CC)\) telle que
      \begin{equation*}
        \tilde{\mA}_{CI3} = 
        \begin{bmatrix}
          \tilde\mH_{CI3}(n_1,k_{z1},\hat\mR_{ex}(n_1,k_{z1}))
          \\
          \vdots
          \\
          \tilde\mH_{CI3}(n_i,k_{zj},\hat\mR_{ex}(n_i,k_{zj}))
          \\
          \vdots
          \\
          \tilde\mH_{CI3}(n_{N_n},k_{zN_{k_z}},\hat\mR_{ex}(n_{N_n},k_{zN_{k_z}}))
        \end{bmatrix}.
      \end{equation*}
      On définit \(\tilde{g}\) le vecteur colonne \(\CC^{4N_{n}N_{k_z}}\) telle que
      \begin{equation*}
        \tilde{g} = 
        \begin{bmatrix}
          \tilde{b}(n_1,k_{z1},\hat\mR_{ex}(n_1,k_{z1}))
          \\
          \vdots
          \\
          \tilde{b}(n_i,k_{zj},\hat\mR_{ex}(n_i,k_{zj}))
          \\
          \vdots
          \\
          \tilde{b}(n_{N_n},k_{zN_{k_z}},\hat\mR_{ex}(n_{N_n},k_{zN_{k_z}}))
        \end{bmatrix}.
      \end{equation*}
    \end{defn}

    On peut alors calculer les coefficients de la CI3:

    \begin{defn}
      On définit la fonctionnelle \(J_R\)
      \begin{equation*}
        J_R(X) = \norm{\tilde{\mA}_{CI3}X - \tilde{g}}.
      \end{equation*}
    \end{defn}

    \begin{thm}[Minimisation sans contraintes pour la CI3]

      Les coefficients de la CIOE sont solutions du problème

      Trouver \(X^* \in \CC^5\) tel que
      \begin{equation*}
        X^* = \argmin{X\in \CC^5} J_R(X).
      \end{equation*}
    \end{thm}

    \begin{prop}
      Si \(\conj{\tilde{\mA}_{CI3}^t}\tilde{\mA}_{CI3}\) est inversible alors
      \begin{equation*}
        X^* = (\conj{\tilde{\mA}_{CI3}^t}\tilde{\mA}_{CI3})^{-1}\conj{\tilde{\mA}_{CI3}^t}\tilde{g}.
      \end{equation*}
    \end{prop}
    \begin{proof}
      Même méthode que pour la proposition \ref{prop:cylindre:minimisation:minimum_sans_contraintes} sur l'impédance.
    \end{proof}

    Nous n'avons pas réussi à démontrer que cette matrice était définie pour tout empilement et toute incidence.

    \begin{thm}[Minimisation avec contraintes pour la CI3]

      Soit \(\CSU[3]{CI3}\) le sous-espace de \(\CC^5\) issu de la définition \ref{def:csu:ci3-3}.
      Alors les coefficients de la CIOE respectant les CSU sont solutions du problème

      Trouver \(X^* \in \CC^5\) tel que
      \begin{equation*}
        X^* = \argmin{X\in \CSU[3]{CI3}} J_R(X).
      \end{equation*}
    \end{thm}
