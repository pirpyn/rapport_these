\section{Calcul des coefficients des CIOE par moindres carrés sur les coefficients de la série de Fourier}

  On remarque alors que sans contraintes, la CI3 n'est pas adaptée au cylindre, même en incidence radiale. Cependant, on peut aussi résoudre l'erreur entre les coefficients de Fourier. Pour cela, on adapte au cylindre le lemme permettant de déduire d'une impédance la matrice de réflexion qui contient les coefficients de la série de Fourier.

  \begin{defn}[Matrice de réflexion associée à une impédance]\label{def:cylindre:reflexion_from_impedance}{}~
    Soit \(\mM_\mJ\) et \(\mM_\mH\) les fonctions introduite à la définition \ref{def:cylindre:matrices_MJ-MH}.

    On définit la fonction \(\hat\mR\) de \(\NN\times\RR\times\mathcal{M}_2(\CC) \rightarrow \mathcal{M}_2(\CC)\) telle que
    \begin{multline*}
      \hat\mR(n,k_z,\mA) = - \mM_\mH(r_c^+,n,k_z,\mA)^{-1}\mM_\mJ(r_c^+,n,k_z,\mA)
    \end{multline*}
  \end{defn}
    \begin{prop}
      Soit \(r_c^+\) l'interface cylindre-vide.

      Soit \(\hat{\mZ}(n,k_z)\) la matrice d'impédance à cette interface.
      \begin{align*}
        \hat{\vE}_t(r_c^+,n,k_z) &= \hat{\mZ}(n,k_z) \left(\vect{e_r} \pvect \hat{\vH}_t(r_c^+,n,k_z)\right)
      \end{align*}
      alors il existe \(\vect{C}(n,k_z) \in \CC^2\) tel que 
      \begin{align*}
        \hat{\vE}_t(r_c^+,n,k_z) &= \left(\mJ_E(r_c^+,n,k_z) + \mH_E(r_c^+,n,k_z)\hat\mR(n,k_z,\hat\mZ(n,k_z)\right)\vect{C}(n,k_z)
        \\
        \vect{e_r} \pvect \hat{\vH}_t(r_c^+,n,k_z) &= \left(\mJ_H(r_c^+,n,k_z) + \mH_H(r_c^+,n,k_z)\hat\mR(n,k_z,\hat\mZ(n,k_z)\right)\vect{C}(n,k_z)
      \end{align*}
    \end{prop}
    \begin{proof}
      Immédiat depuis la définition des champs \eqref{eq:imp_fourier:cylindre:Et},\eqref{eq:imp_fourier:cylindre:Ht}.
    \end{proof}

    \begin{defn}%[]
      \label{def:cylindre:minimisation:matrices_MR}
      On définit les fonctions \(\hat\mR_{ex}, \hat\mR_{CI3}\) de \(\NN\times\RR\times \rightarrow \mathcal{M}_2(\CC)\) telles que
      \begin{align*}
        \hat\mR_{ex}(n,k_z) &= \hat\mR(n,k_z, \hat\mZ_{ex}(n,k_z))
        \\
        \hat\mR_{CI3}(n,k_z) &= \hat\mR(n,k_z, \hat\mZ_{CI3}(n,k_z))
      \end{align*}
      où \(\hat\mZ_{ex},\hat\mZ_{CI3}\) sont des fonctions définies à la proposition \ref{prop:cylindre:synthese:impedance} et à l'équation \eqref{eq:cylindre:hoibc:ci3}.
    \end{defn}

  \subsection{Expression de la fonctionnelle}

    On utilise les fonctions \(\mN_E, \mN_H\) introduite à la définition \ref{def:cylindre:matrices_NE-NH}.

    \begin{defn}
      On définit \(\tilde{\mH}_{CI3}\) la fonction de \(\NN \times \RR \times \mathcal{M}_2(\CC) \rightarrow \mathcal{M}_{4\times5}(\CC)\) telle que
      \begin{equation*}
        \mH_{CI3}(n,k_z,\mA) = 
        \begin{bmatrix}
          A_0(n,k_z,\mA)_{11} & A_1(n,k_z,\mA)_{11} & A_2(n,k_z,\mA)_{11} & B_1(n,k_z,\mA)_{11} & B_2(n,k_z,\mA)_{11}
          \\
          A_0(n,k_z,\mA)_{12} & A_1(n,k_z,\mA)_{12} & A_2(n,k_z,\mA)_{12} & B_1(n,k_z,\mA)_{12} & B_2(n,k_z,\mA)_{12}
          \\
          A_0(n,k_z,\mA)_{21} & A_1(n,k_z,\mA)_{21} & A_2(n,k_z,\mA)_{21} & B_1(n,k_z,\mA)_{21} & B_2(n,k_z,\mA)_{21}
          \\
          A_0(n,k_z,\mA)_{22} & A_1(n,k_z,\mA)_{22} & A_2(n,k_z,\mA)_{22} & B_1(n,k_z,\mA)_{22} & B_2(n,k_z,\mA)_{22}
        \end{bmatrix}
        \end{equation*}
        où
        \begin{align*}
          A_0(n,k_z,\mA) &= \mN_E(r_c^+,n,k_z,\mA)
          \\
          A_1(n,k_z,\mA) &= \hat{\mLD}(n,k_z)\mN_E(r_c^+,n,k_z,\mA)
          \\
          A_2(n,k_z,\mA) &= -\hat{\mLR}(n,k_z)\mN_E(r_c^+,n,k_z,\mA)
          \\
          B_1(n,k_z,\mA) &= \hat{\mLD}(n,k_z)\mN_H(r_c^+,n,k_z,\mA)
          \\
          B_2(n,k_z,\mA) &= -\hat{\mLR}(n,k_z)\mN_H(r_c^+,n,k_z,\mA)            
        \end{align*}

        On définit \(\tilde{b}\) la fonction de \(\NN \times \RR \times \mathcal{M}_2(\CC) \rightarrow \mathcal{M}_{4\times1}(\CC)\) telle que
        \begin{equation*}
          \tilde{b}(n,k_z,\mA) = 
          \begin{bmatrix}
            \mN_H(r_c^+,n,k_z,\mA)_{11}
            \\
            \mN_H(r_c^+,n,k_z,\mA)_{12}
            \\
            \mN_H(r_c^+,n,k_z,\mA)_{21}
            \\
            \mN_H(r_c^+,n,k_z,\mA)_{22}
          \end{bmatrix}
        \end{equation*}
      \end{defn}

    \begin{prop}
      Soit \(X = (a_0,a_1,a_2,b_1,b_2)\), \((n,k_z)\) fixé et \(\hat\mR_{ex}\) la matrice définie en \ref{def:cylindre:minimisation:matrices_MR}, alors
      \begin{multline*}
        \argmin{X\in\CC^5} \norm{\hat\mR_{CI3}(n,k_z,X) - \hat\mR_{ex}(n,k_z)} = 
        \\
        \argmin{X\in\CC^5} \norm{\tilde{\mH}_{CI3}(n,k_z,\hat\mR_{ex}(n,k_z))X - \tilde{b}(n,k_z,\hat\mR_{ex}(n,k_z))}
      \end{multline*}
    \end{prop}

\begin{proof}
      C'est la même méthodologie que pour l`impédance.
      On rappelle de la section précédente
      \begin{multline*}
        \hat{\mZ}_{CI3}(n,k_z) = \left(\mI + b_1 \hat{\mLD}(n,k_z) - b_2 \hat{\mLR}(n,k_z) \right)^{-1}\\\left(a_0 \mI + a_1 {\hat{\mLD}(n,k_z)} - a_2 {\hat{\mLR}(n,k_z)}\right)
      \end{multline*}
      On pose \(\hat\mZ_D(n,k_z) = \mI + b_1 \hat{\mLD}(n,k_z) - b_2 \hat{\mLR}(n,k_z)\) et \(\hat\mZ_N(n,k_z) = a_0 \mI + a_1 {\hat{\mLD}(n,k_z)} - a_2 {\hat{\mLR}(n,k_z)}\) donc

      \begin{align*}
      &{}~ \argmin{X\in\CC^5} \norm{\hat\mR_{CI3}(n,k_z,X) - \hat\mR_{ex}(n,k_z)}
      \intertext{D'après la définition \ref{def:cylindre:matrices_MJ-MH} des fonctions \(\mM_\mJ, \mM_\mH\),}
      & = \argmin{X\in\CC^5} \norm{ - \left(\mM_\mH(n,k_z,\hat\mZ_{CI3}\right)^{-1}\left(\mM_\mJ(n,k_z,\hat\mZ_{CI3}\right) - \hat\mR_{ex}(n,k_z) }
      \\
      &= \argmin{X\in\CC^5} \norm{\hat\mZ_D(n,k_z)^{-1}\left(\hat\mZ_N(n,k_z) - \hat\mZ_D(n,k_z)\hat\mZ_{ex}(n,k_z)\right) }
      \\
      &= \argmin{X\in\CC^5} \norm{\hat\mZ_N(n,k_z) - \hat\mZ_D(n,k_z)\hat\mZ_{ex}(n,k_z)}
      \\
      &= \argmin{X\in\CC^5} \norm{\hat\mZ_N(n,k_z) - \left(b_1 \hat{\mLD}(n,k_z) - b_2 \hat{\mLR}(n,k_z)\right)\hat\mZ_{ex}(n,k_z) - \hat\mZ_{ex}(n,k_z) }
      \\
      &= \argmin{X\in\CC^5} \norm{\mH_{CI3}(n,k_z,\hat\mZ_{ex}(n,k_z))X - b(\hat\mZ_{ex}(n,k_z))}
      \end{align*}
    \end{proof}

    \subsection{Résultats numériques sur l'approximation des coefficients de la série de Fourier}
