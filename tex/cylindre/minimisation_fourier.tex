\section{Calcul des coefficients des CIOE par moindres carrés sur les coefficients de la série de Fourier}

    On remarque alors que sans contraintes, la CI3 n'est pas adaptée au cylindre, même en incidence radiale. Cependant, on peut aussi résoudre l'erreur entre les coefficients de Fourier. Pour cela, on adapte au cylindre le lemme permettant de déduire d'une impédance la matrice de réflexion qui contient les coefficients de la série de Fourier.

    \begin{lemme}[Matrice de réflexion associée à une impédance]
        \label{lem:cylindre:reflexion_from_impedance}
        Soit \(r_m\) une interface entre deux matériaux, et on se place juste au dessus (\(r>r_m\)). 

        Soit \(\hat{\mZ}(n,k_z)\) la matrice d'impédance à cette interface.
        \begin{align*}
            \hat{\vE}_t(r_m^+,n,k_z) &= \hat{\mZ}(n,k_z) \left(\vect{e_r} \pvect \hat{\vH}_t(r_m^+,n,k_z)\right)
        \end{align*}
        Alors la matrice de réflexion est fonction de la matrice d'impédance et
        \begin{multline*}
                \hat\mR(n,k_z) = - \left(\mH_E(r_m^+,n,k_z) - \hat{\mZ}(n,k_z) \mH_H(r_m^+,n,k_z) \right)^{-1}\\
                \left(\mJ_E(r_m^+,n,k_z) - \hat{\mZ}(n,k_z) \mJ_H(r_m^+,n,k_z) \right)
        \end{multline*}
    \end{lemme}

    \subsection{Expression des moindre carrés dans le cadre de l'approximation cylindre infini pour une incidence}

    \subsection{Expression des moindre carrés dans le cadre de l'approximation cylindre infini avec un balayage en incidence}

    \subsection{Résultats numériques sur l'approximation des coefficients de la série de Fourier}
