\section{Calcul des coefficients de la CI3 par moindres carrés sur l'impédance}

  \subsection{Expression de la fonctionnelle}

    \begin{defn}
      On définit \(\mH_{CI3}\) la fonction de \(\NN \times \RR \times \mathcal{M}_2(\CC) \rightarrow \mathcal{M}_{4\times5}(\CC)\) telle que
      \begin{equation*}
        \mH_{CI3}(n,k_z,\mA) = \begin{bmatrix}
        1 & \hat{\mLD}(n,k_z)_{11} & -\hat{\mLR}(n,k_z)_{11} & -\left(\hat{\mLD}(n,k_z){\mA}\right)_{11} & \left(\hat{\mLR}(n,k_z){\mA}\right)_{11}
        \\
        0 & \hat{\mLD}(n,k_z)_{12} & -\hat{\mLR}(n,k_z)_{12} & -\left(\hat{\mLD}(n,k_z){\mA}\right)_{12} & \left(\hat{\mLR}(n,k_z){\mA}\right)_{12}
        \\
        0 & \hat{\mLD}(n,k_z)_{21} & -\hat{\mLR}(n,k_z)_{21} & -\left(\hat{\mLD}(n,k_z){\mA}\right)_{21} & \left(\hat{\mLR}(n,k_z){\mA}\right)_{21}
        \\
        1 & \hat{\mLD}(n,k_z)_{22} & -\hat{\mLR}(n,k_z)_{22} & -\left(\hat{\mLD}(n,k_z){\mA}\right)_{22} & \left(\hat{\mLR}(n,k_z){\mA}\right)_{22}
        \end{bmatrix}
      \end{equation*}
      On définit \(b\) la fonction de \(\mathcal{M}_2(\CC) \rightarrow \mathcal{M}_{4\times1}(\CC)\) telle que
      \begin{equation*}
        b(\mA) = \begin{bmatrix}
        {\mA}_{11}
        \\
        {\mA}_{12}
        \\
        {\mA}_{21}
        \\
        {\mA}_{22}
        \end{bmatrix}
      \end{equation*}
    \end{defn}

    \begin{prop}
      Soit \(X = (a_0,a_1,a_2,b_1,b_2)\), \((n,k_z)\) fixé et \(\hat\mZ_{ex}\) l'opérateur d'impédance exact du cylindre, alors
      \begin{equation*}
        \argmin{X\in\CC^5} \norm{\hat\mZ_{CI3}(n,k_z,X) - \hat\mZ_{ex}(n,k_z)} = \argmin{X\in\CC^5} \norm{\mH_{CI3}(n,k_z,\hat\mZ_{ex}(n,k_z))X - b(\hat\mZ_{ex}(n,k_z))}
      \end{equation*}
    \end{prop}

    \begin{proof}
      C'est la même méthodologie que pour le plan.
      On rappelle que dans la section précédente, on a introduit
      \begin{multline*}
        \hat{\mZ}_{CI3}(n,k_z) = \left(\mI + b_1 \hat{\mLD}(n,k_z) - b_2 \hat{\mLR}(n,k_z) \right)^{-1}\\\left(a_0 \mI + a_1 {\hat{\mLD}(n,k_z)} - a_2 {\hat{\mLR}(n,k_z)}\right)
      \end{multline*}
      On pose \(\hat\mZ_D(n,k_z) = \mI + b_1 \hat{\mLD}(n,k_z) - b_2 \hat{\mLR}(n,k_z)\) et \(\hat\mZ_N(n,k_z) = a_0 \mI + a_1 {\hat{\mLD}(n,k_z)} - a_2 {\hat{\mLR}(n,k_z)}\) donc

      \begin{align*}
      &{}~ \argmin{X\in\CC^5} \norm{\hat\mZ_{CI3}(n,k_z,X) - \hat\mZ_{ex}(n,k_z)}
      \\
      & = \argmin{X\in\CC^5} \norm{\hat\mZ_D(n,k_z)^{-1}\hat\mZ_N(n,k_z) - \hat\mZ_{ex}(n,k_z) }
      \\
      &= \argmin{X\in\CC^5} \norm{\hat\mZ_D(n,k_z)^{-1}\left(\hat\mZ_N(n,k_z) - \hat\mZ_D(n,k_z)\hat\mZ_{ex}(n,k_z)\right) }
      \\
      &= \argmin{X\in\CC^5} \norm{\hat\mZ_N(n,k_z) - \hat\mZ_D(n,k_z)\hat\mZ_{ex}(n,k_z)}
      \\
      &= \argmin{X\in\CC^5} \norm{\hat\mZ_N(n,k_z) - \left(b_1 \hat{\mLD}(n,k_z) - b_2 \hat{\mLR}(n,k_z)\right)\hat\mZ_{ex}(n,k_z) - \hat\mZ_{ex}(n,k_z) }
      \\
      &= \argmin{X\in\CC^5} \norm{\mH_{CI3}(n,k_z,\hat\mZ_{ex}(n,k_z))X - b(\hat\mZ_{ex}(n,k_z))}
      \end{align*}
    \end{proof}

    On tronque la série de Fourier à \(N_{n}\) termes et on se dote de \(N_{k_z}\) \(k_z\). Il existe donc \(N_{n}N_{k_z}\) couples tels que \((n_i,k_{zj}) = (n,k_z)_{(j-1)N_{n}+i}\).
    \begin{defn}
      On définit \(\mA_{CI3}\) la matrice de \(\mathcal{M}_{4N_{n}N_{k_z}\times5}(\CC)\) telle que
      \begin{equation*}
        \mA_{CI3} = 
        \begin{bmatrix}
          \mH_{CI3}(n_1,k_{z1},\hat\mZ_{ex}(n_1,k_{z1}))
          \\
          \vdots
          \\
          \mH_{CI3}(n_i,k_{zj},\hat\mZ_{ex}(n_i,k_{zj}))
          \\
          \vdots
          \\
          \mH_{CI3}(n_{N_n},k_{zN_{k_z}},\hat\mZ_{ex}(n_{N_n},k_{zN_{k_z}}))
        \end{bmatrix}
      \end{equation*}
      On définit \(g\) la matrice de \(\mathcal{M}_{4N_{n}N_{k_z}\times1}(\CC)\) telle que
      \begin{equation*}
        g = 
        \begin{bmatrix}
          b(\hat\mZ_{ex}(n_1,k_{z1}))
          \\
          \vdots
          \\
          b(\hat\mZ_{ex}(n_i,k_{zj}))
          \\
          \vdots
          \\
          b(\hat\mZ_{ex}(n_{N_n},k_{zN_{k_z}}))
        \end{bmatrix}
      \end{equation*}
    \end{defn}

    On peut alors calculer les coefficients de la CI3
    \begin{defn}
      On définit la fonctionnelle \(J_Z\)
      \begin{equation*}
        J_Z(X) = \norm{{\mA}_{CI3}X - {g}}
      \end{equation*}
    \end{defn}
    \begin{thm}[Minimisation sans contraintes pour la CI3]

      Les coefficients de la CIOE sont solutions du problème

      Trouver \(X^* \in \CC^5\) tel que
      \begin{equation*}
        X^* = \argmin{X\in \CC^5}  J_Z(X)
      \end{equation*}
    \end{thm}

    \begin{prop}
      \label{prop:cylindre:minimisation:minimum_sans_contraintes}
      Si \(\conj{\mA_{CI3}^t}\mA_{CI3}\) est inversible alors
      \begin{equation*}
        X^* = (\conj{\mA_{CI3}^t}\mA_{CI3})^{-1}\conj{\mA_{CI3}^t}g
      \end{equation*}
    \end{prop}

    \begin{proof}
      Soit \(\left<\cdot,\cdot\right>\) le produit scalaire associé à la norme \(\norm{\cdot}\). Soit \(J\) la fonctionnelle telle que
      \begin{align*}
      J(X) &= \norm{\mA_{CI3}X - g} 
      \\
      &= \left<\mA_{CI3}X - g,\mA_{CI3}X - g\right>
      \intertext{le produit scalaire est symétrique donc \(\left<a,b\right> = \left<b,a\right>\)}
      &=\left<\mA_{CI3}X,\mA_{CI3}X\right> - 2\left<\mA_{CI3}X,g\right> + \left<g,g\right>
      \\
      &=\left<X,\conj{\mA_{CI3}^t}\mA_{CI3}X\right> - 2\left<X,\conj{\mA_{CI3}^t}g\right> + \left<g,g\right>
      \end{align*}
      Si la matrice \(\conj{\mA_{CI3}^t}\mA_{CI3}\) est inversible, comme elle est par définition hermitienne et positive alors la fonctionnelle est quadratique. Donc il existe une unique minimum où le gradient de cette fonctionnelle s'annule. Or
      \begin{align*}
        \nabla J(X^*) &= 2\conj{\mA_{CI3}^t}\mA_{CI3}X^* - 2\conj{\mA_{CI3}^t}g
        \\ 
        &= 0
      \end{align*}
    \end{proof}

    Nous n'avons pas réussi à démontrer que cette matrice était définie pour tout empilement et tout incidence.

    \begin{thm}[Minimisation avec contraintes pour la CI3]

      Soit \(\CSU[3]{CI3}\) le sous-espace de \(\CC^5\) issu de la définition \ref{def:csu:ci3-3}.
      Alors les coefficients de la CIOE respectant les CSU sont solutions du problème

      Trouver \(X^* \in \CC^5\) tel que
      \begin{equation*}
        X^* = \argmin{X\in \CSU[3]{CI3}}  J_Z(X)
      \end{equation*}
    \end{thm}


  \subsection{Résultats numériques sur l'approximation de la matrice d'impédance}

    Dans cette partie, toutes les CIOE faisant intervenir des opérateurs différentielles normalisent ces derniers par \(k_0^2\). Ainsi la CI3 s'exprime
    \begin{equation*}
      \left(\oI + c_1\frac{\LD}{k_0^2} -c_2\frac{\LR}{k_0^2}\right)\vE_t = \left(a_0\oI + b_1\frac{\LD}{k_0^2} - b_2 \frac{\LR}{k_0^2} \right)\vJ
    \end{equation*}

    La figure \ref{fig:imp_fourier:cylindre:hoppe:62:hoibc:mode_2} permet de vérifier les résultats de \cite[p.~62]{hoppe_impedance_1995} pour une couche de matériau sans perte. On prend un incidence normale au cylindre, donc les matrices d'impédance exactes et approchées sont diagonales. On fait varier \(n\) de 0 à \(\lceil k_0 (r_0+d) \rceil\). On trace sur cette figure les parties imaginaires des termes diagonaux pour chaque \(n\) de cette matrice, les parties réelles étant nulles. Les coefficients sont calculés sans contraintes.

    \begin{figure}[!hbt]
      \centering
      \tikzsetnextfilename{Z_HOPPE_62_cylindre_hoibc_mode_2.TM}
\begin{tikzpicture}[scale=1]
    \begin{axis}[
            title={Polarisation TM},
            ylabel={\(\Im(\hat{Z}(n,0)\)},
            xlabel={\(k_t\slash k_0\)},
            width=0.4\textwidth,
            xmin=0,
            xmax=1.5,
            mark repeat=1,
            legend pos=outer north east
        ]
        \addplot [black,mark=square*] table [col sep=comma, x={s1}, y={Im(z_ex.tm)}] {csv/HOPPE_62/HOPPE_62.z_ex.MODE_2_TYPE_C_+3.000E-02.csv};

        \addplot [blue,mark=x] table [col sep=comma, x={s1}, y={Im(z_ibc0.tm)}] {csv/HOPPE_62/HOPPE_62.z_ibc.IBC_ibc0_SUC_F_MODE_2_TYPE_C_+3.000E-02.csv};

        \addplot [red,mark=diamond*] table [col sep=comma, x={s1}, y={Im(z_ibc3.tm)}] {csv/HOPPE_62/HOPPE_62.z_ibc.IBC_ibc3_SUC_F_MODE_2_TYPE_C_+3.000E-02.csv};

        \addplot [violet,mark=triangle*] table [col sep=comma, x={s1}, y={Im(z_ibc6.tm)}] {csv/HOPPE_62/HOPPE_62.z_ibc.IBC_ibc6_SUC_F_MODE_2_TYPE_C_+3.000E-02.csv};
    \end{axis}
\end{tikzpicture}
\tikzsetnextfilename{Z_HOPPE_62_cylindre_hoibc_mode_2.TE}
\begin{tikzpicture}[scale=1]
    \begin{axis}[
            title={Polarisation TE},
            ylabel={},
            xlabel={\(k_t\slash k_0\)},
            width=0.4\textwidth,
            xmin=0,
            xmax=1.5,
            mark repeat=1,
            legend pos=outer north east
        ]
        \addplot [black,mark=square*] table [col sep=comma, x={s1}, y={Im(z_ex.te)}] {csv/HOPPE_62/HOPPE_62.z_ex.MODE_2_TYPE_C_+3.000E-02.csv};
        \addlegendentry{Exact};

        \addplot [blue,mark=x] table [col sep=comma, x={s1}, y={Im(z_ibc0.te)},color=] {csv/HOPPE_62/HOPPE_62.z_ibc.IBC_ibc0_SUC_F_MODE_2_TYPE_C_+3.000E-02.csv};
        \addlegendentry{CI0};

        \addplot [red,mark=diamond*] table [col sep=comma, x={s1}, y={Im(z_ibc3.te)}] {csv/HOPPE_62/HOPPE_62.z_ibc.IBC_ibc3_SUC_F_MODE_2_TYPE_C_+3.000E-02.csv};
        \addlegendentry{CI3};

        \addplot [violet,mark=triangle*] table [col sep=comma, x={s1}, y={Im(z_ibc6.te)}] {csv/HOPPE_62/HOPPE_62.z_ibc.IBC_ibc6_SUC_F_MODE_2_TYPE_C_+3.000E-02.csv};
        \addlegendentry{CI6};
    \end{axis}
\end{tikzpicture}
      \caption[CIOE sur empilement de Hoppe & Rahmat-Samii p.~62]{Partie imaginaire des termes diagonaux des matrices d'impédance pour l'empilement \(\eps = 6\), \(\mu = 1\), \(d=0.0225\text{m}\), \(f=1\text{GHz}\), \(r_0=0.03\text{m}\) en fonction de \(k_t = n / (r_0+d)\).}
      \label{fig:imp_fourier:cylindre:hoppe:62:hoibc:mode_2}
    \end{figure}
    
    On remarque que la CI3 si performante dans l'approximation plan infini ne donnent pas de bons résultats dans l’approximation cylindre infini. 
    En effet, pour \(n=k_z=0\) la matrice d'impédance exacte n'est pas une constante mais une matrice diagonale. 
    Or par définition la CIOE est une constante pour ce couple, c'est la CI0. On subit donc cette erreur dans les résultats. 

    Une CIOE plus intéressante, que l'on nomme CI6 et qui est inspirée de \cite[p.~60]{hoppe_impedance_1995}, serait alors:

    \begin{equation*}
      \left(\oI + c_1\frac{\LD}{k_0^2} -c_2\frac{\LR}{k_0^2}\right)\vE_t = \left(\diag{a_1}{a_2} + b_1\frac{\LD}{k_0^2} - b_2 \frac{\LR}{k_0^2} \right)\vJ
    \end{equation*}

    \begin{table}[!hbt]
      \centering
      % On fait deux tables de même hauteur
      \begin{minipage}[t]{0.49\textwidth}
        \vspace{0pt}
        \centering
        \begin{coefftable}{\hyperlink{ci0}{CI0}}
          \input{csv/HOPPE_62/HOPPE_62.IBC_ibc0_SUC_F_MODE_2_TYPE_C_+3.000E-02.coeff.txt}
        \end{coefftable}
        \begin{coefftable}{\hyperlink{ci3}{CI3}}
          \input{csv/HOPPE_62/HOPPE_62.IBC_ibc3_SUC_F_MODE_2_TYPE_C_+3.000E-02.coeff.txt}
        \end{coefftable}
      \end{minipage}
      \begin{minipage}[t]{0.49\textwidth}
        \vspace{0pt}
        \centering
        \begin{coefftable}{\hyperlink{ci6}{CI6}}
          \input{csv/HOPPE_62/HOPPE_62.IBC_ibc6_SUC_F_MODE_2_TYPE_C_+3.000E-02.coeff.txt}
        \end{coefftable}
      \end{minipage}
      \caption{Coefficients associés à la figure \ref{fig:imp_fourier:cylindre:hoppe:62:hoibc:mode_2}}
      \label{tab:imp_fourier:cylindre:hoppe:62:hoibc:mode_2}
    \end{table}

    Cependant cette CIOE ne sera pas retenue car son implémentation dans le code équation intégrale nécessite une modification de ce dernier.


