\section{Calcul des coefficients de la CI3 par moindres carrés sur l'impédance}

  \subsection{Expression de la fonctionnelle}

    \begin{defn}
      On définit \(\mH_{CI3}\) la fonction de \(\NN \times \RR \times \mathcal{M}_2(\CC) \rightarrow \mathcal{M}_{4\times5}(\CC)\) telle que
      \begin{equation*}
        \mH_{CI3}(n,k_z,\mA) = \begin{bmatrix}
        1 & \hat{\mLD}(n,k_z)_{11} & -\hat{\mLR}(n,k_z)_{11} & -\left(\hat{\mLD}(n,k_z){\mA}\right)_{11} & \left(\hat{\mLR}(n,k_z){\mA}\right)_{11}
        \\
        0 & \hat{\mLD}(n,k_z)_{12} & -\hat{\mLR}(n,k_z)_{12} & -\left(\hat{\mLD}(n,k_z){\mA}\right)_{12} & \left(\hat{\mLR}(n,k_z){\mA}\right)_{12}
        \\
        0 & \hat{\mLD}(n,k_z)_{21} & -\hat{\mLR}(n,k_z)_{21} & -\left(\hat{\mLD}(n,k_z){\mA}\right)_{21} & \left(\hat{\mLR}(n,k_z){\mA}\right)_{21}
        \\
        1 & \hat{\mLD}(n,k_z)_{22} & -\hat{\mLR}(n,k_z)_{22} & -\left(\hat{\mLD}(n,k_z){\mA}\right)_{22} & \left(\hat{\mLR}(n,k_z){\mA}\right)_{22}
        \end{bmatrix}
      \end{equation*}
      On définit \(b\) la fonction de \(\mathcal{M}_2(\CC) \rightarrow \mathcal{M}_{4\times1}(\CC)\) telle que
      \begin{equation*}
        b(\mA) = \begin{bmatrix}
        {\mA}_{11}
        \\
        {\mA}_{12}
        \\
        {\mA}_{21}
        \\
        {\mA}_{22}
        \end{bmatrix}
      \end{equation*}
    \end{defn}

    \begin{prop}
      Soit \(X = (a_0,a_1,a_2,b_1,b_2)\), \((n,k_z)\) fixé et \(\hat\mZ_{ex}\) l'opérateur d'impédance exact du cylindre, alors
      \begin{equation*}
        \argmin{X\in\CC^5} \norm{\hat\mZ_{CI3}(n,k_z,X) - \hat\mZ_{ex}(n,k_z)} = \argmin{X\in\CC^5} \norm{\mH_{CI3}(n,k_z,\hat\mZ_{ex}(n,k_z))X - b(\hat\mZ_{ex}(n,k_z))}^2
      \end{equation*}
    \end{prop}

    \begin{proof}
      C'est la même méthodologie que pour le plan.
      On rappelle que dans la section précédente, on a introduit
      \begin{multline*}
        \hat{\mZ}_{CI3}(n,k_z) = \left(\mI + b_1 \hat{\mLD}(n,k_z) - b_2 \hat{\mLR}(n,k_z) \right)^{-1}\\\left(a_0 \mI + a_1 {\hat{\mLD}(n,k_z)} - a_2 {\hat{\mLR}(n,k_z)}\right)
      \end{multline*}
      On pose \(\hat\mZ_D(n,k_z) = \mI + b_1 \hat{\mLD}(n,k_z) - b_2 \hat{\mLR}(n,k_z)\) et \(\hat\mZ_N(n,k_z) = a_0 \mI + a_1 {\hat{\mLD}(n,k_z)} - a_2 {\hat{\mLR}(n,k_z)}\) donc

      \begin{align*}
      &{}~ \argmin{X\in\CC^5} \norm{\hat\mZ_{CI3}(n,k_z,X) - \hat\mZ_{ex}(n,k_z)}
      \\
      & = \argmin{X\in\CC^5} \norm{\hat\mZ_D(n,k_z)^{-1}\hat\mZ_N(n,k_z) - \hat\mZ_{ex}(n,k_z) }
      \\
      &= \argmin{X\in\CC^5} \norm{\hat\mZ_D(n,k_z)^{-1}\left(\hat\mZ_N(n,k_z) - \hat\mZ_D(n,k_z)\hat\mZ_{ex}(n,k_z)\right) }
      \\
      &= \argmin{X\in\CC^5} \norm{\hat\mZ_N(n,k_z) - \hat\mZ_D(n,k_z)\hat\mZ_{ex}(n,k_z)}
      \\
      &= \argmin{X\in\CC^5} \norm{\hat\mZ_N(n,k_z) - \left(b_1 \hat{\mLD}(n,k_z) - b_2 \hat{\mLR}(n,k_z)\right)\hat\mZ_{ex}(n,k_z) - \hat\mZ_{ex}(n,k_z) }
      \\
      &= \argmin{X\in\CC^5} \norm{\mH_{CI3}(n,k_z,\hat\mZ_{ex}(n,k_z))X - b(\hat\mZ_{ex}(n,k_z))}
      \end{align*}
    \end{proof}

    On tronque la série de Fourier à \(N_{n}\) termes et on se dote de \(N_{k_z}\) \(k_z\). Il existe donc \(N_{n}N_{k_z}\) couples tels que \((n_i,k_{zj}) = (n,k_z)_{(j-1)N_{n}+i}\).
    \begin{defn}
      On définit \(\mA_{CI3}\) la matrice de \(\mathcal{M}_{4N_{n}N_{k_z}\times5}(\CC)\) telle que
      \begin{equation*}
        \mA_{CI3} = 
        \begin{bmatrix}
          \mH_{CI3}(n_1,k_{z1},\hat\mZ_{ex}(n_1,k_{z1}))
          \\
          \vdots
          \\
          \mH_{CI3}(n_i,k_{zj},\hat\mZ_{ex}(n_i,k_{zj}))
          \\
          \vdots
          \\
          \mH_{CI3}(n_{N_n},k_{zN_{k_z}},\hat\mZ_{ex}(n_{N_n},k_{zN_{k_z}}))
        \end{bmatrix}
      \end{equation*}
      On définit \(g\) la matrice de \(\mathcal{M}_{4N_{n}N_{k_z}\times1}(\CC)\) telle que
      \begin{equation*}
        g = 
        \begin{bmatrix}
          b(\hat\mZ_{ex}(n_1,k_{z1}))
          \\
          \vdots
          \\
          b(\hat\mZ_{ex}(n_i,k_{zj}))
          \\
          \vdots
          \\
          b(\hat\mZ_{ex}(n_{N_n},k_{zN_{k_z}}))
        \end{bmatrix}
      \end{equation*}
    \end{defn}

    On peut alors calculer les coefficients de la CI3
    \begin{defn}
      On définit la fonctionnelle \(J_Z\)
      \begin{equation*}
        J_Z(X) = \norm{{\mA}_{CI3}X - {g}}^2
      \end{equation*}
    \end{defn}
    \begin{thm}[Minimisation sans contraintes pour la CI3]

      Les coefficients de la CIOE sont solutions du problème

      Trouver \(X^* \in \CC^5\) tel que
      \begin{equation*}
        X^* = \argmin{X\in \CC^5}  J_Z(X)
      \end{equation*}
    \end{thm}

    \begin{prop}
      \label{prop:cylindre:minimisation:minimum_sans_contraintes}
      Si \(\conj{\mA_{CI3}^t}\mA_{CI3}\) est inversible alors
      \begin{equation*}
        X^* = (\conj{\mA_{CI3}^t}\mA_{CI3})^{-1}\conj{\mA_{CI3}^t}g
      \end{equation*}
    \end{prop}

    \begin{proof}
      Soit \(\ps{\cdot}{\cdot}\) le produit scalaire associé à la norme \(\norm{\cdot}\). Soit \(J\) la fonctionnelle telle que
      \begin{align*}
      J(X) &= \norm{\mA_{CI3}X - g}^2
      \\
      &= \ps{\mA_{CI3}X - g}{\mA_{CI3}X - g}
      \intertext{le produit scalaire est symétrique donc \(\ps{a}{b} = \ps{b}{a}\)}
      &=\ps{\mA_{CI3}X}{\mA_{CI3}X} - 2\ps{\mA_{CI3}X}{g} + \ps{g}{g}
      \\
      &=\ps{X}{\conj{\mA_{CI3}^t}\mA_{CI3}X} - 2\ps{X}{\conj{\mA_{CI3}^t}g} + \ps{g}{g}
      \end{align*}
      Si la matrice \(\conj{\mA_{CI3}^t}\mA_{CI3}\) est inversible, comme elle est par définition hermitienne et positive alors la fonctionnelle est quadratique. Donc il existe une unique minimum où le gradient de cette fonctionnelle s'annule. Or
      \begin{align*}
        \nabla J_Z(X^*) &= 2\conj{\mA_{CI3}^t}\mA_{CI3}X^* - 2\conj{\mA_{CI3}^t}g
        \\ 
        &= 0
      \end{align*}
    \end{proof}

    Nous n'avons pas réussi à démontrer que cette matrice était définie pour tout empilement et tout incidence.

    \begin{thm}[Minimisation avec contraintes pour la CI3]

      Soit \(\CSU[3]{CI3}\) le sous-espace de \(\CC^5\) issu de la définition \ref{def:csu:ci3-3}.
      Alors les coefficients de la CIOE respectant les CSU sont solutions du problème

      Trouver \(X^* \in \CC^5\) tel que
      \begin{equation*}
        X^* = \argmin{X\in \CSU[3]{CI3}}  J_Z(X)
      \end{equation*}
    \end{thm}
