\section{Calcul des coefficients des CIOE par moindres carrés sur l'impédance}

  \subsection{Expression des moindre carrés dans le cadre de l'approximation cylindre infini pour une incidence}

  \subsection{Expression des moindre carrés dans le cadre de l'approximation cylindre infini avec un balayage en incidence}

  \subsection{Résultats numériques sur l'approximation de la matrice d'impédance}

    La figure \ref{fig:imp_fourier:plan:hoppe:62:hoibc:ibc6} permet de vérifier les résultats de \cite[p.~62]{hoppe_impedance_1995} pour une couche de matériau sans perte.

    % \begin{figure}[!hbt]
    %   \centering
    %   \begin{tikzpicture}[scale=1]
    \begin{axis}[
            title={Polarisation TM},
            ylabel={\(\Im(\hat{Z}(k_t r_{1},0)\)},
            xlabel={\(k_t\slash k_0\)},
            width=0.4\textwidth,
            xmin=0,
            xmax=2.5,
            mark repeat=20,
            legend pos=outer north east
        ]
        \addplot [black,mark=square*] table [col sep=comma, x={s1}, y={Im(z_ex.tm)}] {csv/HOPPE_62/HOPPE_62.z_ex.C_+3.000E-02.csv};

        \addplot [blue,mark=x] table [col sep=comma, x={s1}, y={Im(z_ibc0.tm)}] {csv/HOPPE_62/HOPPE_62.z_ibc.IBC_ibc0_TYPE_C_+3.000E-02_SUC_F.csv};

        \addplot [red,mark=diamond*] table [col sep=comma, x={s1}, y={Im(z_ibc3.tm)}] {csv/HOPPE_62/HOPPE_62.z_ibc.IBC_ibc3_TYPE_C_+3.000E-02_SUC_F.csv};
    \end{axis}
\end{tikzpicture}
\begin{tikzpicture}[scale=1]
    \begin{axis}[
            title={Polarisation TE},
            ylabel={},
            xlabel={\(k_t\slash k_0\)},
            width=0.4\textwidth,
            xmin=0,
            xmax=2.5,
            mark repeat=20,
            legend pos=outer north east
        ]
        \addplot [black,mark=square*] table [col sep=comma, x={s1}, y={Im(z_ex.te)}] {csv/HOPPE_62/HOPPE_62.z_ex.C_+3.000E-02.csv};
        \addlegendentry{Exact};

        \addplot [blue,mark=x] table [col sep=comma, x={s1}, y={Im(z_ibc0.te)},color=] {csv/HOPPE_62/HOPPE_62.z_ibc.IBC_ibc0_TYPE_C_+3.000E-02_SUC_F.csv};
        \addlegendentry{CI0};

        \addplot [red,mark=diamond*] table [col sep=comma, x={s1}, y={Im(z_ibc3.te)}] {csv/HOPPE_62/HOPPE_62.z_ibc.IBC_ibc3_TYPE_C_+3.000E-02_SUC_F.csv};
        \addlegendentry{CI3};
    \end{axis}
\end{tikzpicture}
    %   \caption[CIOE sur empilement de Hoppe & Rahmat-Samii p.~62]{\(\eps = 6, \mu = 1, d=0.0225\text{m}, f=1\text{GHz}, r_0=0.03\text{m}\)}
    %   \label{fig:imp_fourier:plan:hoppe:62:hoibc}
    % \end{figure}
    % \begin{table}[!hbt]
    %   \centering
    %   % On fait deux tables de même hauteur
    %   \begin{coefftable}{\hyperlink{ci0}{CI0}}
    %     \input{csv/HOPPE_62/HOPPE_62.IBC_ibc0_SUC_F_MODE_2_TYPE_C_+3.000E-02.coeff.txt}
    %     \\
    %     \\
    %     \\
    %     \\
    %     \\
    %   \end{coefftable}
    %   \begin{coefftable}{\hyperlink{ci3}{CI3}}
    %     \input{csv/HOPPE_62/HOPPE_62.IBC_ibc3_SUC_F_MODE_2_TYPE_C_+3.000E-02.coeff.txt}
    %   \end{coefftable}
    %   \caption{Coefficients associés à la figure \ref{fig:imp_fourier:plan:hoppe:62:hoibc}}
    %   \label{tab:imp_fourier:plan:hoppe:62:hoibc}
    % \end{table}

    On remarque que la CI3 si performante dans l'approximation plan infini ne donnent pas de bons résultats dans l’approximation cylindre infini. 
    En effet, la matrice d'impédance exacte n'est pas une constante mais une matrice diagonale pour \(n=k_z=0\). 
    Or par définition la CIOE est une constante pour ce couple, c'est la CI0. On subit donc cette erreur dans les résultats. 

    Une CIOE plus intéressante, que l'on nomme CI6 et qui est inspirée de \cite[p.~60]{hoppe_impedance_1995}, serait alors:

    \begin{equation}
      \left(\oI + c_1\frac{\LD}{k_0^2} -c_2\frac{\LR}{k_0^2}\right)\vE_t = \left(\diag{a_1}{a_2} + b_1\frac{\LD}{k_0^2} - b_2 \frac{\LR}{k_0^2} \right)\vJ
    \end{equation}

    \begin{figure}[!hbt]
      \centering
      \input{tikz/plot/Z_HOPPE_62_cylindre_hoibc_ibc6.tikz}
      \caption[CIOE sur empilement de Hoppe & Rahmat-Samii p.~62]{\(\eps = 6, \mu = 1, d=0.0225\text{m}, f=1\text{GHz}, r_0=0.03\text{m}\)}
      \label{fig:imp_fourier:plan:hoppe:62:hoibc:ibc6}
    \end{figure}
    \begin{table}[!hbt]
      \centering
      % On fait deux tables de même hauteur
      \begin{minipage}[t]{0.49\textwidth}
        \vspace{0pt}
        \centering
        \begin{coefftable}{\hyperlink{ci0}{CI0}}
          \input{csv/HOPPE_62/HOPPE_62.IBC_ibc0_SUC_F_MODE_2_TYPE_C_+3.000E-02.coeff.txt}
        \end{coefftable}
        \begin{coefftable}{\hyperlink{ci3}{CI3}}
          \input{csv/HOPPE_62/HOPPE_62.IBC_ibc3_SUC_F_MODE_2_TYPE_C_+3.000E-02.coeff.txt}
        \end{coefftable}
      \end{minipage}
      \begin{minipage}[t]{0.49\textwidth}
        \vspace{0pt}
        \centering
        \begin{coefftable}{\hyperlink{ci6}{CI6}}
          \input{csv/HOPPE_62/HOPPE_62.IBC_ibc6_SUC_F_MODE_2_TYPE_C_+3.000E-02.coeff.txt}
        \end{coefftable}
      \end{minipage}
      \caption{Coefficients associés à la figure \ref{fig:imp_fourier:plan:hoppe:62:hoibc:ibc6}}
      \label{tab:imp_fourier:plan:hoppe:62:hoibc:ibc6}
    \end{table}

    Cependant cette CIOE ne sera pas retenue car son implémentation dans le code équation intégrale nécessite une modification de ce dernier. On présente néanmoins dans la figure \ref{fig:imp_fourier:plan:hoppe:62:hoibc:ibc6} sa performance vis à vis de la CI3 sur le cylindre.


