\section{Approximation de la matrice d'impédance pour un plan infini par une CIOE}

    Dans le cadre de cette thèse, nous nous intéressons à la condition d'impédance d'\cite{aubakirov_electromagnetic_2014} et à ces dérivées qui s'exprime en tout point de la surface extérieur de notre objet telle que

    \begin{align}
        \left(I + b_1 \frac{\LD}{k_0^2} - b_2 \frac{\LR}{k_0^2} \right)\vE_t = \left(a_0 I + a_1 \frac{\LD}{k_0^2} - a_2 \frac{\LR}{k_0^2} \right)(\vn \pvect \vH) & \forall x \in \Gamma
    \end{align}

    Nous la nommons \hyperlink{ci3}{CI3}.

    Pour approcher la matrice \(\hat{\mZ}\) que l'on a défini précédemment, il convient donc d'exprimer les opérateurs \(\LD, \LR\) en tant que multiplicateur de Fourier.

  \subsection[Expression des opérateurs LD LR en Fourier]{Expression des opérateurs \(\LD,\LR\) en Fourier}

    Par définition de \(\LD\), on a
    \begin{align}
      \LD \vE_t & = \vgrads{} \vdivs{} \vE_t
    \end{align}

    Or d’après la définition de la transformée de Fourier

    \begin{align}
      \vE_t(x,y,z) & = \frac{1}{2\pi}\int_\RR\int_\RR \hat{\vE_t}(k_x,k_y,z)e^{ik_xx + ik_yy}\dd{k_x}\dd{k_y}
    \end{align}

    donc si on applique l'opérateur à ce vecteur

    \begin{align}
      \LD \vE_t
      & = \vgrads{} \vdivs{} \vE_t
      \\
      &=\frac{1}{2\pi}\vgrads{} \int_\RR\int_\RR \hat{\vE_t}(k_x,k_y,z) \cdot \vgrads{} e^{ik_xx + ik_yy}\dd{k_x}\dd{k_y}
      \\
      &=\frac{1}{2\pi}\int_\RR \int_\RR \vhesss{}\left(\left( e^{ik_xx + ik_yy} \right) \hat{\vE_t}\right)(k_x,k_y,z)\dd{k_x}\dd{k_y}
    \end{align}

    On définit \(\hat{\mLD}\) l'opérateur matriciel tel que
    \begin{align}
      \LD \vE_t
      &= \frac{1}{2\pi}\int_\RR\int_\RR \hat{\mLD} \hat{\vE_t}(k_x,k_y,z)\dd{k_x}\dd{k_y}
    \end{align}

    Son expression est de ce qui précède
    \begin{equation}
      \label{eq:plan:fourier:LD}
      \hat{\LD}(k_x,k_y) = -
      \begin{bmatrix}
        k_x^2 & k_x k_y
        \\
        k_x k_y & k_y^2
      \end{bmatrix}
    \end{equation}
    Voilà pourquoi lors de l'expression de la matrice \(\hat\mZ\), nous avions nommée cette matrice ainsi.

    On reprend exactement la même méthode pour l'opérateur \(\LR\):
    \begin{align}
      \LR \vE_t & = \vrots{} \vrots{} \vE_t
      \\
      &=\frac{1}{2\pi}\vrots{} \int_\RR\int_\RR \vgrads{}\left(e^{ik_xx + ik_yy}\right) \pvect \hat{\vE_t}(k_x,k_y,z)\dd{k_x}\dd{k_y}
      \\
      &= \frac{1}{2\pi}\int_\RR \int_\RR \left(\vhesss - \vlapls\right) \left(\left(e^{ik_xx + ik_yy}\right) \hat{\vE_t}\right)(k_x,k_y,z)\dd{k_x}\dd{k_y}
    \end{align}

    On définit \(\hat{\LR}\) l'opérateur matriciel tel que
    \begin{align}
      \LR \vE_t
      &= \frac{1}{2\pi}\int_\RR\int_\RR \hat{\LR} \hat{\vE_t}(k_x,k_y,z)\dd{k_x}\dd{k_y}
    \end{align}

    \begin{equation}
      \hat{\LR}(k_x,k_y) =
      \begin{bmatrix}
        k_y^2 & -k_x k_y
        \\
        -k_x k_y & k_x^2
      \end{bmatrix}
    \end{equation}

  \subsection{Expression de la matrice d'impédance approchée par la CI3}

    On peut donc définir \(\hat{\mZ}_{CI3}\) l’opérateur matriciel associé à la condition d'impédance.

    \begin{multline}
        \hat{\mZ}_{CI3}(k_x,k_y) = \left(\mI + b_1 \frac{\hat{\mLD}}{k_0^2}(k_x,k_y) - b_2 \frac{\hat{\mLR}}{k_0^2}(k_x,k_y) \right)^{-1}
        \\
        \left(a_0 \mI + a_1 \frac{\hat{\mLD}}{k_0^2}(k_x,k_y) - a_2 \frac{\hat{\mLR}}{k_0^2}(k_x,k_y)\right)
    \end{multline}

    Les CIOE dérivées de la CI3 s'obtiennent tout aussi facilement.

    \begin{equation}
        \hat{\mZ}_{CI4}(k_x,k_y) = a_0 \mI + a_1 \frac{\hat{\mLD}}{k_0^2}(k_x,k_y) - a_2 \frac{\hat{\mLR}}{k_0^2}(k_x,k_y)
    \end{equation}
    \begin{equation}
        \hat{\mZ}_{CI1}(k_x,k_y) =  \left(\mI + b \frac{\hat{\mLD} - \hat{\mLR}}{k_0^2}(k_x,k_y) \right)^{-1}\left(a_0\mI + a_1 \frac{\hat{\mLD} - \hat{\mLR}}{k_0^2}(k_x,k_y) \right)
    \end{equation}
    \begin{equation}
        \hat{\mZ}_{CI01}(k_x,k_y) =  a_0\mI + a_1 \frac{\hat{\mLD} - \hat{\mLR}}{k_0^2}(k_x,k_y)
    \end{equation}
    Pour calculer les coefficients de la CIOE, il faut minimiser la distance entre le symbole \(\hat{\mZ}(k_x,k_y)\) et \(\hat{\mZ}_{CI3}(k_x,k_y)\) \footnote{La distance est la norme de Frobenius}. Évidemment, il existe une infinité de combinaisons pour un couple \((k_x,k_y)\) données. Nous avons choisis de nous donner un grand nombre de couples et de minimiser au sens des moindres carrés.

  \subsection{Le cas spécial de la condition de Leontovich}

    La condition de Leontovich consiste à approcher le symbole par une constante, et donc d'avoir \(a_1=a_2=b_1=b_2=0\). Dans le paragraphe précèdent, cela revient à chercher la valeur moyenne du symbole. Cependant, cette condition rend compte exactement de l'impédance à incidence normale. Dans ce cas particulier, nous ne réaliserons pas de minimisation mais on fixera cette constante à la valeur du symbole en \((0,0)\).

    \begin{align}
      \hat{\mZ}_{CI0} & = a_0\mI 
      \\
      &= \hat{\mZ}(0,0)
    \end{align}