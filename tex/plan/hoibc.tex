\section{Approximation par une CIOE de l'opérateur de Calderón plan}

  \subsection[Expression des opérateurs LD, LR en Fourier]{Expression des opérateurs \(\LD\), \(\LR\) en Fourier}
    \label{sec:plan:hoibc:LD-LR}
    Soient \((x,y,z)\) les coordonnées plan d'un point de l'espace et un plan d'équation \(z=0\).

    Soit \(V = \left(\mathcal{C}^\infty(\RR^2)\right)^2 \cap L^2(\RR^2)\) l'espace des fonctions infiniment dérivables définies sur ce plan et de carré intégrable.

    \begin{defn}
      \label{eq:plan:fourier:LD}
      On définit \(\LD\) l'opérateur de \(V\) tel que
      % \begin{REM}
      %   Ce n'est pas un endomorphisme, je te l'ai dit car tu n'es pas \(L^2\) pour les composantes et c'est \((L^2(\RR^2))^2\).
      %   C'est plutôt un symbole de Fourier ou un multiplicateur de Fourier matriciel.
      %   Et la LD agit sur les champs tangents du plan.
      % \end{REM}
      \begin{align*}
        \LD \vect{U}(x,y,z) & = \vgrads{} \vdivs{} \vect{U}(x,y,z).
      \end{align*}

      On définit \(\hat{\mLD}\) la fonction de \(\RR\times\RR \rightarrow \mathcal{M}_2(\RR)\) telle que
      \begin{equation*}
        \hat{\mLD}(k_x,k_y) = -
        \begin{bmatrix}
          k_x^2 & k_xk_y
          \\
          k_xk_y & k_y^2
        \end{bmatrix}.
      \end{equation*}
    \end{defn}

    \begin{prop}
      Soit \(\vect{U} \in V\), alors
      % \begin{REM}
      %   Si tu en as besoin, il faut prendre \(\vect{U}\in (\mathcal{S}(\RR^2))^2\)
      % \end{REM}
      \begin{equation*}
        \widehat{\LD \vect{U}} (k_x,k_y,0) = \hat{\mLD}(k_x,k_y) \hat{\vect{U}}(k_x,k_y,0).
      \end{equation*}
    \end{prop}

    \begin{proof}
      Par définition de \(\LD\), on a
      \begin{align*}
        \LD \vect{U} & = \vgrads{} \vdivs{} \vect{U}.
      \end{align*}
      On utilise les expressions en coordonnées cartésiennes des opérateurs aux dérivées partielles (voir annexe \ref{sec:annexe:div_grad_rot}).
      \begin{align*}
        \vdivs{\vect{U}}(x,y,z) &= \ddr{x}{U_x}(x,y,z) + \ddr{y}{U_z}(x,y,z),
        \\
        \vgrads{f}(x,y,z) &= \ddr{x}{f}(x,y,z)\vect{e_x} + \ddr{y}{f}(x,y,z)\vect{e_y}.
      \end{align*}
      Or d’après la définition de la transformée de Fourier
      \begin{align*}
        \vect{U}(x,y,z) & = \frac{1}{2\pi} \int_\RR \int_\RR \hat{\vect{U}}(k_x,k_y,z)e^{ik_xx + ik_yy}\dd{k_x}\dd{k_y},
      \end{align*}
      on a
      % \begin{REM}
      %   mal dit
      % \end{REM}
      \begin{align*}
        \widehat{\vdivs{\vect{U}}}(k_x,k_y,z) &= ik_x{\hat{U}_x}(k_x,k_y,z) + ik_y{\hat{U}_y}(k_x,k_y,z),
        \\
        \widehat{\vgrads{f}}(k_x,k_y,z) &= ik_x\hat{f}(k_x,k_y,z)\vect{e_x} + ik_y\hat{f}(k_x,k_y,z)\vect{e_y},
      \end{align*}
      donc
      \begin{align*}
        \widehat{\LD{\vect{U}}}(k_x,k_y,z) =  \left(-k_x^2\vect{e_x} -k_xk_y\vect{e_y}\right){\hat{U}_x}(k_x,k_y,z) + \left(-k_xk_y\vect{e_x} - {k_y^2}\vect{e_y}\right){\hat{U}_z}(k_x,k_y,z).
      \end{align*}

    \end{proof}


    \begin{defn}
      \label{eq:plan:fourier:LR}

      On définit \(\LR\) l'opérateur de \(V\) tel que
      \begin{align*}
        \LR \vect{U}(x,y,z) & = \vrots{} (\rots{} \vect{U})(x,y,z).
      \end{align*}

      On définit \(\hat{\mLR}\) la fonction de \(\RR\times\RR \rightarrow \mathcal{M}_2(\RR)\) telle que
      \begin{equation*}
        \hat{\mLR}(k_x,k_y) = 
        \begin{bmatrix}
          k_y^2 & -k_xk_y
          \\
          -k_xk_y & k_x^2
        \end{bmatrix}.
      \end{equation*}
    \end{defn}

    \begin{prop}
      Soit \(\vect{U} \in V\), alors
      \begin{equation*}
        \widehat{\LR \vect{U}} (k_x,k_y,0^+) = \hat{\mLR}(k_x,k_y) \hat{\vect{U}}(k_x,k_y,0^+).
      \end{equation*}
    \end{prop}

    \begin{proof}
      Par définition de \(\LR\), on a
      \begin{align*}
        \LR \vect{U} & = \vrots{} (\rots{} \vect{U}).
      \end{align*}
      On utilise les expressions en coordonnées cartésiennes des opérateurs aux dérivées partielles (voir annexe \ref{sec:annexe:div_grad_rot}).
      \begin{align*}
        \rots{\vect{U}}(x,y,z) &= \ddr{x}{U_y}(x,y,z) - \ddr{y}{U_x}(x,y,z),
        \\
        \vrots{f}(x,y,z) &= \ddr{y}{f}(x,y,z)\vect{e_x} - \ddr{x}{f}(x,y,z)\vect{e_y},
      \end{align*}
      donc comme pour l'opérateur \(\LD\)
      \begin{align*}
        \widehat{\rots{\vect{U}}}(k_x,k_y,z) &= ik_x{\hat{U}_y}(k_x,k_y,z) - ik_y{\hat{U}_x}(k_x,k_y,z),
        \\
        \widehat{\vrots{f}}(k_x,k_y,z) &=  ik_y\hat{f}(k_x,k_y,z)\vect{e_x} - ik_x\hat{f}(k_x,k_y,z)\vect{e_y},
      \end{align*}
      donc
      \begin{align*}
        \widehat{\LR{\vect{U}}}(k_x,k_y,z) =  \left({k_y^2}\vect{e_x} -k_xk_y\vect{e_y}\right){\hat{U}_x}(k_x,k_y,z) + \left(-k_xk_y\vect{e_x} + k_x^2\vect{e_y}\right){\hat{U}_z}(k_x,k_y,z).
      \end{align*}
    \end{proof}

  \subsection{Expression de la matrice d'impédance approchée par une CI}

    On peut donc définir \(\hat{\mZ}_{CI3}\) l’opérateur matriciel associé à la condition d'impédance.

    \begin{multline}
      \label{eq:plan:hoibc:ci3}
      \hat{\mZ}_{CI3}(k_x,k_y) = \left(\mI + b_1 \frac{\hat{\mLD}(k_x,k_y)}{k_0^2} - b_2 \frac{\hat{\mLR}(k_x,k_y)}{k_0^2} \right)^{-1}\\\left(a_0 \mI + a_1 \frac{\hat{\mLD}(k_x,k_y)}{k_0^2} - a_2 \frac{\hat{\mLR}(k_x,k_y)}{k_0^2}\right).
    \end{multline}


    Les CIOE dérivées de la CI3 s'obtiennent tout aussi facilement.

    \begin{equation*}
        \hat{\mZ}_{CI4}(k_x,k_y) = a_0 \mI + a_1 \frac{\hat{\mLD}}{k_0^2}(k_x,k_y) - a_2 \frac{\hat{\mLR}}{k_0^2}(k_x,k_y),
    \end{equation*}
    \begin{equation*}
        \hat{\mZ}_{CI1}(k_x,k_y) =  \left(\mI + b \frac{\hat{\mLD} - \hat{\mLR}}{k_0^2}(k_x,k_y) \right)^{-1}\left(a_0\mI + a_1 \frac{\hat{\mLD} - \hat{\mLR}}{k_0^2}(k_x,k_y) \right),
    \end{equation*}
    \begin{equation*}
        \hat{\mZ}_{CI01}(k_x,k_y) =  a_0\mI + a_1 \frac{\hat{\mLD} - \hat{\mLR}}{k_0^2}(k_x,k_y).
    \end{equation*}
    Les coefficients de la CIOE peuvent être choisis comme ceux qui minimisent la distance entre \(\hat{\mZ}(k_x,k_y)\) et \(\hat{\mZ}_{CI}(k_x,k_y)\)\footnote{La distance choisie est la norme de Frobenius}.
    % \begin{REM}
    %   Cette phrase ne veut pas dire ce que tu crois. Pour trouver les coefficients de la CIOE que l'on prendra dans le calcul, on choisit de minimiser. ( Pourquoi "il faut", pourquoi ce choix et le calcul. )
    % \end{REM}
    Il existe une infinité de combinaisons pour un couple \((k_x,k_y)\) donné.
    Nous avons choisi de nous donner un grand nombre de couples et de minimiser au sens des moindres carrés.

  \subsection{Le cas spécial de la condition de Leontovich}

    La condition de Leontovich consiste à approcher la matrice par une constante, et donc à avoir \(a_1=a_2=b_1=b_2=0\). Cependant, cette condition rend compte exactement de l'impédance à incidence normale. Dans ce cas particulier, nous ne réaliserons pas de minimisation, mais on fixera cette constante à la valeur de la matrice quand \((k_x,k_y) = (0,0)\) donc \( \hat{\mZ}_{CI0}  = \hat{\mZ}(0,0)\). Cette dernière est diagonale.
