\section{Expressions exactes des matrices d'impédance et des matrices de réflexions pour un plan infini}
    % Ce cas est très bien documenté (\cite{senior_approximate_x995},\cite{hoppe_impedance_x995}) et pose la méthodologie à adopter pour les objets courbes.

    Dans un premier temps, on peut sans perte de généralités faire une rotation du repère pour avoir le plan orthogonal à \(\vect{z}\).

    \begin{figure}[!h]
        \begin{center}
            \tikzsetnextfilename{plan_1_couche}
            \begin{tikzpicture}
                \tikzmath{
    \largeur = 6;
    \hauteur = 1;
    \milieu = 1.3;
    \xC = \largeur;
    \xA = 0;
}

%% 1ere couche
\tikzmath{
    \yC = \hauteur;
    \yA = 0;
}

\coordinate (A) at (\xA,\yA);
\coordinate (B) at (\xA,\yC);
\coordinate (C) at (\xC,\yC);

\draw ($(B)!0.5!(C)$) node [above] {vide};


\fill [lightgray] (A) rectangle (C);
\draw ($(A)!0.5!(C)$) node {$\peps,\pmu,d$};
\draw (B) -- (C) node [right] {$\z = 0$};

%% Le repère
\tikzmath{
    \xD = \xC + 1.5;
}

\coordinate (n) at (\xD,\yA);

\draw [->] (n) -- ++(0,1) node [at end, right] {$\v{\z}$};
\draw [->] (n) -- ++(1,0) node [at end, right] {$\v{\x}$};

\draw (n) circle(0.1cm) node [below=0.1cm] {$\v{\y}$};
\draw (n) +(135:0.1cm) -- +(315:0.1cm);
\draw (n) +(45:0.1cm) -- +(225:0.1cm);

%% Le conducteur
\tikzmath{
    \yC = \yC - \hauteur;
    \yA = \yA - 0.5*\hauteur;
}

\coordinate (A) at (\xA,\yA);
\coordinate (B) at (\xA,\yC);
\coordinate (C) at (\xC,\yC);
\draw (B) -- (C);

\fill [pattern=north east lines] (A) rectangle (C);



            \end{tikzpicture}
        \end{center}
    \end{figure}

    % Comme il est infini dans les directions \(\vect{e_x} ,\vect{e_y}\) et que le matériau est homogène isotrope, 
    Nous utilisons une transformée partielle en \(x, y\).

    \begin{equation}
        \vE(x,y,z) = \frac{1}{2\pi}\iint_{\RR^2} e^{i(k_x x + k_y y)}\hat{\vE} (k_x,k_y,z) \dd{k_x}\dd{k_y}
    \end{equation}

    Grâce à cela, nous simplifions le problème en étudiant \( \hat{\vE}\) grâce aux multiplicateurs de Fourier associés aux variables \(x,y\). 

    \begin{prop}
        Dans le matériau, on définit \(k_3\) le nombre complexe tel que \(k_3^2 = k^2 - k_x^2 - k_y^2\).
        Alors existe \((c_i(k_x,k_y))_{1\le i \le4}, \in \CC(\RR^2)^4\) tels que
        \begin{subequations}
            \begin{align}
                \hat{\vE}_y(k_x,k_y,z) & = e^{ik_3z}c_1(k_x,k_y) + e^{-ik_3z}c_2(k_x,k_y)
                \\
                \hat{\vH}_y(k_x,k_y,z) & = e^{ik_3z}c_3(k_x,k_y) + e^{-ik_3z}c_4(k_x,k_y)
            \end{align}
        \end{subequations}
    \end{prop}

    \begin{proof}
        On commence par coupler les équations de Maxwell pour obtenir des équations sur chaque champ seul
        \begin{align}
            \vlapl{\vE} + k^2 \vE & = 0
            \\
            \vlapl{\vH} + k^2 \vH & = 0
        \end{align}

        Donc chaque composante de chaque champ est solution d'une équation de Helmholtz. On s’intéresse alors à la composante en \(y\) seulement:
        \begin{align}
            \lapl{\vE_y} + k^2 \vE_y & = 0
            \\
            \lapl{\vH_y} + k^2 \vH_y & = 0
        \end{align}

        Si l'on applique ces équations différentielles aux transformées de Fourier, on obtient
        \begin{align}
            \ddr[2]{z}{\hat{\vE}_y}(k_x,k_y,z) + \left( k^2 - k_x^2 - k_y^2\right) \hat{\vE}_y(k_x,k_y,z) & = 0
            \\
            \ddr[2]{z}{\hat{\vH}_y}(k_x,k_y,z) + \left( k^2 - k_x^2 - k_y^2\right) \hat{\vH}_y(k_x,k_y,z) & = 0
        \end{align}

        Une solution générale s'écrit alors
        \begin{align}
            \hat{\vE}_y(k_x,k_y,z) & = e^{ik_3z}c_1(k_x,k_y) + e^{-ik_3z}c_2(k_x,k_y)
            \\
            \hat{\vH}_y(k_x,k_y,z) & = e^{ik_3z}c_3(k_x,k_y) + e^{-ik_3z}c_4(k_x,k_y)
        \end{align}
    \end{proof}

    Ce choix est motivé par l'invariance par rotation du plan infini. Or il est d'usage de la représenter comme sur le schéma, avec un propagation de l'onde dans le plan \(xOz\). Souvent, on parle alors de polarisation pour simplifier les calculs. En effet, une polarisation \gls{acr-te} consiste à dire que \(\vE\) n'a qu'une composante dans le plan transverse au plan de propagation, donc suivant \(\vect{e_y}\). De même une polarisation \gls{acr-tm} consiste à dire que \(\vH\) n'a qu'une composante suivant \(\vect{e_y}\). C'est ce qui motive ce choix absolument abstrait d'utiliser les deuxièmes composantes.

    \subsection{Expressions des champs tangentiels dans chaque couche}

        \begin{defn}
            On définit les matrices \(\mA_{E}(k_x,k_y,z)\),\(\mB_{E}(k_x,k_y,z)\),\(\mA_{H}(k_x,k_y,z)\),\(\mB_{H}(k_x,k_y,z)\)
            \begin{align}
              \mA_{E}(k_x,k_y,z) &= e^{ik_3z}
              \begin{bmatrix}
                -\frac{k_xk_y}{k^2 - k_y^2} & -\frac{k\eta k_3}{k^2 - k_y^2}
                \\
                1 & 0
              \end{bmatrix}
              \\
              \mB_{E}(k_x,k_y,z) &= e^{-ik_3z}
              \begin{bmatrix}
                -\frac{k_xk_y}{k^2 - k_y^2} & \frac{k\eta k_3}{k^2 - k_y^2}
                \\
                1 & 0
              \end{bmatrix}
              \\
              \mA_{H}(k_x,k_y,z) &= e^{ik_3z}
              \begin{bmatrix}
                0 & -1
                \\
                \frac{k}{\eta}\frac{k_3}{k^2 - k_y^2} & -\frac{k_xk_y}{k^2 - k_y^2}
              \end{bmatrix}
              \\
              \mB_{H}(k_x,k_y,z) &= e^{-ik_3z}
              \begin{bmatrix}
                0 & -1
                \\
                -\frac{k}{\eta}\frac{k_3}{k^2 - k_y^2} & -\frac{k_xk_y}{k^2 - k_y^2}
              \end{bmatrix}
            \end{align}
        \end{defn}

        \begin{prop}
            Dans chaque couche, il existe \((c_i(k_x,k_y))_{1\le i \le4}, \in \CC(\RR^2)^4\) tels que
            \begin{align}
                \hat{\vE}_t(k_x,k_y,z) &= \mA_E(k_x,k_y,z) \begin{bmatrix}c_1(k_x,k_y) \\ c_3(k_x,k_y)\end{bmatrix} + \mB_E(k_x,k_y,z) \begin{bmatrix}c_2(k_x,k_y) \\ c_4(k_x,k_y)\end{bmatrix}
                \\
                (\vn \pvect \hat{\vH})_t(k_x,k_y,z) &= \mA_H(k_x,k_y,z) \begin{bmatrix}c_1(k_x,k_y) \\ c_3(k_x,k_y)\end{bmatrix} + \mB_H(k_x,k_y,z) \begin{bmatrix}c_2(k_x,k_y) \\ c_4(k_x,k_y)\end{bmatrix}
            \end{align}
        \end{prop}

        \begin{proof}
            On reprend les équations de Maxwell, appliquées sur les transformées de Fourier. À partir de la première et de la dernière composante de chaque équation de Maxwell, on peut déterminer \(\hat{E_x},\hat{E_z},\hat{H_x},\hat{H_z}\).
            \begin{align*}
                \left\lbrace
                \begin{matrix}
                    ik_y \hat{E_z}  - \ddr{z}{\hat{E_y}} = -i \w \mu \hat{H_x}
                    \\
                    ik_x \hat{E_y} - ik_y \hat{E_x} = -i\w \mu \hat{H_z}
                \end{matrix}
                \right. \quad
                \left\lbrace
                \begin{matrix}
                    ik_y \hat{H_z}  - \ddr{z}{\hat{H_y}} = i \w \eps \hat{E_x}
                    \\
                    ik_x \hat{H_y} - ik_y \hat{H_x} = i\w \eps \hat{E_z}
                \end{matrix}
                \right.
            \end{align*}

            Cela revient à résoudre \(Y = \mat{M}X\) où la matrice \(\mat{M}\) et les vecteurs \(X, Y\) sont définis tels que
            \begin{equation}
              \mat{M} =
              \begin{bmatrix}
              0 & -ik_y & -i\w\mu & 0
              \\
              ik_y & 0 & 0 & -i\w\mu
              \\
              i\w\eps & 0 & 0 & -ik_y
              \\
              0 & i\w\eps & ik_y & 0
              \end{bmatrix}
              \,
              X =
              \begin{bmatrix}
                \hat{E_x}\\
                \hat{E_z}\\
                \hat{H_x}\\
                \hat{H_z}
              \end{bmatrix}
              \,
              Y =
              \begin{bmatrix}
                -\ddr{z}{\hat{E_y}}\\
                ik_x\hat{E_z}\\
                -\ddr{z}{\hat{H_y}}\\
                ik_x\hat{H_y}
              \end{bmatrix}
            \end{equation}

            Cette matrice est inversible si \(\det(\mM) = (k^2 - k_y^2)^2 \) est non nul.
            On suppose ce dernier non nul, on peut en déduire \(X\)
            \begin{equation}
              \begin{bmatrix}
                \hat{E_x}\\
                \hat{E_z}\\
                \hat{H_x}\\
                \hat{H_z}
              \end{bmatrix} =
              \frac{1}{k^2 - k_y^2}
              \begin{bmatrix}
              0 & ik_y & -i\w\mu & 0
              \\
              -ik_y & 0 & 0 & -i\w\mu
              \\
              i\w\eps & 0 & 0 & ik_y
              \\
              0 & i\w\eps & -ik_y & 0
              \end{bmatrix}
              \begin{bmatrix}
                -\ddr{z}{\hat{E_y}}\\
                ik_x\hat{E_y}\\
                -\ddr{z}{\hat{H_y}}\\
                ik_x\hat{H_y}
              \end{bmatrix}
            \end{equation}

            On extrait alors uniquement les composantes suivant \(x,y\) de \(\hat{\vE}\) et \(\vect{e_z} \pvect \hat{\vH}\)
            \begin{align}
                \hat{E_x} &= \frac{1}{k^2 - k_y^2}\left(ik_yik_x\hat{E_y} + i\w\mu\ddr{z}{\hat{H_y}}\right)
                \\
                \hat{E_y} &= e^{ik_3z}c_1 + e^{-ik_3z}c_2
                \\
                -\hat{H_y} &= -e^{ik_3z}c_3 - e^{-ik_3z}c_4
                \\
                \hat{H_x} &= \frac{1}{k^2 - k_y^2}\left(-i\w\eps\ddr{z}{\hat{E_y}} + ik_yik_x\hat{H_y}\right)
            \end{align}
            Soit si l'on factorise par les exponentielles
            \begin{align}
                \hat{E_x} &= \frac{1}{k^2 - k_y^2}\left(-\left(k_yk_x c_1 + \w\mu k_3 c_3 \right)e^{ik_3z} - \left(k_yk_xc_2 - \w\mu k_3 c_4\right)e^{-ik_3z}\right)
                \\
                \hat{E_y} &= e^{ik_3z}c_1 + e^{-ik_3z}c_2
                \\
                -\hat{H_y} &= -e^{ik_3z}c_3 - e^{-ik_3z}c_4
                \\
                \hat{H_x} &= \frac{1}{k^2 - k_y^2}\left(\left(\w\eps k_3c_1 - k_y k_x c_3\right)e^{ik_3z} + \left(-\w\eps k_3c_2 - k_y k_x c_4\right)e^{-ik_3z}\right)
            \end{align}

            On obtient alors
            \begin{align}
                \hat{\vE}_t(k_x,k_y,z) &= \mA_E(k_x,k_y,z) \begin{bmatrix}c_1(k_x,k_y) \\ c_3(k_x,k_y)\end{bmatrix} + \mB_E(k_x,k_y,z) \begin{bmatrix}c_2(k_x,k_y) \\ c_4(k_x,k_y)\end{bmatrix}
                \\
                (\vn \pvect \hat{\vH})_t(k_x,k_y,z) &= \mA_H(k_x,k_y,z) \begin{bmatrix}c_1(k_x,k_y) \\ c_3(k_x,k_y)\end{bmatrix} + \mB_H(k_x,k_y,z) \begin{bmatrix}c_2(k_x,k_y) \\ c_4(k_x,k_y)\end{bmatrix}
            \end{align}

        \end{proof}

        Ce résultat est encore valable dans le vide mais les \(c_i\) y sont différents et il faut replacer \(k\) par \(k_0\) et \(\eta\) par \(1\).
        Cette décomposition est intéressante car elle contient des termes de propagation vers le plan, ce sont les matrices \(\mA_E, \mA_H\) et des termes de propagation vers l'infini, ce sont les matrices \(\mB_E, \mB_H\). En effet, le terme \(e^{i\w t}\) omis dans les expression est multiplié par \(e^{\pm i k_3 z}\). 

        Supposons par exemple que \(k_3 = \sqrt{k_0^2 - k_x^2 - k_y^2}\) soit réel, donc positif. Si l'on veut que par exemple \(\w t + k_3 z\) soit constant quand \(t\) augmente, il faut que \(z\) diminue, autrement dit, quand le temps avance, l'onde se rapproche du plan. C'est la partie onde incidente et de fait l'autre partie est l'onde réfléchie.

        Supposons maintenant que \(k_3 = i\sqrt{k_0^2 - k_x^2 - k_y^2}\)  soit imaginaire. Alors si l'on veut des champs réfléchis finis à l'infini, il faut que \(\Im(k_3) = \Im(\sqrt{k_0 - k_x^2 - k_y^2})\) soit négatif, ainsi on a une évolution en \(e^{\Im(k_3)z}\), qui tend bien vers \(0\).

    \subsection{Expression de la matrice d'impédance pour une couche de matériau}

    %%%%%%%%%%%%%%%%%%%%%%%%%%%%%%%%%%%%%%%%%%%%%%%%%%%%%%%%%%%%%%%%%%%%%%%
    %%%%%%%%%%%%%%%%%%%%%%%%%%%%%%%%%%%%%%%%%%%%%%%%%%%%%%%%%%%%%%%%%%%%%%%
    %%%%%%%%%%%%%%%%%%%%%%%%%%%%%%%%%%%%%%%%%%%%%%%%%%%%%%%%%%%%%%%%%%%%%%%

        \begin{defn}
            \label{def:plan:impedance}
            On définit \(\hat \mZ(k_x,k_y)\) l'impédance, le multiplicateur de Fourier matriciel tel que
            \begin{equation*}
                \hat \vE_t(k_x,k_y,0) = \hat \mZ(k_x,k_y) \left(\vect{e_z} \pvect \hat \vH_t(k_x,k_y,0)\right)
            \end{equation*}
        \end{defn}

        On commence par montrer un lemme très utile pour trouver l'expression de \(\hat \mZ(k_x,k_y)\)
        \begin{lemme}[Continuité des impédances]
            \label{lem:plan:continuite_impedance}
            Soit \(z_m\) une interface entre deux matériaux. On considère les impédances de part et d'autre de l'interface
            \begin{align*}
                \hat{\vE}_t(k_x,k_y,z_m^+) &= \hat{\mZ}^+(k_x,k_y) \left(\vect{e_z} \pvect \hat{\vH}_t(k_x,k_y,z_m^+)\right)
                \\
                \hat{\vE}_t(k_x,k_y,z_m^-) &= \hat{\mZ}^-(k_x,k_y) \left(\vect{e_z} \pvect \hat{\vH}_t(k_x,k_y,z_m^-)\right)
            \end{align*}
            alors il y a continuité de l'impédance au travers de l'interface
            \begin{equation*}
            \hat \mZ^+(k_x,k_y) = \hat \mZ^-(k_x,k_y)
            \end{equation*}
        \end{lemme}
        \begin{proof}
            La démonstration est extrêmement rapide grâce à la continuité tangentielle des champs.
            \begin{align}
                \hat{\vE}_t(k_x,k_y,z_m^+) &= \hat{\vE}_t(k_x,k_y,z_m^-)
                \\
                \hat{\vH}_t(k_x,k_y,z_m^+) &= \hat{\vH}_t(k_x,k_y,z_m^-)
            \end{align}

            Donc on la même propriété sur \(\vn \pvect \hat{\vH}\)
            \begin{align}                
                \vn \pvect \hat{\vH}_t(k_x,k_y,z_m^+) &= \vn \pvect \hat{\vH}_t(k_x,k_y,z_m^-)
                \\
                \intertext{On repart de l'expression de l'impédance}
                \hat{\vE}_t(k_x,k_y,z_m^+) &= \hat{\mZ}^+(k_x,k_y) \left(\vect{e_z} \pvect \hat{\vH}_t(k_x,k_y,z_m^+)\right)
                \\
                \intertext{On utilise la continuité des champs}
                \hat{\vE}_t(k_x,k_y,z_m^-) &= \hat{\mZ}^+(k_x,k_y) \left(\vect{e_z} \pvect \hat{\vH}_t(k_x,k_y,z_m^-)\right)
                \\
                \intertext{Ce qui par définition de l'impédance vaut}
                \hat{\vE}_t(k_x,k_y,z_m^-) &= \hat{\mZ}^-(k_x,k_y) \left(\vect{e_z} \pvect \hat{\vH}_t(k_x,k_y,z_m^-)\right)
            \end{align}
        \end{proof}

        \begin{thm}
            Supposons que
            \begin{subequations}
                \label{eq:imp_fourier:plan:hyp_1_c}
                \begin{align}
                    k_3     & \not = 0 \\
                    k_3d    & \not = \frac{\pi}{2}+n\pi\,, \forall n \in \NN
                \end{align}
            \end{subequations}

            Alors
            \begin{align}
            \label{eq:imp_plan:symb_z:1c}
            \hat \mZ(k_x,k_y) &= i\eta\frac{\tan\left(k_3d\right)}{kk_3}
                \begin{bmatrix}
                   k^2-k_x^2  & -k_xk_y\\
                    -k_xk_y & k^2-k_y^2\\
                \end{bmatrix}
            \end{align}
        \end{thm}

        \begin{proof}
            Grâce au lemme \eqref{lem:plan:continuite_impedance}, on sait donc que l'on peut uniquement se placer dans la couche. Nous n'avons donc pas besoin de distinguer les vecteurs indépendant de \(z\) dans l'expression des champs tangentiels. Nous posons donc
            \begin{align}
                \vect{C_1}(k_x,k_y) = \begin{bmatrix}c_1(k_x,k_y) \\ c_3(k_x,k_y)\end{bmatrix} & & \vect{C_2}(k_x,k_y) = \begin{bmatrix}c_2(k_x,k_y) \\ c_4(k_x,k_y)\end{bmatrix}
            \end{align}

            Nous utilisons la condition limite du conducteur parfait
            \begin{align}
                \hat{\vE}_t(k_x,k_y,-d^+) &= 0
                \\
                &=  \mA_E(k_x,k_y,-d^+)\vect{C_1}(k_x,k_y) + \mB_E(k_x,k_y,-d^+)\vect{C_2}(k_x,k_y)
            \end{align}

            Si on suppose que ces matrices sont inversibles, donc que \(k_3\) est non-nul alors on a
            \begin{align}
                \vect{C_2}(k_x,k_y) &= -\left(\mB_E(k_x,k_y,-d^+)\right)^{-1}\mA_E(k_x,k_y,-d^+)\vect{C_1}(k_x,k_y)
            \end{align}

            On part de l'expression de l'impédance, exprimée à l'intérieur
            \begin{align}
                \hat \vE_t(k_x,k_y,0^-) &= \hat \mZ(k_x,k_y) \left(\vect{e_z} \pvect \hat \vH_t(k_x,k_y,0^-)\right)
            \end{align}
            Or on injecte le résultat précèdent
            \begin{multline}
                \hat \vE_t(k_x,k_y,0^-) = \\
                \left(\mA_E(k_x,k_y,0^-)-\left(\mB_E(k_x,k_y,-d^+)\right)^{-1}\mA_E(k_x,k_y,-d^+)\right)\vect{C_1}(k_x,k_y)
            \end{multline}
            \begin{multline}
                \vn \pvect \hat \vH_t(k_x,k_y,0^-) = \\
                 \left(\mA_H(k_x,k_y,0^-)-\left(\mB_E(k_x,k_y,-d^+)\right)^{-1}\mA_E(k_x,k_y,-d^+)\right)\vect{C_1}(k_x,k_y)
            \end{multline}
            On en déduit l'impédance
            \begin{multline}
                \hat \mZ(k_x,k_y) = \\ \left(\mA_E(k_x,k_y,0^-)-\mB_E(k_x,k_y,0^-)\left(\mB_E(k_x,k_y,-d^+)\right)^{-1}\mA_E(k_x,k_y,-d^+)\right)\\
                \left(\mA_H(k_x,k_y,0^-)-\mB_H(k_x,k_y,0^-)\left(\mB_H(k_x,k_y,-d^+)\right)^{-1}\mA_E(k_x,k_y,-d^+)\right)^{-1}
            \end{multline}
            Comme on a supposé les matrices inversibles
            \begin{multline}
                \hat \mZ(k_x,k_y) = \\ \left(\mA_E(k_x,k_y,0^-)\left(\mA_E(k_x,k_y,-d^+)\right)^{-1}-\mB_E(k_x,k_y,0^-)\left(\mB_E(k_x,k_y,-d^+)\right)^{-1}\right) \\
                \left(\mA_H(k_x,k_y,0^-)\left(\mA_E(k_x,k_y,-d^+)\right)^{-1}-\mB_H(k_x,k_y,0^-)\left(\mB_E(k_x,k_y,-d^+)\right)^{-1}\right)^{-1}
            \end{multline}

            On va alors exprimer chacun des termes grâce aux expressions des matrices
            \begin{align}
                \mA_E(k_x,k_y,0^-)\left(\mA_E(k_x,k_y,-d^+)\right)^{-1} &= e^{ik_3d}\mI
                \\
                \mB_E(k_x,k_y,0^-)\left(\mB_E(k_x,k_y,-d^+)\right)^{-1} &= e^{-ik_3d}\mI
                \\
                \mA_H(k_x,k_y,0^-)\left(\mA_E(k_x,k_y,-d^+)\right)^{-1} &= e^{ik_3d}\frac{1}{kk_3\eta}
                \begin{bmatrix}
                    k^2 - k_y^2 & k_x k_y
                    \\
                    k_x k _y & k^2 - k_x^2
                \end{bmatrix}
                \\
                \mB_H(k_x,k_y,0^-)\left(\mB_E(k_x,k_y,-d^+)\right)^{-1} &= -e^{-ik_3d}\frac{1}{kk_3\eta}
                    \begin{bmatrix}
                    k^2 - k_y^2 & k_x k_y
                    \\
                    k_x k _y & k^2 - k_x^2
                \end{bmatrix} 
            \end{align}

            Définissons alors les matrices \(\hat\mLD,\hat\mLR\) telle que
            \begin{align}
                \hat{\mLD} = -\begin{bmatrix}
                k_x^2 & k_x k_y
                \\
                k_x k _y & k_y^2
                \end{bmatrix}
                & & 
                \hat{\mLR} = \begin{bmatrix}
                k_y^2 & -k_x k_y
                \\
                -k_x k _y &  k_x^2
                \end{bmatrix}
            \end{align}

            On remarque alors que 
            \begin{align}
                \mA_H(k_x,k_y,0^-)\left(\mA_E(k_x,k_y,-d^+)\right)^{-1} &=  e^{ik_3d}\frac{1}{kk_3\eta}(k^2\mI  -\hat\mLR)
                \\
                \mB_H(k_x,k_y,0^-)\left(\mB_E(k_x,k_y,-d^+)\right)^{-1} &= -e^{-ik_3d}\frac{1}{kk_3\eta}(k^2\mI  -\hat\mLR)
            \end{align}

            Ces matrices sont inversibles si \(k_3\) est non nul, ce que nous avons déjà supposé.

            Un simple calcul matriciel permet de trouver que \( (k^2\mI - \hat{\mLR})^{-1} = \frac{1}{k^2k_3^2}\left(k^2 \mI + \hat{\mLD}\right) \).

            En supposant \(k_3d \not = \frac{\pi}{2} + n\pi\), on déduit donc que
            \begin{align}
                \hat \mZ(k_x,k_y) &= \frac{\eta}{kk_3} \frac{e^{ik_3d} - e^{-ik_3d}}{e^{ik_3d} + e^{-ik_3d}}\left(k^2\mI + \hat{\mLD}\right)
                \\
                &= i\eta\frac{\tan\left(k_3d\right)}{kk_3}\left(k^2\mI + \hat{\mLD}\right)
            \end{align}

        \end{proof}
        %On remarque que \(\det(\mat{Z}) = i\frac{\eta^2}{k_3}\eta\tan(k_3d)\) et donc pour un matériau \((\eps,\mu,d)\) donné, l'opérateur d'impédance n'est pas inversible pour tous  \((k_x,k_y) \in \RR^2, n \in \NN\), \(k_x^2+k_y^2 =  \w^2\eps\mu - \frac{1}{d^2}\left(\frac{\pi}{2} + n\pi\right)^2\), qui ne peut être vérifié que si \(\eps\mu\) est réel\footnote{Comme \(\eps, \mu\) sont à partie réelle (resp. imaginaire) strictement positive (resp. négative), alors ce n'est vrai pour les matériaux à partie imaginaire nulle.}.

        Dans l'article de \cite{marceaux_high-order_2000}, l'opérateur est exprimé comme
        \begin{equation}
            \vn \pvect \hat{\vE} = \hat{\mathfrak{Z}} \hat{\vH_t}
        \end{equation}

        Pour une couche de matériau, cet article énonce que 
        \begin{align}
            \hat{\mathfrak{Z}} &= -i\eta\frac{\tan(k_3d)}{kk_3}\left(k^2\mI - \hat{\mLR}\right)
        \end{align}

        Sachant que dans le cadre des vecteurs tangents \(\vn\pvect \vx = \mA \vy_t\) est équivalent à \(\vx_t = -\det(\mA)\left(\mA^{-1}\right)^t\left(\vn\pvect\vy\right)\), alors en développant les calculs, on retrouve bien le résultat de l'article de \cite{marceaux_high-order_2000}.\\

        En pratique, on néglige toute les dépendance en \(y\) ce qui revient à fixer \(k_y\) à \(0\). Grâce à cette hypothèse, on trouve que \(\hat \mZ\) est une matrice diagonale.

        Dans ce cas, pour revenir sur la polarisation, introduite dans la section précédente, le champ \(\vE\)-TE correspond à \({E_y} \vect{e_y}\), le champ \(\vE\)-TM à \({E_x} \vect{e_x} + {E_z} \vect{e_z} \), tandis que le champ \(\vH\)-TE correspond à \({H_x} \vect{e_x} + {H_z} \vect{e_z}\) et le champ \(\vH\)-TM correspond à \({H_y} \vect{e_y}\).
        
        Alors la matrice \(\hat \mZ\) peut se réécrire comme
        \begin{equation}
            \hat \mZ =
            \begin{bmatrix}
                \hat Z_{TM} & 0
                \\
                0 & \hat Z_{TE}
            \end{bmatrix}
        \end{equation}


        La figure \ref{fig:imp_fourier:plan:hoppe} permet de vérifier les résultats de \cite[p.~33]{hoppe_impedance_1995} pour une couche de matériau sans perte. Dans cet ouvrage, la matrice n’est pas exactement la même ( voir annexe \ref{sec:annexe:hoppe} ) et on pose \(\hat{\mathfrak{Z}} = \eta_0\hat{\mZ}\begin{bmatrix}0&1\\-1&0\end{bmatrix}\).

        On applique les hypothèses précédentes, donc \(k_y=0\). On remarque que pour \(k_x\slash k_0=2\), \(k_3\) est nul et l'impédance n'est pas calculée en ce point. 

        Comme le matériau n'a pas de perte, la partie réelle de \(\hat{\mathfrak{\mZ}}\) est nulle et n'est donc pas tracée.

        \begin{figure}[!hbt]
            \centering
            \tikzsetnextfilename{Z_HOPPE_33_plan}
\begin{tikzpicture}[scale=1]
    \begin{axis}[
            title={},
            ylabel={\(\Im(\hat{\mathfrak{Z}}(k_x,0)\) (\(\Omega\))},
            xlabel={\(k_x\slash k_0\)},
            width=0.8\textwidth,
            xmin=0,
            xmax=2,
            mark repeat=20,
            legend pos=outer north east
        ]
        \addplot [black] table [col sep=comma, x={s1}, y={Im(z_ex.11)}] {csv/HOPPE_33/HOPPE_33.z_ex.MODE_2_TYPE_P.csv};
        \addlegendentry{TM};

        \addplot [black,dashed] table [col sep=comma, x={s1}, y={Im(z_ex.22)}] {csv/HOPPE_33/HOPPE_33.z_ex.MODE_2_TYPE_P.csv};
        \addlegendentry{TE};
    \end{axis}
\end{tikzpicture}
            \caption[Reproduction résultat Hoppe & Rahmat-Samii p.~33]{Partie imaginaire des coefficients diagonaux de \(\hat{\mathfrak Z}\) pour \(\eps = 4, \mu = 1, d=0.015\text{m}, f=1\text{GHz}\)}
            \label{fig:imp_fourier:plan:hoppe}
        \end{figure}

        La figure \ref{fig:imp_fourier:plan:soudais} permet de vérifier les résultats de \cite{soudais_3d_2017} pour une couche de matériau sans perte où \(k_3d = \frac{\pi}{2}\) pour \(k_x \simeq 0.9 k_0\). On a donc une asymptote bien visible sur le module des coefficients diagonaux de \(\hat\mZ\) pour \(k_x\slash k_0 \simeq 0.94\).

        \begin{figure}[!hbt]
            \centering
            \tikzsetnextfilename{Z_SOUDAIS_plan_large}
\begin{tikzpicture}[scale=1]
    \begin{axis}[
            title={},
            width=0.4\textwidth,
            xmin=0,
            xmax=1.8,
            ylabel={\(\Im(\hat{Z}(k_x,0))\)},
            xlabel={\(k_x\slash k_0\)},
            mark repeat=20,
            legend pos=outer north east
        ]
        \addplot [black] table [x={s1}, y={Im(z_ex.11)},col sep=comma] {csv/SOUDAIS/SOUDAIS.z_ex.MODE_2_TYPE_P.csv};
        \addplot [black,dashed] table [x={s1}, y={Im(z_ex.22)},col sep=comma] {csv/SOUDAIS/SOUDAIS.z_ex.MODE_2_TYPE_P.csv};
    \end{axis}
\end{tikzpicture}
\tikzsetnextfilename{Z_SOUDAIS_plan_zoom}
\begin{tikzpicture}[scale=1]
    \begin{axis}[
            title={},
            width=0.4\textwidth,
            ymin=-100,
            ymax=100,
            xmin=0.8,
            xmax=1,
            restrict y to domain=-200:200,                        
            ylabel={},
            xlabel={\(k_x\slash k_0\)},
            mark repeat=20,
            legend pos=outer north east
        ]
        \addplot [black] table [x={s1}, y={Im(z_ex.11)},col sep=comma] {csv/SOUDAIS/SOUDAIS.z_ex.MODE_2_TYPE_P.csv};
        \addlegendentry{TM};
        \addplot [black,dashed] table [x={s1}, y={Im(z_ex.22)},col sep=comma] {csv/SOUDAIS/SOUDAIS.z_ex.MODE_2_TYPE_P.csv};
        \addlegendentry{TE};
    \end{axis}
\end{tikzpicture}
            \caption[Reproduction résultat P. Soudais p.~11]{Partie imaginaire des coefficients diagonaux de \(\hat\mZ\) pour \(\eps = 4, \mu = 1, d=0.035\text{m}, f=12\text{GHz}\)}
            \label{fig:imp_fourier:plan:soudais}
        \end{figure}


        Grâce à la figure \ref{fig:reflex_fourier:plan:soudais}, on remarque que la matrice de réflexion est parfaitement définie en ce point. De plus, cet empilement permet d'obtenir une onde guidée pour \(k_x\slash k_0 \simeq 1.4\) car le coefficient TM diverge en ce point. C'est donc un cas très intéressant.

        \begin{figure}[!hbt]
            \centering
            \tikzsetnextfilename{R_SOUDAIS_plan}
\begin{tikzpicture}[scale=1]
    \begin{axis}[
            title={},
            width=0.8\textwidth,
            xmin=0,
            xmax=1.8,
            ymin=0,
            ymax=4,
            restrict y to domain=0:10,
            ylabel={\(|\hat{R}(k_x,0)|\)},
            xlabel={\(k_x\slash k_0\)},
            mark repeat=20,
            legend pos=outer north east
        ]
        \addplot [black] table [x={s1}, y={Abs(r_ex.11)},col sep=comma] {csv/SOUDAIS/SOUDAIS.r_ex.MODE_2_TYPE_P.csv};
        \addlegendentry{TM};
        \addplot [black,dashed] table [x={s1}, y={Abs(r_ex.22)},col sep=comma] {csv/SOUDAIS/SOUDAIS.r_ex.MODE_2_TYPE_P.csv};
        \addlegendentry{TE};
    \end{axis}
\end{tikzpicture}
            \caption[Reproduction résultat P. Soudais p.~11]{Module des coefficients diagonaux de \(\mR\) pour \(\eps = 4, \mu = 1, d=0.035\text{m}, f=12\text{GHz}\)}
            \label{fig:reflex_fourier:plan:soudais}
        \end{figure}


    \subsection{Expression de la matrice de réflexion pour une couche de matériau}

        Dans le cas de la diffraction par un plan infini, une grandeur d’intérêt est le rapport entre l'onde réfléchie et l'onde incidente.
         
        \begin{defn}
          On définit la matrice \(\hat\mR\) dépendante de \((k_x,k_y)\) telle que pour tout point de la surface \(z=0_+\)
          \begin{equation}
            \vect{C_2}(k_x,k_y)  = \hat\mR(k_x,k_y) \vect{C_1}(k_x,k_y)
          \end{equation}
        \end{defn}

        On peut toujours définir cette matrice dans chaque couche, mais ce n'est alors plus la matrice de réflexion. Mais par abus de langage, on l'y nommera quand même matrice de réflexion.
        Cette matrice \(\hat\mR\) est propre à chaque couche là où la matrice d'impédance \(\hat\mZ\) est propre à chaque interface.

        \begin{lemme}[Matrice de réflexion associée à une impédance]
            \label{lem:plan:reflexion_from_impedance}
            Soit \(z_m\) une interface entre deux matériaux, et on se place juste au dessus (\(z>z_m\)). 

            Soit \(\hat{\mZ}(k_x,k_y)\) la matrice d'impédance à cette interface.
            \begin{align*}
                \hat{\vE}_t(k_x,k_y,z_m^+) &= \hat{\mZ}(k_x,k_y) \left(\vect{e_z} \pvect \hat{\vH}_t(k_x,k_y,z_m^+)\right)
            \end{align*}
            Alors la matrice de réflexion est fonction de la matrice d'impédance et
            \begin{multline*}
                    \hat\mR(k_x,k_y) = - \left(\mB_E(k_x,k_y,z_m^+) - \hat{\mZ}(k_x,k_y) \mB_H(k_x,k_y,z_m^+) \right)^{-1}\\
                    \left(\mA_E(k_x,k_y,z_m^+) - \hat{\mZ}(k_x,k_y) \mA_H(k_x,k_y,z_m^+) \right)
            \end{multline*}
        \end{lemme}
        \begin{proof}
            Il suffit de partir de la définition des deux matrices pour conclure.
        \end{proof}

        Cependant, il est aussi possible de déterminer la matrice de réflexion directement sans passer par l'impédance. Pour cela, nous avons besoin du lemme suivant

        \begin{lemme}[Discontinuité des matrices de réflexions]
            \label{lem:plan:discontinuite_reflexion}
            Soit \(z_m\) une interface entre deux matériaux \(+\) (\(z>z_m\)) et \(-\) (\(z<z_m\)). On considère les matrices de réflexions de part et d'autre de l'interface
            \begin{align*}
                \vect{C_2}^+(k_x,k_y)  = \hat\mR^+(k_x,k_y) \vect{C_1}^+(k_x,k_y)
                \\
                \vect{C_2}^-(k_x,k_y)  = \hat\mR^-(k_x,k_y) \vect{C_1}^-(k_x,k_y)
            \end{align*}
            Allégeons les notations: omettons les \((k_x,k_y)\) et indiquons par \(\pm\) lorsque l'on se trouve en \(z_m^\pm\).
            Alors il y a discontinuité de la matrice de réflexion au travers de l'interface et
            \begin{multline*}
            \hat\mR^+ = - \left((\mA_E^- + \mB_E^-\hat\mR^-)^{-1}\mB_E^+ - (\mA_H^- + \mB_H^-\hat\mR^-)^{-1}\mB_H^+\right)^{-1}
            \\
            \left((\mA_E^- + \mB_E^-\hat\mR^-)^{-1}\mA_E^+ - (\mA_H^- + \mB_H^-\hat\mR^-)^{-1}\mA_H^+\right)
            \end{multline*}
        \end{lemme}
        \begin{proof}
            Il suffit d'utiliser le lemme précèdent de part et d'autre de l'interface conjointement avec le lemme \ref{lem:plan:continuite_impedance}. La conclusion est immédiate.
        \end{proof}

        On peut exprimer alors la matrice de réflexion dans le vide \(\hat\mR\) en utilisant la condition limite de conducteur parfait.

    %%%%%%%%%%%%%%%%%%%%%%%%%%%%%%%%%%%%%%%%%%%%%%%%%%%%%%%%%%%%%%%%%%%%%%%
    %%%%%%%%%%%%%%%%%%%%%%%%%%%%%%%%%%%%%%%%%%%%%%%%%%%%%%%%%%%%%%%%%%%%%%%
    %%%%%%%%%%%%%%%%%%%%%%%%%%%%%%%%%%%%%%%%%%%%%%%%%%%%%%%%%%%%%%%%%%%%%%%

    \subsection{Expression de la matrice d'impédance pour plusieurs couches de matériaux}
        On suppose que l'on a \(n\) couches de matériaux :

        \begin{figure}[h!btp]
            \centering
            \tikzsetnextfilename{plan_n_couches}            
            \begin{tikzpicture}
                \tikzmath{
    \largeur = 6;
    \hauteur = 0.5;
    \milieu = 1.3;
    \xC = \largeur;
    \xA = 0;
}

%% 1ere couche
\tikzmath{
    \yC = \hauteur;
    \yA = 0;
}

\coordinate (A) at (\xA,\yA);
\coordinate (B) at (\xA,\yC);
\coordinate (C) at (\xC,\yC);

\draw ($(B)!0.5!(C)$) node [above] {vide};


\fill [lightgray] (A) rectangle (C);
\draw ($(A)!0.5!(C)$) node {$\eps_n,\mu_n,d_n$};
\draw (B) -- (C) node [right] {$e_3 = 0$};

%% Des couches
\tikzmath{
    \yC = \yC - \hauteur;
    \yA = \yA - \milieu*\hauteur;
}

\coordinate (A) at (\xA,\yA);
\coordinate (B) at (\xA,\yC);
\coordinate (C) at (\xC,\yC);

\fill [lightgray]    (A) rectangle (C);
\fill [pattern=dots] (A) rectangle (C);
\draw (B) -- (C);

%% N ieme couche
\tikzmath{
    \yC = \yC - \milieu*\hauteur;
    \yA = \yA - \hauteur;
}

\coordinate (A) at (\xA,\yA);
\coordinate (B) at (\xA,\yC);
\coordinate (C) at (\xC,\yC);
\fill [lightgray] (A) rectangle (C);
\draw ($(A)!0.5!(C)$) node {$\eps_1,\mu_1,d_1$};
\draw (B) -- (C);

%% Le repère
\tikzmath{
    \xD = \xC + 0.5;
}

\coordinate (n) at (\xD,\yA);
\draw [->] (n) -- ++(1,0) node [at end, right] {$\v{e_1}$};
\draw [->] (n) -- ++(0,1) node [at end, right] {$\v{e_3}$};

\draw (n) circle(0.1cm) node [below=0.1cm] {$\v{e_2}$};
\draw (n) +(135:0.1cm) -- +(315:0.1cm);
\draw (n) +(45:0.1cm) -- +(225:0.1cm);

%% Le conducteur
\tikzmath{
    \yC = \yC - \hauteur;
    \yA = \yA - 0.5*\hauteur;
}

\coordinate (A) at (\xA,\yA);
\coordinate (B) at (\xA,\yC);
\coordinate (C) at (\xC,\yC);
\draw (B) -- (C);

\fill [pattern=north east lines] (A) rectangle (C);



            \end{tikzpicture}
        \end{figure}

        Pour chaque couche caractérisée par \((k_m,\eta_m,d_m)\), on définit
        \begin{align}
        k_{3m} &= \sqrt{k_m^2 - k_y^2 - k_x^2}
        \\
        z_m &= -\sum_{i=0}^{n-m} d_{n-i}
        \end{align}
        Afin de ne pas alourdir la thèse, on ne redéfinit pas les matrices \(\mA_E,\mB_E,\mA_H,\mB_H\) mais il est très important de considérer que pour \(z_{m-1}<z<z_m\), ces matrices font intervenir les constantes de matériaux \(k_m,\eta_m\) et sont nommées \(\mA_{Em},\mB_{Em},\mA_{Hm},\mB_{Hm}\). De même pour les vecteurs \(\vect{C_{1m}},\vect{C_{2m}}\).

        \begin{defn}
            On définit pour chaque interface, la matrice d'impédance \(\hat{\mZ}_m\) telle que
            \begin{equation*}
                \hat \vE_t(k_x,k_y,z_m) = \hat \mZ_m(k_x,k_y) \left(\vect{e_z} \pvect \hat \vH_t(k_x,k_y,z_m)\right)
            \end{equation*}
            Et on définit pour chaque couche \(z_{m-1}<z<z_m\), la matrice de réflexion \(\hat{\mR}_m\) telle que
            \begin{align*}
                \vect{C_{2m}}(k_x,k_y) &= \hat{\mR}_m \vect{C_{1m}}(k_x,k_y)
                \intertext{et les matrices \(\mM_A, \mM_B\)}
                \\
                \mM_{Am}(k_x,k_y,z_{m-1}^+) &= \mA_{Em}(k_x,k_y,z_{m-1}^+)-\hat{\mZ}_{m-1}\mA_{Hm}(k_x,k_y,z_{m-1}^+)
                \\
                \mM_{Bm}(k_x,k_y,z_{m-1}^+) &= \mB_{Em}(k_x,k_y,z_{m-1}^+)-\hat{\mZ}_{m-1}\mB_{Hm}(k_x,k_y,z_{m-1}^+)
            \end{align*}
        \end{defn}

        \begin{thm}
            \label{thm:imp:fourier:plan:multi_couche}
            Soit \(\hat \mZ_0(k_x,k_y) = \mat{0}_{\mathcal{M}_2(\CC)}\).

            Si pour tout \(0 < m \le n\)
            \begin{align}
                k_{3m} &\not = 0
                \\
                \det\left(\mM_{Am}(k_x,k_y,z_{m-1}^+)\right) &\not = 0 
                \\
                \det\left(\mM_{Bm}(k_x,k_y,z_{m-1}^+)\right) &\not = 0
            \end{align}
            Alors la matrice \(\hat \mZ_n\) est défini par la relation de récurrence :
            \begin{multline*}
                \hat \mZ_m = \\
                \left(\mA_{Em}(z_{m}^-)\left(\mM_{Am}(k_x,k_y,z_{m-1}^+)\right)^{-1} - \mB_{Em}(z_{m}^-) \left(\mM_{Bm}(k_x,k_y,z_{m-1}^+)\right)^{-1}\right)
                \\
                \left(\mA_{Hm}(z_{m}^-)\left(\mM_{Am}(k_x,k_y,z_{m-1}^+)\right)^{-1} - \mB_{Hm}(z_{m}^-) \left(\mM_{Bm}(k_x,k_y,z_{m-1}^+)\right)^{-1}\right)^{-1}
            \end{multline*}
        \end{thm}

        \begin{proof}
            Nous avons déjà démontré l'initialisation de cette récurrence précédemment: la condition limite sur le conducteur impose \(\hat \mZ_0 = \mat{0}_{\mathcal{M}_2(\CC)}\) et on retrouve le résultat pour une couche.

            Pour démontrer la récurrence, on se place dans la couche \(z_{m-1}<z<z_m\). On suppose que l'on ait une matrice d'impédance sur l'interface \(z= z_{m-1}^+\). Grâce au lemme \ref{lem:plan:reflexion_from_impedance}, on peut en déduire la matrice de réflexion dans la couche. Il est alors aisé de déduire l'impédance sur l'interface \(z=z_m^-\).
        \end{proof}
