\section[Opérateur de Calderón pour un plan]{Expressions exactes des matrices d'impédance et des matrices de réflexion pour un plan infini}
    % Ce cas est très bien documenté (\cite{senior_approximate_x995},\cite{hoppe_impedance_x995}) et pose la méthodologie à adopter pour les objets courbes.

    Dans un premier temps, on peut sans perte de généralités faire une rotation du repère pour avoir le plan orthogonal à \(\vect{z}\).
    \begin{figure}[!h]
        \begin{center}
            \tikzsetnextfilename{plan_1_couche}
            \begin{tikzpicture}
                \tikzmath{
    \largeur = 6;
    \hauteur = 1;
    \milieu = 1.3;
    \xC = \largeur;
    \xA = 0;
}

%% 1ere couche
\tikzmath{
    \yC = \hauteur;
    \yA = 0;
}

\coordinate (A) at (\xA,\yA);
\coordinate (B) at (\xA,\yC);
\coordinate (C) at (\xC,\yC);

\draw ($(B)!0.5!(C)$) node [above] {vide};


\fill [lightgray] (A) rectangle (C);
\draw ($(A)!0.5!(C)$) node {$\peps,\pmu,d$};
\draw (B) -- (C) node [right] {$\z = 0$};

%% Le repère
\tikzmath{
    \xD = \xC + 1.5;
}

\coordinate (n) at (\xD,\yA);

\draw [->] (n) -- ++(0,1) node [at end, right] {$\v{\z}$};
\draw [->] (n) -- ++(1,0) node [at end, right] {$\v{\x}$};

\draw (n) circle(0.1cm) node [below=0.1cm] {$\v{\y}$};
\draw (n) +(135:0.1cm) -- +(315:0.1cm);
\draw (n) +(45:0.1cm) -- +(225:0.1cm);

%% Le conducteur
\tikzmath{
    \yC = \yC - \hauteur;
    \yA = \yA - 0.5*\hauteur;
}

\coordinate (A) at (\xA,\yA);
\coordinate (B) at (\xA,\yC);
\coordinate (C) at (\xC,\yC);
\draw (B) -- (C);

\fill [pattern=north east lines] (A) rectangle (C);



            \end{tikzpicture}
        \end{center}
    \end{figure}
    % Comme il est infini dans les directions \(\vect{e_x} ,\vect{e_y}\) et que le matériau est homogène isotrope, 
    Nous utilisons une transformée partielle en \(x, y\).
    \begin{equation*}
        \vE(x,y,z) = \frac{1}{2\pi}\iint_{\RR^2} e^{i(k_x x + k_y y)}\hat{\vE} (k_x,k_y,z) \dd{k_x}\dd{k_y.}
    \end{equation*}
    Grâce à cela, nous simplifions le problème en étudiant \( \hat{\vE}\) grâce aux multiplicateurs de Fourier associés aux variables \(x,y\). 

    Soit \(k\) la fonction constante par morceaux telle que
    \begin{equation*}
    \fonction{k}{[-d,0[\cup[0,\infty[}{\CC}
          {z}{k(z)=
          \begin{cases} 
            w^2\sqrt{\eps\mu} & -d \le z < 0,
            \\
            w^2\sqrt{\eps_0\mu_0} & 0 \le z < \infty.
          \end{cases}}
    \end{equation*}

    \begin{defn}
        Soit \((k_x,k_y) \in \RR^2\) fixé.
        Soit \(k_3\) la fonction constante par morceaux en \(z\) telle que
        \begin{equation*}
          \fonction{k_3}{[-d,0[\cup[0,\infty[}{\CC}
          {z}{
            k_3(z); k_3(z)^2 =
            \begin{cases}
            k^2 - k_x^2 - k_y^2, \Im(k_3(z)) \ge 0 & -d \le z < 0
            \\
            k_0^2 - k_x^2 - k_y^2, \Im(k_3(z)) < 0 & 0 \le z
            \end{cases}
          }
        \end{equation*}
        Le choix du signe de la partie imaginaire se déduit de la décroissance à l'infini des champs diffractés définis ci-dessous.
        % \begin{REM}
        %   Quel choix de racines fais-tu ?
        % \end{REM}
        % \begin{REP}
        %   Par convention d'écriture, partie imaginaire positive
        %   A l’extérieur partie imaginaire négative.
        %   Car dans le matériau l'onde est de la forme \(A e^{ik_3z} + B e^{-ik_3z}\) donc peu importe tandis que la seule onde acceptable à l’extérieur est celle en \(e^{-ik_3 z}\) car si \(k_3\) à une partie imaginaire, alors cela donne une décroissance à l'infini.
        % \end{REP}
    \end{defn}

    \begin{prop}
        On se place dans le matériau et l’on suppose tel que \(\forall z \in [-d,0], k_3(z) \not = 0\).
        Alors existe \((c_i(k_x,k_y))_{1\le i \le4}, \in (\CC(\RR^2))^4\) tels que les champs solutions de \ref{eq:imp_fourier:intro:maxwell_harmonique:form_cea} sont de la forme
        \begin{align*}
            \hat{E}_y(k_x,k_y,z) & = e^{ik_3(z)z}c_1(k_x,k_y) + e^{-ik_3(z)z}c_2(k_x,k_y),
            \\
            \hat{\cH}_y(k_x,k_y,z) & = e^{ik_3(z)z}c_3(k_x,k_y) + e^{-ik_3(z)z}c_4(k_x,k_y).
        \end{align*}
        Les constantes \(c_i\) ne peuvent en réalité pas être choisies arbitrairement car les exponentielles complexes divergent quand \(k_x^2 + k_y^2 \rightarrow \infty\) et ces solutions n'ont pas de transformées de Fourier. Nous supposons que les constantes \(c_i\) satisfassent la propriété \eqref{prop:unicite:interieur:postulat:multi-couche}, où il n'y a pas de résonances.
        % \begin{REM}
        %   Faux: Il faut que je te réexplique par Skype. Ce n'est pas la condition de résonance mais \(z \in [z_0,z_2]\) par exemple + une estimations sur les coefficients.
        % \end{REM}
        Nous supposons que les constantes \(c_i\) satisfassent la propriété \eqref{prop:unicite:interieur:postulat:multi-couche}.

    \end{prop}

    \begin{proof}
      On commence par coupler les équations de Maxwell pour obtenir des équations sur chaque champ seul
      \begin{align*}
          \vlapl{\vE} + k(z)^2 \vE & = 0,
          \\
          \vlapl{\vH} + k(z)^2 \vH & = 0.
      \end{align*}
      Donc chaque composante de chaque champ est solution d'une équation de Helmholtz. On s’intéresse alors à la composante en \(y\) seulement:
      \begin{align*}
          \lapl{E_y} + k(z)^2 E_y & = 0,
          \\
          \lapl{\cH_y} + k(z)^2 \cH_y & = 0.
      \end{align*}
      On applique les transformées partielles de Fourier en \((x,y)\) aux \glspl{acr-edp}, on obtient
      \begin{align*}
          \ddr[2]{z}{\hat{E_y}}(k_x,k_y,z) + \left( k(z)^2 - k_x^2 - k_y^2\right) \hat{E}_y(k_x,k_y,z) & = 0,
          \\
          \ddr[2]{z}{\hat{\cH}_y}(k_x,k_y,z) + \left( k(z)^2 - k_x^2 - k_y^2\right) \hat{\cH}_y(k_x,k_y,z) & = 0.
      \end{align*}

      On suppose alors que \(k_3(z)^2\) ne s'annule pas dans la couche. S'il s'annule, les solutions sont linéaires en \(z\) et nous renvoyons à l'annexe \ref{sec:annexe:plan:k3_nul}.

      Une solution générale s'écrit alors
      \begin{align*}
          \hat{E}_y(k_x,k_y,z) & = e^{ik_3(z)z}c_1(k_x,k_y) + e^{-ik_3(z)z}c_2(k_x,k_y),
          \\
          \hat{\cH}_y(k_x,k_y,z) & = e^{ik_3(z)z}c_3(k_x,k_y) + e^{-ik_3(z)z}c_4(k_x,k_y).
      \end{align*}
    \end{proof}

    Par abus de notation, on identifie  \(\eps\equiv \eps(-d^+)=\eps(0^-)\), \(\mu\equiv\mu(-d^+)=\mu(0^-)\), \(k\equiv k(-d^+)=k(0^-)\), \(k_3 \equiv k_3(-d^+) = k_3(0^-)\).

    % Ce choix est motivé par l'invariance par rotation du plan infini. Or il est d'usage de la représenter comme sur le schéma, avec un propagation de l'onde dans le plan \(xOz\). Souvent, on parle alors de polarisation pour simplifier les calculs. En effet, une polarisation \gls{acr-te} consiste à dire que \(\vE\) n'a qu'une composante dans le plan transverse au plan de propagation, donc suivant \(\vect{e_y}\). De même une polarisation \gls{acr-tm} consiste à dire que \(\vH\) n'a qu'une composante suivant \(\vect{e_y}\). C'est ce qui motive ce choix absolument abstrait d'utiliser les deuxièmes composantes.

    \subsection{Expressions des champs tangentiels dans chaque couche}

      \begin{defn}
        On définit les matrices \(\mA_{E}(k_x,k_y,z)\), \(\mB_{E}(k_x,k_y,z)\), \(\mA_{H}(k_x,k_y,z)\), \(\mB_{H}(k_x,k_y,z)\) constantes par morceaux en \(z\)
        \begin{align*}
          \mA_{E}(k_x,k_y,z) &= e^{ik_3(z)z}
          \begin{bmatrix}
            -\frac{k_xk_y}{k(z)^2 - k_y^2} & -\frac{k(z)\eta_r(z)k_3(z)}{k(z)^2 - k_y^2}
            \\
            1 & 0
          \end{bmatrix},
          \\
          \mB_{E}(k_x,k_y,z) &= e^{-ik_3(z)z}
          \begin{bmatrix}
            -\frac{k_xk_y}{k(z)^2 - k_y^2} & \frac{k(z)\eta_r(z)(z) k_3(z)}{k(z)^2 - k_y^2}
            \\
            1 & 0
          \end{bmatrix},
          \\
          \mA_{H}(k_x,k_y,z) &= e^{ik_3(z)z}
          \begin{bmatrix}
            0 & -1
            \\
            \frac{k(z)k_3(z)}{\eta_r(z)(k(z)^2 - k_y^2)} & -\frac{k_xk_y}{k(z)^2 - k_y^2}
          \end{bmatrix},
          \\
          \mB_{H}(k_x,k_y,z) &= e^{-ik_3(z)z}
          \begin{bmatrix}
            0 & -1
            \\
            -\frac{k(z)k_3(z)}{\eta_r(z)(k(z)^2 - k_y^2)} & -\frac{k_xk_y}{k(z)^2 - k_y^2}
          \end{bmatrix}.
        \end{align*}
      \end{defn}
      On remarque que \(\eps,\mu\) sont constantes par morceaux, donc \(k,\eta_r\) aussi.
      Sur une interface d'équation \(z=z_p\) entre deux matériaux, il y a donc un saut de valeurs pour ces matrices: \(\lim_{\delta\rightarrow 0 } \mA_E(k_x,k_y,z_p+ \delta) - \mA_E(k_x,k_y,z_p - \delta) \not = 0\).
      \begin{prop}
          Il existe \((c_i(k_x,k_y,z))_{1\le i \le4}\) constants par morceaux en \(z\) tels que les composantes en \(\vect{e_x},\vect{e_y}\) des champs, dites composantes tangentielles par abus de langage, sont
          \begin{align*}
              \hat{\vE}_t(k_x,k_y,z) &= \mA_E(k_x,k_y,z) \begin{bmatrix}c_1(k_x,k_y,z) \\ c_3(k_x,k_y,z)\end{bmatrix} + \mB_E(k_x,k_y,z) \begin{bmatrix}c_2(k_x,k_y,z) \\ c_4(k_x,k_y,z)\end{bmatrix},
              \\
              (\vn \pvect \hat{\vH})_t(k_x,k_y,z) &= \mA_H(k_x,k_y,z) \begin{bmatrix}c_1(k_x,k_y,z) \\ c_3(k_x,k_y,z)\end{bmatrix} + \mB_H(k_x,k_y,z) \begin{bmatrix}c_2(k_x,k_y,z) \\ c_4(k_x,k_y,z)\end{bmatrix}.
          \end{align*}
      \end{prop}

      \begin{proof}
        On se place dans une couche, les constantes de matériaux sont donc fixées.
        On reprend les équations de Maxwell appliquées sur les transformées de Fourier. À partir des composantes \(\vect{e_x},\vect{e_y}\) des équations de Maxwell, on  détermine un système d'\glspl{acr-edo} sur \(\hat{E_x},\hat{E_z},\hat{\mathcal{H}_x},\hat{\mathcal{H}_z}\) à partir de \(\hat{E_z}, \hat{\mathcal{H}_z}\).
        % \begin{REM}
        %   Je te suggère d'ajouter comme on en a discuté le fait qu'il y a y un système d'EDO sur \(E_y,E_x,H_y,H_x\) en remplaçant \(E_z,H_z\).
        %   Dans ta présentation les calculs sont corrects presque tout le temps.
        % \end{REM}
        \begin{align*}
            \left\lbrace
            \begin{matrix}
                ik_y \hat{E_z}  - \ddr{z}{\hat{E_y}} = -i k \eta_r  \hat{\mathcal{H}_x},
                \\
                ik_x \hat{E_y} - ik_y \hat{E_x} = -i k \eta_r  \hat{\mathcal{H}_z},
            \end{matrix}
            \right. 
            &&
            \left\lbrace
            \begin{matrix}
                ik_y \hat{\mathcal{H}_z}  - \ddr{z}{\hat{\mathcal{H}_y}} = i k \eta_r^{-1} \hat{E_x},
                \\
                ik_x \hat{\mathcal{H}_y} - ik_y \hat{\mathcal{H}_x} = i k \eta_r^{-1} \hat{E_z}.
            \end{matrix}
            \right.
        \end{align*}
        Cela revient à résoudre \(Y = \mat{M}X\) où la matrice \(\mat{M}\) et les vecteurs \(X, Y\) sont définis tels que
        \begin{align*}
          \mat{M} =
          \begin{bmatrix}
          0 & -ik_y & -ik \eta_r & 0
          \\
          ik_y & 0 & 0 & -ik \eta_r
          \\
          ik \eta_r^{-1} & 0 & 0 & -ik_y
          \\
          0 & ik \eta_r^{-1} & ik_y & 0
          \end{bmatrix},
          &&
          X =
          \begin{bmatrix}
            \hat{E_x}\\
            \hat{E_z}\\
            \hat{\mathcal{H}_x}\\
            \hat{\mathcal{H}_z}
          \end{bmatrix},
          &&
          Y =
          \begin{bmatrix}
            -\ddr{z}{\hat{E_y}}\\
            ik_x\hat{E_y}\\
            -\ddr{z}{\hat{\mathcal{H}_y}}\\
            ik_x\hat{\mathcal{H}_y}
          \end{bmatrix}.
        \end{align*}
        Cette matrice est inversible si \(\det(\mM) = (k^2 - k_y^2)^2 \) est non nul.
        On suppose ce dernier non nul, on peut en déduire \(X\)
        \begin{equation*}
          \begin{bmatrix}
            \hat{E_x}\\
            \hat{E_z}\\
            \hat{\mathcal{H}_x}\\
            \hat{\mathcal{H}_z}
          \end{bmatrix} =
          \frac{1}{k^2 - k_y^2}
          \begin{bmatrix}
          0 & ik_y & -ik \eta_r & 0
          \\
          -ik_y & 0 & 0 & -ik \eta_r
          \\
          ik \eta_r^{-1} & 0 & 0 & ik_y
          \\
          0 & ik \eta_r^{-1} & -ik_y & 0
          \end{bmatrix}
          \begin{bmatrix}
            -\ddr{z}{\hat{E_y}}\\
            ik_x\hat{E_y}\\
            -\ddr{z}{\hat{\mathcal{H}_y}}\\
            ik_x\hat{\mathcal{H}_y}
          \end{bmatrix}.
        \end{equation*}
        On extrait alors uniquement les composantes tangentielles
        \begin{align*}
          \left\lbrace
          \begin{aligned}
            \hat{E_x} &= \frac{1}{k^2 - k_y^2}\left(ik_yik_x\hat{E_y} + ik\eta_r\ddr{z}{\hat{\mathcal{H}_y}}\right),
            \\
            \hat{E_y} &= e^{ik_3z}c_1 + e^{-ik_3z}c_2,
          \end{aligned}
          \right.
          &&
          \left\lbrace
          \begin{aligned}
            -\hat{\cH_y} &= -e^{ik_3z}c_3 - e^{-ik_3z}c_4,
            \\
            \hat{\cH_x} &= \frac{1}{k^2 - k_y^2}\left(-ik\eta_r^{-1}\ddr{z}{\hat{E_y}} + ik_yik_x\hat{\mathcal{H}_y}\right).
          \end{aligned}
          \right.
        \end{align*}
        Soit si l'on factorise par les exponentielles
        \begin{align*}
          &\left\lbrace
          \begin{aligned}
            \hat{E_x} &= \frac{1}{k^2 - k_y^2}\left(-\left(k_yk_x c_1 + k\eta_r k_3 c_3 \right)e^{ik_3z} - \left(k_yk_xc_2 -k\eta_r k_3 c_4\right)e^{-ik_3z}\right),
            \\
            \hat{E_y} &= e^{ik_3z}c_1 + e^{-ik_3z}c_2,
          \end{aligned}
          \right.
          \\
          &\left\lbrace
          \begin{aligned}
            -\hat{\mathcal{H}_y} &= -e^{ik_3z}c_3 - e^{-ik_3z}c_4,
            \\
            \hat{\mathcal{H}_x} &= \frac{1}{k^2 - k_y^2}\left(\left(k\eta_r^{-1} k_3c_1 - k_y k_x c_3\right)e^{ik_3z} + \left(-k\eta_r^{-1} k_3c_2 - k_y k_x c_4\right)e^{-ik_3z}\right).
          \end{aligned}
          \right.
        \end{align*}
        On obtient alors
        \begin{align*}
            \hat{\vE}_t(k_x,k_y,z) &= \mA_E(k_x,k_y,z) \begin{bmatrix}c_1(k_x,k_y) \\ c_3(k_x,k_y)\end{bmatrix} + \mB_E(k_x,k_y,z) \begin{bmatrix}c_2(k_x,k_y) \\ c_4(k_x,k_y)\end{bmatrix},
            \\
            (\vn \pvect \hat{\vH})_t(k_x,k_y,z) &= \mA_H(k_x,k_y,z) \begin{bmatrix}c_1(k_x,k_y) \\ c_3(k_x,k_y)\end{bmatrix} + \mB_H(k_x,k_y,z) \begin{bmatrix}c_2(k_x,k_y) \\ c_4(k_x,k_y)\end{bmatrix}.
        \end{align*}
      \end{proof}

        % Ce résultat est encore valable dans le vide mais les \(c_i\) y sont différents et il faut replacer \(k\) par \(k_0\) et \(\eta\) par \(1\).
        % Cette décomposition est intéressante car elle contient des termes de propagation vers le plan, ce sont les matrices \(\mA_E, \mA_H\) et des termes de propagation vers l'infini, ce sont les matrices \(\mB_E, \mB_H\). En effet, le terme \(e^{i\w t}\) omis dans les expression est multiplié par \(e^{\pm i k_3 z}\). 

        % Supposons par exemple que \(k_3 = \sqrt{k_0^2 - k_x^2 - k_y^2}\) soit réel, donc positif. Si l'on veut que par exemple \(\w t + k_3 z\) soit constant quand \(t\) augmente, il faut que \(z\) diminue, autrement dit, quand le temps avance, l'onde se rapproche du plan. C'est la partie onde incidente et de fait l'autre partie est l'onde réfléchie.

        % Supposons maintenant que \(k_3 = i\sqrt{k_0^2 - k_x^2 - k_y^2}\)  soit imaginaire. Alors si l'on veut des champs réfléchis finis à l'infini, il faut que \(\Im(k_3) = \Im(\sqrt{k_0^2 - k_x^2 - k_y^2})\) soit négatif, ainsi on a une évolution en \(e^{\Im(k_3)z}\), qui tend bien vers \(0\).

    \subsection{Expression de la matrice d'impédance pour une couche de matériau}

    %%%%%%%%%%%%%%%%%%%%%%%%%%%%%%%%%%%%%%%%%%%%%%%%%%%%%%%%%%%%%%%%%%%%%%%
    %%%%%%%%%%%%%%%%%%%%%%%%%%%%%%%%%%%%%%%%%%%%%%%%%%%%%%%%%%%%%%%%%%%%%%%
    %%%%%%%%%%%%%%%%%%%%%%%%%%%%%%%%%%%%%%%%%%%%%%%%%%%%%%%%%%%%%%%%%%%%%%%

        % \begin{figure}[!h]
        %     \begin{center}
        %         \tikzsetnextfilename{plan_1_couche}
        %         \begin{tikzpicture}
        %             \tikzmath{
    \largeur = 6;
    \hauteur = 1;
    \milieu = 1.3;
    \xC = \largeur;
    \xA = 0;
}

%% 1ere couche
\tikzmath{
    \yC = \hauteur;
    \yA = 0;
}

\coordinate (A) at (\xA,\yA);
\coordinate (B) at (\xA,\yC);
\coordinate (C) at (\xC,\yC);

\draw ($(B)!0.5!(C)$) node [above] {vide};


\fill [lightgray] (A) rectangle (C);
\draw ($(A)!0.5!(C)$) node {$\peps,\pmu,d$};
\draw (B) -- (C) node [right] {$\z = 0$};

%% Le repère
\tikzmath{
    \xD = \xC + 1.5;
}

\coordinate (n) at (\xD,\yA);

\draw [->] (n) -- ++(0,1) node [at end, right] {$\v{\z}$};
\draw [->] (n) -- ++(1,0) node [at end, right] {$\v{\x}$};

\draw (n) circle(0.1cm) node [below=0.1cm] {$\v{\y}$};
\draw (n) +(135:0.1cm) -- +(315:0.1cm);
\draw (n) +(45:0.1cm) -- +(225:0.1cm);

%% Le conducteur
\tikzmath{
    \yC = \yC - \hauteur;
    \yA = \yA - 0.5*\hauteur;
}

\coordinate (A) at (\xA,\yA);
\coordinate (B) at (\xA,\yC);
\coordinate (C) at (\xC,\yC);
\draw (B) -- (C);

\fill [pattern=north east lines] (A) rectangle (C);



        %         \end{tikzpicture}
        %     \end{center}
        % \end{figure}

        On commence par montrer un lemme très utile pour trouver l'expression de \(\hat \mZ(k_x,k_y)\).
        \begin{lemme}[Continuité des impédances]
          \label{lem:plan:continuite_impedance}
          Soit \(z_m\) une interface entre deux matériaux.
          On suppose qu'il existe des matrices de part et d'autre de l'interface telles que
          \begin{align*}
              \hat{\vE}_t(k_x,k_y,z_m^+) &= \hat{\mZ}^+(k_x,k_y) \left(\vect{e_z} \pvect \hat{\vH}_t(k_x,k_y,z_m^+)\right),
              \\
              \hat{\vE}_t(k_x,k_y,z_m^-) &= \hat{\mZ}^-(k_x,k_y) \left(\vect{e_z} \pvect \hat{\vH}_t(k_x,k_y,z_m^-)\right).
          \end{align*}
          Alors
          \begin{equation*}
          \hat \mZ^+(k_x,k_y) = \hat \mZ^-(k_x,k_y).
          \end{equation*}
        \end{lemme}

        % \begin{REM}
        %   Tu peux utiliser  cette notation mais cela ne s'appelle pas, une fois sur \(Z\) une matrice d'impédance.
        %   Comme je t'ai dit d'un coté (en \(z_m^-\)) c'est bien une matrice d'impédance qui a donc été déterminé par le pb en \([z_0,z_m^+]\).
        % \end{REM}

        \begin{proof}
          La démonstration est extrêmement rapide grâce à la continuité tangentielle des champs.
          \begin{align*}
            \hat{\vE}_t(k_x,k_y,z_m^+) &= \hat{\vE}_t(k_x,k_y,z_m^-),
            \\
            \hat{\vH}_t(k_x,k_y,z_m^+) &= \hat{\vH}_t(k_x,k_y,z_m^-).
          \end{align*}
          Donc on la même propriété sur \(\vn \pvect \hat{\vH}\),
          \begin{align*}                
            \vn \pvect \hat{\vH}_t(k_x,k_y,z_m^+) &= \vn \pvect \hat{\vH}_t(k_x,k_y,z_m^-).
            \\
            \intertext{On repart de la propriété de la matrice \(\hat{\mZ}^+(k_x,k_y)\),}
            \hat{\vE}_t(k_x,k_y,z_m^+) &= \hat{\mZ}^+(k_x,k_y) \left(\vect{e_z} \pvect \hat{\vH}_t(k_x,k_y,z_m^+)\right).
            \\
            \intertext{On utilise la continuité des champs}
            \hat{\vE}_t(k_x,k_y,z_m^-) &= \hat{\mZ}^+(k_x,k_y) \left(\vect{e_z} \pvect \hat{\vH}_t(k_x,k_y,z_m^-)\right).
            \\
            \intertext{Ce qui s'identifie à \(\hat{\mZ}^-(k_x,k_y)\)}
            \hat{\vE}_t(k_x,k_y,z_m^-) &= \hat{\mZ}^-(k_x,k_y) \left(\vect{e_z} \pvect \hat{\vH}_t(k_x,k_y,z_m^-)\right).
          \end{align*}
        \end{proof}

        \begin{defn}
          \label{def:plan:impedance:1c}
          On définit \(\hat \mZ(k_x,k_y)\) la fonction de \(\RR \times \RR\) dans \(\mathcal{M}_{2\times2}(\CC^2)\) telle que
          \begin{align*}
            k_3 &= \sqrt{k^2 - k_x^2 -k_y^2} \not = 0,
            \hat \mZ(k_x,k_y) &= i\eta_r\frac{\tan\left(k_3d\right)}{k k_3}
            \begin{bmatrix}
              k^2-k_x^2  & -k_xk_y
              \\
              -k_xk_y & k^2-k_y^2
            \end{bmatrix}.
          \end{align*}
        \end{defn}
        % \begin{REM}
        %    C'est une fonction de \(\RR^2\) dans \(\mathcal{M}_{2\times2}(\CC^2)\) et si \(k_3\in\CC\) tu as des formes alternatives avec tanh.
        % \end{REM}
        \begin{prop}
            \label{prop:imp_plan:symb_z:1c}
            On suppose que
            \begin{align*}
                k_3     & \not = 0, \\
                k_3d    & \not = \frac{\pi}{2}+n\pi\,, \forall n \in \NN.
            \end{align*}
            % \begin{REM}
            %   Remarquons que ces conditions sont toujours vérifiées si \(\eps\mu\not\in\RR_+\).
            % \end{REM}
            Alors
            \begin{align*}
              \hat \vE_t(k_x,k_y,0^-) &= \hat \mZ(k_x,k_y) \left(\vect{e_z} \pvect \vH(k_x,k_y,0^-)\right).
            \end{align*}
        \end{prop}

        \begin{proof}
            Grâce au lemme \eqref{lem:plan:continuite_impedance}, on sait donc que l'on peut uniquement se placer dans la couche.

            Il existe une répartition des constantes \((c_1,c_2,c_3c,_4)\) telles que \((c_i,c_j)\) s'exprime en fonction de \((c_k,c_l)\).
            Ici nous choisissons
            %Nous n'avons donc pas besoin de distinguer les vecteurs indépendants de \(z\) mais qui sont différents dans chaque couche, dans l'expression des champs tangentiels. Nous posons donc
            \begin{align*}
                \vect{C_1}(k_x,k_y) = \begin{bmatrix}c_1(k_x,k_y) \\ c_3(k_x,k_y)\end{bmatrix}, 
                && 
                \vect{C_2}(k_x,k_y) = \begin{bmatrix}c_2(k_x,k_y) \\ c_4(k_x,k_y)\end{bmatrix}.
            \end{align*}
            Nous utilisons la condition limite du conducteur parfait
            \begin{align*}
                \hat{\vE}_t(k_x,k_y,-d^+) &= 0,
                \\
                &=  \mA_E(k_x,k_y,-d^+)\vect{C_1}(k_x,k_y) + \mB_E(k_x,k_y,-d^+)\vect{C_2}(k_x,k_y).
            \end{align*}
            Si l’on suppose que ces matrices sont inversibles, donc que \(k_3\) est non-nul alors on a
            \begin{align*}
                \vect{C_2}(k_x,k_y) &= -\left(\mB_E(k_x,k_y,-d^+)\right)^{-1}\mA_E(k_x,k_y,-d^+)\vect{C_1}(k_x,k_y).
            \end{align*}
            % \begin{REM}
            %   La tu peux ajouter: Il existe une répartition des constantes \(c_1,c_2,c_3,c_4\) telles que \((c_i,c_j)\) s'exprime en fonction de \((c_k,c_l)\), Ici, nous en avons choisis une particulière.
            % \end{REM}
            % On part de l'expression de l'impédance, exprimée à l'intérieur
            % \begin{align*}
            %     \hat \vE_t(k_x,k_y,0^-) &= \hat \mZ(k_x,k_y) \left(\vect{e_z} \pvect \hat \vH_t(k_x,k_y,0^-)\right)
            % \end{align*}
            Or on injecte le résultat précédent
            \begin{multline*}
                \hat \vE_t(k_x,k_y,0^-) =
                \\
                \left(\mA_E(k_x,k_y,0^-)-\left(\mB_E(k_x,k_y,-d^+)\right)^{-1}\mA_E(k_x,k_y,-d^+)\right)\vect{C_1}(k_x,k_y),
            \end{multline*}
            \begin{multline*}
                \vn \pvect \hat \vH_t(k_x,k_y,0^-) =
                \\
                \left(\mA_H(k_x,k_y,0^-)-\left(\mB_E(k_x,k_y,-d^+)\right)^{-1}\mA_E(k_x,k_y,-d^+)\right)\vect{C_1}(k_x,k_y).
            \end{multline*}
            On en déduit l'impédance
            \begin{multline*}
                \hat \mZ(k_x,k_y) =
                \\
                \left(\mA_E(k_x,k_y,0^-)-\mB_E(k_x,k_y,0^-)\left(\mB_E(k_x,k_y,-d^+)\right)^{-1}\mA_E(k_x,k_y,-d^+)\right)
                \\
                \left(\mA_H(k_x,k_y,0^-)-\mB_H(k_x,k_y,0^-)\left(\mB_H(k_x,k_y,-d^+)\right)^{-1}\mA_E(k_x,k_y,-d^+)\right)^{-1}.
            \end{multline*}
            Comme on a supposé les matrices inversibles
            \begin{multline*}
                \hat \mZ(k_x,k_y) =
                \\ \left(\mA_E(k_x,k_y,0^-)\left(\mA_E(k_x,k_y,-d^+)\right)^{-1}-\mB_E(k_x,k_y,0^-)\left(\mB_E(k_x,k_y,-d^+)\right)^{-1}\right) 
                \\
                \left(\mA_H(k_x,k_y,0^-)\left(\mA_E(k_x,k_y,-d^+)\right)^{-1}-\mB_H(k_x,k_y,0^-)\left(\mB_E(k_x,k_y,-d^+)\right)^{-1}\right)^{-1}.
            \end{multline*}
            On va alors exprimer chacun des termes grâce aux expressions des matrices
            \begin{align*}
                \mA_E(k_x,k_y,0^-)\left(\mA_E(k_x,k_y,-d^+)\right)^{-1} &= e^{ik_3d}\mI,
                \\
                \mB_E(k_x,k_y,0^-)\left(\mB_E(k_x,k_y,-d^+)\right)^{-1} &= e^{-ik_3d}\mI,
                \\
                \mA_H(k_x,k_y,0^-)\left(\mA_E(k_x,k_y,-d^+)\right)^{-1} &= e^{ik_3d}\frac{1}{k\eta_r k_3}
                \begin{bmatrix}
                    k^2 - k_y^2 & k_x k_y
                    \\
                    k_x k _y & k^2 - k_x^2
                \end{bmatrix},
                \\
                \mB_H(k_x,k_y,0^-)\left(\mB_E(k_x,k_y,-d^+)\right)^{-1} &= -e^{-ik_3d}\frac{1}{k\eta_r k_3}
                    \begin{bmatrix}
                    k^2 - k_y^2 & k_x k_y
                    \\
                    k_x k _y & k^2 - k_x^2
                \end{bmatrix}.
            \end{align*}
            Soient les matrices \(\hat\mLD,\hat\mLR\) définies à la section \ref{sec:plan:hoibc:LD-LR}
            \begin{align*}
                \hat{\mLD} = -\begin{bmatrix}
                k_x^2 & k_x k_y
                \\
                k_x k _y & k_y^2
                \end{bmatrix},
                & & 
                \hat{\mLR} = \begin{bmatrix}
                k_y^2 & -k_x k_y
                \\
                -k_x k _y &  k_x^2
                \end{bmatrix}.
            \end{align*}
            On remarque alors que 
            \begin{align*}
                \mA_H(k_x,k_y,0^-)\left(\mA_E(k_x,k_y,-d^+)\right)^{-1} &=  e^{ik_3d}\frac{1}{k\eta_r k_3}(k^2\mI  -\hat\mLR),
                \\
                \mB_H(k_x,k_y,0^-)\left(\mB_E(k_x,k_y,-d^+)\right)^{-1} &= -e^{-ik_3d}\frac{1}{k\eta_r k_3}(k^2\mI  -\hat\mLR).
            \end{align*}
            Ces matrices sont inversibles si \(k_3\) est non nul, ce que nous avons déjà supposé.

            Un simple calcul matriciel permet de trouver que \( (k^2\mI - \hat{\mLR})^{-1} = \frac{1}{k^2k_3^2}\left(k^2 \mI + \hat{\mLD}\right) \).

            En supposant \(k_3d \not = \frac{\pi}{2} + n\pi\), donc \(e^{ik_3d}+e^{-ik_3d}\not=0\) on déduit que
            \begin{align*}
                \hat \mZ(k_x,k_y) &= \frac{\eta_r}{k k_3} \frac{e^{ik_3d} - e^{-ik_3d}}{e^{ik_3d} + e^{-ik_3d}}\left(k^2\mI + \hat{\mLD}\right),
                \\
                &= i\eta_r\frac{\tan\left(k_3d\right)}{k k_3}\left(k^2\mI + \hat{\mLD}\right).
            \end{align*}
        \end{proof}
        %On remarque que \(\det(\mat{Z}) = i\frac{\eta^2}{k_3}\eta\tan(k_3d)\) et donc pour un matériau \((\eps,\mu,d)\) donné, l'opérateur d'impédance n'est pas inversible pour tous  \((k_x,k_y) \in \RR^2, n \in \NN\), \(k_x^2+k_y^2 =  \w^2\eps\mu - \frac{1}{d^2}\left(\frac{\pi}{2} + n\pi\right)^2\), qui ne peut être vérifié que si \(\eps\mu\) est réel\footnote{Comme \(\eps, \mu\) sont à partie réelle (resp. imaginaire) strictement positive (resp. négative), alors ce n'est vrai pour les matériaux à partie imaginaire nulle.}.
        Dans l'article de \cite{marceaux_high-order_2000}, la matrice d'impédance est telle que
        \begin{align*}
             \vn \pvect \hat{\vE} = \hat{\mathfrak{Z}} \hat{\vH_t}, && \hat{\mathfrak{Z}} = -i\eta_r\frac{\tan(k_3d)}{kk_3}\left(k^2\mI - \hat{\mLR}\right).
        \end{align*}
        C'est une forme équivalente à la nôtre où \(\hat{\mZ} = \mT\hat{\mathfrak{Z}}\mT\) avec \(\mT = \begin{bmatrix}0&-1\\1&0\end{bmatrix}\).

        Notons que \(\hat\mZ(k_x,0)\) est une matrice diagonale.
        
        % \begin{REM}
        %   Ca ce n'est pas math. Mets Notons que \(\mZ(k_x,0)\) est une matrice diagonale.
        % \end{REM}
        Dans la littérature (\cite{stupfel_implementation_2015,aubakirov_electromagnetic_2014,hoppe_impedance_1995}), on nomme les termes diagonaux \(\hat Z_{TM}, \hat Z_{TE}\) tels que
         % le champ \(\vE\)-TE correspond à \({E_y} \vect{e_y}\), le champ \(\vE\)-TM à \({E_x} \vect{e_x} + {E_z} \vect{e_z} \), tandis que le champ \(\vH\)-TE correspond à \({H_x} \vect{e_x} + {H_z} \vect{e_z}\) et le champ \(\vH\)-TM correspond à \({H_y} \vect{e_y}\).
        
        % Alors la matrice \(\hat \mZ\) peut se réécrire comme
        \begin{equation*}
            \hat \mZ =
            \begin{bmatrix}
                \hat Z_{TM} & 0
                \\
                0 & \hat Z_{TE}
            \end{bmatrix}.
        \end{equation*}

  \subsection{Expression de la matrice d'impédance pour plusieurs couches}
    On suppose que l'on a \(n\) couches de matériaux :
    \begin{figure}[h!btp]
        \centering
        \tikzsetnextfilename{plan_n_couches}            
        \begin{tikzpicture}
            \tikzmath{
    \largeur = 6;
    \hauteur = 0.5;
    \milieu = 1.3;
    \xC = \largeur;
    \xA = 0;
}

%% 1ere couche
\tikzmath{
    \yC = \hauteur;
    \yA = 0;
}

\coordinate (A) at (\xA,\yA);
\coordinate (B) at (\xA,\yC);
\coordinate (C) at (\xC,\yC);

\draw ($(B)!0.5!(C)$) node [above] {vide};


\fill [lightgray] (A) rectangle (C);
\draw ($(A)!0.5!(C)$) node {$\eps_n,\mu_n,d_n$};
\draw (B) -- (C) node [right] {$e_3 = 0$};

%% Des couches
\tikzmath{
    \yC = \yC - \hauteur;
    \yA = \yA - \milieu*\hauteur;
}

\coordinate (A) at (\xA,\yA);
\coordinate (B) at (\xA,\yC);
\coordinate (C) at (\xC,\yC);

\fill [lightgray]    (A) rectangle (C);
\fill [pattern=dots] (A) rectangle (C);
\draw (B) -- (C);

%% N ieme couche
\tikzmath{
    \yC = \yC - \milieu*\hauteur;
    \yA = \yA - \hauteur;
}

\coordinate (A) at (\xA,\yA);
\coordinate (B) at (\xA,\yC);
\coordinate (C) at (\xC,\yC);
\fill [lightgray] (A) rectangle (C);
\draw ($(A)!0.5!(C)$) node {$\eps_1,\mu_1,d_1$};
\draw (B) -- (C);

%% Le repère
\tikzmath{
    \xD = \xC + 0.5;
}

\coordinate (n) at (\xD,\yA);
\draw [->] (n) -- ++(1,0) node [at end, right] {$\v{e_1}$};
\draw [->] (n) -- ++(0,1) node [at end, right] {$\v{e_3}$};

\draw (n) circle(0.1cm) node [below=0.1cm] {$\v{e_2}$};
\draw (n) +(135:0.1cm) -- +(315:0.1cm);
\draw (n) +(45:0.1cm) -- +(225:0.1cm);

%% Le conducteur
\tikzmath{
    \yC = \yC - \hauteur;
    \yA = \yA - 0.5*\hauteur;
}

\coordinate (A) at (\xA,\yA);
\coordinate (B) at (\xA,\yC);
\coordinate (C) at (\xC,\yC);
\draw (B) -- (C);

\fill [pattern=north east lines] (A) rectangle (C);



        \end{tikzpicture}
    \end{figure}

    On suppose qu'il n'y pas de résonances, et nous renvoyons à la proposition \ref{prop:unicite:interieur:postulat:multi-couche} l’expression de ces dernières.

    % \begin{REM}
    %   Rappelle au départ le problème bien posé avec conditions de Dirichlet homogène a une unique solution si \(\w^2\eps\mu\) n'est pas une résonance.
    %   Il faut quand même que tu mentionnes que d'après la théorie, les composantes qui permettent de calculer la solution sont deux des quatre constantes \(c_i\) que tu as choisis \((c_2,c_4)=f(c_1,c_3)\).
    % \end{REM}
    Soit \(z_p\) la hauteur de l'interface \(p\), \(z_p = -\sum_{i=p+1}^{n} d_{i}\). On dit que l'on se trouve dans la couche \(p\) si \(z_{p-1} \le z < z_p \).

    \begin{defn}
      \label{def:plan:matrices_MA-MB}
      On définit les fonctions de \(\RR\times\RR\times [z_{p-1}, z_p[ \times \mathcal{M}_{2}(\CC) \rightarrow \mathcal{M}_{2}(\CC)\) :
      \begin{align*}
        \mM_{A}(k_x,k_y,z,\mZ) &= \mA_{E}(k_x,k_y,z)-\mZ\mA_{H}(k_x,k_y,z),
        \\
        \mM_{B}(k_x,k_y,z,\mZ) &= \mB_{E}(k_x,k_y,z)-\mZ\mB_{H}(k_x,k_y,z).
      \end{align*}
    \end{defn}

    \begin{defn}
      \label{def:plan:reflexion:impedance}
      On définit la fonction de \(\RR\times\RR\times [z_{p-1}, z_p[ \times \mathcal{M}_{2}(\CC) \rightarrow \mathcal{M}_{2}(\CC)\) :
      \begin{align*}
        \hat\mR(k_x,k_y,z,\mZ) &= -\mM_{B}(k_x,k_y,z,\mZ)^{-1}\mM_{A}(k_x,k_y,z,\mZ).
      \end{align*}
    \end{defn}
    A priori, pour \(k_x,k_y,z\) donnés, \(\hat\mR(k_x,k_y,z,\mZ)\) n'est pas définie pour toute matrice \(\mZ\).
    On prolonge ces définitions aux autres couches.

    \begin{defn}%[Fonction de transfert]{}~
      \label{def:plan:transfert:impedance}

      On définit \(\mT_p\) la fonction de \(\RR^2\times\times [z_{p-1}, z_p[^2\times\mathcal{M}_2(\CC)\rightarrow \mathcal{M}_{2}(\CC)\) :
      \begin{multline*}
        \mT_p(k_x,k_y,z,z',\mZ) = \\
          \left(\mA_{E}(k_x,k_y,z,)\mM_{A}(k_x,k_y,z',\mZ)^{-1} - \mB_{E}(k_x,k_y,z,)\mM_{B}(k_x,k_y,z',\mZ)^{-1}\right) 
          \\
          \left(\mA_{H}(k_x,k_y,z,)\mM_{A}(k_x,k_y,z',\mZ)^{-1} - \mB_{H}(k_x,k_y,z,)\mM_{B}(k_x,k_y,z',\mZ)^{-1}\right)^{-1}.
      \end{multline*}
    \end{defn}

    A priori, pour \((k_x,k_y,z,z')\) donnés, \(\mT_p(k_x,k_y,z,z',\mZ)\) n'est pas définie pour toute matrice \(\mZ\).

    \begin{prop}%[Théorème de transfert]~
      \label{prop:plan:transfert:impedance}

      Soient \(\hat\vE,\hat\vH\) tels que \(\vE_t(k_x,k_y,z_p^-) = \hat\mZ_{p}(k_x,k_y)(\vn \pvect \hat\vH(k_x,k_y,z_p^-)\).

      Si les matrices suivantes sont inversibles
      \begin{align*}
        \mM_{A}(k_x,k_y,z_p^-, \hat\mZ_{p}(k_x,k_y)), && \mM_{B}(k_x,k_y,z_p^-, \hat\mZ_{p}(k_x,k_y)),
      \end{align*}
      \begin{align*}
        \mA_{H}(k_x,k_y,z_{p-1})\mM_{A}(k_x,k_y,z_p^-, \hat\mZ_{p}(k_x,k_y))^{-1} - \mB_{H}(k_x,k_y,z_{p-1})\mM_{B}(k_x,k_y,z_p^-, \hat\mZ_{p}(k_x,k_y))^{-1},
      \end{align*}

      alors \(\hat\vE_t(k_x,k_y,z_{p-1}) = \mT_p(k_x,k_y,z_{p-1},z_p^-,\hat\mZ_{p}(k_x,k_y))(\vn \pvect \hat\vH(k_x,k_y,z_{p-1}))\).

      Une condition limite sur le bord supérieur d'une couche détermine la condition limite sur le bord inférieur.
    \end{prop}


    \begin{proof}
      On se situe dans la couche \(p\) (\(z_{p-1}\le z < z_p\)) et l'on sait qu'il existe dans cette couche des constantes \(c_i(k_x,k_y)\) telles que les champs vérifient
      \begin{multline*}
        \mA_{E}(k_x,k_y,z_p^-)
        \begin{bmatrix}
          c_1(k_x,k_y) \\
          c_3(k_x,k_y)
        \end{bmatrix}
        +
        \mB_{E}(k_x,k_y,z_p^-)
        \begin{bmatrix}
          c_2(k_x,k_y) \\
          c_4(k_x,k_y)
        \end{bmatrix}
        =
        \\
        \hat \mZ_{p}(k_x,k_y)
        \left(
          \mA_{H}(k_x,k_y,z_p^-)
          \begin{bmatrix}
            c_1(k_x,k_y) \\
            c_3(k_x,k_y)
          \end{bmatrix}
          +
          \mB_{H}(k_x,k_y,z_p^-)
          \begin{bmatrix}
            c_2(k_x,k_y) \\
            c_4(k_x,k_y)
          \end{bmatrix}
        \right).
      \end{multline*}

      Ce qui revient à 
      \begin{equation*}
        \mM_{A}(k_x,k_y,z_p^-,\hat\mZ_p(k_x,k_y))
        \begin{bmatrix}
          c_1(k_x,k_y) \\
          c_3(k_x,k_y)
        \end{bmatrix}
        =
        -\mM_{B}(k_x,k_y,z_p^-,\hat\mZ_p(k_x,k_y))
        \begin{bmatrix}
          c_2(k_x,k_y) \\
          c_4(k_x,k_y)
        \end{bmatrix}.
      \end{equation*}

      On suppose que les matrices \(\mM_{A}(k_x,k_y,z_p^-,\hat\mZ_p(k_x,k_y)), \mM_{B}(k_x,k_y,z_p^-,\hat\mZ_p(k_x,k_y))\) sont inversibles donc
      \begin{equation*}
        \begin{bmatrix}
          c_2(k_x,k_y) \\
          c_4(k_x,k_y)
        \end{bmatrix}
        =
        \hat\mR(k_x,k_y,z_p^-,\hat\mZ_p(k_x,k_y))
        \begin{bmatrix}
          c_1(k_x,k_y) \\
          c_3(k_x,k_y)
        \end{bmatrix}.
      \end{equation*}

      On injecte ce qui précède en \(z = z_{p-1}\) :
      \begin{multline*}
        \hat{\vE}_t(k_x,k_y,z_{p-1}) = 
        \\
        \left(\mB_{E}(k_x,k_y,z_{p-1})\hat\mR(k_x,k_y,z_p^-,\hat{\mZ}_p(k_x,k_y)) + \mA_{E}(k_x,k_y,z_{p-1})\right)
        \begin{bmatrix}
          c_1(k_x,k_y) \\
          c_3(k_x,k_y)
        \end{bmatrix},
      \end{multline*}        
      \begin{multline*}
        \vect{e_r}\times\hat{\vH}_t(k_x,k_y,z_{p-1}) =
        \\
        \left(\mB_{H}(k_x,k_y,z_{p-1})\hat\mR(k_x,k_y,z_p^-,\hat{\mZ}_p(k_x,k_y)) + \mA_{H}(k_x,k_y,z_{p-1}))\right)
        \begin{bmatrix}
          c_1(k_x,k_y) \\
          c_3(k_x,k_y)
        \end{bmatrix}.
      \end{multline*}

      On suppose alors que cette dernière est inversible pour tout \((k_x,k_y)\).

      On obtient
      \begin{multline*}
        \hat{\vE}_t(k_x,k_y,z_{p-1}) =
        \\
        \left(\mA_{E}(k_x,k_y,z_{p-1}) + \mB_{E}(k_x,k_y,z_{p-1})\hat\mR(k_x,k_y,z_p^-,\hat{\mZ}_p(k_x,k_y))\right) \\
        \left(\mA_{H}(k_x,k_y,z_{p-1}) + \mB_{H}(k_x,k_y,z_{p-1})\hat\mR(k_x,k_y,z_p^-,\hat{\mZ}_p(k_x,k_y))\right)^{-1}
        \\
        \vect{e_r}\times\hat{\vH}_t(k_x,k_y,z_{p-1}).
      \end{multline*}

      Comme on a supposé l'inversibilité des deux matrices \(\mM_{A}\), \(\mM_{B}\), on simplifie afin d'obtenir la propriété.
    \end{proof}

    \begin{prop}%[Théorème de relèvement]~
      \label{prop:plan:relevement:impedance}

      Soient \(\hat\vE_t,\hat\vH_t\) tels que \(\vE_t(k_x,k_y,z_{p-1}) = \hat\mZ_{p-1}(k_x,k_y)(\vn \pvect \hat\vH(k_x,k_y,z_{p-1})\).

      Si les matrices suivantes sont inversibles
      \begin{align*}
        \mM_{A}(k_x,k_y,z_{p-1},\hat\mZ_{p-1}(k_x,k_y)), && \mM_{B}(k_x,k_y,z_{p-1},\hat\mZ_{p-1}(k_x,k_y)),
      \end{align*}
      \begin{align*}
        \mB_{H}(k_x,k_y,z_p^-)\mM_{B}(k_x,k_y,z_{p-1},\hat\mZ_{p-1}(k_x,k_y))^{-1} - \mA_{H}(k_x,k_y,z_p^-)\mM_{A}(k_x,k_y,z_{p-1},\hat\mZ_{p-1}(k_x,k_y))^{-1},
      \end{align*}

      alors \(\hat\vE_t(k_x,k_y,z_p^-) = \mT_p(k_x,k_y,z_p^-,z_{p-1},\hat\mZ_{p-1}(k_x,k_y))(\vn \pvect \hat\vH(k_x,k_y,z_p^-))\).

      Une condition d'impédance sur le bord inférieur d'une couche détermine la condition limite sur le bord supérieur.
    \end{prop}

    % \begin{REM}
    %   Il faudrait le formuler plus précisément.
    %   En effet, prend les 3 pbs suivants: \(\OO_j=\OO_0\cap C_1 \dots \cap C_j\) ( \(\OO\) recouvert de \(j\) couches ).
      
    %   Le premier problème est \(u_{|\partial \OO_0} = 0\), \(u_{|\partial C_j^+} = u_0\) donné. Si \(\w^2\eps\mu\) n'est pas une résonance alors il existe un unique \(u\) solution et donc le Calderón est déterminé \(\partial_{nj}u=C_j(u_{|\partial ????? })\).

    %   Le deuxième problème est \(v_{|\partial \OO_0} = 0\), \(v_{|\partial C_{j+1}^+} = v_0\)  donné. Si \(\w^2\eps\mu\) n'est pas une résonance alors il existe un unique \(v\) solution et donc le Calderón est déterminé.

    %   Enfin tu regardes le 3\ieme problème, \(u_{|\partial C_j^+}\) et \(\partial_{n} u_{|\partial C_j^+}\) sont liés par le Calderón \(C_j\). \(u_{|\partial C_{j+1}^+}=v_0\) donné, et tu dis que ce 3\ieme problème donne exactement \(v\) dans  la couche \(j+1\). C'est à démontrer aussi.

    %   Tu as le pb des résonances et de la solution du 3\ieme pb.
    % \end{REM}
    % \begin{REP}
    %   Ce n'est pas du tout clair ce que tu demandes. Quel est le but ?
    % \end{REP}
    \begin{proof}
      Même méthodologie que pour la proposition \ref{prop:plan:transfert:impedance}.
    \end{proof}


    \begin{prop}%[Corollaire aux théorèmes de transfert et de relèvement.]{}~
      \label{prop:plan:synthese:impedance}{}~

      Soient \(\hat\vE_t,\hat\vH_t\) tels que 
      \begin{align*}
      \vE_t(k_x,k_y,z_{p-1}) &= \hat\mZ_{p-1}(k_x,k_y)(\vn \pvect \hat\vH(k_x,k_y,z_{p-1})),
      \\
      \vE_t(k_x,k_y,z_p^-) &= \hat\mZ_{p}(k_x,k_y)(\vn \pvect \hat\vH(k_x,k_y,z_p^-)).
      \end{align*}

      Si les matrices suivantes sont inversibles
      \begin{align*}
        \mM_{A}(k_x,k_y,z_p^-,\hat\mZ_{p}(k_x,k_y)), && \mM_{A}(k_x,k_y,z_{p-1},\hat\mZ_{p-1}(k_x,k_y)),
        \\
        \mM_{B}(k_x,k_y,z_p^-,\hat\mZ_{p}(k_x,k_y)), && \mM_{B}(k_x,k_y,z_{p-1},\hat\mZ_{p-1}(k_x,k_y)),
      \end{align*}
      \begin{align*}
        \mA_{H}(k_x,k_y,z_p^-)\mM_{A}(k_x,k_y,z_{p-1},\hat\mZ_{p-1}(k_x,k_y))^{-1} - \mB_{H}(k_x,k_y,z_p^-)\mM_{B}(k_x,k_y,z_{p-1},\hat\mZ_{p-1}(k_x,k_y))^{-1},
        \\
        \mA_{H}(k_x,k_y,z_{p-1})\mM_{A}(k_x,k_y,z_p^-,\hat\mZ_{p}(k_x,k_y))^{-1} - \mB_{H}(k_x,k_y,z_{p-1})\mM_{B}(k_x,k_y,z_p^-,\hat\mZ_{p}(k_x,k_y))^{-1},
      \end{align*}

      alors 
      \begin{align*}
        \hat\mZ_{p-1}(k_x,k_y) &= \mT_p(k_x,k_y,z_{p-1},z_p^-,\hat\mZ_{p}(k_x,k_y)),
        \\
        \hat\mZ_{p}(k_x,k_y) &= \mT_p(k_x,k_y,z_p^-,z_{p-1},\hat\mZ_{p-1}(k_x,k_y)).
      \end{align*}

    \end{prop}

    On peut donc déterminer itérativement les matrices \(\hat\mZ_{p}\) pour obtenir la matrice associée à l'opérateur de Calderón.
    Dans notre cadre d'étude, la présence d'un conducteur parfait sur l'interface \(z= z_0\) implique \(\hat\mZ_{0}(k_x,k_y) = 0\).

    %%%%%%%%%%%%%%%%%%%%%%%%%%%%%%%%%%%%%%%%%%%%%%%%%%%%%%%%%%%%%%%%%%%%%%%
    %%%%%%%%%%%%%%%%%%%%%%%%%%%%%%%%%%%%%%%%%%%%%%%%%%%%%%%%%%%%%%%%%%%%%%%
    %%%%%%%%%%%%%%%%%%%%%%%%%%%%%%%%%%%%%%%%%%%%%%%%%%%%%%%%%%%%%%%%%%%%%%%

\subsection{Expression des coefficients de la série de Fourier}

    On se place de part et d'autre de l'interface \(p\) donc \(z_{p-1} \le z < z_{p+1} \).

    \begin{defn}
      \label{def:plan:matrices_NE-NH}
      On définit les fonctions de \(\RR^2 \times [z_{p-1},z_{p+1}[ \times \mathcal{M}_{2}(\CC) \rightarrow \mathcal{M}_{2}(\CC)\) :
      \begin{align*}
        \mN_{E}(k_x,k_y,z,\mR) &= \mA_{E}(k_x,k_y,z) + \mB_{E}(k_x,k_y,z)\mR,
        \\
        \mN_{H}(k_x,k_y,z,\mR) &= \mA_{H}(k_x,k_y,z) + \mB_{H}(k_x,k_y,z)\mR.
      \end{align*}
    \end{defn}

    \begin{defn}%[Fonction de transfert]{}~
      \label{def:plan:transfert:reflexion}{}~

      On définit \(\mathfrak{T}_p\) la fonction de \(\RR^2 \times [z_{p-1}, z_p[\times[z_p, z_{p+1}[\times\mathcal{M}_2(\CC)\rightarrow \mathcal{M}_{2}(\CC)\) :
      \begin{multline*}
        \mathfrak{T}_p(k_x,k_y,z,z',\mZ) = \\
          -\left(\mN_{E}(k_x,k_y,z',\mZ)^{-1}\mB_{E}(k_x,k_y,z) - \mN_{H}(k_x,k_y,z',\mZ)^{-1}\mB_{H}(k_x,k_y,z)\right)^{-1},
          \\
          \left(\mN_{E}(k_x,k_y,z',\mZ)^{-1}\mA_{E}(k_x,k_y,z) - \mN_{H}(k_x,k_y,z',\mZ)^{-1}\mA_{H}(k_x,k_y,z)\right).
      \end{multline*}
    \end{defn}
    A priori, pour \(k_x,k_y,z,z'\) donnés, \(\mathfrak{T}_p(k_x,k_y,z,z',\mZ)\) n'est pas définie pour toute matrice \(\mZ\).

    \begin{prop}%[Théorème de transfert]~
      \label{prop:plan:transfert:reflexion}{}~

      On suppose qu'il existe \(\hat\mR_{p+1}(k_x,k_y)\) telle que 
      \begin{align*}
        \vE_t(k_x,k_y,z_p^+) &= \mN_{E}(k_x,k_y,z_p^+,\hat\mR_{p+1}(k_x,k_y))\vect{C}_{p+1}(k_x,k_y),
        \\
        \vect{e_r}\pvect\vH(k_x,k_y,z_p^+) &= \mN_{H}(k_x,k_y,z_p^+,\hat\mR_{p+1}(k_x,k_y))\vect{C}_{p+1}(k_x,k_y).
      \end{align*}

      Si les matrices suivantes sont inversibles
      \begin{align*}
        \mN_{E}(k_x,k_y,z_p^+,\hat\mR_{p+1}(k_x,k_y)), && \mN_{H}(k_x,k_y,z_p^+,\hat\mR_{p+1}(k_x,k_y)),
      \end{align*}
      \begin{align*}
        \mN_{E}(k_x,k_y,z_p^+,\hat\mR_{p+1}(k_x,k_y))^{-1}\mB_{E}(k_x,k_y,z_p^-) - \mN_{H}(k_x,k_y,z_p^+,\hat\mR_{p+1}(k_x,k_y))^{-1}\mB_{H}(k_x,k_y,z_p^-),
      \end{align*}
      alors
      \begin{align*}
        \vE_t(k_x,k_y,z_p^-) &= \mN_{E}(k_x,k_y,z_p^-,\mathfrak{T}_p(k_x,k_y,z_p^-,z_p^+,\hat\mR_{p+1}(k_x,k_y)))\vect{C}_{p}(k_x,k_y),
        \\
        \vE_t(k_x,k_y,z_p^-) &= \mN_{H}(k_x,k_y,z_p^-,\mathfrak{T}_p(k_x,k_y,z_p^-,z_p^+,\hat\mR_{p+1}(k_x,k_y)))\vect{C}_{p}(k_x,k_y).
      \end{align*}
    \end{prop}

    \begin{proof}
      De part et d'autre de \(z=z_p\), on a 
      \begin{align*}
        \vE_t(k_x,k_y,z_p^+) &= \mN_{E}(k_x,k_y,z_p^+,\hat\mR_{p+1}(k_x,k_y))\vect{C}_{1}^+(k_x,k_y),
        \\
        \vE_t(k_x,k_y,z_p^-) &= \mA_E(k_x,k_y,z_p^-)\vect{C}_{1}^-(k_x,k_y) + \mB_E(k_x,k_y,z_p^-)\vect{C}_{2}^-(k_x,k_y),
      \end{align*}
      \begin{align*}
        \vect{e_r}\pvect\vH(k_x,k_y,z_p^+) &= \mN_{H}(k_x,k_y,z_p^+,\hat\mR_{p+1}(k_x,k_y))\vect{C}_{1}^+(k_x,k_y),
        \\
        \vect{e_r}\pvect\vH(k_x,k_y,z_p^-) &= \mA_H(k_x,k_y,z_p^-)\vect{C}_{1}^-(k_x,k_y) + \mB_H(k_x,k_y,z_p^-)\vect{C}_{2}^-(k_x,k_y).
      \end{align*}
      Il y a continuité des champs à l'interface donc
      \begin{align*}
        \mA_E(k_x,k_y,z_p^-)\vect{C}_{1}^-(k_x,k_y) + \mB_E(k_x,k_y,z_p^-)\vect{C}_{2}^-(k_x,k_y) &= \mN_{E}(k_x,k_y,z_p^+,\hat\mR_{p+1}(k_x,k_y))\vect{C}_{1}^+(k_x,k_y),
        \\
        \mA_H(k_x,k_y,z_p^-)\vect{C}_{1}^-(k_x,k_y) + \mB_H(k_x,k_y,z_p^-)\vect{C}_{2}^-(k_x,k_y) &= \mN_{H}(k_x,k_y,z_p^+,\hat\mR_{p+1}(k_x,k_y))\vect{C}_{1}^+(k_x,k_y),
      \end{align*}
      donc si on suppose les matrices \(\mN_E, \mN_H\) inversibles
      \begin{multline*}
        \mN_{E}(k_x,k_y,z_p^+,\hat\mR_{p+1}(k_x,k_y))^{-1}\left(\mA_E(k_x,k_y,z_p^-)\vect{C}_{1}^-(k_x,k_y) + \mB_E(k_x,k_y,z_p^-)\vect{C}_{2}^-(k_x,k_y)\right) =
        \\
        \mN_{H}(k_x,k_y,z_p^+,\hat\mR_{p+1}(k_x,k_y))^{-1}\left(\mA_H(k_x,k_y,z_p^-)\vect{C}_{1}^-(k_x,k_y) + \mB_H(k_x,k_y,z_p^-)\vect{C}_{2}^-(k_x,k_y)\right).
      \end{multline*}

      On regroupe les termes pour obtenir une relation entre les deux vecteurs,
      \begin{multline*}
        \left(
        \mN_{E}(k_x,k_y,z_p^+,\hat\mR_{p+1}(k_x,k_y))^{-1}\mA_E(k_x,k_y,z_p^-)
        \right.
        \\
        \left.
        - \mN_{H}(k_x,k_y,z_p^+,\hat\mR_{p+1}(k_x,k_y))^{-1}\mA_H(k_x,k_y,z_p^-)
        \right)\vect{C}_{1}^-(k_x,k_y) =
        \\
        -\left(
        \mN_{E}(k_x,k_y,z_p^+,\hat\mR_{p+1}(k_x,k_y))^{-1}\mB_E(k_x,k_y,z_p^-)
        \right.
        \\
        \left.
        - \mN_{H}(k_x,k_y,z_p^+,\hat\mR_{p+1}(k_x,k_y))^{-1}\mB_H(k_x,k_y,z_p^-)
        \right)\vect{C}_{2}^-(k_x,k_y).
      \end{multline*}

      On suppose l'inversibilité de la matrice devant \(\vect{C}_{1}^-(k_x,k_y)\), et alors
      \begin{equation*}
        \vect{C}_{2}^-(k_x,k_y) = \mathfrak{T}_p(k_x,k_y,z_p^-,z_p^+,\hat\mR_{p+1}(k_x,k_y)) \vect{C}_{1}^-(k_x,k_y).
      \end{equation*}
    \end{proof}

    \begin{prop}%[Théorème de relévement]~
      \label{prop:plan:relevement:reflexion}{}~

      On suppose qu'il existe \(\hat\mR_{p}(k_x,k_y)\) telle que 
      \begin{align*}
        \vE_t(k_x,k_y,z_p^-) &= \mN_{E}(k_x,k_y,z_p^-,\hat\mR_{p}(k_x,k_y))\vect{C}_{p}(k_x,k_y),
        \\
        \vect{e_r}\pvect\vH(k_x,k_y,z_p^-) &= \mN_{H}(k_x,k_y,z_p^-,\hat\mR_{p}(k_x,k_y))\vect{C}_{p}(k_x,k_y).
      \end{align*}

      Si les matrices suivantes sont inversibles
      \begin{align*}
        \mN_{E}(k_x,k_y,z_p^-,\mR_{p}(k_x,k_y)), && \mN_{H}(k_x,k_y,z_p^-,\hat\mR_{p}(k_x,k_y)),
      \end{align*}
      \begin{align*}
        \mN_{E}(k_x,k_y,z_p^-,\mR_{p}(k_x,k_y))^{-1}\mB_{E}(k_x,k_y,z_p^+) - \mN_{H}(k_x,k_y,z_p^-,\hat\mR_{p}(k_x,k_y))^{-1}\mB_{H}(k_x,k_y,z_p^+),
      \end{align*}
      alors
      \begin{align*}
        \vE_t(k_x,k_y,z_p^+) &= \mN_{E}(k_x,k_y,z_p^+,\mathfrak{T}_p(k_x,k_y,z_p^+,z_p^-,\hat\mR_{p}(k_x,k_y)))\vect{C}_{p+1}(k_x,k_y),
        \\
        \vE_t(k_x,k_y,z_p^+) &= \mN_{H}(k_x,k_y,z_p^+,\mathfrak{T}_p(k_x,k_y,z_p^+,z_p^-,\hat\mR_{p}(k_x,k_y)))\vect{C}_{p+1}(k_x,k_y).
      \end{align*}
    \end{prop}

    \begin{proof}
     Preuve identique à la proposition \ref{prop:plan:transfert:reflexion}.
    \end{proof}

    \begin{prop}%[Théorème de relévement]~
      \label{prop:plan:synthese:reflexion}{}~

      On suppose qu'il existe \(\hat\mR_{p}(k_x,k_y)\) et \(\hat\mR_{p+1}(k_x,k_y)\) telles que 
      \begin{align*}
      &\left\lbrace\begin{aligned}
        \vE_t(k_x,k_y,z_p^-) &= \mN_{E}(k_x,k_y,z_p^-,\hat\mR_{p}(k_x,k_y))\vect{C}_{p}(k_x,k_y),
        \\
        \vect{e_r}\pvect\vH(k_x,k_y,z_p^-) &= \mN_{H}(k_x,k_y,z_p^-,\hat\mR_{p}(k_x,k_y))\vect{C}_{p}(k_x,k_y),
        \end{aligned}
      \right.
      \\
      &\left\lbrace\begin{aligned}
        \vE_t(k_x,k_y,z_p^+) &= \mN_{E}(k_x,k_y,z_p^+,\hat\mR_{p+1}(k_x,k_y))\vect{C}_{p+1}(k_x,k_y),
        \\
        \vect{e_r}\pvect\vH(k_x,k_y,z_p^+) &= \mN_{H}(k_x,k_y,z_p^+,\hat\mR_{p+1}(k_x,k_y))\vect{C}_{p+1}(k_x,k_y).
        \end{aligned}
      \right.      
      \end{align*}

      Si les matrices suivantes sont inversibles
      \begin{align*}
        \mN_{E}(k_x,k_y,z_p^-,\mR_{p}(k_x,k_y)), && \mN_{H}(k_x,k_y,z_p^-,\hat\mR_{p}(k_x,k_y)),
        \\
        \mN_{E}(k_x,k_y,z_p^+,\mR_{p+1}(k_x,k_y)), && \mN_{H}(k_x,k_y,z_p^+,\hat\mR_{p+1}(k_x,k_y)),
      \end{align*}
      \begin{align*}
        \mN_{E}(k_x,k_y,z_p^-,\mR_{p}(k_x,k_y))^{-1}\mB_{E}(k_x,k_y,z_p^+) - \mN_{H}(k_x,k_y,z_p^-,\hat\mR_{p}(k_x,k_y))^{-1}\mB_{H}(k_x,k_y,z_p^+),
        \\
        \mN_{E}(k_x,k_y,z_p^+,\mR_{p+1}(k_x,k_y))^{-1}\mB_{E}(k_x,k_y,z_p^-) - \mN_{H}(k_x,k_y,z_p^+,\hat\mR_{p+1}(k_x,k_y))^{-1}\mB_{H}(k_x,k_y,z_p^-),
      \end{align*}
      alors
      \begin{align*}
        \hat\mR_{p+1}(k_x,k_y) &= \mathfrak{T}_p(k_x,k_y,z_p^+,z_p^-,\hat\mR_{p}(k_x,k_y)),
        \\
        \hat\mR_{p}(k_x,k_y) &= \mathfrak{T}_p(k_x,k_y,z_p^-,z_p^+,\hat\mR_{p+1}(k_x,k_y)).
      \end{align*}
    \end{prop}

    On peut donc déterminer itérativement les matrices \(\hat\mR\) que l'on nomme matrice de réflexion.
    % \begin{REM}
    %   Il faut savoir quel nom tu donnes.
    % \end{REM}
    Dans notre cadre d'étude, la présence d'un conducteur parfait sur l'interface \(z=z_0^+\) implique \(\mR_{1}(k_x,k_y) = -\mB_E(k_x,k_y,z_0^+)^{-1}\mA_E(k_x,k_y,z_0^+)\).
    % \begin{REM}
    %   Il faut absolument que tu fasses la différence entre les calcul formels et les objets définis par les théorèmes.
    % \end{REM}
    % \begin{REP}
    %   Que veux-tu dire ?
    % \end{REP}

\subsection{Applications numériques}

  La figure \ref{fig:imp_fourier:plan:hoppe} permet de vérifier les résultats de \cite[p.~33]{hoppe_impedance_1995} pour une couche de matériau sans perte.
  Dans cet ouvrage, l'opérateur de Calderón considéré n'est pas le même ( voir annexe \ref{sec:annexe:hoppe} ) et on pose \(\hat{\mathfrak{Z}} = \hat{\mZ}\begin{bmatrix}0&1\\-1&0\end{bmatrix}\) la matrice d'impédance utilisée par Hoppe.

  On applique les hypothèses précédentes d'invariance par rotation, donc \(k_y=0\).
  On remarque que pour \(k_x\slash k_0=2\), \(k_3(k_x,0)\) est nul et l'impédance n'est pas définie en ce point. 

  Comme le matériau n'a pas de perte, la partie réelle de \(\hat{\mathfrak{\mZ}}\) est nulle et n'est donc pas tracée.
  \begin{figure}[!hbt]
      \centering
      \tikzsetnextfilename{Z_HOPPE_33_plan}
\begin{tikzpicture}[scale=1]
    \begin{axis}[
            title={},
            ylabel={\(\Im(\hat{\mathfrak{Z}}(k_x,0)\) (\(\Omega\))},
            xlabel={\(k_x\slash k_0\)},
            width=0.8\textwidth,
            xmin=0,
            xmax=2,
            mark repeat=20,
            legend pos=outer north east
        ]
        \addplot [black] table [col sep=comma, x={s1}, y={Im(z_ex.11)}] {csv/HOPPE_33/HOPPE_33.z_ex.MODE_2_TYPE_P.csv};
        \addlegendentry{TM};

        \addplot [black,dashed] table [col sep=comma, x={s1}, y={Im(z_ex.22)}] {csv/HOPPE_33/HOPPE_33.z_ex.MODE_2_TYPE_P.csv};
        \addlegendentry{TE};
    \end{axis}
\end{tikzpicture}
      \caption[Reproduction résultat Hoppe & Rahmat-Samii p.~33]{Partie imaginaire des coefficients diagonaux de \(\hat{\mathfrak Z}\) pour \(\eps = 4, \mu = 1, d=0.015\text{m}, f=1\text{GHz}\)}
      \label{fig:imp_fourier:plan:hoppe}
  \end{figure}

  La figure \ref{fig:imp_fourier:plan:soudais} permet de vérifier les résultats de \cite{soudais_3d_2017} pour une couche de matériau sans perte où \(k_3d = \frac{\pi}{2}\) pour \(k_x \simeq 0.94 k_0\).
  On a donc une asymptote bien visible sur le module des coefficients diagonaux de \(\hat\mZ\) pour \(k_x\slash k_0 \simeq 0.94\).
  \begin{figure}[!hbt]
      \centering
      \tikzsetnextfilename{Z_SOUDAIS_plan_large}
\begin{tikzpicture}[scale=1]
    \begin{axis}[
            title={},
            width=0.4\textwidth,
            xmin=0,
            xmax=1.8,
            ylabel={\(\Im(\hat{Z}(k_x,0))\)},
            xlabel={\(k_x\slash k_0\)},
            mark repeat=20,
            legend pos=outer north east
        ]
        \addplot [black] table [x={s1}, y={Im(z_ex.11)},col sep=comma] {csv/SOUDAIS/SOUDAIS.z_ex.MODE_2_TYPE_P.csv};
        \addplot [black,dashed] table [x={s1}, y={Im(z_ex.22)},col sep=comma] {csv/SOUDAIS/SOUDAIS.z_ex.MODE_2_TYPE_P.csv};
    \end{axis}
\end{tikzpicture}
\tikzsetnextfilename{Z_SOUDAIS_plan_zoom}
\begin{tikzpicture}[scale=1]
    \begin{axis}[
            title={},
            width=0.4\textwidth,
            ymin=-100,
            ymax=100,
            xmin=0.8,
            xmax=1,
            restrict y to domain=-200:200,                        
            ylabel={},
            xlabel={\(k_x\slash k_0\)},
            mark repeat=20,
            legend pos=outer north east
        ]
        \addplot [black] table [x={s1}, y={Im(z_ex.11)},col sep=comma] {csv/SOUDAIS/SOUDAIS.z_ex.MODE_2_TYPE_P.csv};
        \addlegendentry{TM};
        \addplot [black,dashed] table [x={s1}, y={Im(z_ex.22)},col sep=comma] {csv/SOUDAIS/SOUDAIS.z_ex.MODE_2_TYPE_P.csv};
        \addlegendentry{TE};
    \end{axis}
\end{tikzpicture}
      \caption[Reproduction résultat P. Soudais p.~11]{Partie imaginaire des coefficients diagonaux de \(\hat\mZ\) pour \(\eps = 4, \mu = 1, d=0.035\text{m}, f=12\text{GHz}\)}
      \label{fig:imp_fourier:plan:soudais}
  \end{figure}

  % Grâce à la figure \ref{fig:reflex_fourier:plan:soudais}, on remarque que la matrice de réflexion est parfaitement définie en ce point. De plus, cet empilement permet d'obtenir une onde guidée pour \(k_x\slash k_0 \simeq 1.4\) car le coefficient TM diverge en ce point.
  % \begin{figure}[!hbt]
  %     \centering
  %     \tikzsetnextfilename{R_SOUDAIS_plan}
\begin{tikzpicture}[scale=1]
    \begin{axis}[
            title={},
            width=0.8\textwidth,
            xmin=0,
            xmax=1.8,
            ymin=0,
            ymax=4,
            restrict y to domain=0:10,
            ylabel={\(|\hat{R}(k_x,0)|\)},
            xlabel={\(k_x\slash k_0\)},
            mark repeat=20,
            legend pos=outer north east
        ]
        \addplot [black] table [x={s1}, y={Abs(r_ex.11)},col sep=comma] {csv/SOUDAIS/SOUDAIS.r_ex.MODE_2_TYPE_P.csv};
        \addlegendentry{TM};
        \addplot [black,dashed] table [x={s1}, y={Abs(r_ex.22)},col sep=comma] {csv/SOUDAIS/SOUDAIS.r_ex.MODE_2_TYPE_P.csv};
        \addlegendentry{TE};
    \end{axis}
\end{tikzpicture}
  %     \caption[Reproduction résultat P. Soudais p.~11]{Module des coefficients diagonaux de \(\mR\) pour \(\eps = 4, \mu = 1, d=0.035\text{m}, f=12\text{GHz}\)}
  %     \label{fig:reflex_fourier:plan:soudais}
  % \end{figure}


\FloatBarrier
