\section{Étude de la CI4 par optimisation sous contraintes}

%Soit la condition d'impédance
%\begin{equation}\label{eq:ci4}
%\E_t = ( a_0 + a_1 \LD - a_2 \LR )(\J)
%\end{equation}
\newcommand{\st}{\ds\sum_{i=1}^n t_i}
\newcommand{\stc}{\ds\sum_{i=1}^n t_i^2 }
\newcommand{\mma}{\frac{\left(\st\right)^2 - 2n \stc}{\stc}}
\newcommand{\mmb}{\frac{\left(\st\right)^2}{\stc}}
Soit \(n>1\) un entier.
Soit \(t\) un vecteur réel négatif de taille \(n\), strictement décroissant. Soit \(D\) un réel.
Soit \(Z_{TE}, Z_{TM}\) deux vecteurs de complexe de taille \(n\) dont une composante représente l'impédance du plan infini pour l'onde d'incidence \(t_i\) pour la polar TE, respectivement. TM.
Soit \(M\) la matrice réelle \(3\times3\) suivante:
\[
M = 2\begin{bmatrix}
2n & \st & \st \\
\st & \stc & 0 \\
\st & 0 & \stc
\end{bmatrix}
\]
Soit \(Q\) la matrice réelle de taille \(6 \times 6\) suivante :
\[
Q = \begin{bmatrix}
M & 0_{3 \times 3} \\
0_{3 \times 3} & M
\end{bmatrix}
\]

Soit \(F = \begin{bmatrix}
-1 & 0 & 0 \\
0 & 1 & 0 \\
0 & 0 & 1 \\
0 & 0 & 0 \\
0 & 0 & 0 \\
0 & 0 & 0
\end{bmatrix}\) et soit \(P\) un vecteur de \(\RR^6\) tel que \( F^tP  \in (\RR_-)^3\).

On cherche \(X=(a_0', a_1', a_2',a_0'',a_1'',a_2'') \in \RR^6\) tel que
\begin{align}
X = \argmin{X\in \RR^6} \frac{1}{2}\left<QX,X\right> - \left<P,X\right> + D \text{ s.c } F^tX \in (\RR_-)^3
\end{align}

Cette fonctionnelle possède un unique minimum global car Q est inversible (démontré ci-après) et un unique minimum local à l'espace des contrainte car ce dernier est convexe et  \(Q\) est quadratique ( car correspondante à \(|| Z_{ap}(X)||_2^2\) ).

Pour résoudre ce problème nous allons appliquer la théorie des multiplicateurs de Karush-Khun-Lagrange-Tucker.

Le Lagrangien de ce problème est le suivant :
Soient \(\lambda \in (\RR_+)^3\)
\begin{equation}
\mathcal{L}(X,\lambda) = \frac{1}{2}\left<Q X, X\right>  - \left<P, X\right> + D + \left<\lambda, F^tX\right>
\end{equation}

Pour trouver le point-selle de ce lagrangien, nous allons appliquons le principe du min-max: on minimise par rapport à \(X\), puis on maximise par rapport à \(\lambda\).

On cherche la condition de Karush-Khun-Tucker de stationnarité \(\nabla_X \mathcal L (X,\lambda) = 0\).

\begin{equation}
\nabla_X \mathcal{L}(X,\lambda) = Q X - (P - F\lambda)
\end{equation}

% {\color{red}
% Soit \(F\) le vecteur réel de \(\RR^3\) tel que \(F = \begin{bmatrix} 1 \\ -1 \\ -1 \end{bmatrix}\) et \(\lambda\) le vecteur réel de \(\RR^3\) tel que \(\lambda = \begin{bmatrix} \lambda_1 \\ \lambda_2 \\ \lambda_3 \end{bmatrix}\).
% \begin{equation}
% \nabla_X \mathcal{L}(X,\lambda) = 2Q X - 2 P + \begin{bmatrix}\left<\lambda,F\right> \\ 0_{3\times1} \end{bmatrix}.
% \end{equation}
% }

\begin{prop}[Inversibilité de \(Q\)]~\\

Si \(t\) possède au moins deux composantes différentes alors
\begin{equation}
\det(Q) \not = 0
\end{equation}
\end{prop}

\begin{proof}
Par définition \( \det(Q) = \det(M)^2\). Or
  % \[
  % M = \begin{bmatrix}
  % 2n & \st & \st \\
  % \st & \stc & 0 \\
  % \st & 0 & \stc
  % \end{bmatrix}
  % \]
  \begin{align}
  \det(M) &= 2\left(\sum_{i=1}^n t_i^2\right)\left( n \sum_{i=1}^n t_i^2 - \left(\sum_{i=1}^n t_i\right)^2\right) \\
  &= 2\left<t,t\right>\left( \left<1,1\right> \left<t,t\right> - \left<t,1\right>^2\right)
  \end{align}
  et d'après Cauchy–Schwarz, le terme de droite est non-nul pour tout \(t\) non colinéaire au vecteur dont toutes les composantes valent 1, c'est à dire n'importe quel vecteur ayant au moins  deux composantes différentes.
% De plus, on note que

% \begin{equation}
%   M^{-1} = \frac{1}{\ddet(M)}
%   \begin{bmatrix}
%     \stc & -\st & -\st \\
%     -\st & \mma & \mmb \\
%     -\st & \mmb & \mma
%   \end{bmatrix}
% \end{equation}

et ainsi
\begin{equation}
Q^{-1} =
\begin{bmatrix}
M^{-1} & 0_{3 \times 3} \\
0_{3 \times 3} & M^{-1}
\end{bmatrix}
\end{equation}

\end{proof}
Puisque \(Q\) est inversible, on a la
\begin{prop}[Condition de stationnarité de K.K.T]
\begin{equation}
\nabla_X\mathcal{L}(X,\lambda) = 0 \Leftrightarrow X = Q^{-1}\left(P - F\lambda \right)
\end{equation}
\end{prop}


La fonction duale \(\mathcal D\) est alors le lagrangien \(\mathcal L\) en ce point \(X\)
\begin{align}
\mathcal D (\lambda) &= -\frac{1}{2} \left<P - F\lambda , Q^{-1}\left(P - F\lambda \right) \right > + D\\
&= -\frac{1}{2}\left<F^tQ^{-1}F\lambda,\lambda\right> + \left<F^tQ^{-1}P,\lambda\right> + D
\end{align}


On cherche alors à maximiser la fonction duale.
\begin{equation}
\nabla \mathcal D (\lambda) = -F^tQ^{-1}F\lambda + F^tQ^{-1}P
\end{equation}

\begin{prop}[Inversibilité de \(F^tQ^{-1}F\)]
  \(F^tQ^{-1}F\) est inversible.
\end{prop}
\begin{proof}
\(F^tQ^{-1}F = \tilde FM^{-1}\tilde F\) où \(\tilde F = \begin{bmatrix}
-1 & 0 & 0 \\
0 & 1 & 0 \\
0 & 0 & 1
\end{bmatrix}\).  Comme \(\tilde F\) et \(M\) sont inversibles, alors \(\tilde FM^{-1}\tilde F\) est inversible et \(F^tQ^{-1}F\) l'est aussi.
\end{proof}

Directement on peut trouver le point extremum de la fonction duale.
\begin{align}
\nabla \mathcal D (\lambda) = 0 \Leftrightarrow \lambda &= \left(F^tQ^{-1}F\right)^{-1}\left(F^tQ^{-1}P\right)\\
\intertext{Soit \(\tilde P\in (\RR)^3\) dont les composantes sont les 3 premières de \(P\).}
\lambda &  = \left(\tilde FM^{-1}\tilde F\right)^{-1}\left(\tilde FM^{-1}\tilde P\right) \\
& = \tilde F\tilde P\\
\intertext{Ce qui par définition de \(\tilde F\) et \(\tilde P\) vaut}
& = F^tP
\end{align}

La solution optimale est alors
\begin{align}
X^\star &= Q^{-1}\left(P- F\lambda\right)
\end{align}

On peut vérifier \(X\) vérifie les contraintes : \( F^t X = F^t Q^{-1} P - F^tQ^{-1}F\lambda = 0 \in (\RR_-)^3\).

\subsection{Problème des multiplicateurs nuls}


Le lecteur remarquera que par définition les multiplicateurs se doivent par définition être positif, ce dont le précédent résulat ne tient pas compte. La théorie des multiplicateurs de K.K.T énonce alors une condition supplémentaire.
Soit \(\lambda_i\) est positif et la contrainte associée est active, c'est à dire \(\left<F_{i,:},X\right> = 0\), soit la contrainte est inactive et alors le multiplicateur associé est nul.

Cependant comme \(Q\) n'est pas diagonale, en annulant un multiplicateur, les autres sont modifiés.  Il faut donc reprendre l'ensemble des calculs en supprimant la contrainte correspondante de \(F\). On a alors au plus 8 cas à traiter, où l'on considère toutes les combinaisons possibles de contraintes actives. La solution optimale est alors celle où les multiplicateurs sont tous positifs.

\begin{TODO}
  Montrer ces combinaisons
\end{TODO}


%  Il faut donc savoir quelles sont les contraintes qui seront actives avant de le calculer. On peut alors résoudre géometriquement cette question : l'ensemble des contraintes est une pyramide infini de sommet \((0)\) et d'axes \(F_1,F_2,F_3\). Un point appartient à cette pyramide si les composantes \((a,b,c)\) de ce point respectivement à ces 3 vecteurs sont positives, composantes qui doivent résoudre

% \[
% \begin{bmatrix}
%   |F_1|^2 & F_1 \cdot F_2 & F_1\cdot F_3 \\
%   F_1\cdot F_2 & |F_2|^2 & F_2\cdot F_3 \\
%   F_1\cdot F_3 & F_2\cdot F_3 & |F_3|^2
% \end{bmatrix}
% \begin{bmatrix}
% a\\b\\c
% \end{bmatrix} =
% \begin{bmatrix}
% X\cdot F_1\\
% X\cdot F_2\\
% X\cdot F_3\\
% \end{bmatrix}
% \]

% Si l'une des ces composantes est négative alors le point appartient à l'une des pyramides duales, c'est dire de sommet \((0)\)