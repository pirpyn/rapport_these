\section{Une condition nécessaire et suffisante pour l'unicité des solutions du problème intérieur}

Pour exprimer l'opérateur d'impédance, nous devons exprimer les champs dans l'objet. Il apparaît nécessaire que ces derniers soient donc bien définis.

Nous nous dotons d'un matériaux définis par couches, et allons donc exhiber une condition nécessaire et suffisante d'unicité des solutions du problème de Maxwell-Helmholtz  \eqref{eq:unicite:maxwell_int} dans \(\OO\) est un objet multi-couche creux c'est-a-dire tel que \(\partial \OO = \Gamma \cup \Gamma_0\) où \(\Gamma\) est la surface extérieure et \(\Gamma_0\) la surface intérieure.

\begin{align}
\left\lbrace
  \begin{matrix}
    \vrot \vE(\vx) + i k(\vx)\eta(\vx) \vH(\vx) &= 0
    \\
    \vrot \vH(\vx) - i k(\vx)(\eta(\vx))^{-1} \vE(\vx) &= 0
  \end{matrix}
  \right. && \text{dans \(\OO\).}
  \label{eq:unicite:maxwell_int}
\end{align}

\begin{prop}
  Le problème Maxwell-Helmholtz \eqref{eq:unicite:maxwell_int} avec conditions aux limites de Dirichlet \(\vE_{|\Gamma} = 0\), \(\vE_{|\Gamma_0} = 0\) admet une unique solution si est seulement si le déterminant d'un système linéaire est non-nul.
\end{prop}

\begin{proof}
  Prenons l'exemple de deux couches de matériaux.
  Nous supposons qu'il existe un champ \(\vE\) solution du problème de Maxwell-Helmholtz \eqref{eq:unicite:maxwell_int} avec conditions limite de Dirichlet sur les bords du domaine. La démonstration de l'existence est l'objet de la partie suivante.

  \newcommand{\kk}{\tilde{k}}

  On pose \(\kk_1 = \sqrt{k_1^2 - k_x^2 - k_y^2}\),  \(\kk_2 = \sqrt{k_2^2 - k_x^2 - k_y^2}\).
  \begin{align*}
    \left\lbrace
    \begin{aligned}
      \hat{E}_x &= \sin(\kk_2(z-z_1))a_2 +  \cos(\kk_2(z-z_1))b
      \\
      \hat{E}_y &= \sin(\kk_2(z-z_1))c_2 +  \cos(\kk_2(z-z_1))d
    \end{aligned}
    \right. && z_1 < z < z_2
    \\
    \left\lbrace
    \begin{aligned}
      \hat{E}_x &= \sin(\kk_1(z-z_1))a_1 +  \cos(\kk_1(z-z_1))b
      \\
      \hat{E}_y &= \sin(\kk_1(z-z_1))c_1 +  \cos(\kk_21z-z_1))d
    \end{aligned}
    \right. && z_0 < z < z_1
  \end{align*}
  La composante en z de \(\vE\) se déduit des deux autres car sa divergence est nul.

  On déduit le champ \(\vH\) alors \(\vH(\vx) = \frac{-\vrot{\vE}(\vx)}{i k(\vx)\eta(\vx)}\). On note \(\mLR\) la matrice \(
  \begin{pmatrix}
      k_y^2 & k_xk_y
      \\
      k_xk_y & k_x^2
  \end{pmatrix}\).

  \begin{align*}
    \hat{\vH} &= \frac{\sin(\kk_2(z-z_1))}{-i k_2\kk_2 \eta_2}(\kk_2^2 \mI - \mLR)\begin{bmatrix}d \\ b\end{bmatrix} - \frac{\cos(\kk_2(z-z_1))}{-i \kk_2\kk_2 \eta_2}(k_2^2 \mI - \mLR)\begin{bmatrix}c_2 \\ a_2\end{bmatrix} && \text{pour } z_1 < z < z_2
    \\
    \hat{\vH} &= \frac{\sin(\kk_1(z-z_1))}{-i k_1\kk_1 \eta_1}(\kk_1^2 \mI - \mLR)\begin{bmatrix}d \\ b\end{bmatrix} - \frac{\cos(\kk_1(z-z_1))}{-i \kk_1\kk_1 \eta_1}(k_1^2 \mI - \mLR)\begin{bmatrix}c_2 \\ a_2\end{bmatrix} && \text{pour } z_0 < z < z_1
  \end{align*}

  Pour que le problème soit bien posé, il faut que les constantes complexes \(a_2,c_2,b,d,a_1,c_1\) soient déterminées de façon unique quand on exprime les conditions limites. Ce sont des conditions limites de Dirichlet en \(z=z_0\) et \(z=z_2\) pour \(\vE\) et de sauts nuls en \(z=z_1\) pour \(\vH\) (la condition de saut nul pour \(\vE\) a déjà été utilisé dans l'expression des champs).

  On a unicité si dans le cas de conditions de Dirichlet homogènes, l'unique solution est \(\vE=0\). Ces conditions s'expriment
  \begin{align*}
    a_2\sin{(\kk_2(z_2-z_1))} + b\cos{(\kk_2(z_2-z_1))} &= 0
    \\
    c_2\sin{(\kk_2(z_2-z_1))} + d\cos{(\kk_2(z_2-z_1))} &= 0
    \\
    a_1\sin{(\kk_1(z_0-z_1))} + b\cos{(\kk_1(z_0-z_1))} &= 0
    \\
    c_1\sin{(\kk_1(z_0-z_1))} + d\cos{(\kk_1(z_0-z_1))} &= 0
  \end{align*}

  C'est un système à 4 équations pour 6 inconnus, donc à deux degrés de liberté. Il existe donc \(A,B\) tels que
  \begin{align*}
    a_2 &= A\cos{(\kk_2(z_2-z_1))}\sin{(\kk_1(z_0-z_1))}
    \\
    b &= -A\sin{(\kk_2(z_2-z_1))}\sin{(\kk_1(z_0-z_1))}
    \\
    a_1 &= A\sin{(\kk_2(z_2-z_1))}\cos{(\kk_1(z_0-z_1))}
    \\
    c_2 &= B\cos{(\kk_2(z_2-z_1))}\sin{(\kk_1(z_0-z_1))}
    \\
    d &= -B\sin{(\kk_2(z_2-z_1))}\sin{(\kk_1(z_0-z_1))}
    \\
    c_1 &= B\sin{(\kk_2(z_2-z_1))}\cos{(\kk_1(z_0-z_1))}
  \end{align*}
  soient solutions du système précédent.

  La dernière condition limite s'exprime alors
  \begin{align*}
    \frac{i}{k_2\kk_2\eta_2}(\kk_2 \mI - \mLR)\begin{bmatrix}A\\B\end{bmatrix} \cos(\kk_2(z_2-z_1)) \sin(\kk_1(z_0-z_1))
    \\= 
    \frac{i}{k_1\kk_1\eta_1}(\kk_1 \mI - \mLR)\begin{bmatrix}A\\B\end{bmatrix} \sin(\kk_2(z_2-z_1)) \cos(\kk_1(z_0-z_1))
  \end{align*}

  On voit alors apparaître le système final. Si le déterminant de ce système n'est pas nul, alors il existe des champs non nuls solution du problème homogène, et alors il n'y pas unicité des solutions

\end{proof}