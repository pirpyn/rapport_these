\section{Analyse de Fourier de l'impédance}

Des résultats sur l'analyse spectrale de l'opérateur d'impédance ont déjà été présentés par \cite{hoppe_impedance_1995} pour des conducteurs plans et cylindriques recouverts d'une couche de matériau.
Nous rappellerons ainsi la méthode pour les obtenir puis nous montrerons qu’étendre ces résultats à plusieurs couches est immédiat. 

Soit \(\OO\) un domaine fermé, borné, de frontière régulière. Soit \(\vect{e_1},\vect{e_2},\vect{e_n}\) un repère local adapté à la surface de \(\OO\).

Supposons qu'il existe des champs \((\vE,\vect H)\) qui vérifient les équations de Maxwell:
\begin{equation*}
    \left\lbrace
    \begin{matrix}
    \vrot \vE(\vect{x},t) &=& -\mu(\vx) \ddr{t}{\vect H}(\vx,t),
    \\
    \vrot \vect H(\vect{x},t) &=& \eps(\vx) \ddr{t}{\vE}(\vx,t).
    \end{matrix}
    \right.
\end{equation*}

On suppose aussi que \(\eps\) et \(\mu\) sont constants par couche: \(\eps(x_1,x_2,x_n)=\eps(x_n)\).

On suppose enfin que ces champs \(\vE(\vect{x},t),\vect H(\vect{x},t)\) ont des transformées de Fourier partielles suivant les coordonnées tangentielles et le temps
%(\(\hat{\vE}(k_1,k_2,x_n,\w), \hat{\vect H}(k_1,k_2,x_n,\w)\))
(\cite[Théorème de Plancherel, p.~153]{yosida_functional_1995}) telles que

\begin{align*}
    \hat{\vE} (k_1,k_2,x_n,\w) &= \frac{1}{2\pi}\int_{\RR^2\times\RR_+} e^{-i(k_1x_1+k_2x_2+\w t)}\vE(x_1,x_2,x_n,t) \dd{x_1}\dd{x_2}\dd{t},
    \\
    \vE(x_1,x_2,x_n,t) &= \frac{1}{2\pi}\int_{\RR^2\times\RR_+} e^{i(k_1x_1+k_2x_2+\w t)}\hat{\vE} (k_1,k_2,x_n,\w) \dd{k_1}\dd{k_2} \dd{\w}.
\end{align*}

Pour avoir le droit d'appliquer la transformée de Fourier, nous supposons que pour notre empilement nous n'avons pas de résonances tel que nous l'avions évoqué à la propriété \eqref{prop:unicite:interieur:postulat:multi-couche}. 

Dans cette thèse, on utilise les équations harmoniques de Maxwell, ainsi on réalise au moins une transformée partielle en temps. La variable de Fourier associée est \(\w\), et donc l'opérateur \(\ddr{t}{~}\) est remplacé par \(i\gls{phy-w}\).

On va donc utiliser le système d'équations de Maxwell harmonique:
\begin{equation}
    \left\lbrace
    \begin{matrix}
    \vrot \hat{\gls{phy-e}}(k_1,k_2,x_n,\w)  &=& -i \omega \mu(x_n) \hat{\gls{phy-h}}(k_1,k_2,x_n,\w),
    \\
    \vrot \hat{\gls{phy-h}}(k_1,k_2,x_n,\w)  &=& i \omega \eps(x_n) \hat{\gls{phy-e}}(k_1,k_2,x_n,\w).
    \end{matrix}
    \right.
    \label{eq:imp_fourier:intro:maxwell_harmonique}
\end{equation}

Afin d'utiliser les mêmes conventions que \cite{stupfel_implementation_2015}, on substitue au champ magnétique \gls{phy-h} l’excitation magnétique \gls{phy-h2}. Les équations de Maxwell harmoniques sont en définitive
\begin{equation}
    \left\lbrace
    \begin{matrix}
    \vrot \hat{\gls{phy-e}}(k_1,k_2,x_n,\w)  &=& -i k(x_n) \eta_r(x_n) \hat{\gls{phy-h2}}(k_1,k_2,x_n,\w),  \\
    \vrot \hat{\gls{phy-h2}}(k_1,k_2,x_n,\w)  &=& i k(x_n) \eta_r^{-1}(x_n) \hat{\gls{phy-e}}(k_1,k_2,x_n,\w),
    \end{matrix}
    \right.
    \label{eq:imp_fourier:intro:maxwell_harmonique:form_cea}
\end{equation}
où \(\gls{phy-h2} = \gls{phy-eta0} \gls{phy-h}\), \(\eps(x_n)=\gls{phy-eps}_r(x_n)\gls{phy-eps0}\), \(\mu(x_n)=\gls{phy-mu}_r(x_n)\gls{phy-mu0}\), \(\eta_r(x_n)=\sqrt{\mu_r(x_n)/\eps_r(x_n)}\), \(\gls{phy-k}=\gls{phy-k0}\sqrt{\gls{phy-mu}_r(x_n)\gls{phy-eps}_r(x_n)}\), \(\gls{phy-k0} = \w\sqrt{\mu_0\eps_0}\) tel qu'en tout point extérieur de \(\OO\), \(\gls{phy-eps}_r(x_n) = \gls{phy-mu}_r(x_n)= 1\).

Notre but est de prouver l'existence d'une matrice \(\hat\mZ(k_1,k_2,\w)\) telle que
\begin{equation*}
    \hat{\vE}_t(k_1,k_2,0,\w) =  \hat{\mZ}(k_1,k_2,\w) \left(\vn \pvect \hat{\vH}(k_1,k_2,0,\w) \right).
\end{equation*}

La méthode pour trouver une expression cette dernière est la suivante.
\begin{itemize}
    \item Faire une transformée de Fourier partielle des champs, dépendante de la géométrie.
    \item Réécrire le système d'équations de Maxwell simplifié.
    \item Obtenir des EDO simples à résoudre sur les composantes des champs.
    \item En déduire l'expression générale des transformées partielles de Fourier.
    \item Utiliser les conditions limites pour obtenir les solutions particulières.
    \item En déduire la matrice d'impédance en Fourier.
\end{itemize}

À partir de maintenant, la dépendance en \(\w\) sera implicite.

