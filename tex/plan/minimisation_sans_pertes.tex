\subsection{Problème de la singularité de l'impédance dans le cadre du plan infini pour une couche de matériau sans perte}

Dans la thèse de \cite{aubakirov_electromagnetic_2014}, les constantes de la couche de matériau \(\eps=4,\mu=1,f=12\) GHz, et \(d=3.5\) mm. 
%Ce cas non physique possède un onde guidée pour \((k_x,k_y) = (k_0s^\star,0.)\) où \(\mR(k_0s^\star,0.) = \infty\).

Il existe un \(s_z\) tel que \(\hat\mZ_{ex}(k_0s_z,0.) = \infty\) si \(\eps\mu\in\RR\).
En effet, d'après la formule pour une couche de matériaux de la définition \ref{def:plan:impedance:1c}, 
% \begin{REM}
%   Dis dans le texte (et non seulement dans le titre) que ce n'est possible que si \(\eps\mu\in\RR\).
% \end{REM}
\begin{equation}
  \hat{\mZ}_{ex}(k_x,0.) = i\frac{\eta}{k\sqrt{k^2 - k_x^2}}\tan(\sqrt{k^2 - k_x^2}d)\left(k^2\mI - \hat{\mLR}\right).
\end{equation}
Ainsi, il est facile de voir que l'on a une asymptote à cause de la tangente, et donc pour cet empilement
\begin{equation}
  \label{eq:plan:asymptote_1_couche}
  s_z = \sqrt{\eps \mu - \left(\frac{\pi/2}{k_0 d}\right)^2}.
\end{equation}
% \begin{REM}[asymptote]
%   Et voilà l'explication de \hyperlink{REMnoasymptote}{la precedente remarque}.
% \end{REM}
Le problème est donc que si nous balayons en incidence et que l'on passe par ce point, alors la matrice \(\hat\mZ\) n'est pas définie en ce point.
Or comme le gradient de la fonction est fonction de cette matrice, le gradient n'est pas défini pour tout \(X\).
Si l'on utilise une méthode basée sur le gradient de type Newton, ce que nous avons fait, on comprend pourquoi la méthode numérique échoue à calculer des coefficients.

Dans le code numérique, la discrétisation en incidence est linéaire et nous passions par ce point.
En discrétisant différemment l'incidence, peut-être aurions-nous pu éviter cette difficulté.
\begin{REM}
et une autre méthode de minimisation
\end{REM} 
\begin{REP}
  Question à prévoir.
\end{REP}
\subsubsection{Réduction du nombre de variables de la minimisation}

On décompose alors nos matrices et vecteurs en séparant les parties contentant cette asymptote.

On suppose donc qu'il existe \({\mH}_\infty, {b}_\infty, X_\infty,\) tels que
\begin{align*}
  {\mH} &= {\mH}_\infty + {\mH}_r,
  \\
  {b} &= {b}_\infty + {b}_r,
  \\
  X &= X_\infty + X_r.
\end{align*}

Ces matrices et vecteurs sont reliés par les relations
\begin{align}
  {\mH}_\infty X_\infty &= {b}_\infty,
  \\
  {\mH}_\infty X_r &= 0.
\end{align}

Il faut voir cette décomposition comme deux parties, où l'une est nulle quasiment partout sauf pour le \(s_z\) problématique, et l'autre est définie normalement sauf aux termes correspondant au \(s_z\) où elle est nulle.

Schématiquement, on définit \(i_z\) l'indice d'une ligne telle que \({b}(\hat{\mZ})_{i_z} = \infty\), 

\begin{equation*}
  \begin{matrix}
    {\mH} &=& {\mH}_\infty &+& {\mH}_r,
    \\
    \begin{bmatrix}
      {\mH}_{1,1} & \cdots & {\mH}_{1,N_{CI}}
      \\
      \vdots & \ddots & \vdots
      \\
      {\mH}_{i_z-1,1} & \cdots & {\mH}_{i_z-1,N_{CI}}
      \\
      {\mH}_{i_z,1} & \cdots & {\mH}_{i_z,N_{CI}}
      \\
      {\mH}_{i_z+1,1} & \cdots & {\mH}_{i_z+1,N_{CI}}
      \\
      \vdots & \ddots & \vdots
      \\
      {\mH}_{4N_i,1} & \cdots & {\mH}_{4N_i,N_{CI}}
    \end{bmatrix}
    & = &
    \begin{bmatrix}
      0 & \cdots & 0
      \\
      \vdots & \ddots & \vdots
      \\
      0 & \cdots & 0
      \\
      {\mH}_{i_z,1} & \cdots & {\mH}_{i_z,N_{CI}}
      \\
      0 & \cdots & 0
      \\
      \vdots & \ddots & \vdots
      \\
      0 & \cdots & 0
    \end{bmatrix}
    & + & 
    \begin{bmatrix}
      {\mH}_{1,1} & \cdots & {\mH}_{1,N_{CI}}
      \\
      \vdots & \ddots & \vdots
      \\
      {\mH}_{i_z-1,1} & \cdots & {\mH}_{i_z-1,N_{CI}}
      \\
      0 & \cdots & 0
      \\
      {\mH}_{i_z+1,1} & \cdots & {\mH}_{i_z+1,N_{CI}}
      \\
      \vdots & \ddots & \vdots
      \\
      {\mH}_{4N_i,1} & \cdots & {\mH}_{4N_i,N_{CI}}
    \end{bmatrix}
  \end{matrix}.
\end{equation*}
Dans les faits, \({\mH}_\infty\) a \(4\) lignes non nulles et \({\mH}_r\) en a autant de nulles aux mêmes endroits.


On développe donc la fonction
% \begin{REM}
%   Voir \hyperlink{REMfonction}{avant}.
%   De plus, il est préférable de l'écrire sur le carré de la norme.
% \end{REM}
% \begin{REP}
%   Fait
% \end{REP}
suivant cette décomposition :
\begin{align*}
\argmin{X}\left\rVert {\mH} X - {b} \right \rVert^2 &= \argmin{X_r}\left\rVert \left({\mH}_\infty + {\mH}_r\right)\left( X_\infty + X_r \right) - {b}_\infty - {b}_r \right \rVert^2.
\\
\intertext{On utilise la relation entre \({\mH}_\infty X_\infty\) et \({b}_\infty\)}
&=  \argmin {X_r}\left\rVert {\mH}_\infty X_r + {\mH}_r X_\infty + {\mH}_r X_r - {b}_r \right \rVert^2.
\\
\intertext{Enfin par définition de \({\mH}_\infty\) et \(X_r\), leur produit est nul}
&= \argmin{X_r} \left\rVert {\mH}_r ( X_r + X_\infty)- {b}_r \right \rVert^2.
\end{align*}

On voit alors que l'on peut résoudre le problème
% \begin{REM}
%   Ce n'est pas rigoureux
% \end{REM}
% \begin{REP}
%   Pourtant c'est issus du travail à Montréal.
% \end{REP}
si l'on minimise uniquement sur \(X_r\), les autres étant fixés, et si l'on enlève du système les lignes où l'impédance n'est pas définie.

\subsubsection{Application de la réduction à la CI3}

On rappelle l'expression de la CI3
\begin{equation*}
  \hat{\mZ}_{ap}(k_x,0) = \left(\mI + b_1 \hat{\mLD}(k_x,0) - b_2 \hat{\mLR}(k_x,0) \right)^{-1}\left(a_0\mI + a_1 \hat{\mLD}(k_x,0) - a_2 \hat{\mLR}(k_x,0) \right).
\end{equation*}

% On voit donc que pour faire apparaître une asymptote,
% \begin{REM}
%   Pas math. "Une condition nécessaire et suffisante pour que \(\mZ\) ai une asymptote est".
%   Elle est peut être seulement nécessaire car si \(a_0\mI + a_1\mLD -a_2\mLR\) a un noyau
% \end{REM}
% \begin{REP}
%   FAIT
% \end{REP}
Une condition nécessaire et suffisante pour que \(\hat{\mZ}_{ap}\) ait une asymptote est que la matrice de gauche ne soit pas inversible en \(s_z\).

La matrice \({\mH}_\infty\) est donc nulle partout sauf en 8 termes, placés sur les deux dernières colonnes et les 4 lignes correspondantes à \((k_x,k_y)=(k_0 s_z,0)\).

Connaissant les expressions des matrices \(\hat\mLD,\hat\mLR\) introduites dans la partie précédente alors on déduit que
\begin{align*}
  X_\infty = \begin{bmatrix}
    0\\
    0\\
    0\\
    (k_0 s_z)^{-2}\\
    (k_0 s_z)^{-2}\\
  \end{bmatrix},
  & &
  X_r = \begin{bmatrix}
  a_0\\
  a_1\\
  a_2\\
  0\\
  0\\
  \end{bmatrix}.
\end{align*}

\subsection{Choix de la méthode numérique pour résoudre la minimisation sous contraintes}

  Des méthodes basées sur le gradient sont adaptées, car la fonction est dérivable pour tout \(X\) et les contraintes se comportent comme des polynômes dépendant uniquement des composantes de \(X\).
  Nous avons donc fait le choix de la méthode \gls{acr-sqp} pour les raisons suivantes:
 
  \begin{itemize}
    \item elle est éprouvée depuis \cite{kraft_software_1988} et des sources Fortran à jour sont disponibles à \url{https://github.com/jacobwilliams/slsqp}, ce qui est capital pour l'intégrer dans un code industriel ;
    \item elle est rapide, nous avons observé que cette méthode convergeait en quelques dizaines d'itérations au plus ;
    \item elle accepte des contraintes non linéaires, donc elle est adaptée à nos CSU.
  \end{itemize}

  \subsection{Résultats numérique avec contraintes}

  La figure \ref{fig:imp_fourier:plan:stupfel:hoibc_vs_csu} montre un empilement où les coefficients déterminés sans contraintes ne vérifient pas la \CSU[3]{CI3} de la définition \ref{def:csu:ci3-3}, et où on peut observer une différence entre l'opérateur de Calderón et son approximation avec les coefficients calculés en vérifiant cette CSU.

  \begin{figure}[!hbt]
    \centering
    \tikzsetnextfilename{STUPFEL_ABS_Z_CALDERON_CIOE.TM}
\begin{tikzpicture}[scale=1]
    \begin{axis}[
            title={Polarisation TM},
            ylabel={\(\abs{\hat\mZ(k_x,0)}\)},
            xlabel={\(k_x / k_0 \)},
            width=0.4\textwidth,
            xmin=0,
            xmax=1.8,
            ymin=0.14,
            ymax=0.34,
            restrict y to domain=0:0.5,
            mark repeat=20,
            legend pos=outer north east
        ]
        \addplot [color=black,mark=square*] table [col sep=comma, x={kx/k0}, y={Exact}] {csv/STUPFEL/STUPFEL.abs_z.calderon_cioe.TM.csv};
        \addplot [color=blue,mark=x] table [col sep=comma, x={kx/k0}, y={CI0}] {csv/STUPFEL/STUPFEL.abs_z.calderon_cioe.TM.csv};
        \addplot [color=red,mark=diamond*] table [col sep=comma, x={kx/k0}, y={CI3}] {csv/STUPFEL/STUPFEL.abs_z.calderon_cioe.TM.csv};
        \addplot [color=Gray,mark=+] table [col sep=comma, x={kx/k0}, y={CI3+CSU}] {csv/STUPFEL/STUPFEL.abs_z.calderon_cioe.TM.csv};
    \end{axis}
\end{tikzpicture}

\tikzsetnextfilename{STUPFEL_ABS_Z_CALDERON_CIOE.TE}
\begin{tikzpicture}[scale=1]
    \begin{axis}[
            title={Polarisation TE},
            ylabel={},
            xlabel={\(k_x / k_0 \)},
            width=0.4\textwidth,
            xmin=0,
            xmax=1.8,
            ymin=0.203,
            ymax=0.214,
            restrict y to domain=0.1:0.3,
            mark repeat=20,
            legend pos=outer north east,
            y tick label style={
                /pgf/number format/.cd,
                    fixed,
                    fixed zerofill,
                    precision=3,
                /tikz/.cd
            },
        ]
        \addplot [color=black,mark=square*] table [col sep=comma, x={kx/k0}, y={Exact}] {csv/STUPFEL/STUPFEL.abs_z.calderon_cioe.TE.csv};
        \addlegendentry{Exact};
        \addplot [color=blue,mark=x] table [col sep=comma, x={kx/k0}, y={CI0}] {csv/STUPFEL/STUPFEL.abs_z.calderon_cioe.TE.csv};
        \addlegendentry{CI0};
        \addplot [color=red,mark=diamond*] table [col sep=comma, x={kx/k0}, y={CI3}] {csv/STUPFEL/STUPFEL.abs_z.calderon_cioe.TE.csv};
        \addlegendentry{CI3};
        \addplot [color=Gray,mark=+] table [col sep=comma, x={kx/k0}, y={CI3+CSU}] {csv/STUPFEL/STUPFEL.abs_z.calderon_cioe.TE.csv};
        \addlegendentry{CI3\textsubscript{CSU}};
    \end{axis}
\end{tikzpicture}

    \caption[Module de la matrice d'impédance pour le matériau M1, perte de précision à cause de la CSU]{ Module des coefficients diagnonaux de \(\hat\mZ\) pour \(\eps = 1-i\), \(\mu = 1\), \(d=5\)cm, \(f=200\)MHz.}
    \label{fig:imp_fourier:plan:stupfel:hoibc_vs_csu}
  \end{figure}

  \begin{TODO}
    Mettre coeffs figure \ref{fig:imp_fourier:plan:stupfel:hoibc_vs_csu}
  \end{TODO}

  \FloatBarrier

  La figure \ref{fig:imp_fourier:plan:soudais:hoibc_csu} montre l'effet des matériaux sans perte sur la minimisation.
  Nous complétons la figure \ref{fig:imp_fourier:plan:soudais:hoibc} où nous avions déjà observé la précision de la CI3, dont les coefficients étaient alors calculés sans contraintes.
  L'on observe bien que si l'on n'adapte pas la minimsation pour le cas de ce matériau sans perte en fixant les coefficients \(b_1\) et \(b_2\) (CI3\textsupsub{*}{CSU} sur le graphe), la matrice d'impédance approchée n'est pas du tout une bonne approximation de l'opérateur de Calderón.

  \begin{figure}[!hbt]
    \centering
    \tikzsetnextfilename{Z_SOUDAIS_plan_hoibc.TM}
\begin{tikzpicture}[scale=1]
    \begin{axis}[
            title={Polarisation TM},
            ylabel={\(\Im(\hat{Z}(k_x,0)\)},
            xlabel={\(k_x\slash k_0\)},
            width=0.4\textwidth,
            xmin=0,
            xmax=1.8,
            ymin=-1E+1,
            ymax=1E+1,
            restrict y to domain=-30:30,
            mark repeat=20,
            legend pos=outer north east
        ]
        \addplot [color=black,mark=square*] table [col sep=comma, x={s1}, y={Im(z_ex.11)}] {csv/SOUDAIS/SOUDAIS.z_ex.MODE_2_TYPE_P.csv};
        % \addlegendentry{Exact};

        \addplot [color=blue, mark=x] table [col sep=comma, x={s1}, y={Im(z_ibc0.11)}] {csv/SOUDAIS/SOUDAIS.z_ibc.IBC_ibc0_SUC_F_MODE_2_TYPE_P.csv};
        % \addlegendentry{CI0};

        \addplot [color=red, mark=diamond*] table [col sep=comma, x={s1}, y={Im(z_ibc3.11)}] {csv/SOUDAIS/SOUDAIS.z_ibc.IBC_ibc3_SUC_F_MODE_2_TYPE_P.csv};
        % \addlegendentry{CI3};

        \addplot [color=Gray, dashed, mark=+] table [col sep=comma, x={kx/k0}, y={CI3+CSU}]  {csv/SOUDAIS/SOUDAIS.Im_z.exact_ibc.TM.csv};
        % \addlegendentry{CI3\textsubscript{CSU}};

        \addplot [color=GreenYellow, dotted, mark=+] table [col sep=comma, x={kx/k0}, y={CI3*+CSU}]  {csv/SOUDAIS/SOUDAIS.Im_z.exact_ibc.TM.csv};
        % \addlegendentry{CI3\textsupsub{CSU}{*}};

    \end{axis}
\end{tikzpicture}
\tikzsetnextfilename{Z_SOUDAIS_plan_hoibc.TE}
\begin{tikzpicture}[scale=1]
    \begin{axis}[
            title={Polarisation TE},
            ylabel={},
            xlabel={\(k_x\slash k_0\)},
            width=0.4\textwidth,
            xmin=0,
            xmax=1.8,
            ymin=-1E+1,
            ymax=1E+1,
            restrict y to domain=-30:30,
            mark repeat=20,
            legend pos=outer north east
        ]
        \addplot [color=black,mark=square*] table [col sep=comma, x={s1}, y={Im(z_ex.22)}] {csv/SOUDAIS/SOUDAIS.z_ex.MODE_2_TYPE_P.csv};
        \addlegendentry{Exact};

        \addplot [color=blue, mark=x] table [col sep=comma, x={s1}, y={Im(z_ibc0.22)}] {csv/SOUDAIS/SOUDAIS.z_ibc.IBC_ibc0_SUC_F_MODE_2_TYPE_P.csv};
        \addlegendentry{CI0};

        \addplot [color=red, mark=diamond*] table [col sep=comma, x={s1}, y={Im(z_ibc3.22)}] {csv/SOUDAIS/SOUDAIS.z_ibc.IBC_ibc3_SUC_F_MODE_2_TYPE_P.csv};
        \addlegendentry{CI3};

        \addplot [color=Gray, dashed, mark=+] table [col sep=comma, x={kx/k0}, y={CI3+CSU}]  {csv/SOUDAIS/SOUDAIS.Im_z.exact_ibc.TE.csv};
        \addlegendentry{CI3\textsubscript{CSU}};

        \addplot [color=GreenYellow, dotted, mark=+] table [col sep=comma, x={kx/k0}, y={CI3*+CSU}]  {csv/SOUDAIS/SOUDAIS.Im_z.exact_ibc.TE.csv};
        \addlegendentry{CI3\textsupsub{*}{CSU}};

    \end{axis}
\end{tikzpicture}

    \caption[Partie imaginaire de l'opérateur de Calderón, comparé avec les approximations CI0, CI3, CI3 avec CSU, CI3 avec CSU adaptée pour une couche plane de matériau sans pertes de P.~Soudais]{Partie imaginaire des coefficients diagonaux de \(\hat\mZ\) pour \(\eps = 4, \mu = 1, d=0.035\text{m}, f=12\text{GHz}\)}
    \label{fig:imp_fourier:plan:soudais:hoibc_csu}
  \end{figure}

  \begin{table}[!hbt]
    \centering
    \tablecoeff[0.49]{\hyperlink{ci0}{CI0}}{csv/SOUDAIS/SOUDAIS.IBC_ibc0_SUC_F_MODE_2_TYPE_P.coeff.txt}
    \tablecoeff[0.49]{\hyperlink{ci3}{CI3}}{csv/SOUDAIS/SOUDAIS.IBC_ibc3_SUC_F_MODE_2_TYPE_P.coeff.txt}
    \begin{TODO}
      Ajouter coeff CI3+CSU, CI3*+CSU
    \end{TODO}
    \caption{Coefficients associés à la figure \ref{fig:imp_fourier:plan:soudais:hoibc_csu}}
    \label{tab:imp_fourier:plan:soudais:hoibc_csu}
  \end{table}

  \FloatBarrier