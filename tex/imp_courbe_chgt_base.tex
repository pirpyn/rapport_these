\section{Matrice de changement de base}
On se base sur les travaux de \cite{lafitte_diffraction_1998}. On cherche la matrice \(g=\left(M^tM\right)^{-1}\) où M est la matrice Jacobienne du changement de base.
On note \(X_i\) les coordonnées dans le repère cartésien et \(x_i\) dans le repère local.

\begin{TODO}
  Corriger le transpose en 1er. Changer \(x_2,x_2\) en \(r,\theta\)
\end{TODO}

\subsection{Cylindre}
Quitte à faire une rotation, on peut toujours définir le cylindre d'axe \(X_3\)
\begin{align*}
X_1 &= x_1 \cos(x_2) \\
X_2 &= x_1 \sin(x_2) \\
X_3 &= x_3
\end{align*}

\[
M = \begin{bmatrix}
\cos(x_2) & \sin(x_2) & 0 \\
-x_1\sin(x_2) & x_1\cos(x_2) & 0 \\
0 & 0 & 1
\end{bmatrix} \quad |M| = x_1
\]
\[
M^tM =
\begin{bmatrix}
\cos(x_2)^2 + x_1^2 \sin(x_2)^2 & \cos(x_2)\sin(x_2)(1-x_1^2) & 0 \\
\cos(x_2)\sin(x_2)(1-x_1^2) & \sin(x_2)^2 + x_1^2\cos(x_2)^2) & 0 \\
0 & 0 & 1
\end{bmatrix}
\]
\[
g = \frac{1}{x_1^2}
\begin{bmatrix}
\sin(x_2)^2 + x_1^2\cos(x_2)^2) & -\cos(x_2)\sin(x_2)(1-x_1^2) & 0 \\
-\cos(x_2)\sin(x_2)(1-x_1^2) & \cos(x_2)^2 + x_1^2 \sin(x_2)^2& 0 \\
0 & 0 & 1
\end{bmatrix}
\]
\subsection{Sphère}
\begin{align*}
X_1 &= x_1 \cos(x_2)\cos(x_3) \\
X_2 &= x_1 \sin(x_2)\cos(x_3) \\
X_3 &= x_1 \sin(x_3)
\end{align*}

\[
M = \begin{bmatrix}
\cos(x_2)\cos(x_3) & \sin(x_2)\cos(x_3) & \sin(x_3) \\
-x_1\sin(x_2)\cos(x_3) & x_1\cos(x_2)\cos(x_3) & 0 \\
-x_1 \cos(x_2)\sin(x_3) & -x_1 \cos(x_2)\sin(x_3) & x_1\cos(x_3)
\end{bmatrix} \quad |M| = x_1 \sin(x_3)
\]
\[
M^tM =
\begin{bmatrix}

\end{bmatrix}
\]
\[
g =
\begin{bmatrix}

\end{bmatrix}
\]