\section{Solution exacte de Maxwell sur une sphère: la série de Mie}\label{sec:serie_mie}
%TODO

\TODO{refaire cette partie et expliquer la somme finale sur les champs}

Maintenant que nous connaissons l'ensemble des solutions de Maxwell ainsi que la manière s'approcher la condition d'impédance, nous allons expliciter les coefficients $(a_n,b_n)$  afin d'assurer le condition d'impédance de surface.

% Le but est de déterminer les coefficients qui permettent de décomposer les champs sur les harmoniques sphériques pour aboutir aux séries de Mie, et de déterminer les coefficients de Mie $(a_n,b_n)$

\subsection{Rappel}
Soit $\OO = B_{\RR^3}(0,R)$, $M_{m,n}, N_{m,n}$ les harmoniques sphériques (cf \cite{marceaux_high-order_2000} et section \ref{sec:sol_maxwell})

% Dans la base sphérique :
% \[
%   M_{m,n}=
%   \begin{bmatrix}
%     0\\
%     ime^{im\phi}\frac{\mathbb{P}^m_n(\cos\theta)}{\sin\theta}\\
%     -e^{im\phi}\ddd{\theta}{\mathbb{P}^m_n(\cos\theta)}\\
%   \end{bmatrix}
%   \qquad
%   N_{m,n}=
%   \begin{bmatrix}
%     n(n+1)e^{im\phi}\frac{\mathbb{P}^m_n(\cos\theta)}{k_0r}\int_0^x( Id\, dt)/x\\
%     \frac{e^{im\phi}}{k_0r}\ddd{\theta}{\mathbb{P}^m_n(\cos\theta)}Id\\
%     im\frac{e^{im\phi}}{k_0r\sin\theta}\mathbb{P}^m_n(\cos\theta)Id
%   \end{bmatrix}
% \]

On pose
\begin{align*}
  \gls{mat-conj} &=conj(z)\\
  \gls{mat-gmn} &=\frac{4\pi}{2n+1}n(n+1)\frac{(n+m)!}{(n-m)!}\\
  \gls{mat-tild} &= \ddr{x}{(xu(x))}
\end{align*}

D'après les résultats énoncés en \ref{sec:helmholtz_scal}, on a
\[
  j_n(k_0r)M_{m,n} = j_n(k_0r)
  \begin{bmatrix}
    0\\
    ime^{im\phi}\frac{\mathbb{P}^m_n(\cos\theta)}{\sin\theta}\\
    -e^{im\phi}\ddr{\theta}{\PP^m_n(\cos\theta)}\\
  \end{bmatrix}
\]
et 
\[
  \tilde{j_n}(k_0r)N_{m,n}=
    \begin{bmatrix}
    n(n+1)e^{im\phi}\frac{\mathbb{P}^m_n(\cos\theta)}{k_0r}j_n(k_0r)\\
    \frac{e^{im\phi}}{k_0r}\ddr{\theta}{\PP^m_n(\cos\theta)}\tilde j_n(k_0r)\\
    im\frac{e^{im\phi}}{k_0r\sin\theta}\PP^m_n(\cos\theta)\tilde j_n(k_0r)
  \end{bmatrix}
\]

On a donc les expressions suivantes pour les champs $(\vE,\vH)$ solution de Maxwell, sans condition de rayonnement:
$\forall m,n \in \ZZ\pvect\NN$




On rappelle les relations vectorielles liant $M$ et $N$
\begin{align*}
  &\v e_r \pvect N_{1,n} = -\frac{1}{k_0r}M_{1,n} \\
  &\v e_r \pvect M_{1,n} = k_0r N_{1,n}^t
\end{align*}

\subsection{Expression des champs solutions de Maxwell}
Soient $(a_n,b_n)_n$ des suites complexes, $f_n = \frac{i^{n+1}(2n+1)}{2n(n+1)}$.

D’après \cite{morse_methods_1953}, les solutions $(\vE,\vH)$ (avec $\vH \equiv \vH /\eta_0 $) sont :
\begin{align*}
  \vE^i &= -\sum_{n=1}^\infty f_n \left( j_nM_{1,n} + \tilde j_n N_{1,n} \right) + n(n+1)\left(j_n M_{-1,n} - \tilde j_nN_{-1,n} \right)\\
  \vH^i &= -i\sum_{n=1}^\infty f_n \left( j_nM_{1,n} + \tilde j_nN_{1,n} \right) - n(n+1)\left(j_n M_{-1,n} - \tilde j_n N_{-1,n} \right)\\
  \vE^d &= -\sum_{n=1}^\infty f_n \left( a_n h_nM_{1,n} + b_n\tilde h_nN_{1,n} \right) + n(n+1)\left(a_nh_nM_{-1,n} - b_n\tilde h_nN_{-1,n} \right)\\
  \vH^d &= -i\sum_{n=1}^\infty f_n \left( b_nh_nM_{1,n} + a_n\tilde h_nN_{1,n} \right) - n(n+1)\left(b_nh_nM_{-1,n} - a_n\tilde h_nN_{-1,n} \right)\\
\end{align*}
Et les champs totaux solution de Maxwell sont : 
\begin{align}
  \vE = \vE^i + \vE^d\label{eq:solEmie}\\
  \vH = \vH^i + \vH^d\label{eq:solHmie}
\end{align}

On rappelle les relation d'orthogonalités énoncées dans \cite{marceaux_high-order_2000}, avec $\delta$ le symbole de Kronecker :
\[
  \begin{matrix}
    \ds\int_S \vdiv N_{m,n}^t \vdiv N_{p,q}^{t,*} &=& \frac{n(n+1)}{k_0^2 R^2} \gamma_{m,n} \delta_{m,n} \delta_{p,q}\\
    \ds\int_S N_{m,n}^t \cdot N_{p,q}^{t,*} &=& \frac{1}{k_0^2}\gamma_{m,n}\delta_{m,n} \delta_{p,q}\\
    \ds\int_S \left(\v e_r \cdot \vrot M_{m,n}\right)\left( e_r \cdot \vrot  M_{p,q}^{*}\right) &=& n(n+1) \gamma_{m,n} \delta_{m,n} \delta_{p,q}\\
    \ds\int_S M_{m,n} \cdot M_{p,q}^{*} &=& R^2 \gamma_{m,n} \delta_{m,n} \delta_{p,q}\\
    \ds\int_S M_{m,n} \cdot N_{p,q}^{t,*} &=& 0
  \end{matrix}
\]
Et les relations différentielles suivantes:
\begin{align*}
  \vdiv M_{m,n} &=0 \\
  \v e_r \cdot \vrot N_{m,n} &= 0
\end{align*}

\subsection{Calcul des coefficients de la série de Mie.}
\begin{tcolorbox}
Sauf mention contraire, les $h_n,j_n$ sont toujours calculées sur le cercle de rayon $k_0R$. On omet donc cette dépendance dans l'écriture.
\end{tcolorbox}
Le but est d'expliciter la \gls{acr-cioe} pour trouver des $a_n, b_n$ la satisfaisant.
Soit $k_0 = \w\sqrt{\eps_0\mu_0}$
\begin{equation}
  \label{eq:CIOE} \left(1+\frac{b}{k_0^2}L\right)\vE_t\ = \left(a_0 + \frac{a_1}{k_0^2}L\right)\vJ
\end{equation}
L est l'opérateur tel que 
\[
  \int_{S} L(u)\cdot v = - \int_{S} {\vdiv u}_t {\vdiv v}_t - \int_{S}\left(\vn \cdot \vrot u\right)\left( \vn \cdot \vrot v\right)
\]
\subsubsection{Obtention des $a_n$}
On intègre cette relation après un produit scalaire avec $M_{1,n}^{*}$ et l'on va expliciter chaque terme:
\begin{align*}
  &\int_S \vE_t\cdot M_{1,n}^{*} = - f_n R^2 \gamma_{1,n}[a_n h_n + j_n]\\
  &\int_S L(\vE_t)\cdot M_{1,n}^{*} =  f_n n(n+1)\gamma_{1,n}[a_n h_n + j_n]\\
  &\int_S \vJ \cdot M_{1,n}^{*} =  i f_n \frac{R}{k_0} \gamma_{1,n}[a_n \tilde h_n + \tilde j_n]\\
  &\int_S L(\vJ) \cdot M_{1,n}^{*} =  - i f_n n(n+1) \gamma_{1,n}\frac{1}{k_0 R}[a_n \tilde h_n + \tilde j_n]\\
\end{align*}
En combinant l'ensemble de ces termes, on obtient les $a_n$:

Soit $ \alpha_n =-i\frac{a_0R^2k^2 - a_1 n (n+1)}{k^2R^2 - bn(n+1)}$, alors
\begin{equation}
  \label{eq:an_mie} a_n = -\frac{\tilde{j_n}\frac{1}{k_0R}\alpha_n +j_n}{\tilde{h_n}\frac{1}{k_0R}\alpha_n + h_n}
\end{equation}

\subsubsection{Obtention des $b_n$}
On intègre cette relation après un produit scalaire avec $N_{1,n}^{t,*}$ et l'on va expliciter chaque terme:
\begin{align*}
  &\int_S \vE_t\cdot N_{1,n}^{t,*} = - f_n \frac{1}{k_0^2} \gamma_{1,n}[b_n\tilde{h_n} + \tilde{j_n} ]\\
  &\int_S L(\vE_t)\cdot N_{1,n}^{t,*} =  f_n \frac{n(n+1)}{k_0^2R^2}\gamma_{1,n}[b_n\tilde{h_n} + \tilde{j_n}]\\
  &\int_S \vJ \cdot N_{1,n}^{t,*} = - i f_n \frac{R}{k_0}\gamma_{1,n}[b_n {h_n} + {j_n}]\\
  &\int_S L(\vJ) \cdot N_{1,n}^{t,*} =  i f_n \frac{n(n+1)}{k_0 R} \gamma_{1,n}[b_n {h_n} + {j_n}]\\
\end{align*}
En combinant l'ensemble de ces termes, on obtient les $b_n$ :

%FIXME : corriger équation ci dessous
\begin{equation}
  \label{eq:bn_mie} b_n = -\frac{\tilde{j_n}- \frac{R}{k_0}\alpha_n j_n}{\tilde{h_n}-\frac{R}{k_0}\alpha_nh_n}
\end{equation}
\subsection{Lien entre des CSU et Q}
\begin{tcolorbox}
\[
Q=\Re{\int_{\Gamma}\vJ\cdot(\conj{\vE_t})ds}
\]
\end{tcolorbox}

En utilisant les relations d'orthogonalités et $\gamma_{1,n} = n(n+1)\gamma_{-1,n}$, on trouve
\[
  Q = \sum_ {n=0}^{\infty}2\gamma_{1,n}\frac{|f_n|^2R}{k_0}\Im{
  \left[
    (a_n h_n + j_n)(a_n^* \tilde h_n^* + \tilde j_n^*) 
    %- (b_n \tilde h_n + \tilde j_n)(b_n^*  h_n^* + j_n^*)
  \right]}
\]
