\section{Coercivité du problème de Maxwell avec conditions d'impédance}
\subsection{Condition générale d'unicité}
\subsubsection{Forme variationnelle du problème de Maxwell}

%Considérons le problème énoncé dans \cite{stupfel_sufficient_2011} (convention $e^{i\omega t}$).
\lettrine{S}{oit} une $\Gamma$ une surface régulière fermée dans $\R^3$. 
À l'extérieur de cette surface, on définit $k_0 \in \R_+^*$, le nombre d'onde dans le vide et $\eta_0 \in \R^*$ l'impédance du vide.
À l'intérieur, on définit les constantes relatives $\eps,\mu \in \C$, invariantes par translation.

\begin{tcolorbox}
\centering
La dépendance temporelle est en $e^{i\omega t}$, et l'on identifie $\H \equiv \eta _0 \H$.
\end{tcolorbox}



Soit $(\E,\H)$ dans $(\Hrot(\O))^3 \times (\Hrot(\O))^3$. 

Alors $(\E,\H)$ sont solutions de Maxwell si 
\begin{align}
\label{eq:unicite:form_var:maxwell-E}
\trot \E + ik_0 \mu \H &= 0 && \text{dans $\O$}
\\\label{eq:unicite:form_var:maxwell-H}
\trot \H - ik_0\eps \E &= 0 && \text{dans $\O$}
  % \\\label{eq:unicite:form_var:TR}
  % \Tr(\E_t) &= - \n_{Y_R} \pvect \H && \text{sur $\Gamma(0,R)$}
  \end{align}
% Où $\Tr$ est l'opérateur de capacité \cite[p.~200]{nedelec_acoustic_2001}, $\n_{Y_R}$ la normale unitaire sortante à $\Gamma(0,R)$.\\

Les relations \eqref{eq:unicite:form_var:maxwell-E} et \eqref{eq:unicite:form_var:maxwell-H} permettent grâce aux formules de Green d'obtenir la forme variationnelle suivante :
Trouver $\E \in \left(\Hrot(\O)\right)^3$, $\forall \v \phi \in \left(\Hrot(\O)\right)^3$
\[
a(\E,\v\phi) = 0
\]
où a est une forme sesquilinéaire telle que, soit $\n$ la normale unitaire sortante à $\Gamma$.
\begin{align*}
a(\E,\v\phi) &:=  \frac{1}{-ik_0\mu} \int_\O \trot \E \cdot \trot \conj{\v{\phi}} dx + -ik_0\eps\int_\O\E\cdot\conj{\v{\phi}} dx
 %+ \int_{Y_R} \conj{ \v \phi } \cdot \Tr(\E_t)ds
 - \int_\Gamma \left(\n \pvect \frac{\trot \E}{-ik_0\mu}\right) \cdot \conj{\v \phi} ds \\
 \end{align*}

 \begin{defn}[Coercivité]
 Une forme sesquilinéaire $a(u,v)$ est coercive dans $\Hrot(\O)$ si $\exists \alpha > 0$ tel que 
 \[
 |\Re(a(u,u))| \ge \alpha ||u||_{\Hrot(\O)}^2 = \alpha \left( || \trot u ||_{L^2}^2 + || u ||_{L^2}^2\right) \, \forall u \in \Hrot(\O)
 \]
 \end{defn}

%La forme bilinéaire $a$ est coercive \Gamma'il existe une constante réel positive $\mathcal{C}$ telle que $|a(\E,\E)|^2 \ge \mathcal{C}  || \E ||_{\Hrot}^4 = \mathcal{C}\left( || \trot \E ||_{L^2}^2  + || \E ||_{L^2}^2 \right)^2 $.

%Supposons $\eps, \mu$ constants et
Posons les définitions suivantes:
\begin{align*}
% X&:= \int_\O | \trot \E | ^2 dx  =|| \trot \E ||_{L^2}^2
% \\B&:= \int_\O | \E | ^2 dx  = || \E ||_{L^2}^2
%\\C&:= \int_{Y_R} \conj{E_t}\cdot T_R \E_t ds
\J &:=  \n \pvect \frac{\trot \E}{-ik_0\mu} && \text{sur $\Gamma$}
\\X&:= \int_\Gamma \J \cdot \conj{\E_t} ds
\end{align*}
La partie réelle de la forme bilinéaire $a$ écrit donc
\begin{equation}
\label{eq:unicite:form_var:decomp_form_bilin_1}
|\Re(a(\E,\E))| = \left|\frac{\Im(\mu)}{k_0} || \trot \E ||_{L^2}^2  + k_0 \Im(\eps) || \E ||_{L^2}^2
%+ \Re(C) 
- \Re(X)\right|
\end{equation}

\begin{hyp}[Hypothèses de coercivité en convention $i\omega t$]\label{hyp:unicite:form_var:hyp_coercivite}
~{}

\begin{enumerate}
  \item $\Im(\eps)$ et $\Im(\mu)$ sont de même signe.
  \item $\Im(\eps)$ et $\Re(X)$ sont de signes opposés\footnote{En convention $-i\omega t$, il faut que le signe soit le même, et donc $\Re(X) \le 0$.}.
  %\item $\Re(C)$ et $\Re(X)$ sont de même signe.
\end{enumerate}
\end{hyp}
Si l'hypothèse \ref{hyp:unicite:form_var:hyp_coercivite} est vérifiée alors $a$ est coercive ( $\alpha = \min(-\Im(\mu)k_0^{-1},-\Im(\eps)k_0)$) donc il y a unicité des solutions du problème de Maxwell avec conditions d'impédance.

Comme en convention $e^{i\omega t}$, le signe de $\Im(\eps)$ et $\Im(\mu)$ est négatif
%, sachant que d'après \cite[p.~97]{nedelec_acoustic_2001} $\Re(C)\ge 0$
alors l'unicité est assurée par la
\begin{defn}[\gls{acr-cgu}]~\\
\begin{equation}\label{eq:unicite:form_var:cgu}
\Re(X) = \Re\left(\int_\Gamma \J \cdot \conj{\E_t} ds\right) \ge 0
\end{equation}
\end{defn}

\subsubsection{Cas des matériaux sans pertes}

Si $\Im(\mu) = \Im(\eps) = 0$, le résultat précèdent n'est plus valable car $\alpha = 0$.

{%TODO ! MAT SANS PERTES
\color{red}
Voir ce qui se passe dans ce cas
}
%%%%%%%%%%%%%%%%%%%%%%%%%%%%%%%%%%%%%%%%%%%%%%%%%%
\subsection{CSU des conditions d'impédance d'ordre élevé}

La plupart des résultats de cette section ont déjà été trouvés dans \cite{stupfel_sufficient_2011}.

%On impose sur $\Gamma$  la relation $\E_t = Z \J$, où $Z$ est un opérateur explicité ci-après.

Pour tous $\forall u, v \in (\mathcal C^\infty(\Gamma))^3$ des fonctions vecteurs complexes tangentes à la surface $\Gamma$, on définit $L$ un opérateur antisymétrique négatif, i.e : 
\begin{align*}
\int_\Gamma \v u\cdot L(\conj{\v v}) &= \int_\Gamma \conj{\v v}\cdot L(\v u)\\
\int_\Gamma \v u\cdot L(\conj{\v u}) &\le 0
\end{align*}

\subsubsection{Cas de la condition d'ordre 0 (\cite{stupfel_sufficient_2011})}
Utilisons la condition d’impédance d'ordre 0, valable sur la surface $\Gamma$: 

Soit $a_0 \in \C$ tel que
\[
\E_t = a_0 \J
\]

On a alors
\begin{equation*}
X = \conj{a_0}||\J||_{L_2(\Gamma)}^2
\end{equation*}

De \eqref{eq:unicite:form_var:cgu}, on déduit la \gls{acr-csu} suivante:
\begin{equation}
\Re\left(a_0\right) \ge 0
\end{equation}

%%%%%%%%%%%%%%%%%%%%%%%%%%%%%%%%%%%%%%%%%%%%%%%%
\subsubsection{Cas de la condition d'ordre 01 (\cite{stupfel_sufficient_2011})}
Utilisons la condition d’impédance d'ordre 01, valable sur la surface $\Gamma$:

Soit $(a_0, a_1) \in \C\times\C$ tel que

\[
\E_t = (a_0 +a_1L)\J
\]
On pose:
\begin{align*}
F&:= || \J|| ^2 \ge 0  & G&:= -\int_\Gamma \J\cdot L\conj{\J} ds \ge 0
\end{align*}

On a alors
\begin{equation*}
X = \conj{a_0}F - \conj{a_1}G
\end{equation*}

De \eqref{eq:unicite:form_var:cgu}, on déduit des CSU suivantes:
\begin{align}
\Re\left(a_0\right) \ge 0\\
\Re\left(a_1\right) \le 0
\end{align}
%%%%%%%%%%%%%%%%%%%%%%%%%%%%%%%%%%%%%%%%%%%%%%%%
\subsubsection{Cas de la condition d'ordre 1}


Utilisons la condition d’impédance d'ordre 1, valable sur la surface $\Gamma$: 

Soit $(a_0, a_1,b) \in \C^3$ tel que
\[
\E_t = (1 + b L)^{-1} (a_0 + a_1 L) \J
\]

\paragraph{Première méthode (\cite{stupfel_sufficient_2011})}~

On utilise l'identité $(a_1-a_0\conj{b}) = (a_1(1+\conj{b}L) - \conj{b}(a_0+a_1L))$:

\begin{align*}
(a_1-a_0\conj{b})X &= \int_\Gamma \left(a_1(1+\conj{b}L) \J\right)\cdot\conj{\E_t} - \left(\conj{b}(a_0+a_1L)\J\right)\cdot\conj{\E_t} ds\\
&= \int_\Gamma \left(a_1(1+\conj{b}L) \conj{\E_t}\right)\cdot\J ds - \int_\Gamma \left(\conj{b}(a_0+a_1L)\J\right)\cdot\conj{\E_t} ds\\
&= \int_\Gamma \left(a_1(\conj{a_0}+\conj{a_1}L) \conj{\J}\right)\cdot\J ds  - \int_\Gamma \left(\conj{b}(1+bL)\E_t\right)\cdot\conj{\E_t} ds\\
&= a_1\conj{a_0} ||\J||^2 + |a_1|^2 \int_\Gamma \J L \conj{\J} ds - \conj{b} ||\E_t||^2 - |b|^2 \int_\Gamma \E_t L \conj{\E_t} ds
\end{align*}

On note $\Delta = a_1 -a_0\conj{b}$, $F = -\int_\Gamma \J L \conj{\J} ds \ge 0 $, $G = -\int_\Gamma \E_t L \conj{\E_t} ds \ge 0 $ . 

Si on décompose les parties réelles et imaginaires de cette expression, on a
\begin{align*}
\Re(\Delta)\Re(X) - \Im(\Delta)\Im(X) &= \Re(a_1\conj{a_0}) ||\J||^2 - \Re(\conj{b})||\E_t||^2 -|a_1|^2 F + |b|^2 G \\
\Im(\Delta)\Re(X) + \Re(\Delta)\Im(X) &= \Im(a_1\conj{a_0}) ||\J||^2 - \Im(\conj{b})||\E_t||^2
\end{align*}
La première relation nous empêche de conclure sur le signe de $\Re(X)$ car il y des signes différents entre les deux derniers termes, sauf si nous imposons $\Re( \Delta)= 0$ auquel cas, nous pouvons conclure grâce à la deuxième relation. Les CSU sont alors

\begin{align}
&\Re(\Delta) &= 0\\
&\Im(\Delta)\Im(b) &\ge 0\\
&\Im(\Delta)\Im(a_1\conj{a_0})&\ge 0
\end{align}

\paragraph{Deuxième méthode}
~
\subparagraph{Cas $a_1\not=0$}
~

En supposant $a_1 \not=0$, on utilise l'identité $(a_0 + a_1 L)^{-1}(1 + b L)  = \frac{b}{a_1} I_d + \left(1-b\frac{a_0}{a_1}\right)(a_0+a_1L)^{-1}$:
\[
X = \int_\Gamma \left(\left(\frac{b}{a_1} I_d + \left(1-b\frac{a_0}{a_1}\right)(a_0+a_1L)^{-1}\right)\E_t\right) \cdot \conj{\E_t} ds
\]

On pose:
\begin{align*}
\v D &:= (a_0 + a_1 L)^{-1}\E_t & F&:= || \v D || ^2 \ge 0  \\
G&:= -\int_\Gamma \v D \cdot L\conj{\v{D}} ds \ge 0 & H &:= || \E_t || ^2 \ge 0
\end{align*}
Comme $\conj{E_t} = (\conj{a_0} + \conj{a_1}L)D$ alors $\ds\int_\Gamma (a_0 +a_1 L) ^{-1}\E_t\cdot \conj{\E_t} ds = \conj{a_0} F - \conj{a_1} G$ et l'on peut alors écrire

\begin{equation}
\label{eq:unicite:form_var:decomp_cgu_ci1_a1}
X = \frac{b}{a_1}H   + \left(1-b\frac{a_0}{a_1}\right)\left(\conj{a_0} F - \conj{a_1} G\right)
\end{equation}
De \eqref{eq:unicite:form_var:cgu}, on déduit des CSU suivantes:
\begin{align}
\Re\left(\frac{b}{a_1}\right) \ge 0 \\
\Re\left(a_0\right) \ge 0 \\
\Re\left(\left(a_1-a_0 b\right)\frac{\conj{a_1}}{a_1}\right) \le 0
\end{align}


\subparagraph{Cas $a_1=0$}
~
\[
X = \int_\Gamma \left( \frac{1}{a_0}\left(1+bL\right)\E_t\right) \cdot \conj{\E_t} ds
\]

On pose:
\begin{align*}
F&:= \int_\Gamma | \E_t | ^2 ds \ge 0 & G &:= -\int_\Gamma \E_t \cdot L\conj{\E_t} ds \ge 0
\end{align*}

On a alors
\begin{equation}
\label{eq:unicite:form_var:decomp_cgu_ci1_a1_nul}
X = \frac{1}{a_0}F - \frac{b}{a_0}G
\end{equation}

De \eqref{eq:unicite:form_var:cgu}, on déduit des CSU suivantes:
\begin{align}
a_0 \not= 0\\
\Re\left(a_0\right) \ge 0\\
\Re\left(b\conj{a_0}\right) \le 0
\end{align}




\subsection{Autres CSU}
Méthode \cite{stupfel_implementation_2015}

On cherche à résoudre le problème suivante:
\[
\begin{pmatrix}
1 & \conj{b} \\
a_0 & a_1
\end{pmatrix}
\begin{pmatrix}
\int_\Gamma \J \cdot \conj{\E_t} \\
\int_\Gamma \J\cdot L\conj{\E_t}
\end{pmatrix}
=
\begin{pmatrix}
\conj{a_0}||\J||_2^2 + \conj{a_1}\int_\Gamma \J \cdot L\conj{\J} \\
||\E_t||_2^2 + b\int_\Gamma \conj{\E_t} \cdot L\E_t
\end{pmatrix}
\]

Si la matrice est inversible, on pose $\Delta = a_1 - a_0\conj{b}$ et alors
\[
\begin{pmatrix}
\int_\Gamma \J \cdot \conj{\E_t} \\
\int_\Gamma \J\cdot L\conj{\E_t}
\end{pmatrix}
=\frac{1}{\Delta}
\begin{pmatrix}
a_1 & -\conj{b} \\
-a_0 & 1
\end{pmatrix}
\begin{pmatrix}
\conj{a_0}||\J||_2^2 + \conj{a_1}\int_\Gamma \J \cdot L\conj{\J} \\
||\E_t||_2^2 + b\int_\Gamma \conj{\E_t} \cdot L\E_t
\end{pmatrix}
\]
Donc
\begin{align*}
X &=  \frac{a_1\conj{a_0}}{\Delta}||\J^2||_2^2 - \frac{\conj{b}}{\Delta}||\E_t||_2^2 \\
&~+\frac{|a_1|^2}{\Delta}\int_\Gamma\J\cdot L \conj{\J} - \frac{|b|^2}{\Delta}\int_\Gamma\conj{\E_t}\cdot L\E
\end{align*}
Les CSU sont alors

\begin{align}
\Re\left(a_0\conj{a_1}\Delta\right) &\ge 0\\
\Re\left(b\Delta\right) &\le 0\\
\Re\left(|a_1|^2\Delta\right) &\le 0\\
\Re\left(|b|^2\Delta\right) &\ge 0
\end{align}
Si la matrice n'est pas inversible, alors on cherche à résoudre
\[
\begin{pmatrix}
1 & \conj{b} \\
a_0 & a_0\conj{b}
\end{pmatrix}
\begin{pmatrix}
\int_\Gamma \J \cdot \conj{\E_t} \\
\int_\Gamma \J\cdot L\conj{\E_t}
\end{pmatrix}
=
\begin{pmatrix}
\conj{a_0}||\J||_2^2 + \conj{a_1}\int_\Gamma \J \cdot L\conj{\J} \\
||\E_t||_2^2 + b\int_\Gamma \conj{\E_t} \cdot L\E_t
\end{pmatrix}
\]

Le noyau de la matrice est alors $\Vect{\begin{pmatrix}\conj{b}\\-1\end{pmatrix}}$ dont l'orthogonal est  $\Vect{\begin{pmatrix}1\\\conj{b}\end{pmatrix}}$.
Pour tout $\int_\Gamma \J\cdot L\conj{\E_t} = \conj{b} \int_\Gamma \J \cdot \conj{\E_t} $, on a unicité des solutions. On déduit alors que

\[
(1 + \conj{b}^2) X = \conj{a_0} ||\J||_2^2 + \conj{a_1}\int_\Gamma \J \cdot L\conj{\J}
\]

Les CSU sont alors

\begin{align}
\Re\left(\frac{\conj{a_0}}{1 + \conj{b}^2}\right) &\ge 0 \\
\Re\left(\frac{\conj{a_1}}{1 + \conj{b}^2}\right) &\le 0
\end{align}

% \begin{proof}
% Les 3 CSU originales sont

% $\hfill
% \bullet \Re\left(\frac{b}{a_1}\right) \ge 0 \hfill
% \bullet \Re\left(\left(1-a_0\frac{b}{a_1}\right)\conj{a_0}\right) \ge 0 \hfill
% \bullet \Re\left(\left(1-a_0\frac{b}{a_1}\right)\conj{a_1}\right) \le 0
% \hfill$

% On va séparer la deuxième CSU et faire apparaître la première CSU dont on va utiliser le résultat pour simplifier le résultat.
% \begin{align*}
% \Re\left(\left(1-a_0\frac{b}{a_1}\right)\conj{a_0}\right)& \ge 0 \\
% \Re\left(\conj{a_0}\right) &\ge \Re\left(|a_0|^2\frac{b}{a_1}\right)\\
% \Re\left(a_0\right) &\ge 0
% \end{align*}
% \end{proof}
%%%%%%%%%%%%%%%%%%%%%%%%%%%%%%%%%%%%%%%%%%%%%%%%%%%%%%%%%%%%%%%%%%%%%%%%%%%%%%%%%%%%%%%%%%%%%%%%%%%%%%%%%%%%%%%%%%%%%









\subsubsection{Avec condition d'impédance à coefficients dépendant des opérateurs LD, LR.}

Soit la CIOE énoncé dans \cite{soudais_3d_2017} que l'on nommera CI3 :
\begin{equation}
\label{eq:unicite:ci3:ci3}
( 1 + b_1 L_D - b_2 L_R)\E_t = (a_0 + a_1 L_D - a_2 L_R ) \J
\end{equation}

On rappelle les expressions des opérateurs \gls{ope-LD} et \gls{ope-LR} pour des vecteurs tangents $\v U \in (C^\infty(\Gamma))^3$, $ \v V \in (C^\infty(\Gamma))^3$ :
\begin{align*}
\LD(\v U) &= \tgrads \tdivs \v U\\
\LR(\v V) &= \trots( \n ( \n \cdot \trots \v V))\\
\end{align*}
\begin{prop}
Soit $\O$ un domaine borné de $\R^3$ , de surface $\Gamma$ fermée et régulière, où $\v n$ y est la normale unitaire
sortante
\begin{equation}
\begin{matrix}
\forall \v U \in (C^\infty(\Gamma))^3 ,& \LR(\LD(\v U)) = \LD(\LR(\v U)) = 0
\end{matrix}
\end{equation}
\end{prop}
\begin{proof}
Soient un vecteur \textbf{tangent} $\v U \in (C^\infty(\Gamma))^3$.

Montrons que $\LR\LD = 0$. D’après \cite[p.~1029, A3.42]{bladel_electromagnetic_2007}, $\n \cdot \trots\tgrads f = 0$
\begin{align*}
\LR(\LD \v U)  &= \trots \left(\n \left(\n \cdot \trots \left( \tgrads \left(\tdivs \v U\right)\right)\right)\right) \\
&= 0
\end{align*}
Montrons que $\LD\LR = 0$. D’après \cite[p.~1029, A3.43]{bladel_electromagnetic_2007}, $\tdivs \trots (X\n) = 0$.
\begin{align*}
\LD(\LR \v U) &= \tgrads \tdivs \trots (\n (\n \cdot \trots \v U)) \\
&= 0
\end{align*}
\end{proof}
% Une relation importante qui découle des propriétés des opérateurs différentiels surfacique \secref{eq:op-LD-LR:prop:LDLR0} est :

% \begin{equation}
% \int_\Gamma L_D(\v U) \cdot L_R(\v V) ds = 0 , \forall \v U, \v V \in (H^1(\O))^3
% \end{equation}

% Cette relation \Gamma'exprime sous forme forte par $L_DL_R\equiv0$. Elle est là aussi symétrique entre les deux opérateurs.
On prend \eqref{eq:unicite:ci3:ci3} et on l’intègre avec des produits scalaires judicieusement choisis.

\begin{multline}
\label{eq:unicite:ci3:csu3-1}
\int_\Gamma \J\cdot\conj{\eqref{eq:unicite:ci3:ci3}}ds \Rightarrow
\int_\Gamma \J \cdot \conj{\E_t} ds  + \conj{b_1} \int_\Gamma \J\cdot L_D\conj{\E_t} ds - \conj{b_2} \int_\Gamma \J L_R\conj{\E_t} ds \\
= \conj{a_0} \int_\Gamma |\J|^2ds - \conj{a_1} \int_\Gamma |\tdivs \J|^2 ds - \conj{a_2} \int_\Gamma |\n \cdot \trots \J|^2 ds
\end{multline}
\begin{multline}
\label{eq:unicite:ci3:csu3-2}
\int_\Gamma \eqref{eq:unicite:ci3:ci3} \cdot \conj{\E_t} ds \Rightarrow
\int_\Gamma |\E_t|^2 ds  - b_1 \int_\Gamma | \tdivs \E |^2 ds - b_2 \int_\Gamma | \n \cdot \trots \E_t|^2 ds \\
= a_0 \int_\Gamma \J\cdot \conj{\E_t}ds + a_1 \int_\Gamma \conj{\E_t} L_D \J ds - a_2 \int_\Gamma \conj{\E_t} \cdot L_R \J ds
\end{multline}
\begin{multline}
\label{eq:unicite:ci3:csu3-3}
\int_\Gamma \J \cdot L_R ( \conj{\eqref{eq:unicite:ci3:ci3}} ) ds \Rightarrow
\int_\Gamma \J \cdot L_R \conj{\E_t} ds  - \conj{b_2} \int_\Gamma L_R \J \cdot L_R \conj{\E_t} ds \\
=  \conj{a_0} \int_\Gamma |\n \cdot \trots \J|^2ds - \conj{a_2} \int_\Gamma | L_R \J|^2 ds
\end{multline}
\begin{multline}
\label{eq:unicite:ci3:csu3-4}
\int_\Gamma  L_R ( \eqref{eq:unicite:ci3:ci3} ) \cdot \conj{\E_t} ds \Rightarrow
\int_\Gamma | \n \cdot \trots \E_t |^2 ds  - \conj{b_2} \int_\Gamma | L_R \E_t|^2 ds \\
= a_0 \int_\Gamma \conj{\E_t} L_R \J ds - a_2 \int_\Gamma L_R \conj{\E_t} \cdot L_R \J ds
\end{multline}
\begin{multline}
\label{eq:unicite:ci3:csu3-5}
\int_\Gamma \J \cdot L_D ( \conj{\eqref{eq:unicite:ci3:ci3}} ) ds \Rightarrow
\int_\Gamma \J \cdot L_D \conj{\E_t} ds  + \conj{b_1} \int_\Gamma L_D \J \cdot L_D \conj{\E_t} ds \\
= - \conj{a_0} \int_\Gamma |\tdivs \J|^2ds + \conj{a_1} \int_\Gamma | L_D \J|^2 ds
\end{multline}
\begin{multline}
\label{eq:unicite:ci3:csu3-6}
\int_\Gamma  L_D ( \eqref{eq:unicite:ci3:ci3} ) \cdot \conj{\E_t} ds \Rightarrow
-\int_\Gamma | \tdivs \E_t |^2 ds  + \conj{b_1} \int_\Gamma | L_D \E_t|^2 ds \\
= a_0 \int_\Gamma \conj{\E_t} L_D \J ds + a_1 \int_\Gamma L_D \conj{\E_t} \cdot L_D \J ds
\end{multline}
On pose alors les définitions suivantes :
\begin{align*}
X&:= \int_\Gamma \J \cdot \conj{\E_t} ds\\
Y_D&:= \int_\Gamma \J \cdot L_D \conj{\E_t} ds
&Y_R&:= \int_\Gamma \J \cdot L_R \conj{\E_t} ds\\
Z_D&:= \int_\Gamma L_D \J \cdot L_D \conj{\E_t} ds
&Z_R&:= \int_\Gamma L_R \J \cdot L_R \conj{\E_t} ds
\end{align*}

Les équations \eqref{eq:unicite:ci3:csu3-1} à \eqref{eq:unicite:ci3:csu3-4} sont équivalentes au système $M_R X_R = F_R$ où

\begin{align*}
M_R&:=
\begin{pmatrix}
1&\conj{b_1}&-\conj{b_2}&0\\
a_0&a_1&-a_2&0\\
0&0&1&-\conj{b_2}\\
0&0&a_0&-a_2\\
\end{pmatrix},\;
X_R =
\begin{pmatrix}
X\\
Y_D\\
Y_R\\
Z_R
\end{pmatrix}\\
F_R &=
\begin{pmatrix}
\conj{a_0} \int_\Gamma |\J|^2ds - \conj{a_1} \int_\Gamma |\tdivs \J|^2 ds - \conj{a_2} \int_\Gamma |\n \cdot \trots \J|^2 ds \\
\int_\Gamma |\E_t|^2 ds  - b_1 \int_\Gamma | \tdivs \E |^2 ds - b_2 \int_\Gamma | \n \cdot \trots \E_t|^2 ds \\
\conj{a_0} \int_\Gamma |\n \cdot \trots \J|^2ds - \conj{a_2} \int_\Gamma | L_R \J|^2 ds \\
\int_\Gamma | \n \cdot \trots \E_t |^2 ds  - \conj{b_2} \int_\Gamma | L_R \E_t|^2 ds
\end{pmatrix},\;
\end{align*}

Tandis que les équations \eqref{eq:unicite:ci3:csu3-1},\eqref{eq:unicite:ci3:csu3-2},\eqref{eq:unicite:ci3:csu3-5},\eqref{eq:unicite:ci3:csu3-6} sont équivalentes au système $M_D X_D= F_D$ où

\begin{align*}
M_D&:=
\begin{pmatrix}
1&-\conj{b_2}&\conj{b_1}&0\\
a_0&-a_2&a_1&0\\
0&0&1&\conj{b_1}\\
0&0&a_0&a_1\\
\end{pmatrix},\;
X_D =
\begin{pmatrix}
X\\
Y_R\\
Y_D\\
Z_D
\end{pmatrix}\\
F_D &=
\begin{pmatrix}
\conj{a_0} \int_\Gamma |\J|^2ds - \conj{a_1} \int_\Gamma |\tdivs \J|^2 ds - \conj{a_2} \int_\Gamma |\n \cdot \trots \J|^2 ds \\
\int_\Gamma |\E_t|^2 ds  - b_1 \int_\Gamma | \tdivs \E |^2 ds - b_2 \int_\Gamma | \n \cdot \trots \E_t|^2 ds \\
-\conj{a_0} \int_\Gamma |\tdivs \J|^2ds + \conj{a_1} \int_\Gamma | L_R \J|^2 ds \\
-\int_\Gamma | \tdivs \E_t |^2 ds  + \conj{b_1} \int_\Gamma | L_R \E_t|^2 ds
\end{pmatrix},\;
\end{align*}

On note dans la suite $\Delta_i = a_i-\conj{b_i}a_0$, $i=1,2$. Rendre ces système linéaire inversibles permet d'obtenir la première  condition suffisante:

\begin{equation}
\label{eq:unicite:ci3:csu3-cn-det}
\Delta_1\Delta_2 \not = 0
\end{equation}

\begin{minipage}{0.49\textwidth}
\textbf{Cas LR}:
\begin{align}
\label{eq:unicite:ci3:csu3r-j2}&\Re\left(a_0\conj{a_2}\Delta_2\right) \ge 0 \\
\label{eq:unicite:ci3:csu3r-e2}&\Re\left(\frac{\conj{b_2}}{\Delta_2}\right) \le 0 \\
\label{eq:unicite:ci3:csu3r-jdj}&\Re\left(\conj{a_0}a_1\left(\frac{\conj{b_1}}{\Delta_1}-\frac{\conj{b_2}}{\Delta_2}\right) + \frac{\conj{a_1}a_2}{\Delta_2} \right)\le 0\\
\label{eq:unicite:ci3:csu3r-ede}&\Re\left(2\Re(b_1)\frac{\conj{b_2}}{\Delta_2}-\frac{\conj{b_1}^2}{\Delta_1}\right) \ge 0\\
\label{eq:unicite:ci3:csu3r-jrj}&\Re\left(|a_2|^2\Delta_2\right) \le 0 \\
\label{eq:unicite:ci3:csu3r-ere}&\Re\left(|b_2|^2\Delta_2\right) \ge 0 \\
\label{eq:unicite:ci3:csu3r-rj2}&\Re\left(|a_1|^2\left(\frac{\conj{b_1}}{\Delta_1}-\frac{\conj{b_2}}{\Delta_2}\right)\right)\ge 0\\
\label{eq:unicite:ci3:csu3r-re2}&\Re\left(|b_1|^2\left(\frac{\conj{b_1}}{\Delta_1}-\frac{\conj{b_2}}{\Delta_2}\right)\right)\le 0
\end{align}
Les conditions \eqref{eq:unicite:ci3:csu3r-jrj} et \eqref{eq:unicite:ci3:csu3r-ere} impliquent :
\begin{equation}
\Re\left(\Delta_2\right) = 0\\\
\end{equation}
Les conditions \eqref{eq:unicite:ci3:csu3r-rj2} et \eqref{eq:unicite:ci3:csu3r-re2} impliquent :
\begin{equation}
\Re\left(\frac{\conj{b_1}}{\Delta_1}-\frac{\conj{b_2}}{\Delta_2}\right) = 0\\\
\end{equation}
\end{minipage}
\begin{minipage}{0.49\textwidth}
\textbf{Cas LD}:
\begin{align}
\label{eq:unicite:ci3:csu3d-j2}&\Re\left(a_0\conj{a_1}\Delta_1\right) \ge 0 \\
\label{eq:unicite:ci3:csu3d-e2}&\Re\left(\frac{\conj{b_1}}{\Delta_1}\right) \le 0 \\
\label{eq:unicite:ci3:csu3d-jrj}&\Re\left(\conj{a_0}a_2\left(\frac{\conj{b_2}}{\Delta_2}-\frac{\conj{b_2}}{\Delta_2}\right) + \frac{\conj{a_2}a_1}{\Delta_1} \right)\le 0\\
\label{eq:unicite:ci3:csu3d-ere}&\Re\left(2\Re(b_2)\frac{\conj{b_1}}{\Delta_1}-\frac{\conj{b_2}^2}{\Delta_2}\right) \ge 0\\
\label{eq:unicite:ci3:csu3d-jdj}&\Re\left(|a_1|^2\Delta_1\right) \le 0 \\
\label{eq:unicite:ci3:csu3d-ede}&\Re\left(|b_1|^2\Delta_1\right) \ge 0 \\
\label{eq:unicite:ci3:csu3d-dj2}&\Re\left(|a_2|^2\left(\frac{\conj{b_2}}{\Delta_2}-\frac{\conj{b_1}}{\Delta_1}\right)\right)\ge 0\\
\label{eq:unicite:ci3:csu3d-de2}&\Re\left(|b_2|^2\left(\frac{\conj{b_2}}{\Delta_2}-\frac{\conj{b_1}}{\Delta_1}\right)\right)\le 0
\end{align}
Les conditions \eqref{eq:unicite:ci3:csu3d-jdj} et \eqref{eq:unicite:ci3:csu3d-ede} impliquent :
\begin{equation}
\Re\left(\Delta_1\right) = 0\\\
\end{equation}
Les conditions \eqref{eq:unicite:ci3:csu3d-dj2} et \eqref{eq:unicite:ci3:csu3d-de2} impliquent :
\begin{equation}
\Re\left(\frac{\conj{b_1}}{\Delta_1}-\frac{\conj{b_2}}{\Delta_2}\right) = 0\\\
\end{equation}
\end{minipage}

%Pour le système $M_D X_D = F_D$, les conditions sont identiques à une permutation des indices 1 et 2 près.
Ces CSU sont très contraigantes et ne permettent pas de retrouver des CSU des CIOE d'ordres inférieurs lorsque l'on annule les coefficients $b_1, b_2$.
\subsubsection{Autres CSU}
On remarque que les inconnus $(Y_R,Z_R)$ (resp. $(Y_D,Z_R)$) sont déterminées par les équations \eqref{eq:unicite:ci3:csu3-3} et \eqref{eq:unicite:ci3:csu3-4} (resp. \eqref{eq:unicite:ci3:csu3-5} et \eqref{eq:unicite:ci3:csu3-6})

On déduit donc que si $\Delta_1 \not = 0$ et $\Delta_2 \not = 0$ alors

\begin{align}
  Y_R &= \frac{1}{\Delta_2}\left(a_2\left[\conj{a_0}\int_\Gamma \J\cdot\LR\conj{\J} - \conj{a_2}||\LR J||^2\right]  -\conj{b_2}\left[\int_\Gamma \conj{\E}\LR{\E} - b_2 ||\LR \E ||^2\right]\right) \\
  Y_D &= \frac{1}{\Delta_1}\left(a_1\left[\conj{a_0}\int_\Gamma \J\cdot\LD\conj{\J} + \conj{a_1}||\LD J||^2\right]  -\conj{b_1}\left[\int_\Gamma \conj{\E}\LD{\E} + b_1 ||\LD \E ||^2\right]\right) 
\end{align}

Il reste alors à utiliser l'équation \eqref{eq:unicite:ci3:csu3-1} pour obtenir
\begin{equation}
X = -\conj{b_1} Y_D + \conj{b_2} Y_R + \conj{a_0} || \J ||^2 + \conj{a_1} \int_\Gamma \J \LD \conj{J} - \conj{a_2} \int_\Gamma \J \LR \conj{J} 
\end{equation}

\begin{multline}
X = \conj{a_0} || \J ||^2 - \conj{a_1} || \div \J ||^2 - \conj{a_2} || \n \times \rot \J ||^2\\
 + \frac{\conj{b_2}}{\Delta_2}\left(a_2\left[\conj{a_0}||\n \times \rot \J||^2 - \conj{a_2}||\LR J||^2\right]  -\conj{b_2}\left[||\n\times\rot\E||^2 - b_2 ||\LR \E ||^2\right]\right) \\
 - \frac{\conj{b_1}}{\Delta_1}\left(a_1\left[-\conj{a_0}||\div\J||^2 + \conj{a_1}||\LD J||^2\right]  -\conj{b_1}\left[-||\div\E||^2 + b_1 ||\LD \E ||^2\right]\right)
\end{multline}

Et on obtiens des CSU suivantes :

\begin{align}
\Re(a_0)\ge 0 \\
\Re(a_1 - \frac{\conj{b_1a_0}a_1}{\Delta_1}) \le 0 \\
\Re(a_2 - \frac{\conj{b_2a_0}a_2}{\Delta_2}) \le 0 \\
\Re(b_1\Delta_1) = 0 \\
\Re(b_2\Delta_2) = 0 \\
\Im(b_1\Delta_1)\Im(b_1)\ge 0\\
\Im(b_2\Delta_2)\Im(b_2)\ge 0\\
\end{align}