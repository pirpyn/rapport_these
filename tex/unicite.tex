\section{Coercivité de la formulation variationnelle associé à l'équation de Maxwell avec une condition d'impédance}
	\subsection{Condition générale d'unicité}
		\subsubsection{Forme variationnelle du problème de Maxwell}

			%Considérons le problème énoncé dans \cite{stupfel_sufficient_2011} (convention $e^{i\omega t}$).
			Soit une $\Gamma$ une surface régulière fermée dans $\RR^3$. 
			À l'extérieur de cette surface, on définit $k_0 \in \RR_+^*$, le nombre d'onde dans le vide et $\eta_0 \in \RR^*$ l'impédance du vide.
			À l'intérieur, on définit les constantes relatives $\eps,\mu \in \CC$, invariantes par translation.

			\begin{tcolorbox}
				\centering
				La dépendance temporelle est en $e^{i\w t}$, et l'on identifie $\vH \equiv \eta _0 \vH$.
			\end{tcolorbox}

			Soit $(\vE,\vH)$ dans $(\Hrot(\OO))^3 \times (\Hrot(\OO))^3$. 

			Alors $(\vE,\vH)$ sont solutions de Maxwell si 
			\begin{align}
				\label{eq:unicite:form_var:maxwell-E}
				\trot \vE + ik_0 \mu \vH &= 0 && \text{dans $\OO$}
				\\\label{eq:unicite:form_var:maxwell-H}
				\trot \vH - ik_0\eps \vE &= 0 && \text{dans $\OO$}
				% \\\label{eq:unicite:form_var:TR}
				% \Tr(\vE_t) &= - \vn_{Y_R} \pvect \vH && \text{sur $\Gamma(0,R)$}
			\end{align}
			% Où $\Tr$ est l'opérateur de capacité \cite[p.~200]{nedelec_acoustic_2001}, $\vn_{Y_R}$ la normale unitaire sortante à $\Gamma(0,R)$.\\

			Les relations \eqref{eq:unicite:form_var:maxwell-E} et \eqref{eq:unicite:form_var:maxwell-H} permettent grâce aux formules de Green d'obtenir la forme variationnelle suivante :
			Trouver $\vE \in \left(\Hrot(\OO)\right)^3$, $\forall \v \phi \in \left(\Hrot(\OO)\right)^3$
			\[
				a(\vE,\v\phi) = 0
			\]
			où a est une forme sesquilinéaire telle que, soit $\vn$ la normale unitaire sortante à $\Gamma$.
			\begin{align*}
				a(\vE,\v\phi) &:=  \frac{1}{-ik_0\mu} \int_\OO \trot \vE \cdot \trot \conj{\v{\phi}} dx + -ik_0\eps\int_\OO\vE\cdot\conj{\v{\phi}} dx
				 %+ \int_{Y_R} \conj{ \v \phi } \cdot \Tr(\vE_t)ds
				 - \int_\Gamma \left(\vn \pvect \frac{\trot \vE}{-ik_0\mu}\right) \cdot \conj{\v \phi} ds \\
			 \end{align*}

			\begin{defn}[Coercivité]
				Une forme sesquilinéaire $a(u,v)$ est coercive dans $\Hrot(\OO)$ si $\exists \alpha > 0$ tel que 
				\[
					|\Re(a(u,u))| \ge \alpha ||u||_{\Hrot(\OO)}^2 = \alpha \left( || \trot u ||_{L^2}^2 + || u ||_{L^2}^2\right) \, \forall u \in \Hrot(\OO)
				\]
			 \end{defn}

			%La forme bilinéaire $a$ est coercive \Gamma'il existe une constante réel positive $\mathcal{C}$ telle que $|a(\vE,\vE)|^2 \ge \mathcal{C}  || \vE ||_{\Hrot}^4 = \mathcal{C}\left( || \trot \vE ||_{L^2}^2  + || \vE ||_{L^2}^2 \right)^2 $.

			%Supposons $\eps, \mu$ constants et
			Posons les définitions suivantes:
			\begin{align*}
				% X&:= \int_\OO | \trot \vE | ^2 dx  =|| \trot \vE ||_{L^2}^2
				% \\B&:= \int_\OO | \vE | ^2 dx  = || \vE ||_{L^2}^2
				%\\CC&:= \int_{Y_R} \conj{E_t}\cdot T_R \vE_t ds
				\vJ &:=  \vn \pvect \frac{\trot \vE}{-ik_0\mu} && \text{sur $\Gamma$}
				\\X&:= \int_\Gamma \vJ \cdot \conj{\vE_t} ds
			\end{align*}
			La partie réelle de la forme bilinéaire $a$ écrit donc
			\begin{equation}
				\label{eq:unicite:form_var:decomp_form_bilin_1}
				|\Re(a(\vE,\vE))| = \left|\frac{\Im(\mu)}{k_0} || \trot \vE ||_{L^2}^2  + k_0 \Im(\eps) || \vE ||_{L^2}^2
				%+ \Re(C) 
				- \Re(X)\right|
			\end{equation}

			\begin{hyp}[Hypothèses de coercivité en convention $i\omega t$]\label{hyp:unicite:form_var:hyp_coercivite}
				~{}

				\begin{enumerate}
					\item $\Im(\eps)$ et $\Im(\mu)$ sont de même signe.
					\item $\Im(\eps)$ et $\Re(X)$ sont de signes opposés\footnote{En convention $-i\omega t$, il faut que le signe soit le même, et donc $\Re(X) \le 0$.}.
					%\item $\Re(C)$ et $\Re(X)$ sont de même signe.
				\end{enumerate}
			\end{hyp}

			Si l'hypothèse \ref{hyp:unicite:form_var:hyp_coercivite} est vérifiée alors $a$ est coercive ( $\alpha = \min(-\Im(\mu)k_0^{-1},-\Im(\eps)k_0)$) donc il y a unicité des solutions du problème de Maxwell avec conditions d'impédance.

			Comme en convention $e^{i\omega t}$, le signe de $\Im(\eps)$ et $\Im(\mu)$ est négatif
			%, sachant que d'après \cite[p.~97]{nedelec_acoustic_2001} $\Re(C)\ge 0$
			alors l'unicité est assurée par la
			\begin{defn}[\gls{acr-cgu}]~\\
				\begin{equation}\label{eq:unicite:form_var:cgu}
					\Re(X) = \Re\left(\int_\Gamma \vJ \cdot \conj{\vE_t} ds\right) \ge 0
				\end{equation}
			\end{defn}

		\subsubsection{Cas des matériaux sans pertes}

			Si $\Im(\mu) = \Im(\eps) = 0$, le résultat précédent n'est plus valable car $\alpha = 0$.

			\TODO{Fredholm}

	%%%%%%%%%%%%%%%%%%%%%%%%%%%%%%%%%%%%%%%%%%%%%%%%%%
	\subsection{CSU des conditions d'impédance d'ordre élevé}

		La plupart des résultats de cette section ont déjà été trouvés dans \cite{stupfel_sufficient_2011}.

		% On impose sur $\Gamma$  la relation $\vE_t = Z \vJ$, où $Z$ est un opérateur explicité ci-après.

		Pour tous $\forall u, v \in (\mathcal C^\infty(\Gamma))^3$ des fonctions vecteurs complexes tangentes à la surface $\Gamma$, on définit $L$ un opérateur antisymétrique négatif, i.e : 
		\begin{align*}
			\int_\Gamma \v u\cdot L(\conj{\v v}) &= \int_\Gamma \conj{\v v}\cdot L(\v u)\\
			\int_\Gamma \v u\cdot L(\conj{\v u}) &\le 0
		\end{align*}

		%%%%%%%%%%%%%%%%%%%%%%%%%%%%%%%%%%%%%%%%%%%%%%%%%%
		\subsubsection{Cas de la condition d'ordre 0 (\cite{stupfel_sufficient_2011})}
			Utilisons la condition d’impédance d'ordre 0, valable sur la surface $\Gamma$: 

			Soit $a_0 \in \CC$ tel que
			\[
				\vE_t = a_0 \vJ
			\]

			On a alors
			\begin{equation*}
			X = \conj{a_0}||\vJ||_{L_2(\Gamma)}^2
			\end{equation*}

			De \eqref{eq:unicite:form_var:cgu}, on déduit une \gls{acr-csu}:
			\begin{equation}
			\Re\left(a_0\right) \ge 0
			\end{equation}

		%%%%%%%%%%%%%%%%%%%%%%%%%%%%%%%%%%%%%%%%%%%%%%%%
		\subsubsection{Cas de la condition d'ordre 01 (\cite{stupfel_sufficient_2011})}
			Utilisons la condition d’impédance d'ordre 01, valable sur la surface $\Gamma$:

			Soit $(a_0, a_1) \in \CC\times\CC$ tel que

			\[
				\vE_t = (a_0 +a_1L)\vJ
			\]
			On pose:
			\begin{align*}
				F&:= || \vJ|| ^2 \ge 0  & G&:= -\int_\Gamma \vJ\cdot L\conj{\vJ} ds \ge 0
			\end{align*}

			On a alors
			\begin{equation*}
				X = \conj{a_0}F - \conj{a_1}G
			\end{equation*}

			De \eqref{eq:unicite:form_var:cgu}, on déduit des CSU suivantes:
			\begin{align}
				\Re\left(a_0\right) \ge 0\\
				\Re\left(a_1\right) \le 0
			\end{align}
		%%%%%%%%%%%%%%%%%%%%%%%%%%%%%%%%%%%%%%%%%%%%%%%%
		\subsubsection{Cas de la condition d'ordre 1}


			Utilisons la condition d’impédance d'ordre 1, valable sur la surface $\Gamma$: 

			Soit $(a_0, a_1,b) \in \CC^3$ tel que
			\[
				\vE_t = (1 + b L)^{-1} (a_0 + a_1 L) \vJ
			\]

			\paragraph{Première méthode (\cite{stupfel_sufficient_2011})}~

				On utilise l'identité $(a_1-a_0\conj{b}) = (a_1(1+\conj{b}L) - \conj{b}(a_0+a_1L))$:

				\begin{align*}
					(a_1-a_0\conj{b})X &= \int_\Gamma \left(a_1(1+\conj{b}L) \vJ\right)\cdot\conj{\vE_t} - \left(\conj{b}(a_0+a_1L)\vJ\right)\cdot\conj{\vE_t} ds\\
					&= \int_\Gamma \left(a_1(1+\conj{b}L) \conj{\vE_t}\right)\cdot\vJ ds - \int_\Gamma \left(\conj{b}(a_0+a_1L)\vJ\right)\cdot\conj{\vE_t} ds\\
					&= \int_\Gamma \left(a_1(\conj{a_0}+\conj{a_1}L) \conj{\vJ}\right)\cdot\vJ ds  - \int_\Gamma \left(\conj{b}(1+bL)\vE_t\right)\cdot\conj{\vE_t} ds\\
					&= a_1\conj{a_0} ||\vJ||^2 + |a_1|^2 \int_\Gamma \vJ L \conj{\vJ} ds - \conj{b} ||\vE_t||^2 - |b|^2 \int_\Gamma \vE_t L \conj{\vE_t} ds
				\end{align*}

				On note $\Delta = a_1 -a_0\conj{b}$, $F = -\int_\Gamma \vJ L \conj{\vJ} ds \ge 0 $, $G = -\int_\Gamma \vE_t L \conj{\vE_t} ds \ge 0 $ . 

				Si on décompose les parties réelles et imaginaires de cette expression, on a
				\begin{align*}
					\Re(\Delta)\Re(X) - \Im(\Delta)\Im(X) &= \Re(a_1\conj{a_0}) ||\vJ||^2 - \Re(\conj{b})||\vE_t||^2 -|a_1|^2 F + |b|^2 G \\
					\Im(\Delta)\Re(X) + \Re(\Delta)\Im(X) &= \Im(a_1\conj{a_0}) ||\vJ||^2 - \Im(\conj{b})||\vE_t||^2
				\end{align*}
				La première relation nous empêche de conclure sur le signe de $\Re(X)$ car il y des signes différents entre les deux derniers termes, sauf si nous imposons $\Re( \Delta)= 0$ auquel cas, nous pouvons conclure grâce à la deuxième relation. Les CSU sont alors

				\begin{align}
					&\Re(\Delta) &= 0\\
					&\Im(\Delta)\Im(b) &\ge 0\\
					&\Im(\Delta)\Im(a_1\conj{a_0})&\ge 0
				\end{align}

			\paragraph{Deuxième méthode}
				~
				\subparagraph{Cas $a_1\not=0$}
					~

					En supposant $a_1 \not=0$, on utilise l'identité $(a_0 + a_1 L)^{-1}(1 + b L)  = \frac{b}{a_1} I_d + \left(1-b\frac{a_0}{a_1}\right)(a_0+a_1L)^{-1}$:
					\[
						X = \int_\Gamma \left(\left(\frac{b}{a_1} I_d + \left(1-b\frac{a_0}{a_1}\right)(a_0+a_1L)^{-1}\right)\vE_t\right) \cdot \conj{\vE_t} ds
					\]

					On pose:
					\begin{align*}
						\v D &:= (a_0 + a_1 L)^{-1}\vE_t & F&:= || \v D || ^2 \ge 0  \\
						G&:= -\int_\Gamma \v D \cdot L\conj{\v{D}} ds \ge 0 & H &:= || \vE_t || ^2 \ge 0
					\end{align*}
					Comme $\conj{E_t} = (\conj{a_0} + \conj{a_1}L)D$ alors $\ds\int_\Gamma (a_0 +a_1 L) ^{-1}\vE_t\cdot \conj{\vE_t} ds = \conj{a_0} F - \conj{a_1} G$ et l'on peut alors écrire

					\begin{equation}
						\label{eq:unicite:form_var:decomp_cgu_ci1_a1}
						X = \frac{b}{a_1}H   + \left(1-b\frac{a_0}{a_1}\right)\left(\conj{a_0} F - \conj{a_1} G\right)
					\end{equation}
					De \eqref{eq:unicite:form_var:cgu}, on déduit des CSU suivantes:
					\begin{align}
						\Re\left(\frac{b}{a_1}\right) \ge 0 \\
						\Re\left(a_0\right) \ge 0 \\
						\Re\left(\left(a_1-a_0 b\right)\frac{\conj{a_1}}{a_1}\right) \le 0
					\end{align}

				\subparagraph{Cas $a_1=0$}
					~
					\[
						X = \int_\Gamma \left( \frac{1}{a_0}\left(1+bL\right)\vE_t\right) \cdot \conj{\vE_t} ds
					\]

					On pose:
					\begin{align*}
						F&:= \int_\Gamma | \vE_t | ^2 ds \ge 0 & G &:= -\int_\Gamma \vE_t \cdot L\conj{\vE_t} ds \ge 0
					\end{align*}

					On a alors
					\begin{equation}
						\label{eq:unicite:form_var:decomp_cgu_ci1_a1_nul}
						X = \frac{1}{a_0}F - \frac{b}{a_0}G
					\end{equation}

					De \eqref{eq:unicite:form_var:cgu}, on déduit des CSU suivantes:
					\begin{align}
						a_0 \not= 0\\
						\Re\left(a_0\right) \ge 0\\
						\Re\left(b\conj{a_0}\right) \le 0
					\end{align}

	\subsection{Autres CSU}
		Méthode \cite{stupfel_implementation_2015}

		On cherche à résoudre le problème suivante:
		\[
			\begin{bmatrix}
				1 & \conj{b} \\
				a_0 & a_1
			\end{bmatrix}
			\begin{bmatrix}
				\int_\Gamma \vJ \cdot \conj{\vE_t} \\
				\int_\Gamma \vJ\cdot L\conj{\vE_t}
			\end{bmatrix}
			=
			\begin{bmatrix}
				\conj{a_0}||\vJ||_2^2 + \conj{a_1}\int_\Gamma \vJ \cdot L\conj{\vJ} \\
				||\vE_t||_2^2 + b\int_\Gamma \conj{\vE_t} \cdot L\vE_t
			\end{bmatrix}
		\]

		Si la matrice est inversible, on pose $\Delta = a_1 - a_0\conj{b}$ et alors
		\[
		\begin{bmatrix}
			\int_\Gamma \vJ \cdot \conj{\vE_t} \\
			\int_\Gamma \vJ\cdot L\conj{\vE_t}
		\end{bmatrix}
		=\frac{1}{\Delta}
		\begin{bmatrix}
			a_1 & -\conj{b} \\
			-a_0 & 1
		\end{bmatrix}
		\begin{bmatrix}
			\conj{a_0}||\vJ||_2^2 + \conj{a_1}\int_\Gamma \vJ \cdot L\conj{\vJ} \\
			||\vE_t||_2^2 + b\int_\Gamma \conj{\vE_t} \cdot L\vE_t
		\end{bmatrix}
		\]
		Donc
		\begin{align*}
			X &=  \frac{a_1\conj{a_0}}{\Delta}||\vJ^2||_2^2 - \frac{\conj{b}}{\Delta}||\vE_t||_2^2 \\
			&~+\frac{|a_1|^2}{\Delta}\int_\Gamma\vJ\cdot L \conj{\vJ} - \frac{|b|^2}{\Delta}\int_\Gamma\conj{\vE_t}\cdot L\vE
		\end{align*}
		Les CSU sont alors

			\begin{align}
			\Re\left(a_0\conj{a_1}\Delta\right) &\ge 0\\
			\Re\left(b\Delta\right) &\le 0\\
			\Re\left(|a_1|^2\Delta\right) &\le 0\\
			\Re\left(|b|^2\Delta\right) &\ge 0
		\end{align}
		Si la matrice n'est pas inversible, alors on cherche à résoudre
		\[
			\begin{bmatrix}
				1 & \conj{b} \\
				a_0 & a_0\conj{b}
			\end{bmatrix}
			\begin{bmatrix}
				\int_\Gamma \vJ \cdot \conj{\vE_t} \\
				\int_\Gamma \vJ \cdot L\conj{\vE_t}
			\end{bmatrix}
			=
			\begin{bmatrix}
				\conj{a_0}||\vJ||_2^2 + \conj{a_1}\int_\Gamma \vJ \cdot L\conj{\vJ} \\
				||\vE_t||_2^2 + b\int_\Gamma \conj{\vE_t} \cdot L\vE_t
			\end{bmatrix}
		\]

		Le noyau de la matrice est alors $\Vect{\begin{bmatrix}\conj{b}\\-1\end{bmatrix}}$ dont l'orthogonal est  $\Vect{\begin{bmatrix}1\\\conj{b}\end{bmatrix}}$.
		Pour tout $\int_\Gamma \vJ\cdot L\conj{\vE_t} = \conj{b} \int_\Gamma \vJ \cdot \conj{\vE_t} $, on a unicité des solutions. On déduit alors que

		\[
			(1 + \conj{b}^2) X = \conj{a_0} ||\vJ||_2^2 + \conj{a_1}\int_\Gamma \vJ \cdot L\conj{\vJ}
		\]

		Les CSU sont alors

		\begin{align}
			\Re\left(\frac{\conj{a_0}}{1 + \conj{b}^2}\right) &\ge 0 \\
			\Re\left(\frac{\conj{a_1}}{1 + \conj{b}^2}\right) &\le 0
		\end{align}

		% \begin{proof}
		% Les 3 CSU originales sont

		% $\hfill
		% \bullet \Re\left(\frac{b}{a_1}\right) \ge 0 \hfill
		% \bullet \Re\left(\left(1-a_0\frac{b}{a_1}\right)\conj{a_0}\right) \ge 0 \hfill
		% \bullet \Re\left(\left(1-a_0\frac{b}{a_1}\right)\conj{a_1}\right) \le 0
		% \hfill$

		% On va séparer la deuxième CSU et faire apparaître la première CSU dont on va utiliser le résultat pour simplifier le résultat.
		% \begin{align*}
		% \Re\left(\left(1-a_0\frac{b}{a_1}\right)\conj{a_0}\right)& \ge 0 \\
		% \Re\left(\conj{a_0}\right) &\ge \Re\left(|a_0|^2\frac{b}{a_1}\right)\\
		% \Re\left(a_0\right) &\ge 0
		% \end{align*}
		% \end{proof}
		%%%%%%%%%%%%%%%%%%%%%%%%%%%%%%%%%%%%%%%%%%%%%%%%%%%%%%%%%%%%%%%%%%%%%%%%%%%%%%%%%%%%%%%%%%%%%%%%%%%%%%%%%



		\subsubsection{Avec condition d'impédance à coefficients dépendant des opérateurs LD, LR.}

			Soit la CIOE énoncé dans \cite{soudais_3d_2017} que l'on nommera CI3 :
			\begin{equation}
				\label{eq:unicite:ci3:ci3}
				( 1 + b_1 L_D - b_2 L_R)\vE_t = (a_0 + a_1 L_D - a_2 L_R ) \vJ
			\end{equation}

			On rappelle les expressions des opérateurs \gls{ope-LD} et \gls{ope-LR} pour des vecteurs tangents $\v U \in (C^\infty(\Gamma))^3$, $ \v V \in (C^\infty(\Gamma))^3$ :
			\begin{align*}
				\LD(\v U) &= \tgrads \tdivs \v U\\
				\LR(\v V) &= \trots( \vn ( \vn \cdot \trots \v V))\\
			\end{align*}
			\begin{prop}
				Soit $\OO$ un domaine borné de $\RR^3$ , de surface $\Gamma$ fermée et régulière, où $\v n$ y est la normale unitaire
				sortante
				\begin{equation}
					\begin{matrix}
						\forall \v U \in (C^\infty(\Gamma))^3 ,& \LR(\LD(\v U)) = \LD(\LR(\v U)) = 0
					\end{matrix}
				\end{equation}
			\end{prop}
			\begin{proof}
				Soient un vecteur \textbf{tangent} $\v U \in (C^\infty(\Gamma))^3$.

				Montrons que $\LR\LD = 0$. D’après \cite[p.~1029, A3.42]{bladel_electromagnetic_2007}, $\vn \cdot \trots\tgrads f = 0$
				\begin{align*}
					\LR(\LD \v U)  &= \trots \left(\vn \left(\vn \cdot \trots \left( \tgrads \left(\tdivs \v U\right)\right)\right)\right) \\
					&= 0
				\end{align*}
				Montrons que $\LD\LR = 0$. D’après \cite[p.~1029, A3.43]{bladel_electromagnetic_2007}, $\tdivs \trots (X\vn) = 0$.
				\begin{align*}
					\LD(\LR \v U) &= \tgrads \tdivs \trots (\vn (\vn \cdot \trots \v U)) \\
					&= 0
				\end{align*}
			\end{proof}
			% Une relation importante qui découle des propriétés des opérateurs différentiels surfacique \secref{eq:op-LD-LR:prop:LDLR0} est :

			% \begin{equation}
			% \int_\Gamma L_D(\v U) \cdot L_R(\v V) ds = 0 , \forall \v U, \v V \in (H^1(\OO))^3
			% \end{equation}

			% Cette relation \Gamma'exprime sous forme forte par $L_DL_R\equiv0$. Elle est là aussi symétrique entre les deux opérateurs.
			On prend \eqref{eq:unicite:ci3:ci3} et on l’intègre avec des produits scalaires judicieusement choisis.

			\begin{multline}
				\label{eq:unicite:ci3:csu3-1}
				\int_\Gamma \vJ\cdot\conj{\eqref{eq:unicite:ci3:ci3}}ds \Rightarrow
				\int_\Gamma \vJ \cdot \conj{\vE_t} ds  + \conj{b_1} \int_\Gamma \vJ\cdot L_D\conj{\vE_t} ds - \conj{b_2} \int_\Gamma \vJ L_R\conj{\vE_t} ds \\
				= \conj{a_0} \int_\Gamma |\vJ|^2ds - \conj{a_1} \int_\Gamma |\tdivs \vJ|^2 ds - \conj{a_2} \int_\Gamma |\vn \cdot \trots \vJ|^2 ds
			\end{multline}
			\begin{multline}
				\label{eq:unicite:ci3:csu3-2}
				\int_\Gamma \eqref{eq:unicite:ci3:ci3} \cdot \conj{\vE_t} ds \Rightarrow
				\int_\Gamma |\vE_t|^2 ds  - b_1 \int_\Gamma | \tdivs \vE |^2 ds - b_2 \int_\Gamma | \vn \cdot \trots \vE_t|^2 ds \\
				= a_0 \int_\Gamma \vJ\cdot \conj{\vE_t}ds + a_1 \int_\Gamma \conj{\vE_t} L_D \vJ ds - a_2 \int_\Gamma \conj{\vE_t} \cdot L_R \vJ ds
			\end{multline}
			\begin{multline}
				\label{eq:unicite:ci3:csu3-3}
				\int_\Gamma \vJ \cdot L_R ( \conj{\eqref{eq:unicite:ci3:ci3}} ) ds \Rightarrow
				\int_\Gamma \vJ \cdot L_R \conj{\vE_t} ds  - \conj{b_2} \int_\Gamma L_R \vJ \cdot L_R \conj{\vE_t} ds \\
				=  \conj{a_0} \int_\Gamma |\vn \cdot \trots \vJ|^2ds - \conj{a_2} \int_\Gamma | L_R \vJ|^2 ds
			\end{multline}
			\begin{multline}
				\label{eq:unicite:ci3:csu3-4}
				\int_\Gamma  L_R ( \eqref{eq:unicite:ci3:ci3} ) \cdot \conj{\vE_t} ds \Rightarrow
				\int_\Gamma | \vn \cdot \trots \vE_t |^2 ds  - \conj{b_2} \int_\Gamma | L_R \vE_t|^2 ds \\
				= a_0 \int_\Gamma \conj{\vE_t} L_R \vJ ds - a_2 \int_\Gamma L_R \conj{\vE_t} \cdot L_R \vJ ds
			\end{multline}
				\begin{multline}
				\label{eq:unicite:ci3:csu3-5}
				\int_\Gamma \vJ \cdot L_D ( \conj{\eqref{eq:unicite:ci3:ci3}} ) ds \Rightarrow
				\int_\Gamma \vJ \cdot L_D \conj{\vE_t} ds  + \conj{b_1} \int_\Gamma L_D \vJ \cdot L_D \conj{\vE_t} ds \\
				= - \conj{a_0} \int_\Gamma |\tdivs \vJ|^2ds + \conj{a_1} \int_\Gamma | L_D \vJ|^2 ds
			\end{multline}
			\begin{multline}
				\label{eq:unicite:ci3:csu3-6}
				\int_\Gamma  L_D ( \eqref{eq:unicite:ci3:ci3} ) \cdot \conj{\vE_t} ds \Rightarrow
				-\int_\Gamma | \tdivs \vE_t |^2 ds  + \conj{b_1} \int_\Gamma | L_D \vE_t|^2 ds \\
				= a_0 \int_\Gamma \conj{\vE_t} L_D \vJ ds + a_1 \int_\Gamma L_D \conj{\vE_t} \cdot L_D \vJ ds
			\end{multline}
			On pose alors les définitions suivantes :
			\begin{align*}
				X&:= \int_\Gamma \vJ \cdot \conj{\vE_t} ds\\
				Y_D&:= \int_\Gamma \vJ \cdot L_D \conj{\vE_t} ds
				&Y_R&:= \int_\Gamma \vJ \cdot L_R \conj{\vE_t} ds\\
				Z_D&:= \int_\Gamma L_D \vJ \cdot L_D \conj{\vE_t} ds
				&Z_R&:= \int_\Gamma L_R \vJ \cdot L_R \conj{\vE_t} ds
			\end{align*}

			Les équations \eqref{eq:unicite:ci3:csu3-1} à \eqref{eq:unicite:ci3:csu3-4} sont équivalentes au système $M_R X_R = F_R$ où

			\begin{align*}
				M_R&:=
				\begin{bmatrix}
					1&\conj{b_1}&-\conj{b_2}&0\\
					a_0&a_1&-a_2&0\\
					0&0&1&-\conj{b_2}\\
					0&0&a_0&-a_2\\
				\end{bmatrix},\;
				X_R =
				\begin{bmatrix}
					X\\
					Y_D\\
					Y_R\\
					Z_R
				\end{bmatrix}\\
				F_R &=
				\begin{bmatrix}
					\conj{a_0} \int_\Gamma |\vJ|^2ds - \conj{a_1} \int_\Gamma |\tdivs \vJ|^2 ds - \conj{a_2} \int_\Gamma |\vn \cdot \trots \vJ|^2 ds \\
					\int_\Gamma |\vE_t|^2 ds  - b_1 \int_\Gamma | \tdivs \vE |^2 ds - b_2 \int_\Gamma | \vn \cdot \trots \vE_t|^2 ds \\
					\conj{a_0} \int_\Gamma |\vn \cdot \trots \vJ|^2ds - \conj{a_2} \int_\Gamma | L_R \vJ|^2 ds \\
					\int_\Gamma | \vn \cdot \trots \vE_t |^2 ds  - \conj{b_2} \int_\Gamma | L_R \vE_t|^2 ds
				\end{bmatrix}
			\end{align*}

			Tandis que les équations \eqref{eq:unicite:ci3:csu3-1},\eqref{eq:unicite:ci3:csu3-2},\eqref{eq:unicite:ci3:csu3-5},\eqref{eq:unicite:ci3:csu3-6} sont équivalentes au système $M_D X_D= F_D$ où

			\begin{align*}
				M_D&:=
				\begin{bmatrix}
					1&-\conj{b_2}&\conj{b_1}&0\\
					a_0&-a_2&a_1&0\\
					0&0&1&\conj{b_1}\\
					0&0&a_0&a_1\\
				\end{bmatrix},\;
				X_D =
				\begin{bmatrix}
					X\\
					Y_R\\
					Y_D\\
					Z_D
				\end{bmatrix}\\
				F_D &=
				\begin{bmatrix}
					\conj{a_0} \int_\Gamma |\vJ|^2ds - \conj{a_1} \int_\Gamma |\tdivs \vJ|^2 ds - \conj{a_2} \int_\Gamma |\vn \cdot \trots \vJ|^2 ds \\
					\int_\Gamma |\vE_t|^2 ds  - b_1 \int_\Gamma | \tdivs \vE |^2 ds - b_2 \int_\Gamma | \vn \cdot \trots \vE_t|^2 ds \\
					-\conj{a_0} \int_\Gamma |\tdivs \vJ|^2ds + \conj{a_1} \int_\Gamma | L_R \vJ|^2 ds \\
					-\int_\Gamma | \tdivs \vE_t |^2 ds  + \conj{b_1} \int_\Gamma | L_R \vE_t|^2 ds
				\end{bmatrix},\;
			\end{align*}

			On note dans la suite $\Delta_i = a_i-\conj{b_i}a_0$, $i=1,2$. On suppose que ces système aient une unique solution. Alors on obtient la première condition suffisante:

			\begin{equation}
				\label{eq:unicite:ci3:csu3-cn-det}
				\Delta_1\Delta_2 \not = 0
			\end{equation}

			\begin{minipage}{0.49\textwidth}
				\textbf{Cas LR}:
				\begin{align}
					\label{eq:unicite:ci3:csu3r-j2}&\Re\left(a_0\conj{a_2}\Delta_2\right) \ge 0 \\
					\label{eq:unicite:ci3:csu3r-e2}&\Re\left(\frac{\conj{b_2}}{\Delta_2}\right) \le 0 \\
					\label{eq:unicite:ci3:csu3r-jdj}&\Re\left(\conj{a_0}a_1\left(\frac{\conj{b_1}}{\Delta_1}-\frac{\conj{b_2}}{\Delta_2}\right) + \frac{\conj{a_1}a_2}{\Delta_2} \right)\le 0\\
					\label{eq:unicite:ci3:csu3r-ede}&\Re\left(2\Re(b_1)\frac{\conj{b_2}}{\Delta_2}-\frac{\conj{b_1}^2}{\Delta_1}\right) \ge 0\\
					\label{eq:unicite:ci3:csu3r-jrj}&\Re\left(|a_2|^2\Delta_2\right) \le 0 \\
					\label{eq:unicite:ci3:csu3r-ere}&\Re\left(|b_2|^2\Delta_2\right) \ge 0 \\
					\label{eq:unicite:ci3:csu3r-rj2}&\Re\left(|a_1|^2\left(\frac{\conj{b_1}}{\Delta_1}-\frac{\conj{b_2}}{\Delta_2}\right)\right)\ge 0\\
					\label{eq:unicite:ci3:csu3r-re2}&\Re\left(|b_1|^2\left(\frac{\conj{b_1}}{\Delta_1}-\frac{\conj{b_2}}{\Delta_2}\right)\right)\le 0
				\end{align}
				Les conditions \eqref{eq:unicite:ci3:csu3r-jrj} et \eqref{eq:unicite:ci3:csu3r-ere} impliquent :
				\begin{equation}
					\Re\left(\Delta_2\right) = 0\\\
				\end{equation}
				Les conditions \eqref{eq:unicite:ci3:csu3r-rj2} et \eqref{eq:unicite:ci3:csu3r-re2} impliquent :
				\begin{equation}
					\Re\left(\frac{\conj{b_1}}{\Delta_1}-\frac{\conj{b_2}}{\Delta_2}\right) = 0\\\
				\end{equation}
			\end{minipage}
			\begin{minipage}{0.49\textwidth}
				\textbf{Cas LD}:
				\begin{align}
					\label{eq:unicite:ci3:csu3d-j2}&\Re\left(a_0\conj{a_1}\Delta_1\right) \ge 0 \\
					\label{eq:unicite:ci3:csu3d-e2}&\Re\left(\frac{\conj{b_1}}{\Delta_1}\right) \le 0 \\
					\label{eq:unicite:ci3:csu3d-jrj}&\Re\left(\conj{a_0}a_2\left(\frac{\conj{b_2}}{\Delta_2}-\frac{\conj{b_2}}{\Delta_2}\right) + \frac{\conj{a_2}a_1}{\Delta_1} \right)\le 0\\
					\label{eq:unicite:ci3:csu3d-ere}&\Re\left(2\Re(b_2)\frac{\conj{b_1}}{\Delta_1}-\frac{\conj{b_2}^2}{\Delta_2}\right) \ge 0\\
					\label{eq:unicite:ci3:csu3d-jdj}&\Re\left(|a_1|^2\Delta_1\right) \le 0 \\
					\label{eq:unicite:ci3:csu3d-ede}&\Re\left(|b_1|^2\Delta_1\right) \ge 0 \\
					\label{eq:unicite:ci3:csu3d-dj2}&\Re\left(|a_2|^2\left(\frac{\conj{b_2}}{\Delta_2}-\frac{\conj{b_1}}{\Delta_1}\right)\right)\ge 0\\
					\label{eq:unicite:ci3:csu3d-de2}&\Re\left(|b_2|^2\left(\frac{\conj{b_2}}{\Delta_2}-\frac{\conj{b_1}}{\Delta_1}\right)\right)\le 0
				\end{align}
				Les conditions \eqref{eq:unicite:ci3:csu3d-jdj} et \eqref{eq:unicite:ci3:csu3d-ede} impliquent :
				\begin{equation}
					\Re\left(\Delta_1\right) = 0\\\
				\end{equation}
				Les conditions \eqref{eq:unicite:ci3:csu3d-dj2} et \eqref{eq:unicite:ci3:csu3d-de2} impliquent :
				\begin{equation}
					\Re\left(\frac{\conj{b_1}}{\Delta_1}-\frac{\conj{b_2}}{\Delta_2}\right) = 0\\\
				\end{equation}
			\end{minipage}

			%Pour le système $M_D X_D = F_D$, les conditions sont identiques à une permutation des indices 1 et 2 près.
			Ces CSU sont très contraignantes et ne permettent pas de retrouver des CSU des CIOE d'ordres inférieurs lorsque l'on annule les coefficients $b_1, b_2$.


		\subsubsection{CSU de Payen}
			On remarque que les inconnus $(Y_R,Z_R)$ (resp. $(Y_D,Z_R)$) sont déterminées par les équations \eqref{eq:unicite:ci3:csu3-3} et \eqref{eq:unicite:ci3:csu3-4} (resp. \eqref{eq:unicite:ci3:csu3-5} et \eqref{eq:unicite:ci3:csu3-6})

			On déduit donc que si $\Delta_1 \not = 0$ et $\Delta_2 \not = 0$ alors

			\begin{align}
				Y_R &= \frac{1}{\Delta_2}\left(a_2\left[\conj{a_0}\int_\Gamma \vJ\cdot\LR\conj{\vJ} - \conj{a_2}||\LR J||^2\right]  -\conj{b_2}\left[\int_\Gamma \conj{\vE}\LR{\vE} - b_2 ||\LR \vE ||^2\right]\right) \\
				Y_D &= \frac{1}{\Delta_1}\left(a_1\left[\conj{a_0}\int_\Gamma \vJ\cdot\LD\conj{\vJ} + \conj{a_1}||\LD J||^2\right]  -\conj{b_1}\left[\int_\Gamma \conj{\vE}\LD{\vE} + b_1 ||\LD \vE ||^2\right]\right) 
			\end{align}

			Il reste alors à utiliser l'équation \eqref{eq:unicite:ci3:csu3-1} pour obtenir
			\begin{equation}
				X = -\conj{b_1} Y_D + \conj{b_2} Y_R + \conj{a_0} || \vJ ||^2 + \conj{a_1} \int_\Gamma \vJ \LD \conj{J} - \conj{a_2} \int_\Gamma \vJ \LR \conj{J} 
			\end{equation}

			\begin{multline}
				X = \conj{a_0} || \vJ ||^2 - \conj{a_1} || \vdiv \vJ ||^2 - \conj{a_2} || \vn \times \vrot \vJ ||^2\\
				+ \frac{\conj{b_2}}{\Delta_2}\left(a_2\left[\conj{a_0}||\vn \times \vrot \vJ||^2 - \conj{a_2}||\LR J||^2\right]  -\conj{b_2}\left[||\vn\times\vrot\vE||^2 - b_2 ||\LR \vE ||^2\right]\right) \\
				- \frac{\conj{b_1}}{\Delta_1}\left(a_1\left[-\conj{a_0}||\vdiv\vJ||^2 + \conj{a_1}||\LD J||^2\right]  -\conj{b_1}\left[-||\vdiv\vE||^2 + b_1 ||\LD \vE ||^2\right]\right)
			\end{multline}

			Et on obtiens des CSU suivantes :

			\begin{align}
				\Re(a_0)\ge 0 \\
				\Re(a_1 - \frac{\conj{b_1a_0}a_1}{\Delta_1}) \le 0 \\
				\Re(a_2 - \frac{\conj{b_2a_0}a_2}{\Delta_2}) \le 0 \\
				\Re(b_1\Delta_1) = 0 \\
				\Re(b_2\Delta_2) = 0 \\
				\Im(b_1\Delta_1)\Im(b_1)\ge 0\\
				\Im(b_2\Delta_2)\Im(b_2)\ge 0
			\end{align}

		\subsubsection{CSU de Lafitte-Stupfel}

			\TODO{Copier la démonstration}

			On pose 
			\begin{equation}
				z = \left(1 - \frac{b_1}{a_1}a_0 - \frac{b_2}{a_2}a_0\right)
			\end{equation}

			Les CSU sont alors

			\begin{align}
				\Re(\conj{a_0}z) \ge 0 \\
				\Re(\conj{a_1}z) \le 0 \\
				\Re(\conj{a_2}z) \le 0 \\
				\Re(\frac{b_1}{a_1}) \ge 0 \\
				\Re(\frac{b_2}{a_2}) \ge 0 \\
				\Re(\frac{b_1\conj{a_2}}{a_1\conj{a_0}}) \le 0 \\
				\Re(\frac{b_2\conj{a_1}}{a_2\conj{a_0}}) \le 0 \\
				\Re(a_1) \le 0 \\
				\Re(a_2) \le 0
			\end{align}