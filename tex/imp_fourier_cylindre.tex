\section{Cas d'un objet cylindrique}

  % On rappelle les formules des opérateurs \(\vdiv, \vrot\) en coordonnée cylindrique \((r,\theta,z)\).
  % \begin{align}
  %   \vrot \vect{V} &= \left(\frac{1}{r}\ddr{\theta}{V_z} - \ddr{z}{V_\theta}\right)\vect{e_r} +
  %   \left(\ddr{z}{V_r} - \ddr{r}{V_z}\right)\vect{e_\theta} +
  %   \frac{1}{r}\left(\ddr{r}{(rV_\theta)}-\ddr{\theta}{V_r}\right)\vect{e_z}
  %   \\
  %   \vdiv \vect{V} &= \frac{1}{r}\ddr{r}{(rV_r)}+\frac{1}{r}\ddr{\theta}{V_\theta}+\ddr{z}{V_z}
  %   \\
  %   \vgrad f &= \ddr{r}{f}\vect{e_r}
  %   +\frac{1}{r}\ddr{\theta}{f}\vect{e_\theta} + \ddr{z}{f}\vect{e_z}
  % \end{align}

  \begin{figure}[!hbt]
    \centering
    \begin{tikzpicture}
      \coordinate (mat) at (0,-1.5);
\coordinate (vide) at (0,-2);
\coordinate (c) at (0,0);

\fill [lightgray] (c) circle (2);
\fill [white] (c) circle (1.5);
\fill [pattern=north east lines] (c) circle (1.5);

\draw (c) circle (2);
\draw (c) circle (1.5);


\coordinate (n) at (0,2);

%\draw (vide) node [below] {$\eps_0,\mu_0$};
\draw (mat) node [below] {$\peps,\pmu$};

% Axess
\draw [->] (n) -- ++(0,1) node [at end, right] {$\v{\mr}$};
\draw [->] (n) -- ++(1,0) node [at end, right] {$\v{\mt}$};

\draw (n) ++(0.2,0.2) circle(0.1cm) node [above=0.1cm] {$\v{\mz}$};
\draw (n) ++(0.2,0.2) +(135:0.1cm) -- +(315:0.1cm);
\draw (n) ++(0.2,0.2) +(45:0.1cm) -- +(225:0.1cm);

%\draw [->>,thick] (lt) ++ (1,1) -- (mt) ;


    \end{tikzpicture}
  \end{figure}

  On exprime les équations de Maxwell dans le matériau dans la base cylindrique et sans pertes de généralité, on peut réaliser une transformée de Fourier en \(z\) par invariance en translation et en \(\theta\) par invariance en rotation.
  Cependant, le multiplicateur de Fourier associé à la coordonnée \(\theta\) doit être un entier pour assurer la périodicité. On le note \(n\).

  \begin{equation}
    \vE(r,\theta,z) = \frac{1}{2\pi}\sum_{n=-\infty}^{\infty}\int_{\RR} e^{i(n \theta + k_z z )}\hat{\vE} (r,n,k_z) \dd{k_z}
  \end{equation}

  \begin{prop}
    Soit
    \begin{equation}
      k_3 = \sqrt{\w^2\eps\mu - k_z^2}
    \end{equation}
    et \(J_n\) et \(H_n^{(2)}\) des solutions de l'équation de Bessel d'ordre \(n\).

    Alors \(\exists (c_1,c_2,c_3,c_4) \in \CC^4\) tels que
    \begin{subequations}
      \begin{align}
        \hat{E_z}(r,n,k_z) &= c_1 J_n\left(k_3r\right) + c_2 H_n^{(2)}\left(k_3r\right)
        \\
        \hat{H_z}(r,n,k_z) &= c_3 J_n\left(k_3r\right) + c_4 H_n^{(2)}\left(k_3r\right)
      \end{align}
    \end{subequations}
  \end{prop}

  \begin{proof}

    On peut simplifier les opérateurs différentiels:

    \begin{align}
      \vrot \hat \vE(r,n,k_z) &= i\left(\frac{n}{r}\hat{E_z} - k_z\hat{E_\theta}\right)\vect{e_r} +
      \left(ik_z\hat{E_r} - \ddr{r}{\hat{E_z}}\right)\vect{e_\theta} +
      \frac{1}{r}\left(\ddr{r}{(r\hat{E_\theta})}-in\hat{E_r}\right)\vect{e_z}
      \\
      &=-i\w\mu \hat \vH(r,n,k_z)
    \end{align}

    On remarque que la méthode utilisée pour le plan aboutie à une équation différentielle à coefficients non constants de type \(r\ddr{r}{\vect{X}}(r,n,k_z) = \mat{M}(r,n,k_z)\vect{X}(r,n,k_z)\).
    On ne peut pas exprimer la solution avec les valeurs et vecteurs propres de la matrice.
    %Nous allons donc trouver une équation de Bessel en développant le système de Maxwell.

    Comme l'on cherche \(\hat \vE_t, \hat \vH_t\), on remarque que les 2\ieme composantes des équations de Maxwell permettent de déduire \(\hat\vE_t, \hat\vH_t\) de \( \hat E_z, \hat H_z\).

    On couple les 2 équations du système de Maxwell pour aboutir à une équation sur \(\hat \vE\) seul:

    % \begin{align}
    %   \vrot \vrot \hat \vE &= \w^2\eps\mu \hat \vE
    %   \\
    %   \vdiv \hat \vE &= 0
    % \end{align}

    % \begin{multline}
    %   \vrot \vrot \hat \vE = \dots\\
    %   i\left(\frac{n}{r^2}\left(\ddr{r}{(r\hat{E_\theta})} - in\hat{E_r}\right) - k_z\left(ik_z\hat{E_r} - \ddr{r}{\hat{E_z}}\right)\right)  \vect{e_r} \dots\\
    %   + \left(-k_z\left(\frac{n}{r}\hat{E_z} - k_z\hat{E_\theta}\right) -\ddr{r}{}\left(\frac{1}{r}\left(\ddr{r}{(r\hat{E_\theta})}-in\hat{E_r}\right)\right)\right)  \vect{e_\theta} \dots\\
    %   + \frac{1}{r}\left(\ddr{r}{} \left(r\left(ik_z\hat{E_r} - \ddr{r}{\hat{E_z}}\right)\right) + n \left(\frac{n}{r}\hat{E_z} - k_z\hat{E_\theta}\right)\right) \vect{e_z}
    % \end{multline}

    % On aboutit au système suivant
    \begin{equation}
      \left\lbrace
      \begin{array}{ccc}
        -\left(\w^2\eps\mu -\frac{n^2}{r^2}  - k_z^2\right)\hat{E_r}  +i\frac{n}{r^2}\ddr{r}{(r\hat{E_\theta})}  +k_z\ddr{r}{\hat{E_z}} & = & 0\\
        in\ddr{r}{}\left(\frac{\hat{E_r}}{r}\right) -\left(\w^2\eps\mu - k_z^2\right)\hat{E_\theta} + \ddr{r}{}\left(\frac{1}{r}\ddr{r}{(r\hat{E_\theta})}\right)  - n\frac{k_z}{r}\hat{E_z} & = & 0\\
        i\frac{k_z}{r}\ddr{r}{(r\hat{E_r})}  - n\frac{k_z}{r}\hat{E_\theta}  -\left(\w^2\eps\mu - \frac{n^2}{r^2} \right)\hat{E_z} - \frac{1}{r}\ddr{r}{}\left(r\ddr{r}{\hat{E_z}}\right) & = & 0
      \end{array}
      \right.
    \end{equation}

    De la troisième  équation, on trouve pour \(r\not=0\)
    \begin{equation}
    r^2 \ddr[2]{r}{\hat{E_z}} + r\ddr{r}{\hat{E_z}} + \left(r^2\w^2\eps\mu - n^2\right)\hat{E_z} =ik_zr\ddr{r}{(r\hat{E_r})} -  nk_zr\hat{E_\theta}
    \end{equation}

    Or comme \(\vdiv \hat \vE = 0\), on a
    \begin{align}
      \vdiv\hat \vE &= \frac{1}{r}\ddr{r}{(r\hat{E_r})} + \frac{in}{r}\hat{E_\theta} + ik_z\hat{E_z}
      \\
      k_z^2r^2 \hat{E_z} &= ik_zr\ddr{r}{(r\hat{E_r})} - nk_zr\hat{E_\theta}
    \end{align}

    On obtient donc sur la composante \(\hat{E_z}\):
    \begin{equation}
      r^2 \ddr[2]{r}{\hat{E_z}} + r\ddr{r}{\hat{E_z}} + \left(r^2\left(\w^2\eps\mu - k_z^2\right) - n^2\right)\hat{E_z} = 0
    \end{equation}

    C'est une équation de Bessel (cf \cite[eq (6.80)]{bowman_introduction_1958}), dont des solutions générales sont: soient \((c_1,c_2) \in \CC^2\):
     \begin{equation}
      \hat{E_z}(r,n,k_z) = c_1 J_n\left(k_3r\right) + c_2 H_n^{(2)}\left(k_3r\right)
    \end{equation}
    où \(J_n\) est la fonction de Bessel du premier type, \(H_n^{(2)}\) la fonction de Hankel de deuxième type.
    On sait que l'on peut prendre n'importe quel couple de fonctions de Bessel (cf \eqref{eq:annex:bessel:equiv_bessel}), on choisit ce dernier car les \(J_n\) sont régulières et les \(H_n\) évoluent en \(\frac{1}{\sqrt{r}}\) à l'infini, donc ce choix est adapté à une décomposition en une onde incidente partout définie et une onde réfléchie décroissante à l'infini.

    De plus, d'après \cite[p.~358]{abramowitz_handbook_1964}, on sait qu'une fonction de Bessel d'ordre \(n\) est linéairement dépendante de celle d'ordre \(-n\).
    On peut donc se restreindre à \(n\) entier naturel

    On trouve exactement le même résultat pour \(\hat{H_z}\): soient \((c_3,c_4) \in \CC^2\)
    \begin{equation}
      \hat{H_z}(r,n,k_z) = c_3 J_n\left(k_3r\right) + c_4 H_n^{(2)}\left(k_3r\right)
    \end{equation}
  \end{proof}


  \newcommand{\mJ}{\mat{J}}
  \newcommand{\mH}{\mat{H}}

  \begin{defn}
    On définit les matrices \(\mJ_{E}(r,n,k_z),\mH_{E}(r,n,k_z),\mJ_{H}(r,n,k_z),\mH_{H}(r,n,k_z)\)
    \begin{align}
      \mJ_{E}(r,n,k_z) &=
      \begin{bmatrix}
        -\frac{nk_z}{rk_3^2}J_n(k_3r) & \frac{ik\eta}{k_3}J_n'(k_3r)
        \\
        J_n(k_3r) & 0
      \end{bmatrix}
      \\
      \mH_{E}(r,n,k_z) &=
      \begin{bmatrix}
        -\frac{nk_z}{rk_3^2}H_n^{(2)}(k_3r) & \frac{ik\eta}{k_3}H_n^{(2)}{}'(k_3r)
        \\
        H_n^{(2)}(k_3r) & 0
      \end{bmatrix}
      \\
      \mJ_{H}(r,n,k_z) &=
      \begin{bmatrix}
        0 & -J_n(k_3r)
        \\
        -\frac{ik}{\eta k_3}J_n'(k_3r) & -\frac{nk_z}{rk_3^2}J_n(k_3r)
      \end{bmatrix}
      \\
      \mH_{H}(r,n,k_z) &=
      \begin{bmatrix}
        0 & -H_n^{(2)}(k_3r)
        \\
        -\frac{ik}{\eta k_3}H_n^{(2)}{}'(k_3r) & -\frac{nk_z}{rk_3^2}H_n^{(2)}(k_3r)
      \end{bmatrix}
    \end{align}
  \end{defn}

  \begin{prop}
    Alors les champs tangentiels s'écrivent
    \begin{subequations}
      \begin{align}
        \hat \vE_t(r,n,k_z) &= \mJ_{E}(r,n,k_z)
        \begin{bmatrix}
          c_1 \\
          c_3
        \end{bmatrix}
        +
        \mH_{E}(r,n,k_z)
        \begin{bmatrix}
          c_2 \\
          c_4
        \end{bmatrix}
        \label{eq:imp_fourier:cylindre:Et}\\
        \vect{e_r}\times\hat \vH_t(r,n,k_z) &=
        \mJ_{H}(r,n,k_z)
        \begin{bmatrix}
          c_1 \\
          c_3
        \end{bmatrix}
        +
        \mH_{H}(r,n,k_z)
        \begin{bmatrix}
          c_2 \\
          c_4
        \end{bmatrix}
        \label{eq:imp_fourier:cylindre:Ht}
      \end{align}
    \end{subequations}
  \end{prop}


  \begin{proof}
    À partir des équations de Maxwell restantes, on peut déterminer \(\hat{E_r},\hat{E_\theta},\hat{H_r},\hat{H_\theta}\).
    \begin{equation}
      \left\lbrace
      \begin{matrix}
        -ik_z\hat{E_\theta} + i\w\mu \hat{H_r} = -\frac{in}{r}\hat{E_z}
        \\
        ik_z\hat{E_r} + i\w\mu \hat{H_\theta} = \ddr{r}{\hat{E_z}}
        \\
        i\w\eps \hat{E_r} + ik_z \hat{H_\theta} = \frac{in}{r}\hat{H_z}
        \\
        i\w\eps \hat{E_\theta} - ik_z \hat{H_r} = -\ddr{r}{\hat{H_z}}
      \end{matrix}
      \right.
    \end{equation}

    Cela revient à résoudre \(\vect{Y} = \mat{M}\vect{X}\) où la matrice \(\mat{M}\) et les vecteurs \(\vect{X}, \vect{Y}\) sont définis tels que
    \begin{equation}
      \mat{M} =
      \begin{bmatrix}
      0 & -ik_z & i\w\mu & 0
      \\
      ik_z & 0 & 0 & i\w\mu
      \\
      i\w\eps & 0 & 0 & ik_z
      \\
      0 & i\w\eps & -ik_z & 0
      \end{bmatrix}
      \,
      \vect{X} =
      \begin{bmatrix}
        \hat{E_r}\\
        \hat{E_\theta}\\
        \hat{H_r}\\
        \hat{H_\theta}
      \end{bmatrix}
      \,
      \vect{Y} =
      \begin{bmatrix}
        -\frac{in}{r}\hat{E_z}\\
        \ddr{r}{\hat{E_z}}\\
        \frac{in}{r}\hat{H_z}\\
        -\ddr{r}{\hat{H_z}}
      \end{bmatrix}
    \end{equation}

    On remarque que \(\mM\mM = \left(k_z^2 - \omega^2\eps\mu\right)\mI\) et donc que \(\det(\mat{M}) = \left(ik_3\right)^2\).

    On suppose ce dernier non nul, on peut déduire \(\vect{X}\):

    \begin{equation}
      \begin{bmatrix}
        \hat{E_r}\\
        \hat{E_\theta}\\
        \hat{H_r}\\
        \hat{H_\theta}
      \end{bmatrix} =
      \frac{1}{-k_3^2}
      \begin{bmatrix}
      0 & -ik_z & i\w\mu & 0
      \\
      ik_z & 0 & 0 & i\w\mu
      \\
      i\w\eps & 0 & 0 & ik_z
      \\
      0 & i\w\eps & -ik_z & 0
      \end{bmatrix}
      \begin{bmatrix}
        -\frac{in}{r}\hat{E_z}\\
        \ddr{r}{\hat{E_z}}\\
        \frac{in}{r}\hat{H_z}\\
        -\ddr{r}{\hat{H_z}}
      \end{bmatrix}
    \end{equation}

    On extrait alors \(\hat{E_\theta}, \hat{H_\theta}\) pour obtenir les champs tangentielles à \(\vect{e_r}\) en tout point, sachant déjà \(\hat{E_z}, \hat{H_z}\).

    \begin{align}
      \hat{E_\theta} &= -\frac{1}{k_3^2}\left(\frac{nk_z}{r}\hat{E_z} - i\w\mu\ddr{r}{\hat{H_z}}\right)
      \\
      \hat{E_z} &= c_1 J_n(k_3 r) + c_2 H_n^{(2)}(k_3 r)
      \\
      -\hat{H_z} &= -c_3 J_n(k_3 r) - c_4 H_n^{(2)}(k_3 r)
      \\
      \hat{H_\theta} &= -\frac{1}{k_3^2}\left(i\w\eps\ddr{r}{\hat{E_z}} + \frac{nk_z}{r}\hat{H_z}\right)
    \end{align}

    On dérive les fonctions de Bessel:

     \begin{align}
      \hat{E_\theta} &= -\frac{nk_z}{rk_3^2}\left(c_1J_n(k_3r) + c_2 H_n^{(2)}(k_3r)\right) + \frac{ik\eta}{k_3}\left(c_3J_n'(k_3r) + c_4 H_n^{(2)}{}'(k_3r)\right)
      \\
      \hat{E_z} &= c_1 J_n(k_3 r) + c_2 H_n^{(2)}(k_3 r)
      \\
      -\hat{H_z} &= -c_3 J_n(k_3 r) - c_4 H_n^{(2)}(k_3 r)
      \\
      \hat{H_\theta} &= -\frac{ik}{\eta k_3}\left(c_1J_n'(k_3r) + c_2 H_n^{(2)}{}'(k_3r)\right) - \frac{nk_z}{rk_3^2}\left(c_3J_n(k_3r) + c_4 H_n^{(2)}(k_3r)\right)
    \end{align}

    Et on obtient


    \begin{subequations}
      \label{eq:imp_fourier:cylindre:champs}
      \begin{align}
        \label{eq:imp_fourier:cylindre:champs:E}
        \hat \vE_t(r,n,k_z) &= \mJ_{E}(r)
        \begin{bmatrix}
          c_1 \\
          c_3
        \end{bmatrix}
        +
        \mH_{E}(r)
        \begin{bmatrix}
          c_2 \\
          c_4
        \end{bmatrix}
        \\
        \label{eq:imp_fourier:cylindre:champs:H}
        \vect{e_r}\times\hat \vH_t(r,n,k_z) &=
        \mJ_{H}(r)
        \begin{bmatrix}
          c_1 \\
          c_3
        \end{bmatrix}
        +
        \mH_{H}(r)
        \begin{bmatrix}
          c_2 \\
          c_4
        \end{bmatrix}
      \end{align}
    \end{subequations}

  \end{proof}

  %%%%%%%%%%%%%%%%%%%%%%%%%%%%%%%%%%%%%%%%%%%%%%%%%%%%%%%%%%%%%%%%%%%%%%%%%%%%%%%%%%%%%%%%%%%%%%%%%%%%%%%%
  %%%%%%%%%%%%%%%%%%%%%%%%%%%%%%%%%%%%%%%%%%%%%%%%%%%%%%%%%%%%%%%%%%%%%%%%%%%%%%%%%%%%%%%%%%%%%%%%%%%%%%%%
  %%%%%%%%%%%%%%%%%%%%%%%%%%%%%%%%%%%%%%%%%%%%%%%%%%%%%%%%%%%%%%%%%%%%%%%%%%%%%%%%%%%%%%%%%%%%%%%%%%%%%%%%


  \subsection{Opérateur d'impédance pour une couche}

    Soit \(r_1 = r_0 + d\)
    \begin{defn}
      On définit le symbole \(\hat \mZ(n,k_z)\) de l'opérateur d'impédance la matrice telle que
      \begin{equation}
        \hat \vE_t(r_1,n,k_z) = \hat \mZ(n,k_z) \left(\vect{e_r}\pvect \hat \vH_t(r_1,n,k_z)\right)
      \end{equation}
    \end{defn}

    \begin{thm}
      Si on suppose que les fonctions de Bessel et leurs dérivées ne s’annulent pas en \(k_3r_0\) et que
      la matrice \(\mH_{H}(r_1) - \mJ_{H}(r_1)\mJ_{E}(r_0)^{-1}\mH_{E}(r_0)\) est inversible

      Alors le symbole \(\hat \mZ(n,k_z)\) de l'opérateur d'impédance est
      \begin{multline}
        \hat \mZ(n,k_z) =
        \left(\mH_{E}(r_1)\mH_{E}(r_0)^{-1} - \mJ_{E}(r_1)\mJ_{E}(r_0)^{-1}\right)\\
        \left(\mH_{H}(r_1)\mH_{E}(r_0)^{-1} - \mJ_{H}(r_1)\mJ_{E}(r_0)^{-1}\right)^{-1}
      \end{multline}
    \end{thm}

    \begin{proof}

      On injecte la relation \(\vE_t(r_0,\theta,z) = 0\) équivalente à \(\hat \vE(r_0,n,k_z) = 0\) dans \eqref{eq:imp_fourier:cylindre:Et}.
      \begin{equation}
        \mJ_{E}(r_0)
        \begin{bmatrix}
          c_1 \\
          c_3
        \end{bmatrix}
        =-\mH_{E}(r_0)
        \begin{bmatrix}
          c_2 \\
          c_4
        \end{bmatrix}
      \end{equation}

      Or par définition des matrices,
      \begin{align}
        \det(\mJ_E(r_0)) &= -\frac{ik\eta}{k_3}J_n(k_3r_0)J_n'(k_3r_0)
        \\
        \det(\mH_E(r_0)) &= -\frac{ik\eta}{k_3}H_n^{(2)}(k_3r_0)H_n^{(2)}{}'(k_3r_0)
      \end{align}

      D’après \cite[p.~370]{abramowitz_handbook_1964}, les zéros des fonctions de Bessel d'ordre réel \(>-1\) sont tous réels.
      Donc à condition d'avoir \(k_3\) complexe, comme l'ordre est entier et que l'on se restreint au entiers naturels, ces matrices sont inversibles\footnote{Là encore, il faut étudier le cas des matériaux sans pertes où \(k_3\) est réel pour \(k_z < w\sqrt{\mu\eps}\)}.

      À condition de l'inversibilité de ces deux matrices, on peut donc exprimer les composantes tangentielles
      \begin{align}
        \hat \vE_t(r_1,n,k_z) &=
        \left(\mH_{E}(r_1) - \mJ_{E}(r_1)\mJ_{E}(r_0)^{-1}\mH_{E}(r_0)\right)
        \begin{bmatrix}
          c_2 \\
          c_4
        \end{bmatrix}
        \\
        \vect{e_r}\pvect \hat \vH_t(r_1,n,k_z) &=
        \left(\mH_{H}(r_1) - \mJ_{H}(r_1)\mJ_{E}(r_0)^{-1}\mH_{E}(r_0) \right)
        \begin{bmatrix}
          c_2 \\
          c_4
        \end{bmatrix}
      \end{align}

      Et à condition que \(\mH_{H}(r_1) - \mJ_{H}(r_1)\mJ_{E}(r_0)^{-1}\mH_{E}(r_0)\) soit inversible, le symbole de l'opérateur d'impédance est:
      \begin{TODO}
        Inversibilité de \(\mH_{H}(r_1) - \mJ_{H}(r_1)\mJ_{E}(r_0)^{-1}\mH_{E}(r_0)\)
      \end{TODO}
      \begin{align}
        \hat \mZ &=
        \left(\mH_{E}(r_1) - \mJ_{E}(r_1)\mJ_{E}(r_0)^{-1}\mH_{E}(r_0)\right)
        \left(\mH_{H}(r_1) - \mJ_{H}(r_1)\mJ_{E}(r_0)^{-1}\mH_{E}(r_0)\right)^{-1}
        \\
        &=
        \left(\mH_{E}(r_1)\mH_{E}(r_0)^{-1} - \mJ_{E}(r_1)\mJ_{E}(r_0)^{-1}\right)
        \left(\mH_{H}(r_1)\mH_{E}(r_0)^{-1} - \mJ_{H}(r_1)\mJ_{E}(r_0)^{-1}\right)^{-1}
      \end{align}

      Contrairement au plan, les matrices ne commutent pas et on ne peut pas simplifier le résultat.

    \end{proof}

  %%%%%%%%%%%%%%%%%%%%%%%%%%%%%%%%%%%%%%%%%%%%%%%%%%%%%%%%%%%%%%%%%%%%%%%%%%%%%%%%%%%%%%%%%%%%%%%%%%%%%%%%
  %%%%%%%%%%%%%%%%%%%%%%%%%%%%%%%%%%%%%%%%%%%%%%%%%%%%%%%%%%%%%%%%%%%%%%%%%%%%%%%%%%%%%%%%%%%%%%%%%%%%%%%%
  %%%%%%%%%%%%%%%%%%%%%%%%%%%%%%%%%%%%%%%%%%%%%%%%%%%%%%%%%%%%%%%%%%%%%%%%%%%%%%%%%%%%%%%%%%%%%%%%%%%%%%%%


  \subsection{Opérateur d'impédance pour plusieurs couches}

    \begin{figure}[!hbt]
      \centering
      \begin{tikzpicture}
        \tikzmath{
    \a = 83;
    \b = 97;
    \d = 0.5;
    \ri = 30;
    \re = \ri;
}

% Le conducteur
\tikzmath{
    \ri = \re;
    \re = \ri + 0.5*\d;
    \xa = cos(\a)*\re;
    \ya = sin(\a)*\re;
    \xb = cos(\b)*\ri;
    \yb = sin(\b)*\ri;
}

\coordinate (a) at (\xa,\ya);
\coordinate (b) at (\xb,\yb);

\fill [pattern=north east lines] (a) arc (\a:\b:\re) -- (b) arc (\b:\a:\ri) -- cycle;
\draw (a) arc (\a:\b:\re);
\draw (a) node [right] {$r_0$};

% Le repère
\coordinate (n) at ($(a)+(0.5,-1)$);
%
%
%\draw [->] (n) -- ++(0,1) node [at end, right] {$\v{\pr}$};
%\draw [->] (n) -- ++(1,0) node [at end, right] {$\v{\pt}$};
%
\draw (n) ++(0.2,0.2) circle(0.1cm) node [above=0.1cm] {$\vect{e_z}$};
\draw (n) ++(0.2,0.2) +(135:0.1cm) -- +(315:0.1cm);
\draw (n) ++(0.2,0.2) +(45:0.1cm) -- +(225:0.1cm);

% 1 ere couche

\tikzmath{
    \ri = \re;
    \re = \ri + \d;
    \xa = cos(\a)*\re;
    \ya = sin(\a)*\re;
    \xb = cos(\b)*\ri;
    \yb = sin(\b)*\ri;
    \xc = cos(0.5*(\b+\a))*(\ri+0.5*\d);
    \yc = sin(0.5*(\b+\a))*(\ri+0.5*\d);
}

\coordinate (a) at (\xa,\ya);
\coordinate (b) at (\xb,\yb);
\coordinate (c) at (\xc,\yc);

\fill [lightgray] (a) arc (\a:\b:\re) -- (b) arc (\b:\a:\ri) -- cycle;
\draw (a) arc (\a:\b:\re);
\draw (c) node {$\eps_1,\mu_1,d_1$};


% Des couches

\tikzmath{
    \ri = \re;
    \re = \ri + 2*\d;
    \xa = cos(\a)*\re;
    \ya = sin(\a)*\re;
    \xb = cos(\b)*\ri;
    \yb = sin(\b)*\ri;
    \xc = cos(0.5*(\b+\a))*(\ri+0.5*\d);
    \yc = sin(0.5*(\b+\a))*(\ri+0.5*\d);
}

\coordinate (a) at (\xa,\ya);
\coordinate (b) at (\xb,\yb);
\coordinate (c) at (\xc,\yc);

\fill [lightgray]    (a) arc (\a:\b:\re) -- (b) arc (\b:\a:\ri) -- cycle;
\fill [pattern=dots] (a) arc (\a:\b:\re) -- (b) arc (\b:\a:\ri) -- cycle;
\draw (a) arc (\a:\b:\re);

% n eme couche

\tikzmath{
    \ri = \re;
    \re = \ri + \d;
    \xa = cos(\a)*\re;
    \ya = sin(\a)*\re;
    \xb = cos(\b)*\ri;
    \yb = sin(\b)*\ri;
    \xc = cos(0.5*(\b+\a))*(\ri+0.5*\d);
    \yc = sin(0.5*(\b+\a))*(\ri+0.5*\d);
}

\coordinate (a) at (\xa,\ya);
\coordinate (b) at (\xb,\yb);
\coordinate (c) at (\xc,\yc);

\fill [lightgray] (a) arc (\a:\b:\re) -- (b) arc (\b:\a:\ri) -- cycle;
\draw (a) arc (\a:\b:\re);
\draw (c) node {$\eps_{Nc},\mu_{Nc},d_{Nc}$};

% Le vide
\tikzmath{
    \xc = cos(0.5*(\b+\a))*(\re);
    \yc = sin(0.5*(\b+\a))*(\re);
}

\draw (\xc,\yc) node [above] {vide};


      \end{tikzpicture}
    \end{figure}

    Soit \(r_m\) le rayon de la couche \(m\), \(r_m = r_0 +\sum_{i=1}^{m} d_{i}\).

    \begin{defn}
      On définit pour chaque interface, le symbole \(\hat \mZ_m\) tel que
      \begin{equation}
        \hat \vE_t(r_m,n,k_z) = \hat \mZ_m(n,k_z) \left(\vect{e_r} \pvect \hat \vH_t(r_m,n,k_z)\right)
      \end{equation}
    \end{defn}

    Pour chaque couche caractérisée par \((\eps_m,\mu_m,d_m)\), définissons
    \begin{subequations}
      \begin{align}
        k_{3m} &= \sqrt{w^2\eps_m\mu_m - k_z^2}
        \\
        \mJ_{Em}(r) &=
          \begin{bmatrix}
            -\frac{nk_z}{rk_{3m}^2}J_n(k_{3m}r) & \frac{i\w\mu_m}{k_{3m}}J_n'(k_{3m}r)
            \\
            J_n(k_{3m}r) & 0
          \end{bmatrix}
        \\
        \mH_{Em}(r) &=
          \begin{bmatrix}
            -\frac{nk_z}{rk_{3m}^2}H_n^{(2)}(k_{3m}r) & \frac{i\w\mu_m}{k_{3m}}H_n^{(2)}{}'(k_{3m}r)
            \\
            H_n^{(2)}(k_{3m}r) & 0
          \end{bmatrix}
        \\
        \mJ_{Hm}(r) &=
          \begin{bmatrix}
            0 & -J_n(k_{3m}r)
            \\
            -\frac{i\w\eps_m}{k_{3m}}J_n'(k_{3m}r) & -\frac{nk_z}{rk_{3m}^2}J_n(k_{3m}r)
          \end{bmatrix}
        \\
        \mH_{Hm}(r) &=
          \begin{bmatrix}
            0 & -H_n^{(2)}(k_{3m}r)
            \\
            -\frac{i\w\eps_m}{k_{m3}}H_n^{(2)}{}'(k_{3m}r) & -\frac{nk_z}{rk_{3m}^2}H_n^{(2)}(k_{3m}r)
          \end{bmatrix}
        \\
        \mA_{Jm}(r) &= \mJ_{Em}(r) -  \mZ_{m-1} \mJ_{Hm}(r)
        \\
        \mA_{Hm}(r) &= \mH_{Em}(r) -  \mZ_{m-1} \mH_{Hm}(r)
      \end{align}
    \end{subequations}

    \begin{thm}
      Soit \(\hat \mZ_0(n,k_z) = \mat{0}_{\mathcal{M}_2(\CC)}\).

      Si pour tout \(0 < m < n\)

      \begin{equation}
        \begin{aligned}
          k_{3m} & \not = 0 \\
          \det\left(\mA_{Jm}(r_{m-1})\right) & \not = 0
          \\
          \det\left(\mA_{Hm}(r_{m-1})\right) & \not = 0
          \\
          \det\left(\mH_{Hm}(r_{m})\mA_{Hm}(r_{m-1})^{-1} - \mJ_{Hm}(r_{m})(\mA_{Jm}(r_{m-1}))^{-1}\right) &\not = 0
        \end{aligned}
      \end{equation}

      Alors le symbole \(\hat \mZ_n\) est défini par la relation de récurrence :
      \begin{multline}
        \mZ_m = \left(\mH_{Em}(r_m)\mA_{Hm}(r_{m-1})^{-1} - \mJ_{Em}(r_m)\mA_{Jm}(r_{m-1})^{-1}\right) \\
            \left(\mH_{Hm}(r_m)\mA_{Hm}(r_{m-1})^{-1} - \mJ_{Hm}(r_m)\mA_{Jm}(r_{m-1})^{-1}\right)^{-1}
      \end{multline}
    \end{thm}

    \begin{proof}
      À l'initialisation, on retrouve le résultat pour une couche.

      On résonne par récursivité:

      On se situe dans la couche \(m\) et l'on sait que les champs vérifient
      \begin{equation}
        \begin{bmatrix}
          \hat{E_\theta}(r_{m-1},n,k_z)\\
          \hat{E_z}(r_{m-1},n,k_z)\\
        \end{bmatrix}
        =
        \hat \mZ_{m-1}(n,k_z)
        \begin{bmatrix}
          -\hat{H_z}(r_{m-1},n,k_z)\\
          \hat{H_\theta}(r_{m-1},n,k_z)\\
        \end{bmatrix}
      \end{equation}

      En injectant ce qui précède dans \eqref{eq:imp_fourier:cylindre:champs} en \(r = r_{m-1}\)
      \begin{align}
        \mJ_{Em}(r_{m-1})
        \begin{bmatrix}
          c_1 \\
          c_3
        \end{bmatrix}
        +
        \mH_{Em}(r_{m-1})
        \begin{bmatrix}
          c_2 \\
          c_4
        \end{bmatrix}
        &=
        \hat \mZ_{m-1}
        \left(
          \mJ_{Hm}(r_{m-1})
          \begin{bmatrix}
            c_1 \\
            c_3
          \end{bmatrix}
          +
          \mH_{Hm}(r_{m-1})
          \begin{bmatrix}
            c_2 \\
            c_4
          \end{bmatrix}
        \right)
        \\
        \mA_{Jm}(r_{m-1})
        \begin{bmatrix}
          c_1 \\
          c_3
        \end{bmatrix}
        &=
        -\mA_{Hm}(r_{m-1})
        \begin{bmatrix}
          c_2 \\
          c_4
        \end{bmatrix}
      \end{align}

      \begin{TODO}
        Inversibilité de \(\mA_{Jm}(r_m), \mA_{Hm}(r_m)\).
      \end{TODO}

      On injecte ce qui précède dans \eqref{eq:imp_fourier:cylindre:champs} en \(r = r_{m}\)
      \begin{align}
        \vE_t &=
        \left(\mH_{Em}(r_{m}) - \mJ_{Em}(r_{m})\mA_{Jm}(r_{m-1})^{-1}\mA_{Hm}(r_{m-1})\right)
        \begin{bmatrix}
          c_2 \\
          c_4
        \end{bmatrix}
        \\
        \vect{e_r}\times\vH_t &=
        \left(\mH_{Hm}(r_{m}) - \mJ_{Hm}(r_{m})\mA_{Jm}(r_{m-1})^{-1}\mA_{Hm}(r_{m-1}) \right)
        \begin{bmatrix}
          c_2 \\
          c_4
        \end{bmatrix}
      \end{align}

      \begin{TODO}
        Inversibilité de \(\mH_{Hm}(r_m) - \mJ_{Hm}(r_m)(\mA_{Jm}(r_{m-1}))^{-1}\mA_{Hm}(r_{m-1})\).
      \end{TODO}

      On peut alors conclure sur le symbole

      \begin{multline}
        \hat \mZ_{m} =
          \left(\mH_{Em}(r_m) - \mJ_{Em}(r_m)\mA_{Jm}(r_{m-1})^{-1}\mA_{Hm}(r_{m-1})\right) \\
          \left(\mH_{Hm}(r_m) - \mJ_{Hm}(r_m)\mA_{Jm}(r_{m-1})^{-1}\mA_{Hm}(r_{m-1})\right)^{-1}
      \end{multline}

      \begin{multline}
        \hat \mZ_{m} =
          \left(\mH_{Em}(r_m)\mA_{Hm}(r_{m-1})^{-1} - \mJ_{Em}(r_m)\mA_{Jm}(r_{m-1})^{-1}\right) \\
          \left(\mH_{Hm}(r_m)\mA_{Hm}(r_{m-1})^{-1} - \mJ_{Hm}(r_m)\mA_{Jm}(r_{m-1})^{-1}\right)^{-1}
      \end{multline}

    \end{proof}

  \subsection{Applications numérique}

    La figure \ref{fig:imp_fourier:cylindre:hoppe_p62} permet de vérifier les résultats de \cite[p.~62]{hoppe_impedance_1995} pour une couche de matériau sans perte (voir Figure \ref{fig:annex:hoppe:p62}).

    \begin{TODO}
      Expliquer pourquoi on prendre des valeurs continues de \(n\) et non discrètes
    \end{TODO}

    \begin{figure}[!hbt]
      \centering
      \begin{tikzpicture}[scale=1]
        \begin{axis}[
            title={},
            ylabel={\(\Im(\hat{\mZ}(k_t r_1,0))\)},
            xlabel={\(k_t\slash k_0\)},
            width=0.8\textwidth,
            xmin=0,
            xmax=1.5,
            mark repeat=20,
            legend pos=outer north east
          ]
          \addplot [black] table [x={s1}, y={Im(z_ex.tm)},col sep=comma] {tikz/csv/impedance/HOPPE_62/HOPPE_62.z_ex.C_+3.000E-02.csv};
          \addlegendentry{TM}
          \addplot [black,dashed] table [x={s1}, y={Im(z_ex.te)},col sep=comma]  {tikz/csv/impedance/HOPPE_62/HOPPE_62.z_ex.C_+3.000E-02.csv};
          \addlegendentry{TE}
        \end{axis}
      \end{tikzpicture}
      \caption{\(\eps = 6, \mu = 1, r_0 = 0.0300\text{m}, d=0.0225\text{m}, f=1\text{GHz}\)}
      \label{fig:imp_fourier:cylindre:hoppe_p62}
    \end{figure}

    La figure \ref{fig:imp_fourier:cylindre:hoppe_p62:converge_rayon} montre la convergence du symbole de l'impédance d'un cylindre vers le symbole du plan en fonction du rayon du cylindre.

    \begin{figure}[!hbt]
      \centering
      \begin{tikzpicture}[scale=1]
        \begin{axis}[
            title={},
            ylabel={\(\Im(\hat{\mZ}(k_tr_1,0))\)},
            xlabel={\(k_t \slash k_0\)},
            width=0.7\textwidth,
            xmin=0,
            xmax=1.5,
            mark repeat=20,
            legend pos=outer north east
          ]

          \addplot [black] table [col sep=comma, x={s1}, y={Im(z_ex.tm)}] {tikz/csv/impedance/HOPPE_62/HOPPE_62.z_ex.C_+3.000E-02.csv};
          \addlegendentry{TM \(r_0=0.03m\)}

          \addplot [black,dashed] table [col sep=comma, x={s1}, y={Im(z_ex.te)}] {tikz/csv/impedance/HOPPE_62/HOPPE_62.z_ex.C_+3.000E-02.csv};
          \addlegendentry{TE \(r_0=0.03m\)}
          \addplot [black,mark=*] table [col sep=comma, x={s1}, y={Im(z_ex.tm)}] {tikz/csv/impedance/HOPPE_62/HOPPE_62.z_ex.C_+3.000E-01.csv};
          \addlegendentry{TM \(r_0=0.3m\)}

          \addplot [black,dashed,mark=*] table [col sep=comma, x={s1}, y={Im(z_ex.te)}] {tikz/csv/impedance/HOPPE_62/HOPPE_62.z_ex.C_+3.000E-01.csv};
          \addlegendentry{TE \(r_0=0.3m\)}
        \end{axis}
      \end{tikzpicture}
      \caption{\(\eps = 6, \mu = 1, d=0.0225\text{m}, f=1\text{GHz}\)}
      \label{fig:imp_fourier:cylindre:hoppe_p62:converge_rayon}
    \end{figure}

\begin{TODO}
  Courbes erreurs plan cylindre
\end{TODO}

    % \begin{figure}[!hbt]
    %   \centering
    %   \begin{tikzpicture}[scale=1]
    %     \begin{loglogaxis}[
    %         title={},
    %         ylabel={\(||\hat{\mZ}_{plan} - \hat{\mZ}_{cyl}||_2\)},
    %         xlabel={\(r_0/d\)},
    %         width=0.8\textwidth,
    %         xmin=0.1,
    %         xmax=100,
    %         % mark repeat=20,
    %         legend pos=outer north east
    %       ]
    %       \legend{TM,TE}
    %       \addplot [black] table [x={r0/d}, y={tm},col sep=semicolon] {tikz/csv/impedance/cylindre/hoppe_p62_error.csv};
    %       \addplot [black,dashed] table [x={r0/d}, y={te},col sep=semicolon] {tikz/csv/impedance/cylindre/hoppe_p62_error.csv};
    %     \end{loglogaxis}
    %   \end{tikzpicture}
    %   \caption{\(\eps = 6, \mu = 1, d=0.0225\text{m}, f=1\text{GHz}\)}
    %   \label{fig:imp_fourier:cylindre:hoppe_p62:converge_rayon:error}
    % \end{figure}