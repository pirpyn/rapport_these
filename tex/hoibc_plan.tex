\section{Approximation de l'opérateur d'impédance pour un plan infini par une condition d'impédance d'ordre élevée}

  \subsection{Expression de la condition d'impédance d'ordre élevée}

    On sait que le symbole \(\hat{\mZ}(k_x,k_y)\) s'écrit

    \begin{multline}
        \hat \mZ_m = -k_{3m}
        \left(ik_{3m}\tan\left(k_{3m}d_m\right)\mI - \hat \mZ_{m-1}\mC_m\right) \\
        \left(k_{3m}\mI - i\tan\left(k_{3m}d_m\right)\hat \mZ_{m-1}\mC_m\right)^{-1}
        \mC_m^{-1}
    \end{multline}

    où la matrice \(\mC_m\) est telle que
    \begin{align}
       \mC_m &= \frac{1}{k_m\eta_m}
        \begin{bmatrix}
            k_m^2 - k_y^2 & k_xk_y\\
            k_xk_y & k_m^2 - k_y^2
        \end{bmatrix}
    \end{align}

    Dans le cadre de cette thèse, nous nous intéressons à la condition d'impédance d'\cite{aubakirov_electromagnetic_2014} qui s'exprime en tout point de la surface extérieur de notre objet telle que

    \begin{align}
        \left(I + b_1 \LD{} -b_2 \LR \right)\vE_t = \left(a_0 I + a_1 \LD{} - a_2 \LR \right)(\vn \pvect \vH) & \forall x \in \Gamma
    \end{align}

    Nous la nommons CI3.

    Pour approcher le symbole \(\hat{\mZ}\) que l'on a défini dans l'espace de Fourier, il convient donc d'exprimer les opérateurs \(\LD, \LR\) dans cet espace.

    % Pour cela, nous utiliserons la relation de \cite[Corollary 2,p. ~154]){yosida_functional_1995}:

    Par définition, on a
    \begin{align}
      \LD \vE_t & = \vgrads{} \vdivs{} \vE_t
    \end{align}

    Or

    \begin{align}
      \vE_t(x,y,z) & = \int_\RR\int_\RR \hat{\vE_t}(k_x,k_y,z)e^{ik_xx + ik_yy}\dd{k_x}\dd{k_y}
    \end{align}


    donc

    \begin{align}
      \LD \vE_t
      & = \vgrads{} \vdivs{} \vE_t
      \\
      &=\vgrads{} \int_\RR\int_\RR \hat{\vE_t}(k_x,k_y,z) \cdot \vgrads{} e^{ik_xx + ik_yy}\dd{k_x}\dd{k_y}
      \\
      &=\int_\RR \int_\RR \vhesss{}\left(\left( e^{ik_xx + ik_yy} \right) \hat{\vE_t}\right)(k_x,k_y,z)\dd{k_x}\dd{k_y}
    \end{align}

    On définit \(\hat{\LD}\) l'opérateur matriciel tel que
    \begin{align}
      \LD \vE_t
      &= \int_\RR\int_\RR \hat{\LD} \hat{\vE_t}(k_x,k_y,z)\dd{k_x}\dd{k_y}
    \end{align}

    \begin{equation}
      \hat{\LD}(k_x,k_y) = -
      \begin{bmatrix}
        k_x^2 & k_x k_y
        \\
        k_x k_y & k_y^2
      \end{bmatrix}
    \end{equation}


    Par définition, on a
    \begin{align}
      \LR \vE_t & = \vrots{} \vrots{} \vE_t
    \end{align}

    donc

    \begin{align}
      \LR \vE_t
      & = \vrots{} \vrots{} \vE_t
      \\
      &=\vrots{} \int_\RR\int_\RR \vgrads{}\left(e^{ik_xx + ik_yy}\right) \pvect \hat{\vE_t}(k_x,k_y,z)\dd{k_x}\dd{k_y}
      \\
      &= \int_\RR \int_\RR \left(\vhesss - \vlapls\right) \left(\left(e^{ik_xx + ik_yy}\right) \hat{\vE_t}\right)(k_x,k_y,z)\dd{k_x}\dd{k_y}
      \\
    \end{align}

    On définit \(\hat{\LR}\) l'opérateur matriciel tel que
    \begin{align}
      \LR \vE_t
      &= \int_\RR\int_\RR \hat{\LR} \hat{\vE_t}(k_x,k_y,z)\dd{k_x}\dd{k_y}
    \end{align}

    \begin{equation}
      \hat{\LR}(k_x,k_y) =
      \begin{bmatrix}
        k_y^2 & -k_x k_y
        \\
        -k_x k_y & k_x^2
      \end{bmatrix}
    \end{equation}

    On peut donc définir \(\hat{\mZ}_{IBC}\) l’opérateur matriciel associé à la condition d'impédance.

    \begin{multline}
        \hat{\mZ}_{IBC}(k_x,k_y) = \left(I + b_1 \hat{\LD}(k_x,k_y) - b_2 \hat{\LR}(k_x,k_y) \right)^{-1}
        \\
        \left(a_0 I + a_1 \hat{\LD}(k_x,k_y) - a_2 \hat{\LR}(k_x,k_y)\right)
    \end{multline}

    Pour calculer les coefficients de la CIOE, il faut minimiser la distance entre le symbole \(\hat{\mZ}(k_x,k_y)\) et \(\hat{\mZ}_{IBC}(k_x,k_y)\) \footnote{La distance est la norme de Frobenius}. Évidemment, il existe une infinité de combinaisons pour un couple \((k_x,k_y)\). Nous avons choisis de nous donner un grand nombre de couples et de minimiser au sens des moindres carrés.

    Pour tenir compte d'une onde plane homogène, il faut que \(k_x^2 + k_y^2\) soit inférieur à \(k_0^2\); au delà, nous reproduisons des ondes évanescentes. Nous avons constaté que prendre \(k_x^2 + k_y^2\) inférieur à \(2k_0\) est suffisant pour prendre en compte des ondes guidées dans le cas des matériaux sans pertes.

  \subsection{Le cas spécial de la condition de Leontovich}

    La condition de Leontovich consiste à approcher le symbole par une constante, et donc d'avoir \(a_1=a_2=b_1=b_2=0\). Dans le paragraphe précèdent, cela revient à chercher la valeur moyenne du symbole. Cependant, cette condition rend compte de l'impédance à incidence normale. Dans ce cas particulier, nous ne réaliserons pas de minimisation mais on fixera cette constante à la valeur du symbole en \((0,0)\).

    \begin{equation}
      a_0 = \hat{\mZ}(0,0)
    \end{equation}

  \subsection{Résultats numériques}

      La figure \ref{fig:imp_fourier:plan:hoppe:33:hoibc} permet de vérifier les résultats de \cite[p.~33]{hoppe_impedance_1995} pour une couche de matériau sans perte. La condition d'impédance d'ordre élevé est bien meilleure que la condition de Leontovich.

      \begin{figure}[!hbt]
          \centering
          \begin{tikzpicture}[scale=1]
              \begin{axis}[
                      title={Polarisation TM},
                      ylabel={\(\Im(\hat{\mZ}(k_x,0)\)},
                      xlabel={\(k_x\slash k_0\)},
                      width=0.4\textwidth,
                      xmin=0,
                      xmax=2,
                      mark repeat=20,
                      legend pos=outer north east
                  ]
                  \addplot [color=black,mark=square*] table [col sep=comma, x={s1}, y={Im(z_ex.tm)}] {tikz/csv/impedance/HOPPE_33/HOPPE_33.z_ex.P.csv};
                  % \addlegendentry{Exact};

                  \addplot [color=blue,mark=x] table [col sep=comma, x={s1}, y={Im(z_ibc0.tm)}] {tikz/csv/impedance/HOPPE_33/HOPPE_33.z_ibc.IBC_ibc0_TYPE_P_SUC_F.csv};
                  % \addlegendentry{CI0};

                  \addplot [color=red,mark=diamond*] table [col sep=comma, x={s1}, y={Im(z_ibc3.tm)}] {tikz/csv/impedance/HOPPE_33/HOPPE_33.z_ibc.IBC_ibc3_TYPE_P_SUC_F.csv};
                  % \addlegendentry{CI3};
              \end{axis}
          \end{tikzpicture}
          \begin{tikzpicture}[scale=1]
              \begin{axis}[
                      title={Polarisation TE},
                      ylabel={},
                      xlabel={\(k_x\slash k_0\)},
                      width=0.4\textwidth,
                      xmin=0,
                      xmax=2,
                      mark repeat=20,
                      legend pos=outer north east
                  ]
                  \addplot [color=black,mark=square*] table [col sep=comma, x={s1}, y={Im(z_ex.te)}] {tikz/csv/impedance/HOPPE_33/HOPPE_33.z_ex.P.csv};
                  \addlegendentry{Exact};

                  \addplot [color=blue,mark=x] table [col sep=comma, x={s1}, y={Im(z_ibc0.te)},color=] {tikz/csv/impedance/HOPPE_33/HOPPE_33.z_ibc.IBC_ibc0_TYPE_P_SUC_F.csv};
                  \addlegendentry{CI0};

                  \addplot [color=red,mark=diamond*] table [col sep=comma, x={s1}, y={Im(z_ibc3.te)}] {tikz/csv/impedance/HOPPE_33/HOPPE_33.z_ibc.IBC_ibc3_TYPE_P_SUC_F.csv};
                  \addlegendentry{CI3};
              \end{axis}
          \end{tikzpicture}
          \caption[CIOE sur empilement de Hoppe & Rahmat-Samii p.~33]{\(\eps = 4, \mu = 1, d=0.015\text{m}, f=1\text{GHz}\)}
          \label{fig:imp_fourier:plan:hoppe:33:hoibc}
      \end{figure}

      \begin{figure}[!hbt]
          \centering
          \begin{tikzpicture}[scale=1]
              \begin{axis}[
                      title={Polarisation TM},
                      ylabel={\(\Im(\hat{\mZ}(k_x,0)\)},
                      xlabel={\(k_x\slash k_0\)},
                      width=0.4\textwidth,
                      xmin=0,
                      xmax=1.4,
                      mark repeat=22,
                      legend pos=outer north east
                  ]
                  \addplot [color=black,mark=square*] table [col sep=comma, x={s1}, y={Im(z_ex.tm)}] {tikz/csv/impedance/ICEAA_11/ICEAA_11.z_ex.P.csv};
                  % \addlegendentry{Exact};

                  \addplot [color=blue,mark=x] table [col sep=comma, x={s1}, y={Im(z_ibc0.tm)}] {tikz/csv/impedance/ICEAA_11/ICEAA_11.z_ibc.IBC_ibc0_TYPE_P_SUC_F.csv};
                  % \addlegendentry{CI0};

                  \addplot [color=red,mark=diamond*] table [col sep=comma, x={s1}, y={Im(z_ibc3.tm)}] {tikz/csv/impedance/ICEAA_11/ICEAA_11.z_ibc.IBC_ibc3_TYPE_P_SUC_F.csv};
                  % \addlegendentry{CI3};
              \end{axis}
          \end{tikzpicture}
          \begin{tikzpicture}[scale=1]
              \begin{axis}[
                      title={Polarisation TE},
                      ylabel={},
                      xlabel={\(k_x\slash k_0\)},
                      width=0.4\textwidth,
                      xmin=0,
                      xmax=1.4,
                      mark repeat=22,
                      legend pos=outer north east
                  ]
                  \addplot [color=black,mark=square*] table [col sep=comma, x={s1}, y={Im(z_ex.te)}] {tikz/csv/impedance/ICEAA_11/ICEAA_11.z_ex.P.csv};
                  \addlegendentry{Exact};

                  \addplot [color=blue,mark=x] table [col sep=comma, x={s1}, y={Im(z_ibc0.te)},color=] {tikz/csv/impedance/ICEAA_11/ICEAA_11.z_ibc.IBC_ibc0_TYPE_P_SUC_F.csv};
                  \addlegendentry{CI0};

                  \addplot [color=red,mark=diamond*] table [col sep=comma, x={s1}, y={Im(z_ibc3.te)}] {tikz/csv/impedance/ICEAA_11/ICEAA_11.z_ibc.IBC_ibc3_TYPE_P_SUC_F.csv};
                  \addlegendentry{CI3};
              \end{axis}
          \end{tikzpicture}
            \caption[CIOE sur empilement de P.~Soudais p.~11]{\(\eps = 4, \mu = 1, d=0.035\text{m}, f=12\text{GHz}\)}
          \label{fig:imp_fourier:plan:soudais:hoibc}
      \end{figure}

      On remarque que la CI3 capture très bien l'asymptote, grâce aux termes \(b_1\) et \(b_2\) qui annulent le dénominateur. C'est donc une bonne condition d'impédance pour des matériaux sans pertes, où une ou plusieurs asymptotes peuvent apparaître.

  \subsection{Lien avec les CIOE de \cite{stupfel_implementation_2015}}

    Dans cet article sont introduites les CIOE
    \begin{itemize}
      \item CI01
        \begin{equation}
          \vE_t = \left(\diag{a_{01}}{a_{02}} + a_1\LL\right)\vJ
        \end{equation}
      \item CI1
        \begin{equation}
          \left(\oI + b\LL \right)\vE_t = \left(\diag{a_{01}}{a_{02}} + a_1\LL\right)\vJ
        \end{equation}
    \end{itemize}

    L'opérateur \(\LL\) est le laplacien tangentiel \(\lapls\) fois l'identité (\(\LL = \lapls{}\oI\)). Le multiplicateur de Fourier associé est la matrice
    \begin{equation}
      L = -
      \begin{bmatrix}
        k_x^2 + k_y^2 & 0
        \\
        0 & k_x^2 + k_y^2
      \end{bmatrix}
    \end{equation}

    Nous présentons quelques résultats avec ces CIOE, mais la CIOE CI3 étant plus performante, les CIOE CI01 et CI1 ne seront pas exploitées.

    \begin{figure}[!hbt]
      \centering
      \begin{tikzpicture}[scale=1]
          \begin{axis}[
                  title={Polarisation TM},
                  ylabel={\(|\hat{\mZ}(k_x,0)|\)},
                  xlabel={\(k_x\slash k_0\)},
                  width=0.4\textwidth,
                  xmin=0,
                  xmax=1,
                  ymin=0.15,
                  ymax=0.25,
                  mark repeat=22,
                  legend pos=outer north east
              ]
              \addplot [color=black,mark=square*] table [col sep=comma, x={s1}, y={Abs(z_ex.tm)}] {tikz/csv/impedance/M1/M1.z_ex.P.csv};

              \addplot [color=blue,mark=x] table [col sep=comma, x={s1}, y={Abs(z_ibc0.tm)}] {tikz/csv/impedance/M1/M1.z_ibc.IBC_ibc0_TYPE_P_SUC_F.csv};

              \addplot [color=green!50!black,mark=pentagon*] table [col sep=comma, x={s1}, y={Abs(z_ibc01.tm)}] {tikz/csv/impedance/M1/M1.z_ibc.IBC_ibc01_TYPE_P_SUC_F.csv};

              \addplot [color=orange,mark=*] table [col sep=comma, x={s1}, y={Abs(z_ibc1.tm)}] {tikz/csv/impedance/M1/M1.z_ibc.IBC_ibc1_TYPE_P_SUC_F.csv};

              \addplot [color=red,mark=diamond*] table [col sep=comma, x={s1}, y={Abs(z_ibc3.tm)}] {tikz/csv/impedance/M1/M1.z_ibc.IBC_ibc3_TYPE_P_SUC_F.csv};
          \end{axis}
      \end{tikzpicture}
      \begin{tikzpicture}[scale=1]
          \begin{axis}[
                  title={Polarisation TE},
                  ylabel={},
                  xlabel={\(k_x\slash k_0\)},
                  width=0.4\textwidth,
                  xmin=0,
                  xmax=1,
                  ymin=0.15,
                  ymax=0.25,
                  mark repeat=22,
                  legend pos=outer north east
              ]
              \addplot [color=black,mark=square*] table [col sep=comma, x={s1}, y={Abs(z_ex.te)}] {tikz/csv/impedance/M1/M1.z_ex.P.csv};
              \addlegendentry{Exact};

              \addplot [color=blue,mark=x] table [col sep=comma, x={s1}, y={Abs(z_ibc0.te)},color=] {tikz/csv/impedance/M1/M1.z_ibc.IBC_ibc0_TYPE_P_SUC_F.csv};
              \addlegendentry{CI0};

              \addplot [color=green!50!black,mark=pentagon*] table [col sep=comma, x={s1}, y={Abs(z_ibc01.te)}] {tikz/csv/impedance/M1/M1.z_ibc.IBC_ibc01_TYPE_P_SUC_F.csv};
              \addlegendentry{CI01};

              \addplot [color=orange,mark=*] table [col sep=comma, x={s1}, y={Abs(z_ibc1.te)}] {tikz/csv/impedance/M1/M1.z_ibc.IBC_ibc1_TYPE_P_SUC_F.csv};
              \addlegendentry{CI1};

              \addplot [color=red,mark=diamond*] table [col sep=comma, x={s1}, y={Abs(z_ibc3.te)}] {tikz/csv/impedance/M1/M1.z_ibc.IBC_ibc3_TYPE_P_SUC_F.csv};
              \addlegendentry{CI3};

          \end{axis}
      \end{tikzpicture}
        \caption[CIOE sur empilement de B.~Stupfel p.~1661]{\(\eps = 1-i, \mu = 1, d=0.05\text{m}, f=0.2\text{GHz}\)}
      \label{fig:imp_fourier:plan:stupfel:hoibc}
    \end{figure}

