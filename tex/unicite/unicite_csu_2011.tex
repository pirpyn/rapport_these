\section[Des CSU pour les CIOE de Stupfel et Poget 2011]{Des conditions suffisante pour les CIOE de \cite{stupfel_sufficient_2011}}

  Nous définissons les \glspl{acr-cioe} comme une condition limite liant les traces tangentielles \(\vE_t, \vJ\) des champs sur \(\Gamma\). Une démonstration pour obtenir ces CIOE est en dehors du cadre de cette thèse, nous ne ferons donc qu'utiliser des CIOE existantes. 

  Grâce à ces CIOE, nous allons établir des conditions elles aussi suffisante qui impliquent la \gls{acr-cgu} \eqref{eq:unicite:form_var:cgu}. Par leur nature suffisante, il n'y pas un unique jeu de CSU pour une CIOE donnée. Une difficulté est d'être capable de juger si ce jeu est satisfaisant, et si ce n'est pas le cas, être capable de proposer un autre jeu.

  Les CIOE de \cite{stupfel_sufficient_2011} font intervenir l'opérateur de Hodge \(\mathcal{L}\), commençons par rappeler son expression et quelques propriétés.

  \begin{defn}
    Pour tous \((\vu) \in (\mathcal C^\infty(\CC,\Gamma))^3\)
    \begin{equation}
      \LL(\vu) = \tgrads{\tdivs \vu} - \trots{\trots \vu}
    \end{equation}
  \end{defn}

  \begin{prop}
    Par définition, l’opérateur hermitien \(\LL\) est symétrique négatif.

    Pour tous \(\vu,\vv \in (\mathcal C^\infty(\CC,\Gamma))^3\)
    \begin{align}
      \int_\Gamma \vu\cdot \LL(\conj{\vv}) &= \int_\Gamma \conj{\vv}\cdot \LL(\vu)
      \\
      \int_\Gamma \vu\cdot \LL(\conj{\vu}) &\le 0
    \end{align}
  \end{prop}

  De plus, on rappelle la notation de la section précédente
  \begin{equation}
    X = \int_\Gamma \vJ \cdot \conj{\vE_t}
    \label{eq:unicite:2011:X}
  \end{equation}
  On rappelle que l'on veut trouver des conditions permettant de garantir \eqref{eq:unicite:form_var:cgu} qui est \(\Re(X)\ge0\).

  %%%%%%%%%%%%%%%%%%%%%%%%%%%%%%%%%%%%%%%%%%%%%%%%%%
  \subsection{CSU de la CI0}
    Utilisons la condition d’impédance de Leontovich, la \hyperlink{ci0}{CI0}:

    Soit \(a_0 \in \CC\) tel que
    \begin{align*}
      \vE_t = a_0 \vJ
    \end{align*}

    On a alors
    \begin{equation*}
      X = \conj{a_0}\norm{\vJ}^2
    \end{equation*}

    Il suffit alors que
    \begin{equation}
      \Re\left(a_0\right) \ge 0
    \end{equation}
    pour que \(\Re(X)\ge0\).

    Nous referons à cette condition comme ``la'' CSU-0. La CSU-0 est donc un sous-espace de \(\CC\), le demi-espace des complexes à partie réelle positive.

  %%%%%%%%%%%%%%%%%%%%%%%%%%%%%%%%%%%%%%%%%%%%%%%%
  \subsection{CSU de la CI01}
    Utilisons la condition d’impédance \hyperlink{ci01}{CI01}:

    Soit \((a_0, a_1) \in \CC^2\) tel que

    \begin{align*}
      \vE_t = (a_0 +a_1 \LL)\vJ
    \end{align*}

    On déduit que
    \begin{align*}
      X & = \conj{a_0} \norm{\vJ} ^2 + \conj{a_1}\int_\Gamma \vJ\cdot \LL\conj{\vJ}
    \end{align*}

    Il suffit alors d'imposer les conditions suivantes pour que \(\Re(X)\ge 0\) pour toutes solutions \(\vE_t,\vJ\).
    \begin{align}
      \Re\left(a_0\right) \ge 0\\
      \Re\left(a_1\right) \le 0
    \end{align}

    Nous referons à ces conditions comme ``les'' CSU-01. Les CSU-01 sont donc un sous-espace de \(\CC^2\).


  %%%%%%%%%%%%%%%%%%%%%%%%%%%%%%%%%%%%%%%%%%%%%%%%
  \subsection{CSU de la CI1}

    Utilisons la condition d’impédance \hyperlink{ci1}{CI1}:

    Soit \((a_0, a_1,b) \in \CC^3\) tel que
    \begin{align*}
      (1 + b \LL) \vE_t = (a_0 + a_1 \LL) \vJ
    \end{align*}

    Pour cette CIOE, nous démontrons qu'il existe plusieurs jeux de CSU possibles.

    \subsubsection{CSU de \cite{stupfel_sufficient_2011}}

      \begin{prop}
        Soit \(\Delta = a_1 - a_0\conj{b}\), alors des CSU sont
        \begin{align}
          &\Re(\Delta) &= 0\\
          &\Im(\Delta)\Im(b) &\ge 0\\
          &\Im(\Delta)\Im(a_1\conj{a_0})&\ge 0
        \end{align}
      \end{prop}

      \begin{proof}
        On utilise l'identité \((a_1-a_0\conj{b}) = (a_1(1+\conj{b}\LL) - \conj{b}(a_0+a_1\LL))\):
        \begin{align*}
          (a_1-a_0\conj{b})X &= \int_\Gamma \left(a_1(1+\conj{b}\LL) \vJ\right)\cdot\conj{\vE_t} - \left(\conj{b}(a_0+a_1 \LL)\vJ\right)\cdot\conj{\vE_t} \\
          &= \int_\Gamma \left(a_1(1+\conj{b}\LL) \conj{\vE_t}\right)\cdot\vJ - \int_\Gamma \left(\conj{b}(a_0+a_1 \LL)\vJ\right)\cdot\conj{\vE_t} \\
          &= \int_\Gamma \left(a_1(\conj{a_0}+\conj{a_1}\LL) \conj{\vJ}\right)\cdot\vJ - \int_\Gamma \left(\conj{b}(1+b \LL)\vE_t\right)\cdot\conj{\vE_t} \\
          &= a_1\conj{a_0} \norm{ \vJ }^2 + |a_1|^2 \int_\Gamma \vJ \LL \conj{\vJ} - \conj{b} \norm{ \vE_t }^2 - |b|^2 \int_\Gamma \vE_t \LL \conj{\vE_t} 
        \end{align*}

        On pose 
        \begin{align*}
          F &= -\int_\Gamma \vJ \LL \conj{\vJ} \ge 0
          \\
          G &= -\int_\Gamma \vE_t \LL \conj{\vE_t} \ge 0
        \end{align*}

        Si on décompose les parties réelles et imaginaires de cette expression, on a
        \begin{align*}
          \Re(\Delta)\Re(X) - \Im(\Delta)\Im(X) &= \Re(a_1\conj{a_0}) \norm{\vJ}^2 - \Re(\conj{b})\norm{\vE_t}^2 -|a_1|^2 F + |b|^2 G \\
          \Im(\Delta)\Re(X) + \Re(\Delta)\Im(X) &= \Im(a_1\conj{a_0}) \norm{\vJ}^2 - \Im(\conj{b})\norm{\vE_t}^2
          \intertext{En imposant \(\Re(\Delta)= 0\) nous pouvons conclure grâce à la deuxième relation qui implique}
          \Im(\Delta)^2\Re(X) &= \Im(\Delta)\Im(a_1\conj{a_0}) \norm{\vJ}^2 - \Im(\Delta)\Im(\conj{b})\norm{ \vE_t }^2
        \end{align*}
        Dans le cas où \(\Im(\Delta)\not=0\), il suffit d'imposer que tous ces termes soient positifs pour que \(\Re(X)\) le soit.
        Des CSU sont alors
        \begin{align}
          &\Re(\Delta) &= 0\\
          &\Im(\Delta)\Im(b) &\ge 0\\
          &\Im(\Delta)\Im(a_1\conj{a_0})&\ge 0
        \end{align}
      \end{proof}

      Nous pouvons noter que ces CSU fixent la partie réelle de \(a_1\) en fonction des autres coefficients, ce qui est plus contraignant qu'une condition d'inégalité.
  
      Nous referons à ces conditions comme ``les'' CSU-1-1. Les CSU-1-1 sont un sous-espace de \(\CC^3\). On remarque que la projection de ce sous-espace sur \(\CC^2 \times \lbrace0\rbrace\) (quand \(b=0\) donc quand la CI1 devient la CI01) n'est pas égale aux CSU-01, mais est un sous-espace plus petit. Ce sont donc des CSU trop contraignantes.
    
    \subsubsection{CSU de \cite{stupfel_implementation_2015}}

      \begin{prop}
        Un autre jeu de CSU pour la CI1 est
        \begin{align}
          a_1 = a_0 \conj{b} \\
          \Re\left(\frac{\conj{a_0}}{1 + \conj{b}^2}\right) &\ge 0 \\
          \Re\left(\frac{\conj{a_1}}{1 + \conj{b}^2}\right) &\le 0
        \end{align}
      \end{prop}

      \begin{proof}
        On cherche à résoudre le problème suivant:
        \[
          \begin{bmatrix}
            1 & \conj{b} \\
            a_0 & a_1
          \end{bmatrix}
          \begin{bmatrix}
            \int_\Gamma \vJ \cdot \conj{\vE_t} \\
            \int_\Gamma \vJ\cdot \LL\conj{\vE_t}
          \end{bmatrix}
          =
          \begin{bmatrix}
            \conj{a_0}\norm{\vJ}_2^2 - \conj{a_1}F \\
            \norm{\vE_t}_2^2 - bG
          \end{bmatrix}
        \]

        Si la matrice ci-dessus est inversible, on note son déterminant \(\Delta\) introduit dans la partie précédente et alors
        \[
        \begin{bmatrix}
          \int_\Gamma \vJ \cdot \conj{\vE_t} \\
          \int_\Gamma \vJ\cdot \LL\conj{\vE_t}
        \end{bmatrix}
        =\frac{1}{\Delta}
        \begin{bmatrix}
          a_1 & -\conj{b} \\
          -a_0 & 1
        \end{bmatrix}
        \begin{bmatrix}
          \conj{a_0}\norm{ \vJ }_2^2 - \conj{a_1}F \\
          \norm{ \vE_t }_2^2 - bG
        \end{bmatrix}
        \]
        On ne s’intéresse qu'à la première composante
        \begin{align*}
          X &=  
          \frac{a_1\conj{a_0}}{\Delta}\norm{\vJ^2}_2^2 
          - \frac{\conj{b}}{\Delta}\norm{\vE_t}_2^2 
          - \frac{|a_1|^2}{\Delta}F
          + \frac{|b|^2}{\Delta}G
        \intertext{ et donc }
          \Re(X) &= 
          \Re\left(\frac{a_1\conj{a_0}}{\Delta}\right)\norm{\vJ^2}_2^2 
          - \Re\left({\frac{\conj{b}}{\Delta}}\right)\norm{\vE_t}_2^2 
          - \Re\left(\frac{|a_1|^2}{\Delta}\right)F
          + \Re\left(\frac{|b|^2}{\Delta}\right)G
        \end{align*}

        On impose alors la positivité de tous les termes, ce qui implique que \(\Re(X)\ge0\).

        Des CSU sont alors bien celles que l'on a déjà trouvées

          \begin{align}
          \Re\left(a_0\conj{a_1}\Delta\right) &\ge 0
          \\
          \Re\left(b\Delta\right) &\le 0
          \\
          \Re\left(|a_1|^2\Delta\right) &\le 0
          \\
          \Re\left(|b|^2\Delta\right) &\ge 0
        \end{align}

        Si la matrice n'est pas inversible, alors on cherche à résoudre
        \[
          \begin{bmatrix}
            1 & \conj{b} \\
            a_0 & a_0\conj{b}
          \end{bmatrix}
          \begin{bmatrix}
            \int_\Gamma \vJ \cdot \conj{\vE_t} \\
            \int_\Gamma \vJ \cdot \LL\conj{\vE_t}
          \end{bmatrix}
          =
          \begin{bmatrix}
            \conj{a_0}\norm{\vJ}_2^2 - \conj{a_0}bF
            \\
            \norm{\vE_t}_2^2 - bG
          \end{bmatrix}
        \]

        Le noyau de la matrice est alors \(\Vect{\begin{bmatrix}\conj{b}\\-1\end{bmatrix}}\) dont l'orthogonal est  \(\Vect{\begin{bmatrix}1\\\conj{b}\end{bmatrix}}\).
        Pour tout \(\int_\Gamma \vJ\cdot \LL\conj{\vE_t} = \conj{b} \int_\Gamma \vJ \cdot \conj{\vE_t} \), on a unicité des solutions. On déduit alors que

        \[
          (1 + \conj{b}^2) X = \conj{a_0} \norm{\vJ}_2^2 - \conj{a_0}bF
        \]

        Il suffit d'imposer la positivité de tous les termes pour conclure

        \begin{align}
          \Delta = 0 \\
          \Re\left(\frac{\conj{a_0}}{1 + \conj{b}^2}\right) &\ge 0 \\
          \Re\left(\frac{\conj{a_0}b}{1 + \conj{b}^2}\right) &\le 0
        \end{align}

      \end{proof}

      Mais dans ce cas, on ne retombe pas non plus sur les CSU de la CI01 quand \(b=0\).

    % \paragraph*{Correction des CSU de l'article}

    %   Les CSU présentées à l'équation (43) de cet article contiennent une erreur.

    %   En effet, dans cet article, on change le terme isotrope (\(a_0\)) pour un terme orthotrope (\(a_0^1,a_0^2\)).

    %   L'article énonce alors que des CSU sont
    %   \begin{align*}
        
    %   \end{align*}

    \subsubsection{Nouvelles CSU pour la CI1 qui préservent les CSU de la CI01}
      Dans la partie précédente, nous avons trouvé un certain nombre de CSU pour la CI1 que nous avons toutes jugées insatisfaisantes, car les CSU de la CI01 que nous avions trouvés n'y était pas incluse. Autrement dit, poser \(b=0\) ne redonne pas les CSU de la CI01.

      Nous montrons qu'il est possible d'obtenir pour la CI1 des CSU qui ont cette propriété.

      \paragraph{Cas \(a_1\not=0\)}
        ~
        \begin{prop}
          Des CSU pour la CI1 sont
          \begin{align}
            \Re\left(\frac{b}{a_1}\right) \ge 0 \\
            \Re\left(a_0\right) \ge 0 \\
            \Re\left(\left(a_1-a_0 b\right)\frac{\conj{a_1}}{a_1}\right) \le 0
          \end{align}
        \end{prop}
        \begin{proof}
          En supposant \(a_1 \not=0\), on développe 
          \begin{align*}
            (a_0 + a_1 \LL)^{-1}(1 + b \LL) &= (a_0 + a_1 \LL)^{-1} + b (a_0 + a_1 \LL)^{-1}\LL
            \\
            &=(a_0 + a_1 \LL)^{-1} + \frac{b}{a_1} (a_0 + a_1 \LL)^{-1}(a_0 + a_1\LL -a_0)
            \\
            &=\frac{b}{a_1} + \left(1-b\frac{a_0}{a_1}\right)(a_0+a_1 \LL)^{-1}
          \end{align*}

          et ainsi
          \[
            X = \int_\Gamma \left(\left(\frac{b}{a_1}  + \left(1-b\frac{a_0}{a_1}\right)(a_0+a_1 \LL)^{-1}\right)\vE_t\right) \cdot \conj{\vE_t} 
          \]

          On pose
          \begin{align*}
            \vect{D} &= (a_0 + a_1 \LL)^{-1}\vE_t
            \\
            S &= -\int_\Gamma \vect{D} \cdot \LL\conj{\vect{D}} \ge 0
          \end{align*}
          Comme \(\conj{\vE_t} = (\conj{a_0} + \conj{a_1}\LL)\vect{D}\) alors 
          \begin{align*}
            \int_\Gamma (a_0 +a_1 \LL) ^{-1}\vE_t\cdot \conj{\vE_t} 
            &= \int_\Gamma \vect{D}\cdot (\conj{a_0} + \conj{a_1}\LL)\vect{D} 
            \\
            &= \conj{a_0} \norm{\vect{D}}^2 - \conj{a_1} S
          \end{align*}
          et l'on peut alors écrire

          \begin{equation}
            \label{eq:unicite:form_var:decomp_cgu_ci1_a1}
            X = \frac{b}{a_1}\norm{\vE_t}^2  + \left(1-b\frac{a_0}{a_1}\right)\left(\conj{a_0} \norm{\vect{D}}^2 - \conj{a_1} S\right)
          \end{equation}
          Il suffit alors d'imposer la positivité de la partie réelle de chaque terme pour impliquer \(\Re(X)\ge0\)
          \begin{align}
            a_1 \not = 0\\
            \Re\left(\frac{b}{a_1}\right) \ge 0 \\
            \Re\left(a_0\right) \ge 0 \\
            \Re\left(\left(a_1-a_0 b\right)\frac{\conj{a_1}}{a_1}\right) \le 0
          \end{align}
        \end{proof}

        Quand \(b=0\), on vérifie les CSU dans la CI01.% La condition \(a_1\not=0\) n'est pas contraignante car c'est une propriété presque toujours vrai et donc ce jeu est un meilleur choix dans le cadre d'un code numérique.

      \paragraph{Cas \(a_1=0\)}
        ~
        
        On complète les CSU précédentes
        \begin{prop}
          Des CSU sont
          \begin{align}
            a_1 = 0 \\
            a_0 \not= 0\\
            \Re\left(a_0\right) \ge 0\\
            \Re\left(b\conj{a_0}\right) \le 0
          \end{align}
        \end{prop}
        \begin{proof}
          De la définition de la CIOE, on déduit que
          \[
            X = \int_\Gamma \left( \frac{1}{a_0}\left(1+b\LL\right)\vE_t\right) \cdot \conj{\vE_t} 
          \]

          On a alors
          \begin{equation}
            \label{eq:unicite:form_var:decomp_cgu_ci1_a1_nul}
            X = \frac{1}{a_0}\norm{\vE_t}^2  - \frac{b}{a_0}G
          \end{equation}

          Il suffit alors d'imposer la positivité de la partie réelle de chaque terme pour impliquer \(\Re(X)\ge0\)
          \begin{align}
            a_1 = 0 \\
            a_0 \not= 0\\
            \Re\left(a_0\right) \ge 0\\
            \Re\left(b\conj{a_0}\right) \le 0
          \end{align}
        \end{proof}

        On remarque que si \(b=0\) s’annule, comme on a supposé \(a_1=0\), on retombe bien sur la CSU de la CI0.


      %%%%%%%%%%%%%%%%%%%%%%%%%%%%%%%%%%%%%%%%%%%%%%%%%%%%%%%%%%%%%%%%%%%%%%%%%%%%%%%%%%%%%%%%%%%%%%%%%%%%%%%%%
