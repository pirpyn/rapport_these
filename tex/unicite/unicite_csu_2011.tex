\section[Des CSU pour les CIOE de Stupfel et Poget 2011]{Des conditions suffisante pour les CIOE de \cite{stupfel_sufficient_2011}}

  Nous définissons les \glspl{acr-cioe} comme une condition limite liant \(\vE_t\) et \(\vn\pvect\vH\) sur \(\Gamma\). L'existence des CIOE est en dehors du cadre de cette thèse, nous ne ferons donc qu'utiliser des CIOE existantes.

  Grâce à ces CIOE, nous allons établir des conditions suffisante qui impliquent la \gls{acr-cgu} \eqref{eq:unicite:form_var:cgu}. Par leur nature suffisante, il n'y pas un unique jeu de CSU pour une CIOE donnée. Une difficulté est d'être capable de juger si ce jeu est satisfaisant, et si ce n'est pas le cas, être capable de proposer un autre jeu.

  Les CIOE de \cite{stupfel_sufficient_2011} font intervenir l'opérateur de Hodge \(\mathcal{L}\), commençons par rappeler son expression et quelques propriétés.

  \begin{defn}
    % Pour tout \(\vu \in (\mathcal{C}^\infty(\Gamma))^2\)
    \begin{equation}
      \begin{aligned}
        \LL:(\mathcal{C}^\infty(\Gamma))^2 & \rightarrow & (\mathcal{C}^\infty(\Gamma))^2
        \\
        \vu & \mapsto & \LL(\vu) & = \tgrads{\tdivs \vu} - \tvrots{\tvrots \vu}
      \end{aligned}
    \end{equation}
  \end{defn}

  \begin{prop}
    Par définition, l’opérateur hermitien \(\LL\) est symétrique négatif.

    Pour tous \(\vu,\vv \in (\mathcal C^\infty(\Gamma))^2\)
    \begin{align}
      \int_\Gamma \vu\cdot \LL(\conj{\vv}) &= \int_\Gamma \conj{\vv}\cdot \LL(\vu)
      \\
      \int_\Gamma \vu\cdot \LL(\conj{\vu}) &\le 0
      \label{eq:hodge:negatif}
    \end{align}
  \end{prop}

  On rappelle que l'on veut trouver des conditions permettant de garantir \eqref{eq:unicite:form_var:cgu} soit \(\Re(X)\ge0\) où \(X = \int_\Gamma \vJ \cdot \conj{\vE_t}\).

  %%%%%%%%%%%%%%%%%%%%%%%%%%%%%%%%%%%%%%%%%%%%%%%%%%
  \subsection{CSU de la CI0}
    Considérons la condition d’impédance de Leontovich, la \hyperlink{ci0}{CI0} caractérisé par
    \begin{align}
      \label{eq:unicite:ci0}
      \vE_t = a_0 \vJ && \forall a_0 \in \CC
    \end{align}

    \begin{defn}
      \label{def:csu:ci0}
      On définit le sous-espace de \(\CC\)
      \begin{equation*}
        \CSU{CI0} = \lbrace a_0 \in \CC; \Re(a_0) \ge 0 \rbrace
      \end{equation*}
    \end{defn}

    \begin{prop}[Une CSU pour la CI0]
      Il suffit que
      \begin{equation*}
        a_0 \in \CSU{CI0}
      \end{equation*}
      pour que \(\Re(X)\ge 0\), ce qui entraîne l'unicité.
    \end{prop}
    \begin{proof}
      On a \( X = \conj{a_0}\norm{\vJ}^2\) donc \(\Re(X) = \Re(a_0)\norm{\vJ}^2 \)
    \end{proof}
  %%%%%%%%%%%%%%%%%%%%%%%%%%%%%%%%%%%%%%%%%%%%%%%%
  \subsection{CSU de la CI01}
    Considérons la condition d’impédance \hyperlink{ci01}{CI01}:
    \begin{align}
      \label{eq:unicite:ci01}
      \vE_t = (a_0 + a_1 \LL)\vJ && \forall (a_0, a_1) \in \CC^2
    \end{align}

    \begin{defn}
      \label{def:csu:ci01}
      On définit le sous-espace de \(\CC^2\)
      \begin{equation*}
        \CSU{CI01} = \left\lbrace (a_0,a_1) \in \CC^2; \begin{matrix}
        \Re\left(a_0\right) \ge 0
        \\
        \Re\left(a_1\right) \le 0
        \end{matrix}
        \right\rbrace
      \end{equation*}
    \end{defn}

    \begin{prop}[Une CSU pour la CI01]
      Il suffit que
      \begin{equation*}
        (a_0,a_1) \in \CSU{CI01}
      \end{equation*}
      pour que \(\Re(X)\ge 0\), ce qui entraîne l'unicité.
    \end{prop}
    \begin{proof}
      On a
      \begin{align*}
        X & = \conj{a_0} \norm{\vJ} ^2 + \conj{a_1}\int_\Gamma \vJ\cdot \LL\conj{\vJ}
        \intertext{donc}
        \Re(X) & = \Re{(a_0)} \norm{\vJ} ^2 + \Re{(a_1)}\int_\Gamma \vJ\cdot \LL\conj{\vJ}
      \end{align*}
      Si on suppose \((a_0,a_1) \in \CSU{CI01}\), alors d'après \eqref{eq:hodge:negatif}, tous les termes sont positifs.
    \end{proof}

  %%%%%%%%%%%%%%%%%%%%%%%%%%%%%%%%%%%%%%%%%%%%%%%%
  \subsection{CSU de la CI1}

    Considérons la condition d’impédance \hyperlink{ci1}{CI1}:
    \begin{align}
    \label{eq:unicite:ci1}
      (1 + b \LL) \vE_t = (a_0 + a_1 \LL) \vJ && \forall (a_0, a_1,b) \in \CC^3
    \end{align}

    Pour cette CIOE, nous démontrons qu'il existe plusieurs CSU.

    \subsubsection{CSU de \cite{stupfel_sufficient_2011}}

    On définit \(\Delta: \CC^3 \rightarrow \CC; \Delta(a_0,a_1,b) = a_1 - a_0\conj{b}\) et par abus de notation, on omet les variables \((a_0,a_1,b)\)
    \begin{equation}
       \Delta(a_0,a_1,b) \equiv \Delta
    \end{equation}

    \begin{defn}
      \label{def:csu:ci1-1}

      On définit le sous-espace de \(\CC^3\)
      \begin{equation*}
        \CSU[1]{CI1} = \left\lbrace 
        \begin{matrix}
        (a_0,a_1,b) \in \CC^3
        \\
        \Re(\Delta) = 0
        \\
        \Im(\Delta) = 0
        \\
        \Im(\Delta)\Im(b) \ge 0
        \\
        \Im(\Delta )\Im(a_1\conj{a_0})\ge 0
        \end{matrix}
        \right\rbrace
      \end{equation*}
    \end{defn}

    \begin{prop}[Une première CSU pour la CI1]
      Il suffit que
      \begin{equation*}
        (a_0,a_1,b) \in \CSU[1]{CI1}
      \end{equation*}
      pour que \(\Re(X)\ge 0\), ce qui entraîne l'unicité.
    \end{prop}

      \begin{proof}
        On utilise l'identité \((a_1-a_0\conj{b}) = (a_1(1+\conj{b}\LL) - \conj{b}(a_0+a_1\LL))\):
        \begin{align*}
          (a_1-a_0\conj{b})X &= \int_\Gamma \left(a_1(1+\conj{b}\LL) \vJ\right)\cdot\conj{\vE_t} - \left(\conj{b}(a_0+a_1 \LL)\vJ\right)\cdot\conj{\vE_t}
          \intertext{Comme l'opérateur \(\LL\) est symétrique}
          (a_1-a_0\conj{b})X &= \int_\Gamma \left(a_1(1+\conj{b}\LL) \conj{\vE_t}\right)\cdot\vJ - \int_\Gamma \left(\conj{b}(a_0+a_1 \LL)\vJ\right)\cdot\conj{\vE_t}
          \intertext{Par définition de la CI1}
          (a_1-a_0\conj{b})X &= \int_\Gamma \left(a_1(\conj{a_0}+\conj{a_1}\LL) \conj{\vJ}\right)\cdot\vJ - \int_\Gamma \left(\conj{b}(1+b \LL)\vE_t\right)\cdot\conj{\vE_t} \\
          (a_1-a_0\conj{b})X &= a_1\conj{a_0} \norm{ \vJ }^2 + |a_1|^2 \int_\Gamma \vJ \LL \conj{\vJ} - \conj{b} \norm{ \vE_t }^2 - |b|^2 \int_\Gamma \vE_t \LL \conj{\vE_t} 
        \end{align*}

        On pose 
        \begin{align*}
          F &= -\int_\Gamma \vJ \LL \conj{\vJ} \ge 0
          \\
          G &= -\int_\Gamma \vE_t \LL \conj{\vE_t} \ge 0
        \end{align*}

        Explicitons la partie imaginaire de \( (a_1-a_0\conj{b})X\),
        \begin{align*}
          % \Re(\Delta)\Re(X) - \Im(\Delta)\Im(X) &= \Re(a_1\conj{a_0}) \norm{\vJ}^2 - \Re(\conj{b})\norm{\vE_t}^2 -|a_1|^2 F + |b|^2 G \\
          \Im(\Delta)\Re(X) + \Re(\Delta)\Im(X) &= \Im(a_1\conj{a_0}) \norm{\vJ}^2 - \Im(\conj{b})\norm{\vE_t}^2
          \intertext{En supposant \(\Re(\Delta)= 0\) et \(\Im(\Delta) \not = 0\) nous pouvons conclure car}
          \Im(\Delta)^2\Re(X) &= \Im(\Delta)\Im(a_1\conj{a_0}) \norm{\vJ}^2 - \Im(\Delta)\Im(\conj{b})\norm{\vE_t}^2
        \end{align*}
        Dans le cas où \(\Im(\Delta)\not=0\), il suffit d'imposer que tous ces termes soient positifs pour que \(\Re(X)\) le soit.
      \end{proof}

      On remarque que
      \begin{align}
        \CSU[1]{CI1}\cap(\CC^2 \times \lbrace0\rbrace) &\subsetneq (\CSU{CI01}\times\lbrace0\rbrace)
        \intertext{C'est insatisfaisant, car pour des fonctions infiniment régulières, la CIOE CI1 avec \(b=0\) est équivalente à la CI01.
        Plus précisément, soit \(S = \lbrace (a_0,a_1) \in \CC^2; \Re(a_1)=0 \rbrace\), on a}
        \CSU[1]{CI1}\cap(\CC^2 \times \lbrace0\rbrace) &= ((\CSU{CI01}\cap S) \times\lbrace0\rbrace) 
      \end{align}

    \subsubsection{CSU de \cite{stupfel_implementation_2015}}

    \begin{defn}
      \label{def:csu:ci1-2}

      On définit le sous-espace de \(\CC^3\)
      \begin{equation*}
        \CSU[2]{CI1} = \left\lbrace 
        \begin{matrix}
        (a_0,a_1,b) \in \CC^3
        \\
        \Delta = 0
        \\
        \Im(\Delta)\Im(b) \ge 0
        \\
        \Im(\Delta )\Im(a_1\conj{a_0})\ge 0
        \end{matrix}
        \right\rbrace
      \end{equation*}
    \end{defn}

    \begin{prop}[Une deuxième CSU pour la CI1]
      Il suffit que
      \begin{equation*}
        (a_0,a_1,b) \in \CSU[2]{CI1}
      \end{equation*}
      pour que \(\Re(X)\ge 0\), ce qui entraîne l'unicité.
    \end{prop}

      \begin{proof}
        Intégrons la CI1 \eqref{eq:unicite:ci1} contre \(\conj{\vE_t}\)
        \begin{align*}
          \norm{\vE_t}_2^2 - bG &= a_0 X + a_1 \int_\Gamma \vJ\cdot \LL\conj{\vE_t}
          \intertext{où \(G\ge 0\).
          Intégrons la CI1 conjuguée contre \({\vJ}\)
          }
          X + \conj{b}\int_\Gamma \vJ\cdot \LL\conj{\vE_t}  &= \conj{a_0}\norm{\vJ}_2^2 - \conj{a_1}F
        \end{align*}
        où \(F \ge 0\).

        On cherche à résoudre le problème suivant:
        \[
          \begin{bmatrix}
            1 & \conj{b} \\
            a_0 & a_1
          \end{bmatrix}
          \begin{bmatrix}
            X \\
            \int_\Gamma \vJ\cdot \LL\conj{\vE_t}
          \end{bmatrix}
          =
          \begin{bmatrix}
            \conj{a_0}\norm{\vJ}_2^2 - \conj{a_1}F \\
            \norm{\vE_t}_2^2 - bG
          \end{bmatrix}
        \]

        Si la matrice ci-dessus est inversible, on note son déterminant \(\Delta\) introduit dans la partie précédente et alors
        \[
        \begin{bmatrix}
          X
          \\
          \int_\Gamma \vJ\cdot \LL\conj{\vE_t}
        \end{bmatrix}
        =\frac{1}{\Delta}
        \begin{bmatrix}
          a_1 & -\conj{b} \\
          -a_0 & 1
        \end{bmatrix}
        \begin{bmatrix}
          \conj{a_0}\norm{ \vJ }_2^2 - \conj{a_1}F \\
          \norm{ \vE_t }_2^2 - bG
        \end{bmatrix}
        \]
        On ne s’intéresse qu'à la première composante de vecteurs
        \begin{align*}
          X &=  
          \frac{a_1\conj{a_0}}{\Delta}\norm{\vJ^2}_2^2 
          - \frac{\conj{b}}{\Delta}\norm{\vE_t}_2^2 
          - \frac{|a_1|^2}{\Delta}F
          + \frac{|b|^2}{\Delta}G
        \intertext{ et donc }
          \Re(X) &= 
          \Re\left(\frac{a_1\conj{a_0}}{\Delta}\right)\norm{\vJ^2}_2^2 
          - \Re\left({\frac{\conj{b}}{\Delta}}\right)\norm{\vE_t}_2^2 
          - \Re\left(\frac{|a_1|^2}{\Delta}\right)F
          + \Re\left(\frac{|b|^2}{\Delta}\right)G
        \end{align*}

        On impose alors la positivité de tous les termes, ce qui implique que \(\Re(X)\ge0\).

        On obtient
        \begin{align}
          \Re\left(a_0\conj{a_1}\Delta\right) &\ge 0
          \\
          \Re\left(b\Delta\right) &\le 0
          \\
          \Re\left(|a_1|^2\Delta\right) &\le 0
          \\
          \Re\left(|b|^2\Delta\right) &\ge 0
        \end{align}

        Ce qui est équivalent à \((a_0,a_1,b) \in \CSU[1]{CI1}\).

        Si la matrice n'est pas inversible, alors on cherche à résoudre
        \[
          \begin{bmatrix}
            1 & \conj{b} \\
            a_0 & a_0\conj{b}
          \end{bmatrix}
          \begin{bmatrix}
            \int_\Gamma \vJ \cdot \conj{\vE_t} \\
            \int_\Gamma \vJ \cdot \LL\conj{\vE_t}
          \end{bmatrix}
          =
          \begin{bmatrix}
            \conj{a_0}\norm{\vJ}_2^2 - \conj{a_0}bF
            \\
            \norm{\vE_t}_2^2 - bG
          \end{bmatrix}
        \]

        Le noyau de la matrice est alors \(\Vect{\begin{bmatrix}\conj{b}\\-1\end{bmatrix}}\) dont l'orthogonal est  \(\Vect{\begin{bmatrix}1\\\conj{b}\end{bmatrix}}\).
        Pour tout \(\int_\Gamma \vJ\cdot \LL\conj{\vE_t} = \conj{b} \int_\Gamma \vJ \cdot \conj{\vE_t} \), on a unicité des solutions. On déduit alors que

        \[
          (1 + \conj{b}^2) X = \conj{a_0} \norm{\vJ}_2^2 - \conj{a_0}bF
        \]

        Il suffit d'imposer la positivité de tous les termes pour conclure

        \begin{align}
          \Delta = 0 \\
          \Re\left(\frac{\conj{a_0}}{1 + \conj{b}^2}\right) &\ge 0 \\
          \Re\left(\frac{\conj{a_0}b}{1 + \conj{b}^2}\right) &\le 0
        \end{align}

      \end{proof}

      Nous referons à ces conditions comme les CSU-1-2. Les CSU-1-2 sont un sous-espace de \(\CC^3\). Comme pour les CSU-1-1, on remarque que l'intersection de ce sous-espace avec \(\CC^2 \times \lbrace0\rbrace\) (quand \(b=0\)) est inclus dans l'espace des \(\text{CSU-01}\times\lbrace0\rbrace\) mais pas égal à cause de la condition d'égalité. Ce sont encore des CSU trop contraignantes.
    % \paragraph*{Correction des CSU de l'article}

    %   Les CSU présentées à l'équation (43) de cet article contiennent une erreur.

    %   En effet, dans cet article, on change le terme isotrope (\(a_0\)) pour un terme orthotrope (\(a_0^1,a_0^2\)).

    %   L'article énonce alors que des CSU sont
    %   \begin{align*}
        
    %   \end{align*}

    \subsubsection{De meilleurs CSU pour la CI1}
      Dans la partie précédente, nous avons trouvé un certain nombre de CSU pour la CI1 que nous avons toutes jugées insatisfaisantes, car les CSU de la CI01 que nous avions trouvés n'y était pas incluse.% Autrement dit, poser \(b=0\) ne redonne pas les CSU de la CI01.

      Nous montrons qu'il est possible d'obtenir pour la CI1 des CSU qui ont cette propriété.

      \begin{prop} Des CSU sont
          \begin{align}
            \Re\left(b\conj{a_1}\right) \ge 0
            \\
            \Re\left(a_0\right) \ge 0
            \\
            \Re\left(\left(a_1-a_0 b\right){\conj{a_1}^2}\right) \le 0
            \\
            \Re\left(b\conj{a_0}\right) \le 0
          \end{align}
      \end{prop}


        \paragraph{Cas \(a_1\not=0\)}
          ~
          \begin{prop}
            Des CSU pour la CI1 sont
            \begin{align}
              \Re\left({b}\conj{a_1}\right) \ge 0
              \\
              \Re\left(a_0\right) \ge 0
              \\
              \Re\left(\left(a_1-a_0 b\right){\conj{a_1}^2}\right) \le 0
            \end{align}
          \end{prop}
          \begin{proof}
            En supposant \(a_1 \not=0\), on développe 
            \begin{align*}
              (a_0 + a_1 \LL)^{-1}(1 + b \LL) &= (a_0 + a_1 \LL)^{-1} + b (a_0 + a_1 \LL)^{-1}\LL
              \\
              &=(a_0 + a_1 \LL)^{-1} + \frac{b}{a_1} (a_0 + a_1 \LL)^{-1}(a_0 + a_1\LL -a_0)
              \\
              &=\frac{b}{a_1} + \left(1-b\frac{a_0}{a_1}\right)(a_0+a_1 \LL)^{-1}
            \end{align*}

            et ainsi
            \begin{align*}
              X &= \int_\Gamma (a_0 + a_1 \LL)^{-1}(1 + b \LL)\vE_t \cdot \conj{\vE_t} 
              \\
              &= \int_\Gamma \left(\frac{b}{a_1}  + \left(1-b\frac{a_0}{a_1}\right)(a_0+a_1 \LL)^{-1}\right)\vE_t \cdot \conj{\vE_t} 
            \end{align*}

            On pose
            \begin{align*}
              \vect{D} &= (a_0 + a_1 \LL)^{-1}\vE_t
              \\
              S &= -\int_\Gamma \vect{D} \cdot \LL\conj{\vect{D}} \ge 0
            \end{align*}
            Comme \(\conj{\vE_t} = (\conj{a_0} + \conj{a_1}\LL)\conj{\vect{D}}\) alors 
            \begin{align*}
              \int_\Gamma (a_0 +a_1 \LL) ^{-1}\vE_t\cdot \conj{\vE_t} 
              &= \int_\Gamma \vect{D}\cdot (\conj{a_0} + \conj{a_1}\LL)\conj{\vect{D}}
              \\
              &= \conj{a_0} \norm{\vect{D}}^2 - \conj{a_1} S
            \end{align*}
            et l'on peut alors écrire

            \begin{equation}
              \label{eq:unicite:form_var:decomp_cgu_ci1_a1}
              X = \frac{b}{a_1}\norm{\vE_t}^2  + \left(1-b\frac{a_0}{a_1}\right)\left(\conj{a_0} \norm{\vect{D}}^2 - \conj{a_1} S\right)
            \end{equation}
            Il suffit alors d'imposer la positivité de la partie réelle de chaque terme pour impliquer \(\Re(X)\ge0\).
            \begin{align}
              \Re\left(\frac{b}{a_1}\right) \ge 0
              \\
              \Re\left(\left(a_1-a_0 b\right)\frac{\conj{a_0}}{a_1}\right) \ge 0
              \\
              \Re\left(\left(a_1-a_0 b\right)\frac{\conj{a_1}}{a_1}\right) \le 0
            \end{align}
            ce qui est équivalent à 
            \begin{align}
              \Re\left(b\conj{a_1}\right) \ge 0
              \\
              \Re\left(a_0\right) \ge 0
              \\
              \Re\left(\left(a_1-a_0 b\right){\conj{a_1}^2}\right) \le 0
            \end{align}
          \end{proof}
          Quand \(b = 0\) et \(a_0,a_1\) sont quelconques, on obtient les conditions suffisante \(\Re(a_0)\ge 0 \) et \(\Re(a_1)\le 0 \) qui sont bien les conditions CSU-01.

          Quand \(a_1 \rightarrow 0\) et \(a_0,b\) quelconques, la seule condition qui reste est \(\Re(a_0)\ge 0\) qui est bien la condition CSU-0, mais aucune condition sur \(b\). Il faut donc déterminer les CSU de la CIOE CI1 modifié avec \(a_1 = 0\).
          %Quand \(b=0\), on vérifie les CSU dans la CI01.% La condition \(a_1\not=0\) n'est pas contraignante car c'est une propriété presque toujours vrai et donc ce jeu est un meilleur choix dans le cadre d'un code numérique.

        \paragraph{Cas \(a_1=0\)}
          ~
          
          On a donc la CIOE CI1-01
          \begin{align*}
            \left(1 + b \LL \right) \vE_t = a_0 \vJ
          \end{align*}

          \begin{prop}
            Des CSU sont
            \begin{align}
              a_0 \not= 0\\
              \Re\left(a_0\right) \ge 0\\
              \Re\left(b\conj{a_0}\right) \le 0
            \end{align}
          \end{prop}
          \begin{proof}
            De la définition de la CIOE, on déduit que
            \[
              X = \int_\Gamma \left( \frac{1}{a_0}\left(1+b\LL\right)\vE_t\right) \cdot \conj{\vE_t} 
            \]

            On a alors
            \begin{equation}
              \label{eq:unicite:form_var:decomp_cgu_ci1_a1_nul}
              X = \frac{1}{a_0}\norm{\vE_t}^2  - \frac{b}{a_0}G
            \end{equation}

            Il suffit alors d'imposer la positivité de la partie réelle de chaque terme pour impliquer \(\Re(X)\ge0\)
            \begin{align}
              a_0 \not= 0
              \\
              \Re\left(a_0\right) \ge 0
              \\
              \Re\left(b\conj{a_0}\right) \le 0
            \end{align}
          \end{proof}

          On remarque que si \(b=0\), alors la seule condition à vérifier est \(\Re(a_0) \ge 0\) qui est la CSU-0.

          On nommera ces CSU les CSU-1.

      %%%%%%%%%%%%%%%%%%%%%%%%%%%%%%%%%%%%%%%%%%%%%%%%%%%%%%%%%%%%%%%%%%%%%%%%%%%%%%%%%%%%%%%%%%%%%%%%%%%%%%%%%
