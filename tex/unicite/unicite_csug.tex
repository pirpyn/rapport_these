\section[Une CSU des solutions du problème de Maxwell extérieur]{Une condition suffisante pour l'unicité des solutions du problème de Maxwell extérieur}


  On s’intéresse à la propagation des ondes électromagnétiques à l'extérieur d'un objet fermé borné régulier. Les champs associés sont solutions des équations de Maxwell\footnote{voir annexe \ref{sec:annex:maxwell_equation}} harmonique \(e^{i\w t}\) :

  Soit \(\OO\) l'objet et \(\Gamma=\partial\OO\) sa frontière , on décompose donc tout l'espace entre l'intérieur et l'extérieur 
  \[
    \RR^3 = \OO\cap\OO^c
  \]
  On définit en tout point de \(\Gamma\), supposée régulière, \(\vn\) la normale unitaire sortante à \(\OO\).
 Dans \(\OO^c\) on définit \(k_0=\w / c \in \RR_+^*\), le nombre d'onde dans le vide où \gls{phy-c} est la célérité d'une onde dans le vide.

 % Soit \((\vE,\vH)\) dans \((\Hrot(\OO^c)) \times (\Hrot(\OO^c))\) tels que:
  Soit \((\vE,\vH)\) dans \((\mathcal{C}^\infty(\OO^c)) \times (\mathcal{C}^\infty(\OO^c))\) tels que:
  \begin{align}
  \left\lbrace
    \begin{matrix}
      \trot \vE + i k_0 \vH &= 0
      \\
      \trot \vH - i k_0 \vE &= 0
    \end{matrix}
    \right. && \text{dans \(\OO^c\)}
    \label{eq:unicite:maxwell_ext}
  \end{align}

  En l'absence de source, le rotationnel d'un des champs a la même régularité que le champs lui-même, donc ces champs sont \(\mathcal{C}^\infty\).

  Afin de pouvoir étudier plus facilement ce problème, on se ramène à un ouvert borné de la manière suivante. On définit \(B_R\) le boule de rayon \(R\) suffisamment grand pour englober \(\OO\). On cherche alors les solutions dans \(\OO^c_R=\OO^c\cap B_R\) avec \(\partial\OO^c_R = \Gamma\cup S_R\) , où \(S_R\) est la sphère de rayon \(R\).

  Sur cette dernière est définie la condition de rayonnement \eqref{eq:unicite:TR}.
  \begin{equation}
    \label{eq:unicite:TR}
    \begin{aligned}
    \Tr(\vE_t) = - \vn_{S_R} \pvect \vH && \text{sur \(S(0,R)\)}
    \end{aligned}
  \end{equation}
  \(\Tr\) est l'opérateur de capacité défini par \cite[p.~200]{nedelec_acoustic_2001}, \(\vn_{S_R}\) la normale unitaire sortante à \(B(0,R)\).

  Ces équations aboutissent à la forme variationnelle suivante :
  \begin{prop}[Formulation variationnelle du problème de Maxwell extérieur]
    Trouver \(\vE \in \left(\mathcal{C}^\infty(\OO^c_R)\right)\), \(\forall \vect \phi \in \left(\mathcal{C}^\infty(\OO^c_R)\right)\)
    \[
      a(\vE,\vect\phi) = 0
    \]

    où a est une forme sesquilinéaire telle que
    \begin{equation*}
      \begin{aligned}
      a(\vE,\vect\phi) &:=  \frac{1}{ik_0} \int_{\OO^c_R} \trot \vE(\vx) \cdot \trot \conj{\vect{\phi}}(\vx) \dd{x} + ik_0\int_{\OO^c_R}\vE(\vx)\cdot\conj{\vect{\phi}}(\vx) \dd{x}
        \\ 
        & \quad + \int_{S_R} \conj{ \vect \phi }(\vx) \cdot \Tr(\vE_t)(\vx) \dd{S_R}(\vx) - \int_\Gamma \left(\vn \pvect \frac{\trot \vE(\vx)}{ik_0}\right) \cdot \conj{\vect \phi} (\vx) \dd{\Gamma}(\vx)
      \end{aligned}
    \end{equation*}
  \end{prop}

  \begin{proof}
    Partons de la première équation.
    \begin{align*}
          0 & = \frac{\trot \vE}{ik_0} + \vH
          \intertext{Soit \(\vect{\phi} \in \mathcal{C}^\infty(\OO^c_R)\). On déduit la forme variationnelle }
          0 & = \int_{\OO^c_R} \frac{\trot \vE}{ik_0} \cdot \trot \conj{\vect{\phi}} + \int_{\OO^c_R} \vH \cdot \trot\conj{\vect{\phi}}
          \intertext{Utilisons la formule de Green du rotationnel (voir \cite[eq.~(A1.32)]{bladel_electromagnetic_2007}).
          \[
            \int_V \vect{u}\cdot \vrot{\vect{v}} = \int_V \vrot\vect{u}\cdot \vect{v} - \int_{\partial V} \left( \vect{v}\pvect\vect{u} \right)\cdot \vn_V
          \]
          où \(\vn_V\) est la normale unitaire sortante de \(V\). On rappelle que \(\vn_\Gamma\) est la normale sortante de  \(\OO\) et \(\vn_{S_R}\) la normale sortante de \(B_R\).}
          0 & = \int_{\OO^c_R} \frac{\trot \vE}{ik_0} \cdot \trot\vect{\phi} - \int_{S_R} \left( \conj{\vect \phi} \pvect \vH\right)  \cdot \vn_{S_R}
          \\
          & \qquad \qquad + \int_\Gamma \left( \conj{\vect \phi} \pvect \vH\right)  \cdot \vn_\Gamma +\int_{\OO^c_R} \trot\vH \cdot \conj{\vect{\phi}}
          \intertext{On utilise la condition de rayonnement, la deuxième équation de Maxwell et on permute les termes de l'intégrale sur \(\Gamma\)}
          0 & =\int_{\OO^c_R} \frac{\trot \vE}{ik_0}\cdot \trot \conj{\vect{\phi}} + \int_{S_R} \Tr(\vE_t)  \cdot \conj{\vect{\phi}}
          \\
          & \qquad \qquad - \int_\Gamma \left(\vn_\Gamma \pvect \vH \right) \cdot \conj{\vect \phi} +  \int_{\OO^c_R} ik_0 \vE \cdot \conj{\vect{\phi}}
      \end{align*}
  \end{proof}

  On définit en tout point de \(\Gamma\) la trace tangentielle de \(\vH\) que l'on note \(\vJ = \vn_\Gamma \pvect \vH\).

  \begin{prop}[Une \glsentrydesc{acr-csu}]~\\
    Si la condition suivante est vérifiée
    \begin{equation}
      \label{eq:unicite:form_var:cgu}
      \Re\left(\int_\Gamma \vJ(\vx) \cdot \conj{\vE_t}(\vx) \dd{\Gamma}(\vx)\right) \ge 0
    \end{equation}
    alors l'unicité des solutions du problème de Maxwell extérieur \eqref{eq:unicite:maxwell_ext} avec la condition de rayonnement \eqref{eq:unicite:TR} est assurée.
  \end{prop}

\begin{proof}
  Pour démontrer cela, on utilise le Lemme de Rellich, énoncé dans \cite[p.~74]{cessenat_mathematical_1996}:
  \begin{lemme}[Lemme de Rellich]
    Soit \(\OO^c\) un domaine connexe, complément d'un domaine borné, et soit \(u\) satisfaisant
    \begin{subequations}
      \begin{align}
        \Delta u + k^2 u = 0 & &\text{dans \(\OO^c\)}
        \\
        \lim_{R\rightarrow\infty}\int_{S_R} |u(\vx)|^2 ds(\vx) = 0
      \end{align}
    \end{subequations}
    alors \(u=0\) dans \(\OO^c\).
  \end{lemme}
  \begin{proof}
    Voir \cite[p.~74]{cessenat_mathematical_1996}
  \end{proof}

  On définit les quantités suivantes
  \begin{align}
    X &= \int_\Gamma \vJ \cdot \conj{\vE_t}
    \\
    C &= \int_{S_R} \Tr(\vE_t)  \cdot \conj{\vE_t}
  \end{align}

  En reprenant l'expression de la partie réelle de la forme bilinéaire \(a(\vE,\vE)\), on remarque que
  \begin{align}
    \Re(a(\vE,\vE)) & = 0
    \\
    \Re(C) + \Re(X) & = 0
    \intertext{d'après \cite[eq.~5.3.89, p.~200]{nedelec_acoustic_2001}, \(\Re(C)\ge 0\), si on suppose \eqref{eq:unicite:form_var:cgu} alors } 
    \Re(C) = \Re(X) & = 0
  \end{align}

  Puisque \(\Re(C) = 0\) alors d'après \cite[Théorème~5.3.5, p.~200]{nedelec_acoustic_2001}, \(\vE_t = 0\) sur \(S_R\) or d'après le lemme de Rellich, \(\vE = 0\) dans \(\OO^c_R\).

  On conclut donc qu'en l'absence de sources et si on suppose \eqref{eq:unicite:form_var:cgu}, alors la seule solution est un champ nul, ce qui démontre l'unicité (car le problème est linéaire).
\end{proof}
