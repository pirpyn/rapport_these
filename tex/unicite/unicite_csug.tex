\section[Une CSU des solutions du problème de Maxwell extérieur]{Une condition suffisante assurant l'unicité des solutions du problème de Maxwell extérieur}

  On s’intéresse à la propagation des ondes électromagnétiques à l'extérieur d'un objet fermé borné, sans sources. Les champs associés sont solutions des équations de Maxwell harmoniques \(e^{i\w t}\). 

  Soit \(\gls{mat-ga}=\partial\overline{\gls{mat-om}}\) la frontière de \(\OO\), on décompose donc tout l'espace entre l'intérieur et l'extérieur, soit 
  \[
    \RR^3 = \overline{\OO}\cup{\OO}^c.
  \]
  On définit en tout point de \(\Gamma\), supposée régulière, \(\gls{mat-vn}\) la normale unitaire sortante à \(\OO\).
  Dans \(\OO^c\) on définit \(\gls{phy-k0}=\gls{phy-w} / \gls{phy-c} \in \RR_+^*\), le nombre d'onde dans le vide, où \gls{phy-c} est la célérité d'une onde dans le vide.

  % Soit \((\vE,\vH)\) dans \((\Hrot(\OO^c)) \times (\Hrot(\OO^c))\) tels que:
  Nous utilisons les équations de Maxwell-Helmholtz où les inconnues sont le champ électrique \gls{phy-e} et l'induction magnétique \gls{phy-h2} dans le vide, elles se déduisent du système d'équations de Maxwell (voir annexe \ref{sec:annex:maxwell_equation}). 

  Soit \((\gls{phy-e},\gls{phy-h2})\) dans \((\mathcal{C}^\infty(\OO^c)\cap\Sobolev[rot](\OO^c))^2\) tels que
  \begin{align}
  \left\lbrace
    \begin{matrix}
      \vrot \vE + i k_0 \vH &= 0
      \\
      \vrot \vH - i k_0 \vE &= 0
    \end{matrix}
    \right. && \text{dans \(\OO^c\).}
    \label{eq:unicite:maxwell_ext}
  \end{align}

%  En l'absence de source, le rotationnel d'un des champs a la même régularité que le champs lui-même, donc ces champs sont \(\mathcal{C}^\infty\).

  Pour étudier ce problème numériquement, on se ramène à un ouvert borné de la manière suivante :   soit \(R\in\RR\) tel que le domaine \(\overline{\Omega}\) soit inclus dans \(B_R\), la boule de rayon \(R\). On cherche alors les solutions de \eqref{eq:unicite:maxwell_ext} dans \(\OO^c_R=\OO^c\cap B_R\) avec \(\partial\OO^c_R = \Gamma\cup S_R\), \(S_R\) étant la sphère de rayon \(R\) où est définie la condition de rayonnement à l'infini
  \begin{equation}
    \label{eq:unicite:TR}
    \begin{aligned}
    \Tr(\vE_t) = - \vn_{S_R} \pvect \vH && \text{sur \(S_R\).}
    \end{aligned}
  \end{equation}
  \(\Tr\) est l'opérateur de capacité défini par \cite[p.~200]{nedelec_acoustic_2001}, \(\vn_{S_R}\) la normale unitaire sortante à \(B_R\), définie uniquement sur \(S_R\).

  Soit \(V = \lbrace \vect{u} \in (\mathcal{C}^\infty(\OO^c_R)\cap\Sobolev[rot](\OO^c_R) \rbrace\).

  Le problème \eqref{eq:unicite:maxwell_ext} sans sources muni de la condition \eqref{eq:unicite:TR} est équivalent à
  \begin{prop}[Formulation variationnelle du problème de Maxwell extérieur sans sources]
    Trouver \(\vE \in V\),
    \begin{align*}
      a(\vE,\vect\phi) = 0 && \forall \vect \phi \in V,
    \end{align*}

    où a est la forme sesquilinéaire donnée par
    \begin{equation*}
      \begin{aligned}
      a(\vE,\vect\phi) &:=  \frac{1}{ik_0} \int_{\OO^c_R} \trot \vE(\vx) \cdot \trot \conj{\vect{\phi}}(\vx) \dd{x} + ik_0\int_{\OO^c_R}\vE(\vx)\cdot\conj{\vect{\phi}}(\vx) \dd{x}
        \\ 
        & \quad + \int_{S_R} \conj{ \vect \phi }(\vx) \cdot \Tr(\vE_t)(\vx) \dd{S_R}(\vx) - \int_\Gamma \left(\vn \pvect \frac{\trot \vE(\vx)}{ik_0}\right) \cdot \conj{\vect \phi} (\vx) \dd{\Gamma}(\vx).
      \end{aligned}
    \end{equation*}
  \end{prop}

  \begin{proof}
    Nous omettons les dépendances en \(\vx\).
    Partons de la première équation de Maxwell
    \begin{align*}
          0 & = \frac{\trot \vE}{ik_0} + \vH.
          \intertext{En appliquant cette égalité à \(\trot\vect{\phi}\) pour \(\vect{\phi} \in V\), on déduit la forme variationnelle }
          0 & = \int_{\OO^c_R} \frac{\trot \vE}{ik_0} \cdot \trot \conj{\vect{\phi}}\dd{x} + \int_{\OO^c_R} \vH \cdot \trot\conj{\vect{\phi}}\dd{x}, && \forall \vect{\phi} \in V.
          \intertext{Utilisons la formule de Green du rotationnel (voir \cite[eq.~(A1.32)]{bladel_electromagnetic_2007}).
          \[
            \int_V \vect{u}\cdot \trot{\vect{v}}\dd{x} = \int_V \trot\vect{u}\cdot \vect{v}\dd{x} + \int_{\partial V} \left( \vect{v}\pvect\vect{u} \right)\cdot \vn_V \dd{\Gamma(x)},
          \]
          où \(\vn_V\) est la normale unitaire sortante de \(V\). On rappelle que \(\vn_\Gamma\) est la normale sortante de  \(\OO\) et \(\vn_{S_R}\) la normale sortante de \(B_R\).}
          0 & = \int_{\OO^c_R} \frac{\trot \vE}{ik_0} \cdot \trot\vect{\phi}\dd{x} +\int_{\OO^c_R} \trot\vH \cdot \conj{\vect{\phi}}\dd{x}
          \\
          & \qquad \qquad+ \int_{S_R} \left( \conj{\vect \phi} \pvect \vH\right)  \cdot \vn_{S_R}\dd{S_R(x)} - \int_\Gamma \left( \conj{\vect \phi} \pvect \vH\right)  \cdot \vn_\Gamma \dd{\Gamma(x)}, && \forall \vect{\phi} \in V.
          \intertext{On utilise  la deuxième équation de Maxwell, la condition de rayonnement et on permute les termes de l'intégrale sur \(\Gamma\)}
          0 & =\int_{\OO^c_R} \frac{\trot \vE}{ik_0}\cdot \trot \conj{\vect{\phi}}\dd{x}  +  \int_{\OO^c_R} ik_0 \vE \cdot \conj{\vect{\phi}}\dd{x}
          \\
          & \qquad \qquad + \int_{S_R} \Tr(\vE_t)  \cdot \conj{\vect{\phi}}\dd{S_R(x)} + \int_\Gamma \left(\vn_\Gamma \pvect \vH \right) \cdot \conj{\vect \phi}\dd{\Gamma(x)},
          && \forall \vect{\phi} \in V.
      \end{align*}
  \end{proof}

  Puisque le problème n'a pas de sources, l'unique solution du problème si ce dernier est bien posé est \(\vE=\vH=0\).

  On définit en tout point de \(\Gamma\) la trace tangentielle de \(\vH\) que l'on note \(\vJ = \vn_\Gamma \pvect \vH\).

  \begin{prop}[Une \glsentrydesc{acr-csu}]~\\
    Si l'on suppose que
    \begin{equation}
      \label{eq:unicite:form_var:cgu}
      \Re\left(\int_\Gamma \vJ(\vx) \cdot \conj{\vE_t}(\vx) \dd{\Gamma(\vx)}\right) \ge 0,
    \end{equation}
    alors le problème \eqref{eq:unicite:maxwell_ext}  muni de la condition \eqref{eq:unicite:TR} admet \(\vE=\vH=0\) comme unique solution.
  \end{prop}

\begin{proof}
  Pour démontrer cela, on utilise le lemme de Rellich, énoncé dans \cite[p.~74]{cessenat_mathematical_1996}:
  \begin{lemme}[Lemme de Rellich]
    Soit \(\OO^c\) un domaine connexe, complément d'un domaine borné, et soit \(u\) satisfaisant
    \begin{subequations}
      \begin{align}
        \Delta u + k^2 u = 0 & &\text{dans \(\OO^c\)},
        \\
        \lim_{R\rightarrow\infty}\int_{S_R} |u(\vx)|^2 \dd{S_R}(\vx) = 0,
      \end{align}
    \end{subequations}
    alors \(u=0\) dans \(\OO^c\).
  \end{lemme}
  \begin{proof}
    Voir \cite[p.~74]{cessenat_mathematical_1996}.
  \end{proof}

  On définit les quantités suivantes
  \begin{align}
    X &= \int_\Gamma \vJ(\vx) \cdot \conj{\vE_t}(\vx)\dd{\Gamma(\vx)},
    \label{eq:unicite:x}
    \\
    C &= \int_{S_R} \Tr(\vE_t)(\vx)  \cdot \conj{\vE_t}(\vx)\dd{S_R(\vx)}.
  \end{align}

  \begin{align*}
    \intertext{De}
    a(\vE,\vE) &= 0,
    \\
    &= \frac{1}{ik_0} \int_{\OO^c_R} \norm{\trot \vE(\vx)}^2 \dd{x} + ik_0\int_{\OO^c_R}\norm{\vE(\vx)}^2 \dd{x}
     + X + C,
    \intertext{on déduit}
    \Re(a(\vE,\vE)) & = 0,
    \intertext{ce qui se réecrit}
    \Re(C) + \Re(X) & = 0.
    \intertext{Comme d'après \cite[Théorème~5.3.5, p.~200]{nedelec_acoustic_2001} \(\Re(C) \ge 0\), on en déduit} 
    \Re(X) & \le 0.
  \end{align*}
  L'hypothèse \eqref{eq:unicite:form_var:cgu} \(\Re(X) \ge 0\) impose alors \(\Re(X)=0\) et donc
  \begin{align*} 
    \Re(C) &= 0,
    \intertext{donc d'après le théorème 5.3.5 de Nédélec}
    \vE_t &= 0 && \text{sur \(S_R\)},
    \intertext{et d'après le lemme de Rellich}
    \vE &= 0 && \text{dans \(\OO^c_R\)}.
  \end{align*}

  On en déduit \(\vH = 0 \).

  On conclut donc que si on suppose \eqref{eq:unicite:form_var:cgu}, alors en l'absence de sources le couple \((\vE,\vH)=(0,0)\) est l'unique solution, ce qui démontre l'unicité par linéarité.
\end{proof}
