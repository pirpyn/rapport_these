\section[Une CNS pour l'unicité des solutions de Maxwell intérieur]{Une condition nécessaire et suffisante pour l'unicité des solutions du problème intérieur}
  \subsection{Cas général: alternative de Fredholm}

    Cette partie détaille l'introduction à l'opérateur de Calderón du chapitre \ref{sec:context_math}.

    % \begin{REM}
    %   Pourquoi cette partie ne serait-elle pas résumée au début afin d'introduire.
    % \end{REM}
    % \begin{REP}
    %   Au début de la thèse, introduction, chapitre ?
    % \end{REP}

    Pour exprimer l'opérateur de Calderón,
    % \begin{REM}
    %   Calderón ou impédance
    % \end{REM}
    % \begin{REP}
    %   Je préfère impédance, ça m'évite de parler d'une théorie que je ne connais pas.
    % \end{REP}

    nous devons exprimer les champs dans l'objet. Il apparaît nécessaire que ces derniers soient donc bien définis.

    Nous exhibons une condition nécessaire et suffisante d'unicité des solutions du problème de Maxwell-Helmholtz  \eqref{eq:unicite:maxwell_int} dans \(\OO\), objet multicouche autour d'un \gls{acr-cep}, c'est-a-dire tel que \(\partial \OO = \Gamma \cup \Gamma_0\) où \(\Gamma\) est la surface extérieure et \(\Gamma_0\) la surface intérieure. Le problème de Maxwell harmonique en \(e^{i\w t}\) s'exprime

    \begin{align}
    \left\lbrace
      \begin{matrix}
        \vrot \vE(\vx) + i k(\vx)\eta(\vx) \vH(\vx) &= 0
        \\
        \vrot \vH(\vx) - i k(\vx)\eta(\vx)^{-1} \vE(\vx) &= 0
      \end{matrix}
      \right. && \text{dans \(\OO\).}
      \label{eq:unicite:maxwell_int}
    \end{align}

    Nous rappelons le résultat général de \cite[Théorème~8, p.~111]{cessenat_mathematical_1996}.
    \begin{thm}[Alternative de Fredholm]
      Les champs \(\vE,\vH\) solutions de \eqref{eq:unicite:maxwell_int} avec conditions aux limites de Dirichlet \(\vE_{|\Gamma} = 0\), \(\vE_{|\Gamma_0} = 0\) sont soit uniques si \(\w^2\eps\mu\) n'est pas une valeur propre du Laplacien vectoriel, soit ces champs sont engendrés par les vecteurs propres.
    \end{thm}
    Nous renvoyons à l'ouvrage cité précédemment pour la démonstration, en dehors du cadre de cette thèse.
    D'après ce résultat, nous pouvons avoir des résonances pour le problème intérieur, c'est-à-dire des solutions non nulles du problème sans sources donc non uniques.

    % \begin{REM}
    %   Parler de Calderón générale page 111 112.
    % \end{REM}

  \subsection{Simplification de la géométrie  pour déduire l'opérateur de Calderón.}
    Cette thèse s'attache à déterminer les coefficients des \glspl{acr-cioe} des sections précédentes.
    Nous devons être capables d'exprimer l'opérateur d'impédance exact pour l'approcher par les CIOE.
    Il faut donc que les champs soient déterminés de façon unique pour exprimer cet opérateur.
    Nous ne pouvons nous satisfaire de l'alternative de Fredholm et devons exprimer une condition plus pratique pour déterminer s'il y a unicité.
    Nous allons alors approcher l'objet par une forme simple, mais avec les mêmes caractéristiques pour l'empilement.
    Ainsi nous étudierons trois objets modèles pour le calcul de l'opérateur de Calderón ainsi que pour le calcul des coefficients des CIOE.
    % \begin{REM}
    %   Ce n'est pas une approximation de l'objet mais des objets modèles.
    % \end{REM}
    % \begin{REP}
    %   Oui et non. 
    %   On a bien des objets modèles type sphère, cône-sphère lors du calcul intégral, mais en plus, lors du calcul des coefficients des CIOE, ces objets sont localement approchés par une des 3 géométries. 
    %   Ainsi, l'objet modèle sphère est localement approché par un plan. 
    %   Tout dépend du contexte.
    %   Dans le cadre du calcul des coefficients de CIOE, on approche l'objet modèle par un autre si besoin.
    % \end{REP}
    \begin{enumerate}
      \item Un objet plan infini 
      \item Un objet cylindrique
      \item Un objet sphérique
    \end{enumerate}
   
    Cependant, nous utiliserons des objets différents (cylindre à bouts arrondis, sphère-cône) lors du calcul par équations intégrales, utilisant les coefficients calculés sur ces objets modèles.

    Nous désirons une condition ne dépendant pas de la géométrie pour garantir l'unicité et l'alternative de Fredholm n'étant donc pas utilisable,
    % \begin{REM}
    %   Que veux-tu dire par là ? Que \(\eps\mu\w^2\) n'est jamais valeur propre du Laplacien ?
    % \end{REM}
    % \begin{REP}
    %   Que dire qu'il y a unicité ou que les résonances sont définies par un espace vectoriel n’empêche pas le code de minimisation de planter à cause des singularités quand on tombe sur ces résonances.
    % \end{REP}
    nous allons simplifier la géométrie du problème pour aboutir à une condition nécessaire d'unicité.

    L'objet est représenté schématiquement:
    \begin{figure}[h!btp]
        \centering
        \tikzsetnextfilename{plan_n_couches}            
        \begin{tikzpicture}
            \tikzmath{
    \largeur = 6;
    \hauteur = 0.5;
    \milieu = 1.3;
    \xC = \largeur;
    \xA = 0;
}

%% 1ere couche
\tikzmath{
    \yC = \hauteur;
    \yA = 0;
}

\coordinate (A) at (\xA,\yA);
\coordinate (B) at (\xA,\yC);
\coordinate (C) at (\xC,\yC);

\draw ($(B)!0.5!(C)$) node [above] {vide};


\fill [lightgray] (A) rectangle (C);
\draw ($(A)!0.5!(C)$) node {$\eps_n,\mu_n,d_n$};
\draw (B) -- (C) node [right] {$e_3 = 0$};

%% Des couches
\tikzmath{
    \yC = \yC - \hauteur;
    \yA = \yA - \milieu*\hauteur;
}

\coordinate (A) at (\xA,\yA);
\coordinate (B) at (\xA,\yC);
\coordinate (C) at (\xC,\yC);

\fill [lightgray]    (A) rectangle (C);
\fill [pattern=dots] (A) rectangle (C);
\draw (B) -- (C);

%% N ieme couche
\tikzmath{
    \yC = \yC - \milieu*\hauteur;
    \yA = \yA - \hauteur;
}

\coordinate (A) at (\xA,\yA);
\coordinate (B) at (\xA,\yC);
\coordinate (C) at (\xC,\yC);
\fill [lightgray] (A) rectangle (C);
\draw ($(A)!0.5!(C)$) node {$\eps_1,\mu_1,d_1$};
\draw (B) -- (C);

%% Le repère
\tikzmath{
    \xD = \xC + 0.5;
}

\coordinate (n) at (\xD,\yA);
\draw [->] (n) -- ++(1,0) node [at end, right] {$\v{e_1}$};
\draw [->] (n) -- ++(0,1) node [at end, right] {$\v{e_3}$};

\draw (n) circle(0.1cm) node [below=0.1cm] {$\v{e_2}$};
\draw (n) +(135:0.1cm) -- +(315:0.1cm);
\draw (n) +(45:0.1cm) -- +(225:0.1cm);

%% Le conducteur
\tikzmath{
    \yC = \yC - \hauteur;
    \yA = \yA - 0.5*\hauteur;
}

\coordinate (A) at (\xA,\yA);
\coordinate (B) at (\xA,\yC);
\coordinate (C) at (\xC,\yC);
\draw (B) -- (C);

\fill [pattern=north east lines] (A) rectangle (C);



        \end{tikzpicture}
    \end{figure}

    Les constantes relatives \(\eps,\mu\) (ou \(k,\eta\)) sont constantes par morceaux en fonction de \(z\).
    \begin{prop}
      \label{prop:unicite:interieur:postulat:multi-couche}
      Le problème Maxwell-Helmholtz \eqref{eq:unicite:maxwell_int} avec conditions aux limites de Dirichlet homogènes \(\vE_{|z=z_n} = 0\), \(\vE_{|z=z_0} = 0\) admet une unique solution pour deux couches de matériaux si et seulement si pour tout \((k_x,k_y) \in \RR^2\) tel que \(k_x^2 + k_y^2 \not = 0\)
      \begin{align*}
        \begin{cases}
        \kk_2k_1\eta_2\tan\left(\kk_2(z_2-z_1)\right) & \not = \kk_1k_2\eta_1\tan\left(\kk_1(z_0-z_1)\right),
        \\
        \kk_1k_2\eta_2\tan\left(\kk_2(z_2-z_1)\right) & \not = \kk_2k_1\eta_1\tan\left(\kk_1(z_0-z_1)\right).
        \end{cases}
      \end{align*}
    \end{prop}
    Ces relations ne contiennent pas les conditions de résonance pour chacune des couches.
    \begin{proof}
      % Nous ne traiterons dans cette thèse que l'exemple de deux couches de matériaux.
      % Nous supposons qu'il existe un champ \(\vE\) solution du problème de Maxwell-Helmholtz \eqref{eq:unicite:maxwell_int} avec conditions limite de Dirichlet homogènes sur les bords du domaine.
      % La démonstration de l'existence de ces champs est l'objet de la partie suivante.

      On pose \(\kk_1 = \sqrt{k_1^2 - k_x^2 - k_y^2}\),  \(\kk_2 = \sqrt{k_2^2 - k_x^2 - k_y^2}\).

      On remarques que toutes les composantes du champs \(\vE\) ( idem pour \(\vH\) ) sont solutions de \(\ddr[2]{u}{z} + (k_i^2 k_x^2 + k_y^2) u = 0 \), dont les solutions sont données par la famille \((\sin(\kk_i z),\cos(\kk_i z))\) selon la couche.
      % En utilisant les relations de saut, on obtient formellement,

      % \begin{align*}
      %   \left\lbrace
      %   \begin{aligned}
      %     \hat{E}_x &= a_2\sin(\kk_2(z-z_1)) +  b\cos(\kk_2(z-z_1))
      %     \\
      %     \hat{E}_y &= c_2\sin(\kk_2(z-z_1)) +  d\cos(\kk_2(z-z_1))
      %   \end{aligned}
      %   \right. && z_1 < z < z_2,
      %   \\
      %   \left\lbrace
      %   \begin{aligned}
      %     \hat{E}_x &= a_1\sin(\kk_1(z-z_1)) +  b\cos(\kk_1(z-z_1))
      %     \\
      %     \hat{E}_y &= c_1\sin(\kk_1(z-z_1)) +  d\cos(\kk_2(z-z_1))
      %   \end{aligned}
      %   \right. && z_0 < z < z_1.
      % \end{align*}
      % Ces relations ne sont valables que pour des paramètres \(a_2,c_2,a_1,c_1,b,d\) (constants par rapport à \(z\) mais dépendant de \((k_x,k_y)\)) telles que \((\hat{E}_x,\hat{E}_y)\) sont dans \(\mathcal{S}'(\RR^3)\).
      % La composante en z de \(\vE\) se déduit des deux autres car sa divergence est nulle.

      % % \begin{REM}
      % %   écris le système total car tu as \(\ddr{z}{\hat{E_z}} + ik_x \hat{E_x} + ik_y\hat{E_y} = 0\) et il te manque une constante d'intégration
      % % \end{REM}
      % % \begin{REP}
      % %   Quelle constante ?
      % % \end{REP}

      % On déduit le champ \(\vH(\vx) = \frac{-\vrot{\vE}(\vx)}{i k(\vx)\eta(\vx)}\). 
      % On note \(\mLR\) la matrice \(
      % \begin{bmatrix}
      %     k_y^2 & k_xk_y
      %     \\
      %     k_xk_y & k_x^2
      % \end{bmatrix}
      % \).


      % Ses composantes tangentielles sont
      % \begin{align*}
      %   \hat{\vH}_t &= \frac{\sin(\kk_2(z-z_1))}{-i k_2\kk_2 \eta_2}(\kk_2^2 \mI - \mLR)\begin{bmatrix}d \\ b\end{bmatrix} - \frac{\cos(\kk_2(z-z_1))}{-i k_2\kk_2 \eta_2}(\kk_2^2 \mI - \mLR)\begin{bmatrix}c_2 \\ a_2\end{bmatrix} && \text{pour } z_1 < z < z_2,
      %   \\
      %   \hat{\vH}_t &= \frac{\sin(\kk_1(z-z_1))}{-i k_1\kk_1 \eta_1}(\kk_1^2 \mI - \mLR)\begin{bmatrix}d \\ b\end{bmatrix} - \frac{\cos(\kk_1(z-z_1))}{-i k_1\kk_1 \eta_1}(\kk_1^2 \mI - \mLR)\begin{bmatrix}c_1 \\ a_1\end{bmatrix} && \text{pour } z_0 < z < z_1.
      % \end{align*}

      % Pour que le problème soit bien posé, il faut et il suffit que les constantes complexes \(a_2,c_2,b,d,a_1,c_1\) soient déterminées de façon unique quand on exprime les conditions limites. Ce sont des conditions limites de Dirichlet homogène en \(z=z_0\) et \(z=z_2\) pour \(\vE\) et de sauts nuls en \(z=z_1\) pour \(\vH\) (la condition de saut nul pour \(\vE\) a déjà été utilisée dans l'expression des champs).
      % \begin{REM}
      %   Le fait que \(\hat{\vE}\) et \(\hat{\vH}\) ont un saut nul à l'interface vient du fait que pour chaque \((k_x,k_y)\), on suppose \(\vE,\vH\) dans \(\Sobup[1]{([z_0,z_2])}\).
      % \end{REM}
      % On a unicité si, dans le cas de conditions de Dirichlet homogènes, l'unique solution est \(\vE=0\). Ces conditions s'expriment
      % \begin{equation*}
      %   \label{eq:unicite:interieur:2couches:cldirichlet}
      %   \left\lbrace
      %   \begin{aligned}
      %     a_2\sin{(\kk_2(z_2-z_1))} + b\cos{(\kk_2(z_2-z_1))} &= 0,
      %     \\
      %     c_2\sin{(\kk_2(z_2-z_1))} + d\cos{(\kk_2(z_2-z_1))} &= 0,
      %     \\
      %     a_1\sin{(\kk_1(z_0-z_1))} + b\cos{(\kk_1(z_0-z_1))} &= 0,
      %     \\
      %     c_1\sin{(\kk_1(z_0-z_1))} + d\cos{(\kk_1(z_0-z_1))} &= 0.
      %   \end{aligned}
      %   \right.
      % \end{equation*}

      % C'est un système à 4 équations pour 6 inconnues, donc à deux degrés de liberté. Il existe donc \(A,B\) tels que
      % \begin{equation*}
      %   \left\lbrace
      %   \begin{aligned}
      %   a_2 &= A\cos{(\kk_2(z_2-z_1))}\sin{(\kk_1(z_0-z_1))},
      %   \\
      %   b &= -A\sin{(\kk_2(z_2-z_1))}\sin{(\kk_1(z_0-z_1))},
      %   \\
      %   a_1 &= A\sin{(\kk_2(z_2-z_1))}\cos{(\kk_1(z_0-z_1))},
      %   \\
      %   c_2 &= B\cos{(\kk_2(z_2-z_1))}\sin{(\kk_1(z_0-z_1))},
      %   \\
      %   d &= -B\sin{(\kk_2(z_2-z_1))}\sin{(\kk_1(z_0-z_1))},
      %   \\
      %   c_1 &= B\sin{(\kk_2(z_2-z_1))}\cos{(\kk_1(z_0-z_1))}.
      %   \end{aligned}
      %   \right.
      % \end{equation*}
      % sont solutions du système précédent. 

      % La dernière condition de saut nul en \(z=z_1\) pour \(\hat\vH_t\) s'exprime alors
      % \begin{align*}
      %    \cos(\kk_2(z_2-z_1)) \sin(\kk_1(z_0-z_1))\frac{i}{k_2\kk_2\eta_2}(\kk_2^2 \mI - \mLR)\begin{bmatrix}B\\A\end{bmatrix}
      %   \\= 
      %    \sin(\kk_2(z_2-z_1)) \cos(\kk_1(z_0-z_1))\frac{i}{k_1\kk_1\eta_1}(\kk_1^2 \mI - \mLR)\begin{bmatrix}B\\A\end{bmatrix}.
      % \end{align*}

      % On voit alors apparaître le système final dont le déterminant est
      % \begin{multline*}
      %   -\det{}\left(
      %   \frac{\cos(\kk_2(z_2-z_1)) \sin(\kk_1(z_0-z_1))}{k_2\kk_2\eta_2}(\kk_2^2 \mI - \mLR)
      %   \right.
      %   \\
      %   \left.-
      %    \frac{\sin(\kk_2(z_2-z_1)) \cos(\kk_1(z_0-z_1))}{k_1\kk_1\eta_1}(\kk_1^2 \mI - \mLR)
      %    \right)
      % \end{multline*}

      % \newcommand{\alp}{\frac{\cos(\kk_2(z_2-z_1)) \sin(\kk_1(z_0-z_1))}{k_2\kk_2\eta_2}}
      % \newcommand{\bet}{\frac{\sin(\kk_2(z_2-z_1)) \cos(\kk_1(z_0-z_1))}{k_1\kk_1\eta_1}}

      % \begin{multline*}
      %   =-\left(\alp \kk_2^2 - \bet \kk_1^2\right)
      %   \\
      %   \left(\alp(\kk_2^2-k_y^2)-\bet(\kk_1^2-k_x^2)\right).
      % \end{multline*}


      % Si le déterminant de ce système n'est pas nul, alors il existe des champs non nuls solutions du problème homogène, et il n'y a alors pas unicité des solutions.

      % \begin{REM}
      %   Tu n'as pas repris mes calculs pour simplifier cette expression ? J'avais un résultat intéressant et cela serait dommage qu'elle ne soit pas dans ta thèse.
      % \end{REM}
      % \begin{REP}
      %   Peux-tu me rappeler quand tu m'as envoyé cette note ?
      % \end{REP}

    On peut donc exprimer le champ électrique tangentiel grâce à des paramètres \(a_1,c_1,a_2,c_2,b,d\) dépendant de \((k_x,k_y)\) mais pas de \(z\)
    \begin{align*}
    \hat{E}_x={}& 
    \begin{cases}
    a_2\sin(\kk_2(z-z_1))+b\cos(\kk_2(z-z_1)),&  z_1<z<z_2
    \\
    a_1\sin(\kk_1(z-z_1))+b\cos(\kk_1(z-z_1)),&  z_0<z<z_1
    \end{cases}
    \\
    \hat{E}_y={}&
    \begin{cases}
    c_2\sin(\kk_2(z-z_1))+d\cos(\kk_2(z-z_1)),&   z_1<z<z_2
    \\
    c_1\sin(\kk_1(z-z_1))+d\cos(\kk_1(z-z_1)),&   z_0<z<z_1
    \end{cases}
    \end{align*}
    Ces expressions prenant compte de condition de continuité tangentielle du champ électrique à l'interface \(z=z_1\).

    En supposant \(k_x^2 + k_y^2\) non nul, nous définissons d'autres paramètres \(A_2,C_2, A_1, C_1, B, D\) tels que
    % \[
    % \left\{
    % \begin{array}{ll}
    % (a_2,c_2)&=A_2(k_x,k_y)+C_2(-k_y,k_x)
    % \cr
    % (a_1,c_1)&=A_1(k_x,k_y)+C_1(-k_y,k_x)
    % \cr
    % (b,d)&=B(k_x,k_y)+D(-k_y,k_x)
    % \end{array}
    % \right.
    % \]
    % équivalent à
    \begin{align*}
    a_2 ={}& A_2k_x - C_2k_y, & c_2 ={}& A_2k_y + C_2 k_x,
    \\
    a_1 ={}& A_1k_x - C_1k_y, & c_1 ={}& A_1k_y + C_1 k_x,
    \\
    b ={}& Bk_x - Dk_y, & d ={}& Bk_y + D k_x,
    \end{align*}
    et réciproquement
    \begin{align*}
    A_2 ={}& \frac{a_2 k_x + c_2 k_y}{k_x^2 + k_y^2}, & C_2 ={}& \frac{c_2 k_y - a_2 k_x}{k_x^2 + k_y^2},
    \\
    A_1 ={}& \frac{a_1 k_x + c_1 k_y}{k_x^2 + k_y^2}, & C_1 ={}& \frac{c_1 k_y - a_1 k_x}{k_x^2 + k_y^2},
    \\
    B ={}& \frac{b k_x + d k_y}{k_x^2 + k_y^2}, & D ={}& \frac{d k_y - b k_x}{k_x^2 + k_y^2}.
    \end{align*}

    Les divergences nulle des champs nous permettent de déduire de \(a_2,c_2,a_1,c_1,b,d\) la décomposition sur \((\sin(\kk_i(z-z_i)),\cos(\kk_i(z-z_i)))\) de \(E_z\):

    \[
    E_z=i\frac{k_x^2+k_y^2}{\kk_2}A_2\cos(\kk_2(z-z_1))-i\frac{k_x^2+k_y^2}{\kk_2}B\sin(\kk_2(z-z_1)), z_1<z<z_2,
    \]
    Dans \(z_0<z<z_1\), il faut remplacer \(A_2\) par \(A_1\), \(C_2\) par \(C_1\) et \(\kk_2\) par \(\kk_1\).

    Nous cherchons des résonances, c'est à dire des solutions non nulles du problème de Maxwell harmonique avec conditions de Dirichlet homogènes sur les bords \(z=z_0\) et \(z=z_2\).

    Notons \(\alpha_1=\kk_1(z_0-z_1)\), \(\alpha_2=\kk_2(z_2-z_1)\).

    Les conditions sur \(E_x\) et sur \(E_y\) donnent, après décomposition sur \((k_x,k_y)\) et sur \((-k_y,k_x)\), en \(z=z_0\) et en \(z=z_2\)
    \[
    \left\{
    \begin{aligned}
    B\cos \alpha_1={}& A_1\sin \alpha_1,
    \\
    D\cos \alpha_1={}& C_1\sin \alpha_1,
    \\
    B\cos \alpha_2={}& A_2\sin \alpha_2,
    \\
    D\cos \alpha_2={}& C_2\sin \alpha_2.
    \end{aligned}
    \right.
    \]

    D'autre part, on écrit la condition de saut tangentiel du champ magnétique à l'interface \(z=z_1\).

    Les coordonnées tangentielles de \(\vect F = \vrot \hat\vE\) dans \(z_1<z<z_2\) sont
    \begin{multline*}
    F_x = -k_y\left(\frac{k_x^2+k_y^2}{\kk_2}A_2\cos(\kk_2(z-z_1))-\frac{k_x^2+k_y^2}{\kk_2}B\sin(\kk_2(z-z_1))\right)
    \\
    -\kk_2\left((k_yA_2+k_xC_2)\cos(\kk_2(z-z_1))-(Bk_y+Dk_x)\sin(\kk_2(z-z_1))\right)
    \end{multline*}
    \begin{multline*}
    F_y = \kk_1\left((k_xA_2-k_yC_2)\cos(\kk_2(z-z_1))-(k_xB-k_yD)\sin(\kk_2(z-z_1))\right)
    \\
    +k_x\left(\frac{k_x^2+k_y^2}{\kk_2}A_2\cos(\kk_2(z-z_1))-\frac{k_x^2+k_y^2}{\kk_2}B\sin(\kk_2(z-z_1))\right)
    \end{multline*}

    D'où l'on déduit les coordonnées tangentielles du champ magnétique dans \(z_1<z<z_2\):
    \begin{multline*}
    \hat{\mathcal{H}}_x = -\frac{k_y}{-ik_2\eta_2}\left(\frac{k_x^2+k_y^2}{\kk_2}A_2\cos(\kk_2(z-z_1))-\frac{k_x^2+k_y^2}{\kk_2}B\sin(\kk_2(z-z_1))\right)
    \\
    -\frac{\kk_2}{-ik_2\eta_2}\left((k_yA_2+k_xC_2)\cos(\kk_2(z-z_1))-(Bk_y+Dk_x)\sin(\kk_2(z-z_1))\right)
    \end{multline*}
    \begin{multline*}
    \hat{\mathcal{H_y}} = \frac{\kk_2}{-ik_2\eta_2}\left((k_xA_2-k_yC_2)\cos(\kk_2(z-z_1))-(k_xB-k_yD)\sin(\kk_2(z-z_1))\right)
    \\
    +\frac{k_x}{-ik_2\eta_2}\left(\frac{k_x^2+k_y^2}{\kk_2}A_2\cos(\kk_2(z-z_1))-\frac{k_x^2+k_y^2}{\kk_2}B\sin(\kk_2(z-z_1))\right)
    \end{multline*}
    Dans \(z_0<z<z_1\), il faut remplacer \(A_2\) par \(A_1\), \(C_2\) par \(C_1\), \(k_2\eta_2\) par \(k_1\eta_1\) et \(\kk_2\) par \(\kk_1\).


    Pour satisfaire à la continuité tangentielle du champ magnétique, il faut que
    \[
    \left\{
    \begin{aligned}
    -\frac{k_y}{k_1\eta_1}\frac{k_x^2+k_y^2}{\kk_1}A_1-\frac{\kk_1}{k_1\eta_1}(k_yA_1+k_xC_1)={}&-\frac{k_y}{k_2\eta_2}\frac{k_x^2+k_y^2}{\kk_2}A_2-\frac{\kk_2}{k_2\eta_2}(k_yA_2+k_xC_2),
    \\
    -\frac{\kk_1}{k_1\eta_1}(k_xA_1-k_yC_1)+\frac{k_x}{k_1\eta_1}\frac{k_x^2+k_y^2}{\kk_1}A_1={}&-\frac{\kk_2}{k_2\eta_2}(k_xA_2-k_yC_2)+\frac{k_x}{k_2\eta_2}\frac{k_x^2+k_y^2}{\kk_2^2}A_2.
    \end{aligned}
    \right.
    \]

    On a ainsi un système de 6 équations à 6 inconnues \(A_1,C_1,A_2,C_2,B,D\).

    On introduit les deux matrices
    \begin{align*}
    \gls{mat-mat-abstract}_1={}&
    \begin{bmatrix}
    -\kk_1k_y-k_y\frac{k_x^2+k_y^2}{\kk_1}  &   -\kk_1k_x
    \cr
    \kk_1k_x+k_x\frac{k_x^2+k_y^2}{\kk_1}   &   -\kk_1k_y
    \end{bmatrix},
    &
    \mM_2={}&
    \begin{bmatrix}
    -\kk_2k_y-k_y\frac{k_x^2+k_y^2}{\kk_2}  &   -\kk_2k_x
    \cr
    \kk_2k_x+k_x\frac{k_x^2+k_y^2}{\kk_2}   &   -\kk_2k_y
    \end{bmatrix}.
    \end{align*}

    La continuité tangentielle du champ magnétique se résume à 
    \[
    \frac{1}{k_1\eta_1}\mM_1\begin{bmatrix}A_1\\C_1\end{bmatrix}=\frac{1}{k_2\eta_2}\mM_2\begin{bmatrix}A_2\\C_2\end{bmatrix}.
    \]

    Nous aboutissons en multipliant par \(\sin(\alpha_1)\sin(\alpha_2)\) puis en utilisant les relations en fonction de \(B\) et \(D\) :

    \[
    k_2\eta_2\sin(\alpha_2)\mM_1\begin{bmatrix}A_1\sin(\alpha_1)\\C_1\sin(\alpha_1)\end{bmatrix}
    =
    k_1\eta_1\sin(\alpha_1)\mM_2\begin{bmatrix}A_2\sin(\alpha_2)\\C_2\sin(\alpha_2)\end{bmatrix},
    \]

    \[
    k_2\eta_2\sin(\alpha_2)\cos(\alpha_1)\mM_1\begin{bmatrix}B\\D\end{bmatrix}
    =
    k_1\eta_1\sin(\alpha_1)\cos(\alpha_2)\mM_2\begin{bmatrix}B\\D\end{bmatrix},
    \]
    et le problème de résonance se ramène à 

    \[
    \operatorname{det}(K_1\mM_1-K_2\mM_2)=0
    \]
    avec \(K_1=k_2\eta_2\cos \alpha_1\sin \alpha_2\), \(K_2=k_1\eta_1\cos \alpha_2\sin \alpha_1\).

    Nous remarquons alors que
    \[
    \mM_1=
    \begin{bmatrix}
    -\kk_1k_y-k_y\frac{k_x^2+k_y^2}{\kk_1}  &   -\kk_1k_x
    \\
    \kk_1k_x+k_x\frac{k_x^2+k_y^2}{\kk_1}   &   -\kk_1k_y
    \end{bmatrix}
    =
    \begin{bmatrix}
    -k_y\frac{k_1^2}{\kk_1}   &   -\kk_1k_x
    \\
    k_x\frac{k_1^2}{\kk_1}    &   -\kk_1k_y
    \end{bmatrix},
    \]
    et en modifiant les indices, on obtient un résultat équivalent pour \(\mM_2\).

    En calculant directement le déterminant de \(K_1\mM_1-K_2\mM_2\), on trouve
    \[
    \left(k_x^2+k_y^2\right)
    \left(K_1\frac{k_1^2}{\kk_1} - K_2\frac{k_2^2}{\kk_2}\right)
    \left(K_1\kk_1+K_2\kk_2\right) = 0
    \]
    On obtient donc 3 conditions de résonances.
    La première a déjà été supposée, les deux suivantes s'expriment respectivement en fonction de \((k_x,k_y)\)
    \begin{align*}
    \kk_2k_1\eta_2\tan\left(\kk_2(z_2-z_1)\right) &= \kk_1k_2\eta_1\tan\left(\kk_1(z_0-z_1)\right),
    \\
    \kk_1k_2\eta_2\tan\left(\kk_2(z_2-z_1)\right) &= \kk_2k_1\eta_1\tan\left(\kk_1(z_0-z_1)\right).
    \end{align*}
    Les constantes de matériaux étant complexes, il n'est pas possible d'en déduire des conditions sur \((k_x,k_y)\).
    \end{proof}
