\section{Unicité du problème intérieur}

Pour exprimer l'opérateur d'impédance, nous devons exprimer les champs dans l'objet. Il apparaît nécessaire que ces derniers soient donc bien définis.

Nous nous dotons d'un matériaux définis par couches, et allons donc exhiber une condition nécessaire et suffisante d'unicité des solutions du problème de Maxwell-Helmholtz  \eqref{eq:unicite:maxwell_int} dans \(\OO\) est un objet multi-couche creux c'est-a-dire tel que \(\partial \OO = \Gamma \cup \Gamma_0\) où \(\Gamma\) est la surface extérieure et \(\Gamma_0\) la surface intérieure.

\begin{align}
\left\lbrace
  \begin{matrix}
    \vrot \vE(\vx) + i k(\vx)\eta(\vx) \vH(\vx) &= 0
    \\
    \vrot \vH(\vx) - i k(\vx)(\eta(\vx))^{-1} \vE(\vx) &= 0
  \end{matrix}
  \right. && \text{dans \(\OO\).}
  \label{eq:unicite:maxwell_int}
\end{align}

\begin{prop}
  Le problème Maxwell-Helmholtz \eqref{eq:unicite:maxwell_int} avec conditions aux limites de Dirichlet \(\vE_{|\Gamma} = 0\), \(\vE_{|\Gamma_0} = 0\) admet une unique solution si est seulement si le déterminant d'un système linéaire est non-nul.
\end{prop}
\begin{proof}
\end{proof}