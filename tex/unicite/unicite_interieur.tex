\section{Une condition nécessaire et suffisante pour l'unicité des solutions du problème intérieur}
  \subsection{Cas général: alternative de Fredholm}

    Pour exprimer l'opérateur d'impédance, nous devons exprimer les champs dans l'objet. Il apparaît nécessaire que ces derniers soient donc bien définis.

    Nous exhibons une condition nécessaire et suffisante d'unicité des solutions du problème de Maxwell-Helmholtz  \eqref{eq:unicite:maxwell_int} dans \(\OO\), objet multicouche autour d'un \gls{acr-cep}, c'est-a-dire tel que \(\partial \OO = \Gamma \cup \Gamma_0\) où \(\Gamma\) est la surface extérieure et \(\Gamma_0\) la surface intérieure. Le problème de Maxwell harmonique en \(e^{i\w t}\) s'exprime

    \begin{align}
    \left\lbrace
      \begin{matrix}
        \vrot \vE(\vx) + i \w\mu(\vx) \vH(\vx) &= 0
        \\
        \vrot \vH(\vx) - i \w\eps(\vx) \vE(\vx) &= 0
      \end{matrix}
      \right. && \text{dans \(\OO\).}
      \label{eq:unicite:maxwell_int}
    \end{align}

    Nous rappelons le résultat général de \cite[Théorème~8, p.~111]{cessenat_mathematical_1996}.
    \begin{thm}[Alternative de Fredholm]
      Les champs \(\vE,\vH\) solutions de \eqref{eq:unicite:maxwell_int} avec conditions aux limites de Dirichlet \(\vE_{|\Gamma} = 0\), \(\vE_{|\Gamma_0} = 0\) sont soit uniques si \(\w^2\eps\mu\) n'est pas une valeur propre du Laplacien vectoriel, soit ces champs sont : engendrés par les vecteurs propres.
    \end{thm}
    Nous renvoyons à l'ouvrage cité précédemment pour la démonstration, en dehors du cadre de cette thèse.
    D'après ce résultat, nous pouvons avoir des résonances pour le problème intérieur, c'est-à-dire des solutions non nulles du problème sans sources donc non-uniques.

  \subsection{Cas particuliers de géométries particulières}
    Cette thèse s'attache à déterminer les coefficients des \glspl{acr-cioe} des sections précédentes.
    Nous devons être capables d'exprimer l'opérateur d'impédance exact pour l'approcher par les CIOE.
    Il faut donc que les champs soient déterminés de façon unique pour exprimer cet opérateur. Nous ne pouvons nous satisfaire de l'alternative de Fredholm et devons exprimer une condition plus pratique pour déterminer s'il y a unicité.
    Nous allons alors approcher l'objet par une forme simple mais avec les mêmes caractéristiques pour l'empilement. Ainsi nous étudierons trois approximations :
    \begin{enumerate}
      \item Un objet plan infini 
      \item Un objet cylindrique
      \item Un objet sphérique
    \end{enumerate}
   
    L'alternative de Fredholm n'étant pas utilisable en pratique, nous allons simplifier la géométrie du problème pour aboutir à une condition nécessaire d'unicité.

    L'objet est représenté schématiquement:
    \begin{figure}[h!btp]
        \centering
        \tikzsetnextfilename{plan_n_couches}            
        \begin{tikzpicture}
            \tikzmath{
    \largeur = 6;
    \hauteur = 0.5;
    \milieu = 1.3;
    \xC = \largeur;
    \xA = 0;
}

%% 1ere couche
\tikzmath{
    \yC = \hauteur;
    \yA = 0;
}

\coordinate (A) at (\xA,\yA);
\coordinate (B) at (\xA,\yC);
\coordinate (C) at (\xC,\yC);

\draw ($(B)!0.5!(C)$) node [above] {vide};


\fill [lightgray] (A) rectangle (C);
\draw ($(A)!0.5!(C)$) node {$\eps_n,\mu_n,d_n$};
\draw (B) -- (C) node [right] {$e_3 = 0$};

%% Des couches
\tikzmath{
    \yC = \yC - \hauteur;
    \yA = \yA - \milieu*\hauteur;
}

\coordinate (A) at (\xA,\yA);
\coordinate (B) at (\xA,\yC);
\coordinate (C) at (\xC,\yC);

\fill [lightgray]    (A) rectangle (C);
\fill [pattern=dots] (A) rectangle (C);
\draw (B) -- (C);

%% N ieme couche
\tikzmath{
    \yC = \yC - \milieu*\hauteur;
    \yA = \yA - \hauteur;
}

\coordinate (A) at (\xA,\yA);
\coordinate (B) at (\xA,\yC);
\coordinate (C) at (\xC,\yC);
\fill [lightgray] (A) rectangle (C);
\draw ($(A)!0.5!(C)$) node {$\eps_1,\mu_1,d_1$};
\draw (B) -- (C);

%% Le repère
\tikzmath{
    \xD = \xC + 0.5;
}

\coordinate (n) at (\xD,\yA);
\draw [->] (n) -- ++(1,0) node [at end, right] {$\v{e_1}$};
\draw [->] (n) -- ++(0,1) node [at end, right] {$\v{e_3}$};

\draw (n) circle(0.1cm) node [below=0.1cm] {$\v{e_2}$};
\draw (n) +(135:0.1cm) -- +(315:0.1cm);
\draw (n) +(45:0.1cm) -- +(225:0.1cm);

%% Le conducteur
\tikzmath{
    \yC = \yC - \hauteur;
    \yA = \yA - 0.5*\hauteur;
}

\coordinate (A) at (\xA,\yA);
\coordinate (B) at (\xA,\yC);
\coordinate (C) at (\xC,\yC);
\draw (B) -- (C);

\fill [pattern=north east lines] (A) rectangle (C);



        \end{tikzpicture}
    \end{figure}

    Les constantes relatives \(\eps,\mu\) (ou \(k,\eta\)) sont constantes par morceaux en fonction de \(z\).
    \begin{prop}
      \label{prop:unicite:interieur:postulat:multi-couche}
      Le problème Maxwell-Helmholtz \eqref{eq:unicite:maxwell_int} avec conditions aux limites de Dirichlet \(\vE_{|z=z_n} = 0\), \(\vE_{|z=z_0} = 0\) admet une unique solution si et seulement si le déterminant d'un système linéaire est non-nul.
    \end{prop}

    \begin{proof}
      Nous ne traiterons dans cette thèse que l'exemple de deux couches de matériaux.
      Nous supposons qu'il existe un champ \(\vE\) solution du problème de Maxwell-Helmholtz \eqref{eq:unicite:maxwell_int} avec conditions limite de Dirichlet sur les bords du domaine. La démonstration de l'existence de ces champs est l'objet de la partie suivante.

      \newcommand{\kk}{\tilde{k}}

      On pose \(\kk_1 = \sqrt{k_1^2 - k_x^2 - k_y^2}\),  \(\kk_2 = \sqrt{k_2^2 - k_x^2 - k_y^2}\). On a formellement,
      \begin{align*}
        \left\lbrace
        \begin{aligned}
          \hat{E}_x &= \sin(\kk_2(z-z_1))a_2 +  \cos(\kk_2(z-z_1))b
          \\
          \hat{E}_y &= \sin(\kk_2(z-z_1))c_2 +  \cos(\kk_2(z-z_1))d
        \end{aligned}
        \right. && z_1 < z < z_2,
        \\
        \left\lbrace
        \begin{aligned}
          \hat{E}_x &= \sin(\kk_1(z-z_1))a_1 +  \cos(\kk_1(z-z_1))b
          \\
          \hat{E}_y &= \sin(\kk_1(z-z_1))c_1 +  \cos(\kk_2(z-z_1))d
        \end{aligned}
        \right. && z_0 < z < z_1.
      \end{align*}
      Ces relations ne sont valables que pour des constantes \(a_2,c_2,a_1,c_1,b,d\) telles que \((\hat{E}_x,\hat{E}_y)\) sont dans \(\mathcal{S}'(\RR^3)\).
      La composante en z de \(\vE\) se déduit des deux autres car sa divergence est nulle.

      On déduit le champ \(\vH\) alors \(\vH(\vx) = \frac{-\vrot{\vE}(\vx)}{i k(\vx)\eta(\vx)}\). On note \(\mLR\) la matrice \(
      \begin{bmatrix}
          k_y^2 & k_xk_y
          \\
          k_xk_y & k_x^2
      \end{bmatrix}
      \).


      Le champ s'exprime matriciellement
      \begin{align*}
        \hat{\vH} &= \frac{\sin(\kk_2(z-z_1))}{-i k_2\kk_2 \eta_2}(\kk_2^2 \mI - \mLR)\begin{bmatrix}d \\ b\end{bmatrix} - \frac{\cos(\kk_2(z-z_1))}{-i k_2\kk_2 \eta_2}(\kk_2^2 \mI - \mLR)\begin{bmatrix}c_2 \\ a_2\end{bmatrix} && \text{pour } z_1 < z < z_2,
        \\
        \hat{\vH} &= \frac{\sin(\kk_1(z-z_1))}{-i k_1\kk_1 \eta_1}(\kk_1^2 \mI - \mLR)\begin{bmatrix}d \\ b\end{bmatrix} - \frac{\cos(\kk_1(z-z_1))}{-i k_1\kk_1 \eta_1}(\kk_1^2 \mI - \mLR)\begin{bmatrix}c_1 \\ a_1\end{bmatrix} && \text{pour } z_0 < z < z_1.
      \end{align*}

      Pour que le problème soit bien posé, il faut et il suffit que les constantes complexes \(a_2,c_2,b,d,a_1,c_1\) soient déterminées de façon unique quand on exprime les conditions limites. Ce sont des conditions limites de Dirichlet en \(z=z_0\) et \(z=z_2\) pour \(\vE\) et de sauts nuls en \(z=z_1\) pour \(\vH\) (la condition de saut nul pour \(\vE\) a déjà été utilisé dans l'expression des champs).

      On a unicité si, dans le cas de conditions de Dirichlet homogènes, l'unique solution est \(\vE=0\). Ces conditions s'expriment
      \begin{equation*}
        \label{eq:unicite:interieur:2couches:cldirichlet}
        \left\lbrace
        \begin{aligned}
          a_2\sin{(\kk_2(z_2-z_1))} + b\cos{(\kk_2(z_2-z_1))} &= 0,
          \\
          c_2\sin{(\kk_2(z_2-z_1))} + d\cos{(\kk_2(z_2-z_1))} &= 0,
          \\
          a_1\sin{(\kk_1(z_0-z_1))} + b\cos{(\kk_1(z_0-z_1))} &= 0,
          \\
          c_1\sin{(\kk_1(z_0-z_1))} + d\cos{(\kk_1(z_0-z_1))} &= 0.
        \end{aligned}
        \right.
      \end{equation*}

      C'est un système à 4 équations pour 6 inconnues, donc à deux degrés de liberté. Il existe donc \(A,B\) tels que
      \begin{equation*}
        \left\lbrace
        \begin{aligned}
        a_2 &= A\cos{(\kk_2(z_2-z_1))}\sin{(\kk_1(z_0-z_1))},
        \\
        b &= -A\sin{(\kk_2(z_2-z_1))}\sin{(\kk_1(z_0-z_1))},
        \\
        a_1 &= A\sin{(\kk_2(z_2-z_1))}\cos{(\kk_1(z_0-z_1))},
        \\
        c_2 &= B\cos{(\kk_2(z_2-z_1))}\sin{(\kk_1(z_0-z_1))},
        \\
        d &= -B\sin{(\kk_2(z_2-z_1))}\sin{(\kk_1(z_0-z_1))},
        \\
        c_1 &= B\sin{(\kk_2(z_2-z_1))}\cos{(\kk_1(z_0-z_1))}.

        \end{aligned}
        \right..

      \end{equation*}
      sont solutions du système précédent. 

      La dernière condition de saut nul en \(z=z_1\) pour \(\hat\vH_t\) s'exprime alors
      \begin{align*}
         \cos(\kk_2(z_2-z_1)) \sin(\kk_1(z_0-z_1))\frac{i}{k_2\kk_2\eta_2}(\kk_2^2 \mI - \mLR)\begin{bmatrix}B\\A\end{bmatrix}
        \\= 
         \sin(\kk_2(z_2-z_1)) \cos(\kk_1(z_0-z_1))\frac{i}{k_1\kk_1\eta_1}(\kk_1^2 \mI - \mLR)\begin{bmatrix}B\\A\end{bmatrix}.
      \end{align*}

      On voit alors apparaître le système final dont le déterminant est
      \begin{multline*}
        -\det{}\left(
        \frac{\cos(\kk_2(z_2-z_1)) \sin(\kk_1(z_0-z_1))}{k_2\kk_2\eta_2}(\kk_2^2 \mI - \mLR)
        \right.
        \\
        \left.-
         \frac{\sin(\kk_2(z_2-z_1)) \cos(\kk_1(z_0-z_1))}{k_1\kk_1\eta_1}(\kk_1^2 \mI - \mLR)
         \right)
      \end{multline*}

      \newcommand{\alp}{\frac{\cos(\kk_2(z_2-z_1)) \sin(\kk_1(z_0-z_1))}{k_2\kk_2\eta_2}}
      \newcommand{\bet}{\frac{\sin(\kk_2(z_2-z_1)) \cos(\kk_1(z_0-z_1))}{k_1\kk_1\eta_1}}

      \begin{multline*}
        =-\left(\alp \kk_2^2 - \bet \kk_1^2\right)
        \\
        \left(\alp(\kk_2^2-k_y^2)-\bet(\kk_1^2-k_x^2)\right).
      \end{multline*}


      Si le déterminant de ce système n'est pas nul, alors il existe des champs non nuls solutions du problème homogène, et il n'y a alors pas unicité des solutions.

    \end{proof}
