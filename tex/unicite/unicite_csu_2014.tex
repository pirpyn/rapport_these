\section[CSU pour les CIOE de Aubakirov 2014]{CSU pour les CIOE de \cite{aubakirov_electromagnetic_2014}}

  \begin{defn}
    Les opérateurs \gls{ope-LD} et \gls{ope-LR} sont introduits par \cite[eq.~4]{stupfel_implementation_2015} et s'expriment pour tous vecteurs complexes régulier tangents à \(\Gamma\) 

    Pour tous \(\vect U,\vect V \in (\mathcal{C}^\infty(\Gamma,\CC))^3\): 
    \begin{align*}
      \LD(\vect U) &= \tgrads (\tdivs \vect U)\\
      \LR(\vect V) &= \trots( \trots \vect V)
    \end{align*}
  \end{defn}

  \begin{prop}
    \(\LD\) est hermitien symétrique négatif et \(\LR\) est hermitien symétrique positif.
  \end{prop}

  \begin{prop}
    Soit \(\OO\) un domaine borné de \(\RR^3\) , de surface \(\Gamma\) fermée et régulière où \(\vect n\) y est la normale unitaire
    sortante
    \begin{equation}
      \begin{matrix}
        \forall \vect U \in (\mathcal{C}^\infty(\Gamma,\CC))^3 ,& \LR(\LD(\vect U)) = \LD(\LR(\vect U)) = 0
      \end{matrix}
    \end{equation}
  \end{prop}

  \begin{proof}
    Soit (\(x_1,x_2\)) système de coordonnées locales à \(\Gamma\)

    Soit un vecteur tangent défini en tout point de la surface exprimé sur la base locale \(\vect{u_1},\vect{u_2},\vect{n}\)
    \[
      \vect{U} = 
      \begin{bmatrix}
        U_1(x_1,x_2)
        \\
        U_2(x_1,x_2)
        \\
        0
      \end{bmatrix}
    \]

    D'après \cite[p.~1028, A3.22]{bladel_electromagnetic_2007}, le rotationnel surfacique d'un vecteur tangent est une vecteur normal à la surface
    \[
      \trots{\vect{U}} = \vn \left(\vn\cdot\trots{\vect{U}}\right)
    \]

    Montrons que \(\LR\LD = 0\).

    D’après \cite[p.~1029, A3.42]{bladel_electromagnetic_2007}, soit \(f(x_1,x_2)\) une fonction régulière définie sur \(\Gamma\) alors \(\vn \cdot \trots(\tgrads f(x_1,x_2)) = 0\).

    On exprime les opérateurs
    \begin{align*}
      \LR(\LD \vect U)  &= \trots \left(\vn \left(\vn \cdot \trots \left( \tgrads \left(\tdivs \vect U\right)\right)\right)\right) \\
      &= 0
    \end{align*}

    Montrons que \(\LD\LR = 0\).

    D’après \cite[p.~1029, A3.43]{bladel_electromagnetic_2007}, \(\tdivs \trots (f\vn) = 0\).
    \begin{align*}
      \LD(\LR \vect U) &= \tgrads \tdivs \trots (\vn (\vn \cdot \trots \vect U)) \\
      &= 0
    \end{align*}
  \end{proof}

\subsection{CSU pour la CI4}
  Soit la CIOE que l'on nomme \hyperlink{ci4}{CI4} :
  \begin{equation}
    \label{eq:unicite:ci4:ci4}
    \vE_t = (a_0 + a_1 \LD - a_2 \LR ) \vJ
  \end{equation}

  \begin{prop}
    Des CSU sont
    \begin{align}
      \Re(a_0) \ge 0
      \\
      \Re(a_1) \le 0
      \\
      \Re(a_2) \le 0
    \end{align}
  \end{prop}

  \begin{prop}
    On pose
    \begin{align*}
      F_D &:= \int_\Gamma \vJ\cdot \LD\conj{\vJ}  \le 0 & F_R &:= \int_\Gamma \vJ\cdot \LR\conj{\vJ}  \ge 0
    \end{align*}

    On a alors
    \begin{equation*}
      X = \conj{a_0}\norm{\vJ}^2 + \conj{a_1}F_D - \conj{a_2}F_R
    \end{equation*}

    On impose la positivité de la partie réelle de chaque terme impliquant \(\Re(X)\ge0\)
    \begin{align*}
      \Re\left(a_0\right) \ge 0
      \\
      \Re\left(a_1\right) \le 0
      \\
      \Re\left(a_2\right) \le 0
    \end{align*}
  \end{prop}

  On remarque que si \(a_1=a_2\), ces CSU sont identiques aux CSU de la CI01.

\subsection{CSU pour la CI3}

  Soit la CIOE énoncée dans \cite{aubakirov_electromagnetic_2014} que l'on nomme \hyperlink{ci3}{CI3} :
  \begin{equation}
    \label{eq:unicite:ci3:ci3}
    ( 1 + b_1 \LD - b_2 \LR)\vE_t = (a_0 + a_1 \LD - a_2 \LR ) \vJ
  \end{equation}

\subsubsection{1\iere méthode}

  \begin{prop}
    Soit \(\Delta_i = a_i-\conj{b_i}a_0\), \(i=1,2\). Des CSU sont
    \begin{align}
      \Re\left(a_0\conj{a_1}\Delta_1\right) \ge 0 \\
      \Re\left(\frac{\conj{b_1}}{\Delta_1}\right) \le 0 \\
      \Re\left(\conj{a_0}a_2\left(\frac{\conj{b_2}}{\Delta_2}-\frac{\conj{b_2}}{\Delta_2}\right) + \frac{\conj{a_2}a_1}{\Delta_1} \right)\le 0\\
      \Re\left(2\Re(b_2)\frac{\conj{b_1}}{\Delta_1}-\frac{\conj{b_2}^2}{\Delta_2}\right) \ge 0\\
      \Re\left(a_0\conj{a_2}\Delta_2\right) \ge 0 \\
      \Re\left(\frac{\conj{b_2}}{\Delta_2}\right) \le 0 \\
      \Re\left(\conj{a_0}a_1\left(\frac{\conj{b_1}}{\Delta_1}-\frac{\conj{b_2}}{\Delta_2}\right) + \frac{\conj{a_1}a_2}{\Delta_2} \right)\le 0\\
      \Re\left(2\Re(b_1)\frac{\conj{b_2}}{\Delta_2}-\frac{\conj{b_1}^2}{\Delta_1}\right) \ge 0\\
      \Re\left(\Delta_1\right) = 0 \\
      \Re\left(\Delta_2\right) = 0 \\
      \Re\left(\frac{\conj{b_2}}{\Delta_2}-\frac{\conj{b_1}}{\Delta_1}\right) = 0
    \end{align}
  \end{prop}
  
  \begin{proof}
    On prend l'expression de la CIOE \eqref{eq:unicite:ci3:ci3} et on intègre chaque membre avec des produits scalaires judicieusement choisis.

    \begin{multline}
      \label{eq:unicite:ci3:csu3-1}
      \int_\Gamma \vJ \cdot \conj{\vE_t}   + \conj{b_1} \int_\Gamma \vJ\cdot \LD\conj{\vE_t}  - \conj{b_2} \int_\Gamma \vJ \LR\conj{\vE_t}  \\
      = \conj{a_0} \norm{\vJ}^2ds - \conj{a_1} \norm{\tdivs \vJ}^2  - \conj{a_2} \norm{\trots \vJ}^2 
    \end{multline}
    \begin{multline}
      \label{eq:unicite:ci3:csu3-2}
      \norm{\vE_t}^2   - b_1 \norm{ \tdivs \vE }^2  - b_2 \norm{\trots \vE_t}^2  \\
      = a_0 \int_\Gamma \vJ\cdot \conj{\vE_t} + a_1 \int_\Gamma \conj{\vE_t} \LD \vJ  - a_2 \int_\Gamma \conj{\vE_t} \cdot \LR \vJ 
    \end{multline}
    \begin{equation}
      \label{eq:unicite:ci3:csu3-3}
      \int_\Gamma \vJ \cdot \LR \conj{\vE_t}   - \conj{b_2} \int_\Gamma \LR \vJ \cdot \LR \conj{\vE_t}
      =  \conj{a_0} \norm{\vn \cdot \trots \vJ}^2ds - \conj{a_2} \norm{ \LR \vJ}^2 
    \end{equation}
    \begin{equation}
      \label{eq:unicite:ci3:csu3-4}
      \norm{ \vn \cdot \trots \vE_t }^2   - \conj{b_2} \norm{ \LR \vE_t}^2 
      = a_0 \int_\Gamma \conj{\vE_t} \LR \vJ  - a_2 \int_\Gamma \LR \conj{\vE_t} \cdot \LR \vJ 
    \end{equation}
      \begin{equation}
      \label{eq:unicite:ci3:csu3-5}
      \int_\Gamma \vJ \cdot \LD \conj{\vE_t}   + \conj{b_1} \int_\Gamma \LD \vJ \cdot \LD \conj{\vE_t}
      = - \conj{a_0} \norm{\tdivs \vJ}^2ds + \conj{a_1} \norm{ \LD \vJ}^2 
    \end{equation}
    \begin{equation}
      \label{eq:unicite:ci3:csu3-6}
      -\norm{ \tdivs \vE_t }^2   + \conj{b_1} \norm{ \LD \vE_t}^2
      = a_0 \int_\Gamma \conj{\vE_t} \LD \vJ  + a_1 \int_\Gamma \LD \conj{\vE_t} \cdot \LD \vJ 
    \end{equation}

    On pose alors les définitions suivantes :
    \begin{align*}
      X &= \int_\Gamma \vJ \cdot \conj{\vE_t} 
      \\
      Y_D &= \int_\Gamma \vJ \cdot \LD \conj{\vE_t}  &
      Y_R &= \int_\Gamma \vJ \cdot \LR \conj{\vE_t} 
      \\
      Z_D &= \int_\Gamma \LD \vJ \cdot \LD \conj{\vE_t}  &
      Z_R &= \int_\Gamma \LR \vJ \cdot \LR \conj{\vE_t} 
    \end{align*}

    Les équations \eqref{eq:unicite:ci3:csu3-1} à \eqref{eq:unicite:ci3:csu3-4} sont équivalentes au système \(\mM_R X_R = F_R\) où

    \begin{align*}
      \mM_R & =
      \begin{bmatrix}
        1 & \conj{b_1} & -\conj{b_2} & 0
        \\
        a_0 & a_1 & -a_2 & 0
        \\
        0 & 0 & 1 & -\conj{b_2}
        \\
        0 & 0 & a_0 & -a_2
        \\
      \end{bmatrix}
      &&
      X_R =
      \begin{bmatrix}
        X\\
        Y_D\\
        Y_R\\
        Z_R
      \end{bmatrix}
    \end{align*}
    \begin{align*}
      F_R &=
      \begin{bmatrix}
        \conj{a_0} \norm{\vJ}^2 - \conj{a_1} \norm{\tdivs \vJ}^2 - \conj{a_2} \norm{\trots \vJ}^2
        \\
        \norm{\vE_t}^2  - b_1 \norm{\tdivs \vE}^2  - b_2 \norm{\trots \vE_t}^2
        \\
        \conj{a_0} \norm{\trots \vJ}^2 - \conj{a_2} \norm{\LR \vJ}^2
        \\
        \norm{\trots \vE_t}^2 - \conj{b_2} \norm{\LR \vE_t}^2
      \end{bmatrix}
    \end{align*}

    Tandis que les équations \eqref{eq:unicite:ci3:csu3-1},\eqref{eq:unicite:ci3:csu3-2},\eqref{eq:unicite:ci3:csu3-5},\eqref{eq:unicite:ci3:csu3-6} sont équivalentes au système \(\mM_D X_D= F_D\) où

    \begin{align*}
      \mM_D & =
      \begin{bmatrix}
        1 & -\conj{b_2} & \conj{b_1} & 0
        \\
        a_0 & -a_2 & a_1 & 0
        \\
        0 & 0 & 1 & \conj{b_1}
        \\
        0 & 0 & a_0 & a_1
      \end{bmatrix}
       & &
      X_D =
      \begin{bmatrix}
        X
        \\
        Y_R
        \\
        Y_D
        \\
        Z_D
      \end{bmatrix}
    \end{align*}
    \begin{align*}
      F_D & =
      \begin{bmatrix}
        \conj{a_0} \norm{\vJ}^2ds - \conj{a_1} \norm{\tdivs \vJ}^2 - \conj{a_2} \norm{\trots \vJ}^2  \\
        \norm{\vE_t}^2   - b_1 \norm{\tdivs \vE }^2  - b_2 \norm{\trots \vE_t}^2  \\
        -\conj{a_0} \norm{\tdivs \vJ}^2ds + \conj{a_1} \norm{\LR \vJ}^2  \\
        -\norm{\tdivs \vE_t}^2   + \conj{b_1} \norm{\LR \vE_t}^2 
      \end{bmatrix}
    \end{align*}

    On note dans la suite \(\Delta_i = a_i-\conj{b_i}a_0\), \(i=1,2\). On suppose que ces systèmes soient inversible donc
    \begin{equation*}
      \label{eq:unicite:ci3:csu3-cn-det}
      \Delta_1\Delta_2 \not = 0
    \end{equation*}

    On a alors
    \begin{align*}
      \mM_R^{-1} & =\frac{1}{\Delta_1\Delta_2}
      \begin{bmatrix}
        a_1 \Delta_2 & -\conj{b_1}\Delta_2 & a_1(a_1\conj{b_2}-a_2\conj{b_1}) & -\conj{b_2}(a_1\conj{b_2}-a_2\conj{b_1})
        \\
        -a_0 \Delta_2 & \Delta_2 & a_2\Delta_2 & -\conj{b_2}\Delta_2
        \\
        0 & 0 & a_2\Delta_1 & -\conj{b_2}\Delta_1
        \\
        0 & 0 & a_0\Delta_1 & -\Delta_1
      \end{bmatrix}
      \\
      \mM_D^{-1} & =\frac{1}{\Delta_1\Delta_2}
      \begin{bmatrix}
        a_2 \Delta_1 & -\conj{b_2}\Delta_1 & a_1(a_1\conj{b_2}-a_2\conj{b_1}) & -\conj{b_1}(a_1\conj{b_2}-a_2\conj{b_1})
        \\
        a_0 \Delta_1 & -\Delta_1 & a_1\Delta_1 & -\conj{b_1}\Delta_1
        \\
        0 & 0 & a_1\Delta_2 & -\conj{b_1}\Delta_2
        \\
        0 & 0 & -a_0\Delta_2 & \Delta_2
      \end{bmatrix}
    \end{align*}
    
    Pour chaque système on déduit \(X\), on développe les expressions et enfin on impose la positivité de la partie réelle de chaque terme pour impliquer que \(\Re(X)\ge 0\).

    \begin{minipage}{0.49\textwidth}
      \textbf{Résolution du 1\ier système}:
      \begin{align}
        \label{eq:unicite:ci3:csu3r-j2}&\Re\left(a_0\conj{a_2}\Delta_2\right) \ge 0 \\
        \label{eq:unicite:ci3:csu3r-e2}&\Re\left(\frac{\conj{b_2}}{\Delta_2}\right) \le 0 \\
        \label{eq:unicite:ci3:csu3r-jdj}&\Re\left(\conj{a_0}a_1\left(\frac{\conj{b_1}}{\Delta_1}-\frac{\conj{b_2}}{\Delta_2}\right) + \frac{\conj{a_1}a_2}{\Delta_2} \right)\le 0\\
        \label{eq:unicite:ci3:csu3r-ede}&\Re\left(2\Re(b_1)\frac{\conj{b_2}}{\Delta_2}-\frac{\conj{b_1}^2}{\Delta_1}\right) \ge 0\\
        \label{eq:unicite:ci3:csu3r-jrj}&\Re\left(|a_2|^2\Delta_2\right) \le 0 \\
        \label{eq:unicite:ci3:csu3r-ere}&\Re\left(|b_2|^2\Delta_2\right) \ge 0 \\
        \label{eq:unicite:ci3:csu3r-rj2}&\Re\left(|a_1|^2\left(\frac{\conj{b_1}}{\Delta_1}-\frac{\conj{b_2}}{\Delta_2}\right)\right)\ge 0\\
        \label{eq:unicite:ci3:csu3r-re2}&\Re\left(|b_1|^2\left(\frac{\conj{b_1}}{\Delta_1}-\frac{\conj{b_2}}{\Delta_2}\right)\right)\le 0
      \end{align}
      \eqref{eq:unicite:ci3:csu3r-jrj} et \eqref{eq:unicite:ci3:csu3r-ere} impliquent :
      \begin{equation}
        \Re\left(\Delta_2\right) = 0\\\
      \end{equation}
      \eqref{eq:unicite:ci3:csu3r-rj2} et \eqref{eq:unicite:ci3:csu3r-re2} impliquent :
      \begin{equation}
        \Re\left(\frac{\conj{b_1}}{\Delta_1}-\frac{\conj{b_2}}{\Delta_2}\right) = 0
      \end{equation}
    \end{minipage}
    \begin{minipage}{0.49\textwidth}
      \textbf{Résolution du 2\ieme système}:
      \begin{align}
        \label{eq:unicite:ci3:csu3d-j2}&\Re\left(a_0\conj{a_1}\Delta_1\right) \ge 0 \\
        \label{eq:unicite:ci3:csu3d-e2}&\Re\left(\frac{\conj{b_1}}{\Delta_1}\right) \le 0 \\
        \label{eq:unicite:ci3:csu3d-jrj}&\Re\left(\conj{a_0}a_2\left(\frac{\conj{b_2}}{\Delta_2}-\frac{\conj{b_2}}{\Delta_2}\right) + \frac{\conj{a_2}a_1}{\Delta_1} \right)\le 0\\
        \label{eq:unicite:ci3:csu3d-ere}&\Re\left(2\Re(b_2)\frac{\conj{b_1}}{\Delta_1}-\frac{\conj{b_2}^2}{\Delta_2}\right) \ge 0\\
        \label{eq:unicite:ci3:csu3d-jdj}&\Re\left(|a_1|^2\Delta_1\right) \le 0 \\
        \label{eq:unicite:ci3:csu3d-ede}&\Re\left(|b_1|^2\Delta_1\right) \ge 0 \\
        \label{eq:unicite:ci3:csu3d-dj2}&\Re\left(|a_2|^2\left(\frac{\conj{b_2}}{\Delta_2}-\frac{\conj{b_1}}{\Delta_1}\right)\right)\ge 0\\
        \label{eq:unicite:ci3:csu3d-de2}&\Re\left(|b_2|^2\left(\frac{\conj{b_2}}{\Delta_2}-\frac{\conj{b_1}}{\Delta_1}\right)\right)\le 0
      \end{align}
      \eqref{eq:unicite:ci3:csu3d-jdj} et \eqref{eq:unicite:ci3:csu3d-ede} impliquent :
      \begin{equation}
        \Re\left(\Delta_1\right) = 0\\\
      \end{equation}
      \eqref{eq:unicite:ci3:csu3d-dj2} et \eqref{eq:unicite:ci3:csu3d-de2} impliquent :
      \begin{equation}
        \Re\left(\frac{\conj{b_1}}{\Delta_1}-\frac{\conj{b_2}}{\Delta_2}\right) = 0\\\
      \end{equation}
    \end{minipage}
  \end{proof}

  Pour conclure, on remarque que l'on obtient à 3 groupes de CSU
  \begin{itemize}
    \item Avec le 1\ier système
    \item Avec le 2\ieme système
    \item En combinant ces deux groupes
  \end{itemize}

  Subjectivement, nous choisissons le dernier car il n'y a \textit{a priori} pas de raisons d'appliquer un opérateur et non l'autre.

  On remarque plusieurs conditions d'égalités, ce qui va contraindre certains coefficients. Enfin on remarque que si \(b_1=b_2=0\), alors on n'obtient pas les CSU de la CI4.

\subsubsection{2\ieme méthode}

  \begin{prop}
      Si les conditions suivantes sont vérifiées
      \begin{align}
      \Re\left(a_0\right)\ge 0 \\
      \Re\left(a_1 - \frac{\conj{b_1a_0}a_1}{\Delta_1}\right) \le 0 \\
      \Re\left(a_2 - \frac{\conj{b_2a_0}a_2}{\Delta_2}\right) \le 0 \\
      \Re\left(b_1\Delta_1\right) = 0 \\
      \Re\left(b_2\Delta_2\right) = 0 \\
      \Im\left(b_1\Delta_1\right)\Im(b_1)\ge 0\\
      \Im\left(b_2\Delta_2\right)\Im(b_2)\ge 0
    \end{align}
    alors \(\Re(X)\ge0\)
  \end{prop}

  \begin{proof}
    En se basant sur la méthode précédente, on remarque que l'on peut déterminer les inconnus \((Y_R,Z_R)\) (resp. \((Y_D,Z_R)\)) uniquement en utilisant les équations \eqref{eq:unicite:ci3:csu3-3} et \eqref{eq:unicite:ci3:csu3-4} (resp. \eqref{eq:unicite:ci3:csu3-5} et \eqref{eq:unicite:ci3:csu3-6}).

    On déduit donc que si \(\Delta_1 \not = 0\) et \(\Delta_2 \not = 0\) alors

    \begin{align}
      Y_R &= \frac{1}{\Delta_2}\left(a_2\left(\conj{a_0}\int_\Gamma \vJ\cdot\LR\conj{\vJ} - \conj{a_2}\norm{\LR \vJ}^2\right)  -\conj{b_2}\left(\int_\Gamma \conj{\vE}\LR{\vE} - b_2 \norm{\LR \vE}^2\right)\right) \\
      Y_D &= \frac{1}{\Delta_1}\left(a_1\left(\conj{a_0}\int_\Gamma \vJ\cdot\LD\conj{\vJ} + \conj{a_1}\norm{\LD \vJ}^2\right)  -\conj{b_1}\left(\int_\Gamma \conj{\vE}\LD{\vE} + b_1 \norm{\LD \vE}^2\right)\right)
    \end{align}

    Il reste alors à utiliser l'équation \eqref{eq:unicite:ci3:csu3-1} pour obtenir
    \begin{equation}
      X = -\conj{b_1} Y_D + \conj{b_2} Y_R + \conj{a_0} \norm{\vJ}^2 + \conj{a_1} \int_\Gamma \vJ \cdot \LD \conj{\vJ} - \conj{a_2} \int_\Gamma \vJ \cdot \LR \conj{\vJ}
    \end{equation}

    \begin{multline}
      X = \conj{a_0} \norm{\vJ }^2 - \conj{a_1} \norm{\vdivs \vJ }^2 - \conj{a_2} \norm{\vrots \vJ }^2
      \\
      + \frac{\conj{b_2}}{\Delta_2}\left(a_2\left(\conj{a_0}\norm{\vrots \vJ}^2 - \conj{a_2}\norm{\LR J}^2\right)  -\conj{b_2}\left(\norm{\vrots\vE}^2 - b_2 \norm{\LR \vE }^2\right)\right)
      \\
      - \frac{\conj{b_1}}{\Delta_1}\left(a_1\left(-\conj{a_0}\norm{\vdivs\vJ}^2 + \conj{a_1}\norm{\LD J}^2\right)  -\conj{b_1}\left(-\norm{\vdivs\vE}^2 + b_1 \norm{\LD \vE }^2\right)\right)
    \end{multline}

    On factorise les termes en \(\vJ\)

    \begin{multline}
      X = \conj{a_0} \norm{\vJ }^2 - \left(a_1 - \frac{\conj{b_1a_0}a_1}{\Delta_1}\right) \norm{\vdivs \vJ}^2 - \left(a_2 - \frac{\conj{b_2a_0}a_2}{\Delta_2}\right) \norm{\vrots \vJ }^2
      \\
      + \frac{\conj{b_2}}{\Delta_2}\left( - |a_2|^2\norm{\LR \vJ}^2  - \conj{b_2}\left(\norm{\vrots\vE}^2 - b_2 \norm{\LR \vE }^2\right)\right) 
      \\
      - \frac{\conj{b_1}}{\Delta_1}\left( |a_1|^2\norm{\LD \vJ}^2  - \conj{b_1}\left(-\norm{\vdivs\vE}^2 + b_1 \norm{\LD \vE }^2\right)\right)
    \end{multline}

    On développe tous les termes
    \begin{multline}
      X = \conj{a_0} \norm{\vJ }^2 - \left(a_1 - \frac{\conj{b_1a_0}a_1}{\Delta_1}\right) \norm{\vdivs \vJ }^2 - \left(a_2 - \frac{\conj{b_2a_0}a_2}{\Delta_2}\right) \norm{\vrots \vJ }^2
      \\
      - \frac{\conj{b_2}|a_2|^2}{\Delta_2}\norm{\LR \vJ}^2  -  \frac{\conj{b_2}^2}{\Delta_2}\norm{\vrots\vE}^2 +  \frac{|b_2|}{\Delta_2} \norm{\LR \vE }^2
      \\
      - \frac{\conj{b_1}|a_1|^2}{\Delta_1}\norm{\LD \vJ}^2  - \frac{\conj{b_1}^2}{\Delta_1}\norm{\vdivs\vE}^2 + \frac{|b_1|^2}{\Delta_1} \norm{\LD \vE }^2
    \end{multline}

    On impose alors à la partie réelle de chaque terme d'être positive, et on obtient les CSU suivantes :

    \begin{equation}
      \Re\left(a_0\right)\ge 0
    \end{equation}
    \begin{minipage}{0.5\textwidth}
      \begin{align}
        \Re\left(a_1 - \frac{\conj{b_1a_0}a_1}{\Delta_1}\right) \le 0 \\
        \Re\left(a_2 - \frac{\conj{b_2a_0}a_2}{\Delta_2}\right) \le 0 \\
        \Re\left(\frac{|a_1|^2\conj{b_1}}{\Delta_1}\right) \le 0 \\
        \Re\left(\frac{|a_2|^2\conj{b_2}}{\Delta_2}\right) \le 0
      \end{align}
    \end{minipage}
    \begin{minipage}{0.5\textwidth}
      \begin{align}
        \Re\left(\frac{\conj{b_1}^2}{\Delta_1}\right) \le 0 \\
        \Re\left(\frac{\conj{b_2}^2}{\Delta_2}\right) \le 0 \\
        \Re\left(\frac{|b_1|^2\conj{b_1}}{\Delta_1}\right) \ge 0 \\
        \Re\left(\frac{|b_2|^2\conj{b_2}}{\Delta_2}\right) \ge 0
      \end{align}
    \end{minipage}

    On remarque alors que certaines CSU peuvent se combiner et imposent que les parties réelles de \(b_1\Delta_1,b_2\Delta_2\) soient nulles.

    \begin{align}
      \Re\left(a_0\right)\ge 0 \\
      \Re\left(a_1 - \frac{\conj{b_1a_0}a_1}{\Delta_1}\right) \le 0 \\
      \Re\left(a_2 - \frac{\conj{b_2a_0}a_2}{\Delta_2}\right) \le 0 \\
      \Re\left(b_1\Delta_1\right) = 0 \\
      \Re\left(b_2\Delta_2\right) = 0 \\
      \Im\left(b_1\Delta_1\right)\Im(b_1)\ge 0\\
      \Im\left(b_2\Delta_2\right)\Im(b_2)\ge 0
    \end{align}
  \end{proof}

  On a réussi à réduire le nombre de CSU et en plus, fixer \(b_1=b_2=0\) permet de retomber sur les CSU de la CI4, mais certains coefficients sont définis à partir des autres, ce qui reste trop contraignant à nos yeux.

\subsubsection{3\ieme méthode: CSU de Lafitte-Stupfel}

  Nous avons réussi à trouver des CSU uniquement constituées de contraintes inégalités et en plus qui permettent de retrouver les CSU de la CI4.

  \begin{prop}
    Soit \(z = \left(1 - \frac{b_1a_0}{a_1} - \frac{b_2a_0}{a_2}\right) \). Ces CSU sont
    \begin{align}
      \Re\left(\conj{a_0}z\right) \ge 0
      \\
      \Re\left(\conj{a_1}z\right) \le 0
      \\
      \Re\left(\conj{a_2}z\right) \le 0
      \\
      \Re\left(\frac{b_1}{a_1}\right) \ge 0
      \\
      \Re\left(\frac{b_2}{a_2}\right) \ge 0
      \\
      \Re\left(a_0\right) \ge 0
      \\
      \Re\left(a_1\right) \le 0
      \\
      \Re\left(a_2\right) \le 0
      \\
      \Re\left(\frac{b_1\conj{a_2}}{a_1\conj{a_0}}\right) \le 0
      \\
      \Re\left(\frac{b_2\conj{a_1}}{a_2\conj{a_0}}\right) \le 0
    \end{align}
  \end{prop}

  \begin{proof}
    Par définition de la CIOE, on a

    \begin{align*}
      X &= \int_\Gamma \left(a_0 + a_1 \LD - a_2 \LR \right)^{-1}\left( 1 + b_1 \LD - b_2 \LR \right) \vE_t\cdot \conj{\vE_t}
    \end{align*}

    On développe chaque terme

    \begin{multline*}
      X = \int_\Gamma \left(a_0 + a_1 \LD - a_2 \LR \right)^{-1}
      \\
      + b_1 \left(a_0 + a_1 \LD - a_2 \LR \right)^{-1}\LD
      \\
      \left.
      - b_2 \left(a_0 + a_1 \LD - a_2 \LR \right)^{-1}\LR \right) \vE_t\cdot \conj{\vE_t}
    \end{multline*}

    On suppose \(a_1\not=0\) et \(a_2\not=0\) alors
    \begin{align*}
      \LD & = \frac{a_0 + a_1 \LD - a_0}{a_1}
      \\
      \LR & = -\frac{a_0 - a_2 \LR - a_0}{a_2}
    \end{align*}

    On pose
    \begin{equation}
      z = \left(1 - \frac{b_1a_0}{a_1} - \frac{b_2a_0}{a_2}\right)
    \end{equation}

    On déduit de ce qui précède que

    \begin{multline*}
      X = \int_\Gamma z\left(a_0 + a_1 \LD - a_2 \LR \right)^{-1}
      \\
      + \frac{b_1}{a_1} \left(a_0 + a_1 \LD - a_2 \LR \right)^{-1}\left(a_0+a_1\LD\right)
      \\
      - \frac{b_2}{a_2} \left(a_0 + a_1 \LD - a_2 \LR \right)^{-1}\left(a_0-a_2\LR\right) \vE_t\cdot \conj{\vE_t}
    \end{multline*}

    On définit

    \newcommand{\vV}{\vect{V}}
    \newcommand{\vW}{\vect{W}}

    \begin{align*}
      \vV & = \left(a_0  + a_1 \LD - a_2\LR \right)^{-1} \vE_t
      \\
      \vW_1 & = \left( - a_2 \left( a_0 + a_1\LD\right)^{-1}\LR\right)^{-1} \vE_t
      \\
      \vW_2 & = \left( + a_1 \left( a_0 - a_2\LR\right)^{-1}\LD\right)^{-1} \vE_t
    \end{align*}

    On déduit

    \begin{multline*}
      X = \int_\Gamma z \vV \cdot \left(\conj{a_0}  + \conj{a_1} \LD - \conj{a_2}\LR\right)\conj{\vV}
      \\
      + \frac{b_1}{a_1} \left( - \conj{a_2} \left( \conj{a_0} + \conj{a_1}\LD\right)^{-1}\LR\right)\conj{\vW_1}\cdot\vW_1
      \\
      + \frac{b_2}{a_2} \left( + \conj{a_1} \left( \conj{a_0} - \conj{a_2}\LR\right)^{-1}\LD\right)\conj{\vW_2}\cdot\vW_2
    \end{multline*}

    Notons

    \newcommand{\vR}{\vect{R}}

    \begin{align*}
      \vR_1 & = \left(\conj{a_0}  + \conj{a_1} \LD \right)^{-1}\LR \conj{\vW_1}
      \\
      \vR_2 & = \left(\conj{a_0}  - \conj{a_2} \LR \right)^{-1}\LD \conj{\vW_2}
    \end{align*}

    Donc 
    \begin{equation*}
      X = \int_\Gamma z \vV \cdot \left(\conj{a_0}  + \conj{a_1} \LD - \conj{a_2}\LR\right)\conj{\vV} - \frac{b_1}{a_1}\conj{a_2}\vR_1\cdot\vW_1 + \frac{b_2}{a_2}\conj{a_1}\vR_2\cdot\vW_2
    \end{equation*}

    Puisque \(\LD\) (resp. \(\LR\)) commute avec lui-même, on a les égalités suivantes

    \begin{align*}
      \LD\left(\conj{a_0}  + \conj{a_1} \LD \right)&=\left(\conj{a_0}  + \conj{a_1} \LD \right)\LD
      \\
      \LR\left(\conj{a_0}  - \conj{a_2} \LR \right)&=\left(\conj{a_0}  - \conj{a_2} \LR \right)\LR
    \end{align*}

    Or on a démontré que \(\LD\LR=\LR\LD=0\), et ainsi

    \begin{equation*}
      \begin{aligned}
        \LD\LR\conj{\vW_1} &= \LD\left(\conj{a_0}  + \conj{a_1} \LD \right)\vR_1
        \\
        0 & =\left(\conj{a_0}  + \conj{a_1} \LD \right)\LD\vR_1
      \end{aligned}
    \end{equation*}
    \begin{equation*}
      \begin{aligned}
        \LR\LD\conj{\vW_2} &= \LR\left(\conj{a_0}  - \conj{a_2} \LR \right)\vR_2
        \\
        0 & =\left(\conj{a_0}  - \conj{a_2} \LR \right)\LR\vR_2
      \end{aligned}
    \end{equation*}

    Pour conclure, il faut utiliser le résultat suivant
    \begin{prop}[Injectivité]
      On suppose que
      \begin{align*}
        \Re(a_0)\ge0 && \Re(a_1) \le 0 && \Re(a_2) \le 0
      \end{align*}
      alors \(a_0 + a_1 \LD\)  et \(a_0 - a_2 \LR\) sont injectif
    \end{prop}

    \begin{proof}
      Par définition, \(a_0 + a_1 \LD\) est injectif si pour \(\vect{U}\) vecteur régulier tangent  à  \(\Gamma\)
      \begin{align*}
        \int_\Gamma \left(a_0 + a_1 \LD\right)\vect{U}\cdot\conj{\vect{U}} &= 0 \Rightarrow \vect{U} = 0
        \intertext{Or par définition de l'opérateur \(\LD\)}
        \int_\Gamma \left(a_0 + a_1 \LD\right)\vect{U}\cdot\conj{\vect{U}} &=a_0\norm{\vect{U}}^2 - a_1\norm{\tdivs{\vect{U}}}^2
      \end{align*}
      
      Prenons la partie réelle de cette égalité et injectons les hypothèses sur les coefficients. Alors le membre de droite ne contient que des termes positifs, donc tous ces termes sont nuls. Pour que ce soit vrai pour tous \(a_0,a_1,a_2\), c'est donc que \(\norm{\vect{U}} = 0\) et \(\norm{\tdivs\vect{U}} = 0\) donc \(\vect{U} = 0\).

      Le même raisonnement est valable pour \(a_0 - a_2 \LR\).
    \end{proof}

    On suppose donc \(\Re(a_0) \ge 0 \),\(\Re(a_1) \le 0\) et \(\Re(a_2)\le0\)), donc \(\left(\conj{a_0}  + \conj{a_1} \LD \right)\) et \(\left(\conj{a_0}  - \conj{a_2} \LR \right)\) sont injectifs et donc on déduit que

    \begin{align*}
      \left(\conj{a_0}  + \conj{a_1} \LD \right)\LD\vR_1 = 0 &\Rightarrow \LD\vR_1 = 0
      \\
      \left(\conj{a_0}  - \conj{a_2} \LR \right)\LR\vR_2 = 0 &\Rightarrow \LR\vR_2 = 0
    \end{align*}

    Or par définition \(\LR\conj{\vW_1} = \left(\conj{a_0}  + \conj{a_1} \LD \right)\vR_1\) et \(\LD\conj{\vW_2} = \left(\conj{a_0}  - \conj{a_2} \LR \right)\vR_2\) donc
    \begin{align*}
      \LR\conj{\vW_1} = \conj{a_0}\vR_1 && \LD\conj{\vW_2} = \conj{a_0}\vR_2
      \intertext{donc}
      \vR_1 = \frac{1}{\conj{a_0}}\LR\conj{\vW_1} && \vR_2 = \frac{1}{\conj{a_0}}\LD\conj{\vW_2}
    \end{align*}

    On réinjecte ce résultat dans la définition de \(X\)

    \begin{multline*}
      X = \int_\Gamma z \vV \cdot \left(\conj{a_0}  + \conj{a_1} \LD - \conj{a_2}\LR\right)\conj{\vV}
      \\
      + \frac{b_1}{a_1} \norm{\vW_1}^2 - \frac{b_1\conj{a_2}}{a_1\conj{a_0}}\int_\Gamma \LR\conj{\vW_1}\cdot\vW_1
      \\
      + \frac{b_2}{a_2} \norm{\vW_2}^2 + \frac{b_2\conj{a_1}}{a_2\conj{a_0}}\int_\Gamma \LD\conj{\vW_2}\cdot\vW_2
    \end{multline*}

    Et il suffit d'imposer la positivité de la partie réelle de chaque terme pour avoir \(\Re(X)\ge 0\)

    \begin{minipage}{0.5\textwidth}
    \begin{align}
      \Re\left(\conj{a_0}z\right) \ge 0
      \\
      \Re\left(\conj{a_1}z\right) \le 0
      \\
      \Re\left(\conj{a_2}z\right) \le 0
      \\
      \Re\left(\frac{b_1}{a_1}\right) \ge 0
      \\
      \Re\left(\frac{b_2}{a_2}\right) \ge 0
    \end{align}
    \end{minipage}
    \begin{minipage}{0.49\textwidth}
    \begin{align}
      \Re\left(a_0\right) \ge 0
      \\
      \Re\left(a_1\right) \le 0
      \\
      \Re\left(a_2\right) \le 0
      \\
      \Re\left(\frac{b_1\conj{a_2}}{a_1\conj{a_0}}\right) \le 0
      \\
      \Re\left(\frac{b_2\conj{a_1}}{a_2\conj{a_0}}\right) \le 0
    \end{align}
    \end{minipage}
  \end{proof}

  On obtient donc un nombre de CSU élevé mais il n'y a aucune condition d'égalité, de plus, on retrouve bien les CSU de la CI4. Ce seront les CSU de référence pour cette CIOE.