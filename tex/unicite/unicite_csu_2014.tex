\section[Des CSU pour la CIOE de Marceaux 2000]{Des conditions suffisantes pour la CIOE de \cite{marceaux_high-order_2000}}

  \begin{defn}
    On définit les opérateurs \gls{ope-LD} et \gls{ope-LR} tels que %sont introduits par \cite[eq.~4]{stupfel_implementation_2015} et s'expriment pour tous vecteurs complexes régulier tangents à \(\Gamma\) 

    Pour tous \(\vect U,\vect V \in (\mathcal{C}^\infty(\Gamma))^2\): 
    \begin{align*}
      \LD(\vect U) &= \tgrads (\tdivs \vect U)
      \\
      \LR(\vect V) &= \tvrots( \tvrots \vect V)
    \end{align*}
  \end{defn}

  \begin{prop}
    \(\LD\) est hermitien symétrique négatif et \(\LR\) est hermitien symétrique positif.

    Pour tous \(\vect U \in (\mathcal{C}^\infty(\Gamma))^2\): 
    \begin{align*}
      \int_\Gamma \LD(\vect U)\cdot \conj{\vect U} &= -\norm{\tdivs \vect U}^2
      \\
      \int_\Gamma \LR(\vect U)\cdot \conj{\vect U} &=\norm{\vn \cdot \tvrots \vect U}^2
    \end{align*}
  \end{prop}

  \begin{prop}
    Soit \(\OO\) un domaine borné de \(\RR^3\) , de surface \(\Gamma\) fermée et régulière où \(\vect n\) y est la normale unitaire
    sortante
    \begin{equation}
      \begin{matrix}
        \forall \vect U \in (\mathcal{C}^\infty(\Gamma,\CC)) ,& \LR(\LD(\vect U)) = \LD(\LR(\vect U)) = 0
      \end{matrix}
    \end{equation}
  \end{prop}

  \begin{proof}
    Soit (\(x_1,x_2\)) un système de coordonnées locales à \(\Gamma\)

    Soit un vecteur tangent défini en tout point de la surface, et exprimé dans la base locale \(\vect{u_1},\vect{u_2},\vect{n}\)
    \[
      \vect{U} = 
      \begin{bmatrix}
        U_1(x_1,x_2)
        \\
        U_2(x_1,x_2)
        \\
        0
      \end{bmatrix}
    \]

    D'après \cite[p.~1028, A3.22]{bladel_electromagnetic_2007}, le rotationnel surfacique d'un vecteur tangent est un vecteur normal à la surface
    \[
      \tvrots{\vect{U}} =\vn \left(\vn\cdot\trot{\vect{U}}\right)
    \]

    Montrons que \(\LR\LD = 0\).

    D’après \cite[p.~1029, A3.42]{bladel_electromagnetic_2007}, soit \(f(x_1,x_2)\) une fonction régulière définie sur \(\Gamma\) alors \(\vn \cdot \tvrots(\tgrads f(x_1,x_2)) = 0\).

    On exprime les opérateurs
    \begin{align*}
      \LR(\LD \vect U)  &= \tvrots \left(\vn \left(\vn \cdot \tvrots \left( \tgrads \left(\tdivs \vect U\right)\right)\right)\right) \\
      &= 0
    \end{align*}

    Montrons que \(\LD\LR = 0\).

    D’après \cite[p.~1029, A3.43]{bladel_electromagnetic_2007}, \(\tdivs \trots (f\vn) = 0\).
    \begin{align*}
      \LD(\LR \vect U) &= \tgrads \tdivs \tvrots (\vn (\vn \cdot \tvrots \vect U)) \\
      &= 0
    \end{align*}
  \end{proof}

\subsection{CSU pour la CI4}
  Soit la CIOE que l'on nomme \hyperlink{ci4}{CI4} :
  \begin{equation}
    \label{eq:unicite:ci4:ci4}
    \vE_t = (a_0 + a_1 \LD - a_2 \LR ) \vJ
  \end{equation}

  \begin{defn}
    \label{def:csu:ci4}

    On définit le sous-espace fermé de \(\CC^3\)
    \begin{equation*}
      \CSU{CI4} = \left\lbrace 
      \begin{matrix}
      (a_0,a_1,a_2) \in \CC^3
      \\
      \Re(a_0) \ge 0
      \\
      \Re(a_1) \le 0
      \\
      \Re(a_2) \le 0
      \end{matrix}
      \right\rbrace
    \end{equation*}
  \end{defn}

 \begin{prop}[Une CSU pour la CI4]
    \label{prop:csu:ci4}
    Il suffit que
    \begin{equation*}
      (a_0,a_1,a_2) \in \CSU{CI4}
    \end{equation*}
    pour que \(\Re(X)\ge 0\), ce qui entraîne l'unicité.
  \end{prop}

  \begin{proof}
    Par définition de \(X\) \eqref{eq:unicite:x}, on a
    \begin{align*}
      X &= \conj{a_0}\norm{\vJ}^2 + \conj{a_1}\norm{\vdivs{\vJ}}^2 - \conj{a_2}\norm{\vn \cdot \vrots{\vJ}}^2
      \intertext{donc}
      \Re(X) &= \Re(\conj{a_0})\norm{\vJ}^2 + \Re(\conj{a_1})\norm{\vdivs{\vJ}}^2 - \Re(\conj{a_2})\norm{\vn \cdot \vrots{\vJ}}^2
    \end{align*}
  \end{proof}

  Soit \(S = \left\lbrace (a_0,a_1,a_2) \in \CC^3 ; a_1 = a_2 \right\rbrace \). On remarque que
  \begin{align}
    \CSU{CI4}\cap S &= (\CSU{CI01}\times\CC)\cap S 
  \end{align}
  Pour des fonctions infiniment régulières, la CIOE CI4 avec \(a_1=a_2\) est équivalente à la CI01 et on a équivalence des CSU correspondantes.


\subsection{CSU pour la CI3}

  Soit la CIOE énoncée dans \cite{marceaux_high-order_2000} que l'on nomme \hyperlink{ci3}{CI3} :
  \begin{equation}
    \label{eq:unicite:ci3:ci3}
    ( 1 + b_1 \LD - b_2 \LR)\vE_t = (a_0 + a_1 \LD - a_2 \LR ) \vJ
  \end{equation}

  Par abus de notation, on omet les variables de la fonction \(\Delta\) \eqref{eq:unicite:delta} 
  \begin{align*}
     \Delta(a_0,a_1,b_1) &\equiv \Delta_1
     \\
     \Delta(a_0,a_2,b_2) &\equiv \Delta_2
  \end{align*}

  \begin{defn}
    \label{def:csu:ci3-1}

    On définit le sous-espace ouvert de \(\CC^5\)
    \begin{equation*}
      \CSU[1]{CI3} = \left\lbrace 
      \begin{matrix}
      (a_0,a_1,a_2,b_1,b_2) \in \CC^5
      \\
      \Delta_1 \not = 0
      \\
      \Delta_2 \not = 0
      \\
      \Re\left(a_0\conj{a_1}\Delta_1\right) \ge 0
      \\
      \Re\left(\frac{\conj{b_1}}{\Delta_1}\right) \le 0
      \\
      \Re\left(\conj{a_0}a_2\left(\frac{\conj{b_1}}{\Delta_1}-\frac{\conj{b_2}}{\Delta_2}\right) + \frac{\conj{a_2}a_1}{\Delta_1} \right)\le 0
      \\
      \Re\left(2\Re(b_2)\frac{\conj{b_1}}{\Delta_1}-\frac{\conj{b_2}^2}{\Delta_2}\right) \ge 0
      \\
      \Re\left(a_0\conj{a_2}\Delta_2\right) \ge 0
      \\
      \Re\left(\frac{\conj{b_2}}{\Delta_2}\right) \le 0
      \\
      \Re\left(\conj{a_0}a_1\left(\frac{\conj{b_1}}{\Delta_1}-\frac{\conj{b_2}}{\Delta_2}\right) + \frac{\conj{a_1}a_2}{\Delta_2} \right)\le 0
      \\
      \Re\left(2\Re(b_1)\frac{\conj{b_2}}{\Delta_2}-\frac{\conj{b_1}^2}{\Delta_1}\right) \ge 0
      \\
      \Re\left(\Delta_1\right) = 0
      \\
      \Re\left(\Delta_2\right) = 0
      \\
      \Re\left(\frac{\conj{b_2}}{\Delta_2}-\frac{\conj{b_1}}{\Delta_1}\right) = 0
      \end{matrix}
      \right\rbrace
    \end{equation*}
  \end{defn}
  
 \begin{prop}[Une première CSU pour la CI3]
    \label{prop:csu:ci3-1}
    Il suffit que
    \begin{equation*}
      (a_0,a_1,a_2,b_1,b_2) \in \CSU[1]{CI3}
    \end{equation*}
    pour que \(\Re(X)\ge 0\), ce qui entraîne l'unicité.
  \end{prop}

  \begin{proof}
    Soit \(\LL_3\) l'opérateur tel que
    \begin{equation}
      \label{eq:unicite:ci3:L3}
      \begin{matrix}
      \LL_3:& (\mathcal{C}^\infty(\Gamma)^2)^2 & \rightarrow & (\mathcal{C}^\infty(\Gamma)^2)^2
      \\
      &(\vE_t,\vJ) & \mapsto & ( 1 + b_1 \LD - b_2 \LR) \vE_t - (a_0 + a_1 \LD - a_2 \LR ) \vJ
      \end{matrix}
    \end{equation}

    On calcule \(\int_\Gamma \conj{\LL_3(\vE_t,\vJ)}\cdot \vJ = 0\)
    \begin{multline}
      \label{eq:unicite:ci3:csu3-1}
      \int_\Gamma \vJ \cdot \conj{\vE_t}   + \conj{b_1} \int_\Gamma \vJ\cdot \LD\conj{\vE_t}  - \conj{b_2} \int_\Gamma \vJ \LR\conj{\vE_t}  \\
      = \conj{a_0} \norm{\vJ}^2ds - \conj{a_1} \norm{\tdivs \vJ}^2  - \conj{a_2} \norm{\trots \vJ}^2 
    \end{multline}
    On calcule \(\int_\Gamma {\LL_3(\vE_t,\vJ)}\cdot \conj{\vE_t} = 0\)
    \begin{multline}
      \label{eq:unicite:ci3:csu3-2}
      \norm{\vE_t}^2   - b_1 \norm{ \tdivs \vE }^2  - b_2 \norm{\trots \vE_t}^2  \\
      = a_0 \int_\Gamma \vJ\cdot \conj{\vE_t} + a_1 \int_\Gamma \conj{\vE_t} \LD \vJ  - a_2 \int_\Gamma \conj{\vE_t} \cdot \LR \vJ 
    \end{multline}
    On calcule \(\int_\Gamma \LR\conj{\LL_3(\vE_t,\vJ)}\cdot\vJ = 0\)
    \begin{equation}
      \label{eq:unicite:ci3:csu3-3}
      \int_\Gamma \vJ \cdot \LR \conj{\vE_t}   - \conj{b_2} \int_\Gamma \LR \vJ \cdot \LR \conj{\vE_t}
      =  \conj{a_0} \norm{\vn \cdot \trots \vJ}^2ds - \conj{a_2} \norm{ \LR \vJ}^2 
    \end{equation}
    On calcule \(\int_\Gamma \LR{\LL_3(\vE_t,\vJ)}\cdot\conj{\vE_t}=0\)
    \begin{equation}
      \label{eq:unicite:ci3:csu3-4}
      \norm{ \vn \cdot \trots \vE_t }^2   - \conj{b_2} \norm{ \LR \vE_t}^2 
      = a_0 \int_\Gamma \conj{\vE_t} \LR \vJ  - a_2 \int_\Gamma \LR \conj{\vE_t} \cdot \LR \vJ 
    \end{equation}
    On calcule \(\int_\Gamma \LD\conj{\LL_3(\vE_t,\vJ)}\cdot\vJ=0\)
    \begin{equation}
      \label{eq:unicite:ci3:csu3-5}
      \int_\Gamma \vJ \cdot \LD \conj{\vE_t}   + \conj{b_1} \int_\Gamma \LD \vJ \cdot \LD \conj{\vE_t}
      = - \conj{a_0} \norm{\tdivs \vJ}^2ds + \conj{a_1} \norm{ \LD \vJ}^2 
    \end{equation}
    On calcule \(\int_\Gamma \LD{\LL_3(\vE_t,\vJ)}\cdot\conj{\vE_t}=0\)
    \begin{equation}
      \label{eq:unicite:ci3:csu3-6}
      -\norm{ \tdivs \vE_t }^2   + \conj{b_1} \norm{ \LD \vE_t}^2
      = a_0 \int_\Gamma \conj{\vE_t} \LD \vJ  + a_1 \int_\Gamma \LD \conj{\vE_t} \cdot \LD \vJ 
    \end{equation}

    On note
    \begin{align*}
      Y_D &= \int_\Gamma \vJ \cdot \LD \conj{\vE_t}  &
      Y_R &= \int_\Gamma \vJ \cdot \LR \conj{\vE_t} 
      \\
      Z_D &= \int_\Gamma \LD \vJ \cdot \LD \conj{\vE_t}  &
      Z_R &= \int_\Gamma \LR \vJ \cdot \LR \conj{\vE_t} 
    \end{align*}

    Les égalités \eqref{eq:unicite:ci3:csu3-1} à \eqref{eq:unicite:ci3:csu3-4} sont équivalentes au système

    \begin{align*}
      \mM_R X_R &= F_R
      \\
      \begin{bmatrix}
        1 & \conj{b_1} & -\conj{b_2} & 0
        \\
        a_0 & a_1 & -a_2 & 0
        \\
        0 & 0 & 1 & -\conj{b_2}
        \\
        0 & 0 & a_0 & -a_2
        \\
      \end{bmatrix}
      \begin{bmatrix}
        X\\
        Y_D\\
        Y_R\\
        Z_R
      \end{bmatrix}
      &=
      \begin{bmatrix}
        \conj{a_0} \norm{\vJ}^2 - \conj{a_1} \norm{\tdivs \vJ}^2 - \conj{a_2} \norm{\trots \vJ}^2
        \\
        \norm{\vE_t}^2  - b_1 \norm{\tdivs \vE}^2  - b_2 \norm{\trots \vE_t}^2
        \\
        \conj{a_0} \norm{\trots \vJ}^2 - \conj{a_2} \norm{\LR \vJ}^2
        \\
        \norm{\trots \vE_t}^2 - \conj{b_2} \norm{\LR \vE_t}^2
      \end{bmatrix}
    \end{align*}
    tandis que les égalités \eqref{eq:unicite:ci3:csu3-1},\eqref{eq:unicite:ci3:csu3-2},\eqref{eq:unicite:ci3:csu3-5},\eqref{eq:unicite:ci3:csu3-6} sont équivalentes au système

    \begin{align*}
      \mM_D  X_D & =  F_D
      \\
      \begin{bmatrix}
        1 & -\conj{b_2} & \conj{b_1} & 0
        \\
        a_0 & -a_2 & a_1 & 0
        \\
        0 & 0 & 1 & \conj{b_1}
        \\
        0 & 0 & a_0 & a_1
      \end{bmatrix}
      \begin{bmatrix}
        X
        \\
        Y_R
        \\
        Y_D
        \\
        Z_D
      \end{bmatrix}
      & =
      \begin{bmatrix}
        \conj{a_0} \norm{\vJ}^2ds - \conj{a_1} \norm{\tdivs \vJ}^2 - \conj{a_2} \norm{\trots \vJ}^2  \\
        \norm{\vE_t}^2   - b_1 \norm{\tdivs \vE }^2  - b_2 \norm{\trots \vE_t}^2  \\
        -\conj{a_0} \norm{\tdivs \vJ}^2ds + \conj{a_1} \norm{\LR \vJ}^2  \\
        -\norm{\tdivs \vE_t}^2   + \conj{b_1} \norm{\LR \vE_t}^2 
      \end{bmatrix}
    \end{align*}

    Dans le cas où ces matrices sont inversibles
    \begin{equation*}
      \label{eq:unicite:ci3:csu3-cn-det}
      \Delta_1\Delta_2 \not = 0
    \end{equation*}

    On a alors
    \begin{align*}
      \mM_R^{-1} & =\frac{1}{\Delta_1\Delta_2}
      \begin{bmatrix}
        a_1 \Delta_2 & -\conj{b_1}\Delta_2 & a_2(a_1\conj{b_2}-a_2\conj{b_1}) & -\conj{b_2}(a_1\conj{b_2}-a_2\conj{b_1})
        \\
        -a_0 \Delta_2 & \Delta_2 & a_2\Delta_2 & -\conj{b_2}\Delta_2
        \\
        0 & 0 & a_2\Delta_1 & -\conj{b_2}\Delta_1
        \\
        0 & 0 & a_0\Delta_1 & -\Delta_1
      \end{bmatrix}
      \\
      \mM_D^{-1} & =\frac{1}{\Delta_1\Delta_2}
      \begin{bmatrix}
        a_2 \Delta_1 & -\conj{b_2}\Delta_1 & a_1(a_1\conj{b_2}-a_2\conj{b_1}) & -\conj{b_1}(a_1\conj{b_2}-a_2\conj{b_1})
        \\
        a_0 \Delta_1 & -\Delta_1 & a_1\Delta_1 & -\conj{b_1}\Delta_1
        \\
        0 & 0 & a_1\Delta_2 & -\conj{b_1}\Delta_2
        \\
        0 & 0 & -a_0\Delta_2 & \Delta_2
      \end{bmatrix}
    \end{align*}
    
    On déduit \(X_D\) et \(X_R\) en fonction de toutes les normes présentent dans \(F_D\) et \(F_R\), on développe les expressions et on considère la partie réelle de chaque coefficient. Pour conclure, il suffit qu'elles soient toutes positives.

    On pose \(\Theta=\frac{\conj{b_1}}{\Delta_1}-\frac{\conj{b_2}}{\Delta_2}\).

    \begin{minipage}{0.49\textwidth}
      {Système \(\mM_RX_R=F_R\)}:
      \begin{align}
        \label{eq:unicite:ci3:csu3r-j2}&\Re\left(a_0\conj{a_2}\Delta_2\right) \ge 0 \\
        \label{eq:unicite:ci3:csu3r-e2}&\Re\left(\frac{\conj{b_2}}{\Delta_2}\right) \le 0 \\
        \label{eq:unicite:ci3:csu3r-jdj}&\Re\left(\conj{a_0}a_1\Theta + \frac{\conj{a_1}a_2}{\Delta_2} \right)\le 0\\
        \label{eq:unicite:ci3:csu3r-ede}&\Re\left(2\Re(b_1)\frac{\conj{b_2}}{\Delta_2}-\frac{\conj{b_1}^2}{\Delta_1}\right) \ge 0\\
        \label{eq:unicite:ci3:csu3r-jrj}&\Re\left(|a_2|^2\Delta_2\right) \le 0 \\
        \label{eq:unicite:ci3:csu3r-ere}&\Re\left(|b_2|^2\Delta_2\right) \ge 0 \\
        \label{eq:unicite:ci3:csu3r-rj2}&\Re\left(|a_1|^2\Theta\right)\ge 0\\
        \label{eq:unicite:ci3:csu3r-re2}&\Re\left(|b_1|^2\Theta\right)\le 0
      \end{align}
      \eqref{eq:unicite:ci3:csu3r-jrj} et \eqref{eq:unicite:ci3:csu3r-ere} impliquent :
      \begin{equation}
        \Re\left(\Delta_2\right) = 0
      \end{equation}
      \eqref{eq:unicite:ci3:csu3r-rj2} et \eqref{eq:unicite:ci3:csu3r-re2} impliquent :
      \begin{equation}
        \Re\left(\Theta\right) = 0
      \end{equation}
    \end{minipage}
    \begin{minipage}{0.49\textwidth}
      {Système \(\mM_DX_D=F_D\)}:
      \begin{align}
        \label{eq:unicite:ci3:csu3d-j2}&\Re\left(a_0\conj{a_1}\Delta_1\right) \ge 0 \\
        \label{eq:unicite:ci3:csu3d-e2}&\Re\left(\frac{\conj{b_1}}{\Delta_1}\right) \le 0 \\
        \label{eq:unicite:ci3:csu3d-jrj}&\Re\left(\conj{a_0}a_2\Theta + \frac{\conj{a_2}a_1}{\Delta_1} \right)\le 0\\
        \label{eq:unicite:ci3:csu3d-ere}&\Re\left(2\Re(b_2)\frac{\conj{b_1}}{\Delta_1}-\frac{\conj{b_2}^2}{\Delta_2}\right) \ge 0\\
        \label{eq:unicite:ci3:csu3d-jdj}&\Re\left(|a_1|^2\Delta_1\right) \le 0 \\
        \label{eq:unicite:ci3:csu3d-ede}&\Re\left(|b_1|^2\Delta_1\right) \ge 0 \\
        \label{eq:unicite:ci3:csu3d-dj2}&\Re\left(|a_2|^2\Theta\right)\ge 0\\
        \label{eq:unicite:ci3:csu3d-de2}&\Re\left(|b_2|^2\Theta\right)\le 0
      \end{align}
      \eqref{eq:unicite:ci3:csu3d-jdj} et \eqref{eq:unicite:ci3:csu3d-ede} impliquent :
      \begin{equation}
        \Re\left(\Delta_1\right) = 0
      \end{equation}
      \eqref{eq:unicite:ci3:csu3d-dj2} et \eqref{eq:unicite:ci3:csu3d-de2} impliquent :
      \begin{equation}
        \Re\left(\Theta\right) = 0
      \end{equation}
    \end{minipage}

    Pour conclure, on remarque que l'on obtient 3 CSU
    \begin{itemize}
      \item Les conditions du 1\ier système
      \item Les conditions du 2\ieme système
      \item L'union de ce ces deux ensembles
    \end{itemize}

    Subjectivement, nous choisissons la dernière, car il n'y a \textit{a priori} pas de raisons de privilégier un opérateur et non l'autre.
  \end{proof}

  Soit \(S = \left\lbrace (a_0,a_1,a_2,b_1,b_2) \in \CC^5 ; a_1 = a_2 ; b_1 = b_2 = 0 \right\rbrace \). On remarque que
  \begin{align}
    \CSU[1]{CI3} & \subset \CSU{CI4}\times\CC^2
    \\
    \CSU[1]{CI3}\cap S & \subsetneq (\CSU{CI4}\times\CC^2)\cap S 
  \end{align}

  \begin{defn}
    \label{def:csu:ci3-2}

    On définit le sous-espace ouvert de \(\CC^5\)
    \begin{equation*}
      \CSU[2]{CI3} = \left\lbrace 
      \begin{matrix}
      (a_0,a_1,a_2,b_1,b_2) \in \CC^5
      \\
      \Delta_1 \not = 0
      \\
      \Delta_2 \not = 0
      \\
      \Re\left(a_0\right)\ge 0
      \\
      \Re\left(a_1 - \frac{\conj{b_1a_0}a_1}{\Delta_1}\right) \le 0
      \\
      \Re\left(a_2 - \frac{\conj{b_2a_0}a_2}{\Delta_2}\right) \le 0
      \\
      \Re\left(b_1\Delta_1\right) = 0
      \\
      \Re\left(b_2\Delta_2\right) = 0
      \\
      \Im\left(b_1\Delta_1\right)\Im(b_1)\ge 0
      \\
      \Im\left(b_2\Delta_2\right)\Im(b_2)\ge 0
      \end{matrix}
      \right\rbrace
    \end{equation*}
  \end{defn}

  
 \begin{prop}[Une deuxième CSU pour la CI3]
    \label{prop:csu:ci3-2}
    Il suffit que
    \begin{equation*}
      (a_0,a_1,a_2,b_1,b_2) \in \CSU[2]{CI3}
    \end{equation*}
    pour que \(\Re(X)\ge 0\), ce qui entraîne l'unicité.
  \end{prop}

  \begin{proof}
    En se basant sur la méthode précédente, on remarque que l'on peut déterminer les inconnues \((Y_R,Z_R)\) (resp. \((Y_D,Z_R)\)) uniquement en utilisant les équations \eqref{eq:unicite:ci3:csu3-3} et \eqref{eq:unicite:ci3:csu3-4} (resp. \eqref{eq:unicite:ci3:csu3-5} et \eqref{eq:unicite:ci3:csu3-6}).

    On déduit donc que si \(\Delta_1 \not = 0\) et \(\Delta_2 \not = 0\) alors

    \begin{align}
      Y_R &= \frac{1}{\Delta_2}\left(a_2\left(\conj{a_0}\int_\Gamma \vJ\cdot\LR\conj{\vJ} - \conj{a_2}\norm{\LR \vJ}^2\right)  -\conj{b_2}\left(\int_\Gamma \conj{\vE}\LR{\vE} - b_2 \norm{\LR \vE}^2\right)\right) \\
      Y_D &= \frac{1}{\Delta_1}\left(a_1\left(\conj{a_0}\int_\Gamma \vJ\cdot\LD\conj{\vJ} + \conj{a_1}\norm{\LD \vJ}^2\right)  -\conj{b_1}\left(\int_\Gamma \conj{\vE}\LD{\vE} + b_1 \norm{\LD \vE}^2\right)\right)
    \end{align}

    Il reste alors à utiliser l'équation \eqref{eq:unicite:ci3:csu3-1} pour obtenir
    \begin{equation}
      X = -\conj{b_1} Y_D + \conj{b_2} Y_R + \conj{a_0} \norm{\vJ}^2 + \conj{a_1} \int_\Gamma \vJ \cdot \LD \conj{\vJ} - \conj{a_2} \int_\Gamma \vJ \cdot \LR \conj{\vJ}
    \end{equation}

    \begin{multline}
      X = \conj{a_0} \norm{\vJ }^2 - \conj{a_1} \norm{\vdivs \vJ }^2 - \conj{a_2} \norm{\vrots \vJ }^2
      \\
      + \frac{\conj{b_2}}{\Delta_2}\left(a_2\left(\conj{a_0}\norm{\vrots \vJ}^2 - \conj{a_2}\norm{\LR J}^2\right)  -\conj{b_2}\left(\norm{\vrots\vE}^2 - b_2 \norm{\LR \vE }^2\right)\right)
      \\
      - \frac{\conj{b_1}}{\Delta_1}\left(a_1\left(-\conj{a_0}\norm{\vdivs\vJ}^2 + \conj{a_1}\norm{\LD J}^2\right)  -\conj{b_1}\left(-\norm{\vdivs\vE}^2 + b_1 \norm{\LD \vE }^2\right)\right)
    \end{multline}

    On regroupe les termes en \(\norm{\vJ}^2\), \(\norm{\vdivs\vJ}^2\),  \(\norm{\vrots\vJ}^2\)
    \begin{multline}
      X = \conj{a_0} \norm{\vJ }^2 - \left(a_1 - \frac{\conj{b_1a_0}a_1}{\Delta_1}\right) \norm{\vdivs \vJ}^2 - \left(a_2 - \frac{\conj{b_2a_0}a_2}{\Delta_2}\right) \norm{\vrots \vJ }^2
      \\
      + \frac{\conj{b_2}}{\Delta_2}\left( - |a_2|^2\norm{\LR \vJ}^2  - \conj{b_2}\left(\norm{\vrots\vE}^2 - b_2 \norm{\LR \vE }^2\right)\right) 
      \\
      - \frac{\conj{b_1}}{\Delta_1}\left( |a_1|^2\norm{\LD \vJ}^2  - \conj{b_1}\left(-\norm{\vdivs\vE}^2 + b_1 \norm{\LD \vE }^2\right)\right)
    \end{multline}

    On développe tous les termes
    \begin{multline}
      X = \conj{a_0} \norm{\vJ }^2 - \left(a_1 - \frac{\conj{b_1a_0}a_1}{\Delta_1}\right) \norm{\vdivs \vJ }^2 - \left(a_2 - \frac{\conj{b_2a_0}a_2}{\Delta_2}\right) \norm{\vrots \vJ }^2
      \\
      - \frac{\conj{b_2}|a_2|^2}{\Delta_2}\norm{\LR \vJ}^2  -  \frac{\conj{b_2}^2}{\Delta_2}\norm{\vrots\vE}^2 +  \frac{|b_2|}{\Delta_2} \norm{\LR \vE }^2
      \\
      - \frac{\conj{b_1}|a_1|^2}{\Delta_1}\norm{\LD \vJ}^2  - \frac{\conj{b_1}^2}{\Delta_1}\norm{\vdivs\vE}^2 + \frac{|b_1|^2}{\Delta_1} \norm{\LD \vE }^2
    \end{multline}

    On impose alors à la partie réelle de chaque terme d'être positive, et on obtient
    \begin{equation*}
      \Re\left(a_0\right)\ge 0
    \end{equation*}
    \begin{minipage}{0.5\textwidth}
      \begin{align*}
        \Re\left(a_1 - \frac{\conj{b_1a_0}a_1}{\Delta_1}\right) \le 0 \\
        \Re\left(a_2 - \frac{\conj{b_2a_0}a_2}{\Delta_2}\right) \le 0 \\
        \Re\left(\frac{|a_1|^2\conj{b_1}}{\Delta_1}\right) \le 0 \\
        \Re\left(\frac{|a_2|^2\conj{b_2}}{\Delta_2}\right) \le 0
      \end{align*}
    \end{minipage}
    \begin{minipage}{0.5\textwidth}
      \begin{align*}
        \Re\left(\frac{\conj{b_1}^2}{\Delta_1}\right) \le 0 \\
        \Re\left(\frac{\conj{b_2}^2}{\Delta_2}\right) \le 0 \\
        \Re\left(\frac{|b_1|^2\conj{b_1}}{\Delta_1}\right) \ge 0 \\
        \Re\left(\frac{|b_2|^2\conj{b_2}}{\Delta_2}\right) \ge 0
      \end{align*}
    \end{minipage}

    On remarque alors que certaines CSU peuvent se combiner et imposent que les parties réelles de \(b_1\Delta_1\) et \(b_2\Delta_2\) soient nulles.
  \end{proof}

  Soit \(S = \left\lbrace (a_0,a_1,a_2,b_1,b_2) \in \CC^5 ; a_1 = a_2 ; b_1 = b_2 = 0 \right\rbrace \). On remarque que
  \begin{align}
    \CSU[2]{CI3} & \subset \CSU{CI4}\times\CC^2
    \\
    \CSU[2]{CI3}\cap S & \subsetneq (\CSU{CI4}\times\CC^2)\cap S 
  \end{align}

    On définit 
    \begin{equation}
      \label{eq:fonction:z-ci3}
      \fonction{z}{\CC\times\CC^*\times\CC^*\times \CC \times \CC}{\CC}%
        {(a_0,a_1,a_2,b_1,b_2)}{1 - \frac{b_1a_0}{a_1} - \frac{b_2a_0}{a_2}}
    \end{equation}
    Par abus de notation, on omet les variables \( (a_0,a_1,a_2,b_1,b_2)\)
    \begin{equation}
       z(a_0,a_1,a_2,b_1,b_2) \equiv z
    \end{equation}
  \begin{defn}
    \label{def:csu:ci3-3}

    On définit le sous-espace ouvert de \(\CC^5\)
    \begin{equation*}
      \CSU[3]{CI3} = \left\lbrace 
      \begin{matrix}
        (a_0,a_1,a_2,b_1,b_2) \in \CC^5
        \\
        a_1 \not = 0
        \\
        a_2 \not = 0
        \\
        \Re\left(\conj{a_0}z\right) \ge 0
        \\
        \Re\left(\conj{a_1}z\right) \le 0
        \\
        \Re\left(\conj{a_2}z\right) \le 0
        \\
        \Re\left(\frac{b_1}{a_1}\right) \ge 0
        \\
        \Re\left(\frac{b_2}{a_2}\right) \ge 0
        \\
        \Re\left(a_0\right) \ge 0
        \\
        \Re\left(a_1\right) \le 0
        \\
        \Re\left(a_2\right) \le 0
        \\
        \Re\left(\frac{b_1\conj{a_2}}{a_1\conj{a_0}}\right) \le 0
        \\
        \Re\left(\frac{b_2\conj{a_1}}{a_2\conj{a_0}}\right) \le 0
      \end{matrix}
      \right\rbrace
    \end{equation*}
  \end{defn}

  
  \begin{prop}[Une troisième CSU pour la CI3]
    \label{prop:csu:ci3-3}
    Il suffit que
    \begin{equation*}
      (a_0,a_1,a_2,b_1,b_2) \in \CSU[3]{CI3}
    \end{equation*}
    pour que \(\Re(X)\ge 0\), ce qui entraîne l'unicité.
  \end{prop}

  \begin{proof}
    Par définition de la CIOE, on a

    Supposons l'opérateur \(a_0 + a_1 \LD - a_2 \LR\) inversible.

    \begin{align*}
      X &= \int_\Gamma \left(a_0 + a_1 \LD - a_2 \LR \right)^{-1}\left( 1 + b_1 \LD - b_2 \LR \right) \vE_t\cdot \conj{\vE_t}
    \end{align*}

    On développe chaque terme

    \begin{multline*}
      X = \int_\Gamma \left(a_0 + a_1 \LD - a_2 \LR \right)^{-1}
      \\
      + b_1 \left(a_0 + a_1 \LD - a_2 \LR \right)^{-1}\LD
      \\
      \left.
      - b_2 \left(a_0 + a_1 \LD - a_2 \LR \right)^{-1}\LR \right) \vE_t\cdot \conj{\vE_t}
    \end{multline*}

    On suppose \(a_1\not=0\) et \(a_2\not=0\) alors
    \begin{align*}
      \LD & = \frac{a_0 + a_1 \LD - a_0}{a_1}
      \\
      \LR & = -\frac{a_0 - a_2 \LR - a_0}{a_2}
    \end{align*}


    On déduit de ce qui précède que

    \begin{multline*}
      X = \int_\Gamma z\left(a_0 + a_1 \LD - a_2 \LR \right)^{-1}
      \\
      + \frac{b_1}{a_1} \left(a_0 + a_1 \LD - a_2 \LR \right)^{-1}\left(a_0+a_1\LD\right)
      \\
      - \frac{b_2}{a_2} \left(a_0 + a_1 \LD - a_2 \LR \right)^{-1}\left(a_0-a_2\LR\right) \vE_t\cdot \conj{\vE_t}
    \end{multline*}

    On définit

    \newcommand{\vV}{\vect{V}}
    \newcommand{\vW}{\vect{W}}

    \begin{align*}
      \vV & = \left(a_0  + a_1 \LD - a_2\LR \right)^{-1} \vE_t
      \\
      \vW_1 & = \left( - a_2 \left( a_0 + a_1\LD\right)^{-1}\LR\right)^{-1} \vE_t
      \\
      \vW_2 & = \left( + a_1 \left( a_0 - a_2\LR\right)^{-1}\LD\right)^{-1} \vE_t
    \end{align*}

    On déduit

    \begin{multline*}
      X = \int_\Gamma z \vV \cdot \left(\conj{a_0}  + \conj{a_1} \LD - \conj{a_2}\LR\right)\conj{\vV}
      \\
      + \frac{b_1}{a_1} \left( - \conj{a_2} \left( \conj{a_0} + \conj{a_1}\LD\right)^{-1}\LR\right)\conj{\vW_1}\cdot\vW_1
      \\
      + \frac{b_2}{a_2} \left( + \conj{a_1} \left( \conj{a_0} - \conj{a_2}\LR\right)^{-1}\LD\right)\conj{\vW_2}\cdot\vW_2
    \end{multline*}

    Notons

    \newcommand{\vR}{\vect{R}}

    \begin{align*}
      \vR_1 & = \left(\conj{a_0}  + \conj{a_1} \LD \right)^{-1}\LR \conj{\vW_1}
      \\
      \vR_2 & = \left(\conj{a_0}  - \conj{a_2} \LR \right)^{-1}\LD \conj{\vW_2}
    \end{align*}

    Donc 
    \begin{equation*}
      X = \int_\Gamma z \vV \cdot \left(\conj{a_0}  + \conj{a_1} \LD - \conj{a_2}\LR\right)\conj{\vV} - \frac{b_1}{a_1}\conj{a_2}\vR_1\cdot\vW_1 + \frac{b_2}{a_2}\conj{a_1}\vR_2\cdot\vW_2
    \end{equation*}

    On commute les opérateurs
    \begin{align*}
      \LD\left(\conj{a_0}  + \conj{a_1} \LD \right)&=\left(\conj{a_0}  + \conj{a_1} \LD \right)\LD
      \\
      \LR\left(\conj{a_0}  - \conj{a_2} \LR \right)&=\left(\conj{a_0}  - \conj{a_2} \LR \right)\LR
    \end{align*}

    Or on a démontré que \(\LD\LR=\LR\LD=0\), et ainsi

    \begin{equation*}
      \begin{aligned}
        \LD\LR\conj{\vW_1} &= \LD\left(\conj{a_0}  + \conj{a_1} \LD \right)\vR_1
        \\
        0 & =\left(\conj{a_0}  + \conj{a_1} \LD \right)\LD\vR_1
      \end{aligned}
    \end{equation*}
    \begin{equation*}
      \begin{aligned}
        \LR\LD\conj{\vW_2} &= \LR\left(\conj{a_0}  - \conj{a_2} \LR \right)\vR_2
        \\
        0 & =\left(\conj{a_0}  - \conj{a_2} \LR \right)\LR\vR_2
      \end{aligned}
    \end{equation*}

    Pour conclure, il faut utiliser le résultat suivant
    \begin{prop}[Injectivité]
      On suppose que
      \begin{align*}
        \Re(a_0)\ge0 && \Re(a_1) \le 0 && \Re(a_2) \le 0
      \end{align*}
      alors \(a_0 + a_1 \LD\)  et \(a_0 - a_2 \LR\) sont injectif
    \end{prop}

    \begin{proof}
      Par définition, \(a_0 + a_1 \LD\) est injectif si pour \(\vect{U}\) vecteur régulier tangent  à  \(\Gamma\)
      \begin{align*}
        \int_\Gamma \left(a_0 + a_1 \LD\right)\vect{U}\cdot\conj{\vect{U}} &= 0 \Rightarrow \vect{U} = 0
        \intertext{Or par définition de l'opérateur \(\LD\)}
        \int_\Gamma \left(a_0 + a_1 \LD\right)\vect{U}\cdot\conj{\vect{U}} &=a_0\norm{\vect{U}}^2 - a_1\norm{\tdivs{\vect{U}}}^2
      \end{align*}
      
      Prenons la partie réelle de cette égalité et injectons les hypothèses sur les coefficients. Alors le membre de droite ne contient que des termes positifs, donc tous ces termes sont nuls. Pour que ce soit vrai pour tous \(a_0,a_1,a_2\), c'est donc que \(\norm{\vect{U}} = 0\) et \(\norm{\tdivs\vect{U}} = 0\) donc \(\vect{U} = 0\).

      Le même raisonnement est valable pour \(a_0 - a_2 \LR\).
    \end{proof}

    On suppose donc \(\Re(a_0) \ge 0 \),\(\Re(a_1) \le 0\) et \(\Re(a_2)\le0\)), donc \(\left(\conj{a_0}  + \conj{a_1} \LD \right)\) et \(\left(\conj{a_0}  - \conj{a_2} \LR \right)\) sont injectifs et donc on déduit que

    \begin{align*}
      \left(\conj{a_0}  + \conj{a_1} \LD \right)\LD\vR_1 = 0 &\Rightarrow \LD\vR_1 = 0
      \\
      \left(\conj{a_0}  - \conj{a_2} \LR \right)\LR\vR_2 = 0 &\Rightarrow \LR\vR_2 = 0
    \end{align*}

    Or par définition \(\LR\conj{\vW_1} = \left(\conj{a_0}  + \conj{a_1} \LD \right)\vR_1\) et \(\LD\conj{\vW_2} = \left(\conj{a_0}  - \conj{a_2} \LR \right)\vR_2\) donc
    \begin{align*}
      \LR\conj{\vW_1} = \conj{a_0}\vR_1 && \LD\conj{\vW_2} = \conj{a_0}\vR_2
      \intertext{donc}
      \vR_1 = \frac{1}{\conj{a_0}}\LR\conj{\vW_1} && \vR_2 = \frac{1}{\conj{a_0}}\LD\conj{\vW_2}
    \end{align*}

    On réinjecte ce résultat dans la définition de \(X\)

    \begin{multline*}
      X = \int_\Gamma z \vV \cdot \left(\conj{a_0}  + \conj{a_1} \LD - \conj{a_2}\LR\right)\conj{\vV}
      \\
      + \frac{b_1}{a_1} \norm{\vW_1}^2 - \frac{b_1\conj{a_2}}{a_1\conj{a_0}}\int_\Gamma \LR\conj{\vW_1}\cdot\vW_1
      \\
      + \frac{b_2}{a_2} \norm{\vW_2}^2 + \frac{b_2\conj{a_1}}{a_2\conj{a_0}}\int_\Gamma \LD\conj{\vW_2}\cdot\vW_2
    \end{multline*}

    Et il suffit d'imposer la positivité de la partie réelle de chaque terme pour avoir \(\Re(X)\ge 0\)
  \end{proof}

  Soit \(S = \left\lbrace (a_0,a_1,a_2,b_1,b_2) \in \CC^5 ; a_1 = a_2 ; b_1 = b_2 = 0 \right\rbrace \). On remarque que
  \begin{align}
    \CSU[3]{CI3} & \subset \CSU{CI4}\times\CC^2
    \\ 
    \CSU[3]{CI3}\cap S & = (\CSU{CI4}\times\CC^2)\cap S 
  \end{align}

