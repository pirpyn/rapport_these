\section[Des CSU pour la CIOE de Marceaux et Stupfel 2000]{Des conditions suffisantes pour la CIOE de \cite{marceaux_high-order_2000}}

  Soit \(V=(\mathcal{C}^\infty(\Gamma))^3\) l'ensemble des champs tangents à \(\Gamma\) exprimé dans le repère cartésien.
  Exprimé dans le repère local à la surface, les éléments de cette ensemble n'ont pas de composantes normale.

  \begin{defn}
    On définit les opérateurs \gls{ope-LD} et \gls{ope-LR} tels que %sont introduits par \cite[eq.~4]{stupfel_implementation_2015} et s'expriment pour tous vecteurs complexes régulier tangents à \(\Gamma\) 

    % \begin{REM}
    %   Vivent-ils sur les vecteurs tangents ?
    % \end{REM}
    % \begin{REP}
    %   En commentaire dans le \TeX : [les opérateurs] introduits par \cite[eq.~4]{stupfel_implementation_2015} et s'expriment pour tous vecteurs complexes régulier tangents à \(\Gamma\).
    % \end{REP}

    \begin{equation*}
      \begin{aligned}
        &\fonction{\LD}{V}{V}%
          {\vu}{\vgrads{\vdivs \vu},}
        \\
        &\fonction{\LR}{V}{V}%
          {\vu}{\vrots{\vrots \vu}.}
      \end{aligned}
    \end{equation*}
  \end{defn}

  \begin{prop}
    L'opérateur \(\LD\) est hermitien symétrique négatif et \(\LR\) est hermitien symétrique positif.
  \end{prop}

  % \begin{REM}
  %   Si j'ai bien compris, \(\vu\) peut avoir des composantes tangentes mais pas le résultat ?
  % \end{REM}
  % \begin{REP}
  %   Je ne comprends pas cette remarque.
  % \end{REP}

  \begin{proof}
    Pour tout \(\vu \in V\),
    \begin{align*}
      \int_\Gamma \LD(\vu)\cdot \conj{\vu} &= -\norm{\vdivs \vu}^2,
      \\
      \int_\Gamma \LR(\vu)\cdot \conj{\vu} &=\norm{ \vrots \vu}^2.
    \end{align*}
  \end{proof}

  % \begin{REM}
  %   L'ensemble des champs tangents \(C^\infty\) ?
  % \end{REM}
  % \begin{REP}
  %   Oui ?
  % \end{REP}
  % \begin{REM}
  %   Est-ce que \(\vu\) est un vecteur tangent au bord ?
  %   Si oui, c'est un vecteur de \(C^\infty(\Gamma)^2\).
  %   Dis-le précisément s'il y a une composante normale ( je sais qu'on en a parlé à Montréal )
  % \end{REM}
  % \begin{REP}
  %   Oui.
  %   Quand je met un vecteur à 3 composantes, c'est que la dernière est nulle en pratique.
  % \end{REP}
  \begin{prop}
    \label{prop:unicite:injectif:operateur:LD}
    Soit \(\vu \in V\).

    Soit \(\mathcal{P}_D\) l'opérateur tel que \(\mathcal{P}_D\vu = (a_0\oI + a_1 \LD)\vu\).

    Si \(\Re(a_0)\ge 0\) et \(\Re(a_1)\le 0\) alors l'opérateur \(\mathcal{P}_D\) est injectif sur \(V\).
  \end{prop}
  \begin{prop}
    \label{prop:unicite:injectif:operateur:LR}
    Soit \(\vu \in V\).

    Soit \(\mathcal{P}_R\) l'opérateur tel que \(\mathcal{P}_R\vu = (a_0\oI - a_2 \LR)\vu\).

    Si \(\Re(a_0)\ge 0\) et \(\Re(a_2)\le 0\), alors l'opérateur \(\mathcal{P}_R\) est injectif sur \(V\).
  \end{prop}
  \begin{prop}
    \label{prop:unicite:injectif:operateur:LD-LR}
    Soit \(\vu \in V\).
    
    Soit \(\mathcal{P}\) l'opérateur tel que \(\mathcal{P}\vu = (a_0\oI + a_1 \LD - a_2\LR)\vu\).

    Si \(\Re(a_0)\ge 0\), \(\Re(a_1)\le 0\) et \(\Re(a_2)\le 0\), alors l'opérateur \(\mathcal{P}\) est injectif sur \(V\).
  \end{prop}
  \begin{proof}
   Identique à la démonstration de la propriété \ref{prop:unicite:injectif:operateur:L}.
  \end{proof}

  \begin{prop}
    Soit \(\OO\) un domaine borné de \(\RR^3\), de surface \(\Gamma\) fermée et régulière, de normale unitaire sortante \(\vect n\) et \(\vect{u} \in (\mathcal{C}^\infty(\Gamma))^3\), alors
    \begin{equation*}
        \LR(\LD(\vu)) = \LD(\LR(\vu)) = 0.
    \end{equation*}
  \end{prop}

  \begin{proof}
    Soit (\(x_1,x_2\)) un système de coordonnées locales sur \(\Gamma\).

    Soit un vecteur tangent défini en tout point de la surface. On définit une base locale \(\vect{u_1},\vect{u_2},\vect{n}\) où \((\vect{u_1},\vect{u_2})\) sont tangents à \(\Gamma\) et \(\vn\) est le vecteur normal unitaire sortant à \(\Gamma\).
    \\
    % \[
    %   \vect{u} = 
    %   \begin{bmatrix}
    %     U_1(x_1,x_2)
    %     \\
    %     U_2(x_1,x_2)
    %     \\
    %     0
    %   \end{bmatrix}.
    % \]

    % D'après \cite[p.~1028, A3.22]{bladel_electromagnetic_2007}, le rotationnel surfacique d'un vecteur tangent est un vecteur normal à la surface
    % \[
    %   \vrots{\vect{u}} =\vn \left(\vn\cdot\trot{\vect{u}}\right)
    % \]

    Montrons que \(\LR\LD = 0\).

    D’après \cite[propriété A3.42, p.~1029]{bladel_electromagnetic_2007}, 
    % on sait que l'application de \(\vrots \vgrads\) sur une fonction scalaire renvoi un vecteur tangent à la surface. Or par définition, le rotationnel surfacique d'un
    soit \(f(x_1,x_2)\) une fonction régulière de \(\Gamma\) dans \(\CC\), alors \(\vn \cdot \vrots(\vgrads f(x_1,x_2)) = 0\).

    \begin{align*}
      \LR(\LD \vu) &= \vrots \left(\left(\vn \cdot \vrots \left( \vgrads \left(\vdivs \vu\right)\right)\right)\vn \right),
      \\
      &= \vrots \left(0\vn \right), 
      \\
      &= 0.
    \end{align*}

    Montrons que \(\LD\LR = 0\).

    D’après \cite[p.~1029, A3.43]{bladel_electromagnetic_2007}, \(\vdivs \vrots (f\vn) = 0\).
    \begin{align*}
      \LD(\LR \vu) &= \vgrads \vdivs \vrots (\vn (\vn \cdot \vrots \vu)),
      \\
      &= \vgrads 0,
      \\
      &= 0.
    \end{align*}
  \end{proof}

\subsection{CSU pour la CI4}
  Soit la CIOE que l'on nomme \hyperlink{ci4}{CI4} :
  \begin{equation}
    \label{eq:unicite:ci4}
    \vE_t = (a_0 + a_1 \LD - a_2 \LR ) \vJ.
  \end{equation}

  \begin{defn}
    \label{def:csu:ci4}

    On définit le sous-espace fermé de \(\CC^3\)
    \begin{equation*}
      \CSU{CI4} = \left\lbrace 
      \begin{aligned}
      &(a_0,a_1,a_2) \in \CC^3,
      \\
      & \begin{aligned}
        \Re(a_0) &\ge 0,
        \\
        \Re(a_1) &\le 0,
        \\
        \Re(a_2) &\le 0,
        \end{aligned}
      \end{aligned}
      \right\rbrace.
    \end{equation*}
  \end{defn}

 \begin{prop}[Une CSU pour la CI4]
    \label{prop:csu:ci4}
    On a 
    \begin{equation*}
      (a_0,a_1,a_2) \in \CSU{CI4} \Rightarrow \Re(X)\ge 0,
    \end{equation*}
    ce qui entraîne l'unicité de la solution du problème \{\eqref{eq:unicite:probleme_sans_ci},\eqref{eq:unicite:ci4}\}.
  \end{prop}
  % \begin{REM}
  %   (\eqref{eq:unicite:probleme_sans_ci}, CI4) ou (\eqref{eq:unicite:probleme_sans_ci},\eqref{eq:unicite:ci4}).
  % \end{REM}
  \begin{proof}
    Par définition de \(X\) \eqref{eq:unicite:x}, on a
    \begin{align*}
      X &= \conj{a_0}\norm{\vJ}^2 + \conj{a_1}\norm{\vdivs{\vJ}}^2 - \conj{a_2}\norm{ \vrots{\vJ}}^2,
      \intertext{donc}
      \Re(X) &= \Re(\conj{a_0})\norm{\vJ}^2 + \Re(\conj{a_1})\norm{\vdivs{\vJ}}^2 - \Re(\conj{a_2})\norm{ \vrots{\vJ}}^2.
    \end{align*}
  \end{proof}

  Soit \(S = \left\lbrace (a_0,a_1,a_2) \in \CC^3 ; a_1 = a_2 \right\rbrace \). On remarque que
  \begin{align}
    \CSU{CI4}\cap S &= (\CSU{CI01}\times\CC)\cap S. 
  \end{align}
  Dans le cas \(a_1=a_2\), (\eqref{eq:unicite:probleme_sans_ci},\eqref{eq:unicite:ci01}) et (\eqref{eq:unicite:probleme_sans_ci},\eqref{eq:unicite:ci4}) sont le même problème, et les CSU correspondantes sont les mêmes.
  % \begin{REM}
  %   Mal exprimé.
  %   Dans le cas \(a_1=a_2\), (\eqref{eq:unicite:probleme_sans_ci},\eqref{eq:unicite:ci01}) et (\eqref{eq:unicite:probleme_sans_ci},\eqref{eq:unicite:ci4}) sont le même problème, et les CSU correspondantes sont les mêmes.
  % \end{REM}
  % \begin{REP}
  %   Fait
  % \end{REP}
\subsection{CSU pour la CI3}

  Soit la CIOE énoncée dans \cite{marceaux_high-order_2000} que l'on nomme \hyperlink{ci3}{CI3} :
  \begin{equation}
    \label{eq:unicite:ci3:ci3}
    ( \oI + b_1 \LD - b_2 \LR)\vE_t = (a_0\oI + a_1 \LD - a_2 \LR ) \vJ.
  \end{equation}

  Par abus de notation, on omet les variables de la fonction \(\Delta\) \eqref{eq:unicite:delta} :
  \begin{align*}
     \Delta(a_0,a_1,b_1) &\equiv \Delta_1 = a_1 - a_0\conj{b_1},
     \\
     \Delta(a_0,a_2,b_2) &\equiv \Delta_2 = a_2 - a_0\conj{b_2}.
  \end{align*}

  \begin{defn}
    \label{def:csu:ci3-0}

    On définit le sous-espace de \(\CC^5\)

    \begin{equation*}
      \CSU[0]{CI3} = \left\lbrace 
      \begin{aligned}
      &(a_0,a_1,a_2,b_1,b_2) \in \CC^5,
      \\
      &\begin{aligned}
        &\Delta_1 &\not = 0,
        \\
        &\Delta_2 &\not = 0,
        \\
        &\Re\left(a_0\conj{a_1}\Delta_1\right) &\ge 0,
        \\
        &\Re\left(\frac{\conj{b_1}}{\Delta_1}\right) &\le 0,
        \\
        &\Re\left(\conj{a_0}a_2\left(\frac{\conj{b_1}}{\Delta_1}-\frac{\conj{b_2}}{\Delta_2}\right) + \frac{\conj{a_2}a_1}{\Delta_1} \right)&\le 0,
        \\
        &\Re\left(2\Re(b_2)\frac{\conj{b_1}}{\Delta_1}-\frac{\conj{b_2}^2}{\Delta_2}\right) &\ge 0,
        \\
        &\Re\left(a_0\conj{a_2}\Delta_2\right) &\ge 0,
        \\
        &\Re\left(\frac{\conj{b_2}}{\Delta_2}\right) &\le 0,
        \\
        &\Re\left(\conj{a_0}a_1\left(\frac{\conj{b_1}}{\Delta_1}-\frac{\conj{b_2}}{\Delta_2}\right) + \frac{\conj{a_1}a_2}{\Delta_2} \right)&\le 0,
        \\
        &\Re\left(2\Re(b_1)\frac{\conj{b_2}}{\Delta_2}-\frac{\conj{b_1}^2}{\Delta_1}\right) &\ge 0,
        \\
        &\Re\left(\Delta_1\right) &= 0,
        \\
        &\Re\left(\Delta_2\right) &= 0,
        \\
        &\Re\left(\frac{\conj{b_2}}{\Delta_2}-\frac{\conj{b_1}}{\Delta_1}\right) &= 0,
        \end{aligned}
      \end{aligned}
      \right\rbrace.
    \end{equation*}
  \end{defn}
  
 \begin{prop}[Une première CSU pour la CI3]
    \label{prop:csu:ci3-0}
    On a 
    \begin{equation*}
      (a_0,a_1,a_2,b_1,b_2) \in \CSU[0]{CI3} \Rightarrow \Re(X)\ge 0,
    \end{equation*}
    ce qui entraîne l'unicité de la solution du problème \{\eqref{eq:unicite:probleme_sans_ci},\eqref{eq:unicite:ci3:ci3}\}.
  \end{prop}
  % \begin{REM}
  %   (\eqref{eq:unicite:probleme_sans_ci}, CI3) ou (\eqref{eq:unicite:probleme_sans_ci},\eqref{eq:unicite:ci3:ci3}).
  % \end{REM}
  % \begin{REP}
  %   Fait
  % \end{REP}
  \begin{proof}
    Soit \(\LL_3\) l'opérateur
    \begin{equation}
      \label{eq:unicite:ci3:L3}
      \fonction{\LL_3}{(\mathcal{C}^\infty(\Gamma))^3 \times (\mathcal{C}^\infty(\Gamma))^3}{(\mathcal{C}^\infty(\Gamma))^3}
      {(\vE_t,\vJ)}{( \oI + b_1 \LD - b_2 \LR) \vE_t - (a_0\oI + a_1 \LD - a_2 \LR ) \vJ.}
    \end{equation}

    De la condition limite sur \(\Gamma\), on déduit par produit scalaire \(L^2\) avec un certain nombre de vecteurs bien choisis, des égalités qui sont nécessairement vérifiées.
    On fait cependant remarquer que ces égalités ne  permettent  pas de revenir  à la CL.

    AInsi on utilise l'identité \(\int_\Gamma \conj{\LL_3(\vE_t,\vJ)}\cdot \vJ \dd{\Gamma}(\vx) = 0\) qui découle de la CI3, et dont on déduit
    % \begin{REM}
    %   Là où ça devient plus compliqué, tu pourrais dire:
    %   De la CL, on déduit par produit scalaire \(L^2\) avec un certain nombre de vecteurs bien choisis, des égalités qui sont nécessairement vérifiées. On fait cependant remarquer que ces égalités ne  permettent  pas de revenir  à la CL. 
    % \end{REM}
    \begin{multline}
      \label{eq:unicite:ci3:csu3-1}
      \int_\Gamma \vJ \cdot \conj{\vE_t}\dd{\Gamma}(\vx)   + \conj{b_1} \int_\Gamma \vJ\cdot \LD\conj{\vE_t}\dd{\Gamma}(\vx)  - \conj{b_2} \int_\Gamma \vJ \LR\conj{\vE_t}\dd{\Gamma}(\vx)  \\
      = \conj{a_0} \norm{\vJ}^2 - \conj{a_1} \norm{\vdivs \vJ}^2  - \conj{a_2} \norm{\vrots \vJ}^2. 
    \end{multline}
    On utilise de même \(\int_\Gamma {\LL_3(\vE_t,\vJ)}\cdot \conj{\vE_t} = 0\),
    \begin{multline}
      \label{eq:unicite:ci3:csu3-2}
      \norm{\vE_t}^2   - b_1 \norm{ \vdivs \vE }^2  - b_2 \norm{\vrots \vE_t}^2\dd{\Gamma}(\vx)  \\
      = a_0 \int_\Gamma \vJ\cdot \conj{\vE_t}\dd{\Gamma}(\vx) + a_1 \int_\Gamma \conj{\vE_t} \LD \vJ\dd{\Gamma}(\vx)  - a_2 \int_\Gamma \conj{\vE_t} \cdot \LR \vJ \dd{\Gamma}(\vx).
    \end{multline}
    On utilise \(\int_\Gamma \conj{\LL_3(\vE_t,\vJ)}\cdot\LR\vJ\dd{\Gamma}(\vx) = 0\),
    \begin{equation}
      \label{eq:unicite:ci3:csu3-3}
      \int_\Gamma \vJ \cdot \LR \conj{\vE_t}\dd{\Gamma}(\vx)   - \conj{b_2} \int_\Gamma \LR \vJ \cdot \LR \conj{\vE_t}\dd{\Gamma}(\vx)
      =  \conj{a_0} \norm{ \vrots \vJ}^2ds - \conj{a_2} \norm{ \LR \vJ}^2. 
    \end{equation}
    On utilise \(\int_\Gamma {\LL_3(\vE_t,\vJ)}\cdot\LR\conj{\vE_t}\dd{\Gamma}(\vx)=0\),
    \begin{equation}
      \label{eq:unicite:ci3:csu3-4}
      \norm{ \vrots \vE_t }^2   - {b_2} \norm{ \LR \vE_t}^2 
      = a_0 \int_\Gamma \conj{\vE_t} \cdot \LR \vJ \dd{\Gamma}(\vx) - a_2 \int_\Gamma \LR \conj{\vE_t} \cdot \LR \vJ \dd{\Gamma}(\vx).
    \end{equation}
    On utilise \(\int_\Gamma \conj{\LL_3(\vE_t,\vJ)}\cdot\LD\vJ\dd{\Gamma}(\vx)=0\),
    \begin{equation}
      \label{eq:unicite:ci3:csu3-5}
      \int_\Gamma \vJ \cdot \LD \conj{\vE_t}\dd{\Gamma}(\vx)   + \conj{b_1} \int_\Gamma \LD \vJ \cdot \LD \conj{\vE_t}\dd{\Gamma}(\vx)
      = - \conj{a_0} \norm{\vdivs \vJ}^2 + \conj{a_1} \norm{ \LD \vJ}^2. 
    \end{equation}
    On utilise \(\int_\Gamma {\LL_3(\vE_t,\vJ)}\cdot\LD\conj{\vE_t}\dd{\Gamma}(\vx)=0\),
    \begin{equation}
      \label{eq:unicite:ci3:csu3-6}
      -\norm{ \vdivs \vE_t }^2   + {b_1} \norm{ \LD \vE_t}^2
      = a_0 \int_\Gamma \conj{\vE_t} \cdot \LD \vJ\dd{\Gamma}(\vx)  + a_1 \int_\Gamma \LD \conj{\vE_t} \cdot \LD \vJ \dd{\Gamma}(\vx).
    \end{equation}

    On note
    \begin{align*}
      Y_D &= \int_\Gamma \vJ \cdot \LD \conj{\vE_t}\dd{\Gamma}(\vx),  &
      Y_R &= \int_\Gamma \vJ \cdot \LR \conj{\vE_t} \dd{\Gamma}(\vx),
      \\
      Z_D &= \int_\Gamma \LD \vJ \cdot \LD \conj{\vE_t}\dd{\Gamma}(\vx),  &
      Z_R &= \int_\Gamma \LR \vJ \cdot \LR \conj{\vE_t} \dd{\Gamma}(\vx).
    \end{align*}

    Les 4 égalités \eqref{eq:unicite:ci3:csu3-1} à \eqref{eq:unicite:ci3:csu3-4} sont équivalentes au système

    \begin{align*}
      \mM_R X_R &= F_R,
      \\
      \begin{bmatrix}
        1 & \conj{b_1} & -\conj{b_2} & 0
        \\
        a_0 & a_1 & -a_2 & 0
        \\
        0 & 0 & 1 & -\conj{b_2}
        \\
        0 & 0 & a_0 & -a_2
        \\
      \end{bmatrix}
      \begin{bmatrix}
        X\\
        Y_D\\
        Y_R\\
        Z_R
      \end{bmatrix}
      &=
      \begin{bmatrix}
        \conj{a_0} \norm{\vJ}^2 - \conj{a_1} \norm{\vdivs \vJ}^2 - \conj{a_2} \norm{\vrots \vJ}^2
        \\
        \norm{\vE_t}^2  - b_1 \norm{\vdivs \vE}^2  - b_2 \norm{\vrots \vE_t}^2
        \\
        \conj{a_0} \norm{\vrots \vJ}^2 - \conj{a_2} \norm{\LR \vJ}^2
        \\
        \norm{\vrots \vE_t}^2 - {b_2} \norm{\LR \vE_t}^2
      \end{bmatrix},
    \end{align*}
    
    et les 4 égalités \eqref{eq:unicite:ci3:csu3-1},\eqref{eq:unicite:ci3:csu3-2},\eqref{eq:unicite:ci3:csu3-5},\eqref{eq:unicite:ci3:csu3-6} sont équivalentes au système

    \begin{align*}
      \mM_D  X_D & =  F_D,
      \\
      \begin{bmatrix}
        1 & -\conj{b_2} & \conj{b_1} & 0
        \\
        a_0 & -a_2 & a_1 & 0
        \\
        0 & 0 & 1 & \conj{b_1}
        \\
        0 & 0 & a_0 & a_1
      \end{bmatrix}
      \begin{bmatrix}
        X
        \\
        Y_R
        \\
        Y_D
        \\
        Z_D
      \end{bmatrix}
      & =
      \begin{bmatrix}
        \conj{a_0} \norm{\vJ}^2 - \conj{a_1} \norm{\vdivs \vJ}^2 - \conj{a_2} \norm{\vrots \vJ}^2
        \\
        \norm{\vE_t}^2   - b_1 \norm{\vdivs \vE }^2  - b_2 \norm{\vrots \vE_t}^2
        \\
        -\conj{a_0} \norm{\vdivs \vJ}^2ds + \conj{a_1} \norm{\LR \vJ}^2
        \\
        -\norm{\vdivs \vE_t}^2   + {b_1} \norm{\LR \vE_t}^2 
      \end{bmatrix}.
    \end{align*}

    Le déterminant de ces matrices triangulaires par blocs est \(\Delta_1\Delta_2\). Dans le cas où
    \begin{equation}
      \label{eq:unicite:ci3:csu3-cn-det}
      \Delta_1\Delta_2 \not = 0,
    \end{equation}
    elles sont inversibles et on a alors leurs inverses
    % \begin{REM}
    %   Verification Mathematica ?
    % \end{REM}
    % \begin{REP}
    % \(\mM_R^{-1}\) \href{https://www.wolframalpha.com/input/?i=Simplify%5B%28a1-a0*Conjugate%5Bb1%5D%29*%28a2-a0*Conjugate%5Bb2%5D%29*Inverse+%7B++%7B1%2C+Conjugate%5Bb1%5D%2C+-Conjugate%5Bb2%5D%2C+0%7D%2C++%7Ba0%2C+a1%2C+-a2%2C+0%7D%2C++%7B0%2C+0%2C+1%2C+-Conjugate%5Bb2%5D%7D%2C++%7B0%2C+0%2C+a0%2C+-a2%7D++%7D%5D}{lien web}.

    % \(\mM_D^{-1}\) \href{https://www.wolframalpha.com/input/?i=Simplify%5B%28a1-a0*Conjugate%5Bb1%5D%29*%28a2-a0*Conjugate%5Bb2%5D%29*Inverse+%7B++%7B1%2C-Conjugate%5Bb2%5D%2C+Conjugate%5Bb1%5D%2C+0%7D%2C++%7Ba0%2C+-a2%2C+a1%2C+0%7D%2C++%7B0%2C+0%2C+1%2C+Conjugate%5Bb1%5D%7D%2C++%7B0%2C+0%2C+a0%2C+a1%7D++%7D%5D}{lien web}
    % \end{REP}
    \begin{align*}
      \mM_R^{-1} & =\frac{1}{\Delta_1\Delta_2}
      \begin{bmatrix}
        a_1 \Delta_2 & -\conj{b_1}\Delta_2 & a_2(a_1\conj{b_2}-a_2\conj{b_1}) & -\conj{b_2}(a_1\conj{b_2}-a_2\conj{b_1})
        \\
        -a_0 \Delta_2 & \Delta_2 & a_2\Delta_2 & -\conj{b_2}\Delta_2
        \\
        0 & 0 & a_2\Delta_1 & -\conj{b_2}\Delta_1
        \\
        0 & 0 & a_0\Delta_1 & -\Delta_1
      \end{bmatrix},
      \\
      \mM_D^{-1} & =\frac{1}{\Delta_1\Delta_2}
      \begin{bmatrix}
        a_2 \Delta_1 & -\conj{b_2}\Delta_1 & a_1(a_1\conj{b_2}-a_2\conj{b_1}) & -\conj{b_1}(a_1\conj{b_2}-a_2\conj{b_1})
        \\
        a_0 \Delta_1 & -\Delta_1 & a_1\Delta_1 & -\conj{b_1}\Delta_1
        \\
        0 & 0 & a_1\Delta_2 & -\conj{b_1}\Delta_2
        \\
        0 & 0 & -a_0\Delta_2 & \Delta_2
      \end{bmatrix}.
    \end{align*}
    
    On déduit alors les vecteurs \(X_D\) et \(X_R\) en fonction des normes présentes dans \(F_D\) et \(F_R\), et donc pour chacun de ces vecteurs, on a une expression de \(X\), \(X = (\mM_R^{-1}F_R)_1\) et \(X = (\mM_D^{-1}F_D)_1\).

    Cette dernière quantité est une combinaison linéaire de toutes les normes rencontrées, donc \(X = A_{D/R} \norm{\vJ}^2 + B_{D/R} \norm{\vdivs \vJ}^2 + \ldots \)
    Une condition suffisante pour que la partie réelle de \(X\) soit positive est d'avoir les parties réelles des constantes \(A_{D/R},B_{D/R} \ldots\) positives.
    % Une condition suffisante sur la positivité de la partie réelle de \(X\) est que toutes les parties réelles de ces  constantes \(A_{D/R},B_{D/R} \ldots\) soient positives.

    On pose \(\Theta=\frac{\conj{b_1}}{\Delta_1}-\frac{\conj{b_2}}{\Delta_2}=-\frac{(a_1\conj{b_2}-a_2\conj{b_1})}{\Delta_1\Delta_2}\).

    % \begin{minipage}{0.99\textwidth}
      {En résolvant le système \(\mM_RX_R=F_R\), on obtient},

      \begin{multline*}
        X =
        \frac{a_1}{\Delta_1}
        \left(
          \conj{a_0} \norm{\vJ}^2 - \conj{a_1} \norm{\vdivs \vJ}^2 - \conj{a_2} \norm{\vrots \vJ}^2
        \right)
        \\
        -\frac{\conj{b_1}}{\Delta_1}  
        \left(
          \norm{\vE_t}^2  - b_1 \norm{\vdivs \vE}^2  - b_2 \norm{\vrots \vE_t}^2
        \right)
         \\
        + a_2\frac{(a_1\conj{b_2}-a_2\conj{b_1})}{\Delta_1\Delta_2}
        \left(
          \conj{a_0} \norm{\vrots \vJ}^2 - \conj{a_2} \norm{\LR \vJ}^2
        \right)
        \\
        -\conj{b_2}\frac{(a_1\conj{b_2}-a_2\conj{b_1})}{\Delta_1\Delta_2}
        \left(
          \norm{\vrots \vE_t}^2 - {b_2} \norm{\LR \vE_t}^2
        \right).
      \end{multline*}
      
      Les conditions suivantes garantissent que \(\Re(X)\ge 0\).
      \begin{align}
        \label{eq:unicite:ci3:csu3d-j2}&\Re\left(\frac{\conj{a_0}{a_1}}{\Delta_1}\right) \ge 0,
        \\
        \label{eq:unicite:ci3:csu3d-e2}&\Re\left(\frac{\conj{b_1}}{\Delta_1}\right) \le 0,
        \\
        \label{eq:unicite:ci3:csu3d-jrj}&\Re\left(\conj{a_0}a_2\Theta + \frac{\conj{a_2}a_1}{\Delta_1} \right)\le 0,
        \\
        \label{eq:unicite:ci3:csu3d-ere}&\Re\left(\conj{b_2}\Theta + \frac{\conj{b_1}b_2}{\Delta_1}\right) \ge 0,
        \\
        \label{eq:unicite:ci3:csu3d-jdj}&\Re\left(|a_1|^2\Delta_1\right) \le 0,
        \\
        \label{eq:unicite:ci3:csu3d-ede}&\Re\left(|b_1|^2\Delta_1\right) \ge 0,
        \\
        \label{eq:unicite:ci3:csu3d-dj2}&\Re\left(|a_2|^2\Theta\right)\ge 0,
        \\
        \label{eq:unicite:ci3:csu3d-de2}&\Re\left(|b_2|^2\Theta\right)\le 0.
      \end{align}
      L'on remarque \eqref{eq:unicite:ci3:csu3d-jdj} \& \eqref{eq:unicite:ci3:csu3d-ede} impliquent
      \begin{equation}
        \label{eq:unicite:ci3:csu3d-jde}
        \Re\left(\Delta_1\right) = 0,
      \end{equation}
      et \eqref{eq:unicite:ci3:csu3r-rj2} \& \eqref{eq:unicite:ci3:csu3r-re2} impliquent
      \begin{equation}
        \label{eq:unicite:ci3:csu3-rje2}
        \Re\left(\Theta\right) = 0.
      \end{equation}
    % \end{minipage}

    % \begin{minipage}{0.99\textwidth}
      {Tandis qu'avec le système \(\mM_DX_D=F_D\)}, on a

      \begin{multline*}
        X =
        \frac{a_2}{\Delta_2}
        \left(
        \conj{a_0} \norm{\vJ}^2 - \conj{a_1} \norm{\vdivs \vJ}^2 - \conj{a_2} \norm{\vrots \vJ}^2 
        \right)
        \\
        -\frac{\conj{b_2}}{\Delta_2}
        \left(
        \norm{\vE_t}^2   - b_1 \norm{\vdivs \vE }^2  - b_2 \norm{\vrots \vE_t}^2
        \right)
        \\
        + a_1\frac{(a_1\conj{b_2}-a_2\conj{b_1})}{\Delta_1\Delta_2}
        \left(
        -\conj{a_0} \norm{\vdivs \vJ}^2 + \conj{a_1} \norm{\LR \vJ}^2
        \right)
        \\
        -\conj{b_1}\frac{(a_1\conj{b_2}-a_2\conj{b_1})}{\Delta_1\Delta_2}
        \left(
        -\norm{\vdivs \vE_t}^2   + {b_1} \norm{\LR \vE_t}^2
        \right).
      \end{multline*}
 
      Les conditions suivantes garantissent que \(\Re(X)\ge 0\).

      \begin{align}
        \label{eq:unicite:ci3:csu3r-j2}&\Re\left(\frac{\conj{a_0}{a_2}}{\Delta_2}\right) \ge 0,
        \\
        \label{eq:unicite:ci3:csu3r-e2}&\Re\left(\frac{\conj{b_2}}{\Delta_2}\right) \le 0,
        \\
        \label{eq:unicite:ci3:csu3r-jdj}&\Re\left(\conj{a_0}a_1\Theta - \frac{\conj{a_1}a_2}{\Delta_2} \right)\le 0,
        \\
        \label{eq:unicite:ci3:csu3r-ede}&\Re\left(-\conj{b_1}\Theta + \frac{\conj{b_2}b_1}{\Delta_2} \right) \ge 0,
        \\
        \label{eq:unicite:ci3:csu3r-jrj}&\Re\left(|a_2|^2\Delta_2\right) \le 0,
        \\
        \label{eq:unicite:ci3:csu3r-ere}&\Re\left(|b_2|^2\Delta_2\right) \ge 0,
        \\
        \label{eq:unicite:ci3:csu3r-rj2}&\Re\left(|a_1|^2\Theta\right)\ge 0,
        \\
        \label{eq:unicite:ci3:csu3r-re2}&\Re\left(|b_1|^2\Theta\right)\le 0,
      \end{align}
      L'on remarque \eqref{eq:unicite:ci3:csu3r-jrj} \& \eqref{eq:unicite:ci3:csu3r-ere} impliquent
      \begin{equation}
        \label{eq:unicite:ci3:csu3r-jre}
        \Re\left(\Delta_2\right) = 0,
      \end{equation}
      et \eqref{eq:unicite:ci3:csu3d-dj2} \& \eqref{eq:unicite:ci3:csu3d-de2} impliquent
      \begin{equation}
        \label{eq:unicite:ci3:csu3-dje2}
        \Re\left(\Theta\right) = 0.
      \end{equation}
    % \end{minipage}

    Pour conclure, on remarque que l'on peut utiliser 3 CSU
    \begin{itemize}
      \item les relations \eqref{eq:unicite:ci3:csu3-cn-det}, \eqref{eq:unicite:ci3:csu3d-j2} à \eqref{eq:unicite:ci3:csu3d-ere}, \eqref{eq:unicite:ci3:csu3d-jde}, \eqref{eq:unicite:ci3:csu3-dje2}
      \item les relations \eqref{eq:unicite:ci3:csu3-cn-det}, \eqref{eq:unicite:ci3:csu3r-j2} à \eqref{eq:unicite:ci3:csu3r-ede}, \eqref{eq:unicite:ci3:csu3r-jre}, \eqref{eq:unicite:ci3:csu3-rje2}
      \item l'union de ces deux ensembles qui est la CSU de la proposition.
    \end{itemize}

    \end{proof}
  \begin{defn}
    \label{def:csu:ci3-0}

    On définit le sous-espace de \(\CC^5\)

    \begin{equation*}
      \CSU[1]{CI3} = \left\lbrace 
      \begin{aligned}
      &(a_0,a_1,a_2,b_1,b_2) \in \CC^5,
      \\
      &\begin{aligned}
        &\Delta_1 & \not= 0,
        \\
        &\Delta_2 & \not= 0,
        \\
        &\Re\left(a_0\conj{a_1}\Delta_1\right) &\ge 0,
        \\
        &\Re\left(a_0\conj{a_2}\Delta_2\right) &\ge 0,
        \\
        &\Re(\Delta_1) &= 0,
        \\
        &\Re(\Delta_2) &= 0,
        \\
        &\Im(b_1)\Im(\Delta_2) - \Im(b_2)\Im(\Delta_1) &= 0,
        \\
        &\Re(b_1\Delta_1) &\le 0.
        \end{aligned}
      \end{aligned}
      \right\rbrace.
    \end{equation*}
  \end{defn}
  
 \begin{prop}[CSU de Lafitte : réduction du nombre de relations dans la CSU précédente]
    \label{prop:csu:ci3-1}
    On a 
    \begin{equation*}
      (a_0,a_1,a_2,b_1,b_2) \in \CSU[1]{CI3} \Rightarrow \Re(X)\ge 0,
    \end{equation*}
    ce qui entraîne l'unicité de la solution du problème \{\eqref{eq:unicite:probleme_sans_ci},\eqref{eq:unicite:ci3:ci3}\}.
  \end{prop}
  % \begin{REM}
  %   (\eqref{eq:unicite:probleme_sans_ci}, CI3) ou (\eqref{eq:unicite:probleme_sans_ci},\eqref{eq:unicite:ci3:ci3}).
  % \end{REM}
  % \begin{REP}
  %   Fait
  % \end{REP}
  \begin{proof}
    Utilisant cette dernière CSU, O.~Lafitte a éliminé les redondances et a obtenu une forme plus compacte.
    \begin{align}
    \label{eq:unicite:ci3:csu3rd-D1}\Re(\Delta_1) &= 0 & &\Leftrightarrow & \Re(a_1) - \Re(a_0\conj{b_1}) &= 0,
    \\
    \label{eq:unicite:ci3:csu3rd-D2}\Re(\Delta_2) &= 0 & &\Leftrightarrow & \Re(a_2) - \Re(a_0\conj{b_2}) &= 0,
    \\
    \label{eq:unicite:ci3:csu3rd-T}\Re(\Theta) &= 0 & &\Leftrightarrow & \Im(b_1)\Im(\Delta_2) - \Im(b_2)\Im(\Delta_1) &= 0.
    \intertext{Les conditions \eqref{eq:unicite:ci3:csu3d-e2} et \eqref{eq:unicite:ci3:csu3r-e2} sont alors redondantes et deviennent}
    \label{eq:unicite:ci3:csu3rd-bD} \Im(b_1)\Im(\Delta_1) &\ge 0 & &\Leftrightarrow & \Im(b_2)\Im(\Delta_2) &\ge 0.
    \end{align}
    Utilisant \eqref{eq:unicite:ci3:csu3rd-T}, l'inégalité \eqref{eq:unicite:ci3:csu3d-ere} se simplifie trivialement car
    \begin{align*}
      \conj{b_2}\Theta + \frac{\conj{b_1}b_2}{\Delta_1} & = \conj{b_2}\Theta + \frac{\conj{b_1}b_2}{\Delta_1} + \frac{\conj{b_2}b_2}{\Delta_2} - \frac{\conj{b_2}b_2}{\Delta_2},
      \\
      &=\left(\conj{b_2} + b_2\right)\Theta + \frac{b_2\conj{b_2}}{\Delta_2},
      \\
      &=2\Re(b_2)\Theta + \frac{|b_2|^2}{\Delta_2},
    \end{align*}
    et ce dernier est imaginaire pur d'après \eqref{eq:unicite:ci3:csu3rd-D2} \& \eqref{eq:unicite:ci3:csu3rd-T}.
    De même, pour l'inégalité \eqref{eq:unicite:ci3:csu3r-ede}.
    % \begin{align*}
    %   -\conj{b_1}\Theta + \frac{\conj{b_2}b_1}{\Delta_2} & = -\conj{b_1}\Theta + \frac{\conj{b_2}b_1}{\Delta_2} - \frac{\conj{b_1}b_1}{\Delta_1} + \frac{\conj{b_1}b_1}{\Delta_1},
    %   \\
    %   &=-\left(\conj{b_1} + b_1\right)\Theta + \frac{b_1\conj{b_1}}{\Delta_1},
    %   \\
    %   &=-2\Re(b_1)\Theta + \frac{|b_1|^2}{\Delta_1},
    % \end{align*}
    % qui est aussi imaginaire pur d'après \eqref{eq:unicite:ci3:csu3rd-D2} \& \eqref{eq:unicite:ci3:csu3rd-T}.

    Enfin l'inégalité \eqref{eq:unicite:ci3:csu3d-jrj} se simplifie aussi trivialement si l'on injecte la définition de \(\Delta_1\) et \(\Delta_2\) car
    \begin{align*}
    \conj{a_0}a_2\Theta + \frac{\conj{a_2}a_1}{\Delta_1} & = \conj{a_0}a_2\Theta + \conj{a_2}\frac{\Delta_1 + a_0\conj{b_1}}{\Delta_1},
    \\
    &=\conj{a_0}a_2\Theta + \conj{a_2} + a_0\conj{a_2}\frac{\conj{b_1}}{\Delta_1} - a_0\conj{a_2}\frac{\conj{b_2}}{\Delta_2} + a_0\conj{a_2}\frac{\conj{b_2}}{\Delta_2},
    \\
    &=\left(\conj{a_0}a_2 + a_0\conj{a_2}\right)\Theta + \conj{a_2} + a_0\conj{a_2}\frac{\conj{b_2}}{\Delta_2},
    \\
    &=\left(\conj{a_0}a_2 + a_0\conj{a_2}\right)\Theta + \conj{a_2}\frac{\Delta_2 + a_0\conj{b_2}}{\Delta_2},
    \\
    &=2\Re(\conj{a_0}a_2)\Theta + \frac{|a_2|^2}{\Delta_2},
    \end{align*}
    qui est imaginaire pur.
    De même pour l'inégalité \eqref{eq:unicite:ci3:csu3r-jdj}.

    Ainsi des 13 relations initiales, seules 8 sont conservées
    \begin{align*}
      \eqref{eq:unicite:ci3:csu3-cn-det} && \Delta_1 & \not= 0 && \Leftrightarrow & \Im(\Delta_1) &\not = 0,
      \\
      \eqref{eq:unicite:ci3:csu3-cn-det} && \Delta_2 & \not= 0 && \Leftrightarrow & \Im(\Delta_2) &\not = 0,
      \\
      \eqref{eq:unicite:ci3:csu3d-j2} && \Re\left(a_0\conj{a_1}\Delta_1\right) &\ge 0 & &\Leftrightarrow & \Im(a_0\conj{a_1})\Im(\Delta_1) &\le 0 ,
      \\
      \eqref{eq:unicite:ci3:csu3r-j2} && \Re\left(a_0\conj{a_2}\Delta_2\right) &\ge 0 & &\Leftrightarrow & \Im(a_0\conj{a_2})\Im(\Delta_2) &\le 0,
      \\
      \eqref{eq:unicite:ci3:csu3rd-D1} && \Re(\Delta_1) &= 0 & &\Leftrightarrow & \Re(a_1) - \Re(a_0\conj{b_1}) &= 0,
      \\
      \eqref{eq:unicite:ci3:csu3rd-D2} && \Re(\Delta_2) &= 0 & &\Leftrightarrow & \Re(a_2) - \Re(a_0\conj{b_2}) &= 0,
      \\
      \eqref{eq:unicite:ci3:csu3rd-T} && \Re(\Theta) &= 0 & &\Leftrightarrow & \Im(b_1)\Im(\Delta_2) - \Im(b_2)\Im(\Delta_1) &= 0,
      \\
      \eqref{eq:unicite:ci3:csu3rd-bD} && \Re(b_1\Delta_1) &\le 0 & &\Leftrightarrow & \Im(b_2)\Im(\Delta_2) &\ge 0.
    \end{align*}

\end{proof}


  Soit \(S = \left\lbrace (a_0,a_1,a_2,b_1,b_2) \in \CC^5 ; a_1 = a_2 ; b_1 = b_2 = 0 \right\rbrace \). On remarque que
  \begin{align}
    \CSU[1]{CI3} & \subset \CSU{CI4}\times\CC^2,
    \\
    \CSU[1]{CI3}\cap S & \subsetneq (\CSU{CI4}\times\CC^2)\cap S.
  \end{align}

  \begin{defn}
    \label{def:csu:ci3-2}

    On définit le sous-espace de \(\CC^5\)
    \begin{equation*}
      \CSU[2]{CI3} = \left\lbrace 
      \begin{aligned}
      &(a_0,a_1,a_2,b_1,b_2) \in \CC^5,
      \\
      &\begin{aligned}
        \Delta_1 &\not = 0,
        \\
        \Delta_2 &\not = 0,
        \\
        \Re\left(a_0\right)&\ge 0,
        \\
        \Re\left(a_1 - \frac{\conj{b_1a_0}a_1}{\Delta_1}\right) &\le 0,
        \\
        \Re\left(a_2 - \frac{\conj{b_2a_0}a_2}{\Delta_2}\right) &\le 0,
        \\
        \Re\left(b_1\Delta_1\right) &= 0,
        \\
        \Re\left(b_2\Delta_2\right) &= 0,
        \\
        \Im\left(b_1\Delta_1\right)\Im(b_1)&\ge 0,
        \\
        \Im\left(b_2\Delta_2\right)\Im(b_2)&\ge 0,
        \end{aligned}
      \end{aligned}
      \right\rbrace.
    \end{equation*}
  \end{defn}
  \begin{REM}
    Et là tu peux mettre le passage à la limite que je t'ai envoyé par mail.
  \end{REM}
  \begin{REP}
  A recopier.
  \end{REP}

 \begin{prop}[Une deuxième CSU pour la CI3]
    \label{prop:csu:ci3-2}
    \begin{equation*}
      (a_0,a_1,a_2,b_1,b_2) \in \CSU[2]{CI3} \Rightarrow \Re(X) \ge 0,
    \end{equation*}
    ce qui entraîne l'unicité de la solution du problème de Maxwell extérieur avec CI3.
  \end{prop}

  \begin{proof}
    En se basant sur la démonstration de la \CSU[1]{CI3}, on remarque que l'on peut déterminer les quantités \((Y_R,Z_R)\) (resp. \((Y_D,Z_R)\)) uniquement en utilisant les équations \eqref{eq:unicite:ci3:csu3-3} et \eqref{eq:unicite:ci3:csu3-4} (resp. \eqref{eq:unicite:ci3:csu3-5} et \eqref{eq:unicite:ci3:csu3-6}).

    On déduit donc que si \(\Delta_1 \not = 0\) et \(\Delta_2 \not = 0\), alors

    \begin{align*}
      Y_R &= \frac{1}{\Delta_2}\left(a_2\left(\conj{a_0}\int_\Gamma \vJ\cdot\LR\conj{\vJ} - \conj{a_2}\norm{\LR \vJ}^2\right)  -\conj{b_2}\left(\int_\Gamma \conj{\vE}\cdot\LR{\vE} - b_2 \norm{\LR \vE}^2\right)\right), \\
      Y_D &= \frac{1}{\Delta_1}\left(a_1\left(\conj{a_0}\int_\Gamma \vJ\cdot\LD\conj{\vJ} + \conj{a_1}\norm{\LD \vJ}^2\right)  -\conj{b_1}\left(\int_\Gamma \conj{\vE}\cdot\LD{\vE} + b_1 \norm{\LD \vE}^2\right)\right).
    \end{align*}

    Utilisons \eqref{eq:unicite:ci3:csu3-1}, on obtient
    \begin{equation*}
      X = -\conj{b_1} Y_D + \conj{b_2} Y_R + \conj{a_0} \norm{\vJ}^2 + \conj{a_1} \int_\Gamma \vJ \cdot \LD \conj{\vJ} - \conj{a_2} \int_\Gamma \vJ \cdot \LR \conj{\vJ}.
    \end{equation*}

    On développe les résultats de la démonstration précédente,
    \begin{multline*}
      X = \conj{a_0} \norm{\vJ }^2 - \conj{a_1} \norm{\vdivs \vJ }^2 - \conj{a_2} \norm{\vrots \vJ }^2
      \\
      + \frac{\conj{b_2}}{\Delta_2}\left(a_2\left(\conj{a_0}\norm{\vrots \vJ}^2 - \conj{a_2}\norm{\LR J}^2\right)  -\conj{b_2}\left(\norm{\vrots\vE}^2 - b_2 \norm{\LR \vE }^2\right)\right)
      \\
      - \frac{\conj{b_1}}{\Delta_1}\left(a_1\left(-\conj{a_0}\norm{\vdivs\vJ}^2 + \conj{a_1}\norm{\LD J}^2\right)  -\conj{b_1}\left(-\norm{\vdivs\vE}^2 + b_1 \norm{\LD \vE }^2\right)\right).
    \end{multline*}

    On factorise les termes en \(\norm{\vJ}^2\), \(\norm{\vdivs\vJ}^2\),  \(\norm{\vrots\vJ}^2\)
    \begin{multline*}
      X = \conj{a_0} \norm{\vJ }^2 - \left(\conj{a_1} - \frac{\conj{b_1a_0}a_1}{\Delta_1}\right) \norm{\vdivs \vJ}^2 - \left(\conj{a_2} - \frac{\conj{b_2a_0}a_2}{\Delta_2}\right) \norm{\vrots \vJ }^2
      \\
      + \frac{\conj{b_2}}{\Delta_2}\left( - |a_2|^2\norm{\LR \vJ}^2  - \conj{b_2}\left(\norm{\vrots\vE}^2 - b_2 \norm{\LR \vE }^2\right)\right) 
      \\
      - \frac{\conj{b_1}}{\Delta_1}\left( |a_1|^2\norm{\LD \vJ}^2  - \conj{b_1}\left(-\norm{\vdivs\vE}^2 + b_1 \norm{\LD \vE }^2\right)\right).
    \end{multline*}

    On développe l'expression précédente
    \begin{multline*}
      X = \conj{a_0} \norm{\vJ }^2 - \left(\conj{a_1} - \frac{\conj{b_1a_0}a_1}{\Delta_1}\right) \norm{\vdivs \vJ }^2 - \left(\conj{a_2} - \frac{\conj{b_2a_0}a_2}{\Delta_2}\right) \norm{\vrots \vJ }^2
      \\
      - \frac{\conj{b_2}|a_2|^2}{\Delta_2}\norm{\LR \vJ}^2  -  \frac{\conj{b_2}^2}{\Delta_2}\norm{\vrots\vE}^2 +  \frac{\conj{b_2}|b_2|^2}{\Delta_2} \norm{\LR \vE }^2
      \\
      - \frac{\conj{b_1}|a_1|^2}{\Delta_1}\norm{\LD \vJ}^2  - \frac{\conj{b_1}^2}{\Delta_1}\norm{\vdivs\vE}^2 + \frac{\conj{b_1}|b_1|^2}{\Delta_1} \norm{\LD \vE }^2.
    \end{multline*}

    Comme condition suffisante, on impose à la partie réelle de chaque terme d'être positive, ce qui s'écrit
    \begin{align*}
      \Re\left(a_0\right)\ge 0, && 
      \\
      \Re\left(\conj{a_1} - \frac{\conj{b_1a_0}a_1}{\Delta_1}\right) \le 0, && \Re\left(\frac{\conj{b_1}^2}{\Delta_1}\right) \le 0,
      \\
      \Re\left(\conj{a_2} - \frac{\conj{b_2a_0}a_2}{\Delta_2}\right) \le 0, && \Re\left(\frac{\conj{b_2}^2}{\Delta_2}\right) \le 0,
      \\
      \Re\left(\frac{|a_1|^2\conj{b_1}}{\Delta_1}\right) \le 0, && \Re\left(\frac{|b_1|^2\conj{b_1}}{\Delta_1}\right) \ge 0,
      \\
      \Re\left(\frac{|a_2|^2\conj{b_2}}{\Delta_2}\right) \le 0, && \Re\left(\frac{|b_2|^2\conj{b_2}}{\Delta_2}\right) \ge 0.
    \end{align*}


    On remarque alors que les conditions des deux dernières lignes se combinent et imposent aux parties réelles de \(b_1\Delta_1\) et \(b_2\Delta_2\) d'être nulles.
  \end{proof}

  Soit \(S = \left\lbrace (a_0,a_1,a_2,b_1,b_2) \in \CC^5 ; a_1 = a_2 ; b_1 = b_2 = 0 \right\rbrace \). On remarque que
  \begin{align}
    \CSU[2]{CI3} & \subset \CSU{CI4}\times\CC^2,
    \\
    \CSU[2]{CI3}\cap S & \subsetneq (\CSU{CI4}\times\CC^2)\cap S. 
  \end{align}

    On définit 
    \begin{equation}
      \label{eq:fonction:z-ci3}
      \fonction{z}{\CC\times\CC^*\times\CC^*\times \CC \times \CC}{\CC}%
        {(a_0,a_1,a_2,b_1,b_2)}{1 - \frac{b_1a_0}{a_1} - \frac{b_2a_0}{a_2}.}
    \end{equation}
    Par abus de notation, on omet les variables \( (a_0,a_1,a_2,b_1,b_2)\)
    \begin{equation}
       z(a_0,a_1,a_2,b_1,b_2) \equiv z.
    \end{equation}
  \begin{defn}
    \label{def:csu:ci3-3}

    On définit le sous-espace  de \(\CC^5\)
    \begin{equation*}
      \CSU[3]{CI3} = \left\lbrace
      \begin{aligned}
        &(a_0,a_1,a_2,b_1,b_2) \in \CC^5,
        \\
        &\begin{aligned}
          &a_1 &\not = 0,
          \\
          &a_2 &\not = 0,
          \\
          &\Re\left(\conj{a_0}z\right) &\ge 0,
          \\
          &\Re\left(\conj{a_1}z\right) &\le 0,
          \\
          &\Re\left(\conj{a_2}z\right) &\le 0,
          \\
          &\Re\left(\frac{b_1}{a_1}\right) &\ge 0,
          \\
          &\Re\left(\frac{b_2}{a_2}\right) &\ge 0,
          \\
          &\Re\left(a_0\right) &\ge 0,
          \\
          &\Re\left(a_1\right) &\le 0,
          \\
          &\Re\left(a_2\right) &\le 0,
          \\
          &\Re\left(\frac{b_1\conj{a_2}}{a_1\conj{a_0}}\right) &\le 0,
          \\
          &\Re\left(\frac{b_2\conj{a_1}}{a_2\conj{a_0}}\right) &\le 0,
        \end{aligned}
      \end{aligned}
      \right\rbrace.
    \end{equation*}
  \end{defn}

  
  \begin{prop}[Une troisième CSU pour la CI3]
    \label{prop:csu:ci3-3}
    \begin{equation*}
      (a_0,a_1,a_2,b_1,b_2) \in \CSU[3]{CI3} \Rightarrow \Re(X) \ge 0,
    \end{equation*}
    ce qui entraîne l'unicité du problème \{\eqref{eq:unicite:probleme_sans_ci},\eqref{eq:unicite:ci3:ci3}\}.
  \end{prop}

  \begin{proof}
    Supposons l'opérateur \(a_0\oI + a_1 \LD - a_2 \LR\) inversible.
    \begin{REM}
      Sur \(\Sobolev[1]\) ? Précisez les espaces. Il faut préciser les espaces, de plus il y a un autre problème. Prenons P l'opérateur
      \(Pf=f''+af\). \(Pf=g\) a une infinité de solutions: si on en a une, alors en ajoutant  \(f_0\) telle que \(f_0''+a_f0 =0\), c'est à dire \(f_0=Ae^{\sqrt{-a}t}+Be^{-\sqrt{-a}t}\) alors comment ????????-tu \(f_0\)
    \end{REM}
    \begin{REP}
      On a vu ça ensemble à Montréal: je ne sais pas dans quel espace me placer.
    \end{REP}
   Alors,

    \begin{align*}
      X &= \int_\Gamma \left(a_0\oI + a_1 \LD - a_2 \LR \right)^{-1}\left( \oI + b_1 \LD - b_2 \LR \right) \vE_t\cdot \conj{\vE_t.}
    \end{align*}

    On développe chaque terme

    \begin{multline*}
      X = \int_\Gamma \left(\left(a_0\oI + a_1 \LD - a_2 \LR \right)^{-1}
      \right.
      \\
      + b_1 \left(a_0\oI + a_1 \LD - a_2 \LR \right)^{-1}\LD
      \\
      \left.
      - b_2 \left(a_0\oI + a_1 \LD - a_2 \LR \right)^{-1}\LR \right) \vE_t\cdot \conj{\vE_t.}
    \end{multline*}

    On suppose \(a_1\not=0\) et \(a_2\not=0\), alors
    \begin{align*}
      \LD & = \frac{1}{a_1}\left(a_0\oI + a_1 \LD - a_0\oI\right),
      \\
      \LR & = -\frac{1}{a_2}\left(a_0\oI - a_2 \LR - a_0\oI\right).
    \end{align*}


    On déduit de ce qui précède que

    \begin{multline*}
      X = \int_\Gamma \left(1-\frac{b_1}{a_1} -\frac{b_2}{a_2}\right)\left(a_0\oI + a_1 \LD - a_2 \LR \right)^{-1}
      \\
      + \frac{b_1}{a_1} \left(a_0\oI + a_1 \LD - a_2 \LR \right)^{-1}\left(a_0\oI + a_1\LD\right)
      \\
      + \frac{b_2}{a_2} \left(a_0\oI + a_1 \LD - a_2 \LR \right)^{-1}\left(a_0\oI - a_2\LR\right) \vE_t\cdot \conj{\vE_t.}
    \end{multline*}

    On définit

    \newcommand{\vV}{\vect{V}}
    \newcommand{\vW}{\vect{W}}

    \begin{align*}
      z &= 1-\frac{b_1}{a_1}a_0 -\frac{b_2}{a_2}a_0,
      \\
      \vV & = \left(a_0\oI  + a_1 \LD - a_2\LR \right)^{-1} \vE_t,
      \\
      \vW_1 & = \left( \oI - a_2 \left( a_0\oI + a_1\LD\right)^{-1}\LR\right)^{-1} \vE_t,
      \\
      \vW_2 & = \left( \oI + a_1 \left( a_0\oI - a_2\LR\right)^{-1}\LD\right)^{-1} \vE_t.
    \end{align*}

    On remarque que
    \begin{align*}
      \left( \oI - a_2 \left( a_0\oI + a_1\LD\right)^{-1}\LR\right)\vW_1 &= \vE_t,
      \\
      \left( a_0\oI + a_1 \LD \right)\left( \oI - a_2 \left( a_0\oI + a_1\LD\right)^{-1}\LR\right)\vW_1&= \left( a_0\oI + a_1 \LD \right)\vE_t,
      \\
      \left( a_0\oI + a_1 \LD - a_2 \LR\right)\vW_1 &= \left( a_0\oI + a_1 \LD \right)\vE_t,
    \end{align*}
    \begin{align*}
      \left( \oI + a_1 \left( a_0\oI - a_2\LR\right)^{-1}\LD\right)\vW_2 &= \vE_t,
      \\
      \left( a_0\oI - a_2 \LR \right)\left( \oI + a_1 \left( a_0\oI - a_2\LR\right)^{-1}\LD\right)\vW_2&= \left( a_0\oI - a_2 \LR \right)\vE_t,
      \\
      \left( a_0\oI + a_1 \LR - a_2 \LR\right)\vW_2 &= \left( a_0\oI - a_2 \LR \right)\vE_t.
    \end{align*}

    On déduit

    \begin{multline*}
      X = \int_\Gamma z \vV \cdot \left(\conj{a_0}\oI  + \conj{a_1} \LD - \conj{a_2}\LR\right)\conj{\vV}
      \\
      + \frac{b_1}{a_1} \left( \oI - \conj{a_2} \left( \conj{a_0}\oI + \conj{a_1}\LD\right)^{-1}\LR\right)\conj{\vW_1}\cdot\vW_1
      \\
      + \frac{b_2}{a_2} \left( \oI + \conj{a_1} \left( \conj{a_0}\oI - \conj{a_2}\LR\right)^{-1}\LD\right)\conj{\vW_2}\cdot\vW_2.
    \end{multline*}

    Notons

    \newcommand{\vR}{\vect{R}}

    \begin{align*}
      \vR_1 & = \left(\conj{a_0}\oI  + \conj{a_1} \LD \right)^{-1}\LR \conj{\vW_1},
      \\
      \vR_2 & = \left(\conj{a_0}\oI  - \conj{a_2} \LR \right)^{-1}\LD \conj{\vW_2}.
    \end{align*}

    Donc 
    \begin{equation*}
      X = \int_\Gamma z \vV \cdot \left(\conj{a_0}\oI  + \conj{a_1} \LD - \conj{a_2}\LR\right)\conj{\vV} + \frac{b_1}{a_1} \norm{\vW_1} +\frac{b_2}{a_2} \norm{\vW_2} - \frac{b_1}{a_1}\conj{a_2}\vR_1\cdot\vW_1 + \frac{b_2}{a_2}\conj{a_1}\vR_2\cdot\vW_2.
    \end{equation*}

    Comme
    \begin{align*}
      \LD\left(\conj{a_0}\oI  + \conj{a_1} \LD \right)&=\left(\conj{a_0}\oI  + \conj{a_1} \LD \right)\LD,
      \\
      \LR\left(\conj{a_0}\oI  - \conj{a_2} \LR \right)&=\left(\conj{a_0}\oI  - \conj{a_2} \LR \right)\LR,
    \end{align*}

    et que \(\LD\LR=\LR\LD=0\), on trouve

    \begin{equation*}
      \begin{aligned}
        \LD\LR\conj{\vW_1} &= \LD\left(\conj{a_0}\oI  + \conj{a_1} \LD \right)\vR_1,
        \\
        0 & =\left(\conj{a_0}\oI  + \conj{a_1} \LD \right)\LD\vR_1,
      \end{aligned}
    \end{equation*}
    \begin{equation*}
      \begin{aligned}
        \LR\LD\conj{\vW_2} &= \LR\left(\conj{a_0}\oI  - \conj{a_2} \LR \right)\vR_2,
        \\
        0 & =\left(\conj{a_0}\oI  - \conj{a_2} \LR \right)\LR\vR_2.
      \end{aligned}
    \end{equation*}

    % Pour conclure, il faut utiliser le résultat suivant
    % \begin{prop}[Injectivité]
    %   On suppose que
    %   \begin{align*}
    %     \Re(a_0)\ge0 && \Re(a_1) \le 0 && \Re(a_2) \le 0
    %   \end{align*}
    %   alors \(a_0\oI + a_1 \LD\)  et \(a_0\oI - a_2 \LR\) sont injectif
    % \end{prop}

    % \begin{proof}
    %   Par définition, \(a_0\oI + a_1 \LD\) est injectif si pour \(\vect{U}\) vecteur régulier tangent  à  \(\Gamma\)
    %   \begin{align*}
    %     \int_\Gamma \left(a_0\oI + a_1 \LD\right)\vect{U}\cdot\conj{\vect{U}} &= 0 \Rightarrow \vect{U} = 0
    %     \intertext{Or par définition de l'opérateur \(\LD\)}
    %     \int_\Gamma \left(a_0\oI + a_1 \LD\right)\vect{U}\cdot\conj{\vect{U}} &=a_0\norm{\vect{U}}^2 - a_1\norm{\vdivs{\vect{U}}}^2
    %   \end{align*}
      
    %   Prenons la partie réelle de cette égalité et utilisons les hypothèses sur les coefficients. Alors le membre de droite ne contient que des termes positifs, donc tous ces termes sont nuls, donc \(\norm{\vect{U}} = 0\) et \(\norm{\vdivs\vect{U}} = 0\) donc \(\vect{U} = 0\).

    %   Le même raisonnement est valable pour \(a_0\oI - a_2 \LR\).
    % \end{proof}

    % On utilise les propositions \ref{prop:unicite:injectif:operateur:LD},\ref{prop:unicite:injectif:operateur:LR},\ref{prop:unicite:injectif:operateur:L}. On suppose donc \(\Re(a_0) \ge 0 \),\(\Re(a_1) \le 0\) et \(\Re(a_2)\le0\)), donc \(\left(\conj{a_0}  + \conj{a_1} \LD \right)\) et \(\left(\conj{a_0}  - \conj{a_2} \LR \right)\) sont injectifs et donc on déduit que

    % \begin{align*}
    %   \left(\conj{a_0}\oI  + \conj{a_1} \LD \right)\LD\vR_1 = 0 &\Rightarrow \LD\vR_1 = 0,
    %   \\
    %   \left(\conj{a_0}\oI  - \conj{a_2} \LR \right)\LR\vR_2 = 0 &\Rightarrow \LR\vR_2 = 0
    % \end{align*}

    Donc on déduit que \(\LD\vR_1 = 0\), \(\LR\vR_2 = 0\).

    Or par définition \(\LR\conj{\vW_1} = \left(\conj{a_0}  + \conj{a_1} \LD \right)\vR_1\) et \(\LD\conj{\vW_2} = \left(\conj{a_0}  - \conj{a_2} \LR \right)\vR_2\), donc
    \begin{align*}
      \LR\conj{\vW_1} = \conj{a_0}\vR_1, && \LD\conj{\vW_2} = \conj{a_0}\vR_2,
      \intertext{donc}
      \vR_1 = \frac{1}{\conj{a_0}}\LR\conj{\vW_1}, && \vR_2 = \frac{1}{\conj{a_0}}\LD\conj{\vW_2}.
    \end{align*}

    On réinjecte ce résultat dans la définition de \(X\).

    \begin{multline*}
      X = \int_\Gamma z \vV \cdot \left(\conj{a_0}\oI  + \conj{a_1} \LD - \conj{a_2}\LR\right)\conj{\vV}
      \\
      + \frac{b_1}{a_1} \norm{\vW_1}^2 - \frac{b_1\conj{a_2}}{a_1\conj{a_0}}\int_\Gamma \LR\conj{\vW_1}\cdot\vW_1
      \\
      + \frac{b_2}{a_2} \norm{\vW_2}^2 + \frac{b_2\conj{a_1}}{a_2\conj{a_0}}\int_\Gamma \LD\conj{\vW_2}\cdot\vW_2.
    \end{multline*}

    On impose la positivité de la partie réelle de chaque terme pour avoir \(\Re(X)\ge 0\).
  \end{proof}

  Soit \(S = \left\lbrace (a_0,a_1,a_2,b_1,b_2) \in \CC^5 ; a_1 = a_2 ; b_1 = b_2 = 0 \right\rbrace \). On remarque que
  \begin{align}
    \CSU[3]{CI3} & \subset \CSU{CI4}\times\CC^2,
    \\ 
    \CSU[3]{CI3}\cap S & = (\CSU{CI4}\times\CC^2)\cap S. 
  \end{align}

