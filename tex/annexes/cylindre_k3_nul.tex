\section[Solution dans le cylindre quand k3 = 0]{Solution dans le cylindre quand \(k_3 = 0\)}
  \label{sec:annexe:cylindre:k3_nul}

  Dans chaque couche de l'empilement cylindrique.

  Si \(k_3 = 0\), \(k_z^2 = w^2\eps\mu\) et il existe donc des constantes \((c_1(n),c_2(n)) \in \CC(\NN)^2\) telles que
  \begin{equation*}
    \hat{E_z}(r,n,k_z) = c_1(n) r^n + c_2(n)r^{-n}.
  \end{equation*}
  Dans ce cas \(\hat{E_z}(r,n,k_z)\) est linéairement dépendant de  \(\hat{E_z}(r,-n,k_z)\) et l'on peut se restreindre à \(n\) entier naturel.

  Toute la méthode est aussi valable pour \(\hat H_z\), il existe des constantes \((c_3(n),c_4(n)) \in \CC(\NN)^2\) telles que
  \begin{equation*}
    \hat{H_z}(r,n,k_z) = c_3(n) r^n + c_4(n)r^{-n}.
  \end{equation*}

  À partir des composantes en \(\vect{e_r},\vect{e_\theta}\) des équations de Maxwell, on peut déterminer \(\hat{E_r},\hat{E_\theta},\hat{H_r},\hat{H_\theta}\).
  \begin{align*}
      &\left\lbrace
      \begin{aligned}
          -i\w\sqrt{\mu\eps}\hat{E_\theta} + i\w\mu \hat{H_r} &= -\frac{in}{r}\hat{E_z},
          \\
          i\w\eps \hat{E_\theta} - i\w\sqrt{\mu\eps} \hat{H_r} &= -\ddp{r}{\hat{H_z}},          
          \\
          i\w\sqrt{\mu\eps}\hat{E_r} + i\w\mu \hat{H_\theta} &= \ddp{r}{\hat{E_z}},
          \\
          i\w\eps \hat{E_r} + i\w\sqrt{\mu\eps} \hat{H_\theta} &= \frac{in}{r}\hat{H_z}.
      \end{aligned}
      \right.
  \end{align*}
  Ce système n'est pas inversible. On remarque une redondance dans les termes de gauche des équations.

  \begin{align*}
      &\left\lbrace
      \begin{aligned}
          -i\w\sqrt{\mu\eps}\hat{E_\theta} + i\w\mu \hat{H_r} &= -\frac{in}{r}\left(c_1(n) r^n + c_2(n)r^{-n}\right),
          \\
          i\w\eps \hat{E_\theta} - i\w\sqrt{\mu\eps} \hat{H_r} &= -n\left(c_3(n)r^{n-1} - c_4(n)r^{-n-1}\right),           
          \\
          i\w\sqrt{\mu\eps}\hat{E_r} + i\w\mu \hat{H_\theta} &= n\left(c_1(n)r^{n-1} - c_2(n)r^{-n-1}\right),
          \\
          i\w\eps \hat{E_r} + i\w\sqrt{\mu\eps} \hat{H_\theta} &= \frac{in}{r}\left(c_3(n) r^n + c_4(n)r^{-n}\right).
      \end{aligned}
      \right.
  \end{align*}

  Il est donc nécessaire pour que les égalités soient compatibles que \(c_3(n)=-i\sqrt{\eps\slash\mu} c_1(n)\) et \(c_4(n) = i\sqrt{\eps\slash\mu}c_2(n)\) donc \(\begin{bmatrix}\hat{E_z}(r,n,k_z)\\\hat{H_z}(r,n,k_z)\end{bmatrix}=\begin{bmatrix}r^n&r^{-n}\\-i\sqrt{\eps\slash\mu}r^n&i\sqrt{\eps\slash\mu}r^{-n}\end{bmatrix}\begin{bmatrix}c_1(n)\\c_2(n)\end{bmatrix}\).
  On considère alors les deux égalités non redondantes de ce système ainsi que les composantes \(\vect{e_z}\) des équations de Maxwell.
  \begin{align*}
      &\left\lbrace
      \begin{aligned}
        -i\w\sqrt{\mu\eps}\hat{E_\theta} + i\w\mu \hat{H_r} &= -{in}\left(c_1(n) r^{n-1} + c_2(n)r^{-n-1}\right),
        \\
        i\w\sqrt{\mu\eps}\hat{E_r} + i\w\mu \hat{H_\theta} &= n\left(c_1(n)r^{n-1} - c_2(n)r^{-n-1}\right),
        \\
        \frac{1}{r}\ddr{r}{(r\hat{E_\theta})} - \frac{in}{r}{\hat{E_r}}  &= -\w\sqrt{\eps\mu} \left(c_1(n) r^n - c_2(n)r^{-n}\right),
        \\
        \frac{1}{r}\ddr{r}{(r\hat{H_\theta})} - \frac{in}{r}{\hat{H_r}}  &=  i\w\eps \left(c_1(n) r^n + c_2(n)r^{-n}\right).
      \end{aligned}
      \right.
      \intertext{On injecte les deux premières égalités dans les deux dernières.}
      &\left\lbrace
      \begin{aligned}
        \frac{1}{r}\ddr{r}{(r\hat{E_\theta})} - \frac{n}{\w\sqrt{\mu\eps}r}\left(-i\w\mu \hat{H_\theta} + n\left(c_1(n)r^{n-1} - c_2(n)r^{-n-1}\right)\right)  &= -\w\sqrt{\eps\mu} \left(c_1(n) r^n - c_2(n)r^{-n}\right),
        \\
        \frac{1}{r}\ddr{r}{(r\hat{H_\theta})} - \frac{n}{\w\mu r}\left(i\w\sqrt{\mu\eps}\hat{E_\theta} -{in}\left(c_1(n) r^{n-1} + c_2(n)r^{-n-1}\right)\right)  &=  i\w\eps \left(c_1(n) r^n + c_2(n)r^{-n}\right).  
      \end{aligned}
      \right.
      \intertext{On développe}
      &\left\lbrace
      \begin{aligned}
        \frac{1}{r}\ddr{r}{(r\hat{E_\theta})} + i\frac{n}{r}\sqrt{\frac{\mu}{\eps}}\hat{H_\theta} &= \frac{n^2}{\w\sqrt{\mu\eps}}\left(c_1(n)r^{n-2} - c_2(n)r^{-n-2}\right) -\w\sqrt{\eps\mu} \left(c_1(n) r^n - c_2(n)r^{-n}\right),
        \\
        \frac{1}{r}\ddr{r}{(r\hat{H_\theta})} - i\frac{n}{r}\sqrt{\frac{\eps}{\mu}}\hat{E_\theta} &=  \frac{-{in^2}}{\w\mu}\left( c_1(n) r^{n-2} + c_2(n)r^{-n-2}\right) + i\w\eps \left(c_1(n) r^n + c_2(n)r^{-n}\right).  
      \end{aligned}
      \right.
      \intertext{On factorise}
      &\left\lbrace
      \begin{aligned}
        \frac{1}{r}\ddr{r}{(r\hat{E_\theta})} + i\frac{n}{r}\sqrt{\frac{\mu}{\eps}}\hat{H_\theta} &= -c_1(n)\sqrt{\frac{\mu}{\eps}}\left(\w\eps r^n - \frac{n^2}{\w\mu}r^{n-2}\right) + c_2(n)\sqrt{\frac{\mu}{\eps}}\left(\w\eps r^{-n} - \frac{n^2}{\w\mu}r^{-n-2}\right),
        \\
        \frac{1}{r}\ddr{r}{(r\hat{H_\theta})} - i\frac{n}{r}\sqrt{\frac{\eps}{\mu}}\hat{E_\theta} &=  c_1(n)i\left(\w\eps r^n - \frac{n^2}{\w\mu}r^{n-2}\right) + c_2(n)i\left(\w\eps r^{-n} -\frac{n^2}{\w\mu}r^{-n-2}\right). 
      \end{aligned}
      \right.
  \end{align*}

  On voit que si l'on injecte l'une dans l'autre, on obtiendra une EDO d'ordre 2 inhomogène. L'expression de la solution de cette EDO n'est pas simple ni immédiate, nous n'avons donc pas continué l'étude dans ce cas.

