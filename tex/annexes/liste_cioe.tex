\section{Liste synthétique des CIOE}

Nous rappelons l'expression des opérateurs aux dérivées partielles \(\LD,\LR,\LL\) sur des champs vectoriels tangents
\begin{align*}
  \LD &= \vgrads{}\vdivs{}, &
  \LR &= \vrots{} \vrots{}, &
  \LL &= \LD - \LR.
\end{align*}
L'opérateur \(\oI\) est l'identité. On rappelle que \(\vJ = (\vn \pvect \vH)_t\).

% Dans les codes numériques, tous ces opérateurs différentiels sont normalisés par le carré du nombre d'onde dans le vide.
\begin{itemize}
  \item \hypertarget{ci0}{CI0}
  \begin{itemize}
    \item Autres noms: SIBC, LIBC
    \item Expression: \(\vE_t = a_0 \vJ\)
    \item Référence: \cite{leontovich_investigations_1948}
  \end{itemize}

  \item \hypertarget{ci01}{CI01}
  \begin{itemize}
    \item Autre nom: Taylor 1
    \item Expression: \(\vE_t = \left(a_0\oI + a_1 \LL \right)\vJ\)
    \item Référence: \cite{stupfel_implementation_2015}
  \end{itemize}


  \item \hypertarget{ci1}{CI1}
  \begin{itemize}
    \item Autre nom: Pade 1-1
    \item Expression: \(\left(\oI + b \LL \right)\vE_t = \left(a_0\oI + a_1 \LL \right)\vJ\)
    \item Référence: \cite{stupfel_implementation_2015}
  \end{itemize}

  \item \hypertarget{ci4}{CI4}
  \begin{itemize}
    \item Autre nom: Polynomial
    \item Expression: \(\vE_t = \left(a_0\oI + a_1 \LD - a_2 \LR \right)\vJ\)
    \item Référence: \cite{hoppe_impedance_1995}
  \end{itemize}

  \item \hypertarget{ci3}{CI3}
  \begin{itemize}
    \item Autre nom: Ratio
    \item Expression: \(\left(\oI + b_1 \LD - b_2 \LR \right)\vE_t = \left(a_0\oI + a_1 \LD - a_2 \LR \right)\vJ\)
    \item Références: \cite{hoppe_impedance_1995,marceaux_high-order_2000,aubakirov_electromagnetic_2014}
  \end{itemize}

  \item \hypertarget{ci6}{CI6}
  \begin{itemize}
    \item Autre nom: Curvature ratio
    \item Expression: \(\left(\oI + c_1 \LD - c_2 \LR \right)\vE_t = \left(\diag{a_1}{a_2} + b_1 \LD - b_2 \LR \right)\vJ\)
    \item Référence: \cite{hoppe_impedance_1995}
  \end{itemize}

  \item \hypertarget{ci7}{CI7}
  \begin{itemize}
    \item Expression: \(\left(\oI + \sum_{i=1}^3\left( d_i \LD^i + e_i (-\LR)^i\right) \right)\vE_t = \left(a_0\oI + \sum_{i=1}^3\left( b_i \LD^i + c_i (-\LR)^i\right) \right)\vJ\)
  \end{itemize}
\end{itemize}
