\section{Tableaux des CIOE}

Nous rappelons l'expression des opérateurs \(\LD,\LR,\LL\) sur des champs vectoriels tangents
\begin{align*}
  \LD &= \vgrads{}\vdivs{} &
  \LR &= \vrots{} \vrots{} &
  \LL &= \vlapls{}
\end{align*}
De plus, \(\oI\) est l'identité.

Dans les codes numériques, tous les opérateurs différentiels sont normalisés par le carré du nombre d'onde dans le vide.
\begin{landscape}
\vfill
\begin{center}
\begin{tabular}{llCl}
Nom & Autre noms & \text{Expression sur les champs tangentiels} & Référence
\\
\hline
\hline
\hypertarget{ci0}{CI0} & SIBC, LIBC & \vE_t = a_0 \vJ  & \cite{leontovich_investigations_1948}
\\
\hline
\hline
\hypertarget{ci01}{CI01} & Taylor 1 & \vE_t = \left(a_0\oI + a_1 \frac{\LL}{k_0^2} \right)\vJ & \multirow{2}{*}{\cite{stupfel_implementation_2015}}
\\
\hypertarget{ci1}{CI1}& Pade 1-1 & \left(\oI + b \frac{\LL}{k_0^2} \right)\vE_t = \left(a_0\oI + a_1 \frac{\LL}{k_0^2} \right)\vJ &
\\
\hline
\hline
\hypertarget{ci4}{CI4}& Polynomial &\vE_t = \left(a_0\oI + a_1 \frac{\LD}{k_0^2} - a_2 \frac{\LR}{k_0^2} \right)\vJ & \multirow{3}{*}{\cite{marceaux_high-order_2000}}
\\
\hypertarget{ci3}{CI3} & Ratio &\left(\oI + b_1 \frac{\LD}{k_0^2} - b_2 \frac{\LR}{k_0^2} \right)\vE_t = \left(a_0\oI + a_1 \frac{\LD}{k_0^2} - a_2 \frac{\LR}{k_0^2} \right)\vJ &
\\
\hypertarget{ci6}{CI6} & Curvature ratio & \left(\oI + c_1 \frac{\LD}{k_0^2} - c_2 \frac{\LR}{k_0^2} \right)\vE_t = \left(\diag{a_1}{a_2} + b_1 \frac{\LD}{k_0^2} - b_2 \frac{\LR}{k_0^2} \right)\vJ &
\\
\hline
\hline
\hypertarget{ci7}{CI7}&  & \left(\oI + \sum_{i=1}^3\left( d_i \frac{\LD^i}{k_0^2} + e_i \frac{(-\LR)^i}{k_0^2}\right) \right)\vE_t = \left(a_0\oI + \sum_{i=1}^3\left( b_i \frac{\LD^i}{k_0^2} + c_i \frac{(-\LR)^i}{k_0^2}\right) \right)\vJ &
\end{tabular}
\end{center}
\vfill
\end{landscape}