% \begin{landscape}
\section{Tableaux des CIOE}

Nous rappelons l'expression des opérateurs \(\LD,\LR,\LL\) sur des champs vectoriels tangents
\begin{align*}
  \LD &= \vgrads{}\vdivs{} &
  \LR &= \vrots{} \vrots{} &
  \LL &= \vlapls{}
\end{align*}

Dans les codes numériques, tous les opérateurs sont normalisés par le carré du nombre d'onde dans le vide.

\begin{center}
\begin{tabular}{l|Cl}
Nom & \text{Expression sur les champs tangentiels} & Référence
\\
\hline
\hline
\hypertarget{ci0}{CI0} & \vE_t = a_0 \vJ  & \cite{leontovich_investigations_1948}
\\
\hypertarget{ci01}{CI01} & \vE_t = \left(a_0\oI + a_1 \LL \right)\vJ & \cite{stupfel_implementation_2015}
\\
\hypertarget{ci1}{CI1}& \left(\oI + b \LL \right)\vE_t = \left(a_0\oI + a_1 \LL \right)\vJ & \cite{stupfel_implementation_2015}
\\
\hypertarget{ci4}{CI4}& \vE_t = \left(a_0\oI + a_1 \LD - a_2 \LR \right)\vJ & \cite{hoppe_impedance_1995}
\\
\hypertarget{ci3}{CI3}& \left(\oI + b_1 \LD - b_2 \LR \right)\vE_t = \left(a_0\oI + a_1 \LD - a_2 \LR \right)\vJ & \cite{hoppe_impedance_1995}
\\
\hypertarget{ci6}{CI6}& \left(\oI + c_1 \LD - c_2 \LR \right)\vE_t = \left(\diag{a_1}{a_2} + b_1 \LD - b_2 \LR \right)\vJ & \cite{hoppe_impedance_1995}
\end{tabular}
\end{center}
% \end{landscape}