\section[Solution dans le plan quand k3 = 0]{Solution dans le plan quand \(k_3 = 0\)}
  \label{sec:annexe:plan:k3_nul}

  Dans chaque couche de l'empilement plan.

  Si \(k_3 = k_0^2\eps\mu - k_x^2 - k_y^2 = 0\), il existe \((c_1(k_x,k_y),c_2(k_x,k_y),c_3(k_x,k_y),c_4(k_x,k_y)) \in (\CC(\RR^2))^4\) telles que
  \begin{align*}
    \hat{E_y}(k_x,k_y,z) &= c_1(k_x,k_y) z + c_2(k_x,k_y),
    \\   
    \hat{\cH_y}(k_x,k_y,z) &= c_3(k_x,k_y) z + c_4(k_x,k_y).
  \end{align*}

  À partir des composantes \(\vect{e_x},\vect{e_y}\) des équations de Maxwell, on peut déterminer \(\hat{E_x},\hat{E_z},\hat{\cH_x},\hat{\cH_z}\).
  Cela revient à résoudre \(Y = \mM X\) où la matrice \(\mM\) et les vecteurs \(X, Y\) sont définis tels que
  \begin{align*}
    \mM =
    \begin{bmatrix}
    0 & -ik_y & -ik_0\mu & 0
    \\
    ik_y & 0 & 0 & -ik_0\mu
    \\
    ik_0\eps & 0 & 0 & -ik_y
    \\
    0 & ik_0\eps & ik_y & 0
    \end{bmatrix},
    &&
    X =
    \begin{bmatrix}
      \hat{E_x}\\
      \hat{E_z}\\
      \hat{\cH_x}\\
      \hat{\cH_z}
    \end{bmatrix},
    &&
    Y =
    \begin{bmatrix}
      -\ddr{z}{\hat{E_y}}\\
      ik_x\hat{E_y}\\
      -\ddr{z}{\hat{\cH_y}}\\
      ik_x\hat{\cH_y}
    \end{bmatrix}.
  \end{align*}

  Cette matrice est inversible si \(\det(\mM) = (k^2 - k_y^2)^2 \) est non nul. On suppose donc \(k_x^2\not=0\). Alors \(\mM^{-1}=\frac{1}{k_x^2}\mM\).

  \begin{align*}
  &\left\lbrace
    \begin{aligned}
      \hat{E_x} &= \frac{1}{k_x^2}\left(ik_yik_x\hat{E_y} + ik_0\mu\ddr{z}{\hat{\cH_y}}\right),
      \\
      \hat{E_z} &= \frac{1}{k_x^2}\left(-ik_y\ddr{z}{\hat{E_y}}-ik_0\mu ik_x\hat{\cH_y}\right)
    \end{aligned}
  \right.    
  &&
  \left\lbrace
    \begin{aligned}
      \hat{\cH_x} &= \frac{1}{k_x^2}\left(-ik_0\eps\ddr{z}{\hat{E_y}} + ik_yik_x\hat{\cH_y}\right),
      \\
      \hat{\cH_z} &= \frac{1}{k_x^2}\left(ik_0\eps ik_x \hat{E_y} - ik_y \ddr{z}{\hat{\cH}_y} \right)
    \end{aligned}
    \right.
  \end{align*}

  Alors on extrait alors uniquement les composantes suivant \(x,y\) de \(\hat{\vE}\) et \(\vect{e_z} \pvect \hat{\vH}\)
  \begin{align*}
  &\left\lbrace
    \begin{aligned}
      \hat{E_x} &= \frac{1}{k_x^2}\left(-k_yk_x\hat{E_y} + ik_0\mu c_3(k_x,k_y)\right),
      \\
      \hat{E_y} &= c_1(k_x,k_y)z + c_2(k_x,k_y)
    \end{aligned}
  \right.    
  \\&
  \left\lbrace
    \begin{aligned}
      -\hat{\cH_y} &= -c_3(k_x,k_y)z - c_4(k_x,k_y),
      \\
      \hat{\cH_x} &= -\frac{1}{k_x^2}\left(ik_0\eps c_1(k_x,k_y) + k_yk_x\hat{\cH_y}\right)
    \end{aligned}
    \right.
  \end{align*}

  Vérifions \(\vdiv\vE\).
  \begin{align*}
    \vdiv{\hat{\vE}} &= ik_x\hat{E_x} &+ ik_y\hat{E_y} &&+ \ddr{z}{\hat{E_z}},
    \\
    &= -ik_y\hat{E_y} - \frac{k_0\mu}{k_x}c_3 &+ ik_y \hat{E_y} &&+ \frac{k_0\mu}{k_x}c_3,
    \\
    &= 0.
  \end{align*}

  % {
  % \begin{itshape}
  % Notes pour Olivier: si \(k_y =0\), on l'identification

  % \begin{align*}
  % \hat{E_x} &= i\frac{k_0\mu}{k_x^2} c_3 &&\equiv a_x,
  % \\
  % \hat{E_y} &= c_1 z + c_2 && \equiv c_y z + a_y,
  % \\
  % \hat{E_z} &= \frac{k_0\mu}{k_x}(c_3 z + c_4) &&\equiv -i k_x a_x z + a_z,
  % \\
  % (\vrot\hat{\vE})_x &= -c_1 && \equiv  -c_y,
  % \\
  % (\vrot\hat{\vE})_y &= -i{k_0\mu}(c_3 z + c_4) &&\equiv ik_x(ik_x a_x z - a_z),
  % \\
  % (\vrot\hat{\vE})_z &= ik_x(c_1 z + c_2) && \equiv ik_x(c_y z + a_y).
  % \end{align*}


  % Mon calcul de \(\hat{E_x},\hat{\cH}_x\) est correct, mais pas ce qui suit.

  % \end{itshape}
  % }

  On définit les matrices \(\mA_{E}(k_x,k_y,z)\), \(\mB_{E}(k_x,k_y,z)\), \(\mA_{H}(k_x,k_y,z)\), \(\mB_{H}(k_x,k_y,z)\) constantes par morceaux en \(z\)
  \begin{align*}
    \mA_{E}(k_x,k_y,z) &= z
    \begin{bmatrix}
      -\frac{k_y}{k_x} & -i\frac{k_0\mu(z)}{k_x^2}
      \\
      1 & 0
    \end{bmatrix},
    \\
    \mB_{E}(k_x,k_y,z) &= 
    \begin{bmatrix}
      -\frac{k_y}{k_x} & 0
      \\
      1 & 0
    \end{bmatrix},
    \\
    \mA_{H}(k_x,k_y,z) &= z
    \begin{bmatrix}
      0 & -1
      \\
      \frac{-ik_0\eps(z)}{k_x^2} & -\frac{k_y}{k_x}
    \end{bmatrix},
    \\
    \mB_{H}(k_x,k_y,z) &= 
    \begin{bmatrix}
      0 & -1
      \\
      0 & -\frac{k_y}{k_x}
    \end{bmatrix}.
  \end{align*}

  Les composantes en \(\vect{e_x},\vect{e_y}\) des champs, dites composantes tangentielles par abus de langage sont
  \begin{align*}
      \hat{\vE}_t(k_x,k_y,z) &= \mA_E(k_x,k_y,z) \begin{bmatrix}c_1(k_x,k_y,z) \\ c_3(k_x,k_y,z)\end{bmatrix} + \mB_E(k_x,k_y,z) \begin{bmatrix}c_2(k_x,k_y,z) \\ c_4(k_x,k_y,z)\end{bmatrix},
      \\
      (\vn \pvect \hat{\vH})_t(k_x,k_y,z) &= \mA_H(k_x,k_y,z) \begin{bmatrix}c_1(k_x,k_y,z) \\ c_3(k_x,k_y,z)\end{bmatrix} + \mB_H(k_x,k_y,z) \begin{bmatrix}c_2(k_x,k_y,z) \\ c_4(k_x,k_y,z)\end{bmatrix}.
  \end{align*}

  Mais les matrices \(\mB_{H}, \mB_E\) ne sont pas inversibles et donc on ne pourra pas trouver la matrice \(\hat\mZ\).

 %  {
 %  \begin{itshape}
 %  Notes pour Olivier: 

 %  En fait, le cas \(k_3=0\) montre bien la limite de la simplification: on n'a pas besoin de l'inversion de toutes les matrices pour trouver \(\mZ\).

 %  % Utilisons les relations de saut de la partie tangentielle des champs à une interface \(z_0\), avec \(k_y = 0\).

 %  % \begin{align*}
 %  % i\frac{k_0\mu_1}{k_x^2}c_3^1 &= i\frac{k_0\mu_2}{k_x^2}c_3^2,
 %  % \\
 %  % c_1^1 z_0 + c_2^1 &= c_1^2 z_0 + c_2^2,
 %  % \\
 %  % -i\frac{k_0\eps_1}{k_x^2}c_1^1 &= -i\frac{k_0\eps_2}{k_x^2}c_1^2,
 %  % \\
 %  % c_3^1 z_0 + c_4^1 &= c_3^2 z_0 + c_4^2.
 %  % \end{align*}

 %  % Ce système, d'inconnus \((c_1^1,c_2^1,c_3^1,c_4^1)\), a comme déterminant 
 %  % \begin{align*}
 %  % \begin{vmatrix}
 %  % 0 & 0 & \mu_1 & 0
 %  % \\
 %  % z_0 & 1 & 0 & 0
 %  % \\
 %  % \eps_1 & 0 & 0 & 0
 %  % \\
 %  % 0 & 0 & z_0 & 1 
 %  % \end{vmatrix} = -\eps_1
 %  % \begin{vmatrix}
 %  % 0 & \mu_1 & 0
 %  % \\
 %  % 1 & 0 & 0
 %  % \\
 %  % 0 & z_0 & 1 
 %  % \end{vmatrix} = \eps_1
 %  % \begin{vmatrix}
 %  % \mu_1 & 0
 %  % \\
 %  % z_0 & 1 
 %  % \end{vmatrix} = \eps_1\mu_1,
 %  % \end{align*}
 %  % il est donc inversible.

 %  Utilisons les relations de saut de la partie tangentielle des champs à une interface \(z_0\), avec \(k_y\) quelconque.

 %  \begin{align*}
 %  -\frac{k_y}{k_x}(c_1^1 z_0 + c_2^1) + i\frac{k_0\mu_1}{k_x^2}c_3^1 &= \dots,
 %  \\
 %  c_1^1 z_0 + c_2^1 &= \dots,
 %  \\
 %  -\frac{k_y}{k_x}(c_3^1 z_0 + c_4^1)-i\frac{k_0\eps_1}{k_x^2}c_1^1 &= \dots,
 %  \\
 %  c_3^1 z_0 + c_4^1 &= \dots.
 %  \end{align*}

 %  Ce système, d'inconnus \((c_1^1,c_2^1,c_3^1,c_4^1)\), a comme déterminant 
 %  \begin{align*}
 %  \begin{vmatrix}
 %   -{k_y}z_0 &  -{k_y} & i\frac{k_0\mu_1}{k_x} & 0
 %  \\
 %  z_0 & 1 & 0 & 0
 %  \\
 % -i\frac{k_0\eps_1}{k_x} & 0 & -{k_y}z_0 & -{k_y}
 %  \\
 %  0 & 0 & z_0 & 1 
 %  \end{vmatrix}  &= 
 %  \begin{vmatrix}
 %  0 &  -{k_y} & i\frac{k_0\mu_1}{k_x} & 0
 %  \\
 %  0 & 1 & 0 & 0
 %  \\
 % -i\frac{k_0\eps_1}{k_x} & 0 & -{k_y}z_0 & -{k_y}
 %  \\
 %  0 & 0 & z_0 & 1 
 %  \end{vmatrix} = -i\frac{k_0\eps_1}{k_x}
 %  \begin{vmatrix}
 %  -{k_y} & i\frac{k_0\mu_1}{k_x} & 0
 %  \\
 %  1 & 0 & 0
 %  \\
 %  0 & z_0 & 1 
 %  \end{vmatrix}
 %  \\
 %  & = i\frac{k_0\eps_1}{k_x}
 %  \begin{vmatrix}
 %  i\frac{k_0\mu_1}{k_x} & 0
 %  \\
 %  z_0 & 1 
 %  \end{vmatrix} = -\frac{k_0^2\mu_1\eps_1}{k_x^2} = -\frac{k^2}{k_x^2},
 %  \end{align*}
 %  il est donc inversible si \(k_x\not=0\), ce qu'on a déjà supposé pour trouver l'expression des champs.

 %  Dans ta note sectionsDE \& résonances, tu exprimes une condition d'unicité mais ce que je cherche dans cette annexe est la condition d'impédance.


 %  Point bloquant:

 %  L'unicité des solutions pour deux couches est assurée si le déterminant suivant, d'inconnus  \(
 %  c_1^1, c_2^1, c_3^1, c_4^1, c_1^2, c_2^2, c_3^2, c_4^3
 %  \), est non nul ( champs \(\vE_t\) nul en \(z_0\) et \(z_2\) et saut de \(\vE_t\), \(\vH_t\) nul en \(z_1\) ).

 %  \begin{align*}
 %  &{\,}
 %  \begin{vmatrix}
 %  -{k_y}z_0 &  -{k_y} & i\frac{k_0\mu_1}{k_x} & 0 & 0 & 0 & 0 & 0
 %  \\
 %  z_0 & 1 & 0 & 0 & 0 & 0 & 0 & 0
 %  \\
 %  -{k_y} z_1 & - {k_y} & i\frac{k_0\mu_1}{k_x} & 0 & k_y z_1 & k_y & -i\frac{k_0\mu_2}{k_x} & 0
 %  \\
 %  z_1 & 1 & 0 & 0 & -z_1 & -1 & 0 & 0
 %  \\
 %  -i\frac{k_0\eps_1}{k_x} & 0 & -k_y z_1 & - k_y & i\frac{k_0\eps_2}{k_x} & 0 & k_y z_1 & k_y
 %  \\
 %  0 & 0 & z_1 & 1 & 0 & 0 & -z_1 & -1
 %  \\
 %  0 & 0 & 0 & 0 & -k_y z_2 & -k_y & i\frac{k_0\mu_2}{k_x} & 0
 %  \\
 %  0 & 0 & 0 & 0 & z_2 & 1 & 0 & 0
 %  \end{vmatrix}
 %  % \\
 %  % (C_1 \leftarrow C_1 - z_0 C_2 )&=
 %  % \begin{vmatrix}
 %  % 0 & -{k_y} & i\frac{k_0\mu_1}{k_x} & 0 & 0 & 0 & 0 & 0
 %  % \\
 %  % 0 & 1 & 0 & 0 & 0 & 0 & 0 & 0
 %  % \\
 %  % -{k_y} (z_1-z_0) & - {k_y} & i\frac{k_0\mu_1}{k_x} & 0 & k_y z_1 & k_y & -i\frac{k_0\mu_2}{k_x} & 0
 %  % \\
 %  % z_1-z_0 & 1 & 0 & 0 & -z_1 & -1 & 0 & 0
 %  % \\
 %  % -i\frac{k_0\eps_1}{k_x} & 0 & -k_y z_1 & - k_y & i\frac{k_0\eps_2}{k_x} & 0 & k_y z_1 & k_y
 %  % \\
 %  % 0 & 0 & z_1 & 1 & 0 & 0 & -z_1 & -1
 %  % \\
 %  % 0 & 0 & 0 & 0 & -k_y z_2 & -k_y & i\frac{k_0\mu_2}{k_x} & 0
 %  % \\
 %  % 0 & 0 & 0 & 0 & z_2 & 1 & 0 & 0
 %  % \end{vmatrix}
 %  % \\
 %  % ( / L_2 ) &=
 %  % \begin{vmatrix}
 %  % 0 & i\frac{k_0\mu_1}{k_x} & 0 & 0 & 0 & 0 & 0
 %  % \\
 %  % -{k_y} (z_1-z_0) &  i\frac{k_0\mu_1}{k_x} & 0 & k_y z_1 & k_y & -i\frac{k_0\mu_2}{k_x} & 0
 %  % \\
 %  % z_1-z_0 & 0 & 0 & -z_1 & -1 & 0 & 0
 %  % \\
 %  % -i\frac{k_0\eps_1}{k_x} & -k_y z_1 & - k_y & i\frac{k_0\eps_2}{k_x} & 0 & k_y z_1 & k_y
 %  % \\
 %  % 0 & z_1 & 1 & 0 & 0 & -z_1 & -1
 %  % \\
 %  % 0 &  0 & 0 & -k_y z_2 & -k_y & i\frac{k_0\mu_2}{k_x} & 0
 %  % \\
 %  % 0 & 0 & 0 & z_2 & 1 & 0 & 0
 %  % \end{vmatrix}
 %  % \\
 %  % ( C_5 \leftarrow C_5 - z_2 C_6 ) &=
 %  % \begin{vmatrix}
 %  % 0 & i\frac{k_0\mu_1}{k_x} & 0 & 0 & 0 & 0 & 0
 %  % \\
 %  % -{k_y} (z_1-z_0) &  i\frac{k_0\mu_1}{k_x} & 0 & k_y (z_1-z_2) & k_y & -i\frac{k_0\mu_2}{k_x} & 0
 %  % \\
 %  % z_1-z_0 & 0 & 0 & -(z_1- z_2) & -1 & 0 & 0
 %  % \\
 %  % -i\frac{k_0\eps_1}{k_x} & -k_y z_1 & - k_y & i\frac{k_0\eps_2}{k_x} & 0 & k_y z_1 & k_y
 %  % \\
 %  % 0 & z_1 & 1 & 0 & 0 & -z_1 & -1
 %  % \\
 %  % 0 &  0 & 0 & 0 & -k_y & i\frac{k_0\mu_2}{k_x} & 0
 %  % \\
 %  % 0 & 0 & 0 & 0 & 1 & 0 & 0
 %  % \end{vmatrix}
 %  % \\
 %  % ( / L_7 ) &=
 %  % \begin{vmatrix}
 %  % 0 & i\frac{k_0\mu_1}{k_x} & 0 & 0 & 0 & 0
 %  % \\
 %  % -{k_y} (z_1-z_0) &  i\frac{k_0\mu_1}{k_x} & 0 & k_y (z_1-z_2) & -i\frac{k_0\mu_2}{k_x} & 0
 %  % \\
 %  % z_1-z_0 & 0 & 0 & -(z_1- z_2) & 0 & 0
 %  % \\
 %  % -i\frac{k_0\eps_1}{k_x} & -k_y z_1 & - k_y & i\frac{k_0\eps_2}{k_x} & k_y z_1 & k_y
 %  % \\
 %  % 0 & z_1 & 1 & 0 & -z_1 & -1
 %  % \\
 %  % 0 &  0 & 0 & 0 & i\frac{k_0\mu_2}{k_x} & 0
 %  % \end{vmatrix}
 %  % \\
 %  % ( / L_1 ) &= -i\frac{k_0\mu_1}{k_x}
 %  % \begin{vmatrix}
 %  % -{k_y} (z_1-z_0) & 0 & k_y (z_1-z_2) & -i\frac{k_0\mu_2}{k_x} & 0
 %  % \\
 %  % z_1-z_0 & 0 & -(z_1- z_2) & 0 & 0
 %  % \\
 %  % -i\frac{k_0\eps_1}{k_x} & - k_y & i\frac{k_0\eps_2}{k_x} & k_y z_1 & k_y
 %  % \\
 %  % 0 & 1 & 0 & -z_1 & -1
 %  % \\
 %  % 0 & 0 & 0 & i\frac{k_0\mu_2}{k_x} & 0
 %  % \end{vmatrix}
 %  % \\
 %  % ( / L_5 ) &= -\frac{k_0^2\mu_1\mu_2}{k_x^2}
 %  % \begin{vmatrix}
 %  % -{k_y} (z_1-z_0) & 0 & k_y (z_1-z_2) & 0
 %  % \\
 %  % z_1-z_0 & 0 & -(z_1- z_2) & 0
 %  % \\
 %  % -i\frac{k_0\eps_1}{k_x} & - k_y & i\frac{k_0\eps_2}{k_x} & k_y
 %  % \\
 %  % 0 & 1 & 0 & -1
 %  % \end{vmatrix}
 %  % \\
 %  % ( / C_2 \rightarrow C_2 + C_4 ) &= -\frac{k_0^2\mu_1\mu_2}{k_x^2}
 %  % \begin{vmatrix}
 %  % -{k_y} (z_1-z_0) & 0 & k_y (z_1-z_2) & 0
 %  % \\
 %  % z_1-z_0 & 0 & -(z_1- z_2) & 0
 %  % \\
 %  % -i\frac{k_0\eps_1}{k_x} & 0 & i\frac{k_0\eps_2}{k_x} & k_y
 %  % \\
 %  % 0 & 0 & 0 & -1
 %  % \end{vmatrix}
 %  % \\
 %  % ( / L_4 ) &= -\frac{k_0^2\mu_1\mu_2}{k_x^2}
 %  % \begin{vmatrix}
 %  % -{k_y} (z_1-z_0) & 0 & k_y (z_1-z_2)
 %  % \\
 %  % z_1-z_0 & 0 & -(z_1- z_2)
 %  % \\
 %  % -i\frac{k_0\eps_1}{k_x} & 0 & i\frac{k_0\eps_2}{k_x}
 %  % \end{vmatrix}
 %  \end{align*}

 %  Or la 4\ieme et la 8\ieme colonne sont colinéaires, donc ce dernier est toujours nul. Nous n'avons pas unicité des solutions. Donc il y a redondance dans nos équations et on peut exprimer une des \(c_i^j\) en fonctions des autres.

 %  En fait, si on considère les CL champs \(\vE_t\) nul en \(z_0\), champ \(\vH_t\) nul en \(z_2\) et saut de \(\vE_t\), \(\vH_t\) nul en \(z_1\), on tombe sur une déterminant non nul, mais il faut que tu m'expliques pourquoi dans ta notes, tu considères \(\vE_t, \vH_t\).

 %  La preuve de ce déterminant.
 %  \begin{align*}
 %  &{\,}
 %  \begin{vmatrix}
 %  -{k_y}z_0 &  -{k_y} & i\frac{k_0\mu_1}{k_x} & 0 & 0 & 0 & 0 & 0
 %  \\
 %  z_0 & 1 & 0 & 0 & 0 & 0 & 0 & 0
 %  \\
 %  -{k_y} z_1 & - {k_y} & i\frac{k_0\mu_1}{k_x} & 0 & k_y z_1 & k_y & -i\frac{k_0\mu_2}{k_x} & 0
 %  \\
 %  z_1 & 1 & 0 & 0 & -z_1 & -1 & 0 & 0
 %  \\
 %  -i\frac{k_0\eps_1}{k_x} & 0 & -k_y z_1 & - k_y & i\frac{k_0\eps_2}{k_x} & 0 & k_y z_1 & k_y
 %  \\
 %  0 & 0 & z_1 & 1 & 0 & 0 & -z_1 & -1
 %  \\
 %  0 & 0 & 0 & 0 & -i\frac{k_0\eps_2}{k_x} & 0 & -k_y z_2 & -k_y
 %  \\
 %  0 & 0 & 0 & 0 & 0 & 0 & z_2 & 1
 %  \end{vmatrix}
 %  \\
 %  (C_1 \leftarrow C_1 - z_0 C_2 )&=
 %  \begin{vmatrix}
 %  0 & -{k_y} & i\frac{k_0\mu_1}{k_x} & 0 & 0 & 0 & 0 & 0
 %  \\
 %  0 & 1 & 0 & 0 & 0 & 0 & 0 & 0
 %  \\
 %  -{k_y} (z_1-z_0) & - {k_y} & i\frac{k_0\mu_1}{k_x} & 0 & k_y z_1 & k_y & -i\frac{k_0\mu_2}{k_x} & 0
 %  \\
 %  z_1-z_0 & 1 & 0 & 0 & -z_1 & -1 & 0 & 0
 %  \\
 %  -i\frac{k_0\eps_1}{k_x} & 0 & -k_y z_1 & - k_y & i\frac{k_0\eps_2}{k_x} & 0 & k_y z_1 & k_y
 %  \\
 %  0 & 0 & z_1 & 1 & 0 & 0 & -z_1 & -1
 %  \\
 %  0 & 0 & 0 & 0 & -i\frac{k_0\eps_2}{k_x} & 0 & -k_y z_2 & -k_y
 %  \\
 %  0 & 0 & 0 & 0 & 0 & 0 & z_2 & 1
 %  \end{vmatrix}
 %  \\
 %  ( / L_2 ) &=
 %  \begin{vmatrix}
 %  0 & i\frac{k_0\mu_1}{k_x} & 0 & 0 & 0 & 0 & 0
 %  \\
 %  -{k_y} (z_1-z_0) &  i\frac{k_0\mu_1}{k_x} & 0 & k_y z_1 & k_y & -i\frac{k_0\mu_2}{k_x} & 0
 %  \\
 %  z_1-z_0 & 0 & 0 & -z_1 & -1 & 0 & 0
 %  \\
 %  -i\frac{k_0\eps_1}{k_x} & -k_y z_1 & - k_y & i\frac{k_0\eps_2}{k_x} & 0 & k_y z_1 & k_y
 %  \\
 %  0 & z_1 & 1 & 0 & 0 & -z_1 & -1
 %  \\
 %  0 & 0 & 0 & -i\frac{k_0\eps_2}{k_x} & 0 & -k_y z_2 & -k_y
 %  \\
 %  0 & 0 & 0 & 0 & 0 & z_2 & 1
 %  \end{vmatrix}
 %  \\
 %  ( C_6 \leftarrow C_6 - z_2 C_7 ) &=
 %  \begin{vmatrix}
 %  0 & i\frac{k_0\mu_1}{k_x} & 0 & 0 & 0 & 0 & 0
 %  \\
 %  -{k_y} (z_1-z_0) &  i\frac{k_0\mu_1}{k_x} & 0 & k_y z_1 & k_y & -i\frac{k_0\mu_2}{k_x} & 0
 %  \\
 %  z_1-z_0 & 0 & 0 & -z_1 & -1 & 0 & 0
 %  \\
 %  -i\frac{k_0\eps_1}{k_x} & -k_y z_1 & - k_y & i\frac{k_0\eps_2}{k_x} & 0 & k_y (z_1-z_2) & k_y
 %  \\
 %  0 & z_1 & 1 & 0 & 0 & -(z_1-z_2) & -1
 %  \\
 %  0 & 0 & 0 & -i\frac{k_0\eps_2}{k_x} & 0 & 0 & -k_y
 %  \\
 %  0 & 0 & 0 & 0 & 0 & 0 & 1
 %  \end{vmatrix}
 %  \\
 %  ( / L_7 ) &=
 %  \begin{vmatrix}
 %  0 & i\frac{k_0\mu_1}{k_x} & 0 & 0 & 0 & 0
 %  \\
 %  -{k_y} (z_1-z_0) &  i\frac{k_0\mu_1}{k_x} & 0 & k_y z_1 & k_y & -i\frac{k_0\mu_2}{k_x}
 %  \\
 %  z_1-z_0 & 0 & 0 & -z_1 & -1 & 0
 %  \\
 %  -i\frac{k_0\eps_1}{k_x} & -k_y z_1 & - k_y & i\frac{k_0\eps_2}{k_x} & 0 & k_y (z_1-z_2)
 %  \\
 %  0 & z_1 & 1 & 0 & 0 & -(z_1-z_2)
 %  \\
 %  0 & 0 & 0 & -i\frac{k_0\eps_2}{k_x} & 0 & 0
 %  \end{vmatrix}
 %  \\
 %  ( / L_1 ) &= -i\frac{k_0\mu_1}{k_x}
 %  \begin{vmatrix} 
 %  -{k_y} (z_1-z_0) & 0 & k_y z_1 & k_y & -i\frac{k_0\mu_2}{k_x}
 %  \\
 %  z_1-z_0 & 0 & -z_1 & -1 & 0
 %  \\
 %  -i\frac{k_0\eps_1}{k_x} & - k_y & i\frac{k_0\eps_2}{k_x} & 0 & k_y (z_1-z_2)
 %  \\
 %  0 & 1 & 0 & 0 & -(z_1-z_2)
 %  \\
 %  0 & 0 & -i\frac{k_0\eps_2}{k_x} & 0 & 0
 %  \end{vmatrix}
 %  \\
 %  ( / L_5 ) &= -\frac{k_0^2\mu_1\eps_2}{k_x^2}
 %  \begin{vmatrix} 
 %  -{k_y} (z_1-z_0) & 0 & k_y & -i\frac{k_0\mu_2}{k_x}
 %  \\
 %  z_1-z_0 & 0 & -1 & 0
 %  \\
 %  -i\frac{k_0\eps_1}{k_x} & - k_y & 0 & k_y (z_1-z_2)
 %  \\
 %  0 & 1 & 0 & -(z_1-z_2)
 %  \end{vmatrix}
 %  \\
 %  ( C_4 \leftarrow C_4 + (z_1-z_2)C_2  ) &= -\frac{k_0^2\mu_1\eps_2}{k_x^2}
 %  \begin{vmatrix} 
 %  -{k_y} (z_1-z_0) & 0 & k_y & -i\frac{k_0\mu_2}{k_x} + k_y(z_1-z_2)
 %  \\
 %  z_1-z_0 & 0 & -1 & -(z_1-z_2)
 %  \\
 %  -i\frac{k_0\eps_1}{k_x} & - k_y & 0 & 0
 %  \\
 %  0 & 1 & 0 & 0
 %  \end{vmatrix}
 %  \\
 %  ( / L_4  ) &= -\frac{k_0^2\mu_1\eps_2}{k_x^2}
 %  \begin{vmatrix} 
 %  -{k_y} (z_1-z_0) & k_y & -i\frac{k_0\mu_2}{k_x} + k_y(z_1-z_2)
 %  \\
 %  z_1-z_0 & -1 & -(z_1-z_2)
 %  \\
 %  -i\frac{k_0\eps_1}{k_x} & 0 & 0
 %  \end{vmatrix}
 %  \\
 %  ( L_1  \leftarrow L_1 + k_y L_2) &= -\frac{k_0^2\mu_1\eps_2}{k_x^2}
 %  \begin{vmatrix} 
 %  0 & 0 & -i\frac{k_0\mu_2}{k_x}
 %  \\
 %  z_1-z_0 & -1 & -(z_1-z_2)
 %  \\
 %  -i\frac{k_0\eps_1}{k_x} & 0 & 0
 %  \end{vmatrix}
 %  \\
 %  ( / C_2 ) &= \frac{k_0^2\mu_1\eps_2}{k_x^2}
 %  \begin{vmatrix} 
 %  0 & -i\frac{k_0\mu_2}{k_x}
 %  \\
 %  -i\frac{k_0\eps_1}{k_x} & 0
 %  \end{vmatrix}
 %  = -\frac{k_1^2k_2^2}{k_x^4}
 %  \end{align*}
 %  \end{itshape}
 %  }