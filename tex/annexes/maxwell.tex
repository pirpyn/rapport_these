\section{Formulation des équations de Maxwell}
\label{sec:annex:maxwell_equation}

Uniquement dans cette annexe, les constantes de matériaux relatives sont dénotées par l'indice \(~_r\).

Les équations de Maxwell temporelles sont
\begin{equation}
  \left\lbrace \begin{matrix}
  \vrot \vE + \mu \ddr{t}{\vect{H}} &=& 0,
  \\
  \vrot \vect{H} - \eps \ddr{t}{\vE} &=& 0.
  \end{matrix} \right.
\end{equation}

Donc les équations de Maxwell harmoniques en convention \(e^{i\w t}\) sont
\begin{equation}
  \left\lbrace \begin{matrix}
  \vrot \vE + i\w\mu \vect{H} &=& 0,
  \\
  \vrot \vect{H} - i\w\eps \vE &=& 0.
  \end{matrix} \right.
\end{equation}

On définit les constantes relatives \(\mu_r = \mu \slash \mu_0\), \(\eps_r = \eps \slash \eps_0\), \(k_0 = \w \sqrt {\eps_0\mu_0}\), \(\eta_0=\sqrt {\mu_0/\eps_0}\).

On pose \(\vH = \eta_0 \vect{H}\).
Les équations harmoniques se réécrivent
\begin{equation}
\left\lbrace \begin{matrix}
\vrot \vE & + & ik_0\eta_0\mu_r \vect{H} &= 0,
\\
\vrot \vect{H} & - & ik_0\eta_0^{-1}\eps_r \vE &= 0,
\end{matrix} \right.
\quad
\overset{\vH = \eta_0 \vect{H}}{\Rightarrow}
\quad
\left\lbrace \begin{matrix}
\vrot \vE & + & ik_0\mu_r \vH &= 0,
\\
\vrot \vH & - & ik_0\eps_r \vE &= 0.
\end{matrix} \right.
\end{equation}

On pose \(k=k_0\nu_r\), \(\nu_r = \sqrt{\mu_r\eps_r}\), \(\eta_r = \sqrt{\mu_r \slash \eps_r}\)

\begin{equation}
\left\lbrace \begin{matrix}
\vrot \vE & + & ik\eta_0\eta_r \vect{H} &= 0,
\\
\vrot \vect{H} - ik\eta_0^{-1}\eta_r^{-1} \vE &= 0,
\end{matrix} \right.
\quad
\overset{\vH = \eta_0 \vect{H}}{\Rightarrow}
\quad
\left\lbrace \begin{matrix}
\vrot \vE & + & ik\eta_r \vH &= 0,
\\
\vrot \vH & - & ik\eta_r^{-1} \vE &= 0.
\end{matrix} \right.
\end{equation}