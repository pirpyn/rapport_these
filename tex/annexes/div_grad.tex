\section{Opérateurs gradient, divergent et rotationnel en coordonnées cartésiennes, cylindriques, sphériques et opérateurs surfaciques}
\label{sec:annexe:div_grad_rot}
\subsection{Cartésien}

    \begin{align}
        \vgrad f &= \ddr{x}{f}\vect{e_x} + \ddr{y}{f}\vect{e_y}  + \ddr{z}{f}\vect{e_z} \\
        \vdiv \vu &= \ddr{x}{u_x} + \ddr{y}{u_y} + \ddr{z}{u_z} \\
        \vrot\vu &= \left(\ddr{y}{u_z}-\ddr{z}{u_y}\right)\vect{e_x} + \left(\ddr{z}{u_x}-\ddr{x}{u_y}\right)\vect{e_y} + \left(\ddr{x}{u_y}-\ddr{y}{u_x}\right)\vect{e_z}
    \end{align}

\subsection{Cylindrique}

    \[
        (x,y,z) = (r\cos\theta,r\sin\theta,z)
    \]

    \begin{align}
        \vgrad f &= \ddr{r}{f}\vect{e_r}
        +\frac{1}{r}\ddr{\theta}{f}\vect{e_\theta} + \ddr{z}{f}\vect{e_z}
        \\
        \vdiv \vu &= \frac{1}{r}\ddr{r}{(ru_r)}+\frac{1}{r}\ddr{\theta}{u_\theta}+\ddr{z}{u_z}
        \\
        \vrot \vu &= \left(\frac{1}{r}\ddr{\theta}{u_z} - \ddr{z}{u_\theta}\right)\vect{e_r} +
        \left(\ddr{z}{u_r} - \ddr{r}{u_z}\right)\vect{e_\theta} +
        \frac{1}{r}\left(\ddr{r}{(ru_\theta)}-\ddr{\theta}{u_r}\right)\vect{e_z}
    \end{align}

\subsection{Sphérique}

    \[
        (x,y,z) = (r\cos\phi\sin\theta,r\sin\theta\sin\theta,r\cos\theta)
    \]

    \begin{align}
        \vgrad f &= \ddr{r}{f}\vect{e_r}
        +\frac{1}{r}\ddr{\theta}{f}\vect{e_\theta} + \frac{1}{r\sin\theta}\ddr{\phi}{f}\vect{e_\phi}
        \\
        \vdiv \vu &= \frac{1}{r^2}\ddr{r}{(r^2u_r)}
        + \frac{1}{r\sin\theta}\ddr{\theta}{}\left(\sin(\theta)u_\theta\right) + \frac{1}{r\sin\theta}\ddr{\phi}{u_\phi}
    \end{align}

    \begin{multline}
        \vrot \vu = \frac{1}{r\sin\theta}\left(\ddr{\theta}{}\left(\sin(\theta)u_\phi\right) - \ddr{\phi}{u_\theta}\right)\vect{e_r}\dots
        \\
        + \left(\frac{1}{r\sin\theta}\ddr{\phi}{u_r} - \frac{1}{r}\ddr{r}{(ru_\phi)} \right)\vect{e_\theta} \dots
        \\
        + \frac{1}{r}\left(\ddr{r}{(ru_\theta)}-\ddr{\theta}{u_r}\right)\vect{e_\phi}
    \end{multline}

\subsection{Opérateurs surfaciques}

    Soit \(\Omega\) un objet de surface \(\Gamma\) régulière et \(\vn\) la normale unitaire sortante de cet objet.
    Soit \(f\) et \(\vu\) une fonction et un champ de vecteurs dérivables en tout point \(\vx\) de \(\Gamma\), on rappelle de \cite{nedelec_acoustic_2001} les opérateurs différentiels surfaciques
    \begin{align}
        \vgrads{f}(\vx) &= \vgrad{f}(\vx) - \vn(\vx) (\vn(\vx) \cdot \vgrad{f}(\vx))
        \\
        \vdivs{\vu} &= \vdiv\left(\vu(\vx) - \vn(\vx) (\vn(\vx) \cdot \vu(\vx)\right)
        \\
        \vrots{\vu}(\vx) &= \vn(\vx)\left(\vn(\vx) \cdot \vrot{\vu}(\vx)\right)
    \end{align}
    Si l'on veut appliquer le rotationnel surfacique comme une application sur les champs tangentielles, les opérateurs rotationnel surfacique scalaire et rotationnel surfacique vectoriel
    \begin{align}
        \trots{\vu}(\vx) &= \vn(\vx) \cdot \vrot{\vu}(\vx)
        \\
        \tvrots{f}(\vx) &= - \vn \pvect \vgrads f(\vx)
    \end{align}