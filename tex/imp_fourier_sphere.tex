\section{Cas d'un objet sphérique}

    On exprime les équations de Maxwell dans le matériau et sans pertes de généralité, on peut réaliser une transformée de Fourier en \(\theta\) et \(\phi\) par invariance en rotation.
    Cependant, les multiplicateur de Fourier associés aux coordonnées \(\theta,\phi\) doivent être des entiers pour assurer la \(2\pi\)-périodicité. On les note \(n,m\).

    Par définition du repère sphérique \(\vE(r,\theta+\pi,\phi) =  \vE(r,\pi-\theta,\phi + \pi)\) alors

    \begin{align}
        \vE(r,\theta+\pi,\phi) &= \frac{1}{2\pi}\sum_{n\in\ZZ}\sum_{m\in\ZZ} \hat{\vE}(r,n,m) e^{-in(\theta+\pi)} e^{-im\phi}
        \\
        \vE(r,\phi-\theta,\phi+\pi)&=\frac{1}{2\pi}\sum_{n\in\ZZ}\sum_{m\in\ZZ} \hat{\vE}(r,n,m) e^{-in( \pi - \theta)} e^{-im(\phi+ \pi)}
        \\
        \vE(r,\phi-\theta,\phi+\pi)&=\frac{1}{2\pi}\sum_{n'\in\ZZ}\sum_{m\in\ZZ} \hat{\vE}(r,-n',m) e^{-in'(\theta-\pi)} e^{-im(\phi+ \pi)}
        \\
        0&=\frac{1}{2\pi}\sum_{n\in\ZZ}\sum_{m\in\ZZ} \left( \hat{\vE}(r,-n,m) - \hat{\vE}(r,n,m)e^{-i2n\pi}e^{im\pi}\right)e^{-in\theta}e^{-im\phi}
    \end{align}

    On en déduit donc une relation entre les termes \(n\)-positifs et \(n\)-négatifs de la série

    \begin{align}
        \hat{\vE}(r,-n,m) = \hat{\vE}(r,n,m)e^{im\pi} && \forall n \in\ZZ^*
    \end{align}

    On peut donc récrire la série uniquement avec \(n\in\NN\)

    \begin{align}
        \vE(r,\theta,\phi) &= \frac{1}{2\pi}\sum_{n\in\ZZ}\sum_{m\in\ZZ} \hat{\vE}(r,n,m) e^{-in\theta} e^{-im\phi}
        \\
        &= \frac{1}{2\pi}\sum_{n\in\NN}\sum_{m\in\ZZ} \hat{\vE}(r,n,m) \left( e^{-in\theta} + e^{in\theta}e^{im\pi}\right)e^{-im\phi}
    \end{align}

    Une autre propriété est 
    \begin{equation}
        \ds\ddr{\phi}{\vE}(r,0,\phi) = -\frac{1}{2\pi}\sum_{n\in\ZZ}\sum_{m\in\ZZ} im\hat{\vE}(r,n,m) e^{-im\phi}= 0
    \end{equation}

    Et la dernière est 
    \begin{equation}
        \ds\ddr{\phi}{\vE}(r,\pi,\phi) = -\frac{1}{2\pi}\sum_{n\in\ZZ}\sum_{m\in\ZZ} im\hat{\vE}(r,n,m) e^{-in\pi}e^{-im\phi}= 0
    \end{equation}

    \begin{TODO}
        Problème: les \(e^{in\theta}\) ne sont pas orthogonales sur \([0,\pi]\)
    \end{TODO}

    On réécrit les opérateurs sur les transformées de Fourier afin de déterminer des conditions à vérifié sur ces dernières.

    \begin{multline}
        \vrot \hat{\vE} = \frac{1}{r\sin\theta}\left(\left(\cos(\theta) + in\sin(\theta)\right)\hat{E}_\phi - im \hat{E}_\theta\right)\vect{e_r}\dots 
        \\
        + \left(\frac{im}{r\sin\theta}\hat{E}_r - \frac{1}{r}\ddr{r}{(r\hat{E}_\phi)} \right)\vect{e_\theta} \dots
        \\
        + \frac{1}{r}\left(\ddr{r}{(r\hat{E}_\theta)}-in\hat{E}_r\right)\vect{e_\phi}
    \end{multline}

    \begin{align}
        \vdiv \hat{\vE} &= \frac{1}{r^2}\ddr{r}{(r^2\hat{E}_r)}
        + \frac{1}{r\sin\theta}\left(\cos(\theta) + in\sin(\theta)\right)\hat{E}_\theta + \frac{im}{r\sin\theta}{\hat{E}_\phi}
    \end{align}

    De même qu'avec un cylindre, on cherche à isoler une composante.

    \begin{multline}
        \vrot \vrot \hat{\vE} = \\
        \frac{1}{r\sin\theta}\left(\left(\cos(\theta) + in\sin(\theta)\right)\frac{1}{r}\left(\ddr{r}{(r\hat{E}_\theta)}-in\hat{E}_r\right) - im \left(\frac{im}{r\sin\theta}\hat{E}_r - \frac{1}{r}\ddr{r}{(r\hat{E}_\phi)} \right)\right)\vect{e_r}\dots 
        \\
        + \left(\frac{im}{r\sin\theta}\frac{1}{r\sin\theta}\left(\left(\cos(\theta) + in\sin(\theta)\right)\hat{E}_\phi - im \hat{E}_\theta\right) - \frac{1}{r}\ddr{r}{}\left(r \frac{1}{r}\left(\ddr{r}{(r\hat{E}_\theta)}-in\hat{E}_r\right)\right) \right)\vect{e_\theta} \dots
        \\
        + \frac{1}{r}\left(\ddr{r}{}\left(r \left(\frac{im}{r\sin\theta}\hat{E}_r - \frac{1}{r}\ddr{r}{(r\hat{E}_\phi)} \right)\right)-in\frac{1}{r\sin\theta}\left(\left(\cos(\theta) + in\sin(\theta)\right)\hat{E}_\phi - im \hat{E}_\theta\right)\right)\vect{e_\phi}
    \end{multline}

    \begin{TODO}
      Ce cas est plus complexe: les relations de périodicités donnent une série de Fourier en \(\theta,\phi\) mais il faut rajouter aussi les relations en \(u(r,\theta+\pi,\phi) = u(r,\pi-\theta,\phi + \pi)\), \(u(r,0,\phi) = u(r,0,\phi')\) et \(u(r,\pi,\phi) = u(r,\pi,\phi')\). De plus, les \(e^{i\theta}\) ne sont pas orthogonale sur \([0,\pi]\) ce qui va aussi complexifier les calculs
    \end{TODO}
