\section{Cas d'un objet sphérique}

    On exprime les équations de Maxwell dans le matériau et sans pertes de généralité, on peut réaliser une transformée de Fourier en $\theta$ et $\phi$ par invariance en rotation. Cependant, les multiplicateur de Fourier associés aux coordonnées $\theta,\phi$ doivent être des entiers pour assurer la périodicité. On les note $n,m$.

    \begin{multline}
        \vrot \v{E} = \frac{1}{r\sin\theta}\left(\left(\cos(\theta) + in\sin(\theta)\right)E_\phi - im E_\theta\right)\v{e_r}\dots 
        \\
        + \left(\frac{im}{r\sin\theta}E_r - \frac{1}{r}\ddr{r}{(rE_\phi)} \right)\v{e_\theta} \dots
        \\
        + \frac{1}{r}\left(\ddr{r}{(rE_\theta)}-inE_r\right)\v{e_\phi}
    \end{multline}

    \begin{align}
        \vdiv \v{E} &= \frac{1}{r^2}\ddr{r}{(r^2E_r)}
        + \frac{1}{r\sin\theta}\left(\cos(\theta) + in\sin(\theta)\right)E_\theta + \frac{im}{r\sin\theta}{E_\phi}
    \end{align}

    De même qu'avec un cylindre, on cherche à isoler une composante.

    \begin{multline}
        \vrot \vrot \v{E} = \\
        \frac{1}{r\sin\theta}\left(\left(\cos(\theta) + in\sin(\theta)\right)\frac{1}{r}\left(\ddr{r}{(rE_\theta)}-inE_r\right) - im \left(\frac{im}{r\sin\theta}E_r - \frac{1}{r}\ddr{r}{(rE_\phi)} \right)\right)\v{e_r}\dots 
        \\
        + \left(\frac{im}{r\sin\theta}\frac{1}{r\sin\theta}\left(\left(\cos(\theta) + in\sin(\theta)\right)E_\phi - im E_\theta\right) - \frac{1}{r}\ddr{r}{}\left(r \frac{1}{r}\left(\ddr{r}{(rE_\theta)}-inE_r\right)\right) \right)\v{e_\theta} \dots
        \\
        + \frac{1}{r}\left(\ddr{r}{}\left(r \left(\frac{im}{r\sin\theta}E_r - \frac{1}{r}\ddr{r}{(rE_\phi)} \right)\right)-in\frac{1}{r\sin\theta}\left(\left(\cos(\theta) + in\sin(\theta)\right)E_\phi - im E_\theta\right)\right)\v{e_\phi}
    \end{multline}
