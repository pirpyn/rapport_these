\section{Fonctions de Bessel et Hankel}

Cette anneze référence les résultats de \cite{abramowitz_handbook_1964}.

Soit \(J_n(z)\) la fonction de Bessel de degré \(n\) de 1\iere espèce, \(Y_n(z)\) la fonction de Bessel de degré \(n\) de 2\ieme espèce, solutions de

\begin{equation}
    z^2 \ddp[2]{z}{u}(z) + z\ddp{z}{u}(z)+\left(z^2-n^2\right)u(z) = 0
\end{equation}

La fonctions de Hankel de degré \(n\) de 1\iere et 2\ieme type sont définis comme
\begin{align}
    H_n^{(1)}(z) &= J_n(z) + iY_n(z)\\
    H_n^{(2)}(z) &= J_n(z) - iY_n(z)
\end{align}

On remarque que \(\forall (z_1,z_2) \in \CC^2\)

\begin{equation}
\begin{aligned}
z_1 J_n(z) + z_2 Y_n(z) 
&= ( z_1 - i z_2 ) J_n(z) + iz_2 H_n^{(2)}(z) \\
&= ( z_1 + i z_2 ) J_n(z) - iz_2 H_n^{(1)}(z) \\
&= z_1 H_n^{(2)}(z) + ( z_2 + i z_1 ) Y_n(z) \\
&= z_1 H_n^{(1)}(z) + ( z_2 - i z_1 ) Y_n(z) \\
&= \frac{z_1-iz_2}{2}H_n^{(1)}(z) + \frac{z_1+iz_2}{2}H_n^{(2)}(z)
\end{aligned}
\label{eq:annex:bessel:equiv_bessel}
\end{equation}