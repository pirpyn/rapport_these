\section{Analyse de Fourier de l'opérateur d'impédance}
Soit $\OO$ un domaine fermé, bornée, de frontière régulière. Supposons que ces champs soient $L^2$ en espace et en temps: $(\vE(\v{x},t),\vH(\v{x},t)) \in L^2(\OO \times \RR_+) \cap L^2(\OO^c\times\RR_+)$ et vérifient les équations de Maxwell:
\begin{equation}
    \left\lbrace 
    \begin{matrix}
    \vrot \vE = \mu \ddr{t}{\vH} \\
    \vrot \vH = -\eps \ddr{t}{\vE}
    \end{matrix}
    \right.
\end{equation}

Puisque ces champs $\vE(\v{x},t),\vH(\v{x},t)$ sont $L^2$, on peut définir leurs transformées de Fourier $\hat{\vE}(\v{k},\w), \hat{\vH}(\v{k},\w)$ (\cite[Théorème de Plancherel, p.~153]{yosida_functional_1995}) telle que

\begin{equation}
    \hat{\vE} (\v{k},\w) = \frac{1}{\sqrt{2\pi}^n}\int_{\RR^n} e^{-i(\v{k} \cdot \v{x}+\w t)}\vE(\v{x},t) dxdt\,, \quad dx = \prod\limits_{i=1}^n dx_i
\end{equation}

La méthode pour trouver une expression de l'opérateur d'impédance est la suivante.
\begin{itemize}
\item Faire une transformée de Fourier partielle des champs, dépendante de la géométrie.
\item Réécrire le système d'équation de Maxwell simplifié.
\item Obtenir des EDO simple à résoudre sur les composantes des champs.
\item Utiliser les conditions limites pour les déduire 
\item En déduire l'opérateur d'impédance en Fourier.
\end{itemize}

Dans ce qui suit, on réalise au moins une transformée partielle en temps. La variable de Fourier associée est $\w$, et donc l'opérateur $\ddr{t}{~}$ est remplacé par $i\w$.

\begin{tcolorbox}
On ne différencie pas les champs $\vE,\vH$ de leurs transformées de Fourier $\hat \vE, \hat \vH$.
\end{tcolorbox}

On va donc utiliser le système d'équations de Maxwell harmonique:

\begin{equation}
    \left\lbrace 
    \begin{matrix}
    \vrot \vE = i \omega \mu \vH \\
    \vrot \vH = -i \omega \eps \vE
    \end{matrix}
    \right.
    \label{eq:imp_fourier:intro:maxwell_harmonique}
\end{equation}


\subsection{Forme générale des solutions pour un plan infini}

Ce cas est très bien documenté (\cite{senior_approximate_1995},\cite{hoppe_impedance_1995}) et pose la méthodologie à adopter pour les objets courbes. 

Dans un premier temps, on peut sans perte de généralités faire une rotation du repère pour avoir le plan orthogonal à $\v e_3$. Comme il est infini dans les directions $\v e_1, \v e_2$ et que le matériau est homogène isotrope, on utilise la transformée partielle en $x_1, x_2$ seulement.

\renewcommand{\x}{{\v e_1}}
\renewcommand{\y}{{\v e_2}}
\renewcommand{\z}{{\v e_3}}
\renewcommand{\peps}{{\eps}}
\renewcommand{\pmu}{{\mu}}
\begin{figure}[h!]
\centering
\begin{tikzpicture}
\coordinate (l) at (-2,0);
\coordinate (r) at (2,0);
\coordinate (m) at ($(r)!0.5!(l)$);

\fill [lightgray] (l) rectangle (2,1) ;
\foreach \t in {-2.2,-2.1,...,2} {
\draw plot[domain=0:0.3] (\t + \x, -\x, 0);
}
\coordinate (ll) at (-2.2,-0.3);
\fill [white] (ll) rectangle (l) ;
\coordinate (rr) at (2,-0.31);
\fill [white] (rr) rectangle (2.3,0) ;


\coordinate (n) at (-4,0);

\coordinate (lt) at (-2,1);
\coordinate (rt) at (2,1);
\coordinate (mt) at ($(rt)!0.5!(lt)$);

\draw (l) -- (r) node [at end,right] {$z = -d$};
\draw (lt)  -- (rt) node [at end,right] {$z = 0$};

\draw (lt) node [above right] {$\eps_0,\mu_0$};
\draw ($(l)!0.5!(lt)$) node [right] {$\eps_1,\mu_1$};

\draw [->] (n) -- ++(0,1) node [at end, right] {$z$};
\draw [->] (n) -- ++(1,0) node [at end, right] {$x$};

\draw (n) circle(0.1cm) node [below=0.1cm] {$y$};
\draw (n) +(135:0.1cm) -- +(315:0.1cm);
\draw (n) +(45:0.1cm) -- +(225:0.1cm);

%\draw [->>,thick] (lt) ++ (1,1) -- (mt) ;


\end{tikzpicture}
\end{figure}

En explicitant par composante l'opérateur $\vrot$ , le problème \eqref{eq:imp_fourier:intro:maxwell_harmonique} s'écrit  
\begin{align*}
    \left\lbrace 
    \begin{matrix}
    ik_2 E_3  - \ddr{x_3}{E_ 2} = i \w \mu H_1 \\
    \ddr{x_3}{E_1} - ik_1 E_3 = i\w \mu H_2 \\
    ik_1 E_2 - ik_2 E_1 = i\w \mu H_3 \\
    \end{matrix}
    \right. \quad 
    \left\lbrace 
    \begin{matrix}
    ik_2 H_3  - \ddr{x_3}{H_ 2} = -i \w \eps E_1 \\
    \ddr{x_3}{H_1} - ik_1 H_3 = -i\w \eps E_2 \\
    ik_1 H_2 - ik_2 H_1 = -i\w \eps E_3 \\
    \end{matrix}
    \right.
\end{align*}

Les composantes normales se déduisant des composantes tangentielles, on résout l'EDO matricielle à coefficients constants 
suivante $\ddr{x_3}{}X = M X$ où

\begin{equation}
    X = 
    \begin{bmatrix}
    E_1 \\ 
    E_2 \\ 
    H_1 \\ 
    H_2 \\
    \end{bmatrix}\,
    M = \begin{bmatrix}
    0 & 0 & i\frac{k_1k_2}{\w\eps} & i\left(\w\mu - \frac{k_1^2}{\w\eps}\right)\\
    0 & 0 & -i\left(\w\mu - \frac{k_2^2}{\w\eps}\right) & -i\frac{k_1k_2}{\w\eps}\\
    -i\frac{k_1k_2}{\w\mu} & -i\left(\w\eps - \frac{k_1^2}{\w\mu}\right) & 0 & 0 \\
    i\left(\w\eps - \frac{k_2^2}{\w\mu}\right) & i\frac{k_1k_2}{\w\mu} & 0 & 0 \\
    \end{bmatrix}
\end{equation}

Pour résoudre cette EDO, nous allons chercher les vecteurs propres $V_i$ et les valeurs propres $\lambda_i$ associées de ce système. En effet, une solution générale de ce système s'écrit
\begin{equation}
    X(x_3)= \sum\limits_{i=1}^{4}c_i e^{\lambda_i x_3} V_i \quad , c_i \in \CC
\end{equation}
On pose 
\begin{equation}
    A = \begin{bmatrix}
        i\frac{k_1k_2}{i\w\eps} & i\left(\w\mu - \frac{k_1^2}{\w\eps}\right) \\
        -i\left(\w\mu - \frac{k_2^2}{\w\eps}\right) & -i\frac{k_1k_2}{i\w\eps} \\
    \end{bmatrix}
    \quad
    B = \begin{bmatrix}
        -i\frac{k_1k_2}{\w\mu} & -i\left(\w\eps - \frac{k_1^2}{\w\mu}\right) \\
        i\left(\w\eps - \frac{k_2^2}{\w\mu}\right) & i\frac{k_1k_2}{\w\mu} \\
    \end{bmatrix}
\end{equation}
Le déterminant de $M-\lambda I$ est
\begin{align*}
    \det(M-\lambda I) &= 
    \begin{vmatrix}
        -\lambda I & A \\
        B & -\lambda I
    \end{vmatrix}
        = \frac{\det(- \lambda I - B(-\lambda I)^{-1} A)}{\det((-\lambda I)^{-1})} \\
        &= \det(\lambda^2 I - BA) \\
        &= (\lambda^2 + (\w^2\eps\mu - k_1^2 -k_2^2))^2
\end{align*}
On note alors 
\begin{equation}
k_3=\sqrt{\w^2\eps\mu - k_1^2 -k_2^2}
\end{equation}

Les valeurs propres sont alors 
\begin{equation}
    \lambda_\pm = \pm i k_3
\end{equation}
Les espaces propres associés sont de dimension 2, on a 

\begin{align}
\Ker(M-\lambda_+I)=\Vect{V_+;W_+} \\
    V_+ = 
    \begin{bmatrix}
    \lambda_+ \\
        0 \\
        -i\frac{k_1k_2}{\w\mu} \\
        i\left(\w\eps - \frac{k_2^2}{\w\mu}\right) \\
    \end{bmatrix}
    \,
    W_+ = 
        \begin{bmatrix}
        0 \\
        \lambda_+ \\
        -i\left(\w\eps - \frac{k_1^2}{\w\mu}\right) \\
        i\frac{k_1k_2}{\w\mu} \\
    \end{bmatrix}
\end{align}

\begin{align}
\Ker(M-\lambda_-I)=\Vect{V_-;W_-}\\
    V_- = 
    \begin{bmatrix}
        \lambda_- \\
        0 \\
        -i\frac{k_1k_2}{\w\mu} \\
        i\left(\w\eps - \frac{k_2^2}{\w\mu}\right) \\
    \end{bmatrix}
    \,
    W_- = 
    \begin{bmatrix}
        0 \\
        \lambda_- \\
        -i\left(\w\eps - \frac{k_1^2}{\w\mu}\right) \\
        i\frac{k_1k_2}{\w\mu} \\
    \end{bmatrix}
\end{align}

On a donc une solution générale du système 
\begin{equation}
    X(x_3) = c_1e^{\lambda_+ x_3}V_+  + c_2e^{\lambda_+ x_3}W_+ + c_3e^{\lambda_- x_3}V_- +c_4e^{\lambda_- x_3}W_- \quad c_i \in \CC
\end{equation}

On exprime les champs $\vE_t(x)$ et $\v{e_3} \times \vH_t(x)$ car ce sont des quantités qui nous intéresse:

\begin{align}
    \begin{bmatrix}
        E_1(x)\\
        E_2(x)\\
    \end{bmatrix}
    &=
    \begin{bmatrix}
        c_1 e^{\lambda_+ x} \lambda_{+} + c_3 e^{\lambda_- x} \lambda_{-} \\
        c_2 e^{\lambda_+ x} \lambda_{+} + c_4 e^{\lambda_- x} \lambda_{-}
    \end{bmatrix}\\
    &=ik_3\left( e^{ik_3 x}
    \begin{bmatrix}
        c_1 \\
        c_2
    \end{bmatrix}
    -e^{-ik_3 x}
    \begin{bmatrix}
        c_3 \\
        c_4
    \end{bmatrix}
    \right)
    \label{eq:imp_fourier:plan:generale_E}
\end{align}

\begin{align}
    \begin{bmatrix}
        -H_2(x)\\
        H_1(x)\\
    \end{bmatrix}
    &=
    \begin{bmatrix}
        -i\left(\w\eps - \frac{k_2^2}{\w\mu}\right) \left( c_1 e^{ik_3 x} + c_3 e^{-ik_3 x} \right) - i\frac{k_1k_2}{\w\mu} \left( c_2 e^{ik_3 x} + c_4 e^{-ik_3 x} \right)
        \\
        -i\frac{k_1k_2}{\w\mu} \left( c_1 e^{ik_3 x} + c_3 e^{-ik_3 x} \right) - i\left(\w\eps - \frac{k_1^2}{\w\mu}\right)\left( c_2 e^{ik_3 x} + c_4 e^{-ik_3 x} \right)
    \end{bmatrix} \\
    &=-i
    \begin{bmatrix}
    \left(\w\eps - \frac{k_2^2}{\w\mu}\right) & \frac{k_1k_2}{\w\mu}
    \\
    \frac{k_1k_2}{\w\mu} & \left(\w\eps - \frac{k_1^2}{\w\mu}\right) 
    \end{bmatrix}
    \left(
        e^{ik_3 x}
        \begin{bmatrix}
            c_1 \\
            c_2
        \end{bmatrix}
        +e^{-ik_3 x}
        \begin{bmatrix}
            c_3 \\
            c_4
        \end{bmatrix}
    \right)
    \label{eq:imp_fourier:plan:generale_H}
\end{align}

Notons
\begin{align}
    \mat{C} &=
    \begin{bmatrix}
        \left(\w\eps-\frac{k_2^2}{\w\mu}\right) & \frac{k_1k_2}{\w\mu}\\
        \frac{k_1k_2}{\w\mu} & \left(\w\eps-\frac{k_1^2}{\w\mu}\right)
    \end{bmatrix}
\end{align}

Comme $k_3$ est complexe et non nul, alors $\det(\mat{C}) = k_3^2\frac{\eps}{\mu}=\frac{k_3^2}{\eta^2}$ n'est jamais nul. Nous verrons que cette propriété sera importante.

\subsection{Solutions particulières du plan infini recouvert d'une couche de matériau homogène isotrope}

Nous utilisons la condition limite 
\begin{equation}
    \begin{bmatrix}
        E_1(-d)\\
        E_2(-d)\\
    \end{bmatrix}
    =
    \begin{bmatrix}
        0\\
        0\\
    \end{bmatrix}
\end{equation}

De \eqref{eq:imp_fourier:plan:generale_E}, on déduit

\begin{align}
    \begin{bmatrix}
        c_1 \\
        c_2
    \end{bmatrix}
    = e^{2ik_3 d}
    \begin{bmatrix}
        c_3 \\
        c_4
    \end{bmatrix}
\end{align}

On définit l'opérateur d'impédance la matrice $\mat Z$ tel que 
\begin{equation}
    \vE_t(0) = \mat Z \left(\v{e_3} \times \vH_t(0)\right)
\end{equation}

De ce qui précède on déduit que,

\begin{align}
    \begin{bmatrix}
        E_1(0)\\
        E_2(0)\\
    \end{bmatrix}
    &=ik_3\left( e^{i2k_3 d} -1 \right)
    \begin{bmatrix}
        c_3 \\
        c_4
    \end{bmatrix} \\
    \begin{bmatrix}
        -H_2(0)\\
        H_1(0)\\
    \end{bmatrix}
    & = - i\left(e^{i2k_3 d} +1 \right)
    \mat{C}
    \begin{bmatrix}
    c_3 \\
    c_4
    \end{bmatrix}
\end{align}

On déduit donc que
\begin{align}
    \mat{Z} &=  - \frac{\eta^2}{k_3}\frac{e^{i2dk_3} -1}{e^{i2dk_3} +1}  
        \begin{bmatrix}
           \left(\w\eps-\frac{k_1^2}{\w\mu}\right)  & -\frac{k_1k_2}{\w\mu}\\
            -\frac{k_1k_2}{\w\mu} &  \left(\w\eps-\frac{k_2^2}{\w\mu}\right)
        \end{bmatrix}\\
    &= i\eta\frac{\tan\left(k_3d\right)}{kk_3}
        \begin{bmatrix}
           k^2-k_1^2  & -k_1k_2\\
            -k_1k_2 & k^2-k_2^2\\
        \end{bmatrix}
\end{align}


\subsection{Solutions particulières du plan infini recouvert de couches de matériaux homogènes isotropes}

En premier lieu, une empilement donné peut toujours se ramener à un empilement à deux couches, où l'on impose $z=-d$ la condition:
\begin{equation}
    \begin{bmatrix}
        E_1(-d)\\
        E_2(-d)\\
    \end{bmatrix}
    =
    \mat {Z_{d}} 
    \begin{bmatrix}
        -H_2(-d)\\
        H_1(-d)\\
    \end{bmatrix}
\end{equation}

D'après \eqref{eq:imp_fourier:plan:generale_E} et \eqref{eq:imp_fourier:plan:generale_H}, on déduit que

\begin{equation}
    \begin{bmatrix}
        E_1(-d)\\
        E_2(-d)\\
    \end{bmatrix}
    = ik_3\left( e^{-ik_3 d}
    \begin{bmatrix}
        c_1 \\
        c_2
    \end{bmatrix}
    -e^{ik_3 d}
    \begin{bmatrix}
        c_3 \\
        c_4
    \end{bmatrix}
    \right)
\end{equation}

\begin{equation}
    \begin{bmatrix}
        -H_2(-d)\\
        H_1(-d)\\
    \end{bmatrix}
    =-i
    \mat{C}
    \left(
        e^{-ik_3 d}
        \begin{bmatrix}
            c_1 \\
            c_2
        \end{bmatrix}
        +e^{ik_3 d}
        \begin{bmatrix}
            c_3 \\
            c_4
        \end{bmatrix}
    \right)
\end{equation}

\begin{equation}
    ik_3\left( e^{-ik_3 d}
    \begin{bmatrix}
        c_1 \\
        c_2
    \end{bmatrix}
    -e^{ik_3 d}
    \begin{bmatrix}
        c_3 \\
        c_4
    \end{bmatrix}
    \right)
    =-i\mat{Z_d}\mat{C}
    \left(
        e^{-ik_3 d}
        \begin{bmatrix}
            c_1 \\
            c_2
        \end{bmatrix}
        +e^{ik_3 d}
        \begin{bmatrix}
            c_3 \\
            c_4
        \end{bmatrix}
    \right)
\end{equation}

\begin{equation}
    \left(k_3\mat{I} + \mat{Z_d}\mat{C}\right)
    \begin{bmatrix}
        c_1 \\
        c_2
    \end{bmatrix}
    = e^{i2k_3 d} \left(k_3\mat{I} - \mat{Z_d}\mat{C}\right)
    \begin{bmatrix}
        c_3 \\
        c_4
    \end{bmatrix}
\end{equation}

\TODO{Prouver que $k_3\mat{I} + \mat{Z_d}\mat{C}$ est inversible}

\begin{equation}
    \begin{bmatrix}
        c_1 \\
        c_2
    \end{bmatrix}
    = e^{i2 k_3 d} \left(k_3\mat{I} + \mat{Z_d}\mat{C}\right)^{-1}\left(k_3\mat{I} - \mat{Z_d}\mat{C}\right)
    \begin{bmatrix}
        c_3 \\
        c_4
    \end{bmatrix}
\end{equation}

\begin{align}
    \begin{bmatrix}
        E_1(0)\\
        E_2(0)\\
    \end{bmatrix}
    &=ik_3\left(
    \begin{bmatrix}
        c_1 \\
        c_2
    \end{bmatrix}
    -
    \begin{bmatrix}
        c_3 \\
        c_4
    \end{bmatrix}
    \right)\\
    &=ik_3\left(e^{i2 k_3 d} \left(k_3\mat{I} + \mat{Z_d}\mat{C}\right)^{-1}\left(k_3\mat{I} - \mat{Z_d}\mat{C}\right) - \mat{I} \right)
    \begin{bmatrix}
        c_3 \\
        c_4
    \end{bmatrix}
\end{align}

Et alors l'opérateur d'impédance $\mat{Z}$ est tel que

\begin{align}
    \mat{Z}
    &=-k_3\left(e^{i2 k_3 d} \left(k_3\mat{I} + \mat{Z_d}\mat{C}\right)^{-1}\left(k_3\mat{I} - \mat{Z_d}\mat{C}\right) - \mat{I} \right)
    \left( \mat{C} \left( e^{i2 k_3 d} \left(k_3\mat{I} + \mat{Z_d}\mat{C}\right)^{-1}\left(k_3\mat{I} - \mat{Z_d}\mat{C}\right) + \mat{I} \right)\right)^{-1}
\end{align}


%%%%%%%%%%%%%%%%%%%%%%%%%%%%%%%%%%
%%%%%%%%%%%%%%%%%%%%%%%%%%%%%%%%%%

%% CYLINDRE

%%%%%%%%%%%%%%%%%%%%%%%%%%%%%%%%%%
%%%%%%%%%%%%%%%%%%%%%%%%%%%%%%%%%%




\subsection{Forme générale des solutions pour un cylindre}

\newcommand{\mr}{r}
\newcommand{\mt}{\theta}
\newcommand{\mz}{z}

\begin{figure}[htpb]
\centering
\begin{tikzpicture}
\coordinate (mat) at (0,-1.5);
\coordinate (vide) at (0,-2);
\coordinate (c) at (0,0);

\fill [lightgray] (c) circle (2);
\fill [white] (c) circle (1.5);
\fill [pattern=north east lines] (c) circle (1.5);
\draw (c) circle (2);
\draw (c) circle (1.5);


\coordinate (n) at (0,2);

\draw (vide) node [below] {$\eps_0,\mu_0$};
\draw (mat) node [below] {$\peps,\pmu$};

% Axess
\draw [->] (n) -- ++(0,1) node [at end, right] {$\mr$};
\draw [->] (n) -- ++(1,0) node [at end, right] {$\mt$};

\draw (n) ++(0.2,0.2) circle(0.1cm) node [above=0.1cm] {$\mz$};
\draw (n) ++(0.2,0.2) +(135:0.1cm) -- +(315:0.1cm);
\draw (n) ++(0.2,0.2) +(45:0.1cm) -- +(225:0.1cm);

%\draw [->>,thick] (lt) ++ (1,1) -- (mt) ;


\end{tikzpicture}
\end{figure}

En coordonnée cylindrique $(r,\theta,z)$, on a
\begin{equation}
\vrot \vE = \left(\frac{1}{r}\ddr{\theta}{E_z} - \ddr{z}{E_\theta}\right)\v{e_r} + 
\left(\ddr{z}{E_r} - \ddr{r}{E_z}\right)\v{e_\theta} +
\frac{1}{r}\left(\ddr{r}{(rE_\theta)}-\ddr{\theta}{E_r}\right)\v{e_z}
\end{equation}

Sans pertes de généralité, on peut réaliser une transformée de Fourier en $z$ par invariance en translation et en $\theta$ par invariance en rotation. Cependant, le multiplicateur de Fourier associé à la coordonnée $\theta$ doit être un entier pour assurer la périodicité. On le note $n$.

\begin{equation}
\vrot \vE = i\left(\frac{n}{r}E_z - k_zE_\theta\right)\v{e_r} + 
\left(ik_zE_r - \ddr{r}{E_z}\right)\v{e_\theta} +
\frac{1}{r}\left(\ddr{r}{(rE_\theta)}-inE_r\right)\v{e_z}
\end{equation}

On remarque que dans ce cas, la méthode utilisée pour le plan aboutie à une EDO à coefficients non constant de type $r\ddr{r}{X}(r) = M(r)X(r)$.

On ne peut pas exprimer la solution avec les valeurs et vecteurs propres de la matrice. Nous allons donc trouver une équation de Bessel en développant le système de Maxwell.

\begin{equation}
\vrot \vrot \vE = \w^2\eps\mu \vE
\end{equation}

\begin{multline}
\vrot \vrot \vE = \dots\\
i\left(\frac{n}{r^2}\left(\ddr{r}{(rE_\theta)} - inE_r\right) - k_z\left(ik_zE_r - \ddr{r}{E_z}\right)\right)    \v{e_r} \dots\\ 
+ \left(-k_z\left(\frac{n}{r}E_z - k_zE_\theta\right) -\ddr{r}{}\left(\frac{1}{r}\left(\ddr{r}{(rE_\theta)}-inE_r\right)\right)\right)    \v{e_\theta} \dots\\
+ \frac{1}{r}\left(\ddr{r}{} \left(r\left(ik_zE_r - \ddr{r}{E_z}\right)\right) + n \left(\frac{n}{r}E_z - k_zE_\theta\right)\right) \v{e_z}
\end{multline}

\begin{equation}
\left\lbrace
\begin{array}{ccc}
-\left(\w^2\eps\mu -\frac{n^2}{r^2}  - k_z^2\right)E_r  +i\frac{n}{r^2}\ddr{r}{(rE_\theta)}  +k_z\ddr{r}{E_z} & = & 0\\
in\ddr{r}{}\left(\frac{E_r}{r}\right) -\left(\w^2\eps\mu - k_z^2\right)E_\theta + \ddr{r}{}\left(\frac{1}{r}\ddr{r}{(rE_\theta)}\right)  - n\frac{k_z}{r}E_z & = & 0\\
i\frac{k_z}{r}\ddr{r}{(rE_r)}  - n\frac{k_z}{r}E_\theta  -\left(\w^2\eps\mu - \frac{n^2}{r^2} \right)E_z - \frac{1}{r}\ddr{r}{}\left(r\ddr{r}{E_z}\right) & = & 0
\end{array}
\right.
\end{equation}

%Une solution générale de ce problème sur $\left]0,+\infty\right[$ nécessite de connaître la primitive de $\frac{M(r)}{r}$


\TODO{Passer en cylindrique. Développer $\vrot \vrot$ et trouver des équations de Bessel pour $E_z$, $H_z$. Déduire $E_r$,$H_r$,$E_\theta$,$H_\theta$. Trouver $Z$}
