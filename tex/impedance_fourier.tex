\section{Analyse de Fourier de l'opérateur d'impédance}
Soit $\Omega$ un domaine fermé, bornée, de frontière régulière. Supposons que ces champs soient $L^2$ en espace et en temps: $(\E,\H) \in L^2(\Omega \times \R_+) \cap L^2(\Omega^c\times\R_+)$ et vérifient les équations de Maxwell:
\[
    \left\lbrace 
    \begin{matrix}
    \rot \E = \mu \dr{t}{\H} \\
    \rot \H = -\eps \dr{t}{\E}
    \end{matrix}
    \right.
\]

Puisque ces champs $\E(\v{x}),\H(\v{x})$ sont $L^2$, on peut définir leurs transformées de Fourier $\hat{\E}(\v{k}), \hat{\H}(\v{k})$ (\cite[Théorème de Plancherel, p.~153]{yosida_functional_1995}) telle que
\[
    \hat{\E} (\v{k}) = \frac{1}{\sqrt{2\pi}^n}\int_{\R^n} e^{-i(\v{k} \cdot \v{x})}\E(\v{x}) dx\,, \quad dx = \prod\limits_{i=1}^n dx_i
\]

La méthode pour trouver une expression de l'opérateur d'impédance est la suivante.
\begin{itemize}
\item Faire une transformée partielle des champs dépendante de la géométrie.
\item Obtenir le système d'équations en utilisant les relations multiplicative des dérivées de transformée de Fourier.
\item Obtenir des EDO simple à résoudre sur les composantes des champs.
\item Utiliser les conditions limites pour les déduire ainsi que l'opérateur d'impédance
\end{itemize}

Dans ce qui suit, on réalise au moins une transformée partielle en temps. La variable de Fourier associée est $\omega$, et donc l'opérateur $\dr{t}{~}$ est remplacé par $i\omega$. On étudie donc le problème (les champs $\E,\H$ qui suivent sont les transformées partielles de Fourier en temps par abus de notation)
\[
    \left\lbrace 
    \begin{matrix}
    \rot \E = i \omega \mu \H \\
    \rot \H = -i \omega \eps \E
    \end{matrix}
    \right.
\]
\subsection{Le cas plan infini homogène isotrope}

Dans un premier temps, on peut sans perte de généralités faire une rotation du repère pour avoir le plan orthogonal à $\v e_3$. Comme il est infini dans les directions $\v e_1, \v e_2$ et que le matériau est homogène isotrope, on utilise la transformée partielle en $x_1, x_2$ seulement.

On travaille donc avec 
\begin{align*}
\hat{\E} (k_1,k_2,x_3) &= \frac{1}{\sqrt{2\pi}^2}\int_{\R^2} e^{-i(k_1x_1+k_2x_2 )}\E(x_1,x_2,x_3) dx_1dx_2\\
\hat{\H} (k_1,k_2,x_3) &= \frac{1}{\sqrt{2\pi}^2}\int_{\R^2} e^{-i(k_1x_1+k_2x_2 )}\H(x_1,x_2,x_3) dx_1dx_2
\end{align*}
Encore une fois, par abus de notation, on ne différencie pas les champs $\E,\H$ de leurs transformées $\hat  \E, \hat \H$.
En explicitant par composante l'opérateur $rot$, le problème est 
\begin{align*}
    \left\lbrace 
    \begin{matrix}
    ik_2 E_3  - \dr{x_3}{E_ 2} = i \omega \mu H_1 \\
    \dr{x_3}{E_1} - ik_1 E_3 = i\omega \mu H_2 \\
    ik_1 E_2 - ik_2 E_1 = i\omega \mu H_3 \\
    \end{matrix}
    \right. \\
    \left\lbrace 
    \begin{matrix}
    ik_2 H_3  - \dr{x_3}{H_ 2} = -i \omega \eps E_1 \\
    \dr{x_3}{H_1} - ik_1 H_3 = -i\omega \eps E_2 \\
    ik_1 H_2 - ik_2 H_1 = -i\omega \eps E_3 \\
    \end{matrix}
    \right.
\end{align*}

On veut donc résoudre une EDO matricielle à coefficients constant:

\[
    \dr{x_3}
\]