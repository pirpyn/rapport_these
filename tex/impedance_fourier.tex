\section{Analyse de Fourier de l'opérateur d'impédance}
Soit $\OO$ un domaine fermé, bornée, de frontière régulière. Supposons que ces champs soient $L^2$ en espace et en temps: $(\vE,\vH) \in L^2(\OO \times \RR_+) \cap L^2(\OO^c\times\RR_+)$ et vérifient les équations de Maxwell:
\begin{equation}
    \left\lbrace 
    \begin{matrix}
    \vrot \vE = \mu \ddr{t}{\vH} \\
    \vrot \vH = -\eps \ddr{t}{\vE}
    \end{matrix}
    \right.
\end{equation}

Puisque ces champs $\vE(\v{x}),\vH(\v{x})$ sont $L^2$, on peut définir leurs transformées de Fourier $\hat{\vE}(\v{k}), \hat{\vH}(\v{k})$ (\cite[Théorème de Plancherel, p.~153]{yosida_functional_1995}) telle que
\begin{equation}
    \hat{\vE} (\v{k}) = \frac{1}{\sqrt{2\pi}^n}\int_{\RR^n} e^{-i(\v{k} \cdot \v{x})}\vE(\v{x}) dx\,, \quad dx = \prod\limits_{i=1}^n dx_i
\end{equation}

La méthode pour trouver une expression de l'opérateur d'impédance est la suivante.
\begin{itemize}
\item Faire une transformée partielle des champs dépendante de la géométrie.
\item Obtenir le système d'équations en utilisant les relations multiplicative des dérivées de transformée de Fourier.
\item Obtenir des EDO simple à résoudre sur les composantes des champs.
\item Utiliser les conditions limites pour les déduire ainsi que l'opérateur d'impédance
\end{itemize}

Dans ce qui suit, on réalise au moins une transformée partielle en temps. La variable de Fourier associée est $\omega$, et donc l'opérateur $\ddr{t}{~}$ est remplacé par $i\omega$. On étudie donc le problème (les champs $\vE,\vH$ qui suivent sont les transformées partielles de Fourier en temps par abus de notation)
\begin{equation}
    \left\lbrace 
    \begin{matrix}
    \vrot \vE = i \omega \mu \vH \\
    \vrot \vH = -i \omega \eps \vE
    \end{matrix}
    \right.
\end{equation}
\subsection{Le cas d'un objet plan infini recouvert d'une couche de matériau homogène isotrope}
Ce cas est très bien documenté (\cite{senior_approximate_1995},\cite{hoppe_impedance_1995}) et pose la méthodologie à adopter pour les objets courbes. 

Dans un premier temps, on peut sans perte de généralités faire une rotation du repère pour avoir le plan orthogonal à $\v e_3$. Comme il est infini dans les directions $\v e_1, \v e_2$ et que le matériau est homogène isotrope, on utilise la transformée partielle en $x_1, x_2$ seulement.

\renewcommand{\x}{{\v e_1}}
\renewcommand{\y}{{\v e_2}}
\renewcommand{\z}{{\v e_3}}
\renewcommand{\peps}{{\eps}}
\renewcommand{\pmu}{{\mu}}
\begin{figure}[h!]
\centering
\begin{tikzpicture}
\coordinate (l) at (-2,0);
\coordinate (r) at (2,0);
\coordinate (m) at ($(r)!0.5!(l)$);

\fill [lightgray] (l) rectangle (2,1) ;
\foreach \t in {-2.2,-2.1,...,2} {
\draw plot[domain=0:0.3] (\t + \x, -\x, 0);
}
\coordinate (ll) at (-2.2,-0.3);
\fill [white] (ll) rectangle (l) ;
\coordinate (rr) at (2,-0.31);
\fill [white] (rr) rectangle (2.3,0) ;


\coordinate (n) at (-4,0);

\coordinate (lt) at (-2,1);
\coordinate (rt) at (2,1);
\coordinate (mt) at ($(rt)!0.5!(lt)$);

\draw (l) -- (r) node [at end,right] {$z = -d$};
\draw (lt)  -- (rt) node [at end,right] {$z = 0$};

\draw (lt) node [above right] {$\eps_0,\mu_0$};
\draw ($(l)!0.5!(lt)$) node [right] {$\eps_1,\mu_1$};

\draw [->] (n) -- ++(0,1) node [at end, right] {$z$};
\draw [->] (n) -- ++(1,0) node [at end, right] {$x$};

\draw (n) circle(0.1cm) node [below=0.1cm] {$y$};
\draw (n) +(135:0.1cm) -- +(315:0.1cm);
\draw (n) +(45:0.1cm) -- +(225:0.1cm);

%\draw [->>,thick] (lt) ++ (1,1) -- (mt) ;


\end{tikzpicture}
\end{figure}

On travaille donc avec 
\begin{align}
\hat{\vE} (x_3) &= \int_{\RR^4} e^{-i(k_1x_1+k_2x_2 )}\vE(x_1,x_2,x_3) dx_1dx_2dk_1dk_2\\
\hat{\vH} (x_3) &= \int_{\RR^4} e^{-i(k_1x_1+k_2x_2 )}\vH(x_1,x_2,x_3) dk_1dx_2dx_1dk_2
\end{align}

\begin{tcolorbox}
Par abus de notation, on ne différencie pas les champs $\vE,\vH$ de leurs transformées de Fourier $\hat \vE, \hat \vH$.
\end{tcolorbox}

En explicitant par composante l'opérateur $\vrot$ , le problème est 
\begin{align*}
    \left\lbrace 
    \begin{matrix}
    ik_2 E_3  - \ddr{x_3}{E_ 2} = i \w \mu H_1 \\
    \ddr{x_3}{E_1} - ik_1 E_3 = i\w \mu H_2 \\
    ik_1 E_2 - ik_2 E_1 = i\w \mu H_3 \\
    \end{matrix}
    \right. \quad 
    \left\lbrace 
    \begin{matrix}
    ik_2 H_3  - \ddr{x_3}{H_ 2} = -i \w \eps E_1 \\
    \ddr{x_3}{H_1} - ik_1 H_3 = -i\w \eps E_2 \\
    ik_1 H_2 - ik_2 H_1 = -i\w \eps E_3 \\
    \end{matrix}
    \right.
\end{align*}

Les composantes normales se déduisant des composantes tangentielles, on résout l'EDO matricielle à coefficients constants 
suivante $\ddr{x_3}{}X = M X$ où

\begin{equation}
    X = 
    \begin{bmatrix}
    E_1 \\ 
    E_2 \\ 
    H_1 \\ 
    H_2 \\
    \end{bmatrix}\,
    M = \begin{bmatrix}
    0 & 0 & \frac{k_1k_2}{i\w\eps} & -i\left(\w\mu - \frac{k_1^2}{\w\eps}\right)\\
    0 & 0 & i\left(\w\mu - \frac{k_2^2}{\w\eps}\right) & -\frac{k_1k_2}{i\w\eps}\\
    -\frac{k_1k_2}{i\w\mu} & i\left(\w\eps - \frac{k_1^2}{\w\mu}\right) & 0 & 0 \\
    -i\left(\w\eps - \frac{k_2^2}{\w\mu}\right) & \frac{k_1k_2}{i\w\mu} & 0 & 0 \\
    \end{bmatrix}
\end{equation}

Pour résoudre cette EDO, nous allons chercher les vecteurs propres $V_i$ et les valeurs propres $\lambda_i$ associées de ce système. En effet, une solution générale de ce système s'écrit
\begin{equation}
    X(x_3)= \sum\limits_{i=1}^{4}c_i e^{\lambda_i x_3} V_i \quad , c_i \in \CC
\end{equation}
On pose 
\begin{equation}
    A = \begin{bmatrix}
        \frac{k_1k_2}{i\w\eps} & -\left(i\w\mu + \frac{k_1^2}{i\w\eps}\right) \\
        \left(i\w\mu + \frac{k_2^2}{i\w\eps}\right) & -\frac{k_1k_2}{i\w\eps} \\
    \end{bmatrix}
    \quad
    B = \begin{bmatrix}
        -\frac{k_1k_2}{i\w\mu} & \left(i\w\eps + \frac{k_1^2}{i\w\mu}\right) \\
        -\left(i\w\eps + \frac{k_2^2}{i\w\mu}\right) & \frac{k_1k_2}{i\w\mu} \\
    \end{bmatrix}
\end{equation}
Le déterminant de M est
\begin{align*}
    \det(M) &= 
    \begin{vmatrix}
        -\lambda I & A \\
        B & -\lambda I
    \end{vmatrix}
        = \frac{\det(- \lambda I - B(-\lambda I)^{-1} A)}{\det((-\lambda I)^{-1})} \\
        &= \det(\lambda^2 I - BA) \\
        &= (\lambda^2 + (\w^2\eps\mu - k_1^2 -k_2^2))^2
\end{align*}
On note alors 
\begin{equation}
k_3=\sqrt{\w^2\eps\mu - k_1^2 -k_2^2}
\end{equation}

Les valeurs propres sont alors $\lambda_\pm = \pm i k_3$. Les espaces propres associés sont de dimension 2, on a 

\begin{align}
\Ker(M-\lambda_+I)=\Vect{V_+;W_+} \\
    V_+ = 
    \begin{bmatrix}
    \lambda_+ \\
        0 \\
        -\frac{k_1k_2}{i\w\mu} \\
        -i\left(\w\eps - \frac{k_2^2}{\w\mu}\right) \\
    \end{bmatrix}
    \,
    W_+ = 
        \begin{bmatrix}
        0 \\
        \lambda_+ \\
        i\left(\w\eps - \frac{k_1^2}{\w\mu}\right) \\
        \frac{k_1k_2}{i\w\mu} \\
    \end{bmatrix}
\end{align}

\begin{align}
\Ker(M-\lambda_-I)=\Vect{V_-;W_-}\\
    V_- = 
    \begin{bmatrix}
        \lambda_- \\
        0 \\
        -\frac{k_1k_2}{i\w\eps} \\
        -i\left(\w\mu - \frac{k_2^2}{\w\eps}\right) \\
    \end{bmatrix}
    \,
    W_- = 
    \begin{bmatrix}
        0 \\
        \lambda_- \\
        i\left(\w\mu - \frac{k_1^2}{\w\eps}\right) \\
        \frac{k_1k_2}{i\w\eps} \\
    \end{bmatrix}
\end{align}

\TODO{Les vecteurs propres $V_\pm$ n'en sont pas car par exemple à la première composante de $MV_{+ 1} = \mu\eps^{-1}\lambda_+ V_{+ 1}  \not = \lambda_\pm V_{+ 1}$. Il faut corriger cette erreur, les résultats ci-dessous en dépendent.}

On a donc une solution générale du système 
\begin{equation}
    X(x_3) = c_1e^{\lambda_+ x_3}V_+  + c_2e^{\lambda_+ x_3}W_+ + c_3e^{\lambda_- x_3}V_- +c_4e^{\lambda_- x_3}W_- \quad c_i \in \CC
\end{equation}



Dans un premier temps, nous allons utiliser la condition limite 
\begin{equation}
    \begin{bmatrix}
        E_1(-d)\\
        E_2(-d)\\
    \end{bmatrix}
    =
    \begin{bmatrix}
        c_1 e^{-\lambda_+ d} \lambda_{+} + c_3 e^{-\lambda_- d} \lambda_{-} \\
        c_2 e^{-\lambda_+ d} \lambda_{+} + c_4 e^{-\lambda_- d} \lambda_{-}
    \end{bmatrix}
    =    
    \begin{bmatrix}
        0\\
        0\\
    \end{bmatrix}
\end{equation}
\begin{align}
    c_1 &= -\frac{\lambda_-}{\lambda_+}\frac{e^{-\lambda_-d}}{e^{-\lambda_+d}}c_3 = e^{i2k_3d}c_3\\
    c_2 &= -\frac{\lambda_-}{\lambda_+}\frac{e^{-\lambda_-d}}{e^{-\lambda_+d}}c_4 = e^{i2k_3d}c_4
\end{align}


Immédiatement, on injecte ce résultat pour obtenir
\begin{equation}
    X(x_3) = c_3\left(e^{ik_3(x_3 + 2d)}V_+ + e^{-ik_3x_3}V_- \right)  + c_4\left(e^{ik_3(x_3 + 2d)}W_+ + e^{-ik_3 x_3}W_- \right) \quad c_i \in \CC
\end{equation}

Ainsi
\begin{align}
 E_1(0) &= ik_3\left(e^{i2dk_3} - 1\right)c_3 \\
 E_2(0) &= ik_3\left(e^{i2dk_3} - 1\right)c_4 \\
 H_1(0) &= 
     ik_1k_2\left(\frac{e^{i2dk_3}}{\w\mu} + \frac{1}{\w\eps}\right)c_3 
    + i\left( \left(\w\eps-\frac{k_1^2}{\w\mu}\right)e^{i2dk_3} + \left(\w\mu-\frac{k_1^2}{\w\eps}\right) \right)c_4 \\
 H_2(0) &=
    - i\left( \left(\w\eps-\frac{k_2^2}{\w\mu}\right)e^{i2dk_3} + \left(\w\mu-\frac{k_2^2}{\w\eps}\right) \right)c_3
    -i k_1k_2\left(\frac{e^{i2dk_3}}{\w\mu} + \frac{1}{\w\eps}\right)c_4
\end{align}

Notons
\begin{align}
C &= 
\begin{bmatrix}
- i\left( \left(\w\eps-\frac{k_2^2}{\w\mu}\right)e^{i2dk_3} + \left(\w\mu-\frac{k_2^2}{\w\eps}\right) \right) & -i k_1k_2\left(\frac{e^{i2dk_3}}{\w\mu} + \frac{1}{\w\eps}\right)\\
- ik_1k_2\left(\frac{e^{i2dk_3}}{\w\mu} + \frac{1}{\w\eps}\right) &  - i\left( \left(\w\eps-\frac{k_1^2}{\w\mu}\right)e^{i2dk_3} + \left(\w\mu-\frac{k_1^2}{\w\eps}\right) \right)
\end{bmatrix}
\end{align}

On remarque que 

\begin{align}
    \begin{bmatrix}
        E_1(0)\\
        E_2(0)\\
    \end{bmatrix}
    & = ik_3\left(e^{i2dk_3} - 1\right)
    \begin{bmatrix}
        c_3 \\
        c_4 \\
    \end{bmatrix}\\
    \begin{bmatrix}
        H_2(0)\\
        -H_1(0)\\
    \end{bmatrix}
    & = C
    \begin{bmatrix}
        c_3 \\
        c_4 \\
    \end{bmatrix}
\end{align}

On définit l'opérateur d'impédance comme la matrice $Z$ telle que 
\begin{equation}
    \begin{bmatrix}
        E_1(0)\\
        E_2(0)\\
    \end{bmatrix}
    =
    \begin{bmatrix}
        Z_{11} & Z_{12} \\
        Z_{21} & Z_{22} \\
    \end{bmatrix}
    \begin{bmatrix}
        H_2(0)\\
        -H_1(0)\\
    \end{bmatrix}
\end{equation}

On déduit donc que
\begin{equation}
    Z =  ik_3e^{i2dk_3} C^{-1}
\end{equation}

\TODO{Prouver que C est inversible. Calculer son inverse. Retrouver les résultats connus}

\subsection{Cas d'un objet plan infini recouvert de plusieurs couches de matériaux homogènes isotropes}

\TODO{Changer la condition limite en $\z=-d$. Expliciter le résultat}

\subsection{Cas d'un objet cylindrique recouvert d'une couche de matériau homogène isotrope}

\TODO{Passer en cylindrique. Développer rot rot et trouver des équations de bessel pour Ez Hz. Déduire Er,Hr,Eo,Ho. Trouver Z}
