\section{CSU pour les CIOE de \cite{stupfel_sufficient_2011}}

  Maintenant que nous connaissant une condition suffisante, nous allons montrer comment les CIOE peuvent la garantir. Ce qu'il faut retenir dans cette démarche est que par nature, le caractère suffisant permet d'obtenir pour une CIOE plusieurs jeux de CSU. Cependant, il est intéressant que si l'on ait un jeu de CSU pour une CIOE, alors on y retrouve des CSU d'ordre moins élevés.

  Les CIOE de \cite{stupfel_sufficient_2011} font intervenir l'opérateur de Hodge \(\mathcal{L}\), commençons par rappeler son expression et quelques propriétés propriétés.

  \begin{defn}
    Pour tous \((\vu) \in (\mathcal C^\infty(\CC,\Gamma))^3\)
    \begin{equation}
      \LL(\vu) = \tgrads{\tdivs \vu} - \trots{\trots \vu}
    \end{equation}
  \end{defn}

  \begin{prop}
    Par définition, l’opérateur \(\LL\) est antisymétrique négatif.

    Pour tous \(\vu,\vv \in (\mathcal C^\infty(\CC,\Gamma))^3\)
    \begin{align}
      \int_\Gamma \vu\cdot \LL(\conj{\vv}) &= \int_\Gamma \conj{\vv}\cdot \LL(\vu)
      \\
      \int_\Gamma \vu\cdot \LL(\conj{\vu}) &\le 0
    \end{align}
  \end{prop}

  De plus, on rappelle la notation de la section précédente
  \begin{equation}
    X = \int_\Gamma \vJ \cdot \conj{\vE_t} ds(\vx)
  \end{equation}
  On rappelle que l'on veut trouver des conditions permettant de garantir \eqref{eq:unicite:form_var:cgu} qui est \(\Re(X)\ge0\).

  %%%%%%%%%%%%%%%%%%%%%%%%%%%%%%%%%%%%%%%%%%%%%%%%%%
  \subsection{CSU de la CI0}
    Utilisons la condition d’impédance de Leontovich, la \hyperlink{ci0}{CI0}:

    Soit \(a_0 \in \CC\) tel que
    \[
      \vE_t = a_0 \vJ
    \]

    On a alors
    \begin{equation*}
    X = \conj{a_0}||\vJ||_{L_2(\Gamma)}^2
    \end{equation*}

    De \eqref{eq:unicite:form_var:cgu}, on déduit une \gls{acr-csu}:
    \begin{equation}
    \Re\left(a_0\right) \ge 0
    \end{equation}

  %%%%%%%%%%%%%%%%%%%%%%%%%%%%%%%%%%%%%%%%%%%%%%%%
  \subsection{CSU de la CI01}
    Utilisons la condition d’impédance \hyperlink{ci01}{CI01}:

    Soit \((a_0, a_1) \in \CC\times\CC\) tel que

    \[
      \vE_t = (a_0 +a_1 \LL)\vJ
    \]


    \begin{prop}
      Les CSU qui impliquent la \gls{acr-cgu} sont
      \begin{align}
        \Re\left(a_0\right) \ge 0\\
        \Re\left(a_1\right) \le 0
      \end{align}
    \end{prop}

    \begin{proof}
      On pose:
      \begin{align*}
        F&:= || \vJ|| ^2 \ge 0  & G&:= -\int_\Gamma \vJ\cdot \LL\conj{\vJ} ds \ge 0
      \end{align*}

      On a alors
      \begin{equation*}
        X = \conj{a_0}F - \conj{a_1}G
      \end{equation*}

      De \eqref{eq:unicite:form_var:cgu}, on déduit des CSU suivantes:
      \begin{align}
        \Re\left(a_0\right) \ge 0\\
        \Re\left(a_1\right) \le 0
      \end{align}
    \end{proof}

    Il n'y a pas d'autre CSU qui soient plus évidentes que celles ci et elle permettent de retrouver la CSU de la CI0 quand \(a_1=0\) donc ce sont des CSU satisfaisante pour cette CIOE.

  %%%%%%%%%%%%%%%%%%%%%%%%%%%%%%%%%%%%%%%%%%%%%%%%
  \subsection{CSU de la CI1}

    Utilisons la condition d’impédance \hyperlink{ci1}{CI1}:

    Soit \((a_0, a_1,b) \in \CC^3\) tel que
    \[
      \vE_t = (1 + b \LL)^{-1} (a_0 + a_1 \LL) \vJ
    \]

    Pour cette CIOE, il existe déjà plusieurs jeux de CSU possible.

    \subsubsection{CSU de \cite{stupfel_sufficient_2011}}

      \begin{prop}
        Soit \(\Delta = a_1 - a_0\conj{b}\), alors les CSU sont
        \begin{align}
          &\Re(\Delta) &= 0\\
          &\Im(\Delta)\Im(b) &\ge 0\\
          &\Im(\Delta)\Im(a_1\conj{a_0})&\ge 0
        \end{align}
      \end{prop}

      \begin{proof}
        On utilise l'identité \((a_1-a_0\conj{b}) = (a_1(1+\conj{b}\LL) - \conj{b}(a_0+a_1\LL))\):

        \begin{align*}
          (a_1-a_0\conj{b})X &= \int_\Gamma \left(a_1(1+\conj{b}\LL) \vJ\right)\cdot\conj{\vE_t} - \left(\conj{b}(a_0+a_1 \LL)\vJ\right)\cdot\conj{\vE_t} ds\\
          &= \int_\Gamma \left(a_1(1+\conj{b}\LL) \conj{\vE_t}\right)\cdot\vJ ds - \int_\Gamma \left(\conj{b}(a_0+a_1 \LL)\vJ\right)\cdot\conj{\vE_t} ds\\
          &= \int_\Gamma \left(a_1(\conj{a_0}+\conj{a_1}\LL) \conj{\vJ}\right)\cdot\vJ ds  - \int_\Gamma \left(\conj{b}(1+b \LL)\vE_t\right)\cdot\conj{\vE_t} ds\\
          &= a_1\conj{a_0} ||\vJ||^2 + |a_1|^2 \int_\Gamma \vJ \LL \conj{\vJ} ds - \conj{b} ||\vE_t||^2 - |b|^2 \int_\Gamma \vE_t \LL \conj{\vE_t} ds
        \end{align*}

        On pose \(F = -\int_\Gamma \vJ \LL \conj{\vJ} ds \ge 0 \), \(G = -\int_\Gamma \vE_t \LL \conj{\vE_t} ds \ge 0 \) .

        Si on décompose les parties réelles et imaginaires de cette expression, on a
        \begin{align*}
          \Re(\Delta)\Re(X) - \Im(\Delta)\Im(X) &= \Re(a_1\conj{a_0}) ||\vJ||^2 - \Re(\conj{b})||\vE_t||^2 -|a_1|^2 F + |b|^2 G \\
          \Im(\Delta)\Re(X) + \Re(\Delta)\Im(X) &= \Im(a_1\conj{a_0}) ||\vJ||^2 - \Im(\conj{b})||\vE_t||^2
        \end{align*}
        La première relation nous empêche de conclure sur le signe de \(\Re(X)\) car il y des signes différents entre les deux derniers termes, sauf si nous imposons \(\Re(\Delta)= 0\) auquel cas, nous pouvons conclure grâce à la deuxième relation. Les CSU sont alors
        \begin{align}
          &\Re(\Delta) &= 0\\
          &\Im(\Delta)\Im(b) &\ge 0\\
          &\Im(\Delta)\Im(a_1\conj{a_0})&\ge 0
        \end{align}
      \end{proof}

      Nous pouvons noter que ces CSU fixent par exemple la partie réelle de \(a_1\) en fonction des autres coefficients, ce qui est plus contraignant qu'une condition d'inégalité. De plus, on remarque qu'on ne retombe pas sur les CSU de la CI01 si on annule \(b\). Ce sont donc des CSU trop contraignantes.
    
    \subsubsection{CSU de \cite{stupfel_implementation_2015}}

      Les CSU présentées dans cette article contiennent une erreur. Nous présentons ici une correction qui.

      \begin{prop}
        Un autre jeu de CSU est
        \begin{align}
          a_1 = a_0 \conj{b} \\
          \Re\left(\frac{\conj{a_0}}{1 + \conj{b}^2}\right) &\ge 0 \\
          \Re\left(\frac{\conj{a_1}}{1 + \conj{b}^2}\right) &\le 0
        \end{align}
      \end{prop}

      \begin{proof}
        On cherche à résoudre le problème suivante:
        \[
          \begin{bmatrix}
            1 & \conj{b} \\
            a_0 & a_1
          \end{bmatrix}
          \begin{bmatrix}
            \int_\Gamma \vJ \cdot \conj{\vE_t} \\
            \int_\Gamma \vJ\cdot \LL\conj{\vE_t}
          \end{bmatrix}
          =
          \begin{bmatrix}
            \conj{a_0}||\vJ||_2^2 + \conj{a_1}\int_\Gamma \vJ \cdot \LL\conj{\vJ} \\
            ||\vE_t||_2^2 + b\int_\Gamma \conj{\vE_t} \cdot \LL\vE_t
          \end{bmatrix}
        \]

        Si la matrice est inversible, on pose \(\Delta = a_1 - a_0\conj{b}\) et alors
        \[
        \begin{bmatrix}
          \int_\Gamma \vJ \cdot \conj{\vE_t} \\
          \int_\Gamma \vJ\cdot \LL\conj{\vE_t}
        \end{bmatrix}
        =\frac{1}{\Delta}
        \begin{bmatrix}
          a_1 & -\conj{b} \\
          -a_0 & 1
        \end{bmatrix}
        \begin{bmatrix}
          \conj{a_0}||\vJ||_2^2 + \conj{a_1}\int_\Gamma \vJ \cdot \LL\conj{\vJ} \\
          ||\vE_t||_2^2 + b\int_\Gamma \conj{\vE_t} \cdot \LL\vE_t
        \end{bmatrix}
        \]
        Donc
        \begin{align*}
          X &=  \frac{a_1\conj{a_0}}{\Delta}||\vJ^2||_2^2 - \frac{\conj{b}}{\Delta}||\vE_t||_2^2 \\
          &~+\frac{|a_1|^2}{\Delta}\int_\Gamma\vJ\cdot \LL \conj{\vJ} - \frac{|b|^2}{\Delta}\int_\Gamma\conj{\vE_t}\cdot \LL\vE
        \end{align*}
        Les CSU sont alors bien celles que l'on a déjà trouvées

          \begin{align}
          \Re\left(a_0\conj{a_1}\Delta\right) &\ge 0\\
          \Re\left(b\Delta\right) &\le 0\\
          \Re\left(|a_1|^2\Delta\right) &\le 0\\
          \Re\left(|b|^2\Delta\right) &\ge 0
        \end{align}

        Si la matrice n'est pas inversible, alors on cherche à résoudre
        \[
          \begin{bmatrix}
            1 & \conj{b} \\
            a_0 & a_0\conj{b}
          \end{bmatrix}
          \begin{bmatrix}
            \int_\Gamma \vJ \cdot \conj{\vE_t} \\
            \int_\Gamma \vJ \cdot \LL\conj{\vE_t}
          \end{bmatrix}
          =
          \begin{bmatrix}
            \conj{a_0}||\vJ||_2^2 + \conj{a_1}\int_\Gamma \vJ \cdot \LL\conj{\vJ} \\
            ||\vE_t||_2^2 + b\int_\Gamma \conj{\vE_t} \cdot \LL\vE_t
          \end{bmatrix}
        \]

        Le noyau de la matrice est alors \(\Vect{\begin{bmatrix}\conj{b}\\-1\end{bmatrix}}\) dont l'orthogonal est  \(\Vect{\begin{bmatrix}1\\\conj{b}\end{bmatrix}}\).
        Pour tout \(\int_\Gamma \vJ\cdot \LL\conj{\vE_t} = \conj{b} \int_\Gamma \vJ \cdot \conj{\vE_t} \), on a unicité des solutions. On déduit alors que

        \[
          (1 + \conj{b}^2) X = \conj{a_0} ||\vJ||_2^2 + \conj{a_1}\int_\Gamma \vJ \cdot \LL\conj{\vJ}
        \]

        Les CSU sont alors

        \begin{align}
          \Delta = 0 \\
          \Re\left(\frac{\conj{a_0}}{1 + \conj{b}^2}\right) &\ge 0 \\
          \Re\left(\frac{\conj{a_1}}{1 + \conj{b}^2}\right) &\le 0
        \end{align}

      \end{proof}

      Mais dans ce cas, on ne retombe pas non plus sur les CSU de la CI01 quand \(b\) s'annule.

    \subsubsection{CSU de 2017}
      ~
      Nous avons trouvé qu'il est possible de relâcher un peu les contraintes, si l'on fait une distinction sur la valeur de la constante \(a_1\).
      \paragraph{Cas \(a_1\not=0\)}
        ~
        \begin{prop}
          Les CSU sont
          \begin{align}
            \Re\left(\frac{b}{a_1}\right) \ge 0 \\
            \Re\left(a_0\right) \ge 0 \\
            \Re\left(\left(a_1-a_0 b\right)\frac{\conj{a_1}}{a_1}\right) \le 0
          \end{align}
        \end{prop}
        \begin{proof}
          En supposant \(a_1 \not=0\), on utilise l'identité \((a_0 + a_1 \LL)^{-1}(1 + b \LL)  = \frac{b}{a_1} I_d + \left(1-b\frac{a_0}{a_1}\right)(a_0+a_1 \LL)^{-1}\):
          \[
            X = \int_\Gamma \left(\left(\frac{b}{a_1} I_d + \left(1-b\frac{a_0}{a_1}\right)(a_0+a_1 \LL)^{-1}\right)\vE_t\right) \cdot \conj{\vE_t} ds
          \]

          On pose:
          \begin{align*}
            \vect D &:= (a_0 + a_1 \LL)^{-1}\vE_t & F&:= || \vect D || ^2 \ge 0  \\
            G&:= -\int_\Gamma \vect D \cdot \LL\conj{\v{D}} ds \ge 0 & H &:= || \vE_t || ^2 \ge 0
          \end{align*}
          Comme \(\conj{E_t} = (\conj{a_0} + \conj{a_1}\LL)D\) alors \(\ds\int_\Gamma (a_0 +a_1 \LL) ^{-1}\vE_t\cdot \conj{\vE_t} ds = \conj{a_0} F - \conj{a_1} G\) et l'on peut alors écrire

          \begin{equation}
            \label{eq:unicite:form_var:decomp_cgu_ci1_a1}
            X = \frac{b}{a_1}H   + \left(1-b\frac{a_0}{a_1}\right)\left(\conj{a_0} F - \conj{a_1} G\right)
          \end{equation}
          De \eqref{eq:unicite:form_var:cgu}, on déduit des CSU suivantes:
          \begin{align}
            \Re\left(\frac{b}{a_1}\right) \ge 0 \\
            \Re\left(a_0\right) \ge 0 \\
            \Re\left(\left(a_1-a_0 b\right)\frac{\conj{a_1}}{a_1}\right) \le 0
          \end{align}
        \end{proof}

        Ces CSU permettent de retomber sur les CSU de la CI01 quand \(b\) vaut zéro. De plus, une condition de non-nullité peut n'être vérifiée qu'à posteriori et donc ce jeu est un meilleur choix dans le cadre d'un code numérique.

      \paragraph{Cas \(a_1=0\)}
        ~
        On complète les CSU précédentes
        \begin{prop}
          Les CSU sont
          \begin{align}
            a_1 = 0 \\
            a_0 \not= 0\\
            \Re\left(a_0\right) \ge 0\\
            \Re\left(b\conj{a_0}\right) \le 0
          \end{align}
        \end{prop}
        \begin{proof}
          \[
            X = \int_\Gamma \left( \frac{1}{a_0}\left(1+b\LL\right)\vE_t\right) \cdot \conj{\vE_t} ds
          \]

          On pose:
          \begin{align*}
            F&:= \int_\Gamma | \vE_t | ^2 ds \ge 0 & G &:= -\int_\Gamma \vE_t \cdot \LL\conj{\vE_t} ds \ge 0
          \end{align*}

          On a alors
          \begin{equation}
            \label{eq:unicite:form_var:decomp_cgu_ci1_a1_nul}
            X = \frac{1}{a_0}F - \frac{b}{a_0}G
          \end{equation}

          De \eqref{eq:unicite:form_var:cgu}, on déduit des CSU suivantes:
          \begin{align}
            a_1 = 0 \\
            a_0 \not= 0\\
            \Re\left(a_0\right) \ge 0\\
            \Re\left(b\conj{a_0}\right) \le 0
          \end{align}
        \end{proof}

        On remarque que si \(b\) s’annule, on retombe bien sur la CSU de la CI0. Ce jeu, qui fait la distinction entre deux cas, à malgré tout cet avantage sur le premier.
      %%%%%%%%%%%%%%%%%%%%%%%%%%%%%%%%%%%%%%%%%%%%%%%%%%%%%%%%%%%%%%%%%%%%%%%%%%%%%%%%%%%%%%%%%%%%%%%%%%%%%%%%%
