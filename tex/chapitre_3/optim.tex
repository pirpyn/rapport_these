\section{Calcul des coefficient des CIOE par moindres carrés sous contraintes}

\subsection{Expression des moindre carrés dans le cadre de l'approximation plan infini}
  Pour toutes les CIOE, on cherche à approcher le symbole de l'opérateur d'impédance \(\hat\mZ_{ex}(k_x,k_y)\) par une matrice \(\hat\mZ_{ap}(k_x,k_y)\)

  \begin{prop}
    SOit \(k_x,k_y\) fixés.
    Pour nos CIOE, il existe une matrice \(\mH_{k_x,k_y}(CI,\hat\mZ_{ex}(k_x,k_y))\) et un vecteur \(\vect{X}(CI)\) où CI représente un vecteur de \(\CC^n\) composés des coefficient de la CI telles que minimiser 
    \[
      ||\hat\mZ_{ap}(k_x,k_y)-\hat\mZ_{ex}(k_x,k_y)||^2
    \] 
    revient à minimiser 
    \[ 
      || \mH_{k_x,k_y}(CI,\hat\mZ_{ex}(k_x,k_y)) \vect{X}(CI) - b(\hat\mZ_{ex}(k_x,k_y)) ||^2
    \]
    Les dimensions de \(\mH\) sont de 4 lignes et le nombre de coefficients de la CI colonnes et donc le vecteur colonne b à 4 coefficients.
  \end{prop}

  \begin{proof}
    Nous démontrons ce résultat pour la \hyperlink{ci3}{CI3}, les matrices des autres CIOE s'en déduisent.

    Soient les matrices symétriques \(\hat\mLD, \hat\mLR\) telles que

    \begin{align}
      \hat\mLD(k_x,k_y) & = - \begin{bmatrix} k_x^2 & k_x k_y \\ k_x k_y & k_y^2 \end{bmatrix}
      \\
      \hat\mLR(k_x,k_y) & =  \begin{bmatrix} k_y^2 & -k_x k_y \\ -k_x k_y &  k_x^2 \end{bmatrix}
    \end{align}

    Par définition, on a
    \begin{align}
    ||\hat\mZ_{ap}-\hat\mZ_{ex}||^2 &= ||\left(\mI + b_1 \hat{\mLD} - b_2 \hat{\mLR}\right)^{-1}\left(a_0\mI + a_1 \hat{\mLD} - a_2 \hat{\mLR}\right)-\hat\mZ_{ex} ||^2
    \\
    &= || a_0\mI + a_1 \hat{\mLD} - a_2 \hat{\mLR} ||^2 + ||\hat\mZ_{ex}||^2 - 2 \left<\left(\mI + b_1 \hat{\mLD} - b_2 \hat{\mLR}\right)\hat\mZ_{ex},a_0\mI + a_1 \hat{\mLD} - a_2 \hat{\mLR} \right>
    \\
    &= ||\hat\mZ_{ex}||^2 + \left<a_0\mI + a_1 \hat{\mLD} - a_2 \hat{\mLR} -2\left(\mI + b_1 \hat{\mLD} - b_2 \hat{\mLR}\right)\hat\mZ_{ex},a_0\mI + a_1 \hat{\mLD} - a_2 \hat{\mLR} \right> 
    \end{align}
  \end{proof}