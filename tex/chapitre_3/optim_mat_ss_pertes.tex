\subsection{Problème de la singularité de l'impédance à une certaine incidence dans le cadre du plan infini}
On se place dans le cadre du plan infini de \cite{soudais_3d_2017}, où \(\eps=4,\mu=1,f=12\) Ghz, et \(d=3.5\) mm. Ce cas non physique possède un onde guidée pour \(s=s^\star\) où \(r(s^\star) = \infty\).

De plus, il existe aussi \(s_z\) tel que \(Z(s_z) = \infty\). En effet, d'après la formule pour une couche de matériau \eqref{eq:imp_plan:symb_z:1c}, \(s_z = \sqrt{\eps \mu - \left(\frac{\pi/2}{k_0 d}\right)^2}\).

La fonctionnelle que l'on cherche à minimiser est \(||H X - Z||\) où \(H\) est une matrice complexe \(2n\) par 5, X un vecteur complexe de taille 5 et \(Z\) un vecteur complexe de taille \(2n\) où \(n\) est le nombre de point entre 0 et \(s_{max}\).

\begin{TODO}
  Il y a confusion entre des grandeurs scalaire dépendante de \(s\) et les grandeurs matricielles qui sont des constantes. A éclaircir.
\end{TODO}

On a
\[
  H = \begin{bmatrix}
  \vdots & \vdots & \vdots & \vdots &\vdots \\
  1 & -s_i^2 & 0 & s_i^2 Z_{Mi} & 0 \\
  \vdots & \vdots & \vdots & \vdots &\vdots \\
  \vdots & \vdots & \vdots & \vdots &\vdots \\
  1 & 0 &-s_i^2 & 0 & s_i^2 Z_{Ei} \\
  \vdots & \vdots & \vdots & \vdots &\vdots \\
  \end{bmatrix},
  X = \begin{bmatrix}
  a_0\\
  a_1\\
  a_2\\
  b_1\\
  b_2\\
  \end{bmatrix},
  Z=\begin{bmatrix}
  \vdots\\
  Z_{Mi}\\
  \vdots\\
  \vdots\\
  Z_{Ei}\\
  \vdots
  \end{bmatrix}
\]

D'après la définition \eqref{eq:imp_plan:symb_z:1c}, \(\lim\limits_{s\rightarrow s_z} (s-s_z) Z(s)
= \begin{bmatrix}
  \vdots \\
  0\\
  \alpha_{M}\\
  0 \\
  \vdots\\
  0 \\
  0 \alpha_{E} \\
  0 \\
  \vdots \\
\end{bmatrix}\) et d'après la définition \eqref{eq:unicite:ci3:ci3} \(\lim\limits_{s\rightarrow s_z} (s-s_z)H(s) =
\begin{bmatrix}
  \vdots & \vdots & \vdots & \vdots &\vdots \\
  0 & 0 & 0 & 0 & 0 \\
  0 & 0 & 0 & s_z^2 \alpha_{M} & 0 \\
  0 & 0 & 0 & 0 & 0 \\
  \vdots & \vdots & \vdots & \vdots &\vdots \\
  0 & 0 & 0 & 0 & 0 \\
  0 & 0 & 0 & 0 & s_z^2 \alpha_{E} \\
  0 & 0 & 0 & 0 & 0 \\
  \vdots & \vdots & \vdots & \vdots &\vdots \\
\end{bmatrix}\).


On suppose donc qu'il existe \(Z_\infty, \tilde{Z},X_\infty, \tilde{X},H_\infty, \tilde{H}\) tels que
\begin{align*}
H(s) &= \frac{1}{s-s_z}H_\infty + \tilde{H}(s) \\
Z(s) &= \frac{1}{s-s_z}Z_\infty + \tilde{Z}(s) \\
X(s) &= X_\infty + (s-s_z)\tilde{X}(s) \\
H_\infty X_\infty &= Z_\infty
\end{align*}

Et on en déduit que \(X_\infty = \begin{bmatrix}
  0\\
  0\\
  0\\
  s_z^{-2}\\
  s_z^{-2}\\
  \end{bmatrix}\). Cela veut dire qu'à moins de fixer \(b_1\) et \(b_2\), on se trouvera dans le noyau de \(H_\infty\) et donc la minimisation sera fausse.

On pose \(\tilde{X} = \begin{bmatrix}
  a_0\\
  a_1\\
  a_2\\
  0\\
  0\\
  \end{bmatrix}\) et l'on a un problème réduit:


\begin{align*}
\min\limits_{X\in\CC^5}{||HX-Z||}
&\equiv \min\limits_{X\in\CC^5}{||(s-s_z)(HX-Z)||} \\
&= \min\limits_{X\in\CC^5}{||(H_\infty + (s-s_z)\tilde{H})(X_\infty + (s-s_z)\tilde{X})-(Z_\infty + (s-s_z)\tilde{Z})||}\\
&= \min\limits_{\tilde{X}\in\CC^5}{||(s-s_z)\left[\tilde{H}(X_\infty + (s-s_z)\tilde{X}) + H_\infty\tilde{X} - \tilde{Z}\right]||}\\
&= \min\limits_{\tilde{X}\in\CC^5}{||\tilde{H}(X_\infty + (s-s_z)\tilde{X}) + H_\infty\tilde{X} - \tilde{Z}||}\\
&= \min\limits_{\tilde{X}\in\CC^5}{||(H_\infty + (s-s_z)\tilde{H})\tilde{X} + \tilde{H}X_\infty - \tilde{Z}||}\\
\end{align*}

\subsection{Vérification numérique}
Cette CIOE réduite a été testé. Elle donne de très bon résultats.
