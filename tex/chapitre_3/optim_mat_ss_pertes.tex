\subsection{Problème de la singularité de l'impédance dans le cadre du plan infini pour une couche de matériaux}

Dans la partie précédente nous avons introduit la fonctionnelle que l'on cherche à minimiser qui est \(F(X) = \left\lVert\tilde{\mH} X - b(\mZ_{ex})\right\rVert_{\RR^N}\).

On se place dans le cadre du plan infini de \cite{aubakirov_electromagnetic_2014}, où \(\eps=4,\mu=1,f=12\) Ghz, et \(d=3.5\) mm. %Ce cas non physique possède un onde guidée pour \((k_x,k_y) = (k_0s^\star,0.)\) où \(\mR(k_0s^\star,0.) = \infty\).

Il existe un \(s_z\) tel que \(\hat\mZ_{ex}(k_0s_z,0.) = \infty\). En effet, d'après la formule pour une couche de matériau \eqref{eq:imp_plan:symb_z:1c}, 

\begin{equation}
  \hat{\mZ}_{ex}(k_x,0.) = i\frac{\eta}{k\sqrt{k^2 - k_x^2}}\tan(\sqrt{k^2 - k_x^2}d)\left(k^2\mI - \hat{\mLR}\right)
\end{equation}
Donc il est facile de voir que l'on a une asymptote à cause de la tangente et donc pour cet empilement
\begin{equation}
  s_z = \sqrt{\eps \mu - \left(\frac{\pi/2}{k_0 d}\right)^2}
\end{equation}

Le problème est donc que si nous balayons en incidence et que l'on passe par ce point, alors la matrice \(\hat\mZ\) n'est pas défini en ce point. Or comme le gradient de la fonctionnelle est fonction de cette matrice, le gradient n'est pas défini pour tout \(X\). Si l'on utilise une méthode basée sur le gradient de type Newton, ce que nous avons fait, on comprend pourquoi la méthode numérique échoue à calculer des coefficients.

\subsubsection{Réduction du nombre de variables de la minimisation}

On décompose alors nos matrices et vecteurs en séparant les parties contentant cette asymptote.

On suppose donc qu'il existe \(\tilde{\mH}_\infty, \tilde{b}_\infty, X_\infty,\) tels que
\begin{align*}
  \tilde{\mH} &= \tilde{\mH}_\infty + \tilde{\mH}_r
  \\
  \tilde{b} &= \tilde{b}_\infty + \tilde{b}_r
  \\
  X &= X_\infty + X_r
\end{align*}

Ces matrices et vecteurs sont reliés par les relations
\begin{align}
  \tilde{\mH}_\infty X_\infty &= \tilde{b}_\infty
  \\
  \tilde{\mH}_\infty X_r &= 0
\end{align}

Il faut vraiment voir cette décomposition comme deux partie, où l'une est nulle quasiment partout sauf pour le \(s_z\) problématique et l'autre est définie normalement sauf aux termes correspondant au \(s_z\) où elle est nulle.

Schématiquement on définit \(i_z\) l'indice d'une ligne telle que \(\tilde{b}(\hat{\mZ})_{i_z} = \infty\)

\begin{equation*}
  \begin{matrix}
    \tilde{\mH} &=& \tilde{\mH}_\infty &+& \tilde{\mH}_r
    \\
    \begin{bmatrix}
      \tilde{\mH}_{1,1} & \cdots & \tilde{\mH}_{1,N_{CI}}
      \\
      \vdots & \ddots & \vdots
      \\
      \tilde{\mH}_{i_z-1,1} & \cdots & \tilde{\mH}_{i_z-1,N_{CI}}
      \\
      \tilde{\mH}_{i_z,1} & \cdots & \tilde{\mH}_{i_z,N_{CI}}
      \\
      \tilde{\mH}_{i_z+1,1} & \cdots & \tilde{\mH}_{i_z+1,N_{CI}}
      \\
      \vdots & \ddots & \vdots
      \\
      \tilde{\mH}_{4N_i,1} & \cdots & \tilde{\mH}_{4N_i,N_{CI}}
    \end{bmatrix}
    & = &
    \begin{bmatrix}
      0 & \cdots & 0
      \\
      \vdots & \ddots & \vdots
      \\
      0 & \cdots & 0
      \\
      \tilde{\mH}_{i_z,1} & \cdots & \tilde{\mH}_{i_z,N_{CI}}
      \\
      0 & \cdots & 0
      \\
      \vdots & \ddots & \vdots
      \\
      0 & \cdots & 0
    \end{bmatrix}
    & + & 
    \begin{bmatrix}
      \tilde{\mH}_{1,1} & \cdots & \tilde{\mH}_{1,N_{CI}}
      \\
      \vdots & \ddots & \vdots
      \\
      \tilde{\mH}_{i_z-1,1} & \cdots & \tilde{\mH}_{i_z-1,N_{CI}}
      \\
      0 & \cdots & 0
      \\
      \tilde{\mH}_{i_z+1,1} & \cdots & \tilde{\mH}_{i_z+1,N_{CI}}
      \\
      \vdots & \ddots & \vdots
      \\
      \tilde{\mH}_{4N_i,1} & \cdots & \tilde{\mH}_{4N_i,N_{CI}}
    \end{bmatrix}
  \end{matrix}
\end{equation*}
Dans les fait, \(\tilde{\mH}_\infty\) à \(4\) lignes non-nulles et \(\tilde{\mH}_r\) en a autant de nulles au même endroits.


On développe donc la fonctionnelle suivant cette décomposition.
\begin{align*}
\argmin{X}\left\rVert \tilde{\mH} X - \tilde{b} \right \rVert &= \argmin{X_r}\left\rVert \left(\tilde{\mH}_\infty + \tilde{\mH}_r\right)\left( X_\infty + X_r \right) - \tilde{b}_\infty - \tilde{b}_r \right \rVert
\\
\intertext{On utilise la relation entre \(\tilde{\mH}_\infty X_\infty\) et \(\tilde{b}_\infty\)}
&=  \argmin {X_r}\left\rVert \tilde{\mH}_\infty X_r + \tilde{\mH}_r X_\infty + \tilde{\mH}_r X_r - \tilde{b}_r \right \rVert
\\
\intertext{Enfin par définition de \(\tilde{\mH}_\infty\) et \(X_r\), leur produit est nul}
&= \argmin{X_r} \left\rVert \tilde{\mH}_r ( X_r + X_\infty)- \tilde{b}_r \right \rVert
\end{align*}

On voit alors que l'on peut résoudre le problème si l'on minimise uniquement sur les \(X_r\), les autres étant fixés et que l'on enlève du système les lignes où l'impédance n'est pas définie.

\subsubsection{Application de la réduction à la CI3}

On rappelle l'expression de la CI3
\begin{equation}
  \hat{\mZ}_{ap}(k_x,0.) = \left(\mI + b_1 \hat{\mLD} - b_2 \hat{\mLR} \right)^{-1}\left(a_0\mI + a_1 \hat{\mLD} - a_2 \hat{\mLR} \right)
\end{equation}

On voit donc que pour faire apparaître une asymptote, il faut que la matrice de gauche ne soit pas inversible en \(s_z\).

La matrice \(\tilde{\mH}_\infty\) est donc nulle partout sauf en 8 termes, placés sur les deux dernières colonnes et les 4 lignes correspondantes à \((k_x,k_t)=(k_0 s_z,0)\).

Connaissant les expressions des matrices \(\hat\mLD,\hat\mLR\) introduites dans la partie précédente alors on déduit que
\begin{align}
  X_\infty = \begin{bmatrix}
    0\\
    0\\
    0\\
    (k_0 s_z)^{-2}\\
    (k_0 s_z)^{-2}\\
  \end{bmatrix}
  & &
  X_r = \begin{bmatrix}
  a_0\\
  a_1\\
  a_2\\
  0\\
  0\\
  \end{bmatrix}
\end{align}
