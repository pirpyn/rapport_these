\subsection{CSU pour la CI4}
  Soit la CIOE que l'on nomme \hyperlink{ci4}{CI4} :
  \begin{equation}
    \label{eq:unicite:ci4:ci4}
    \vE_t = (a_0\oI + a_1 \LD - a_2 \LR ) \vJ
  \end{equation}

  \begin{prop}
    Des CSU sont
    \begin{align}
      \Re(a_0) \ge 0
      \\
      \Re(a_1) \le 0
      \\
      \Re(a_2) \le 0
    \end{align}
  \end{prop}

  \begin{prop}
    On pose:
    \begin{align*}
      F &:= ||\vJ|| ^2 \ge 0  & G_1 &:= \int_\Gamma \vJ\cdot \LD\conj{\vJ} ds \le 0 & G_2 &:= \int_\Gamma \vJ\cdot \LR\conj{\vJ} ds \ge 0
    \end{align*}

    On a alors
    \begin{equation*}
      X = \conj{a_0}F + \conj{a_1}G - \conj{a_2}G
    \end{equation*}

    De \eqref{eq:unicite:form_var:cgu}, on déduit des CSU suivantes:
    \begin{align}
      \Re\left(a_0\right) \ge 0
      \\
      \Re\left(a_1\right) \le 0
      \\
      \Re\left(a_2\right) \le 0
    \end{align}
  \end{prop}

  On remarque que ces CSU redonnent les CSU de la CI01 quand \(a_1=a_2\). Ce jeu semble être le plus naturel, sans être trop contraignant.

\section{CSU pour la CIOE CI3 de \cite{aubakirov_electromagnetic_2014}}

  Soit la CIOE énoncé dans \cite{aubakirov_electromagnetic_2014} que l'on nomme \hyperlink{ci3}{CI3} :
  \begin{equation}
    \label{eq:unicite:ci3:ci3}
    ( \oI + b_1 \LD - b_2 \LR)\vE_t = (a_0\oI + a_1 \LD - a_2 \LR ) \vJ
  \end{equation}

  \begin{defn}
    On rappelle les expressions des opérateurs \gls{ope-LD} et \gls{ope-LR} pour des vecteurs tangents \(\vect U,\vect V \in (\mathcal{C}^\infty(\CC,\Gamma))^3\): 
    \begin{align*}
      \LD(\vect U) &= \tgrads \tdivs \vect U\\
      \LR(\vect V) &= \trots( \vn ( \vn \cdot \trots \vect V))\\
    \end{align*}
  \end{defn}

  \begin{prop}
    Par définition, \(\LD\) est antisymétrique négatif et \(\LR\) antisymétrique positif.
  \end{prop}

  \begin{prop}
    Soit \(\OO\) un domaine borné de \(\RR^3\) , de surface \(\Gamma\) fermée et régulière, où \(\vect n\) y est la normale unitaire
    sortante
    \begin{equation}
      \begin{matrix}
        \forall \vect U \in (\mathcal{C}^\infty(\CC,\Gamma))^3 ,& \LR(\LD(\vect U)) = \LD(\LR(\vect U)) = 0
      \end{matrix}
    \end{equation}
  \end{prop}
  \begin{proof}

    Soient un vecteur \textbf{tangent} \(\vect U \in (\mathcal{C}^\infty(\CC,\Gamma)^3)\). 

    Montrons que \(\LR\LD = 0\).
    D’après \cite[p.~1029, A3.42]{bladel_electromagnetic_2007}, \(\vn \cdot \trots\tgrads f = 0\)
    \begin{align*}
      \LR(\LD \vect U)  &= \trots \left(\vn \left(\vn \cdot \trots \left( \tgrads \left(\tdivs \vect U\right)\right)\right)\right) \\
      &= 0
    \end{align*}
    Montrons que \(\LD\LR = 0\).
    D’après \cite[p.~1029, A3.43]{bladel_electromagnetic_2007}, \(\tdivs \trots (f\vn) = 0\).
    \begin{align*}
      \LD(\LR \vect U) &= \tgrads \tdivs \trots (\vn (\vn \cdot \trots \vect U)) \\
      &= 0
    \end{align*}
  \end{proof}
  % Une relation importante qui découle des propriétés des opérateurs différentiels surfacique \secref{eq:op-LD-LR:prop:LDLR0} est :

  % \begin{equation}
  % \int_\Gamma \LD(\vect U) \cdot \LR(\vect V) ds = 0 , \forall \vect U, \vect V \in (H^1(\OO))^3
  % \end{equation}

  % Cette relation \Gamma'exprime sous forme forte par \(\LD\LR\equiv0\). Elle est là aussi symétrique entre les deux opérateurs.

\subsection{CSU de Stupfel}

  \begin{prop}
    Soit \(\Delta_i = a_i-\conj{b_i}a_0\), \(i=1,2\). Des CSU sont
    \begin{align}
      \Re\left(a_0\conj{a_1}\Delta_1\right) \ge 0 \\
      \Re\left(\frac{\conj{b_1}}{\Delta_1}\right) \le 0 \\
      \Re\left(\conj{a_0}a_2\left(\frac{\conj{b_2}}{\Delta_2}-\frac{\conj{b_2}}{\Delta_2}\right) + \frac{\conj{a_2}a_1}{\Delta_1} \right)\le 0\\
      \Re\left(2\Re(b_2)\frac{\conj{b_1}}{\Delta_1}-\frac{\conj{b_2}^2}{\Delta_2}\right) \ge 0\\
      \Re\left(a_0\conj{a_2}\Delta_2\right) \ge 0 \\
      \Re\left(\frac{\conj{b_2}}{\Delta_2}\right) \le 0 \\
      \Re\left(\conj{a_0}a_1\left(\frac{\conj{b_1}}{\Delta_1}-\frac{\conj{b_2}}{\Delta_2}\right) + \frac{\conj{a_1}a_2}{\Delta_2} \right)\le 0\\
      \Re\left(2\Re(b_1)\frac{\conj{b_2}}{\Delta_2}-\frac{\conj{b_1}^2}{\Delta_1}\right) \ge 0\\
      \Re\left(\Delta_1\right) = 0 \\
      \Re\left(\Delta_2\right) = 0 \\
      \Re\left(\frac{\conj{b_2}}{\Delta_2}-\frac{\conj{b_1}}{\Delta_1}\right) = 0\\
    \end{align}
  \end{prop}
  
  \begin{proof}
    On prend l'expression de la CIOE \eqref{eq:unicite:ci3:ci3} et on l’intègre avec des produits scalaires judicieusement choisis.

    \begin{multline}
      \label{eq:unicite:ci3:csu3-1}
      \int_\Gamma \vJ\cdot\conj{\eqref{eq:unicite:ci3:ci3}}ds \Rightarrow
      \int_\Gamma \vJ \cdot \conj{\vE_t} ds  + \conj{b_1} \int_\Gamma \vJ\cdot \LD\conj{\vE_t} ds - \conj{b_2} \int_\Gamma \vJ \LR\conj{\vE_t} ds \\
      = \conj{a_0} \int_\Gamma |\vJ|^2ds - \conj{a_1} \int_\Gamma |\tdivs \vJ|^2 ds - \conj{a_2} \int_\Gamma |\vn \cdot \trots \vJ|^2 ds
    \end{multline}
    \begin{multline}
      \label{eq:unicite:ci3:csu3-2}
      \int_\Gamma \eqref{eq:unicite:ci3:ci3} \cdot \conj{\vE_t} ds \Rightarrow
      \int_\Gamma |\vE_t|^2 ds  - b_1 \int_\Gamma | \tdivs \vE |^2 ds - b_2 \int_\Gamma | \vn \cdot \trots \vE_t|^2 ds \\
      = a_0 \int_\Gamma \vJ\cdot \conj{\vE_t}ds + a_1 \int_\Gamma \conj{\vE_t} \LD \vJ ds - a_2 \int_\Gamma \conj{\vE_t} \cdot \LR \vJ ds
    \end{multline}
    \begin{multline}
      \label{eq:unicite:ci3:csu3-3}
      \int_\Gamma \vJ \cdot \LR ( \conj{\eqref{eq:unicite:ci3:ci3}} ) ds \Rightarrow
      \int_\Gamma \vJ \cdot \LR \conj{\vE_t} ds  - \conj{b_2} \int_\Gamma \LR \vJ \cdot \LR \conj{\vE_t} ds \\
      =  \conj{a_0} \int_\Gamma |\vn \cdot \trots \vJ|^2ds - \conj{a_2} \int_\Gamma | \LR \vJ|^2 ds
    \end{multline}
    \begin{multline}
      \label{eq:unicite:ci3:csu3-4}
      \int_\Gamma  \LR ( \eqref{eq:unicite:ci3:ci3} ) \cdot \conj{\vE_t} ds \Rightarrow
      \int_\Gamma | \vn \cdot \trots \vE_t |^2 ds  - \conj{b_2} \int_\Gamma | \LR \vE_t|^2 ds \\
      = a_0 \int_\Gamma \conj{\vE_t} \LR \vJ ds - a_2 \int_\Gamma \LR \conj{\vE_t} \cdot \LR \vJ ds
    \end{multline}
      \begin{multline}
      \label{eq:unicite:ci3:csu3-5}
      \int_\Gamma \vJ \cdot \LD ( \conj{\eqref{eq:unicite:ci3:ci3}} ) ds \Rightarrow
      \int_\Gamma \vJ \cdot \LD \conj{\vE_t} ds  + \conj{b_1} \int_\Gamma \LD \vJ \cdot \LD \conj{\vE_t} ds \\
      = - \conj{a_0} \int_\Gamma |\tdivs \vJ|^2ds + \conj{a_1} \int_\Gamma | \LD \vJ|^2 ds
    \end{multline}
    \begin{multline}
      \label{eq:unicite:ci3:csu3-6}
      \int_\Gamma  \LD ( \eqref{eq:unicite:ci3:ci3} ) \cdot \conj{\vE_t} ds \Rightarrow
      -\int_\Gamma | \tdivs \vE_t |^2 ds  + \conj{b_1} \int_\Gamma | \LD \vE_t|^2 ds \\
      = a_0 \int_\Gamma \conj{\vE_t} \LD \vJ ds + a_1 \int_\Gamma \LD \conj{\vE_t} \cdot \LD \vJ ds
    \end{multline}
    On pose alors les définitions suivantes :
    \begin{align*}
      X&:= \int_\Gamma \vJ \cdot \conj{\vE_t} ds\\
      Y_D&:= \int_\Gamma \vJ \cdot \LD \conj{\vE_t} ds
      &Y_R&:= \int_\Gamma \vJ \cdot \LR \conj{\vE_t} ds\\
      Z_D&:= \int_\Gamma \LD \vJ \cdot \LD \conj{\vE_t} ds
      &Z_R&:= \int_\Gamma \LR \vJ \cdot \LR \conj{\vE_t} ds
    \end{align*}

    Les équations \eqref{eq:unicite:ci3:csu3-1} à \eqref{eq:unicite:ci3:csu3-4} sont équivalentes au système \(M_R X_R = F_R\) où

    \begin{align*}
      M_R&:=
      \begin{bmatrix}
        1&\conj{b_1}&-\conj{b_2}&0\\
        a_0&a_1&-a_2&0\\
        0&0&1&-\conj{b_2}\\
        0&0&a_0&-a_2\\
      \end{bmatrix},\;
      X_R =
      \begin{bmatrix}
        X\\
        Y_D\\
        Y_R\\
        Z_R
      \end{bmatrix}\\
      F_R &=
      \begin{bmatrix}
        \conj{a_0} \int_\Gamma |\vJ|^2ds - \conj{a_1} \int_\Gamma |\tdivs \vJ|^2 ds - \conj{a_2} \int_\Gamma |\vn \cdot \trots \vJ|^2 ds \\
        \int_\Gamma |\vE_t|^2 ds  - b_1 \int_\Gamma | \tdivs \vE |^2 ds - b_2 \int_\Gamma | \vn \cdot \trots \vE_t|^2 ds \\
        \conj{a_0} \int_\Gamma |\vn \cdot \trots \vJ|^2ds - \conj{a_2} \int_\Gamma | \LR \vJ|^2 ds \\
        \int_\Gamma | \vn \cdot \trots \vE_t |^2 ds  - \conj{b_2} \int_\Gamma | \LR \vE_t|^2 ds
      \end{bmatrix}
    \end{align*}

    Tandis que les équations \eqref{eq:unicite:ci3:csu3-1},\eqref{eq:unicite:ci3:csu3-2},\eqref{eq:unicite:ci3:csu3-5},\eqref{eq:unicite:ci3:csu3-6} sont équivalentes au système \(M_D X_D= F_D\) où

    \begin{align*}
      M_D&:=
      \begin{bmatrix}
        1&-\conj{b_2}&\conj{b_1}&0\\
        a_0&-a_2&a_1&0\\
        0&0&1&\conj{b_1}\\
        0&0&a_0&a_1\\
      \end{bmatrix},\;
      X_D =
      \begin{bmatrix}
        X\\
        Y_R\\
        Y_D\\
        Z_D
      \end{bmatrix}\\
      F_D &=
      \begin{bmatrix}
        \conj{a_0} \int_\Gamma |\vJ|^2ds - \conj{a_1} \int_\Gamma |\tdivs \vJ|^2 ds - \conj{a_2} \int_\Gamma |\vn \cdot \trots \vJ|^2 ds \\
        \int_\Gamma |\vE_t|^2 ds  - b_1 \int_\Gamma | \tdivs \vE |^2 ds - b_2 \int_\Gamma | \vn \cdot \trots \vE_t|^2 ds \\
        -\conj{a_0} \int_\Gamma |\tdivs \vJ|^2ds + \conj{a_1} \int_\Gamma | \LR \vJ|^2 ds \\
        -\int_\Gamma | \tdivs \vE_t |^2 ds  + \conj{b_1} \int_\Gamma | \LR \vE_t|^2 ds
      \end{bmatrix},\;
    \end{align*}

    On note dans la suite \(\Delta_i = a_i-\conj{b_i}a_0\), \(i=1,2\). On suppose que ces système aient une unique solution. Alors on obtient la première condition suffisante:

    \begin{equation}
      \label{eq:unicite:ci3:csu3-cn-det}
      \Delta_1\Delta_2 \not = 0
    \end{equation}

    \begin{minipage}{0.49\textwidth}
      \textbf{Cas LR}:
      \begin{align}
        \label{eq:unicite:ci3:csu3r-j2}&\Re\left(a_0\conj{a_2}\Delta_2\right) \ge 0 \\
        \label{eq:unicite:ci3:csu3r-e2}&\Re\left(\frac{\conj{b_2}}{\Delta_2}\right) \le 0 \\
        \label{eq:unicite:ci3:csu3r-jdj}&\Re\left(\conj{a_0}a_1\left(\frac{\conj{b_1}}{\Delta_1}-\frac{\conj{b_2}}{\Delta_2}\right) + \frac{\conj{a_1}a_2}{\Delta_2} \right)\le 0\\
        \label{eq:unicite:ci3:csu3r-ede}&\Re\left(2\Re(b_1)\frac{\conj{b_2}}{\Delta_2}-\frac{\conj{b_1}^2}{\Delta_1}\right) \ge 0\\
        \label{eq:unicite:ci3:csu3r-jrj}&\Re\left(|a_2|^2\Delta_2\right) \le 0 \\
        \label{eq:unicite:ci3:csu3r-ere}&\Re\left(|b_2|^2\Delta_2\right) \ge 0 \\
        \label{eq:unicite:ci3:csu3r-rj2}&\Re\left(|a_1|^2\left(\frac{\conj{b_1}}{\Delta_1}-\frac{\conj{b_2}}{\Delta_2}\right)\right)\ge 0\\
        \label{eq:unicite:ci3:csu3r-re2}&\Re\left(|b_1|^2\left(\frac{\conj{b_1}}{\Delta_1}-\frac{\conj{b_2}}{\Delta_2}\right)\right)\le 0
      \end{align}
      \eqref{eq:unicite:ci3:csu3r-jrj} et \eqref{eq:unicite:ci3:csu3r-ere} impliquent :
      \begin{equation}
        \Re\left(\Delta_2\right) = 0\\\
      \end{equation}
      \eqref{eq:unicite:ci3:csu3r-rj2} et \eqref{eq:unicite:ci3:csu3r-re2} impliquent :
      \begin{equation}
        \Re\left(\frac{\conj{b_1}}{\Delta_1}-\frac{\conj{b_2}}{\Delta_2}\right) = 0\\\
      \end{equation}
    \end{minipage}
    \begin{minipage}{0.49\textwidth}
      \textbf{Cas LD}:
      \begin{align}
        \label{eq:unicite:ci3:csu3d-j2}&\Re\left(a_0\conj{a_1}\Delta_1\right) \ge 0 \\
        \label{eq:unicite:ci3:csu3d-e2}&\Re\left(\frac{\conj{b_1}}{\Delta_1}\right) \le 0 \\
        \label{eq:unicite:ci3:csu3d-jrj}&\Re\left(\conj{a_0}a_2\left(\frac{\conj{b_2}}{\Delta_2}-\frac{\conj{b_2}}{\Delta_2}\right) + \frac{\conj{a_2}a_1}{\Delta_1} \right)\le 0\\
        \label{eq:unicite:ci3:csu3d-ere}&\Re\left(2\Re(b_2)\frac{\conj{b_1}}{\Delta_1}-\frac{\conj{b_2}^2}{\Delta_2}\right) \ge 0\\
        \label{eq:unicite:ci3:csu3d-jdj}&\Re\left(|a_1|^2\Delta_1\right) \le 0 \\
        \label{eq:unicite:ci3:csu3d-ede}&\Re\left(|b_1|^2\Delta_1\right) \ge 0 \\
        \label{eq:unicite:ci3:csu3d-dj2}&\Re\left(|a_2|^2\left(\frac{\conj{b_2}}{\Delta_2}-\frac{\conj{b_1}}{\Delta_1}\right)\right)\ge 0\\
        \label{eq:unicite:ci3:csu3d-de2}&\Re\left(|b_2|^2\left(\frac{\conj{b_2}}{\Delta_2}-\frac{\conj{b_1}}{\Delta_1}\right)\right)\le 0
      \end{align}
      \eqref{eq:unicite:ci3:csu3d-jdj} et \eqref{eq:unicite:ci3:csu3d-ede} impliquent :
      \begin{equation}
        \Re\left(\Delta_1\right) = 0\\\
      \end{equation}
      \eqref{eq:unicite:ci3:csu3d-dj2} et \eqref{eq:unicite:ci3:csu3d-de2} impliquent :
      \begin{equation}
        \Re\left(\frac{\conj{b_1}}{\Delta_1}-\frac{\conj{b_2}}{\Delta_2}\right) = 0\\\
      \end{equation}
    \end{minipage}
  \end{proof}
  %Pour le système \(M_D X_D = F_D\), les conditions sont identiques à une permutation des indices 1 et 2 près.
  De part leur nombre, ces CSU sont très contraignantes et ne permettent pas de retrouver des CSU des CIOE d'ordres inférieurs lorsque l'on annule les coefficients \(b_1, b_2\). 


\subsection{CSU de Payen}

  \begin{prop}
    D'autre CSU sont
    \begin{align}
      \Re\left(a_0\right)\ge 0 \\
      \Re\left(a_1 - \frac{\conj{b_1a_0}a_1}{\Delta_1}\right) \le 0 \\
      \Re\left(a_2 - \frac{\conj{b_2a_0}a_2}{\Delta_2}\right) \le 0 \\
      \Re\left(b_1\Delta_1\right) = 0 \\
      \Re\left(b_2\Delta_2\right) = 0 \\
      \Im\left(b_1\Delta_1\right)\Im(b_1)\ge 0\\
      \Im\left(b_2\Delta_2\right)\Im(b_2)\ge 0
    \end{align}
  \end{prop}

  \begin{proof}
    En se basant sur la méthode précédente, on remarque que l'on peut déterminer les inconnus \((Y_R,Z_R)\) (resp. \((Y_D,Z_R)\)) uniquement en fonction des équations \eqref{eq:unicite:ci3:csu3-3} et \eqref{eq:unicite:ci3:csu3-4} (resp. \eqref{eq:unicite:ci3:csu3-5} et \eqref{eq:unicite:ci3:csu3-6}).

    On déduit donc que si \(\Delta_1 \not = 0\) et \(\Delta_2 \not = 0\) alors

    \begin{align}
      Y_R &= \frac{1}{\Delta_2}\left(a_2\left[\conj{a_0}\int_\Gamma \vJ\cdot\LR\conj{\vJ} - \conj{a_2}||\LR J||^2\right]  -\conj{b_2}\left[\int_\Gamma \conj{\vE}\LR{\vE} - b_2 ||\LR \vE ||^2\right]\right) \\
      Y_D &= \frac{1}{\Delta_1}\left(a_1\left[\conj{a_0}\int_\Gamma \vJ\cdot\LD\conj{\vJ} + \conj{a_1}||\LD J||^2\right]  -\conj{b_1}\left[\int_\Gamma \conj{\vE}\LD{\vE} + b_1 ||\LD \vE ||^2\right]\right)
    \end{align}

    Il reste alors à utiliser l'équation \eqref{eq:unicite:ci3:csu3-1} pour obtenir
    \begin{equation}
      X = -\conj{b_1} Y_D + \conj{b_2} Y_R + \conj{a_0} || \vJ ||^2 + \conj{a_1} \int_\Gamma \vJ \cdot \LD \conj{\vJ} - \conj{a_2} \int_\Gamma \vJ \cdot \LR \conj{\vJ}
    \end{equation}

    \begin{multline}
      X = \conj{a_0} || \vJ ||^2 - \conj{a_1} || \vdivs \vJ ||^2 - \conj{a_2} || \vrots \vJ ||^2
      \\
      + \frac{\conj{b_2}}{\Delta_2}\left(a_2\left(\conj{a_0}||\vrots \vJ||^2 - \conj{a_2}||\LR J||^2\right)  -\conj{b_2}\left(||\vrots\vE||^2 - b_2 ||\LR \vE ||^2\right)\right)
      \\
      - \frac{\conj{b_1}}{\Delta_1}\left(a_1\left(-\conj{a_0}||\vdivs\vJ||^2 + \conj{a_1}||\LD J||^2\right)  -\conj{b_1}\left(-||\vdivs\vE||^2 + b_1 ||\LD \vE ||^2\right)\right)
    \end{multline}

    On factorise les termes en \(\vJ\)

    \begin{multline}
      X = \conj{a_0} || \vJ ||^2 - \left(a_1 - \frac{\conj{b_1a_0}a_1}{\Delta_1}\right) || \vdivs \vJ ||^2 - \left(a_2 - \frac{\conj{b_2a_0}a_2}{\Delta_2}\right) || \vrots \vJ ||^2
      \\
      + \frac{\conj{b_2}}{\Delta_2}\left( - |a_2|^2||\LR \vJ||^2  - \conj{b_2}\left(||\vrots\vE||^2 - b_2 ||\LR \vE ||^2\right)\right) 
      \\
      - \frac{\conj{b_1}}{\Delta_1}\left( |a_1|^2||\LD \vJ||^2  - \conj{b_1}\left(-||\vdivs\vE||^2 + b_1 ||\LD \vE ||^2\right)\right)
    \end{multline}

    On développe tous les termes
    \begin{multline}
      X = \conj{a_0} || \vJ ||^2 - \left(a_1 - \frac{\conj{b_1a_0}a_1}{\Delta_1}\right) || \vdivs \vJ ||^2 - \left(a_2 - \frac{\conj{b_2a_0}a_2}{\Delta_2}\right) || \vrots \vJ ||^2
      \\
      - \frac{\conj{b_2}|a_2|^2}{\Delta_2}||\LR \vJ||^2  -  \frac{\conj{b_2}^2}{\Delta_2}||\vrots\vE||^2 +  \frac{|b_2|}{\Delta_2} ||\LR \vE ||^2
      \\
      - \frac{\conj{b_1}|a_1|^2}{\Delta_1}||\LD \vJ||^2  - \frac{\conj{b_1}^2}{\Delta_1}||\vdivs\vE||^2 + \frac{|b_1|^2}{\Delta_1} ||\LD \vE ||^2
    \end{multline}

    On impose alors à la partie réelle de chaque terme d'être positive, et on obtient les CSU suivantes :

    \begin{equation}
      \Re\left(a_0\right)\ge 0
    \end{equation}
    \begin{minipage}{0.5\textwidth}
      \begin{align}
        \Re\left(a_1 - \frac{\conj{b_1a_0}a_1}{\Delta_1}\right) \le 0 \\
        \Re\left(a_2 - \frac{\conj{b_2a_0}a_2}{\Delta_2}\right) \le 0 \\
        \Re\left(\frac{|a_1|^2\conj{b_1}}{\Delta_1}\right) \le 0 \\
        \Re\left(\frac{|a_2|^2\conj{b_2}}{\Delta_2}\right) \le 0
      \end{align}
    \end{minipage}
    \begin{minipage}{0.5\textwidth}
      \begin{align}
        \Re\left(\frac{\conj{b_1}^2}{\Delta_1}\right) \le 0 \\
        \Re\left(\frac{\conj{b_2}^2}{\Delta_2}\right) \le 0 \\
        \Re\left(\frac{|b_1|^2\conj{b_1}}{\Delta_1}\right) \ge 0 \\
        \Re\left(\frac{|b_2|^2\conj{b_2}}{\Delta_2}\right) \ge 0
      \end{align}
    \end{minipage}

    On remarque alors que certaines CSU peuvent se combiner et imposent que les parties réelles de \(b_1\Delta_1,b_2\Delta_2\) soient nulles.

    \begin{align}
      \Re\left(a_0\right)\ge 0 \\
      \Re\left(a_1 - \frac{\conj{b_1a_0}a_1}{\Delta_1}\right) \le 0 \\
      \Re\left(a_2 - \frac{\conj{b_2a_0}a_2}{\Delta_2}\right) \le 0 \\
      \Re\left(b_1\Delta_1\right) = 0 \\
      \Re\left(b_2\Delta_2\right) = 0 \\
      \Im\left(b_1\Delta_1\right)\Im(b_1)\ge 0\\
      \Im\left(b_2\Delta_2\right)\Im(b_2)\ge 0
    \end{align}
  \end{proof}

  On a réussi à réduire le nombre de CSU et en plus, fixer \(b_1=b_2=0\) permet de retomber sur les CSU de la CI4.

\subsection{CSU de Lafitte-Stupfel}

  \begin{prop}
    Soit \(z = \left(1 - \frac{b_1a_0}{a_1} - \frac{b_2a_0}{a_2}\right) \). Des CSU qui assurent la \gls{acr-cgu} sont
    \begin{align}
      \Re\left(\conj{a_0}z\right) \ge 0
      \\
      \Re\left(\conj{a_1}z\right) \le 0
      \\
      \Re\left(\conj{a_2}z\right) \le 0
      \\
      \Re\left(\frac{b_1}{a_1}\right) \ge 0
      \\
      \Re\left(\frac{b_2}{a_2}\right) \ge 0
      \\
      \Re\left(a_0\right) \ge 0
      \\
      \Re\left(a_1\right) \le 0
      \\
      \Re\left(a_2\right) \le 0
      \\
      \Re\left(\frac{b_1\conj{a_2}}{a_1\conj{a_0}}\right) \le 0
      \\
      \Re\left(\frac{b_2\conj{a_1}}{a_2\conj{a_0}}\right) \le 0
    \end{align}
  \end{prop}
  \begin{proof}
    Par définition de la CIOE, on a

    \begin{align}
      X &= \int_\Gamma \left(a_0\oI + a_1 \LD - a_2 \LR \right)^{-1}\left(\oI + b_1 \LD - b_2 \LR \right) \vE_t\cdot \conj{\vE_t}
    \end{align}

    On développe simplement chaque terme

    \begin{multline}
      X = \int_\Gamma \left(a_0\oI + a_1 \LD - a_2 \LR \right)^{-1}
      \\
      + b_1 \left(a_0\oI + a_1 \LD - a_2 \LR \right)^{-1}\LD
      \\
      \left.
      - b_2 \left(a_0\oI + a_1 \LD - a_2 \LR \right)^{-1}\LR \right) \vE_t\cdot \conj{\vE_t}
    \end{multline}

    L'astuce pour obtenir réside dans les égalités suivantes, valables si \(a_1\) et \(a_2\) sont non-nuls.
    \begin{align}
      \LD & = \frac{a_0 + a_1 \LD - a_0}{a_1}
      \\
      \LR & = -\frac{a_0 - a_2 \LD - a_0}{a_2}
    \end{align}

    On pose
    \begin{equation}
      z = \left(1 - \frac{b_1a_0}{a_1} - \frac{b_2a_0}{a_2}\right)
    \end{equation}

    On déduit de ce qui précède que

    \begin{multline}
      X = \int_\Gamma z\left(a_0\oI + a_1 \LD - a_2 \LR \right)^{-1}
      \\
      + \frac{b_1}{a_1} \left(a_0\oI + a_1 \LD - a_2 \LR \right)^{-1}\left(a_0+a_1\LD\right)
      \\
      - \frac{b_2}{a_2} \left(a_0\oI + a_1 \LD - a_2 \LR \right)^{-1}\left(a_0-a_2\LR\right) \vE_t\cdot \conj{\vE_t}
    \end{multline}

    On définit

    \newcommand{\vD}{\vect{D}}
    \newcommand{\vF}{\vect{F}}

    \begin{align}
      \vD & = \left(a_0 \oI + a_1 \LD - a_2\LR \right)^{-1} \vE_t
      \\
      \vF_1 & = \left(\oI - a_2 \left( a_0 + a_1\LD\right)^{-1}\LR\right)^{-1} \vE_t
      \\
      \vF_2 & = \left(\oI + a_1 \left( a_0 - a_2\LR\right)^{-1}\LD\right)^{-1} \vE_t
    \end{align}

    Alors immédiatement, on a

    \begin{multline}
      X = \int_\Gamma z \vD \cdot \left(\conj{a_0} \oI + \conj{a_1} \LD - \conj{a_2}\LR\right)\conj{\vD}
      \\
      + \frac{b_1}{a_1} \left(\oI - \conj{a_2} \left( \conj{a_0} + \conj{a_1}\LD\right)^{-1}\LR\right)\conj{\vF_1}\cdot\vF_1
      \\
      + \frac{b_2}{a_2} \left(\oI + \conj{a_1} \left( \conj{a_0} - \conj{a_2}\LR\right)^{-1}\LD\right)\conj{\vF_2}\cdot\vF_2
    \end{multline}

    Finalement posons

    \newcommand{\vG}{\vect{G}}

    \begin{align}
      \vG_1 & = \left(\conj{a_0} \oI + \conj{a_1} \LD \right)^{-1}\LR \conj{\vF_1}
      \\
      \vG_2 & = \left(\conj{a_0} \oI - \conj{a_2} \LR \right)^{-1}\LD \conj{\vF_2}
    \end{align}

    Puisque \(\LD\) (resp. \(\LR\)) commute avec lui-même, on a les égalités suivantes

    \begin{align}
      \LD\left(\conj{a_0} \oI + \conj{a_1} \LD \right)&=\left(\conj{a_0} \oI + \conj{a_1} \LD \right)\LD
      \\
      \LR\left(\conj{a_0} \oI - \conj{a_2} \LR \right)&=\left(\conj{a_0} \oI - \conj{a_2} \LR \right)\LR
    \end{align}

    Or on a démontré que \(\LD\LR=\LR\LD=0\), et ainsi

    \begin{align}
      \LD\LR\conj{\vF_1} &= \LD\left(\conj{a_0} \oI + \conj{a_1} \LD \right)\vG_1
      \\
      0 & =\left(\conj{a_0} \oI + \conj{a_1} \LD \right)\LD\vG_1
    \end{align}

    Si l'on suppose alors que \(\Re(a_0) \ge 0 \) et \(\Re(a_1) \le 0\) (resp. \(\Re(a_2)\le0\)), alors \(\left(\conj{a_0} \oI + \conj{a_1} \LD \right)\) (resp. \(\left(\conj{a_0} \oI - \conj{a_2} \LR \right)\)) est injectif et donc on déduit que

    \begin{align}
      \LD\vG_1 = 0
      \\
      \LR\vG_2 = 0
    \end{align}

  % \begin{TODO}
  %   En fait plus largement, il faut que \(\Re(a_0)\) et \(\Re(a_{1/2})\) soit de signes opposées. Mais il suffit d'avoir une des deux.
  % \end{TODO}

    Or par définition \(\LR\conj{\vF_1} = \left(\conj{a_0} \oI + \conj{a_1} \LD \right)\vG_1\) (resp. \(\LR\conj{\vF_2} = \left(\conj{a_0} \oI - \conj{a_2} \LR \right)\vG_2\)) donc
    \begin{align}
      \LR\conj{\vF_1} &= \conj{a_0}\vG_1
      \\
      \LD\conj{\vF_2} &= \conj{a_0}\vG_2
    \end{align}

    On réinjecte ce résultat dans la définition de \(X\)

    \begin{multline}
      X = \int_\Gamma z \vD \cdot \left(\conj{a_0} \oI + \conj{a_1} \LD - \conj{a_2}\LR\right)\conj{\vD}
      \\
      + \frac{b_1}{a_1} ||\vF_1||^2 - \frac{b_1\conj{a_2}}{a_1\conj{a_0}} \LR\conj{\vF_1}\cdot\vF_1
      \\
      + \frac{b_2}{a_2} ||\vF_2||^2 + \frac{b_2\conj{a_1}}{a_2\conj{a_0}} \LD\conj{\vF_2}\cdot\vF_2
    \end{multline}

    Les CSU sont alors

    \begin{minipage}{0.5\textwidth}
    \begin{align}
      \Re\left(\conj{a_0}z\right) \ge 0
      \\
      \Re\left(\conj{a_1}z\right) \le 0
      \\
      \Re\left(\conj{a_2}z\right) \le 0
      \\
      \Re\left(\frac{b_1}{a_1}\right) \ge 0
      \\
      \Re\left(\frac{b_2}{a_2}\right) \ge 0
    \end{align}
    \end{minipage}
    \begin{minipage}{0.49\textwidth}
    \begin{align}
      \Re\left(a_0\right) \ge 0
      \\
      \Re\left(a_1\right) \le 0
      \\
      \Re\left(a_2\right) \le 0
      \\
      \Re\left(\frac{b_1\conj{a_2}}{a_1\conj{a_0}}\right) \le 0
      \\
      \Re\left(\frac{b_2\conj{a_1}}{a_2\conj{a_0}}\right) \le 0
    \end{align}
    \end{minipage}
  \end{proof}