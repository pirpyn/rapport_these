\section{Condition suffisante d'unicité générale}

  On rappelle les résultats d’unicité démontrés par \cite[chapitre 2, section 11]{cessenat_mathematical_1996}.

  Soit \(\OO\) un domaine fermé borné de \(\RR^3\) et \(\Gamma\) sa frontière.
  À l'extérieur de cette surface, on définit \(k_0 \in \RR_+^*\), le nombre d'onde dans le vide et \(\eta_0 \in \RR^*\) l'impédance du vide.
  À l'intérieur, on définit les constantes relatives \(\eps,\mu \in \CC\), invariantes par translation dans ce domaine, dont on déduit le nombre d'onde intérieur \(k = k_0 \sqrt{\mu\eps}\).

  Soit \((\vE,\vH)\) dans \((\Hrot(\OO)) \times (\Hrot(\OO))\) tels que:
  \begin{align}
  \left\lbrace
    \begin{matrix}
      \trot \vE + ik \mu \vH &= 0
      \\
      \trot \vH - ik \eps \vE &= 0
      % \\\label{eq:unicite:form_var:TR}
      % \Tr(\vE_t) &= - \vn_{Y_R} \pvect \vH && \text{sur \(\Gamma(0,R)\)}
    \end{matrix}
    \right. && \text{dans \(\OO\)}
  \end{align}
  % Où \(\Tr\) est l'opérateur de capacité \cite[p.~200]{nedelec_acoustic_2001}, \(\vn_{Y_R}\) la normale unitaire sortante à \(\Gamma(0,R)\).\\

  Soit \(\vect{\phi}\) dans \(\Hrot(\OO)\) une fonction test. Ces équations aboutissent à la forme variationnelle suivante :
  \begin{prop}
    Trouver \(\vE \in \left(\Hrot(\OO)\right)\), \(\forall \vect \phi \in \left(\Hrot(\OO)\right)\)
    \[
      a(\vE,\vect\phi) = 0
    \]
  \end{prop}

  où a est une forme sesquilinéaire telle que, soit \(\vn\) la normale unitaire sortante à \(\Gamma\).
  \begin{align*}
    a(\vE,\vect\phi) &:=  \frac{1}{ik\mu} \int_\OO \trot \vE \cdot \trot \conj{\vect{\phi}} dx + ik\eps\int_\OO\vE\cdot\conj{\vect{\phi}} dx
     %+ \int_{Y_R} \conj{ \vect \phi } \cdot \Tr(\vE_t)ds
      - \int_\Gamma \left(\vn \pvect \frac{\trot \vE}{ik\mu}\right) \cdot \conj{\vect \phi} ds(\vx) \\
   \end{align*}

  \begin{proof}
    Partons de la première équation.
    \begin{align}
          0 & = \frac{\trot \vE}{ik\mu} + \vH
          \\ \intertext{Appliquons lui le rotationnel.}
          0 & = \trot \frac{\trot \vE}{ik\mu} + \trot \vH
          \\ \intertext{On peut utiliser la seconde équation pour n'avoir plus que \(\vE\).}
          0 & = \trot \frac{\trot \vE}{ik\mu} + ik\eps \vE
          \\ \intertext{Réalisons le produit scalaire avec la fonction test.}
          0 & = \int_\OO{ \trot \frac{\trot \vE}{ik\mu}\cdot \conj{\vect{\phi}}} +  \int_\OO ik\eps \vE \cdot \conj{\vect{\phi}}
          \\ \intertext{Utilisons la formule de Green du rotationnel (voir \cite[eq.~(A1.32)]{bladel_electromagnetic_2007}).}
          0 & = \int_\OO{ \frac{\trot \vE}{ik\mu}\cdot \trot \conj{\vect{\phi}}} - \int_\Gamma \left( \conj{\vect \phi} \pvect \frac{\trot \vE}{ik\mu}\right)  \cdot \vn ds(\vx) + \int_\OO ik\eps \vE \cdot \conj{\vect{\phi}}
          \\ \intertext{On arrange les termes.}
          0 & = \int_\OO \frac{\trot \vE}{ik\mu}\cdot \trot \conj{\vect{\phi}} - \int_\Gamma \left(\vn \pvect \frac{\trot \vE}{ik\mu}\right) \cdot \conj{\vect \phi} ds(\vx) +  \int_\OO ik\eps \vE \cdot \conj{\vect{\phi}}
      \end{align}
  \end{proof}

  On rappelle la définition de \cite[p.~59]{cessenat_mathematical_1996}.
  \begin{defn}[Coercivité d'une forme sesquilinéaire]
    Une forme sesquilinéaire \(a(\vu,\vv)\) est coercive dans \(\Hrot(\OO)\) si \(\exists \alpha > 0\) tel que
    \[
      |\Re(a(\vu,\vu))| \ge \alpha ||\vu||_{\Hrot(\OO)}^2 = \alpha \left( || \trot \vu ||_{L^2}^2 + || \vu ||_{L^2}^2\right) \, \forall \vu \in \Hrot(\OO)
    \]
   \end{defn}

  %La forme bilinéaire \(a\) est coercive \Gamma'il existe une constante réel positive \(\mathcal{C}\) telle que \(|a(\vE,\vE)|^2 \ge \mathcal{C}  || \vE ||_{\Hrot}^4 = \mathcal{C}\left( || \trot \vE ||_{L^2}^2  + || \vE ||_{L^2}^2 \right)^2 \).

  %Supposons \(\eps, \mu\) constants et
  On définit en tout point de \(\Gamma\) la trace tangentielle de \(\vH\) que l'on note \(\vJ = \vn \pvect \vH\). On définit alors la quantité suivante
  \begin{align}
    X &= \int_\Gamma \vJ \cdot \conj{\vect \phi} ds(\vx)
  \end{align}

  La partie réelle de la forme bilinéaire \(a\) s'écrit donc
  \begin{equation}
    \label{eq:unicite:form_var:decomp_form_bilin_1}
    |\Re(a(\vE,\vE))| = \left|\frac{\Im(\mu)}{k} || \trot \vE ||_{L^2}^2  + k \Im(\eps) || \vE ||_{L^2}^2
    %+ \Re(C)
    - \Re(X)\right|
  \end{equation}

\subsection{Conditions suffisantes d'unicités}

  \begin{hyp}[Hypothèses de coercivité en convention \(i\omega t\)]\label{hyp:unicite:form_var:hyp_coercivite}
    ~{}

    \begin{enumerate}
      \item \(\Im(\eps)\) et \(\Im(\mu)\) sont de même signe.
      \item \(\Im(\eps)\) et \(\Re(X)\) sont de signes opposés\footnote{En convention \(-i\omega t\), il faut que le signe soit le même}.
      %\item \(\Re(C)\) et \(\Re(X)\) sont de même signe.
    \end{enumerate}
  \end{hyp}

\subsection{Cas des matériaux avec pertes}

  Les matériaux avec pertes sont tels que au moins l'une des deux constantes a une partie imaginaire non-nulle.

  Si l'hypothèse \ref{hyp:unicite:form_var:hyp_coercivite} est vérifiée alors \(a\) est coercive ( \(\alpha = \min(-\Im(\mu)k^{-1},-\Im(\eps)k)\)) donc il y a unicité des solutions du problème de Maxwell d'après Lax-Milgram (voir \cite{cessenat_mathematical_1996}).

  Comme nous sommes en convention \(e^{i\omega t}\) et que nous imposons que le signe de \(\Im(\eps)\) et \(\Im(\mu)\) soit négatif
  %, sachant que d'après \cite[p.~97]{nedelec_acoustic_2001} \(\Re(C)\ge 0\)
  alors l'unicité est assurée par la
  \begin{defn}[\glsentrydesc{acr-cgu}]~\\
    \begin{equation}\label{eq:unicite:form_var:cgu}
      \Re\left(\int_\Gamma \vJ \cdot \conj{\vE_t} ds\right) \ge 0
    \end{equation}
  \end{defn}
  Dans la suite de la thèse, nous noterons souvent cette quantité par \gls{mat-q}. C'est par définition la partie réelle du flux sortant du vecteur de Poynting à travers la surface de l'objet. 

\subsection{Cas des matériaux sans pertes}

  Si \(\Im(\mu) = \Im(\eps) = 0\), le résultat précédent n'est plus valable car \(\alpha = 0\). Cependant, l'unicité est assuré par l’alternative de Fredholm (voir \cite[Théorème 8]{cessenat_mathematical_1996}) dont la conclusion est aussi que \ref{eq:unicite:form_var:cgu} doit être vérifiée pour assurée l'unicité.

  Maintenant que nous connaissant la condition suffisante à vérifier, nous allons montrer quelles conditions sur les coefficients des CIOE permettent la vérifier.