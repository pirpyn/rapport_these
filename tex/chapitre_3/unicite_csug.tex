\section{Condition suffisante d'unicité générale}

  Soit \(\OO\) un domaine fermé borné de \(\RR^3\) plongé dans le vide et \(\Gamma\) sa frontière.
  À l'extérieur de cette surface, on définit \(k_0 \in \RR_+^*\), le nombre d'onde dans le vide.

  On s’intéresse à la propagation des ondes électromagnétiques à l'extérieur de cette objet. Les champs associés sont solutions des équations de Maxwell harmonique \(e^{i\w t}\)\footnote{voir annexe \ref{sec:annex:maxwell_equation}} :

  Soit \((\vE,\vH)\) dans \((\Hrot(\OO^c)) \times (\Hrot(\OO^c))\) tels que:
  \begin{align}
  \left\lbrace
    \begin{matrix}
      \trot \vE + i k_0 \vH &= 0
      \\
      \trot \vH - i k_0 \vE &= 0
    \end{matrix}
    \right. && \text{dans \(\OO^c\)}
  \end{align}

  Ce problème étant non bornée, on définit \(B_R\) le boule de rayon \(R\) suffisamment grand et qui englobe \(\OO\). On cherche alors les solutions dans l'espace entre \(\Gamma\) et \(S_R\), la sphère de rayon \(R\) où est définit la condition de rayonnement \eqref{eq:unicite:form_var:TR}.
  \begin{equation}
    \label{eq:unicite:form_var:TR}
    \begin{aligned}
    \Tr(\vE_t) = - \vn_{S_R} \pvect \vH && \text{sur \(S(0,R)\)}
    \end{aligned}
  \end{equation}
  \(\Tr\) est l'opérateur de capacité définit par \cite[p.~200]{nedelec_acoustic_2001}, \(\vn_{S_R}\) la normale unitaire sortante à \(S(0,R)\).

  Soit \(\vect{\phi}\) dans \(\Hrot(\OO^c)\) une fonction test. Ces équations aboutissent à la forme variationnelle suivante :
  \begin{prop}
    Trouver \(\vE \in \left(\Hrot(\OO^c)\right)\), \(\forall \vect \phi \in \left(\Hrot(\OO^c)\right)\)
    \[
      a(\vE,\vect\phi) = 0
    \]
  \end{prop}

  où a est une forme sesquilinéaire telle que, soit \(\vn\) la normale unitaire sortante à \(\Gamma\).
  \begin{equation*}
    \begin{aligned}
    a(\vE,\vect\phi) &:=  \frac{1}{ik_0} \int_\OO \trot \vE \cdot \trot \conj{\vect{\phi}} dx + ik_0\int_\OO\vE\cdot\conj{\vect{\phi}} dx
      \\ 
      & \quad + \int_{S_R} \conj{ \vect \phi } \cdot \Tr(\vE_t)ds - \int_\Gamma \left(\vn \pvect \frac{\trot \vE}{ik_0}\right) \cdot \conj{\vect \phi} ds(\vx)
    \end{aligned}
   \end{equation*}

  \begin{proof}
    Partons de la première équation.
    \begin{align}
          0 & = \frac{\trot \vE}{ik_0} + \vH
          \\ \intertext{Appliquons lui le rotationnel.}
          0 & = \trot \frac{\trot \vE}{ik_0} + \trot \vH
          \\ \intertext{On peut utiliser la seconde équation pour n'avoir plus que \(\vE\).}
          0 & = \trot \frac{\trot \vE}{ik_0} + ik_0 \vE
          \\ \intertext{Réalisons le produit scalaire avec la fonction test.}
          0 & = \int_\OO{ \trot \frac{\trot \vE}{ik_0}\cdot \conj{\vect{\phi}}} +  \int_\OO ik_0 \vE \cdot \conj{\vect{\phi}}
          \\ \intertext{Utilisons la formule de Green du rotationnel (voir \cite[eq.~(A1.32)]{bladel_electromagnetic_2007}). On rappelle que \(\vn_\Gamma\) est la normale entrante à travers \(\Gamma\) et \(\vn_{S_R}\) la normale sortante à travers \(S_R\) }
          0 & = \int_\OO{ \frac{\trot \vE}{ik_0}\cdot \trot \conj{\vect{\phi}}} - \int_{S_R} \left( \conj{\vect \phi} \pvect \frac{\trot \vE}{ik_0}\right)  \cdot \vn_{S_R} ds(\vx) 
          \\ \notag
          & \qquad \qquad + \int_\Gamma \left( \conj{\vect \phi} \pvect \frac{\trot \vE}{ik_0}\right)  \cdot \vn_\Gamma ds(\vx) + \int_\OO ik_0 \vE \cdot \conj{\vect{\phi}}
          \\ \intertext{On utilise la condition de rayonnement}
          0 & = \int_\OO{ \frac{\trot \vE}{ik_0}\cdot \trot \conj{\vect{\phi}}} - \int_{S_R} \Tr(\vE_t)  \cdot \conj{\vect{\phi}} ds(\vx) 
          \\ \notag
          & \qquad \qquad + \int_\Gamma \left( \conj{\vect \phi} \pvect \frac{\trot \vE}{ik_0}\right)  \cdot \vn_\Gamma ds(\vx) + \int_\OO ik_0 \vE \cdot \conj{\vect{\phi}}
          \\ \intertext{On permute les termes de l'intégrale sur \(\Gamma\).}
          0 & = \int_\OO \frac{\trot \vE}{ik_0}\cdot \trot \conj{\vect{\phi}} + \int_{S_R} \Tr(\vE_t)  \cdot \conj{\vect{\phi}} ds(\vx) 
          \\ \notag
          & \qquad \qquad - \int_\Gamma \left(\vn_\Gamma \pvect \frac{\trot \vE}{ik_0}\right) \cdot \conj{\vect \phi} ds(\vx) +  \int_\OO ik_0 \vE \cdot \conj{\vect{\phi}}
      \end{align}
  \end{proof}

  % On peut alors conclure et établir une condition suffisante sur l'unicité des solutions.

  % On rappelle la définition de \cite[p.~59]{cessenat_mathematical_1996}.
  % \begin{defn}[Coercivité d'une forme sesquilinéaire]
  %   Une forme sesquilinéaire \(a(\vu,\vv)\) est coercive dans \(\Hrot(\OO)\) si \(\exists \alpha > 0\) tel que
  %   \[
  %     |\Re(a(\vu,\vu))| \ge \alpha ||\vu||_{\Hrot(\OO)}^2 = \alpha \left( || \trot \vu ||_{L^2}^2 + || \vu ||_{L^2}^2\right) \, \forall \vu \in \Hrot(\OO)
  %   \]
  %  \end{defn}

  On définit en tout point de \(\Gamma\) la trace tangentielle de \(\vH\) que l'on note \(\vJ = \vn_\Gamma \pvect \vH\). On définit les quantité suivantes
  \begin{align}
    X &= \int_\Gamma \vJ \cdot \conj{\vect \phi} ds(\vx)
    \\
    C &= \int_{S_R} \Tr(\vE_t)  \cdot \conj{\vE} ds(\vx)
  \end{align}

  % Sachant que \(k_0\) est réel, la partie réelle de la forme bilinéaire \(a\) s'écrit donc
  % \begin{equation}
  %   \label{eq:unicite:form_var:decomp_form_bilin_1}
  %   |\Re(a(\vE,\vE))| = \left| \Re(C) + \Re(X)\right|
  % \end{equation}


% \subsection{Conditions suffisantes d'unicités}

%   \begin{hyp}[Hypothèses de coercivité en convention \(i\omega t\)]\label{hyp:unicite:form_var:hyp_coercivite}
%     ~{}

%     \begin{enumerate}
%       \item \(\Im(\eps)\) et \(\Im(\mu)\) sont de même signe.
%       \item \(\Im(\eps)\) et \(\Re(X)\) sont de signes opposés\footnote{En convention \(-i\omega t\), il faut que le signe soit le même}.
%       %\item \(\Re(C)\) et \(\Re(X)\) sont de même signe.
%     \end{enumerate}
%   \end{hyp}



% \subsection{Cas des matériaux avec pertes}

%   Les matériaux avec pertes sont tels que au moins l'une des deux constantes a une partie imaginaire non-nulle.

%   Si l'hypothèse \ref{hyp:unicite:form_var:hyp_coercivite} est vérifiée alors \(a\) est coercive ( \(\alpha = \min(-\Im(\mu)k^{-1},-\Im(\eps)k)\)) donc il y a unicité des solutions du problème de Maxwell d'après Lax-Milgram (voir \cite{cessenat_mathematical_1996}).

%   Comme nous sommes en convention \(e^{i\omega t}\) et que nous imposons que le signe de \(\Im(\eps)\) et \(\Im(\mu)\) soit négatif
  \begin{prop}[\glsentrydesc{acr-cgu}]~\\
    Si la condition suivante est vérifiée
    \begin{equation}\label{eq:unicite:form_var:cgu}
      \Re\left(\int_\Gamma \vJ \cdot \conj{\vE_t} ds\right) \ge 0
    \end{equation}
    alors l'unicité des solutions du problème de Maxwell extérieur est assurée.
  \end{prop}

  Pour démontrer cela, il faut faire appel au Lemme de Rellich, énoncé dans \cite[p.~74]{cessenat_mathematical_1996}:
  \begin{lemme}[Rellich]
    Soit \(\OO^c\) un domaine connexe, complément d'un domaine bornée, et soit \(u\) satisfaisant
    \begin{subequations}
      \begin{align}
        \Delta u + k^2 u = 0 & &\text{dans \(\OO^c\)}
        \\
        \int_{S_R} |u(\vx)|^2 ds(\vx) = 0 
      \end{align}
    \end{subequations}
    alors \(u=0\) dans \(\OO^c\).
  \end{lemme}

  En reprenant l'expression de la partie réelle de la forme bilinéaire \(a(\vE,\vE)\), on remarque que la condition \eqref{eq:unicite:form_var:cgu} implique
  \begin{align}
    \Re(a(\vE,\vE)) & = 0
    \\
    \Re(C) + \Re(X) & = 0
    \\ \intertext{d'après \cite[eq.~5.3.89]{nedelec_acoustic_2001} \(\Re(C)\ge 0\) donc } 
    \Re(C) = \Re(X) & = 0
    \\ \intertext{Pusique \(\Re(C) = 0\) alors \(\vE_t = 0\) sur \(S_R\) or d'après le lemme de Rellich}
    \vE &= 0  &\text{ dans } \OO^c
  \end{align}

%   Dans la suite de la thèse, nous noterons souvent cette quantité par \gls{mat-q}. C'est par définition la partie réelle du flux sortant du vecteur de Poynting à travers la surface de l'objet. 

% \subsection{Cas des matériaux sans pertes}

%   Si \(\Im(\mu) = \Im(\eps) = 0\), le résultat précédent n'est plus valable car \(\alpha = 0\). Cependant, l'unicité est assuré par l’alternative de Fredholm (voir \cite[Théorème 8]{cessenat_mathematical_1996}) dont la conclusion est aussi que \ref{eq:unicite:form_var:cgu} doit être vérifiée pour assurée l'unicité.

%   Maintenant que nous connaissant la condition suffisante à vérifier, nous allons montrer quelles conditions sur les coefficients des CIOE permettent la vérifier.