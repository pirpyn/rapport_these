\section{Résolution des équations intégrales par la méthode des éléments finis}
  \subsection{Fonctions de base}
    Soit \(\OO_h\) un maillage fermé de \(\OO\) en \(N_e\) éléments triangulaires. Soit \(N\) le nombre d'arête du maillage.

    % Les commandes qui suivent servent pour les 3 prochains schémas
    \newcommand{\ncouche}{8}
    \newcommand{\xa}{-3}
    \newcommand{\ya}{0.5}
    \newcommand{\xb}{-0.5}
    \newcommand{\yb}{-2}
    \newcommand{\xc}{0.5}
    \newcommand{\yc}{2.5}
    \newcommand{\xd}{5}
    \newcommand{\yd}{-0.5}

    \subsubsection[Fonctions phi Hdiv-conforme]{Fonctions \(\vect{\phi}\) \(\Hdiv\)-conforme}

      Ces fonctions ont été introduites par \cite[eq.~6]{rao_electromagnetic_1982}.
    \begin{minipage}{0.58\textwidth}
      \begin{center}
        \begin{tikzpicture}[scale=1]
          %%% Triangle de gauche

\tikzmath
{
    \ra = sqrt((\xb-\xc)^2 + (\yb-\yc)^2);
    \rb = sqrt((\xc-\xa)^2 + (\yc-\ya)^2);
    \rc = sqrt((\xa-\xb)^2 + (\ya-\yb)^2);
    \aire = 0.25*sqrt((\ra+\rb-\rc)*(\ra-\rb+\rc)*(-\ra+\rb+\rc)*(\ra+\rb+\rc));
}

\coordinate (na) at (\xa,\ya);
\coordinate (nb) at (\xb,\yb);
\coordinate (nc) at (\xc,\yc);

\coordinate (ec) at ($(nb)-(na)$);
\coordinate (ea) at ($(nc)-(nb)$);

\draw (na) -- (nc) -- (nb) -- cycle;

\foreach \couche in {1,...,\ncouche}
{
    \tikzmath
    {
        \a = \couche/\ncouche;
    }
    \foreach \n in {0,...,\couche}
    {
        \tikzmath
        {
            \b = \n/\couche;
        }
        \coordinate (phi) at ($\a*(ec)+\a*\b*(ea)$);
        \coordinate (x) at ($(na)+(phi)$);
        % \draw (x) node {\(x_\n^\couche\)};
        \draw [->] (x) -- ($(x)+1/\aire*(phi)$);
    }
}

%%% Triangle de droite

\tikzmath
{
    \rd = sqrt((\xb-\xc)^2 + (\yb-\yc)^2);
    \rb = sqrt((\xc-\xd)^2 + (\yc-\yd)^2);
    \rc = sqrt((\xd-\xb)^2 + (\yd-\yb)^2);
    \aire = 0.25*sqrt((\rd+\rb-\rc)*(\rd-\rb+\rc)*(-\rd+\rb+\rc)*(\rd+\rb+\rc));
}

\coordinate (nd) at (\xd,\yd);

\coordinate (ed) at ($(nc)-(nb)$);
\coordinate (ec) at ($(nb)-(nd)$);

\draw (nb) -- (nd) -- (nc) -- cycle;

\foreach \couche in {1,...,\ncouche}
{
    \tikzmath
    {
        \a = \couche/\ncouche;
    }
    \foreach \n in {0,...,\couche}
    {
        \tikzmath
        {
            \b = \n/\couche;
        }
        \coordinate (phi) at ($\a*(ec)+\a*\b*(ed)$);
        \coordinate (x) at ($(nd)+(phi)$);
        % \draw (x) node {\(x_\n^\couche\)};
        \draw [->] (x) -- ($(x)-1/\aire*(phi)$);
    }
}

        \end{tikzpicture}
      \end{center}
    \end{minipage} 
    \begin{minipage}{0.4\textwidth}
      En coordonnées locales au triangle \(T_j^l\) \(\vx = (\xi_1,\xi_2\),
      \begin{align*}
        phi_j(\vx) &= \frac{1}{2|K_j^l|}\left(\xi_1 \vect{e_1} + \xi_2 \vect{e_2}\right)
      \end{align*}

      En coordonnées globales  \((x_1,x_2,x_3)\),
      \begin{align*}
        \phi_j(\vx) &= \mathbf{1}_{K_j^l}(\vx)\frac{\vx-\vect{o_j^l}}{2|K_j^l|} \\&- \mathbf{1}_{K_j^r}(\vx)\frac{\vx-\vect{o_j^r}}{2|K_j^r|}
      \end{align*}
    \end{minipage}

    \subsubsection[Fonctions p Hcurl-conforme]{Fonctions \(\vect{p}\) \(\Hrot\)-conforme}

      \begin{center}
        \begin{tikzpicture}[scale=1]
          %%% Triangle de gauche

\tikzmath
{
    \ra = sqrt((\xb-\xc)^2 + (\yb-\yc)^2);
    \rb = sqrt((\xc-\xa)^2 + (\yc-\ya)^2);
    \rc = sqrt((\xa-\xb)^2 + (\ya-\yb)^2);
    \aire = 0.25*sqrt((\ra+\rb-\rc)*(\ra-\rb+\rc)*(-\ra+\rb+\rc)*(\ra+\rb+\rc));
}

\coordinate (na) at (\xa,\ya);
\coordinate (nb) at (\xb,\yb);
\coordinate (nc) at (\xc,\yc);

\coordinate (ec) at ($(nb)-(na)$);
\coordinate (ea) at ($(nc)-(nb)$);

\draw (na) -- (nc) -- (nb) -- cycle;

\foreach \couche in {1,...,\ncouche}
{
    \tikzmath
    {
        \a = \couche/\ncouche;
    }
    \foreach \n in {0,...,\couche}
    {
        \tikzmath
        {
            \b = \n/\couche;
        }
        \coordinate (phi) at ($\a*(ec)+\a*\b*(ea)$);
        \coordinate (x) at ($(na)+(phi)$);
        \path let \p1=(phi) in coordinate (p) at (\y1,-\x1);
        % \draw (x) node {\(x_\n^\couche\)};
        \draw [->] (x) -- ($(x)+1/\aire*(p)$);
    }
}

%%% Triangle de droite

\tikzmath
{
    \rd = sqrt((\xb-\xc)^2 + (\yb-\yc)^2);
    \rb = sqrt((\xc-\xd)^2 + (\yc-\yd)^2);
    \rc = sqrt((\xd-\xb)^2 + (\yd-\yb)^2);
    \aire = 0.25*sqrt((\rd+\rb-\rc)*(\rd-\rb+\rc)*(-\rd+\rb+\rc)*(\rd+\rb+\rc));
}

\coordinate (nd) at (\xd,\yd);

\coordinate (ed) at ($(nc)-(nb)$);
\coordinate (ec) at ($(nb)-(nd)$);

\draw (nb) -- (nd) -- (nc) -- cycle;

\foreach \couche in {1,...,\ncouche}
{
    \tikzmath
    {
        \a = \couche/\ncouche;
    }
    \foreach \n in {0,...,\couche}
    {
        \tikzmath
        {
            \b = \n/\couche;
        }
        \coordinate (phi) at ($\a*(ec)+\a*\b*(ed)$);
        \coordinate (x) at ($(nd)+(phi)$);
        \path let \p1=(phi) in coordinate (p) at (\y1,-\x1);
        % \draw (x) node {\(x_\n^\couche\)};
        \draw [->] (x) -- ($(x)-1/\aire*(p)$);
    }
}

        \end{tikzpicture}
      \end{center}


    \subsubsection[Fonctions q Hcurl-conforme]{Fonctions \(\vect{q}\) \(\Hrot\)-conforme}

      Ces fonctions ont été introduites par \cite[eq.~28]{bendali_boundary-element_1999}.

      \begin{center}
        \begin{tikzpicture}[scale=1]
          %%% Triangle de gauche
%%% Triangle de gauche

\coordinate (na) at (\xa,\ya);
\coordinate (nb) at (\xb,\yb);
\coordinate (nc) at (\xc,\yc);

\coordinate (ec) at ($(nb)-(na)$);
\coordinate (ea) at ($(nc)-(nb)$);

\draw (na) -- (nc) -- (nb) -- cycle;


\tikzmath
{
    \ra = sqrt((\xb-\xc)^2 + (\yb-\yc)^2);
    \rb = sqrt((\xc-\xa)^2 + (\yc-\ya)^2);
    \rc = sqrt((\xa-\xb)^2 + (\ya-\yb)^2);
    \aire = 0.25*sqrt((\ra+\rb-\rc)*(\ra-\rb+\rc)*(-\ra+\rb+\rc)*(\ra+\rb+\rc));
}

\coordinate (q) at (ea);
\coordinate (phi) at (0,0);
\coordinate (x) at ($(na)+(phi)$);
\draw [-Latex] (x) -- ($(x)+1/\aire*(q)$);


\foreach \couche in {2,...,\ncouche}
{
    \tikzmath
    {
        \a = (\couche-1)/(\ncouche-1);
        \c = 1 - 2*\a;
    }
    \coordinate (q) at ($\c*(ea)$);
    \foreach \n in {1,...,\couche}
    {
        \tikzmath
        {
            \b = (\n-1)/(\couche-1);
        }
        \coordinate (phi) at ($\a*(ec)+\a*\b*(ea)$);
        \coordinate (x) at ($(na)+(phi)$);
        \draw [-Latex] (x) -- ($(x)+1/\aire*(q)$);
    }
}

%%% Triangle de droite
%%% Triangle de droite
\coordinate (na) at (\xa,\ya);
\coordinate (nb) at (\xb,\yb);
\coordinate (nc) at (\xc,\yc);
\coordinate (nd) at (\xd,\yd);

\coordinate (ed) at ($(nc)-(nb)$);
\coordinate (ec) at ($(nb)-(nd)$);

\draw (nb) -- (nd) -- (nc) -- cycle;




\tikzmath
{
    \rd = sqrt((\xb-\xc)^2 + (\yb-\yc)^2);
    \rb = sqrt((\xc-\xd)^2 + (\yc-\yd)^2);
    \rc = sqrt((\xd-\xb)^2 + (\yd-\yb)^2);
    \aire = 0.25*sqrt((\rd+\rb-\rc)*(\rd-\rb+\rc)*(-\rd+\rb+\rc)*(\rd+\rb+\rc));
}

\coordinate (q) at (ed);
\coordinate (phi) at (0,0);
\coordinate (x) at ($(nd)+(phi)$);
\draw [-Latex] (x) -- ($(x)+1/\aire*(q)$);


\foreach \couche in {2,...,\ncouche}
{
    \tikzmath
    {
        \a = (\couche-1)/(\ncouche-1);
        \c = 1 - 2*\a;
    }
    \coordinate (q) at ($\c*(ed)$);
    \foreach \n in {1,...,\couche}
    {
        \tikzmath
        {
            \b = (\n-1)/(\couche-1);
        }
        \coordinate (phi) at ($\a*(ec)+\a*\b*(ed)$);
        \coordinate (x) at ($(nd)+(phi)$);
        \draw [-Latex] (x) -- ($(x)+1/\aire*(q)$);
    }
}

        \end{tikzpicture}
      \end{center}

      \begin{TODO}
        Les fonctions \(\vect{p}\) ne semblent pas conforme. A vérifier.
      \end{TODO}

    \subsection{Fonction de bases}

    Soit \(\phi_j\) la fonction de base associée à l’arête orientée \(v_j\) conjointe aux éléments  \(K_j^l\) (orientation directe) et \(K_j^r\) (orientation indirecte).\\

    Soit \(\vx = (x_1,x_2,x_3)\).

    %TODO
    \begin{TODO}
      mettre source des fonctions RT - RWG.
    \end{TODO}


    \begin{minipage}{0.58\textwidth}
      \centering
      \begin{tikzpicture}[scale=2.5]
        \coordinate (ol) at (-1,0); % noeud gauche
\coordinate (or) at (2,.5); % noeud droit
\coordinate (ou) at (0,1); % noeud haut
\coordinate (od) at (1,-1); %noeud bas
\coordinate (x1) at ($1/3*(od)+1/3*(ol)+1/3*(ou)$); % barycentre du 1er triangle
\coordinate (x2) at ($1/3*(od)+1/2*(or)+1/3*(ou)$); % du 2eme triangle

\draw[thick] (1,1) circle(0.05cm) node [right=0.06cm] {\(\vn\)}; % on affiche la normale
\fill (1,1) circle(0.01cm); %on rajoute un petit cercle au centre

% On affiche le nom des noeuds
\draw (ou) node [above] {\(\vect{o_j^u}\)};
\draw (or) node [right] {\(\vect{o_j^r}\)};
\draw (ol) node [left] {\(\vect{o_j^l}\)};
\draw (od) node [below] {\(\vect{o_j^d}\)};

% Des barycentres
\draw (x1) node [right] {\(x\)};
\draw (x2) node [left] {\(x'\)};

% On affiche le nom des triangles
\draw ($1/3*(od)+1/3*(ol)+1/3*(ou)$) node [above=2pt] {\(K_j^l\)};
\draw ($1/3*(od)+1/3*(or)+1/3*(ou)$) node [above=2pt] {\(K_j^r\)};

% l'arete commune
\draw (ou) -- (od);
% Les deux triangles
\draw (ol) -- (ou) -- (od) -- cycle;
\draw (or) -- (ou) -- (od) -- cycle;

% vecteur vj pour qj
\draw [->,black!10!purple,thick] (od) -- (ou) node [near end,left] {\(\vect{v_j}\)};

% fonctions phij
\coordinate (phi) at ($(x1)-(ol)$);
\draw [->,red] (x1) -- (phi) node [sloped,midway,above] {\(\vect{\phi_j}(\vx)\)};

% fonctions pj = -n x phij
\path let \p1=(phi) in coordinate (p) at (\y1,-\x1);
\draw [->,black!60!green] (x1) -- ($(x1)+(p)$) node [midway,left] {\(\vect{p_j}(\vx)\)};

% fonctions qj
\coordinate (vj) at ($(ou)!0.5!(od)$); %point milieu de ou od
\coordinate (vj) at ($(vj)-(ou)$); % vecteur de ou a vj
\draw [->,black!60!blue] (x1) -- ($(x1)+(vj)$) node [midway,right] {\(\vect{q_j}(\vx)\)};

% idem deuxieme triangle

\draw [->,red] (x2) -- (or) node [sloped,midway,below] {\(\vect{\phi_j}(x')\)};

\coordinate (phi) at ($(or)-(x2)$);
\path let \p1=(phi) in coordinate (p) at (\y1,-\x1);
\coordinate (xp) at ($(x2)+(p)$);
\draw [->,black!60!green] (x2) -- (xp) node [midway,left] {\(\vect{p_j}(x')\)};

\coordinate (vj) at ($(ou)!0.2!(od)$); % poit à 0.2 de ou vers od
\coordinate (vj) at ($(vj)-(ou)$);
\draw [->,black!60!blue] (x2) -- ($(x2)-(vj)$) node [midway,right] {\(\vect{q_j}(x')\)};
      \end{tikzpicture}
    \end{minipage} 
    \begin{minipage}{0.4\textwidth}
      \begin{align*}
        \color{red}{\phi_j(\vx)} &= \mathbf{1}_{K_j^l}(\vx)\frac{\vx-\vect{o_j^l}}{2|K_j^l|} \\&- \mathbf{1}_{K_j^r}(\vx)\frac{\vx-\vect{o_j^r}}{2|K_j^r|}\\
        \color{black!60!green}{p_j(\vx)} &= -\vn \pvect \phi_j(\vx)\\
        \color{black!60!blue}{q_j(\vx)} &= \mathbf{1}_{K_j^l}(\vx) ( { \color{purple}{\vect{v}_j } } \pvect \vn ) \cdot ( \vx - \frac{\vect{o^l_j} + \vect{o^d_j}}{2} ) \\
        &- \mathbf{1}_{K_j^r}(\vx) ( { \color{purple}{\vect{v}_j } } \pvect \vn ) \cdot ( \vx - \frac{\vect{o^r_j} + \vect{o^d_j}}{2} )
      \end{align*}
    \end{minipage}

    En notations locale au triangle \(K_j^l, \xi_i = \frac{\sqrt{(\vx-\vect{o^l_j}).\vect{e_i}}}{||\vect{e_i}||}\).

    \begin{minipage}{0.53\textwidth}
      \centering
      \begin{tikzpicture}[scale=2.5]
        \coordinate (ol) at (0,0);
\coordinate (od) at (2,0);
\coordinate (ou) at (0,2);


\draw (ou) node [above] {$o_j^u$};
\draw (ol) node [left] {$o_j^l$};
\draw (od) node [below] {$o_j^d$};

%\draw ($1/3*(od) + 1/3*(ou)$) node [above=2pt] {$K_j^l$};
\draw (ol) -- (ou) -- (od) -- cycle;

\draw [->,black!10!orange,thick] (ol) -- (od) node [midway,below] {\color{black!30!orange}{$\v e_1$}};
\draw [->,black!10!orange,thick] (ol) -- (ou) node [midway,left] {\color{black!30!orange}{$\v e_2$}};

\draw[thick] (ou) +(0.3,0) circle(0.05cm) node [right=0.06cm] {$\v n$};
\fill (ou) + (0.3,0) circle(0.01cm);

\draw [dashdotted] ($0.5*(ou)+0.5*(ol)$) -- ($0.5*(od)+0.5*(ol)$);

% 1
\coordinate (x) at ($1/2*(ou)+1/4*(od)$);
\draw (x) node [right] {$x$};

\draw [->,red] (ol) -- (x) node [sloped,midway,above] {$\phi_j(x)$};

\coordinate (vj) at ($(ou)!0.3!(od)$);
\coordinate (vj) at ($(vj)-(ou)$);

\coordinate (phi) at ($(x)-(ol)$);
\path let \p1=(phi) in coordinate (p) at (\y1,-\x1);
\coordinate (xp) at ($(x)+(p)$);
\draw [->,black!60!green] (x) -- (xp) node [midway,right] {$p_j(x)$};

\coordinate (xp) at ($(x)+(vj)$);
\draw [->,black!60!blue] (x) -- (xp) node [midway,left] {$q_j(x)$};

%\draw [dotted] (x) -- ($(ol)!(x)!(od)$) node [at end, below] {$\xi_1$};
%\draw [dotted] (x) -- ($(ol)!(x)!(ou)$) node [at end, left] {$\xi_2$};

      \end{tikzpicture}
    \end{minipage}
    \begin{minipage}{0.45\textwidth}
      \begin{align*}
        \color{red}{\phi_j(\vx)} &= \frac{1}{2|K_j^l|}\left(\xi_1 \vect{e_1} + \xi_2 \vect{e_2}\right)\\
        \color{black!60!green}{p_j(\vx)} &= \frac{1}{2|K_j^l|}\left(\xi_2 \vect{e_1} -\xi_1 \vect{e_2}\right)\\
        \color{black!60!blue}{q_j(\vx)} &= \frac{1}{2|K_j^l|}(1-2(\xi_1+\xi_2))(\vect{e_1} -\vect{e_2})
      \end{align*}
      %Les fonctions \(q_j\) s'annulent notamment sur le segment \(\xi_1+\xi_2=\frac{1}{2}\) représenté en pointillés. 
    \end{minipage}

    \begin{prop}
      Au sens des distributions, \(\forall j,\, \vrot \phi_j = \delta_x\)
    \end{prop}
    \begin{proof}
    \end{proof}

    \begin{TODO}
      Montrer que les RT sont dans Hdiv et non Hrot
    \end{TODO}

  \subsection{Matrices de masses}
    Soit \(U\) une fonction définit sur \(\Gamma\), approchée par ces fonctions: 
    \[ U(\vx) \simeq \sum\limits_{j=1}^N u_j \phi_j(\vx) \simeq \sum\limits_{j=1}^N v_j p_j(\vx) \simeq \sum\limits_{j=1}^N w_j q_j(\vx)
    \]

    \begin{TODO}
      A priori les espaces fonctionnels ne sont pas les mêmes. Il faut corriger la relation ci-dessous.
    \end{TODO}

    \begin{defn}
      On définit la matrice \(\mG{u}{v}\) telle que
      \begin{equation}
        \mG{u}{v}_{ij} = \int_\Gamma u_i(\vx)\cdot v_j(\vx) \dd{s}(\vx)
      \end{equation}
    \end{defn}

    \begin{prop}
      Par définition
      \begin{equation}
        \left(\mG{u}{v}\right)^t = \mG{v}{u}
      \end{equation}
    \end{prop}

    On remarque alors que les matrices \(\mG{\phi}{\phi}\) et \(\mG{p}{p}\) sont identiques tandis que la matrice \(\mG{p}{\phi}\) est anti-symétrique.

  \subsection{Discrétisation de l'équation intégrale}
    Soit \(g\) la fonction de Green. Soit \(\vE^{inc}_t,\vH^{inc}_t\) le champ incident tangentiel à \(\Gamma\).

    \subsubsection{EFIE}
      On définit l'\gls{acr-efie} telle que
      \begin{multline}
        \label{eq:form_int:EFIE}
        \vE^{inc}_t(\vx) =
          \frac{\vE_t(\vx)}{2} 
            - \int_\Gamma \vgrad g(\vx,\vy) \pvect \vK(\vy) \dd{s}(\vy) \cdots \\
          + \frac{i}{k}\vgrad \int_\Gamma  g(\vx,\vy)\vdiv\vJ(\vy)(\vx) \dd{s}(\vy)
            +  ik\int_\Gamma g(\vx,\vy)\vJ(\vy) \dd{s}(\vy)
      \end{multline}
      \begin{TODO}
        Mettre source de l'EFIE (Nédélec)
      \end{TODO}

      On décompose \(\vJ,\vK,\vE_t,\vH_t\) sur les fonctions de base \(\phi_j\):
      \begin{subequations}
        \begin{align}
          \vJ &\simeq \sum_{j=1}^N I_j \phi_j & \vE_t &\simeq \sum_{j=1}^N e_j \phi_j\\
          \vK &\simeq \sum_{j=1}^N k_j \phi_j & \vH_t &\simeq \sum_{j=1}^N h_j \phi_j
        \end{align}
      \end{subequations}

      On utilise la méthode de Galerkin, on prend comme fonction test \(\phi_i\): 
      \begin{multline}
      \int_\Gamma \vE^{inc}_t(\vx) \cdot \phi_i(\vx) \dd{s}(\vx) = \sum_{j=1}^{N} \left( 
        \frac{e_j}{2} \int_\Gamma \phi_j(\vx) \cdot \phi_i(\vx) \dd{s}(\vx)\right. \\
        -  k_j \int_\Gamma\left(\int_\Gamma \vgrad g(\vx,\vy) \pvect \phi_j(\vy) \dd{s}(\vy)\right) \cdot \phi_i(\vx) \dd{s}(\vx) \\
        + I_j\frac{i}{k}\int_\Gamma
          \left(\int_\Gamma g(\vx,\vy)\vdiv\phi_j(\vy) \dd{s}(\vy)\right) \vdiv \phi_i(\vx) \dd{s}(\vx) \\
      \left.
        + I_j ik \int_\Gamma \left(\int_\Gamma g(\vx,\vy)\phi_j(\vy) \dd{s}(\vy)\right)\cdot \phi_i(\vx) \dd{s}(\vx)
      \right)
      \end{multline}

      On définit alors les matrices \(\mS\), \(\mSt\) telles que
      \begin{align}
        S_{pq} 
          &= 4\pi ik \int_\Gamma \left( \int_\Gamma g(\vx,\vy) \phi_p(\vy) \dd{s}(\vy) \right) \cdot \phi_q (\vx)\dd{s}(\vx) \cdots\\
          &~+ 4\pi \frac{i}{k}\int_\Gamma \left( \int_\Gamma g(\vx,\vy) \vdiv \phi_p (\vy) \dd{s}(\vy) \right) \vdiv \phi_q(\vx)\dd{s}(\vx)\\ 
        \tilde{S}_{pq}
          &= 4\pi \int_\Gamma \left( \int_\Gamma \vgrad g(\vx,\vy) \pvect \phi_p(\vy) \dd{s}(\vy) \right) \cdot \phi_q(\vx)\dd{s}(\vx)
      \end{align}

      \begin{prop}
        Les matrices \(\mS\) et \(\mSt\) sont symétriques.
      \end{prop}
      \begin{proof}
        Pour \(\mS\), c'est une conséquence immédiate des propriétés de l'intégrale et de la symétrie de la fonction de Green.
        Pour \(\mSt\), 
      \end{proof}

      Le second membre est obtenu en projetant le champ incident 
      \begin{equation}
        b_i^E = \int_\Gamma \vE^{inc}_t(\vx) \cdot \phi_i(\vx)\dd{s}(\vx)
      \end{equation}

      On obtient alors 
      \begin{equation}
        \frac{\mS}{4\pi}\vect{I}
        + \frac{\mG{\phi}{\phi}}{2}\vect{e}
        + \frac{\mSt^t}{4\pi}\vect{k}
        = \vect{b^E}
      \end{equation}
    
    \subsubsection{MFIE}
      On définit la \gls{acr-mfie} telle que
      \begin{multline}
        \label{eq:form_int:EFIE}
        \vH_i(\vx) = 
        \frac{\vH_t(\vx)}{2} 
          - \int_\Gamma \vgrad g(\vx,\vy) \pvect \vJ(\vy) \dd{s}(\vy) \cdots \\
        - \frac{i}{k} \vgrad \int_\Gamma  g(\vx,\vy)\vdiv\vK(\vy)\dd{s}(\vy) 
          - ik \int_\Gamma g(\vx,\vy)\vK(\vy)\dd{s}(\vy)
      \end{multline}
      
      On utilise la méthode de Galerkin, on prend comme fonction test \(\phi_i\): 
      \begin{multline}
        \int_\Gamma \vH^{inc}_t(\vx) \cdot \phi_i(\vx) \dd{s}(\vx) = \sum_{j=1}^{N} \left(
          \frac{h_j}{2}\int_\Gamma \phi_j(\vx) \cdot \phi_i(\vx) \dd{s}(\vx) \right. \\
          -  I_j\int_\Gamma  \left(\int_\Gamma \vgrad g(\vx,\vy) \pvect \phi_j(\vy) \dd{s}(\vy) \right) \cdot \phi_i(\vx) \dd{s}(\vx) \\
          - k_j\frac{i}{k} \int_\Gamma \left(\int_\Gamma g(\vx,\vy)\vdiv \phi_j(\vy) \dd{s}(\vy) \right) \vdiv \phi_i(\vx) \dd{s}(\vx) \\
        \left.
          + ik\int_\Gamma \left(\int_\Gamma g(\vx,\vy)\phi_j(\vy)\dd{s}{(\vy)}\right)\cdot \phi_i(\vx)\dd{s}(\vx)
        \right)
      \end{multline}

      Le second membre est obtenu en projetant le champ incident 
      \begin{equation}
        b_i^H = \int_\Gamma \vH^{inc}_t(\vx) \cdot \phi_i(\vx)\dd{s}(\vx)
      \end{equation}

      On obtient alors 
      \begin{equation}
        \frac{\mSt}{4\pi}\vect{I} 
        + \frac{\mG{\phi}{\phi}}{2}\vect{h}
        - \frac{\mS}{4\pi}\vect{k} 
        = \vect{b^H}
      \end{equation}

  \subsection{Discrétisation de l'équation intégrale avec condition d'impédance}
    Pour diminuer le nombre d'inconnus, on utilise une \gls{acr-cioe} pour exprimer \(\vect{e}, \vect{h}\) selon \(\vect{J}, \vect{K}\). Selon la CIOE, nous pourrons aboutir à différents systèmes linéaire selon la projection employée. 

    Les CIOE font intervenir les opérateurs différentiels \gls{ope-LD}, \gls{ope-LR}, \gls{ope-L}. On introduit donc les matrices de rigidités \(\mLD,\mLR,\mL\).

    Comme les \(\phi_j\) sont dans \(\Hdiv\), de l'opérateur \gls{ope-LD}, on définit la matrice \(\mLD\) :
    \begin{align}
      \int_\Gamma L_D(U)(\vx) \cdot \phi_i(\vx) \dd{s}(\vx)  
        & \simeq - \sum_{j=1}^N u_j \int_\Gamma \vdiv \phi_j(\vx) \vdiv \phi_i(\vx) \dd{s}(\vx) = \mLD \vu
    \end{align}

    L'opérateur \(L_R\) est définit sur \(\Hrot\) et nécessite donc un changement de base (voir \cite[(13)]{stupfel_implementation_2015}):

    \begin{defn}
      On suppose que la matrice \(\mG{w}{u}\) est inversible alors on définit la projection sur la base \(w\) de la base \(u\) vers \(v\)
      \begin{equation}
        \mP{u}{v}{w} = \left(\mG{w}{u}\right)^{-1}\mG{w}{v}
      \end{equation}
    \end{defn}

    \begin{prop}
      Par définition
      \begin{equation}
        \left(\mP{u}{v}{w}\right)^t = \mG{v}{w}\left(\mG{u}{w}\right)^{-1}
      \end{equation}
    \end{prop}

    \begin{prop}
      On suppose de plus que la matrice \(\mG{w}{v}\) est inversible alors
      \begin{equation}
        \left(\mP{u}{v}{w}\right)^{-1} = \mP{v}{u}{w}
      \end{equation}
    \end{prop}

    Alors on définit \(\mLR\) telle que (voir \cite{stupfel_implementation_2015})
    \begin{align}
      \mLR & = \left(\mP{p}{\phi}{p}\right)^t\mLD\mP{p}{\phi}{p} \\
      \intertext{ou alors}
      \mLR & = \left(\mP{p}{\phi}{q}\right)^t\mLD\mP{p}{\phi}{q} 
    \end{align}

    \begin{TODO}
      Le choix du LR doit dépendre de certains propriétés des fdb ou des matrices. Les expliciter.
    \end{TODO}

    La matrice \(\mL\) s’obtient immédiatement de la définition de l'opérateur \gls{ope-L}: 
    \begin{equation}
      \mL = \mLD - \mLR
    \end{equation}
        
    \subsection{CI0}
      Par définition de la CIOE,
      \begin{equation}
          \vE_t = a_0 \vJ
      \end{equation}

      En projetant sur la base des \(\phi_j\)
      \begin{align}
        \vect{e} &= a_0 \vect{I} \\
        \vect{h} &= -\frac{1}{a_0} \vect{k}
      \end{align}

    On cherche alors \(\vect{I},\vect{k}\) solution de 

    \begin{equation}
      4\pi
      \begin{bmatrix}
        \mat{S^0}+2\pi a_0 \mM & \mat{S^G} \\
        \mat{S^G} & -\mat{S^0} - \frac{2\pi}{a_0}\mM
      \end{bmatrix}
      \begin{bmatrix}
        \vect{I}\\
        \vect{k}
      \end{bmatrix}
      =
      \begin{bmatrix}
        \vect{b^E}\\
        \vect{b^H}
      \end{bmatrix}     
    \end{equation}

    \subsection{CI01}
      Par définition de la CIOE,
      \begin{equation}
          \vE_t = \left(a_0I_d + a_1 L\right) \vJ
      \end{equation}

      L'opérateur différentiel \gls{ope-L} nécessite que les fonctions de base appartiennent à \(\Hdiv\cap\Hrot\). Une projection des $\phi_j$ sur les $p_j$ ou $q_j$ est alors nécessaire.

      En projetant sur la base des \(\phi_j\)
      \begin{align}
        \vect{e} &= a_0 \vect{I} \\
        \vect{h} &= \frac{1}{a_0} \vect{k}
      \end{align}

    On cherche alors \(\vect{I},\vect{k}\) solution de 

    \begin{equation}
      4\pi
      \begin{bmatrix}
        \mat{S^0}+2\pi a_0 \mM & \mat{S^G} \\
        \mat{S^G} & -\mat{S^0} +  \frac{2\pi}{a_0}\mM
      \end{bmatrix}
      \begin{bmatrix}
        \vect{I}\\
        \vect{k}
      \end{bmatrix}
      =
      \begin{bmatrix}
        \vect{b^E}\\
        \vect{b^H}
      \end{bmatrix}     
    \end{equation}

    \subsection{CI1}
      On rappelle,
      \begin{equation}
        \vE_t = (a_0I_d+a_1L)\vJ -bL\vE_t
      \end{equation}
      \subsubsection{E en fonction de J}
    D’après la CIOE \(\vE_t = Z\vJ\) : \(M\vect e = Z M\vect I\).
    \[
    \left[\frac{a_0}{2}M + \frac{L}{2}(a_1I_d - bM^{-1}ZM) + \frac{S^0}{4\pi}\right]\vect I + \frac{\tilde S^G }{4\pi}\vect k = \vect b_i^E
    \]
    \subsubsection{E en fonction de K}
    \(\vn \pvect \vE_t = \vK\).
    \(\vK = \sum k_j \phi_j = \sum e_j \vn \pvect \phi_j = -\sum e_j p_j \).
    \(M \vect k = - \tilde M \vect e\)
    \[
    \left[\frac{a_0}{2}G + \frac{a_1}{2}L+\frac{S^0}{4\pi}\right]\vect I + \left[\frac{\tilde S^G }{4\pi}+  \frac{b}{2}L\tilde M^{-1}M \right]\vect k = \vect b_i^E
    \]
    On rappelle que d’après la CIOE, \(\vn \pvect \vE_t = \vn \pvect (a_0I_d+a_1L)\vJ -b\vn\pvect L\vE_t\). Or \(\vn\pvect \vJ = -\vH_t\). Donc la CIOE peut s'exprimer
    \[
    \vH_t = \frac{1}{a_0}\left(-\vK + a_1 \vn\pvect L\vJ -b\vn\pvect L\vE_t\right)
    \]
        On remarque \( \vn \pvect L(p_j)\cdot \phi_i = L(p_j)\cdot(\phi_i\pvect\vn) = L(p_j)\cdot p_i = X_{ij}\). 
    \[
    \left[\frac{1}{2a_0}M + \frac{S^0}{4\pi}\right]\vect I + \frac{a_1}{2a_0}X\vect I' - \frac{b}{2a_0}X\vect e' - \left[ \frac{\tilde S^G}{4\pi}\right]\vect k = \vect b_i^H
    \]
    D’après Bruno TODO \(X = -9\tilde N^tL^D\tilde N - L^D\) %TODO

    \(\vect I' = \tilde M^{-1}M\vect I\)

    \(\vect e' = \vect k\)

    On obtient alors 
    \[
    \left[\frac{1}{2a_0}M + \frac{a_1}{2a_0}X\tilde M^{-1}M + \frac{S^0}{4\pi}\right]\vect I - \left[\frac{b}{2a_0}X+ \frac{\tilde S^G}{4\pi}\right]\vect k = \vect b_i^H
    \]

    \subsection{CI4}
    \subsection{CI3}