\section{Résolution des formulation intégrales des équations de Maxwell par éléments finis}
On écrit la CIOE comme
\[
  (I_d + bL)E_t = (a_0I_d+a_1L)J
\]
où L est l'opérateur de Hodge.
\subsection{Fonction de base}
On a maillage 3D surfacique de $N_e$ éléments triangulaires.

Soit $\phi_j$ la fonction de base associée à l’arête orientée $v_j$ conjointe aux éléments $K_p$ (orientation directe) et $K_q$ (orientation indirecte), et l'on notera $K_j^l$ pour $K_p$ et $K_j^r$ pour $K_q$.\\

En notations globale au repère 3D. 
%En prenant en compte l'orientation des arêtes entre 2 éléments. 
Soit $\v x = (x_1,x_2,x_3)$.

%TODO
\TODO{mettre source des fonctions RT - RWG.}

\begin{minipage}{0.4\textwidth}
\begin{align*}
	\color{red}{\phi_j(x)} &= \mathbf{1}_{K_j^l}(x)\frac{x-o_j^l}{2|K_j^l|} \\&- \mathbf{1}_{K_j^r}(x)\frac{x-o_j^r}{2|K_j^r|}\\
	\color{black!60!green}{p_j(x)} &= -\n \pvect \phi_j(x)\\
	\color{black!60!blue}{q_j(x)} &= \mathbf{1}_{K_j^l}(x) ( { \color{purple}{\v v_j } } \pvect \n ) \cdot ( x - \frac{o^l_j + o^d_j}{2} ) \\
 &- \mathbf{1}_{K_j^r}(x) ( { \color{purple}{\v v_j } } \pvect \n ) \cdot ( x - \frac{o^r_j + o^d_j}{2} )
\end{align*}
\end{minipage}
\begin{minipage}{0.58\textwidth}
	\centering
	\begin{tikzpicture}[scale=2.5]
	\coordinate (ol) at (-1,0); % noeud gauche
\coordinate (or) at (2,.5); % noeud droit
\coordinate (ou) at (0,1); % noeud haut
\coordinate (od) at (1,-1); %noeud bas
\coordinate (x1) at ($1/3*(od)+1/3*(ol)+1/3*(ou)$); % barycentre du 1er triangle
\coordinate (x2) at ($1/3*(od)+1/2*(or)+1/3*(ou)$); % du 2eme triangle

\draw[thick] (1,1) circle(0.05cm) node [right=0.06cm] {\(\vn\)}; % on affiche la normale
\fill (1,1) circle(0.01cm); %on rajoute un petit cercle au centre

% On affiche le nom des noeuds
\draw (ou) node [above] {\(\vect{o_j^u}\)};
\draw (or) node [right] {\(\vect{o_j^r}\)};
\draw (ol) node [left] {\(\vect{o_j^l}\)};
\draw (od) node [below] {\(\vect{o_j^d}\)};

% Des barycentres
\draw (x1) node [right] {\(x\)};
\draw (x2) node [left] {\(x'\)};

% On affiche le nom des triangles
\draw ($1/3*(od)+1/3*(ol)+1/3*(ou)$) node [above=2pt] {\(K_j^l\)};
\draw ($1/3*(od)+1/3*(or)+1/3*(ou)$) node [above=2pt] {\(K_j^r\)};

% l'arete commune
\draw (ou) -- (od);
% Les deux triangles
\draw (ol) -- (ou) -- (od) -- cycle;
\draw (or) -- (ou) -- (od) -- cycle;

% vecteur vj pour qj
\draw [->,black!10!purple,thick] (od) -- (ou) node [near end,left] {\(\vect{v_j}\)};

% fonctions phij
\coordinate (phi) at ($(x1)-(ol)$);
\draw [->,red] (x1) -- (phi) node [sloped,midway,above] {\(\vect{\phi_j}(\vx)\)};

% fonctions pj = -n x phij
\path let \p1=(phi) in coordinate (p) at (\y1,-\x1);
\draw [->,black!60!green] (x1) -- ($(x1)+(p)$) node [midway,left] {\(\vect{p_j}(\vx)\)};

% fonctions qj
\coordinate (vj) at ($(ou)!0.5!(od)$); %point milieu de ou od
\coordinate (vj) at ($(vj)-(ou)$); % vecteur de ou a vj
\draw [->,black!60!blue] (x1) -- ($(x1)+(vj)$) node [midway,right] {\(\vect{q_j}(\vx)\)};

% idem deuxieme triangle

\draw [->,red] (x2) -- (or) node [sloped,midway,below] {\(\vect{\phi_j}(x')\)};

\coordinate (phi) at ($(or)-(x2)$);
\path let \p1=(phi) in coordinate (p) at (\y1,-\x1);
\coordinate (xp) at ($(x2)+(p)$);
\draw [->,black!60!green] (x2) -- (xp) node [midway,left] {\(\vect{p_j}(x')\)};

\coordinate (vj) at ($(ou)!0.2!(od)$); % poit à 0.2 de ou vers od
\coordinate (vj) at ($(vj)-(ou)$);
\draw [->,black!60!blue] (x2) -- ($(x2)-(vj)$) node [midway,right] {\(\vect{q_j}(x')\)};
	\end{tikzpicture}
\end{minipage} 

En notations locale au triangle $K_j^l, \xi_i = \frac{\sqrt{(x-o^l_j).\v e_i}}{||e_i||}$.

\begin{minipage}{0.45\textwidth}
\begin{align*}
	\color{red}{\phi_j(x)} &= \frac{1}{2|K_j^l|}\left(\xi_1 \v e_1 + \xi_2 \v e_2\right)\\
	\color{black!60!green}{p_j(x)} &= \frac{1}{2|K_j^l|}\left(\xi_2 \v e_1 -\xi_1 \v e_2\right)\\
	\color{black!60!blue}{q_j(x)} &= \frac{1}{2|K_j^l|}(1-2(\xi_1+\xi_2))(\v e_1 -\v e_2)
\end{align*}
Les fonctions $q_j$ s'annulent notamment sur le segment $\xi_1+\xi_2=\frac{1}{2}$ représenté en pointillés. 
\end{minipage}
\begin{minipage}{0.53\textwidth}
	\centering
	\begin{tikzpicture}[scale=2.5]
	\coordinate (ol) at (0,0);
\coordinate (od) at (2,0);
\coordinate (ou) at (0,2);


\draw (ou) node [above] {$o_j^u$};
\draw (ol) node [left] {$o_j^l$};
\draw (od) node [below] {$o_j^d$};

%\draw ($1/3*(od) + 1/3*(ou)$) node [above=2pt] {$K_j^l$};
\draw (ol) -- (ou) -- (od) -- cycle;

\draw [->,black!10!orange,thick] (ol) -- (od) node [midway,below] {\color{black!30!orange}{$\v e_1$}};
\draw [->,black!10!orange,thick] (ol) -- (ou) node [midway,left] {\color{black!30!orange}{$\v e_2$}};

\draw[thick] (ou) +(0.3,0) circle(0.05cm) node [right=0.06cm] {$\v n$};
\fill (ou) + (0.3,0) circle(0.01cm);

\draw [dashdotted] ($0.5*(ou)+0.5*(ol)$) -- ($0.5*(od)+0.5*(ol)$);

% 1
\coordinate (x) at ($1/2*(ou)+1/4*(od)$);
\draw (x) node [right] {$x$};

\draw [->,red] (ol) -- (x) node [sloped,midway,above] {$\phi_j(x)$};

\coordinate (vj) at ($(ou)!0.3!(od)$);
\coordinate (vj) at ($(vj)-(ou)$);

\coordinate (phi) at ($(x)-(ol)$);
\path let \p1=(phi) in coordinate (p) at (\y1,-\x1);
\coordinate (xp) at ($(x)+(p)$);
\draw [->,black!60!green] (x) -- (xp) node [midway,right] {$p_j(x)$};

\coordinate (xp) at ($(x)+(vj)$);
\draw [->,black!60!blue] (x) -- (xp) node [midway,left] {$q_j(x)$};

%\draw [dotted] (x) -- ($(ol)!(x)!(od)$) node [at end, below] {$\xi_1$};
%\draw [dotted] (x) -- ($(ol)!(x)!(ou)$) node [at end, left] {$\xi_2$};

	\end{tikzpicture}
\end{minipage}

\begin{prop}
Au sens des distributions, $\forall j,\, \rot \phi_j =
\delta_x$
\end{prop}
\begin{proof}
\end{proof}

\TODO{Montrer que les RT sont dans Hdiv et non Hrot}

\subsection{Relations de passages et matrices}
Soit $ U \simeq \sum\limits_{j=1}^N u_j \phi_j \simeq \sum\limits_{j=1}^N v_j p_j \simeq \sum\limits_{j=1}^N w_j q_j$

On définit les matrices $M,\tilde M, N,\tilde N$ telles que:
\begin{align*}
  \int U \cdot \phi_i &\simeq \sum u_j \int \phi_j \cdot \phi_i = M\v u \\
  & \simeq \sum v_j \int p_j \cdot \phi_i = -N\v v\\
  & \simeq \sum w_j \int q_j \cdot \phi_i = \tilde{N}\v w \\
  \int U \cdot p_i &\simeq \sum v_j \int p_j \cdot p_i = M\v v \\
  &\simeq \sum w_j \int q_j \cdot p_i = \tilde M\v w
\end{align*}
%TODO 
%On a donc les relations de passage des bases $ H_{\text{div}}$ vers $H_{\text{rot}}$
%\begin{align*}
%  \v v &= G'^{-1} G\v u \\
%  \v v &= -G^{-1}G'\v u \\
%\end{align*}
%%% SECTION EFIE
\subsection{EFIE}
Soit g la fonction de Green
\[
  \frac{E_t}{2}+\grad{i}{k}\grad \int  g\div\J +  ik\int g\J - \int \grad g \pvect \K = \E^{inc}
  \]
On décompose $\J,\K,\E^t$ sur les fonctions de bases $\phi_j$ : $\J = \sum I_j \phi_j,\K = \sum k_j \phi_j,\E^t = \sum e_j \phi_j$. 

On rappelle que d’après la CIOE,
\[
  \E_t = (a_0I_d+a_1L)\J -bL\E_t
\]

On va injecter la CIOE dans l'EFIE, faire un produit scalaire avec une fonction de base $\phi_i$ puis intégrer : 
\begin{multline}
  \sum_j \left [\frac{I_j(a_0\int \phi_j \cdot \phi_i +a_1\int L(\phi_j) \cdot \phi_i) -be_j \int L(\phi_j)\cdot \phi_i}{2} \right. \\
  +I_j\frac{i}{k}\int (\int g\div\phi_j)\div \phi_i +  I_j ik\int (\int g\phi_j)\cdot \phi_i\\
  \left. -  k_j\int(\int \grad g \pvect \phi_j) \cdot \phi_i \right]= \int \E^{inc} \cdot \phi_i
\end{multline}
Soit $L_{ij} = \int L(\phi_i)\cdot\phi_j$\\
Soit $S^0_{ij}=4\pi i\left(k\int(\int g\phi_i)\cdot\phi_j + \frac{1}{k}\int(\int g\div\phi_i)\div\phi_j\right)$\\ 
Soit $\tilde S^G_{ij} = 4\pi\left(\int(\int \grad g \pvect \phi_i) \cdot \phi_j \right)$\\
Soit $b_i^E = \int \E^{inc} \cdot \phi_i$.
On obtient alors 
\[
  \left[\frac{a_0}{2}M + \frac{a_1}{2}L + \frac{S^0}{4\pi}\right]\v I - \frac{b}{2}L\v e+ \frac{\tilde  S^G \v k}{4\pi} = \v b_i^E
\]
\subsubsection{E en fonction de J}
D’après la CIOE $\E_t = Z\J$ : $M\v e = Z M\v I$.
\[
  \left[\frac{a_0}{2}M + \frac{L}{2}(a_1I_d - bM^{-1}ZM) + \frac{S^0}{4\pi}\right]\v I + \frac{\tilde S^G }{4\pi}\v k = \v b_i^E
\]
\subsubsection{E en fonction de K}
$\n \pvect \E_t = \K$.
$\K = \sum k_j \phi_j = \sum e_j \n \pvect \phi_j = -\sum e_j p_j $.
$M \v k = - \tilde M \v e$
\[
  \left[\frac{a_0}{2}G + \frac{a_1}{2}L+\frac{S^0}{4\pi}\right]\v I + \left[\frac{\tilde S^G }{4\pi}+  \frac{b}{2}L\tilde M^{-1}M \right]\v k = \v b_i^E
\]
%%% SECTION MFIE
\subsection{MFIE}
Soit g la fonction de Green
\[
  \frac{H_t}{2}-\frac{i}{k}\grad \int  g\div\K -  ik\int g\K - \int \grad g \pvect \J = \H^{inc}
\]

On rappelle que d’après la CIOE, $\n \pvect \E_t = \n \pvect (a_0I_d+a_1L)\J -b\n\pvect L\E_t$. Or $\n\pvect \J = -\H_t$. Donc la CIOE peut s'exprimer
\[
  \H_t = \frac{1}{a_0}\left(-K + a_1 \n\pvect L\J -b\n\pvect L\E_t\right)
\]

On décompose les fonctions sur les $\phi_j$ sauf les termes sur lesquels s'applique $L$, où l'on décompose sur $p_j$ 

On va injecter la CIOE dans la MFIE, faire un produit scalaire avec une fonction de base $\phi_i$ puis intégrer : 
\begin{multline}
  \sum_j \left [\frac{-k_j\int\phi_j\cdot\phi_i + a_1 I'_j \int \n \pvect L (p_j) \cdot \phi_i  -be'_j\int \n \pvect L(p_j)\cdot \phi_i}{2a_0} \right. \\
  +\left(k_j\frac{i}{k}\int (\int g\div\phi_j)\div \phi_i + ik\int (\int g\phi_j)\cdot \phi_i \right)\\
  \left. -  I_j\int(\int \grad g \pvect \phi_j) \cdot \phi_i \right]= \int \H^{inc} \cdot \phi_i
\end{multline}
On remarque $ \n \pvect L(p_j)\cdot \phi_i = L(p_j)\cdot(\phi_i\pvect\n) = L(p_j)\cdot p_i = X_{ij}$. 
\[
  \left[\frac{1}{2a_0}M + \frac{S^0}{4\pi}\right]\v I + \frac{a_1}{2a_0}X\v I' - \frac{b}{2a_0}X\v e' - \left[ \frac{\tilde S^G}{4\pi}\right]\v k = \v b_i^H
\]
D’après Bruno TODO $X = -9\tilde N^tL^D\tilde N - L^D$ %TODO

$\v I' = \tilde M^{-1}M\v I$

$\v e' = \v k$

On obtient alors 
\[
  \left[\frac{1}{2a_0}M + \frac{a_1}{2a_0}X\tilde M^{-1}M + \frac{S^0}{4\pi}\right]\v I - \left[\frac{b}{2a_0}X+ \frac{\tilde S^G}{4\pi}\right]\v k = \v b_i^H
\]

