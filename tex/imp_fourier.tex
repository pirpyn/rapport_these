\section{Analyse de Fourier de l'opérateur d'impédance}

Des résultat sur l'analyse spectrale de l'opérateur d'impédance ont déjà été présenté par \cite{hoppe_impedance_1995} pour des conducteurs plans et des cylindriques recouvert d'une couche de matériau. Nous les rappellerons ainsi la méthode pour les obtenir puis nous expliciterons ces derniers à des matériaux multi-couches sur les même objets

% \TODO{
%     Généraliser ce qui suit à des matériaux non borné à l'aide de fonctions rapidement décroissantes
% }

Soit $\OO$ un domaine fermé, bornée, de frontière régulière. Supposons que ces champs soient $L^2$ en espace et en temps: $(\vE(\v{x},t),\vH(\v{x},t)) \in L^2(\OO \times \RR_+) \cap L^2(\OO^c\times\RR_+)$ et vérifient les équations de Maxwell:
\begin{equation}
    \left\lbrace 
    \begin{matrix}
    \vrot \vE = -\mu \ddr{t}{\vH} \\
    \vrot \vH = \eps \ddr{t}{\vE}
    \end{matrix}
    \right.
\end{equation}

Puisque ces champs $\vE(\v{x},t),\vH(\v{x},t)$ sont $L^2$, on peut définir leurs transformées de Fourier $\hat{\vE}(\v{k},\w), \hat{\vH}(\v{k},\w)$ (\cite[Théorème de Plancherel, p.~153]{yosida_functional_1995}) telle que

\begin{equation}
    \hat{\vE} (\v{k},\w) = \frac{1}{\sqrt{2\pi}^3}\int_{\RR^3\times\RR_+} e^{-i(\v{k} \cdot \v{x}+\w t)}\vE(\v{x},t) \dd{x}\dd{t}\,, \quad \dd x = \prod\limits_{i=1}^3 \dd{x_i}
\end{equation}
\begin{equation}
    \vE(\v{x},t) = \frac{1}{\sqrt{2\pi}^3}\int_{\RR^3\times\RR_+} e^{i(\v{k} \cdot \v{x}+\w t)}\hat{\vE} (\v{k},\w) \dd{k} \dd{\w}\,, \quad \dd k = \prod\limits_{i=1}^3 \dd{k_i}
\end{equation}
La méthode pour trouver une expression de l'opérateur d'impédance est la suivante.
\begin{itemize}
\item Faire une transformée de Fourier partielle des champs, dépendante de la géométrie.
\item Réécrire le système d'équation de Maxwell simplifié.
\item Obtenir des EDO simple à résoudre sur les composantes des champs.
\item Utiliser les conditions limites pour obtenir les solutions particulières de cette EDO. 
\item En déduire l'opérateur d'impédance en Fourier.
\end{itemize}

Dans ce qui suit, on réalise au moins une transformée partielle en temps. La variable de Fourier associée est $\w$, et donc l'opérateur $\ddr{t}{~}$ est remplacé par $i\w$.

% \begin{tcolorbox}
% On ne différencie pas les champs $\vE,\vH$ de leurs transformées de Fourier $\hat \vE, \hat \vH$.
% \end{tcolorbox}

On va donc utiliser le système d'équations de Maxwell harmonique:
\begin{equation}
    \left\lbrace 
    \begin{matrix}
    \vrot \hat \vE(\v{x},\w)  &=& -i \omega \mu \hat \vH(\v{x},\w)  \\
    \vrot \hat \vH(\v{x},\w)  &=& i \omega \eps \hat \vE(\v{x},\w) 
    \end{matrix}
    \right.
    \label{eq:imp_fourier:intro:maxwell_harmonique}
\end{equation}

A partir de maintenant, la dépendance en $\w$ sera implicite: $\hat \vE(\v x) \equiv \hat \vE (\v x, \w)$.