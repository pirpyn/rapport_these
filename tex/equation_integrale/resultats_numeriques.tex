\section{Résultats numériques}

  \begin{figure}[!hbt]
    \centering
    \includegraphics[width=\textwidth]{images/ser/cone_sphere_mono.png}
    \caption{SER monostatique d'un cône sphère par équations intégrales couplées avec une CIOE où les coefficients sont calculés dans le cadre de l'approximation du plan tangent.}
    \label{fig:ser:cone-sphere-mono-M1}
  \end{figure}


  \begin{figure}[!hbt]
    \centering
    \includegraphics[width=\textwidth]{images/ser/sphere_bis.png}
    \caption{SER bistatique d'une sphère par équations intégrales couplées avec une CIOE où les coefficients sont calculés dans le cadre de l'approximation du plan tangent.}
    \label{fig:ser:sphere-bis-M1}
  \end{figure}

  Ces figures montrent que la SER calculée par EI avec CIOE est très proche de la solution de référence, calculée par un code axis symétrique de type équations intégrales +
\begin{REM}
  pas correct, une thèse, ce n'est pas un ensemble de notes!!!
\end{REM} 
   éléments finis avec maillage des matériaux.
