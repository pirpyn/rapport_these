% Ensemble de commande et de module pour ecrire/corriger plus facilement la these

% Pour chaque \newcommand{cmd} présent ici, un \newcommand{cmd}{} doit être present dans final

%\usepackage{draftwatermark}
%    \SetWatermarkText{BROUILLON} % Ecrire Brouillon en traversdes pages
%    \SetWatermarkScale{1}

% Creation de boite REM, qui apparaissent dans les signets
\newcounter{REM}
\newenvironment{REM}
    {%
        \stepcounter{REM}
        \hypertarget{REM\theREM}{}%
        \pdfbookmark[0]{REM \theREM}{REM\theREM}
        \begin{tcolorbox}[%
                title={Remarque \theREM},%
                colback=red!30!white,%
                colframe=red!75!black,%
                halign=flush center%
            ]
    }
    {
        \end{tcolorbox}%
        %
    }%

% \newcounter{REP}
\newenvironment{REP}
    {%
        % \stepcounter{REP}
        \hypertarget{REP\theREM}{}%
        \pdfbookmark[1]{REP \theREM}{REP\theREM}%
        \begin{tcolorbox}[%
                title={Réponse \theREM},%
                colback=green!30!white,%
                colframe=green!35!black,%
                halign=flush center%
            ]
    }
    {
        \end{tcolorbox}%
    }%

% Affiche devant chaque ligne en mode texte, le numéro de la ligne pour le corriger plus efficacement
\usepackage{lineno}
    \linenumbers

\geometry{
    top    = 2.5cm,%
    bottom = 2.5cm,%
    inner  = 1.5cm,% On décale pour avoir une marge plus grande pour annoter
    outer  = 4.5cm%
    }
% No show frame when using externalisation for plots
\ifcsname tikzexternalrealjob\endcsname
\else
\geometry{
   showframe, % Afficher les marges pour voir ce qui dépasse
}
\fi

