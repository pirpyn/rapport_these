\theoremstyle{plain} % style plain
\newtheorem{thm}{Théorème}[chapter]
\newtheorem{cor}[thm]{Corollaire}
\newtheorem{prop}[thm]{Proposition}
\newtheorem{lem}[thm]{Lemme}
%\newtheorem{conj}[thm]{Conjecture}
\newtheorem{rmq}[thm]{Remarque}
\newtheorem*{thmstar}{Théorème} % théorème non numéroté
\newtheorem*{conjstar}{Conjecture} % conjecture non numérotée
\newtheorem{defn}[thm]{Définition}
\newtheorem{hyp}[thm]{Hypothèse}

\theoremstyle{definition} % style definition
\newtheorem{exemple}[thm]{Exemple}
\newtheorem{question}[thm]{Question}
\newtheorem{remarque}[thm]{Remarque}
\newtheorem{notation}[thm]{Notation}

% Pour renommer ``preuve'' en ``démonstration''
\renewcommand{\proofname}{Démonstration}

% POUR L'ECRITURE de la thèse
\usepackage{bookmark}
\newcounter{TODO}
\newcommand{\TODO}[1]{%
    \stepcounter{TODO}
    \hypertarget{TODO\theTODO}{}%
    \begin{tcolorbox}[%
            title={TODO \theTODO},%
            colback=red!30!white,%
            colframe=red!75!black,%
            halign=flush center%
        ]
        #1%
    \end{tcolorbox}%
    \bookmark[dest={TODO\theTODO},level=-2]{TODO \theTODO}%
}

%redefinition
\renewcommand{\frac}[2]{\dfrac{#1}{#2}}
\renewcommand{\tilde}[1]{\widetilde{#1}}
\newcommand{\tagit}{\addtocounter{equation}{1}\tag{\theequation}}
\newcommand{\ds}{\displaystyle}

\newcommand{\secref}[1]{(\S \ref{#1})}
\newcommand{\pvect}{\wedge}
%vectorial notation
\renewcommand{\v}{}
\DeclareRobustCommand\v[1]{\underline{#1}}

%
\newcommand{\w}{\omega}

%mathematical symbol
\newcommand{\R}{\mathbb R}
\newcommand{\C}{\mathbb C}
\newcommand{\N}{\mathbb N}
\newcommand{\Z}{\mathbb Z}
\newcommand{\eps}{\epsilon}
\renewcommand{\d}{\partial}
\renewcommand{\O}{\Omega}

\newcommand{\dd}[1]{\dfrac{\d}{\d #1}}
\newcommand{\dr}[2]{\dfrac{\d #2}{\d #1}}

%Operator
\newcommand{\Hgrad}{\operatorname{H_{grad}}}
\newcommand{\Hrot}{\operatorname{H_{rot}}}
\newcommand{\Hdiv}{\operatorname{H_{div}}}
\newcommand{\Hess}{\v{\v{\operatorname{Hess}}}}
\renewcommand{\Re}{\operatorname{Re}}
\renewcommand{\Im}{\operatorname{Im}}

\newcommand{\sign}[1]{{\operatorname{sign}\left(#1\right)}}
\newcommand{\argmin}[1]{{\operatorname{argmin}_{#1}}}
\newcommand{\conj}[1]{{{#1}^*}} % opérateur conjugé
\newcommand{\Ker}{\operatorname{Ker}}
\newcommand{\Img}{\operatorname{Img}}
\newcommand{\Vect}[1]{\operatorname{Vect}\left\lbrace#1\right\rbrace}

\newcommand{\Tr}{\operatorname{T_R}}
\newcommand{\LD}{\operatorname{L_D}}
\newcommand{\LR}{\operatorname{L_R}}

\newcommand{\EFIE}{\operatorname{EFIE}}
\newcommand{\MFIE}{\operatorname{MFIE}}
\newcommand{\CFIE}{\operatorname{CFIE}}

%Maxwell notations
\newcommand{\E}{\v E}
\renewcommand{\H}{\v H}
\newcommand{\n}{\v n}
\newcommand{\J}{\v J}
\newcommand{\K}{\v K}
\newcommand{\M}{\v M}
\newcommand{\A}{\v A}

%differential operator
\newcommand{\grad}{\v \nabla}
\renewcommand{\div}{\v \nabla \cdot}
\newcommand{\rot}{\v \nabla \pvect}

\newcommand{\tgrad}{{\v{\operatorname{grad}}}}
\newcommand{\trot}{{\v{\operatorname{rot}}}}
\newcommand{\tdiv}{{\operatorname{div}}}

\newcommand{\grads}{\v \nabla_s}
\newcommand{\divs}{\v \nabla_s \cdot}
\newcommand{\rots}{\v \nabla_s \pvect}

\newcommand{\tgrads}{{\v{\operatorname{grad}}_t}}
\newcommand{\trots}{{\v{\operatorname{rot}}_t}}
\newcommand{\tdivs}{{\operatorname{div}_t}}

\newcommand{\lapl}{\v \Delta}

% helmotlz
\newcommand{\rtp}{r,\theta,\phi}
\newcommand{\tp}{\theta,\phi}

\renewcommand{\P}{\mathbb{P}}
\newcommand{\Pmn}{\P^m_n}

% cioe
\newcommand{\ov}[1]{\overline{#1}}

\newcommand{\x}{{x}}
\newcommand{\y}{{y}}
\newcommand{\z}{{z}}

\newcommand{\kO}{\mathbf{k_0}}
\newcommand{\etaO}{\boldsymbol{\eta_0}}
\newcommand{\muO}{\boldsymbol{\mu_0}}
\newcommand{\epsO}{\boldsymbol{\eps_0}}
\newcommand{\nuO}{\boldsymbol{\nu_0}}