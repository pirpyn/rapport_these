\section{Cas d'un objet plan infini}
    % Ce cas est très bien documenté (\cite{senior_approximate_x995},\cite{hoppe_impedance_x995}) et pose la méthodologie à adopter pour les objets courbes. 

    Dans un premier temps, on peut sans perte de généralités faire une rotation du repère pour avoir le plan orthogonal à $\v e_z$. Comme il est infini dans les directions $\v e_x, \v e_y$ et que le matériau est homogène isotrope, on utilise la transformée partielle en $x_1, x_2$ seulement.

    En explicitant par composante l'opérateur $\vrot$ , le problème \eqref{eq:imp_fourier:intro:maxwell_harmonique} s'écrit  
    \begin{align*}
        \left\lbrace 
        \begin{matrix}
        ik_y E_z  - \ddr{z}{E_y} = i \w \mu H_x \\
        \ddr{z}{E_x} - ik_x E_z = i\w \mu H_y \\
        ik_x E_y - ik_y E_x = i\w \mu H_z \\
        \end{matrix}
        \right. \quad 
        \left\lbrace 
        \begin{matrix}
        ik_y H_z  - \ddr{z}{H_y} = -i \w \eps E_x \\
        \ddr{z}{H_x} - ik_x H_z = -i\w \eps E_y \\
        ik_x H_y - ik_y H_x = -i\w \eps E_z \\
        \end{matrix}
        \right.
    \end{align*}

    Les composantes normales se déduisant des composantes tangentielles, on résout l'EDO matricielle à coefficients constants 
    suivante $\ddr{z}{}\v{X} = \mat{M} \v{X}$ où

    \begin{equation}
        \v{X} = 
        \begin{bmatrix}
        E_x \\ 
        E_y \\ 
        H_x \\ 
        H_y \\
        \end{bmatrix}\,,
        \mat{M} = \begin{bmatrix}
        0 & 0 & i\frac{k_xk_y}{\w\eps} & i\left(\w\mu - \frac{k_x^2}{\w\eps}\right)\\
        0 & 0 & -i\left(\w\mu - \frac{k_y^2}{\w\eps}\right) & -i\frac{k_xk_y}{\w\eps}\\
        -i\frac{k_xk_y}{\w\mu} & -i\left(\w\eps - \frac{k_x^2}{\w\mu}\right) & 0 & 0 \\
        i\left(\w\eps - \frac{k_y^2}{\w\mu}\right) & i\frac{k_xk_y}{\w\mu} & 0 & 0 \\
        \end{bmatrix}
    \end{equation}

    Pour résoudre cette EDO, nous allons chercher les vecteurs propres $V_i$ et les valeurs propres $\lambda_i$ associées de ce système. En effet, une solution générale de ce système s'écrit
    \begin{equation}
        \v{X}(z)= \sum\limits_{i=1}^{4}c_i e^{\lambda_i z} \v{V}_i \, c_i \in \CC
    \end{equation}
    On pose 
    \begin{equation}
        \mat{A} = \begin{bmatrix}
            i\frac{k_xk_y}{i\w\eps} & i\left(\w\mu - \frac{k_x^2}{\w\eps}\right) \\
            -i\left(\w\mu - \frac{k_y^2}{\w\eps}\right) & -i\frac{k_xk_y}{i\w\eps} \\
        \end{bmatrix}
        \quad
        \mat{B} = \begin{bmatrix}
            -i\frac{k_xk_y}{\w\mu} & -i\left(\w\eps - \frac{k_x^2}{\w\mu}\right) \\
            i\left(\w\eps - \frac{k_y^2}{\w\mu}\right) & i\frac{k_xk_y}{\w\mu} \\
        \end{bmatrix}
    \end{equation}
    Le déterminant de $\mat{M}-\lambda \mat{I}$ est
    \begin{align*}
        \det(\mat{M}-\lambda \mat{I}) &= 
        \begin{vmatrix}
            -\lambda \mI & \mA \\
            \mB & -\lambda \mI
        \end{vmatrix}
            = \frac{\det(- \lambda \mI - \mB(-\lambda \mI)^{-1} \mA)}{\det((-\lambda \mI)^{-1})} \\
            &= \det(\lambda^2 \mI - \mB\mA) \\
            &= (\lambda^2 + (\w^2\eps\mu - k_x^2 -k_y^2))^2
    \end{align*}
    On note alors 
    \begin{equation}
    k_3=\sqrt{\w^2\eps\mu - k_x^2 -k_y^2}
    \end{equation}

    Les valeurs propres sont alors 
    \begin{equation}
        \lambda_\pm = \pm i k_3
    \end{equation}
    Les espaces propres associés sont de dimension 2, on a 

    \begin{align}
    \Ker(\mat{M}-\lambda_+\mI)=\Vect{\v{V_+};\v{W_+}} \\
        \v{V_+} = 
        \begin{bmatrix}
        \lambda_+ \\
            0 \\
            -i\frac{k_xk_y}{\w\mu} \\
            i\left(\w\eps - \frac{k_y^2}{\w\mu}\right) \\
        \end{bmatrix}
        \,
        \v{W_+} = 
            \begin{bmatrix}
            0 \\
            \lambda_+ \\
            -i\left(\w\eps - \frac{k_x^2}{\w\mu}\right) \\
            i\frac{k_xk_y}{\w\mu} \\
        \end{bmatrix}
    \end{align}

    \begin{align}
    \Ker(\mat{M}-\lambda_-\mI)=\Vect{\v{V_-};\v{W_-}}\\
        \v{V_-} = 
        \begin{bmatrix}
            \lambda_- \\
            0 \\
            -i\frac{k_xk_y}{\w\mu} \\
            i\left(\w\eps - \frac{k_y^2}{\w\mu}\right) \\
        \end{bmatrix}
        \,
        \v{W_-} = 
        \begin{bmatrix}
            0 \\
            \lambda_- \\
            -i\left(\w\eps - \frac{k_x^2}{\w\mu}\right) \\
            i\frac{k_xk_y}{\w\mu} \\
        \end{bmatrix}
    \end{align}

    On a donc une solution générale du système: soient $(c_i)_{i} \in \CC^4$
    \begin{equation}
        \v{X}(z) = c_1e^{\lambda_+ z}\v{V_+}  + c_2e^{\lambda_+ z}\v{W_+} + c_3e^{\lambda_- z}\v{V_-} +c_4e^{\lambda_- z}\v{W_-}
    \end{equation}

    On exprime les champs $\vE_t(z)$ et $\v{e_z} \times \vH_t(z)$ car ce sont des quantités qui nous intéresse:

    \begin{align}
        \begin{bmatrix}
            E_x(z)\\
            E_y(z)\\
        \end{bmatrix}
        &=
        \begin{bmatrix}
            c_1 e^{\lambda_+ z} \lambda_{+} + c_3 e^{\lambda_- z} \lambda_{-} \\
            c_2 e^{\lambda_+ z} \lambda_{+} + c_4 e^{\lambda_- z} \lambda_{-}
        \end{bmatrix}\\
        &=ik_3\left( e^{ik_3 z}
        \begin{bmatrix}
            c_1 \\
            c_2
        \end{bmatrix}
        -e^{-ik_3 z}
        \begin{bmatrix}
            c_3 \\
            c_4
        \end{bmatrix}
        \right)
        \label{eq:imp_fourier:plan:generale_E}
    \end{align}

    \begin{align}
        \begin{bmatrix}
            -H_y(z)\\
            H_x(z)\\
        \end{bmatrix}
        &=
        \begin{bmatrix}
            -i\left(\w\eps - \frac{k_y^2}{\w\mu}\right) \left( c_1 e^{ik_3 z} + c_3 e^{-ik_3 z} \right) - i\frac{k_xk_y}{\w\mu} \left( c_2 e^{ik_3 z} + c_4 e^{-ik_3 z} \right)
            \\
            -i\frac{k_xk_y}{\w\mu} \left( c_1 e^{ik_3 z} + c_3 e^{-ik_3 z} \right) - i\left(\w\eps - \frac{k_x^2}{\w\mu}\right)\left( c_2 e^{ik_3 z} + c_4 e^{-ik_3 z} \right)
        \end{bmatrix} \\
        &=-i
        \begin{bmatrix}
        \left(\w\eps - \frac{k_y^2}{\w\mu}\right) & \frac{k_xk_y}{\w\mu}
        \\
        \frac{k_xk_y}{\w\mu} & \left(\w\eps - \frac{k_x^2}{\w\mu}\right) 
        \end{bmatrix}
        \left(
            e^{ik_3 z}
            \begin{bmatrix}
                c_1 \\
                c_2
            \end{bmatrix}
            +e^{-ik_3 z}
            \begin{bmatrix}
                c_3 \\
                c_4
            \end{bmatrix}
        \right)
        \label{eq:imp_fourier:plan:generale_H}
    \end{align}

    Notons
    \begin{align}
        \mC &=
        \begin{bmatrix}
            \left(\w\eps-\frac{k_y^2}{\w\mu}\right) & \frac{k_xk_y}{\w\mu}\\
            \frac{k_xk_y}{\w\mu} & \left(\w\eps-\frac{k_x^2}{\w\mu}\right)
        \end{bmatrix}
    \end{align}

    Comme $\det(\mC) = k_3^2\frac{\eps}{\mu}=\frac{k_3^2}{\eta^2}$ alors une condition nécessaire pour trouver l'opérateur d'impédance est que $k_3$ soit non nul\footnote{$k_3$ peut s'annuler pour des $\eps,\mu$ réels.}.
    % On peut noter d'après \cite[eq.~(6)]{stupfel_2011}

    % \begin{equation}
    %     \mC^{-1}= \frac{\eta^2}{k_3^2}\left(k^2\mI - \mat{L_R}\right)
    %     \label{eq:imp_fourier:plan:C}
    % \end{equation}


    %%%%%%%%%%%%%%%%%%%%%%%%%%%%%%%%%%%%%%%%%%%%%%%%%%%%%%%%%%%%%%%%%%%%%%%
    %%%%%%%%%%%%%%%%%%%%%%%%%%%%%%%%%%%%%%%%%%%%%%%%%%%%%%%%%%%%%%%%%%%%%%%
    %%%%%%%%%%%%%%%%%%%%%%%%%%%%%%%%%%%%%%%%%%%%%%%%%%%%%%%%%%%%%%%%%%%%%%%


    \subsection{Opérateur d'impédance pour une couche}

        \begin{figure}[h!]
        \centering
        \begin{tikzpicture}
        \tikzmath{
    \largeur = 6;
    \hauteur = 1;
    \milieu = 1.3;
    \xC = \largeur;
    \xA = 0;
}

%% 1ere couche
\tikzmath{
    \yC = \hauteur;
    \yA = 0;
}

\coordinate (A) at (\xA,\yA);
\coordinate (B) at (\xA,\yC);
\coordinate (C) at (\xC,\yC);

\draw ($(B)!0.5!(C)$) node [above] {vide};


\fill [lightgray] (A) rectangle (C);
\draw ($(A)!0.5!(C)$) node {$\peps,\pmu,d$};
\draw (B) -- (C) node [right] {$\z = 0$};

%% Le repère
\tikzmath{
    \xD = \xC + 1.5;
}

\coordinate (n) at (\xD,\yA);

\draw [->] (n) -- ++(0,1) node [at end, right] {$\v{\z}$};
\draw [->] (n) -- ++(1,0) node [at end, right] {$\v{\x}$};

\draw (n) circle(0.1cm) node [below=0.1cm] {$\v{\y}$};
\draw (n) +(135:0.1cm) -- +(315:0.1cm);
\draw (n) +(45:0.1cm) -- +(225:0.1cm);

%% Le conducteur
\tikzmath{
    \yC = \yC - \hauteur;
    \yA = \yA - 0.5*\hauteur;
}

\coordinate (A) at (\xA,\yA);
\coordinate (B) at (\xA,\yC);
\coordinate (C) at (\xC,\yC);
\draw (B) -- (C);

\fill [pattern=north east lines] (A) rectangle (C);



        \end{tikzpicture}
        \end{figure}

        \begin{thm}
            Si on suppose
                \begin{align}
                k_3d &\not = \frac{\pi}{2}+n\pi\,, \forall n \in \NN
            \end{align}
            Alors l'opérateur d'impédance $\mZ$ est défini par la relation de récurrence : 
            \begin{align}
            \mZ_m &= -i\eta\frac{\tan\left(k_3d\right)}{kk_3}
                \begin{bmatrix}
                   k^2-k_x^2  & -k_xk_y\\
                    -k_xk_y & k^2-k_y^2\\
                \end{bmatrix}
            \end{align}
        \end{thm}

        \begin{proof}
            Nous utilisons la condition limite 
            \begin{equation}
                \begin{bmatrix}
                    E_x(-d)\\
                    E_y(-d)\\
                \end{bmatrix}
                =
                \begin{bmatrix}
                    0\\
                    0\\
                \end{bmatrix}
            \end{equation}

            De \eqref{eq:imp_fourier:plan:generale_E}, on déduit

            \begin{align}
                \begin{bmatrix}
                    c_1 \\
                    c_2
                \end{bmatrix}
                = e^{2ik_3 d}
                \begin{bmatrix}
                    c_3 \\
                    c_4
                \end{bmatrix}
            \end{align}

            On définit l'opérateur d'impédance la matrice $\mat Z$ tel que 
            \begin{equation}
                \vE_t(0) = \mat Z \left(\v{e_z} \times \vH_t(0)\right)
            \end{equation}

            De ce qui précède on déduit que,

            \begin{align}
                \begin{bmatrix}
                    E_x(0)\\
                    E_y(0)\\
                \end{bmatrix}
                &=ik_3\left( e^{i2k_3 d} -1 \right)
                \begin{bmatrix}
                    c_3 \\
                    c_4
                \end{bmatrix} \\
                \begin{bmatrix}
                    -H_y(0)\\
                    H_x(0)\\
                \end{bmatrix}
                & = - i\left(e^{i2k_3 d} +1 \right)
                \mC
                \begin{bmatrix}
                c_3 \\
                c_4
                \end{bmatrix}
            \end{align}

            En supposant $k_3d \not = \frac{\pi}{2} + n\pi$, on déduit donc que
            \begin{align}
                \mat{Z} &=  - k_3 \frac{e^{i2k_3d} -1}{e^{i2k_3d} +1} \mC^{-1} 
                \\
                &= -\frac{\eta^2}{k_3} \frac{e^{i2k_3d} -1}{e^{i2k_3d} +1}
                    \begin{bmatrix}
                       \left(\w\eps-\frac{k_x^2}{\w\mu}\right)  & -\frac{k_xk_y}{\w\mu}\\
                        -\frac{k_xk_y}{\w\mu} &  \left(\w\eps-\frac{k_y^2}{\w\mu}\right)
                    \end{bmatrix}
                \\
                &= -i\eta\frac{\tan\left(k_3d\right)}{kk_3}
                    \begin{bmatrix}
                       k^2-k_x^2  & -k_xk_y\\
                        -k_xk_y & k^2-k_y^2\\
                    \end{bmatrix}
            \end{align}

        \end{proof}
        %On remarque que $\det(\mat{Z}) = i\frac{\eta^2}{k_3}\eta\tan(k_3d)$ et donc pour un matériau $(\eps,\mu,d)$ donné, l'opérateur d'impédance n'est pas inversible pour tous  $(k_x,k_y) \in \RR^2, n \in \NN$, $k_x^2+k_y^2 =  \w^2\eps\mu - \frac{1}{d^2}\left(\frac{\pi}{2} + n\pi\right)^2$, qui ne peut être vérifié que si $\eps\mu$ est réel\footnote{Comme $\eps, \mu$ sont à partie réelle (resp. imaginaire) strictement positive (resp. négative), alors ce n'est vrai pour les matériaux à partie imaginaire nulle.}. 

        En pratique, on simplifie $k_y = 0$ soit des solutions se propageant dans le plan $xz$. Grâce à cette hypothèse, on trouve que $\mC, \mZ$ sont des matrices diagonales. 

        De plus, on exprime souvent l'impédance selon la polarisation. Dans le cas plan, le champ $\vE$-TE correspond à $E_y \v{e_y}$, le champ $\vE$-TM à $E_x \v{e_x}$, tandis que le champ $\vH$-TM correspond à $H_x \v{e_x}$ et le champ $\vH$-TE correspond à $H_y \v{e_y}$.
        L'opérateur $\mZ$ peut se réécrire comme 
        \begin{equation}
            \mZ = 
            \begin{bmatrix}
                Z_{TM} & 0 
                \\
                0 & Z_{TE}
            \end{bmatrix}
        \end{equation}

    %%%%%%%%%%%%%%%%%%%%%%%%%%%%%%%%%%%%%%%%%%%%%%%%%%%%%%%%%%%%%%%%%%%%%%%
    %%%%%%%%%%%%%%%%%%%%%%%%%%%%%%%%%%%%%%%%%%%%%%%%%%%%%%%%%%%%%%%%%%%%%%%
    %%%%%%%%%%%%%%%%%%%%%%%%%%%%%%%%%%%%%%%%%%%%%%%%%%%%%%%%%%%%%%%%%%%%%%%

    \subsection{Opérateur d'impédance pour plusieurs couches}
        On suppose que l'on a $n$ couches de matériaux : 

        \begin{figure}[h!btp]
            \centering
            \begin{tikzpicture}
                \tikzmath{
    \largeur = 6;
    \hauteur = 0.5;
    \milieu = 1.3;
    \xC = \largeur;
    \xA = 0;
}

%% 1ere couche
\tikzmath{
    \yC = \hauteur;
    \yA = 0;
}

\coordinate (A) at (\xA,\yA);
\coordinate (B) at (\xA,\yC);
\coordinate (C) at (\xC,\yC);

\draw ($(B)!0.5!(C)$) node [above] {vide};


\fill [lightgray] (A) rectangle (C);
\draw ($(A)!0.5!(C)$) node {$\eps_n,\mu_n,d_n$};
\draw (B) -- (C) node [right] {$e_3 = 0$};

%% Des couches
\tikzmath{
    \yC = \yC - \hauteur;
    \yA = \yA - \milieu*\hauteur;
}

\coordinate (A) at (\xA,\yA);
\coordinate (B) at (\xA,\yC);
\coordinate (C) at (\xC,\yC);

\fill [lightgray]    (A) rectangle (C);
\fill [pattern=dots] (A) rectangle (C);
\draw (B) -- (C);

%% N ieme couche
\tikzmath{
    \yC = \yC - \milieu*\hauteur;
    \yA = \yA - \hauteur;
}

\coordinate (A) at (\xA,\yA);
\coordinate (B) at (\xA,\yC);
\coordinate (C) at (\xC,\yC);
\fill [lightgray] (A) rectangle (C);
\draw ($(A)!0.5!(C)$) node {$\eps_1,\mu_1,d_1$};
\draw (B) -- (C);

%% Le repère
\tikzmath{
    \xD = \xC + 0.5;
}

\coordinate (n) at (\xD,\yA);
\draw [->] (n) -- ++(1,0) node [at end, right] {$\v{e_1}$};
\draw [->] (n) -- ++(0,1) node [at end, right] {$\v{e_3}$};

\draw (n) circle(0.1cm) node [below=0.1cm] {$\v{e_2}$};
\draw (n) +(135:0.1cm) -- +(315:0.1cm);
\draw (n) +(45:0.1cm) -- +(225:0.1cm);

%% Le conducteur
\tikzmath{
    \yC = \yC - \hauteur;
    \yA = \yA - 0.5*\hauteur;
}

\coordinate (A) at (\xA,\yA);
\coordinate (B) at (\xA,\yC);
\coordinate (C) at (\xC,\yC);
\draw (B) -- (C);

\fill [pattern=north east lines] (A) rectangle (C);



            \end{tikzpicture}
        \end{figure}

        Pour chaque couche caractérisée par $(\eps_m,\mu_m,d_m)$, on définit:
        \begin{align}
        k_{3m} &= \sqrt{w^2\eps_m\mu_m - k_y^2 - k_x^2}
        \\
        \mC_m &=
            \begin{bmatrix}
                \left(\w\eps_m-\frac{k_y^2}{\w\mu_m}\right) & \frac{k_xk_y}{\w\mu_m}\\
                \frac{k_xk_y}{\w\mu_m} & \left(\w\eps_m-\frac{k_x^2}{\w\mu_m}\right)
            \end{bmatrix}
        \end{align}

        On définit aussi la profondeur de la couche $m$, $l_m = -\sum_{i=1}^{n-m} d_{m} $. 
        On définit pour chaque interface, l'opérateur $\mZ_m$ tel que $\vE_t(l_m) = \mZ_m \left(\v{e_z} \times \vH_t(l_m)\right)$. 

        On cherche alors l'opérateur pour la couche la moins profonde: $\mZ_n$ telle que $\vE_t(0) = \mZ_n\vH_t(0)$

        \begin{thm}
            Soit $\mZ_0 = \mat{0}_{\mathcal{M}_2(\CC)}$.

            Si pour tout $0<m < n$
                \begin{align}
                \det{\mC_m} = \frac{k_{3m}\eps_m^2}{\mu_m^2} &\not = 0 \\
                \det\left(k_{3m}\mI \pm \mZ_{m-1}\mC_m \right) &\not = 0 \\
                k_{3m}d_m &\not = \frac{\pi}{2}+n\pi\,, \forall n \in \NN \\
                \det\left(k_{3m}\mI - i\tan(k_{3m}d_m)\mZ_{m-1}\mC_m\right) &\not = 0
            \end{align}
            Alors l'opérateur d'impédance $\mZ =  \mZ_n$ est défini par la relation de récurrence : 
            \begin{align}
            \mZ_m &= -k_{3m}
            \left(ik_{3m}\tan\left(k_{3m}d_m\right)\mI - \mZ_{m-1}\mC_m\right)
            \left(k_{3m}\mI - i\tan\left(k_{3m}d_m\right)\mZ_{m-1}\mC_m\right)^{-1}
            \mC_m^{-1}
            \end{align}
        \end{thm}

        \begin{proof}
            Par récurrence, un empilement à $n$ couches se ramène à un empilement à une couche avec la condition:
            \begin{equation}
                \begin{bmatrix}
                    E_x(-d)\\
                    E_y(-d)\\
                \end{bmatrix}
                =
                \mat {Z_{d}} 
                \begin{bmatrix}
                    -H_y(-d)\\
                    H_x(-d)\\
                \end{bmatrix}
            \end{equation}

            À l'initialisation, la condition limite sur le conducteur impose $\mat{Z} = \mat{0}_{\mathcal{M}_2(\CC)}$.

            On reprend donc tous les résultats de la partie précédente. Notamment, de \eqref{eq:imp_fourier:plan:generale_E} et \eqref{eq:imp_fourier:plan:generale_H}, on déduit que

            \begin{equation}
                \begin{bmatrix}
                    E_x(-d)\\
                    E_y(-d)\\
                \end{bmatrix}
                = ik_3\left( e^{-ik_3 d}
                \begin{bmatrix}
                    c_1 \\
                    c_2
                \end{bmatrix}
                -e^{ik_3 d}
                \begin{bmatrix}
                    c_3 \\
                    c_4
                \end{bmatrix}
                \right)
            \end{equation}

            \begin{equation}
                \begin{bmatrix}
                    -H_y(-d)\\
                    H_x(-d)\\
                \end{bmatrix}
                =-i
                \mC
                \left(
                    e^{-ik_3 d}
                    \begin{bmatrix}
                        c_1 \\
                        c_2
                    \end{bmatrix}
                    +e^{ik_3 d}
                    \begin{bmatrix}
                        c_3 \\
                        c_4
                    \end{bmatrix}
                \right)
            \end{equation}

            \begin{equation}
                ik_3\left( e^{-ik_3 d}
                \begin{bmatrix}
                    c_1 \\
                    c_2
                \end{bmatrix}
                -e^{ik_3 d}
                \begin{bmatrix}
                    c_3 \\
                    c_4
                \end{bmatrix}
                \right)
                =-i\mat{Z_d}\mC
                \left(
                    e^{-ik_3 d}
                    \begin{bmatrix}
                        c_1 \\
                        c_2
                    \end{bmatrix}
                    +e^{ik_3 d}
                    \begin{bmatrix}
                        c_3 \\
                        c_4
                    \end{bmatrix}
                \right)
            \end{equation}

            \begin{equation}
                \left(k_3\mI + \mat{Z_d}\mC\right)
                \begin{bmatrix}
                    c_1 \\
                    c_2
                \end{bmatrix}
                = e^{i2k_3 d} \left(k_3\mI - \mat{Z_d}\mC\right)
                \begin{bmatrix}
                    c_3 \\
                    c_4
                \end{bmatrix}
            \end{equation}

            On pose
            \begin{align}
                \mA_\pm &= k_3\mI \pm \mat{Z_d}\mC
            \end{align}

            On remarque que par définition, $\mA_+$ et $\mA_-$ commutent.

            Pour continuer il faut exprimer un vecteur en fonction de l'autre. On suppose donc $\pm k_3$ ne sont pas des valeurs propres de $\mat{Z_d}\mC$ et l'on déduit que

            \begin{align}
                \begin{bmatrix}
                    c_1 \\
                    c_2
                \end{bmatrix}
                &= e^{i2 k_3 d} \mA_+^{-1}\mA_-
                \begin{bmatrix}
                    c_3 \\
                    c_4
                \end{bmatrix}
                \\
                & = \mat{F}
                \begin{bmatrix}
                    c_3 \\
                    c_4
                \end{bmatrix}
            \end{align}

            \begin{align}
                \begin{bmatrix}
                    E_x(0)\\
                    E_y(0)\\
                \end{bmatrix}
                &=ik_3\left(\mat{F} - \mI \right)
                \begin{bmatrix}
                    c_3 \\
                    c_4
                \end{bmatrix}
            \end{align}

            \begin{align}
                \begin{bmatrix}
                    -H_y(0)\\
                    H_x(0)\\
                \end{bmatrix}
                &=-i\mC \left(  \mat{F} + \mI  \right)
                \begin{bmatrix}
                        c_3 \\
                        c_4
                \end{bmatrix}
            \end{align}

            On suppose qu'en plus de $\mA_+$ et $\mA_-$, $\mat{F} + \mI$ est inversible, on va utiliser la commutativité de $\mA_+$ et $\mA_-$.

            Alors l'opérateur d'impédance $\mat{Z}$ s'exprime

            \begin{align}
                \mat{Z}
                &=-k_3\left(\mat{F} - \mI \right)\left(\mat{F}+ \mI \right)^{-1}\mC^{-1}
                \\
                &=-k_3\mA_+^{-1}\left(e^{i2 k_3 d}\mA_- - \mA_+ \right)\left(e^{i2 k_3 d}\mA_- + \mA_+ \right)^{-1}\mA_+\mC^{-1}
                \\
                &= -k_3\left( e^{i2 k_3 d} \mA_- -  \mA_+\right)
                \left( e^{i2 k_3 d} \mA_- + \mA_+ \right)^{-1}\mC^{-1}
                \\
                &= -k_3\left(\left( e^{i2 k_3 d} - 1 \right)\mI - \left( e^{i2 k_3 d} + 1 \right) \mat{Z_d}\mC \right)
                \left( \left( e^{i2 k_3 d} + 1 \right)\mI - \left( e^{i2 k_3 d} - 1 \right)\mat{Z_d}\mC \right)^{-1}\mC^{-1}   
            \end{align}

            En supposant que $\forall n \in \NN \,, k_3d\not = \frac{\pi}{2}+n\pi$, on a

            \begin{equation}
            \mZ = -k_3\left(ik_3\tan(k_3 d)\mI - \mat{Z_d}\mC \right)
                \left( k_3\mI - i\tan(k_3 d)\mat{Z_d}\mC \right)^{-1}\mC^{-1} 
            \end{equation}

            à condition que 
            \begin{align}
                \det\left(k_3\mI \pm \mat{Z_d}\mC \right) \not = 0 \\
                k_3d\not = \frac{\pi}{2}+n\pi\,, \forall n \in \NN \\
                \det\left(k_3\mI - i\tan(k_3d)\mat{Z_d}\mC\right) \not = 0
            \end{align}

        \end{proof}

    %%%%%%%%%%%%%%%%%%%%%%%%%%%%%%%%%%%%%%%%%%%%%%%%%%%%%%%%%%%%%%%%%%%%%%%
    %%%%%%%%%%%%%%%%%%%%%%%%%%%%%%%%%%%%%%%%%%%%%%%%%%%%%%%%%%%%%%%%%%%%%%%
    %%%%%%%%%%%%%%%%%%%%%%%%%%%%%%%%%%%%%%%%%%%%%%%%%%%%%%%%%%%%%%%%%%%%%%%

    \subsection{Applications numériques}

        La figure \ref{fig:imp_fourier:plan:hoppe} permet de vérifier les résultats de \cite[p.~33]{hoppe_impedance_1995} pour une couche de matériau sans perte.

        \pgfplotstableread[col sep=semicolon]{tikz/csv/impedance_plan_1_couche.csv}{\hoppeimpplan}
        \begin{figure}[!hbt]
            \centering
            \begin{tikzpicture}[scale=1]
                \begin{axis}[
                        title={},
                        ylabel={$\Im(Z)$},
                        xlabel={$k_x\slash k_0$},
                        mark repeat=20,
                        legend pos=outer north east
                    ]
                    \legend{TM,TE}
                    \addplot [black,thick,mark=x] table [x={s}, y={imag(z.tm)}] {\hoppeimpplan};
                    \addplot [black,thick,mark=*] table [x={s}, y={imag(z.te)}] {\hoppeimpplan};
                \end{axis}
            \end{tikzpicture}
            \caption{$\eps = 4, \mu = 1, d=0.015\text{m}, f=1\text{GHz}$}
            \label{fig:imp_fourier:plan:hoppe}
        \end{figure}

        La figure \ref{fig:imp_fourier:plan:soudais} permet de vérifier les résultats de \cite{soudais_3d_2017} pour une couche de matériau sans perte où $\mC$ n'est pas inversible pour un $s\simeq 0.9$.

        \pgfplotstableread[col sep=semicolon]{tikz/csv/impedance_plan_1_couche_soudais.csv}{\soudaisimpplan}
        \begin{figure}[!hbt]
            \centering
            \begin{tikzpicture}[scale=1]
                \begin{axis}[
                        title={},
                        ylabel={$\Im(Z)$},
                        xlabel={$k_x\slash k_0$},
                        mark repeat=20,
                        legend pos=outer north east
                    ]
                    \legend{TM,TE}
                    \addplot [black,thick,mark=x] table [x={s}, y={imag(z.tm)}] {\soudaisimpplan};
                    \addplot [black,thick,mark=*] table [x={s}, y={imag(z.te)}] {\soudaisimpplan};
                \end{axis}
            \end{tikzpicture}
            \caption{$\eps = 4, \mu = 1, d=0.035\text{m}, f=12\text{GHz}$}
            \label{fig:imp_fourier:plan:soudais}
        \end{figure}
