
\subsection{Forme générale des solutions pour un plan infini}

\TODO{Passer de $x_1,x_2,x_3$ à $x,y,z$ ? Peut être plus lisible.}

% Ce cas est très bien documenté (\cite{senior_approximate_1995},\cite{hoppe_impedance_1995}) et pose la méthodologie à adopter pour les objets courbes. 

Dans un premier temps, on peut sans perte de généralités faire une rotation du repère pour avoir le plan orthogonal à $\v e_3$. Comme il est infini dans les directions $\v e_1, \v e_2$ et que le matériau est homogène isotrope, on utilise la transformée partielle en $x_1, x_2$ seulement.

\renewcommand{\x}{e_1}
\renewcommand{\y}{e_2}
\renewcommand{\z}{e_3}
\renewcommand{\peps}{\eps}
\renewcommand{\pmu}{{\mu}}
\begin{figure}[h!]
\centering
\begin{tikzpicture}
\coordinate (l) at (-2,0);
\coordinate (r) at (2,0);
\coordinate (m) at ($(r)!0.5!(l)$);

\fill [lightgray] (l) rectangle (2,1) ;
\foreach \t in {-2.2,-2.1,...,2} {
\draw plot[domain=0:0.3] (\t + \x, -\x, 0);
}
\coordinate (ll) at (-2.2,-0.3);
\fill [white] (ll) rectangle (l) ;
\coordinate (rr) at (2,-0.31);
\fill [white] (rr) rectangle (2.3,0) ;


\coordinate (n) at (-4,0);

\coordinate (lt) at (-2,1);
\coordinate (rt) at (2,1);
\coordinate (mt) at ($(rt)!0.5!(lt)$);

\draw (l) -- (r) node [at end,right] {$z = -d$};
\draw (lt)  -- (rt) node [at end,right] {$z = 0$};

\draw (lt) node [above right] {$\eps_0,\mu_0$};
\draw ($(l)!0.5!(lt)$) node [right] {$\eps_1,\mu_1$};

\draw [->] (n) -- ++(0,1) node [at end, right] {$z$};
\draw [->] (n) -- ++(1,0) node [at end, right] {$x$};

\draw (n) circle(0.1cm) node [below=0.1cm] {$y$};
\draw (n) +(135:0.1cm) -- +(315:0.1cm);
\draw (n) +(45:0.1cm) -- +(225:0.1cm);

%\draw [->>,thick] (lt) ++ (1,1) -- (mt) ;


\end{tikzpicture}
\end{figure}

En explicitant par composante l'opérateur $\vrot$ , le problème \eqref{eq:imp_fourier:intro:maxwell_harmonique} s'écrit  
\begin{align*}
    \left\lbrace 
    \begin{matrix}
    ik_2 E_3  - \ddr{x_3}{E_ 2} = i \w \mu H_1 \\
    \ddr{x_3}{E_1} - ik_1 E_3 = i\w \mu H_2 \\
    ik_1 E_2 - ik_2 E_1 = i\w \mu H_3 \\
    \end{matrix}
    \right. \quad 
    \left\lbrace 
    \begin{matrix}
    ik_2 H_3  - \ddr{x_3}{H_ 2} = -i \w \eps E_1 \\
    \ddr{x_3}{H_1} - ik_1 H_3 = -i\w \eps E_2 \\
    ik_1 H_2 - ik_2 H_1 = -i\w \eps E_3 \\
    \end{matrix}
    \right.
\end{align*}

Les composantes normales se déduisant des composantes tangentielles, on résout l'EDO matricielle à coefficients constants 
suivante $\ddr{x_3}{}\v{X} = \mat{M} \v{X}$ où

\begin{equation}
    \v{X} = 
    \begin{bmatrix}
    E_1 \\ 
    E_2 \\ 
    H_1 \\ 
    H_2 \\
    \end{bmatrix}\,,
    \mat{M} = \begin{bmatrix}
    0 & 0 & i\frac{k_1k_2}{\w\eps} & i\left(\w\mu - \frac{k_1^2}{\w\eps}\right)\\
    0 & 0 & -i\left(\w\mu - \frac{k_2^2}{\w\eps}\right) & -i\frac{k_1k_2}{\w\eps}\\
    -i\frac{k_1k_2}{\w\mu} & -i\left(\w\eps - \frac{k_1^2}{\w\mu}\right) & 0 & 0 \\
    i\left(\w\eps - \frac{k_2^2}{\w\mu}\right) & i\frac{k_1k_2}{\w\mu} & 0 & 0 \\
    \end{bmatrix}
\end{equation}

Pour résoudre cette EDO, nous allons chercher les vecteurs propres $V_i$ et les valeurs propres $\lambda_i$ associées de ce système. En effet, une solution générale de ce système s'écrit
\begin{equation}
    \v{X}(x_3)= \sum\limits_{i=1}^{4}c_i e^{\lambda_i x_3} \v{V}_i \, c_i \in \CC
\end{equation}
On pose 
\begin{equation}
    \mat{A} = \begin{bmatrix}
        i\frac{k_1k_2}{i\w\eps} & i\left(\w\mu - \frac{k_1^2}{\w\eps}\right) \\
        -i\left(\w\mu - \frac{k_2^2}{\w\eps}\right) & -i\frac{k_1k_2}{i\w\eps} \\
    \end{bmatrix}
    \quad
    \mat{B} = \begin{bmatrix}
        -i\frac{k_1k_2}{\w\mu} & -i\left(\w\eps - \frac{k_1^2}{\w\mu}\right) \\
        i\left(\w\eps - \frac{k_2^2}{\w\mu}\right) & i\frac{k_1k_2}{\w\mu} \\
    \end{bmatrix}
\end{equation}
Le déterminant de $\mat{M}-\lambda \mat{I}$ est
\begin{align*}
    \det(\mat{M}-\lambda \mat{I}) &= 
    \begin{vmatrix}
        -\lambda \mI & \mA \\
        \mB & -\lambda \mI
    \end{vmatrix}
        = \frac{\det(- \lambda \mI - \mB(-\lambda \mI)^{-1} \mA)}{\det((-\lambda \mI)^{-1})} \\
        &= \det(\lambda^2 \mI - \mB\mA) \\
        &= (\lambda^2 + (\w^2\eps\mu - k_1^2 -k_2^2))^2
\end{align*}
On note alors 
\begin{equation}
k_3=\sqrt{\w^2\eps\mu - k_1^2 -k_2^2}
\end{equation}

Les valeurs propres sont alors 
\begin{equation}
    \lambda_\pm = \pm i k_3
\end{equation}
Les espaces propres associés sont de dimension 2, on a 

\begin{align}
\Ker(\mat{M}-\lambda_+\mI)=\Vect{\v{V_+};\v{W_+}} \\
    \v{V_+} = 
    \begin{bmatrix}
    \lambda_+ \\
        0 \\
        -i\frac{k_1k_2}{\w\mu} \\
        i\left(\w\eps - \frac{k_2^2}{\w\mu}\right) \\
    \end{bmatrix}
    \,
    \v{W_+} = 
        \begin{bmatrix}
        0 \\
        \lambda_+ \\
        -i\left(\w\eps - \frac{k_1^2}{\w\mu}\right) \\
        i\frac{k_1k_2}{\w\mu} \\
    \end{bmatrix}
\end{align}

\begin{align}
\Ker(\mat{M}-\lambda_-\mI)=\Vect{\v{V_-};\v{W_-}}\\
    \v{V_-} = 
    \begin{bmatrix}
        \lambda_- \\
        0 \\
        -i\frac{k_1k_2}{\w\mu} \\
        i\left(\w\eps - \frac{k_2^2}{\w\mu}\right) \\
    \end{bmatrix}
    \,
    \v{W_-} = 
    \begin{bmatrix}
        0 \\
        \lambda_- \\
        -i\left(\w\eps - \frac{k_1^2}{\w\mu}\right) \\
        i\frac{k_1k_2}{\w\mu} \\
    \end{bmatrix}
\end{align}

On a donc une solution générale du système: soient $(c_i)_{i} \in \CC^4$
\begin{equation}
    \v{X}(x_3) = c_1e^{\lambda_+ x_3}\v{V_+}  + c_2e^{\lambda_+ x_3}\v{W_+} + c_3e^{\lambda_- x_3}\v{V_-} +c_4e^{\lambda_- x_3}\v{W_-}
\end{equation}

On exprime les champs $\vE_t(x_3)$ et $\v{e_3} \times \vH_t(x_3)$ car ce sont des quantités qui nous intéresse:

\begin{align}
    \begin{bmatrix}
        E_1(x_3)\\
        E_2(x_3)\\
    \end{bmatrix}
    &=
    \begin{bmatrix}
        c_1 e^{\lambda_+ x_3} \lambda_{+} + c_3 e^{\lambda_- x_3} \lambda_{-} \\
        c_2 e^{\lambda_+ x_3} \lambda_{+} + c_4 e^{\lambda_- x_3} \lambda_{-}
    \end{bmatrix}\\
    &=ik_3\left( e^{ik_3 x_3}
    \begin{bmatrix}
        c_1 \\
        c_2
    \end{bmatrix}
    -e^{-ik_3 x_3}
    \begin{bmatrix}
        c_3 \\
        c_4
    \end{bmatrix}
    \right)
    \label{eq:imp_fourier:plan:generale_E}
\end{align}

\begin{align}
    \begin{bmatrix}
        -H_2(x_3)\\
        H_1(x_3)\\
    \end{bmatrix}
    &=
    \begin{bmatrix}
        -i\left(\w\eps - \frac{k_2^2}{\w\mu}\right) \left( c_1 e^{ik_3 x_3} + c_3 e^{-ik_3 x_3} \right) - i\frac{k_1k_2}{\w\mu} \left( c_2 e^{ik_3 x_3} + c_4 e^{-ik_3 x_3} \right)
        \\
        -i\frac{k_1k_2}{\w\mu} \left( c_1 e^{ik_3 x_3} + c_3 e^{-ik_3 x_3} \right) - i\left(\w\eps - \frac{k_1^2}{\w\mu}\right)\left( c_2 e^{ik_3 x_3} + c_4 e^{-ik_3 x_3} \right)
    \end{bmatrix} \\
    &=-i
    \begin{bmatrix}
    \left(\w\eps - \frac{k_2^2}{\w\mu}\right) & \frac{k_1k_2}{\w\mu}
    \\
    \frac{k_1k_2}{\w\mu} & \left(\w\eps - \frac{k_1^2}{\w\mu}\right) 
    \end{bmatrix}
    \left(
        e^{ik_3 x_3}
        \begin{bmatrix}
            c_1 \\
            c_2
        \end{bmatrix}
        +e^{-ik_3 x_3}
        \begin{bmatrix}
            c_3 \\
            c_4
        \end{bmatrix}
    \right)
    \label{eq:imp_fourier:plan:generale_H}
\end{align}

Notons
\begin{align}
    \mC &=
    \begin{bmatrix}
        \left(\w\eps-\frac{k_2^2}{\w\mu}\right) & \frac{k_1k_2}{\w\mu}\\
        \frac{k_1k_2}{\w\mu} & \left(\w\eps-\frac{k_1^2}{\w\mu}\right)
    \end{bmatrix}
\end{align}

Comme $\det(\mC) = k_3^2\frac{\eps}{\mu}=\frac{k_3^2}{\eta^2}$ alors une condition nécessaire pour trouver l'opérateur d'impédance est que $k_3$ soit non nul\footnote{$k_3$ peut s'annuler pour des $\eps,\mu$ réels.}.
% On peut noter d'après \cite[eq.~(6)]{stupfel_2011}

% \begin{equation}
%     \mC^{-1}= \frac{\eta^2}{k_3^2}\left(k^2\mI - \mat{L_R}\right)
%     \label{eq:imp_fourier:plan:C}
% \end{equation}

\subsection{Plan infini avec une couche}

\begin{thm}
    Si on suppose
        \begin{align}
        k_3d &\not = \frac{\pi}{2}+n\pi\,, \forall n \in \NN
    \end{align}
    Alors l'opérateur d'impédance $\mZ$ est défini par la relation de récurrence : 
    \begin{align}
    \mZ_m &= i\eta\frac{\tan\left(k_3d\right)}{kk_3}
        \begin{bmatrix}
           k^2-k_1^2  & -k_1k_2\\
            -k_1k_2 & k^2-k_2^2\\
        \end{bmatrix}
    \end{align}
\end{thm}

\begin{proof}
    Nous utilisons la condition limite 
    \begin{equation}
        \begin{bmatrix}
            E_1(-d)\\
            E_2(-d)\\
        \end{bmatrix}
        =
        \begin{bmatrix}
            0\\
            0\\
        \end{bmatrix}
    \end{equation}

    De \eqref{eq:imp_fourier:plan:generale_E}, on déduit

    \begin{align}
        \begin{bmatrix}
            c_1 \\
            c_2
        \end{bmatrix}
        = e^{2ik_3 d}
        \begin{bmatrix}
            c_3 \\
            c_4
        \end{bmatrix}
    \end{align}

    On définit l'opérateur d'impédance la matrice $\mat Z$ tel que 
    \begin{equation}
        \vE_t(0) = \mat Z \left(\v{e_3} \times \vH_t(0)\right)
    \end{equation}

    De ce qui précède on déduit que,

    \begin{align}
        \begin{bmatrix}
            E_1(0)\\
            E_2(0)\\
        \end{bmatrix}
        &=ik_3\left( e^{i2k_3 d} -1 \right)
        \begin{bmatrix}
            c_3 \\
            c_4
        \end{bmatrix} \\
        \begin{bmatrix}
            -H_2(0)\\
            H_1(0)\\
        \end{bmatrix}
        & = - i\left(e^{i2k_3 d} +1 \right)
        \mC
        \begin{bmatrix}
        c_3 \\
        c_4
        \end{bmatrix}
    \end{align}

    En supposant $k_3d \not = \frac{\pi}{2} + n\pi$, on déduit donc que
    \begin{align}
        \mat{Z} &=  - k_3 \frac{e^{i2k_3d} -1}{e^{i2k_3d} +1} \mC^{-1} 
        \\
        &= -\frac{\eta^2}{k_3} \frac{e^{i2k_3d} -1}{e^{i2k_3d} +1}
            \begin{bmatrix}
               \left(\w\eps-\frac{k_1^2}{\w\mu}\right)  & -\frac{k_1k_2}{\w\mu}\\
                -\frac{k_1k_2}{\w\mu} &  \left(\w\eps-\frac{k_2^2}{\w\mu}\right)
            \end{bmatrix}
        \\
        &= i\eta\frac{\tan\left(k_3d\right)}{kk_3}
            \begin{bmatrix}
               k^2-k_1^2  & -k_1k_2\\
                -k_1k_2 & k^2-k_2^2\\
            \end{bmatrix}
    \end{align}

\end{proof}
%On remarque que $\det(\mat{Z}) = i\frac{\eta^2}{k_3}\eta\tan(k_3d)$ et donc pour un matériau $(\eps,\mu,d)$ donné, l'opérateur d'impédance n'est pas inversible pour tous  $(k_1,k_2) \in \RR^2, n \in \NN$, $k_1^2+k_2^2 =  \w^2\eps\mu - \frac{1}{d^2}\left(\frac{\pi}{2} + n\pi\right)^2$, qui ne peut être vérifié que si $\eps\mu$ est réel\footnote{Comme $\eps, \mu$ sont à partie réelle (resp. imaginaire) strictement positive (resp. négative), alors ce n'est vrai pour les matériaux à partie imaginaire nulle.}. 

En pratique, on simplifie $k_2 = 0$ soit des solutions se propageant dans le plan $xz$. Grâce à cette hypothèse, on trouve que $\mC, \mZ$ sont des matrices diagonales. 

De plus, on exprime souvent l'impédance selon la polarisation. Dans le cas plan, le champ $\vE$-TE correspond à $E_2 \v{e_2}$, le champ $\vE$-TM à $E_1 \v{e_1}$, tandis que le champ $\vH$-TM correspond à $H_1 \v{e_1}$ et le champ $\vH$-TE correspond à $H_2 \v{e_2}$.
L'opérateur $\mZ$ peut se réécrire comme 
\begin{equation}
    \mZ = 
    \begin{bmatrix}
        Z_{TM} & 0 
        \\
        0 & Z_{TE}
    \end{bmatrix}
\end{equation}

\subsection{Plan infini avec plusieurs couches}
On suppose que l'on a $n$ couches de matériaux : 

\renewcommand{\z}{e_3}
\renewcommand{\x}{e_1}
\renewcommand{\y}{e_2}
\begin{figure}[h!btp]
    \centering
    \begin{tikzpicture}
        \tikzmath{
    \largeur = 6;
    \hauteur = 1;
    \milieu = 1.3;
    \xC = \largeur;
    \xA = 0;
}

%% 1ere couche
\tikzmath{
    \yC = \hauteur;
    \yA = 0;
}

\coordinate (A) at (\xA,\yA);
\coordinate (B) at (\xA,\yC);
\coordinate (C) at (\xC,\yC);

\draw ($(B)!0.5!(C)$) node [above] {vide};


\fill [lightgray] (A) rectangle (C);
\draw ($(A)!0.5!(C)$) node {$\peps_n,\pmu_n,d_n$};
\draw (B) -- (C) node [right] {$\z = 0$};

%% Des couches
\tikzmath{
    \yC = \yC - \hauteur;
    \yA = \yA - \milieu*\hauteur;
}

\coordinate (A) at (\xA,\yA);
\coordinate (B) at (\xA,\yC);
\coordinate (C) at (\xC,\yC);

\fill [lightgray]    (A) rectangle (C);
\fill [pattern=dots] (A) rectangle (C);
\draw (B) -- (C);

%% N ieme couche
\tikzmath{
    \yC = \yC - \milieu*\hauteur;
    \yA = \yA - \hauteur;
}

\coordinate (A) at (\xA,\yA);
\coordinate (B) at (\xA,\yC);
\coordinate (C) at (\xC,\yC);
\fill [lightgray] (A) rectangle (C);
\draw ($(A)!0.5!(C)$) node {$\peps_1,\pmu_1,d_1$};
\draw (B) -- (C);

%% Le repère
\tikzmath{
    \xD = \xC + 0.5;
}

\coordinate (n) at (\xD,\yA);

\draw [->] (n) -- ++(0,1) node [at end, right] {$\v{\z}$};
\draw [->] (n) -- ++(1,0) node [at end, right] {$\v{\x}$};

\draw (n) circle(0.1cm) node [below=0.1cm] {$\v{\y}$};
\draw (n) +(135:0.1cm) -- +(315:0.1cm);
\draw (n) +(45:0.1cm) -- +(225:0.1cm);

%% Le conducteur
\tikzmath{
    \yC = \yC - \hauteur;
    \yA = \yA - 0.5*\hauteur;
}

\coordinate (A) at (\xA,\yA);
\coordinate (B) at (\xA,\yC);
\coordinate (C) at (\xC,\yC);
\draw (B) -- (C);

\fill [pattern=north east lines] (A) rectangle (C);



    \end{tikzpicture}
\end{figure}

Pour chaque couche caractérisée par $(\eps_m,\mu_m,d_m)$, on définit:
\begin{align}
k_{3m} &= \sqrt{w^2\eps_m\mu_m - k_2^2 - k_1^2}
\\
\mC_m &=
    \begin{bmatrix}
        \left(\w\eps_m-\frac{k_2^2}{\w\mu_m}\right) & \frac{k_1k_2}{\w\mu_m}\\
        \frac{k_1k_2}{\w\mu_m} & \left(\w\eps_m-\frac{k_1^2}{\w\mu_m}\right)
    \end{bmatrix}
\end{align}

On définit aussi la profondeur de la couche $m$, $l_m = -\sum_{i=1}^{n-m} d_{m} $. On donc pour chaque interface, l'opérateur $\mZ_m$ tel que $\vE_t(l_m) = \mZ_m \left(\v{e_3} \times \vH_t(l_m)\right)$. 

On cherche alors l'opérateur pour la couche la moins profonde: $\mZ_n$ telle que $\vE_t(0) = \mZ_n\vH_t(0)$

\begin{thm}
    Soit $\mZ_0 = \mat{0}_{\mathcal{M}_2(\CC)}$.

    Si pour tout $0<m < n$
        \begin{align}
        \det{\mC_m} = \frac{k_{3m}\eps_m^2}{\mu_m^2} &\not = 0 \\
        \det\left(k_{3m}\mI \pm \mZ_{m-1}\mC_m \right) &\not = 0 \\
        k_{3m}d_m &\not = \frac{\pi}{2}+n\pi\,, \forall n \in \NN \\
        \det\left(\mI + i\tan(k_{3m}d_m)\mZ_{m-1}\mC_m\right) &\not = 0
    \end{align}
    Alors l'opérateur d'impédance $\mZ =  \mZ_n$ est défini par la relation de récurrence : 
    \begin{align}
    \mZ_m &= k_{3m}
    \left(i\tan\left(k_{3m}d_m\right)\mI + \mZ_{l_{m-1}}\mC_m\right)
    \left(\mI + i\tan\left(k_{3m}d_m\right)\mZ_{l_{m-1}}\mC_m\right)^{-1}
    \mC_m^{-1}
    \end{align}
\end{thm}

\begin{proof}
    Par récurrence, un empilement à $n$ couches se ramène à un empilement à une couche avec la condition:
    \begin{equation}
        \begin{bmatrix}
            E_1(-d)\\
            E_2(-d)\\
        \end{bmatrix}
        =
        \mat {Z_{d}} 
        \begin{bmatrix}
            -H_2(-d)\\
            H_1(-d)\\
        \end{bmatrix}
    \end{equation}

    À l'initialisation, la condition limite sur le conducteur impose $\mat{Z} = \mat{0}_{\mathcal{M}_2(\CC)}$.

    On reprend donc tous les résultats de la partie précédente. Notamment, de \eqref{eq:imp_fourier:plan:generale_E} et \eqref{eq:imp_fourier:plan:generale_H}, on déduit que

    \begin{equation}
        \begin{bmatrix}
            E_1(-d)\\
            E_2(-d)\\
        \end{bmatrix}
        = ik_3\left( e^{-ik_3 d}
        \begin{bmatrix}
            c_1 \\
            c_2
        \end{bmatrix}
        -e^{ik_3 d}
        \begin{bmatrix}
            c_3 \\
            c_4
        \end{bmatrix}
        \right)
    \end{equation}

    \begin{equation}
        \begin{bmatrix}
            -H_2(-d)\\
            H_1(-d)\\
        \end{bmatrix}
        =-i
        \mC
        \left(
            e^{-ik_3 d}
            \begin{bmatrix}
                c_1 \\
                c_2
            \end{bmatrix}
            +e^{ik_3 d}
            \begin{bmatrix}
                c_3 \\
                c_4
            \end{bmatrix}
        \right)
    \end{equation}

    \begin{equation}
        ik_3\left( e^{-ik_3 d}
        \begin{bmatrix}
            c_1 \\
            c_2
        \end{bmatrix}
        -e^{ik_3 d}
        \begin{bmatrix}
            c_3 \\
            c_4
        \end{bmatrix}
        \right)
        =-i\mat{Z_d}\mC
        \left(
            e^{-ik_3 d}
            \begin{bmatrix}
                c_1 \\
                c_2
            \end{bmatrix}
            +e^{ik_3 d}
            \begin{bmatrix}
                c_3 \\
                c_4
            \end{bmatrix}
        \right)
    \end{equation}

    \begin{equation}
        \left(k_3\mI + \mat{Z_d}\mC\right)
        \begin{bmatrix}
            c_1 \\
            c_2
        \end{bmatrix}
        = e^{i2k_3 d} \left(k_3\mI - \mat{Z_d}\mC\right)
        \begin{bmatrix}
            c_3 \\
            c_4
        \end{bmatrix}
    \end{equation}

    On pose
    \begin{align}
        \mA_\pm &= k_3\mI \pm \mat{Z_d}\mC
    \end{align}

    On remarque que par définition, $\mA_+$ et $\mA_-$ commutent.

    Pour continuer il faut exprimer un vecteur en fonction de l'autre. On suppose donc $\pm k_3$ ne sont pas des valeurs propres de $\mat{Z_d}\mC$ et l'on déduit que

    \begin{align}
        \begin{bmatrix}
            c_1 \\
            c_2
        \end{bmatrix}
        &= e^{i2 k_3 d} \mA_+^{-1}\mA_-
        \begin{bmatrix}
            c_3 \\
            c_4
        \end{bmatrix}
        \\
        & = \mat{F}
        \begin{bmatrix}
            c_3 \\
            c_4
        \end{bmatrix}
    \end{align}

    \begin{align}
        \begin{bmatrix}
            E_1(0)\\
            E_2(0)\\
        \end{bmatrix}
        &=ik_3\left(\mat{F} - \mI \right)
        \begin{bmatrix}
            c_3 \\
            c_4
        \end{bmatrix}
    \end{align}

    \begin{align}
        \begin{bmatrix}
            -H_2(0)\\
            H_1(0)\\
        \end{bmatrix}
        &=-i\mC \left(  \mat{F} + \mI  \right)
        \begin{bmatrix}
                c_3 \\
                c_4
        \end{bmatrix}
    \end{align}

    On suppose qu'en plus de $\mA_+$ et $\mA_-$, $\mat{F} + \mI$ est inversible, on va utiliser la commutativité de $\mA_+$ et $\mA_-$.

    Alors l'opérateur d'impédance $\mat{Z}$ s'exprime

    \begin{align}
        \mat{Z}
        &=-k_3\left(\mat{F} - \mI \right)\left(\mat{F}+ \mI \right)^{-1}\mC^{-1}
        \\
        &=-k_3\mA_+^{-1}\left(e^{i2 k_3 d}\mA_- - \mA_+ \right)\left(e^{i2 k_3 d}\mA_- + \mA_+ \right)^{-1}\mA_+\mC^{-1}
        \\
        &= -k_3\left( e^{i2 k_3 d} \mA_- -  \mA_+\right)
        \left( e^{i2 k_3 d} \mA_- + \mA_+ \right)^{-1}\mC^{-1}
        \\
        &= -k_3\left(\left( e^{i2 k_3 d} - 1 \right)\mI - \left( e^{i2 k_3 d} + 1 \right) \mat{Z_d}\mC \right)
        \left( \left( e^{i2 k_3 d} + 1 \right)\mI - \left( e^{i2 k_3 d} - 1 \right)\mat{Z_d}\mC \right)^{-1}\mC^{-1}   
    \end{align}

    En supposant que $\forall n \in \NN \,, k_3d\not = \frac{\pi}{2}+n\pi$, on a

    \begin{equation}
    \mZ = k_3\left(i\tan(k_3 d)\mI + \mat{Z_d}\mC \right)
        \left( \mI + i\tan(k_3 d)\mat{Z_d}\mC \right)^{-1}\mC^{-1} 
    \end{equation}

    à condition que 
    \begin{align}
        \det\left(k_3\mI \pm \mat{Z_d}\mC \right) \not = 0 \\
        k_3d\not = \frac{\pi}{2}+n\pi\,, \forall n \in \NN \\
        \det\left(\mI + i\tan(k_3d)\mat{Z_d}\mC\right) \not = 0
    \end{align}

\end{proof}

\subsection{Applications numériques}

\TODO{
    Mettre des applications numériques sur les cas litigieux.
}
