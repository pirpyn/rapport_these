\section{Cas d'un objet plan infini}
    % Ce cas est très bien documenté (\cite{senior_approximate_x995},\cite{hoppe_impedance_x995}) et pose la méthodologie à adopter pour les objets courbes.

    \begin{figure}[!h]
        \begin{center}
            \tikzsetnextfilename{plan_1_couche}
            \begin{tikzpicture}
                \tikzmath{
    \largeur = 6;
    \hauteur = 1;
    \milieu = 1.3;
    \xC = \largeur;
    \xA = 0;
}

%% 1ere couche
\tikzmath{
    \yC = \hauteur;
    \yA = 0;
}

\coordinate (A) at (\xA,\yA);
\coordinate (B) at (\xA,\yC);
\coordinate (C) at (\xC,\yC);

\draw ($(B)!0.5!(C)$) node [above] {vide};


\fill [lightgray] (A) rectangle (C);
\draw ($(A)!0.5!(C)$) node {$\peps,\pmu,d$};
\draw (B) -- (C) node [right] {$\z = 0$};

%% Le repère
\tikzmath{
    \xD = \xC + 1.5;
}

\coordinate (n) at (\xD,\yA);

\draw [->] (n) -- ++(0,1) node [at end, right] {$\v{\z}$};
\draw [->] (n) -- ++(1,0) node [at end, right] {$\v{\x}$};

\draw (n) circle(0.1cm) node [below=0.1cm] {$\v{\y}$};
\draw (n) +(135:0.1cm) -- +(315:0.1cm);
\draw (n) +(45:0.1cm) -- +(225:0.1cm);

%% Le conducteur
\tikzmath{
    \yC = \yC - \hauteur;
    \yA = \yA - 0.5*\hauteur;
}

\coordinate (A) at (\xA,\yA);
\coordinate (B) at (\xA,\yC);
\coordinate (C) at (\xC,\yC);
\draw (B) -- (C);

\fill [pattern=north east lines] (A) rectangle (C);



            \end{tikzpicture}
        \end{center}
    \end{figure}

    Dans un premier temps, on peut sans perte de généralités faire une rotation du repère pour avoir le plan orthogonal à \(\vect{z}\). Comme il est infini dans les directions \(\vect{e_x} \vect{e_y}\) et que le matériau est homogène isotrope, on utilise la transformée partielle en \(x, y\) seulement.
    \begin{equation}
        \vE(x,y,z) = \frac{1}{2\pi}\iint_{\RR^2} e^{i(k_x x + k_y y)}\hat{\vE} (k_x,k_y,z) \dd{k_x}\dd{k_y}
    \end{equation}

    \begin{prop}
        Soient \(\vect{X}(k_x,k_z,z) =
        \begin{bmatrix}
        \hat{E_x} &
        \hat{E_y} &
        \hat{H_x} &
        \hat{H_y}
        \end{bmatrix}^t\),
        où \((\hat \vE,\hat \vH)\) sont des solutions du problème \eqref{eq:imp_fourier:intro:maxwell_harmonique}, alors il existe une matrice \(\mat{M}\) ne dépendant pas de \(z\) telle que
        \begin{equation}
            \ddr{z}{}\vect{X}(k_x,k_z,z) = \mat{M}(k_x,k_z) \vect{X}(k_x,k_z,z)
            \label{eq:imp_fourier:plan:edo}
        \end{equation}
    \end{prop}

    \begin{proof}
        En utilisant les multiplicateur de Fourier associés aux opérateurs différentiels, le problème \eqref{eq:imp_fourier:intro:maxwell_harmonique} s'écrit
        \begin{align*}
            \left\lbrace
            \begin{matrix}
            ik_y \hat{E_z}  - \ddr{z}{\hat{E_y}} = -i \w \mu \hat{H_x} \\
            \ddr{z}{\hat{E_x}} - ik_x \hat{E_z} = -i\w \mu \hat{H_y} \\
            ik_x \hat{E_y} - ik_y \hat{E_x} = -i\w \mu \hat{H_z} \\
            \end{matrix}
            \right. \quad
            \left\lbrace
            \begin{matrix}
            ik_y \hat{H_z}  - \ddr{z}{\hat{H_y}} = i \w \eps \hat{E_x} \\
            \ddr{z}{\hat{H_x}} - ik_x \hat{H_z} = i\w \eps \hat{E_y} \\
            ik_x \hat{H_y} - ik_y \hat{H_x} = i\w \eps \hat{E_z} \\
            \end{matrix}
            \right.
        \end{align*}

        Les composantes normales se déduisant des composantes tangentielles, on résout l'équation différentielle  matricielle à coefficients constants
        suivante \(\ddr{z}{}\vect{X} = \mat{M} \vect{X}\) où

        \begin{equation}
            \mat{M} = \begin{bmatrix}
            0 & 0 & -i\frac{k_xk_y}{\w\eps} & -i\left(\w\mu - \frac{k_x^2}{\w\eps}\right)\\
            0 & 0 & i\left(\w\mu - \frac{k_y^2}{\w\eps}\right) & i\frac{k_xk_y}{\w\eps}\\
            i\frac{k_xk_y}{\w\mu} & i\left(\w\eps - \frac{k_x^2}{\w\mu}\right) & 0 & 0 \\
            -i\left(\w\eps - \frac{k_y^2}{\w\mu}\right) & -i\frac{k_xk_y}{\w\mu} & 0 & 0 \\
            \end{bmatrix}
        \end{equation}
    \end{proof}

    \begin{prop}
        On définit
        \begin{equation}
            k_3=\sqrt{\w^2\eps\mu - k_x^2 -k_y^2} \quad \lambda_\pm=\pm i k_3
        \end{equation}
        \begin{equation}
            \vect{V_\pm} =
            \begin{bmatrix}
            \lambda_\pm \\
                0 \\
                i\frac{k_xk_y}{\w\mu} \\
                -i\left(\w\eps - \frac{k_y^2}{\w\mu}\right)
            \end{bmatrix}
            \quad
            \vect{W_\pm} =
                \begin{bmatrix}
                0 \\
                \lambda_\pm \\
                i\left(\w\eps - \frac{k_x^2}{\w\mu}\right) \\
                -i\frac{k_xk_y}{\w\mu}
            \end{bmatrix}
        \end{equation}
        alors
        \begin{subequations}
            \begin{align}
                \Ker(\mat{M}-\lambda_+\mI)=\Vect{\vect{V_+};\vect{W_+}}\\
                \Ker(\mat{M}-\lambda_-\mI)=\Vect{\vect{V_-};\vect{W_-}}
            \end{align}
        \end{subequations}
    \end{prop}

    \begin{proof}
        % Pour résoudre cette EDO, nous allons chercher les vecteurs propres \(V_i\) et les valeurs propres \(\lambda_i\) associées de ce système. En effet, une solution générale de ce système s'écrit
        % \begin{equation}
        %     \vect{X}(z)= \sum\limits_{i=1}^{4}c_i e^{\lambda_i z} \vect{V}_i \, c_i \in \CC
        % \end{equation}
        On pose
        \begin{equation*}
            \mat{A} = -\begin{bmatrix}
                i\frac{k_xk_y}{i\w\eps} & i\left(\w\mu - \frac{k_x^2}{\w\eps}\right) \\
                -i\left(\w\mu - \frac{k_y^2}{\w\eps}\right) & -i\frac{k_xk_y}{i\w\eps} \\
            \end{bmatrix}
            \quad
            \mat{B} = -\begin{bmatrix}
                -i\frac{k_xk_y}{\w\mu} & -i\left(\w\eps - \frac{k_x^2}{\w\mu}\right) \\
                i\left(\w\eps - \frac{k_y^2}{\w\mu}\right) & i\frac{k_xk_y}{\w\mu} \\
            \end{bmatrix}
        \end{equation*}
        Le déterminant de \(\mat{M}-\lambda \mat{I}\) est
        \begin{align*}
            \det(\mat{M}-\lambda \mat{I}) &=
            \begin{vmatrix}
                -\lambda \mI & \mA \\
                \mB & -\lambda \mI
            \end{vmatrix}
                = \frac{\det(- \lambda \mI - \mB(-\lambda \mI)^{-1} \mA)}{\det((-\lambda \mI)^{-1})} \\
                &= \det(\lambda^2 \mI - \mB\mA) \\
                &= (\lambda^2 + (\w^2\eps\mu - k_x^2 -k_y^2))^2 \\
                &= (\lambda^2 - \lambda_\pm^2)^2
        \end{align*}
        % On note alors
        % \begin{equation}
        % k_3=\sqrt{\w^2\eps\mu - k_x^2 -k_y^2}
        % \end{equation}

        % Les valeurs propres sont alors
        % \begin{equation}
        %     \lambda_\pm = \pm i k_3
        % \end{equation}
        % Les espaces propres associés sont de dimension 2, on a

        Par un calcul immédiat, on vérifie que \(\mat{M}\vect{V}_\pm = \lambda_\pm\vect{V}_\pm\) et \(\mat{M}\vect{W}_\pm = \lambda_\pm\vect{W}_\pm\).
    \end{proof}

    \begin{prop}
        On pose
        \begin{align}
            \mC &=
            \begin{bmatrix}
                \left(\w\eps-\frac{k_y^2}{\w\mu}\right) & \frac{k_xk_y}{\w\mu}\\
                \frac{k_xk_y}{\w\mu} & \left(\w\eps-\frac{k_x^2}{\w\mu}\right)
            \end{bmatrix}
        \end{align}
        alors
        \begin{subequations}
            \label{eq:imp_fourier:plan:champs}
            \begin{align}
                \begin{bmatrix}
                    \hat{E_x}(k_x,k_y,z)\\
                    \hat{E_y}(k_x,k_y,z)\\
                \end{bmatrix}
                &=ik_3\left( e^{ik_3 z}
                \begin{bmatrix}
                    c_1 \\
                    c_2
                \end{bmatrix}
                -e^{-ik_3 z}
                \begin{bmatrix}
                    c_3 \\
                    c_4
                \end{bmatrix}
                \right)
                \label{eq:imp_fourier:plan:champs:E}
                \\
                \begin{bmatrix}
                    -\hat{H_y}(k_x,k_y,z)\\
                    \hat{H_x}(k_x,k_y,z)\\
                \end{bmatrix}
                &=i
                \mC
                \left(
                    e^{ik_3 z}
                    \begin{bmatrix}
                        c_1 \\
                        c_2
                    \end{bmatrix}
                    +e^{-ik_3 z}
                    \begin{bmatrix}
                        c_3 \\
                        c_4
                    \end{bmatrix}
                \right)
                \label{eq:imp_fourier:plan:champs:H}
            \end{align}
        \end{subequations}
    \end{prop}

    \begin{proof}
        On déduit des vecteur propres une solution générale de \eqref{eq:imp_fourier:plan:edo}.

        Soient \((c_i)_{i} \in \CC^4\)
        \begin{equation}
            \vect{X}(k_x,k_y,z) = c_1e^{\lambda_+ z}\vect{V_+}  + c_2e^{\lambda_+ z}\vect{W_+} + c_3e^{\lambda_- z}\vect{V_-} +c_4e^{\lambda_- z}\vect{W_-}
        \end{equation}

        On exprime \(\hat \vE_t(k_x,k_y,z)\) et \(\vect{e_z} \pvect \hat \vH_t(k_x,k_y,z)\)
        \begin{align}
            \begin{bmatrix}
                \hat{E_x}(k_x,k_y,z)\\
                \hat{E_y}(k_x,k_y,z)\\
            \end{bmatrix}
            &=
            \begin{bmatrix}
                c_1 e^{\lambda_+ z} \lambda_{+} + c_3 e^{\lambda_- z} \lambda_{-} \\
                c_2 e^{\lambda_+ z} \lambda_{+} + c_4 e^{\lambda_- z} \lambda_{-}
            \end{bmatrix}\\
            &=ik_3\left( e^{ik_3 z}
            \begin{bmatrix}
                c_1 \\
                c_2
            \end{bmatrix}
            -e^{-ik_3 z}
            \begin{bmatrix}
                c_3 \\
                c_4
            \end{bmatrix}
            \right)
        \end{align}

        \begin{align}
            \begin{bmatrix}
                -\hat{H_y}(k_x,k_y,z)\\
                \hat{H_x}(k_x,k_y,z)\\
            \end{bmatrix}
            &=
            \begin{bmatrix}
                i\left(\w\eps - \frac{k_y^2}{\w\mu}\right) \left( c_1 e^{ik_3 z} + c_3 e^{-ik_3 z} \right) + i\frac{k_xk_y}{\w\mu} \left( c_2 e^{ik_3 z} + c_4 e^{-ik_3 z} \right)
                \\
                i\frac{k_xk_y}{\w\mu} \left( c_1 e^{ik_3 z} + c_3 e^{-ik_3 z} \right) + i\left(\w\eps - \frac{k_x^2}{\w\mu}\right)\left( c_2 e^{ik_3 z} + c_4 e^{-ik_3 z} \right)
            \end{bmatrix} \\
            &=i
            \mC
            \left(
                e^{ik_3 z}
                \begin{bmatrix}
                    c_1 \\
                    c_2
                \end{bmatrix}
                +e^{-ik_3 z}
                \begin{bmatrix}
                    c_3 \\
                    c_4
                \end{bmatrix}
            \right)
        \end{align}

        Comme \(\det(\mC) = k_3^2\frac{\eps}{\mu}=\frac{k_3^2}{\eta^2}\) alors une condition nécessaire pour trouver l'opérateur d'impédance est que \(k_3\) soit non nul ce qui aussi une hypothèse à vérifier car les valeurs propres doivent être non nulles.\footnote{\(k_3\) peut s'annuler pour des \(\eps,\mu\) réels.}.

    \end{proof}

    On définit la matrice \(\hat{\mLR}\)\footnote{Nous verrons plus tard que ce choix est dicté par les opérateurs différentiel qui approchent le symbole.} telle que
    \begin{equation}
        \mC = \frac{1}{k\eta}\left(k^2\mI  - \hat{\mLR}\right)
    \end{equation}

    On définit la matrice \(\hat{\mLD}\)\footnote{Idem.} telle que
    \begin{equation}
        \mC^{-1} = \frac{\eta}{kk_3^2}\left(k^2\mI + \hat{\mLD}\right)
    \end{equation}

    % On peut noter d'après \cite[eq.~(6)]{stupfel_2011}

    % \begin{equation}
    %     \mC^{-1}= \frac{\eta^2}{k_3^2}\left(k^2\mI - \mat{L_R}\right)
    %     \label{eq:imp_fourier:plan:C}
    % \end{equation}


    %%%%%%%%%%%%%%%%%%%%%%%%%%%%%%%%%%%%%%%%%%%%%%%%%%%%%%%%%%%%%%%%%%%%%%%
    %%%%%%%%%%%%%%%%%%%%%%%%%%%%%%%%%%%%%%%%%%%%%%%%%%%%%%%%%%%%%%%%%%%%%%%
    %%%%%%%%%%%%%%%%%%%%%%%%%%%%%%%%%%%%%%%%%%%%%%%%%%%%%%%%%%%%%%%%%%%%%%%


    \subsection{Opérateur d'impédance pour une couche}

        \begin{defn}
            On définit le symbole de l'opérateur d'impédance, la matrice \(\hat \mZ(k_x,k_y)\) tel que
            \begin{equation}
                \hat \vE_t(k_x,k_y,0) = \hat \mZ(k_x,k_y) \left(\vect{e_z} \pvect \hat \vH_t(k_x,k_y,0)\right)
            \end{equation}
        \end{defn}

        \begin{thm}
            Supposons que
            \begin{subequations}
                \label{eq:imp_fourier:plan:hyp_1_c}
                \begin{align}
                    k_3     & \not =0 \\
                    k_3d    & \not = \frac{\pi}{2}+n\pi\,, \forall n \in \NN
                \end{align}
            \end{subequations}

            Alors
            \begin{align}
            \label{eq:imp_plan:symb_z:1c}
            \hat \mZ(k_x,k_y) &= i\eta\frac{\tan\left(k_3d\right)}{kk_3}
                \begin{bmatrix}
                   k^2-k_x^2  & -k_xk_y\\
                    -k_xk_y & k^2-k_y^2\\
                \end{bmatrix}
            \end{align}
        \end{thm}

        \begin{proof}
            Nous utilisons la condition limite
            \begin{equation}
                \begin{bmatrix}
                    \hat{E_x}(k_x,k_y,-d)\\
                    \hat{E_y}(k_x,k_y,-d)\\
                \end{bmatrix}
                =
                \begin{bmatrix}
                    0\\
                    0\\
                \end{bmatrix}
            \end{equation}

            De \eqref{eq:imp_fourier:plan:champs:E}, on déduit
            \begin{align}
                \begin{bmatrix}
                    c_1 \\
                    c_2
                \end{bmatrix}
                = e^{2ik_3 d}
                \begin{bmatrix}
                    c_3 \\
                    c_4
                \end{bmatrix}
            \end{align}

            En injectant ce qui précède dans \eqref{eq:imp_fourier:plan:champs}, on déduit que
            \begin{align}
                \begin{bmatrix}
                    \hat{E_x}(k_x,k_y,0)\\
                    \hat{E_y}(k_x,k_y,0)\\
                \end{bmatrix}
                &=ik_3\left( e^{i2k_3 d} -1 \right)
                \begin{bmatrix}
                    c_3 \\
                    c_4
                \end{bmatrix} \\
                \begin{bmatrix}
                    -\hat{H_y}(k_x,k_y,0)\\
                    \hat{H_x}(k_x,k_y,0)\\
                \end{bmatrix}
                & = i\left(e^{i2k_3 d} +1 \right)
                \mC
                \begin{bmatrix}
                c_3 \\
                c_4
                \end{bmatrix}
            \end{align}

            En supposant \(k_3d \not = \frac{\pi}{2} + n\pi\), on déduit donc que
            \begin{align}
                \hat \mZ(k_x,k_y) &=  k_3 \frac{e^{i2k_3d} -1}{e^{i2k_3d} +1} \mC^{-1}
                \\
                &= \frac{\eta}{kk_3} \frac{e^{i2k_3d} -1}{e^{i2k_3d} +1}\left(k^2\mI + \hat{\mLD}\right)
                \\
                &= i\eta\frac{\tan\left(k_3d\right)}{kk_3}\left(k^2\mI + \hat{\mLD}\right)
            \end{align}

        \end{proof}
        %On remarque que \(\det(\mat{Z}) = i\frac{\eta^2}{k_3}\eta\tan(k_3d)\) et donc pour un matériau \((\eps,\mu,d)\) donné, l'opérateur d'impédance n'est pas inversible pour tous  \((k_x,k_y) \in \RR^2, n \in \NN\), \(k_x^2+k_y^2 =  \w^2\eps\mu - \frac{1}{d^2}\left(\frac{\pi}{2} + n\pi\right)^2\), qui ne peut être vérifié que si \(\eps\mu\) est réel\footnote{Comme \(\eps, \mu\) sont à partie réelle (resp. imaginaire) strictement positive (resp. négative), alors ce n'est vrai pour les matériaux à partie imaginaire nulle.}.

        Dans l'article de \cite{marceaux_high-order_2000}, l'opérateur est exprimé comme
        \begin{equation}
            \vn \pvect \hat{\vE} = \hat{\mathfrak{Z}} \hat{\vH_t}
        \end{equation}

        Pour une couche de matériau, cet article énonce que 
        \begin{align}
            \hat{\mathfrak{Z}} &= -i\eta\frac{\tan(k_3d)}{kk_3}\left(k^2\mI - \hat{\mLR}\right)
        \end{align}

        Sachant que dans le cadre des vecteurs tangents \(\vn\pvect \vx = \mA \vy_t\) est égal à \(\vx_t = -\det(\mA)\left(\mA^{-1}\right)^t\left(\vn\pvect\vy\right)\), alors en développant les calculs, on retrouve bien le résultat de l'article de \cite{marceaux_high-order_2000}.\\


        En pratique, on néglige toute les dépendance en \(y\) ce qui revient à fixer \(k_y\) à \(0\). Grâce à cette hypothèse, on trouve que \(\mC\), et par conséquence \(\hat \mZ\), sont des matrices diagonales.

        Puisque la propagation n'a alors plus lieu que dans le plan \(xOz\), on peut décomposer ce problème par polarisation.

        C'est à dire que l'on suppose que le champ \(\vE\) (resp. \(\vH\)) est uniquement suivant \(\vect{e_y}\), on parle alors de polarisation Transverse Électrique (resp. Magnétique) ou abrégé TE (resp. TM).

        Dans ce cas, le champ \(\vE\)-TE correspond à \({E_y} \vect{e_y}\), le champ \(\vE\)-TM à \({E_x} \vect{e_x} + {E_z} \vect{e_z} \), tandis que le champ \(\vH\)-TM correspond à \({H_x} \vect{e_x} + {H_z} \vect{e_z}\) et le champ \(\vH\)-TE correspond à \({H_y} \vect{e_y}\).
        
        Alors le symbole \(\hat \mZ\) peut se réécrire comme
        \begin{equation}
            \hat \mZ =
            \begin{bmatrix}
                \hat Z_{TM} & 0
                \\
                0 & \hat Z_{TE}
            \end{bmatrix}
        \end{equation}


        La figure \ref{fig:imp_fourier:plan:hoppe} permet de vérifier les résultats de \cite[p.~33]{hoppe_impedance_1995} pour une couche de matériau sans perte (voir en annexe la figure \ref{fig:annex:hoppe:p33}). On applique les hypothèses précédentes, donc \(k_y=0\). On remarque que pour \(k_x\slash k_0=2\), \(k_3\) est nul et donc \(\mC\) n'est pas inversible. De plus comme le matériau et sans pertes, la partie réelle de \(\hat{\mZ}\) est nulle.

        \begin{figure}[!hbt]
            \centering
            \tikzsetnextfilename{Z_HOPPE_33_plan}
\begin{tikzpicture}[scale=1]
    \begin{axis}[
            title={},
            ylabel={\(\Im(\hat{\mathfrak{Z}}(k_x,0)\) (\(\Omega\))},
            xlabel={\(k_x\slash k_0\)},
            width=0.8\textwidth,
            xmin=0,
            xmax=2,
            mark repeat=20,
            legend pos=outer north east
        ]
        \addplot [black] table [col sep=comma, x={s1}, y={Im(z_ex.11)}] {csv/HOPPE_33/HOPPE_33.z_ex.MODE_2_TYPE_P.csv};
        \addlegendentry{TM};

        \addplot [black,dashed] table [col sep=comma, x={s1}, y={Im(z_ex.22)}] {csv/HOPPE_33/HOPPE_33.z_ex.MODE_2_TYPE_P.csv};
        \addlegendentry{TE};
    \end{axis}
\end{tikzpicture}
            \caption[Reproduction résultat Hoppe & Rahmat-Samii p.~33]{Partie imaginaire des coefficients diagonaux de \(\hat{\mZ}\) pour \(\eps = 4, \mu = 1, d=0.015\text{m}, f=1\text{GHz}\)}
            \label{fig:imp_fourier:plan:hoppe}
        \end{figure}

        La figure \ref{fig:imp_fourier:plan:soudais} permet de vérifier les résultats de \cite{soudais_3d_2017} pour une couche de matériau sans perte où \(k_3d = \frac{\pi}{2}\) pour \(k_x \simeq 0.9 k_0\).

        \begin{figure}[!hbt]
            \centering
            \tikzsetnextfilename{Z_ICEAA_11_plan_large}
\begin{tikzpicture}[scale=1]
    \begin{axis}[
            title={},
            width=0.4\textwidth,
            xmin=0,
            xmax=1.8,
            ylabel={\(\Im(\hat{Z}(k_x,0))\)},
            xlabel={\(k_x\slash k_0\)},
            mark repeat=20,
            legend pos=outer north east
        ]
        \addplot [black] table [x={s1}, y={Im(z_ex.tm)},col sep=comma] {csv/ICEAA_11/ICEAA_11.z_ex.P.csv};
        \addplot [black,dashed] table [x={s1}, y={Im(z_ex.te)},col sep=comma] {csv/ICEAA_11/ICEAA_11.z_ex.P.csv};
    \end{axis}
\end{tikzpicture}
\tikzsetnextfilename{Z_ICEAA_11_plan_zoom}
\begin{tikzpicture}[scale=1]
    \begin{axis}[
            title={},
            width=0.4\textwidth,
            ymin=-100,
            ymax=100,
            xmin=0.8,
            xmax=1,
            restrict y to domain=-200:200,                        
            ylabel={},
            xlabel={\(k_x\slash k_0\)},
            mark repeat=20,
            legend pos=outer north east
        ]
        \addplot [black] table [x={s1}, y={Im(z_ex.tm)},col sep=comma] {csv/ICEAA_11/ICEAA_11.z_ex.P.csv};
        \addlegendentry{TM};
        \addplot [black,dashed] table [x={s1}, y={Im(z_ex.te)},col sep=comma] {csv/ICEAA_11/ICEAA_11.z_ex.P.csv};
        \addlegendentry{TE};
    \end{axis}
\end{tikzpicture}
            \caption[Reproduction résultat P. Soudais p.~11]{Partie imaginaire des coefficients diagonaux de \(\hat\mZ\) pour \(\eps = 4, \mu = 1, d=0.035\text{m}, f=12\text{GHz}\)}
            \label{fig:imp_fourier:plan:soudais}
        \end{figure}

    %%%%%%%%%%%%%%%%%%%%%%%%%%%%%%%%%%%%%%%%%%%%%%%%%%%%%%%%%%%%%%%%%%%%%%%
    %%%%%%%%%%%%%%%%%%%%%%%%%%%%%%%%%%%%%%%%%%%%%%%%%%%%%%%%%%%%%%%%%%%%%%%
    %%%%%%%%%%%%%%%%%%%%%%%%%%%%%%%%%%%%%%%%%%%%%%%%%%%%%%%%%%%%%%%%%%%%%%%

    \subsection{Opérateur d'impédance pour plusieurs couches}
        On suppose que l'on a \(n\) couches de matériaux :

        \begin{figure}[h!btp]
            \centering
            \tikzsetnextfilename{plan_n_couches}            
            \begin{tikzpicture}
                \tikzmath{
    \largeur = 6;
    \hauteur = 0.5;
    \milieu = 1.3;
    \xC = \largeur;
    \xA = 0;
}

%% 1ere couche
\tikzmath{
    \yC = \hauteur;
    \yA = 0;
}

\coordinate (A) at (\xA,\yA);
\coordinate (B) at (\xA,\yC);
\coordinate (C) at (\xC,\yC);

\draw ($(B)!0.5!(C)$) node [above] {vide};


\fill [lightgray] (A) rectangle (C);
\draw ($(A)!0.5!(C)$) node {$\eps_n,\mu_n,d_n$};
\draw (B) -- (C) node [right] {$e_3 = 0$};

%% Des couches
\tikzmath{
    \yC = \yC - \hauteur;
    \yA = \yA - \milieu*\hauteur;
}

\coordinate (A) at (\xA,\yA);
\coordinate (B) at (\xA,\yC);
\coordinate (C) at (\xC,\yC);

\fill [lightgray]    (A) rectangle (C);
\fill [pattern=dots] (A) rectangle (C);
\draw (B) -- (C);

%% N ieme couche
\tikzmath{
    \yC = \yC - \milieu*\hauteur;
    \yA = \yA - \hauteur;
}

\coordinate (A) at (\xA,\yA);
\coordinate (B) at (\xA,\yC);
\coordinate (C) at (\xC,\yC);
\fill [lightgray] (A) rectangle (C);
\draw ($(A)!0.5!(C)$) node {$\eps_1,\mu_1,d_1$};
\draw (B) -- (C);

%% Le repère
\tikzmath{
    \xD = \xC + 0.5;
}

\coordinate (n) at (\xD,\yA);
\draw [->] (n) -- ++(1,0) node [at end, right] {$\v{e_1}$};
\draw [->] (n) -- ++(0,1) node [at end, right] {$\v{e_3}$};

\draw (n) circle(0.1cm) node [below=0.1cm] {$\v{e_2}$};
\draw (n) +(135:0.1cm) -- +(315:0.1cm);
\draw (n) +(45:0.1cm) -- +(225:0.1cm);

%% Le conducteur
\tikzmath{
    \yC = \yC - \hauteur;
    \yA = \yA - 0.5*\hauteur;
}

\coordinate (A) at (\xA,\yA);
\coordinate (B) at (\xA,\yC);
\coordinate (C) at (\xC,\yC);
\draw (B) -- (C);

\fill [pattern=north east lines] (A) rectangle (C);



            \end{tikzpicture}
        \end{figure}

        Pour chaque couche caractérisée par \((\eps_m,\mu_m,d_m)\), on définit:
        \begin{align}
        k_{3m} &= \sqrt{w^2\eps_m\mu_m - k_y^2 - k_x^2}
        \\
        \mC_m &=
            \begin{bmatrix}
                \left(\w\eps_m-\frac{k_y^2}{\w\mu_m}\right) & \frac{k_xk_y}{\w\mu_m}\\
                \frac{k_xk_y}{\w\mu_m} & \left(\w\eps_m-\frac{k_x^2}{\w\mu_m}\right)
            \end{bmatrix}
            \\
            &= \frac{1}{k_m\eta_m}\left(k_m^2 \mI - \hat{\mLR}\right)
        \end{align}

        On rappelle que \(\det{\mC_m} = \frac{k_{3m}^2}{\eta_m^2}\) et \(\mC_m^{-1}= \frac{k_m\eta_m}{k_mk_{3m}^2}\left(k_m^2 \mI + \hat{\mLD}\right)\).

        On définit aussi la profondeur de la couche \(m\), \(l_m = -\sum_{i=0}^{n-m} d_{n-i} \).

        \begin{defn}
            On définit pour chaque interface, le symbole \(\hat \mZ_m\) tel que
            \begin{equation}
                \hat \vE_t(k_x,k_y,l_m) = \hat \mZ_m(k_x,k_y) \left(\vect{e_z} \pvect \hat \vH_t(k_x,k_y,l_m)\right)
            \end{equation}
        \end{defn}

        \begin{thm}
            \label{thm:imp:fourier:plan:multi_couche}
            Soit \(\hat \mZ_0(k_x,k_y) = \mat{0}_{\mathcal{M}_2(\CC)}\).

            Si pour tout \(0<m < n\)
            \begin{align}
                k_{3m} &\not = 0 \\
                \det\left(k_{3m}\mI \pm \hat \mZ_{m-1}\mC_m \right) &\not = 0 \\
                k_{3m}d_m &\not = \frac{\pi}{2}+n\pi\,, \forall n \in \NN \\
                \det\left(k_{3m}\mI - i\tan(k_{3m}d_m)\mZ_{m-1}\mC_m\right) &\not = 0
            \end{align}
            Alors le symbole \(\hat \mZ_n\) est défini par la relation de récurrence :
            \begin{multline}
                \hat \mZ_m = k_{3m}
                \left(ik_{3m}\tan\left(k_{3m}d_m\right)\mI + \hat \mZ_{m-1}\mC_m\right) \\
                \left(k_{3m}\mI + i\tan\left(k_{3m}d_m\right)\hat \mZ_{m-1}\mC_m\right)^{-1}
                \mC_m^{-1}
            \end{multline}
        \end{thm}

        \begin{proof}
            À l'initialisation, la condition limite sur le conducteur impose \(\hat \mZ_0 = \mat{0}_{\mathcal{M}_2(\CC)}\) et on retrouve le résultat pour une couche.

            Par récurrence, un empilement à \(n\) couches se ramène à un empilement à une couche avec la condition:
            \begin{equation}
                \begin{bmatrix}
                    \hat{E_x}(k_x,k_y,-d)\\
                    \hat{E_y}(k_x,k_y,-d)\\
                \end{bmatrix}
                =
                \mZ_m
                \begin{bmatrix}
                    -\hat{H_y}(k_x,k_y,-d)\\
                    \hat{H_x}(k_x,k_y,-d)\\
                \end{bmatrix}
            \end{equation}

            On reprend donc tous les résultats de la partie précédente. Notamment, de \eqref{eq:imp_fourier:plan:champs:E} et \eqref{eq:imp_fourier:plan:champs:H}, on déduit que

            \begin{equation}
                \begin{bmatrix}
                    \hat{E_x}(k_x,k_y,-d)\\
                    \hat{E_y}(k_x,k_y,-d)\\
                \end{bmatrix}
                = ik_3\left( e^{-ik_3 d}
                \begin{bmatrix}
                    c_1 \\
                    c_2
                \end{bmatrix}
                -e^{ik_3 d}
                \begin{bmatrix}
                    c_3 \\
                    c_4
                \end{bmatrix}
                \right)
            \end{equation}

            \begin{equation}
                \begin{bmatrix}
                    -\hat{H_y}(k_x,k_y,-d)\\
                    \hat{H_x}(k_x,k_y,-d)\\
                \end{bmatrix}
                =i
                \mC
                \left(
                    e^{-ik_3 d}
                    \begin{bmatrix}
                        c_1 \\
                        c_2
                    \end{bmatrix}
                    +e^{ik_3 d}
                    \begin{bmatrix}
                        c_3 \\
                        c_4
                    \end{bmatrix}
                \right)
            \end{equation}

            \begin{equation}
                ik_3\left( e^{-ik_3 d}
                \begin{bmatrix}
                    c_1 \\
                    c_2
                \end{bmatrix}
                -e^{ik_3 d}
                \begin{bmatrix}
                    c_3 \\
                    c_4
                \end{bmatrix}
                \right)
                =i\hat\mZ_m\mC
                \left(
                    e^{-ik_3 d}
                    \begin{bmatrix}
                        c_1 \\
                        c_2
                    \end{bmatrix}
                    +e^{ik_3 d}
                    \begin{bmatrix}
                        c_3 \\
                        c_4
                    \end{bmatrix}
                \right)
            \end{equation}

            \begin{equation}
                \left(k_3\mI - \hat\mZ_m\mC\right)
                \begin{bmatrix}
                    c_1 \\
                    c_2
                \end{bmatrix}
                = e^{i2k_3 d} \left(k_3\mI + \hat\mZ_m\mC\right)
                \begin{bmatrix}
                    c_3 \\
                    c_4
                \end{bmatrix}
            \end{equation}

            On pose
            \begin{align}
                \mA_\pm &= k_3\mI \pm \hat\mZ_m\mC
            \end{align}

            On remarque que par définition, \(\mA_+\) et \(\mA_-\) commutent.

            Pour continuer il faut exprimer un vecteur en fonction de l'autre. On suppose donc \(\pm k_3\) ne sont pas des valeurs propres de \(\hat\mZ_m\mC\) et l'on déduit que

            \begin{align}
                \begin{bmatrix}
                    c_1 \\
                    c_2
                \end{bmatrix}
                &= e^{i2 k_3 d} \mA_-^{-1}\mA_+
                \begin{bmatrix}
                    c_3 \\
                    c_4
                \end{bmatrix}
                \\
                & = \mat{F}
                \begin{bmatrix}
                    c_3 \\
                    c_4
                \end{bmatrix}
            \end{align}

            \begin{align}
                \begin{bmatrix}
                    \hat{E_x}(k_x,k_y,0)\\
                    \hat{E_y}(k_x,k_y,0)\\
                \end{bmatrix}
                &=ik_3\left(\mat{F} - \mI \right)
                \begin{bmatrix}
                    c_3 \\
                    c_4
                \end{bmatrix}\\
                \begin{bmatrix}
                    -\hat{H_y}(k_x,k_y,0)\\
                    \hat{H_x}(k_x,k_y,0)\\
                \end{bmatrix}
                &=i\mC \left(\mat{F} + \mI \right)
                \begin{bmatrix}
                        c_3 \\
                        c_4
                \end{bmatrix}
            \end{align}

            On suppose qu'en plus de \(\mA_+\) et \(\mA_-\), \(\mat{F} + \mI\) est inversible, on va utiliser la commutativité de \(\mA_+\) et \(\mA_-\).

            Alors le symbole \(\hat \mZ\) s'exprime

            \begin{align}
                \hat{\mat{Z}}_{m+1}
                &=k_3\left(\mat{F} - \mI \right)\left(\mat{F}+ \mI \right)^{-1}\mC^{-1}
                \\
                &-k_3\mA_-^{-1}\left(e^{i2 k_3 d}\mA_+ - \mA_- \right)\left(e^{i2 k_3 d}\mA_+ + \mA_- \right)^{-1}\mA_-\mC^{-1}
                \\
                &=k_3\left( e^{i2 k_3 d} \mA_+ -  \mA_-\right)
                \left( e^{i2 k_3 d} \mA_+ + \mA_- \right)^{-1}\mC^{-1}
                \\
                &=k_3\left(k_3\left( e^{i2 k_3 d} - 1 \right)\mI + \left( e^{i2 k_3 d} + 1 \right) \hat\mZ_m\mC \right)
                \left( k_3\left( e^{i2 k_3 d} + 1 \right)\mI + \left( e^{i2 k_3 d} - 1 \right)\hat\mZ_m\mC \right)^{-1}\mC^{-1}
            \end{align}

            En supposant que \(\forall n \in \NN \,, k_3d\not = \frac{\pi}{2}+n\pi\), on a

            \begin{equation}
                \hat\mZ_{m+1} = k_3\left(ik_3\tan(k_3 d)\mI + \hat\mZ_m\mC \right)
                    \left( k_3\mI + i\tan(k_3 d)\hat\mZ_m\mC \right)^{-1}\mC^{-1}
            \end{equation}

            % à condition que
            % \begin{align}
            %     \det\left(k_3\mI \pm \hat\mZ_m\mC \right) \not = 0 \\
            %     k_3d\not = \frac{\pi}{2}+n\pi\,, \forall n \in \NN \\
            %     \det\left(k_3\mI - i\tan(k_3d)\hat\mZ_m\mC\right) \not = 0
            % \end{align}

        \end{proof}

        Si l'on veut uniquement exprimer le symbole avec les opérateurs \(\hat{\mLD}, \hat{\mLR}\), on obtient à la relation suivante

        \begin{multline}
            \hat \mZ_m = \frac{\eta_m}{k_mk_{3m}}
            \left(i\frac{k_{m}\eta_m}{k_{3m}}\tan\left(k_{3m}d_m\right)\left(k_m^2\mI+\hat{\mLD}\right) + \hat \mZ_{m-1}\right) \\
            \left(\frac{k_{m}\eta_m}{k_{3m}}\left(k_m^2\mI+\hat{\mLD}\right) + i\tan\left(k_{3m}d_m\right)\hat \mZ_{m-1}\right)^{-1}
            \left(k_m^2\mI+\hat{\mLD}\right)
        \end{multline}  

        Cette relation supprime deux produits de matrices et une inversion ce qui dans un code numérique peut s'avérer utile\footnote{Nous n'avons pas codé cette formule mais celle du théorème \ref{thm:imp:fourier:plan:multi_couche}}\footnote{Ce sont des matrices 2x2 donc la complexité numérique est faible et cet avantage est discutable.}.


  \subsubsection{Coefficients de réflexions}

    Dans le cas de la diffraction par un plan infini, une grandeur d’intérêt est le rapport entre l'onde réfléchie et l'onde incidente.

    On définit les vecteurs de \(\vect{C_1},\vect{C_2} \in \CC^2\) tels que
    \begin{align}
        \hat{\vE_t}(k_x,k_y,z) &= ik_{3,0}\left(e^{ik_{3,0}z}\vect{C_1} - e^{-ik_{3,0}z}\vect{C_2}\right)
        \\
        \left(\vect{e_z}\pvect\hat{\vH}\right)_t(k_x,k_y,z) &= i\mC_{m+1}\left(e^{ik_{3,0}z}\vect{C_1} + e^{-ik_{3,0}z}\vect{C_2}\right)
    \end{align}
 
    L'onde incidente se propageant depuis l'infini vers le plan, vu l'orientation du schéma et la dépendance en \(i\omega t\), l'onde incidente est la partie en \(e^{ik_3z}\) du champ total extérieur et donc l'onde réfléchie la partie en \(e^{-ik_3z}\).\footnote{Car \(\omega t + k z\) reste constant quand \(t\) augmente si \(z\) diminue proportionnellement.}

    \begin{TODO}
        TE mauvais signe. Demander Olivier
    \end{TODO}

    \begin{prop}
      On définit la matrice \(\mR\) dépendant de \((k_x,k_y)\) telle que pour tout point de la surface \(z=0_+\)
      \begin{equation}
        \vect{C_2}  = \mR \vect{C_1}
      \end{equation}
      alors
      \begin{equation}
          \mR = 
          \left(k_0^2\mI + \hat{\mLD} + k_0k_{3,0}\hat{\mZ}_m\right)^{-1}\left(k_0^2\mI + \hat{\mLD} - k_0k_{3,0}\hat{\mZ}_m\right)
      \end{equation}
    \end{prop}

    \begin{proof}
        On sait que 
        \begin{align}
            \hat{\vE_t}(k_x,k_y,0_+) &= ik_{3,0}\left(\vect{C_1}-\vect{C_2}\right)
            \\
            \left(\vect{e_z}\pvect \hat{\vH}\right)_t(k_x,k_y,0_+) &= i\mC_{m+1}\left(\vect{C_1}+\vect{C_2}\right)
        \end{align}
        et que les champs sont liés sur la surface de l'objet par la condition d'impédance
        \begin{align}
            \hat{\vE_t}(k_x,k_y,0_+) &= \hat{\mZ}_m \left(\vect{e_z}\pvect \hat{\vH}(k_x,k_y,0_+)\right)
            \\
            \intertext{donc en injectant les formules des champs}
            ik_{3,0}\left(\vect{C_1}-\vect{C_2}\right) &= i\hat{\mZ}_m\mC_{m+1}\left(\vect{C_1}+\vect{C_2}\right)
            \\
            \intertext{puis en séparant chaque vecteur}
            \left(k_{3,0}\mI - \hat{\mZ}_m\mC_{m+1} \right)\vect{C_1} & = \left(k_{3,0}\mI+\hat{\mZ}_m\mC_{m+1}\right)\vect{C_2}
        \end{align}
        et on termine en remplaçant \(\mC_{m+1}^{-1}\) par son expression avec \(\hat{\mLD}\) à l'extérieur (\(\eta_r = 1\)).
        \begin{equation}
            \mC_{m+1}^{-1} = \frac{1}{k_0k_{3,0}^2} \left(k_0^2 \mI + \hat{\mLD} \right)
        \end{equation}
        \begin{align}
            \mR &= \left(k_0^2\mI + \hat{\mLD} + k_0k_{3,0}\hat{\mZ}_m\right)^{-1}\left(k_0^2\mI + \hat{\mLD} - k_0k_{3,0}\hat{\mZ}_m\right)
        \end{align}
    \end{proof}

    Par construction et avec les mêmes hypothèses que pour \(\hat\mZ\) ( \(k_y = 0\) ), cette matrice est diagonale. On trace les termes diagonaux.

    Alors que le symbole a une asymptote pour \(k_x\slash k_0 \simeq 0.9\), la matrice de réflexion est parfaitement définie en ce point. De plus, on remarque que cet empilement permet d'obtenir une onde guidée pour \(k_x\slash k_0 \simeq 1.42\) car le coefficient TM diverge en ce point. C'est donc un cas très intéressant.

    \begin{figure}[!hbt]
        \centering
        \tikzsetnextfilename{R_ICEAA_11_plan}
\begin{tikzpicture}[scale=1]
    \begin{axis}[
            title={},
            width=0.8\textwidth,
            xmin=0,
            xmax=1.8,
            ymin=0,
            ymax=10,
            restrict y to domain=0:40,
            ylabel={\(|\hat{R}(k_x,0)|\)},
            xlabel={\(k_x\slash k_0\)},
            mark repeat=20,
            legend pos=outer north east
        ]
        \addplot [black] table [x={s1}, y={Abs(r_ex.tm)},col sep=comma] {csv/ICEAA_11/ICEAA_11.r_ex.P.csv};
        \addlegendentry{TM};
        \addplot [black,dashed] table [x={s1}, y={Abs(r_ex.te)},col sep=comma] {csv/ICEAA_11/ICEAA_11.r_ex.P.csv};
        \addlegendentry{TE};
    \end{axis}
\end{tikzpicture}
        \caption[Reproduction résultat P. Soudais p.~11]{Module des coefficients diagonaux de \(\mR\) pour \(\eps = 4, \mu = 1, d=0.035\text{m}, f=12\text{GHz}\)}
        \label{fig:reflex_fourier:plan:soudais}
    \end{figure}

    %%%%%%%%%%%%%%%%%%%%%%%%%%%%%%%%%%%%%%%%%%%%%%%%%%%%%%%%%%%%%%%%%%%%%%%
    %%%%%%%%%%%%%%%%%%%%%%%%%%%%%%%%%%%%%%%%%%%%%%%%%%%%%%%%%%%%%%%%%%%%%%%
    %%%%%%%%%%%%%%%%%%%%%%%%%%%%%%%%%%%%%%%%%%%%%%%%%%%%%%%%%%%%%%%%%%%%%%%
