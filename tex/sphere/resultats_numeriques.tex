\section{Résultats numériques}

  La figure \ref{fig:imp_fourier:sphere:hoppe:62:hoibc:mode_2} trace les valeurs des termes diagonaux de la matrice \(\hat\mZ\) quand les coefficients sont calculés en minimisant \(J_Z\) sans contraintes.
  \begin{figure}[!hbt]
    \centering
    \tikzsetnextfilename{Z_HOPPE_62_sphere_hoibc_mode_2.TM}
\begin{tikzpicture}[scale=1]
    \begin{axis}[
            title={Polarisation TM},
            ylabel={\(\Im(\hat{Z}(n,0)\)},
            xlabel={\(k_t\slash k_0\)},
            width=0.4\textwidth,
            xmin=0,
            xmax=1.5,
            mark repeat=1,
            legend pos=outer north east
        ]
        \addplot [black,mark=square*] table [col sep=comma, x={s2}, y={Im(z_ex.22)}] {csv/HOPPE_62/HOPPE_62.z_ex.MODE_2_TYPE_S_+3.000E-02m.csv};

        \addplot [blue,mark=x] table [col sep=comma, x={s2}, y={Im(z_ibc0.22)}] {csv/HOPPE_62/HOPPE_62.z_ibc.IBC_ibc0_SUC_F_MODE_2_TYPE_S_+3.000E-02m.csv};

        \addplot [red,mark=diamond*] table [col sep=comma, x={s2}, y={Im(z_ibc3.22)}] {csv/HOPPE_62/HOPPE_62.z_ibc.IBC_ibc3_SUC_F_MODE_2_TYPE_S_+3.000E-02m.csv};
    \end{axis}
\end{tikzpicture}
\tikzsetnextfilename{Z_HOPPE_62_sphere_hoibc_mode_2.TE}
\begin{tikzpicture}[scale=1]
    \begin{axis}[
            title={Polarisation TE},
            ylabel={},
            xlabel={\(k_t\slash k_0\)},
            width=0.4\textwidth,
            xmin=0,
            xmax=1.5,
            mark repeat=1,
            legend pos=outer north east
        ]
        \addplot [black,mark=square*] table [col sep=comma, x={s2}, y={Im(z_ex.11)}] {csv/HOPPE_62/HOPPE_62.z_ex.MODE_2_TYPE_S_+3.000E-02m.csv};
        \addlegendentry{Exact};

        \addplot [blue,mark=x] table [col sep=comma, x={s2}, y={Im(z_ibc0.11)}] {csv/HOPPE_62/HOPPE_62.z_ibc.IBC_ibc0_SUC_F_MODE_2_TYPE_S_+3.000E-02m.csv};
        \addlegendentry{CI0};

        \addplot [red,mark=diamond*] table [col sep=comma, x={s2}, y={Im(z_ibc3.11)}] {csv/HOPPE_62/HOPPE_62.z_ibc.IBC_ibc3_SUC_F_MODE_2_TYPE_S_+3.000E-02m.csv};
        \addlegendentry{CI3};
    \end{axis}
\end{tikzpicture}
    \caption[CIOE sur empilement de Hoppe & Rahmat-Samii p.~62]{Minimisation de \(J_Z\): Partie imaginaire des termes diagonaux des matrices d'impédance pour l'empilement \(\eps = 6\), \(\mu = 1\), \(d=0.0225\text{m}\), \(f=1\text{GHz}\), \(r_0=0.03\text{m}\) en fonction de \(k_t = n / (r_0+d)\).}
    \label{fig:imp_fourier:sphere:hoppe:62:hoibc:mode_2}
  \end{figure}
  On remarque alors que la CI3 semble être une bonne approximation. 

  \begin{table}[!hbt]
    \centering
    % On fait deux tables de même hauteur
    \begin{minipage}[t]{0.49\textwidth}
      \vspace{0pt}
      \centering
      \begin{coefftable}{\hyperlink{ci0}{CI0}}
        \input{csv/HOPPE_62/HOPPE_62.IBC_ibc0_SUC_F_MODE_2_TYPE_S_+3.000E-02m.coeff.txt}
      \end{coefftable}
    \end{minipage}
    \begin{minipage}[t]{0.49\textwidth}
      \vspace{0pt}
      \centering
      \begin{coefftable}{\hyperlink{ci3}{CI3}}
        \input{csv/HOPPE_62/HOPPE_62.IBC_ibc3_SUC_F_MODE_2_TYPE_S_+3.000E-02m.coeff.txt}
      \end{coefftable}
    \end{minipage}
    \caption{Coefficients associés à la figure \ref{fig:imp_fourier:sphere:hoppe:62:hoibc:mode_2}}
    \label{tab:imp_fourier:sphere:hoppe:62:hoibc:mode_2}
  \end{table}

  La figure \ref{fig:imp_fourier:sphere:hoppe:62:hoibc:mode_1} minimise \(J_R\) sans contraintes.
  \begin{figure}[!hbt]
    \centering
    \tikzsetnextfilename{Z_HOPPE_62_sphere_hoibc_mode_1.TM}
\begin{tikzpicture}[scale=1]
    \begin{axis}[
            title={Polarisation TM},
            ylabel={\(\Im(\hat{Z}(n,0))\)},
            xlabel={\(k_t\slash k_0\)},
            width=0.4\textwidth,
            xmin=0,
            xmax=1.5,
            mark repeat=1,
            legend pos=outer north east
        ]
        \addplot [black,mark=square*] table [col sep=comma, x={s2}, y={Im(z_ex.22)}] {csv/HOPPE_62/HOPPE_62.z_ex.MODE_1_TYPE_S_+3.000E-02m.csv};

        \addplot [\ccio,mark=x] table [col sep=comma, x={s2}, y={Im(z_ibc0.22)}] {csv/HOPPE_62/HOPPE_62.z_ibc.IBC_ibc0_SUC_F_MODE_1_TYPE_S_+3.000E-02m.csv};

        \addplot [\ccit,mark=diamond*] table [col sep=comma, x={s2}, y={Im(z_ibc3.22)}] {csv/HOPPE_62/HOPPE_62.z_ibc.IBC_ibc3_SUC_F_MODE_1_TYPE_S_+3.000E-02m.csv};
    \end{axis}
\end{tikzpicture}
\tikzsetnextfilename{Z_HOPPE_62_sphere_hoibc_mode_1.TE}
\begin{tikzpicture}[scale=1]
    \begin{axis}[
            title={Polarisation TE},
            ylabel={},
            xlabel={\(k_t\slash k_0\)},
            width=0.4\textwidth,
            xmin=0,
            xmax=1.5,
            mark repeat=1,
            legend pos=outer north east
        ]
        \addplot [black,mark=square*] table [col sep=comma, x={s2}, y={Im(z_ex.11)}] {csv/HOPPE_62/HOPPE_62.z_ex.MODE_1_TYPE_S_+3.000E-02m.csv};
        \addlegendentry{Exact};

        \addplot [\ccio,mark=x] table [col sep=comma, x={s2}, y={Im(z_ibc0.11)}] {csv/HOPPE_62/HOPPE_62.z_ibc.IBC_ibc0_SUC_F_MODE_1_TYPE_S_+3.000E-02m.csv};
        \addlegendentry{CI0};

        \addplot [\ccit,mark=diamond*] table [col sep=comma, x={s2}, y={Im(z_ibc3.11)}] {csv/HOPPE_62/HOPPE_62.z_ibc.IBC_ibc3_SUC_F_MODE_1_TYPE_S_+3.000E-02m.csv};
        \addlegendentry{CI3};
    \end{axis}
\end{tikzpicture}
    \caption[CIOE sur empilement de Hoppe & Rahmat-Samii p.~62]{Minimisation de \(J_R\): Partie imaginaire des termes diagonaux des matrices d'impédance pour l'empilement \(\eps = 6\), \(\mu = 1\), \(d=0.0225\text{m}\), \(f=1\text{GHz}\), \(r_0=0.03\text{m}\) en fonction de \(k_t = n / (r_0+d)\).}
    \label{fig:imp_fourier:sphere:hoppe:62:hoibc:mode_1}
  \end{figure}
  Nous ne savons pas expliquer pourquoi la valeur calculée pour la condition de Leontovich est si différente.

  \begin{table}[!hbt]
    \centering
    % On fait deux tables de même hauteur
    \begin{minipage}[t]{0.49\textwidth}
      \vspace{0pt}
      \centering
      \begin{coefftable}{\hyperlink{ci0}{CI0}}
        \input{csv/HOPPE_62/HOPPE_62.IBC_ibc0_SUC_F_MODE_1_TYPE_S_+3.000E-02m.coeff.txt}
      \end{coefftable}
    \end{minipage}
    \begin{minipage}[t]{0.49\textwidth}
      \vspace{0pt}
      \centering
      \begin{coefftable}{\hyperlink{ci3}{CI3}}
        \input{csv/HOPPE_62/HOPPE_62.IBC_ibc3_SUC_F_MODE_1_TYPE_S_+3.000E-02m.coeff.txt}
      \end{coefftable}
    \end{minipage}
    \caption{Coefficients associés à la figure \ref{fig:imp_fourier:sphere:hoppe:62:hoibc:mode_1}}
    \label{tab:imp_fourier:sphere:hoppe:62:hoibc:mode_1}
  \end{table}
