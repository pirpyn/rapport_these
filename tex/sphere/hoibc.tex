\section{Approximation de la matrice d'impédance pour une sphère par une CIOE}

  \subsection[Expression des opérateurs LD,LR en Fourier]{Expression des opérateurs \(\LD,\LR\) en Fourier}

    Par définition de \(\LD\), on a
    \begin{align}
      \LD \vE_t & = \vgrads{} \vdivs{} \vE_t
    \end{align}
    Or 
    \begin{align*}
      \vE_t(\rtp) &= \sum_{n\in\ZZ}\sum_{m\in\ZZ} (a_{mn} j_n + c_{mn}h_n) \Umn(\rtp) + (b_{mn} \tilde{j_n}+d_{mn} \tilde{h_n}) \Umn^\perp(\rtp)
      %\intertext{donc}
      %\LD\vE_t(\rtp) &= \sum_{n\in\ZZ}\sum_{m\in\ZZ} (a_{mn} j_n + c_{mn}h_n) \LD\Umn(\rtp) + (b_{mn} \tilde{j_n}+d_{mn} \tilde{h_n}) \LD\Umn^\perp(\rtp)
    \end{align*}

    On rappelle  les expressions des vecteurs (voir \eqref{eq:defUmn_tgt}, \eqref{eq:defNmn_tgt}) dans la base sphérique (\(\vect{e_r},\vect{e_\theta},\vect{e_\phi}\))
    \begin{align*}
      \Umn(\tp) =
      \begin{bmatrix}
          0
          \\
          \frac{im}{\sin\theta}\Pmn(\cos(\theta))e^{im\phi}
          \\
          - \ddr{\theta}{\Pmn(\cos(\theta))}e^{im\phi}
      \end{bmatrix}
      &&
      \Umn^\perp(\tp) =
      \begin{bmatrix}
        0
        \\
        \ddr{\theta}{\Pmn(\cos(\theta))}e^{im\phi}
        \\
        \frac{im}{\sin\theta}\Pmn(\cos(\theta))e^{im\phi}
      \end{bmatrix}
    \end{align*}

    On commence par calculer le divergent surfacique (cf annexe \ref{sec:annexe:div_grad_rot}) en sphérique
    \begin{align*}
      \vdivs{}\Umn(\rtp) &= \frac{1}{r\sin(\theta)} \ddr{\theta}{(\sin(\theta)\Umn_\theta)} + \frac{1}{r\sin(\theta)}\ddr{\phi}{(\Umn_\phi)}
      \\
      &=\frac{ime^{im\phi}}{r\sin(\theta)}\left( \ddr{\theta}{\Pmn(\cos(\theta))} - \ddr{\theta}{\Pmn(\cos(\theta))} \right)
      \\
      &= 0
    \end{align*}
    Donc 
    \begin{align*}
      \LD\Umn(\rtp) = 0
    \end{align*}

    Calculons maintenant l'action de \(\LD\) sur \(\Umn^\perp\)
    \begin{align*}
      \vdivs{}\Umn^\perp(\rtp) &= \frac{1}{r\sin(\theta)} \ddr{\theta}{(\sin(\theta)\Umn^\perp_\theta)} + \frac{1}{r\sin(\theta)}\ddr{\phi}{(\Umn^\perp_\phi)}
      \\
      &= \frac{e^{im\phi}}{r\sin(\theta)}
      \left(
        \ddr{\theta}{}\left(\sin(\theta)\ddr{\theta}{\Pmn(\cos(\theta))}\right) - \frac{m^2}{\sin\theta}\Pmn(\cos(\theta))
      \right)
    \end{align*}
    Or d’après \cite[\href{https://dlmf.nist.gov/14.10}{sec.~14.10}]{dlmf_nist_2019} les fonctions de Legendre sont récurrentes
    \begin{align}
      \sin(\theta)\ddr{\theta}{\Pmn(\cos(\theta))} &= (n+m)\PP_{n-1}^m(\cos(\theta)) - n\cos(\theta)\Pmn(\cos(\theta))
    \end{align}

    Donc 
    \begin{align*}
      \LD\Umn^\perp(\rtp) = -\frac{n(n+1)}{r^2}\Umn^\perp(\rtp)
    \end{align*}

    On utilise les résultats de \cite{marceaux_high-order_2000}

    \begin{align*}
      \vgrads{}\vdivs{} \Mmn[z_n]_t(r,\theta,\phi) &= 0
      \\
      \vgrads{}\vdivs{} \Nmn[z_n]_t(r,\theta,\phi) &= -\frac{n(n+1)}{r^2}\Nmn[z_n]_t(r,\theta,\phi)
    \end{align*}

    On définit \(\hat{\mLD}\) l'opérateur matriciel tel que
    \begin{align}
      \LD \vE_t (r_{ext},\theta,\phi)
      &= \frac{1}{2\pi}\sum_{n=-\infty}^\infty\sum_{m=-n}^n \hat{\mLD} \hat{\vE_t}(r_{ext},n,m)
    \end{align}

    Son expression est de ce qui précède
    \begin{equation}
      \label{eq:cylindre:fourier:LD}
      \hat{\mLD}(n,m) = -
      \begin{bmatrix}
        0 & 0
        \\
        0 & \frac{n(n+1)}{r_{ext}^2}
      \end{bmatrix}
    \end{equation}

    On reprend exactement la même méthode pour l'opérateur \(\LR\).
    Par définition de \(\LR\), on a
    \begin{align}
      \LR \vE_t & = \vrots{} \vrots{} \vE_t
    \end{align}

    On utilise les résultats de \cite{marceaux_high-order_2000}

    \begin{align*}
      \vrots{}\vrots{} \Mmn[z_n]_t(r,\theta,\phi) &= \frac{n(n+1)}{r^2}\Mmn[z_n]_t(r,\theta,\phi)
      \\
      \vrots{}\vrots{} \Nmn[z_n]_t(r,\theta,\phi) &= 0
    \end{align*}

    On définit \(\hat{\LR}\) l'opérateur matriciel tel que
    \begin{align}
      \LR \vE_t(r_{ext},\theta,\phi)
      &= \frac{1}{2\pi}\sum_{n=-\infty}^\infty\sum_{m=-n}^n \hat{\LR} \hat{\vE_t}(r_{ext},n,m)
    \end{align}

    \begin{equation}
      \hat{\mLR}(n,m) =
      \begin{bmatrix}
        \frac{n(n+1)}{r_{ext}^2} & 0
        \\
        0 & 0
      \end{bmatrix}
    \end{equation}

  \subsection{Expression de la matrice d'impédance approchée par la CI3}

    Tout comme dans le cas du plan infini, on peut donc définir \(\hat{\mZ}_{IBC}\) l’opérateur matriciel associé à la condition d'impédance.

    \begin{multline}
        \hat{\mZ}_{CI3}(n,m) = \left(\mI + b_1 \frac{\hat{\mLD}(n,m)}{k_0^2} - b_2 \frac{\hat{\mLR}(n,m)}{k_0^2} \right)^{-1}\\
        \left(a_0 \mI + a_1 \frac{\hat{\mLD}(n,m)}{k_0^2} - a_2 \frac{\hat{\mLR}(n,m)}{k_0^2}\right)
    \end{multline}
