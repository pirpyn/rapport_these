\section[Approximation par une CIOE de l'opérateur de Calderón de la sphère]{Approximation de la matrice d'impédance pour une sphère par une CIOE}

  \subsection[Expression des opérateurs LD,LR en Fourier]{Expression des opérateurs \(\LD,\LR\) en Fourier}


Soit \(S(0,r_s)\) une sphère de centre 0, de rayon \(r_s\) et \((r,\theta,\phi)\) les coordonnées sphériques d'un point de l'espace.

    Soit \(V = \left(\mathcal{C}^\infty(S(0,r_s))\right)^2 \cap L^2(S(0,r_s)))\) l'espace des fonctions infiniment dérivables définies sur cette sphère et de carré intégrable.

    \begin{defn}
      \label{eq:sphere:fourier:LD}
      On définit \(\LD\) l'opérateur de \(V\) tel que
      \begin{align*}
        \LD \vect{U}(r_s,\theta,\phi) & = \vgrads{} \vdivs{} \vect{U}(r_s,\theta,\phi).
      \end{align*}

      On définit \(\hat{\mLD}\) la fonction de \(\NN \rightarrow \mathcal{M}_2(\RR)\) telle que
      \begin{equation*}
        \hat{\mLD}(n) = -
        \begin{bmatrix}
          0 & 0
          \\
          0 & \frac{n(n+1)}{r_s^2}
        \end{bmatrix}.
      \end{equation*}
    \end{defn}

    \begin{prop}
      Soit \(\vE_t\) dans \(V\) donc il existe \(a_{nm},b_{nm},c_{nm},d_{nm}\) tels que
      \begin{multline*}
      \vE_t(r_s,\theta,\phi) =
      \\
        \sum_{n\in\ZZ}\sum_{m\in\ZZ}
            \begin{bmatrix}
              \Umn(\tp) & \Umn^\perp(\tp)
            \end{bmatrix}
            \left(
              \mj_E(r_s,n)
              \begin{bmatrix}
                  a_{mn}
                  \\
                  b_{mn}
              \end{bmatrix}
              +
              \mh_E(r_s,n)
              \begin{bmatrix}
                  c_{mn}
                  \\
                  d_{mn}
              \end{bmatrix}
            \right),
      \end{multline*}
      alors
      \begin{multline*}
        \LD\vE_t(r_s,\theta,\phi) =
        \\
          \sum_{n\in\ZZ}\sum_{m\in\ZZ}
            \begin{bmatrix}
              \Umn(\tp) & \Umn^\perp(\tp)
            \end{bmatrix}
            \hat\mLD(n)
            \left(
              \mj_E(r_s,n)
              \begin{bmatrix}
                  a_{mn}
                  \\
                  b_{mn}
              \end{bmatrix}
              +
              \mh_E(r_s,n)
              \begin{bmatrix}
                  c_{mn}
                  \\
                  d_{mn}
              \end{bmatrix}
            \right).
      \end{multline*}
    \end{prop}

    \begin{proof}
      Par définition de \(\LD\), on a
      \begin{align*}
        \LD \vE_t & = \vgrads{} \vdivs{} \vE_t.
      \end{align*}
      De plus, 
      \begin{multline*}
        \vE_t(r_s,\theta,\phi) = \sum_{n\in\ZZ}\sum_{m\in\ZZ} \left( (a_{mn} j_n(kr_s) + c_{mn}h_n(kr_s)) \Umn(\tp) 
        \right.
        \\
        \left.
        + (b_{mn} \tilde{j_n}(kr_s)+d_{mn} \tilde{h_n}(kr_s)) \Umn^\perp(\tp)\right)
        %\intertext{donc}
        %\LD\vE_t(r_s,\theta,\phi) &= \sum_{n\in\ZZ}\sum_{m\in\ZZ} (a_{mn} j_n + c_{mn}h_n) \LD\Umn(r_s,\theta,\phi) + (b_{mn} \tilde{j_n}+d_{mn} \tilde{h_n}) \LD\Umn^\perp(r_s,\theta,\phi)
        .
      \end{multline*}

      On rappelle  les expressions des vecteurs (voir \eqref{eq:defUmn_tgt}, \eqref{eq:defNmn_tgt}) dans la base sphérique (\(\vect{e_r},\vect{e_\theta},\vect{e_\phi}\)): 
      \begin{align*}
        \Umn(\tp) =
        \begin{bmatrix}
            0
            \\
            \frac{im}{\sin\theta}\Pmn(\cos(\theta))e^{im\phi}
            \\
            - \ddr{\theta}{\Pmn(\cos(\theta))}e^{im\phi}
        \end{bmatrix},
        &&
        \Umn^\perp(\tp) =
        \begin{bmatrix}
          0
          \\
          \ddr{\theta}{\Pmn(\cos(\theta))}e^{im\phi}
          \\
          \frac{im}{\sin\theta}\Pmn(\cos(\theta))e^{im\phi}
        \end{bmatrix}.
      \end{align*}

      On commence par calculer le divergent surfacique (cf annexe \ref{sec:annexe:div_grad_rot}) en sphérique.
      \begin{align*}
        \vdivs{}\Umn(r_s,\theta,\phi) &= \frac{1}{r_s\sin(\theta)} \ddr{\theta}{(\sin(\theta)\Umn_\theta)} + \frac{1}{r_s\sin(\theta)}\ddr{\phi}{(\Umn_\phi)},
        \\
        &=\frac{ime^{im\phi}}{r_s\sin(\theta)}\left( \ddr{\theta}{\Pmn(\cos(\theta))} - \ddr{\theta}{\Pmn(\cos(\theta))} \right),
        \\
        &= 0.
      \end{align*}
      Donc 
      \begin{align*}
        \LD\Umn(r_s,\theta,\phi) = 0.
      \end{align*}

      Calculons maintenant l'action de \(\LD\) sur \(\Umn^\perp\).
      \begin{align*}
        \vdivs{}\Umn^\perp(r_s,\theta,\phi) &= \frac{1}{r_s\sin(\theta)} \ddr{\theta}{(\sin(\theta)\Umn^\perp_\theta)} + \frac{1}{r_s\sin(\theta)}\ddr{\phi}{(\Umn^\perp_\phi)},
        \\
        &= \frac{e^{im\phi}}{r_s\sin(\theta)}
        \left(
          \ddr{\theta}{}\left(\sin(\theta)\ddr{\theta}{\Pmn(\cos(\theta))}\right) - \frac{m^2}{\sin\theta}\Pmn(\cos(\theta))
        \right),
      \end{align*}
      or les fonctions de Legendre sont solutions de l'équation différentielle éponyme, donc
      \begin{align*}
        \frac{1}{\sin(\theta)}\ddr{\theta}{}\left(\sin(\theta)\ddr{\theta}{\Pmn(\cos(\theta))}\right) + \left(n(n+1)-\frac{m^2}{\sin(\theta)^2}\right)\Pmn(\cos(\theta) = 0.
      \end{align*}

      Nous déduisons 
      \begin{align*}
         \vdivs{}\Umn^\perp(r_s,\theta,\phi) = -\frac{n(n+1)}{r_s}\Pmn(\cos(\theta))e^{im\phi}.
      \end{align*}
      Enfin d’après la définition du gradient surfacique en coordonnées sphériques,
      \begin{align*}
         \vgrads{}\vdivs{}\Umn^\perp(r_s,\theta,\phi) &= -\frac{n(n+1)}{r_s}\left(\frac{1}{r_s}\ddr{\theta}{\Pmn(\cos(\theta))}e^{im\phi}\vect{e_\theta}+\frac{1}{r_s\sin(\theta)}\ddr{\phi}{e^{im\phi}}\Pmn(\cos(\theta))\vect{e_\phi}\right),
         \\
         &=-\frac{n(n+1)}{r_s^2}\Umn^\perp(r_s,\theta,\phi).
      \end{align*}
    \end{proof}


    \begin{defn}
      \label{eq:sphere:fourier:LR}
      On définit \(\LR\) l'opérateur de \(V\) tel que
      \begin{align*}
        \LR \vect{U}(r_s,\theta,\phi) & = \vrots{} \vrots{} \vect{U}(r_s,\theta,\phi).
      \end{align*}

      On définit \(\hat{\mLR}\) la fonction de \(\NN \rightarrow \mathcal{M}_2(\RR)\) telle que
      \begin{equation*}
        \hat{\mLR}(n) =
        \begin{bmatrix}
          \frac{n(n+1)}{r_s^2} & 0
          \\
          0 & 0
        \end{bmatrix}.
      \end{equation*}
    \end{defn}

    \begin{prop}
      Soit \(\vE_t\) dans \(V\). Il existe \((a_{nm},b_{nm},c_{nm},d_{nm})\) tels que
      \begin{align*}
      \vE_t(r_s,\theta,\phi) &= \sum_{n\in\ZZ}\sum_{m\in\ZZ}
            \begin{bmatrix}
              \Umn(\tp) & \Umn^\perp(\tp)
            \end{bmatrix}
            \left(
              \mj_E(r_s,n)
              \begin{bmatrix}
                  a_{mn}
                  \\
                  b_{mn}
              \end{bmatrix}
              +
              \mh_E(r_s,n)
              \begin{bmatrix}
                  c_{mn}
                  \\
                  d_{mn}
              \end{bmatrix}
            \right),
      \end{align*}
      alors
      \begin{multline*}
        \LR\vE_t(r_s,\theta,\phi) = 
        \\
         \sum_{n\in\ZZ}\sum_{m\in\ZZ}
            \begin{bmatrix}
              \Umn(\tp) & \Umn^\perp(\tp)
            \end{bmatrix}
            \hat\mLR(n)
            \left(
              \mj_E(r_s,n)
              \begin{bmatrix}
                  a_{mn}
                  \\
                  b_{mn}
              \end{bmatrix}
              +
              \mh_E(r_s,n)
              \begin{bmatrix}
                  c_{mn}
                  \\
                  d_{mn}
              \end{bmatrix}
            \right).
      \end{multline*}
    \end{prop}

    \begin{proof}
      On reprend exactement la même méthode que pour \(\LD\).
    \end{proof}

  \subsection{Expression de la matrice d'impédance approchée par la CI3}

    Tout comme dans le cas du plan infini, on peut donc définir \(\hat{\mZ}_{IBC}\) l’opérateur matriciel associé à la condition d'impédance:
    \begin{align}
        \label{eq:sphere:hoibc:ci3}
        \hat{\mZ}_{CI3}(n) = \left(\mI + b_1 \frac{\hat{\mLD}(n)}{k_0^2} - b_2 \frac{\hat{\mLR}(n)}{k_0^2} \right)^{-1}
        \left(a_0 \mI + a_1 \frac{\hat{\mLD}(n)}{k_0^2} - a_2 \frac{\hat{\mLR}(n)}{k_0^2}\right).
    \end{align}
