\section{Calcul des coefficients de la CI3 par moindres carrés sur l'impédance}

  \subsection{Expression de la fonctionnelle}

    \begin{defn}
      On définit \(\mH_{CI3}\) la fonction de \(\NN \times \RR \times \mathcal{M}_2(\CC) \rightarrow \mathcal{M}_{4\times5}(\CC)\) telle que
      \begin{equation*}
        \mH_{CI3}(n,\mZ) = \begin{bmatrix}
        1 & \hat{\mLD}(n)_{11} & -\hat{\mLR}(n)_{11} & -\left(\hat{\mLD}(n){\mZ}\right)_{11} & \left(\hat{\mLR}(n){\mZ}\right)_{11}
        \\
        0 & \hat{\mLD}(n)_{12} & -\hat{\mLR}(n)_{12} & -\left(\hat{\mLD}(n){\mZ}\right)_{12} & \left(\hat{\mLR}(n){\mZ}\right)_{12}
        \\
        0 & \hat{\mLD}(n)_{21} & -\hat{\mLR}(n)_{21} & -\left(\hat{\mLD}(n){\mZ}\right)_{21} & \left(\hat{\mLR}(n){\mZ}\right)_{21}
        \\
        1 & \hat{\mLD}(n)_{22} & -\hat{\mLR}(n)_{22} & -\left(\hat{\mLD}(n){\mZ}\right)_{22} & \left(\hat{\mLR}(n){\mZ}\right)_{22}
        \end{bmatrix}
      \end{equation*}
      On définit \(b\) la fonction de \(\mathcal{M}_2(\CC) \rightarrow \mathcal{M}_{4\times1}(\CC)\) telle que
      \begin{equation*}
        b(\mZ) = \begin{bmatrix}
        {\mZ}_{11}
        \\
        {\mZ}_{12}
        \\
        {\mZ}_{21}
        \\
        {\mZ}_{22}
        \end{bmatrix}
      \end{equation*}
    \end{defn}

    \begin{prop}
      Soit \(X = (a_0,a_1,a_2,b_1,b_2)\), \(n\) fixé et \(\hat\mZ_{ex}\) l'opérateur d'impédance exact du cylindre, alors
      \begin{equation*}
        \argmin{X\in\CC^5} \norm{\hat\mZ_{CI3}(n,X) - \hat\mZ_{ex}(n)} = \argmin{X\in\CC^5} \norm{\mH_{CI3}(n,\hat\mZ_{ex}(n))X - b(\hat\mZ_{ex}(n))}^2
      \end{equation*}
    \end{prop}

    \begin{proof}
      C'est la même méthodologie que pour le plan.
      On rappelle que dans la section précédente, on a introduit
      \begin{multline*}
        \hat{\mZ}_{CI3}(n) = \left(\mI + b_1 \hat{\mLD}(n) - b_2 \hat{\mLR}(n) \right)^{-1}\\\left(a_0 \mI + a_1 {\hat{\mLD}(n)} - a_2 {\hat{\mLR}(n)}\right)
      \end{multline*}
      On pose \(\hat\mZ_D(n) = \mI + b_1 \hat{\mLD}(n) - b_2 \hat{\mLR}(n)\) et \(\hat\mZ_N(n) = a_0 \mI + a_1 {\hat{\mLD}(n)} - a_2 {\hat{\mLR}(n)}\) donc

      \begin{align*}
      &{}~ \argmin{X\in\CC^5} \norm{\hat\mZ_{CI3}(n,X) - \hat\mZ_{ex}(n)}
      \\
      & = \argmin{X\in\CC^5} \norm{\hat\mZ_D(n)^{-1}\hat\mZ_N(n) - \hat\mZ_{ex}(n) }
      \\
      &= \argmin{X\in\CC^5} \norm{\hat\mZ_D(n)^{-1}\left(\hat\mZ_N(n) - \hat\mZ_D(n)\hat\mZ_{ex}(n)\right) }
      \\
      &= \argmin{X\in\CC^5} \norm{\hat\mZ_N(n) - \hat\mZ_D(n)\hat\mZ_{ex}(n)}
      \\
      &= \argmin{X\in\CC^5} \norm{\hat\mZ_N(n) - \left(b_1 \hat{\mLD}(n) - b_2 \hat{\mLR}(n)\right)\hat\mZ_{ex}(n) - \hat\mZ_{ex}(n) }
      \\
      &= \argmin{X\in\CC^5} \norm{\mH_{CI3}(n,\hat\mZ_{ex}(n))X - b(\hat\mZ_{ex}(n))}
      \end{align*}
    \end{proof}

    On tronque la série de Mie à \(N_{n}\) termes.
    \begin{defn}
      On définit \(\mA_{CI3}\) la matrice de \(\mathcal{M}_{4N_{n}\times5}(\CC)\) telle que
      \begin{equation*}
        \mA_{CI3} = 
        \begin{bmatrix}
          \mH_{CI3}(n_1,\hat\mZ_{ex}(n_1))
          \\
          \vdots
          \\
          \mH_{CI3}(n_i,\hat\mZ_{ex}(n_i))
          \\
          \vdots
          \\
          \mH_{CI3}(n_{N_n},k_{z},\hat\mZ_{ex}(n_{N_n},k_{z}))
        \end{bmatrix}
      \end{equation*}
      On définit \(g\) la matrice de \(\mathcal{M}_{4N_{n}\times1}(\CC)\) telle que
      \begin{equation*}
        g = 
        \begin{bmatrix}
          b(\hat\mZ_{ex}(n_1))
          \\
          \vdots
          \\
          b(\hat\mZ_{ex}(n_i))
          \\
          \vdots
          \\
          b(\hat\mZ_{ex}(n_{N_n},k_{z}))
        \end{bmatrix}
      \end{equation*}
    \end{defn}

    On peut alors calculer les coefficients de la CI3
    \begin{defn}
      On définit la fonctionnelle \(J_Z\)
      \begin{equation*}
        J_Z(X) = \norm{{\mA}_{CI3}X - {g}}^2
      \end{equation*}
    \end{defn}
    \begin{thm}[Minimisation sans contraintes pour la CI3]

      Les coefficients de la CIOE sont solutions du problème

      Trouver \(X^* \in \CC^5\) tel que
      \begin{equation*}
        X^* = \argmin{X\in \CC^5}  J_Z(X)
      \end{equation*}
    \end{thm}

    \begin{prop}
      \label{prop:cylindre:minimisation:minimum_sans_contraintes}
      Si \(\conj{\mA_{CI3}^t}\mA_{CI3}\) est inversible alors
      \begin{equation*}
        X^* = (\conj{\mA_{CI3}^t}\mA_{CI3})^{-1}\conj{\mA_{CI3}^t}g
      \end{equation*}
    \end{prop}

    \begin{proof}
      Même démonstration que pour le théorème \ref{prop:plan:minimisation:minimum_sans_contraintes}.
    \end{proof}

    Nous n'avons pas réussi à démontrer que cette matrice était définie pour tout empilement et tout incidence.

    \begin{thm}[Minimisation avec contraintes pour la CI3]

      Soit \(\CSU[3]{CI3}\) le sous-espace de \(\CC^5\) issu de la définition \ref{def:csu:ci3-3}.
      Alors les coefficients de la CIOE respectant les CSU sont solutions du problème

      Trouver \(X^* \in \CC^5\) tel que
      \begin{equation*}
        X^* = \argmin{X\in \CSU[3]{CI3}}  J_Z(X)
      \end{equation*}
    \end{thm}
