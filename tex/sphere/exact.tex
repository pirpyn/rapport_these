Nous rappelons d'abord de \cite{nedelec_acoustic_2001} des résultats connus sur les fonctions spéciales usuelles.

\section{Les harmoniques sphériques vectorielles}

    \begin{defn}[Harmonique sphérique d'ordre \(n\) et de degré \(m\)]
        \label{def:sphere:harmoniques_spheriques}
        Soit \(\PP^m_n\) la fonction associée de Legendre (voir \cite[\href{https://dlmf.nist.gov/14}{Chapitre.~14}]{dlmf_nist_2019}) d'ordre \(n\) et de degré \(m\). 

        On appelle harmoniques sphériques d'ordre \(n\) et de degré \(m\) les fonctions de \(\RR\times\RR \rightarrow \CC\) telles que
        \begin{align*}
            Y_{m,n}(\tp) &= (-1)^m\sqrt{\frac{2n+1}{4\pi}\frac{(n-m)!}{(n+m)!}}e^{im\phi}\PP^m_n(\cos \theta)
        \end{align*}
    \end{defn}

    \begin{prop}
        Les harmoniques sphériques \(Y_{m,n}\) sont solutions de 
        \begin{align*}
            \lapls{} u(\tp) = 0 && \forall (\tp) \in [0,\pi]\times[0,2\pi]
        \end{align*}
    \end{prop}
    \begin{prop}
        Soit \(S\) la sphère unité. Les harmoniques sphériques forment une base orthonormale de \(L^2(S)\).
    \end{prop}
    \begin{proof}
        Voir \cite[p.~14]{nedelec_acoustic_2001}
    \end{proof}

    \begin{defn}
        \label{def:sphere:bessel_spheriques}
        On appelle fonction de Bessel sphérique d'ordre \(n\) les fonctions de \(\CC \rightarrow \CC\) solution de
        \begin{align*}
            z^{2}\ddp[2]{z}{u(z)}+2z\ddp{z}{u(z)}+\left(z^{2}-n(n+1)\right)u(z)=0
        \end{align*}
    \end{defn}
    Ces fonctions sont détaillées dans \cite[\href{https://dlmf.nist.gov/10.47}{section.~10.47}]{dlmf_nist_2019}.
    On note \(j_n\) la fonction de Bessel sphérique de première espèce et \(h^{(2)_n}\) la fonction de Hankel de deuxième espèce.

    \begin{prop}
        Soit \(z_n\) une fonction de Bessel sphérique d'ordre \(n\) et \(Y_{m,n}\) une harmonique sphérique d'ordre \(n\) et de degré \(m\). Les fonctions \((\rtp) \mapsto z_n(k(r)r)Y_{m,n}(\tp)\) sont solutions de 
        \begin{align*}
            \lapl u(\rtp) + k^2u(\rtp) = 0 && \forall (\rtp) \in \RR\times[0,\pi]\times[0,2\pi]
        \end{align*}
    \end{prop}
    \begin{proof}
        Voir \cite[section.~2.6.1]{nedelec_acoustic_2001}
    \end{proof}

    \begin{defn}
        On définit l’opérateur \gls{mat-tild} agissant sur les fonctions dérivables tel que soit \(f \in C^1(\CC)\),
        \begin{align*}
        &\tilde{f}(z) = f(z) + zf'(z)&& \forall z \in \CC
        \end{align*}
    \end{defn}

    \begin{defn}
        \label{def:sphere:harmoniques_spheriques_vect}
        Soit \(z_n\) une fonction de Bessel sphérique d'ordre \(n\) et \(Y_{m,n}\) une harmonique sphérique d'ordre \(n\) et de degré \(m\).
        On appelle harmoniques sphériques vectorielles les fonctions \(\gls{phy-Mmn},\gls{phy-Nmn}\) telles que
        \begin{align}
            \label{eq:defMmn}
            \Mmn[z_n](\rtp) &:= \vrot \left( \vect{r} z_n(k(r)r) Y_{m,n}(\tp) \right)
            \\
            &= z_n(k(r)r)
            \begin{bmatrix}
                0
                \\
                \frac{im}{\sin\theta}Y_{mn}(\tp)
                \\
                - \ddr{\theta}{Y_{mn}}(\tp)
            \end{bmatrix}
            \\
            \label{eq:defNnn}
            \Nmn[z_n](\rtp) &:= \frac{\vrot \Mmn[z_n]}{k}(\rtp) 
            \\
            &= \frac{1}{k(r)r}\begin{bmatrix}
                z_n(k(r)r)n(n+1)Y_{mn}(\tp)
                \\
                \tilde{z_n}(k(r)r)\ddr{\theta}{Y_{mn}}(\tp)
                \\
                \tilde{z_n}(k(r)r)\frac{im}{\sin\theta}Y_{mn}(\tp)
            \end{bmatrix}
        \end{align}
    \end{defn}

    \begin{prop}
        Ces fonctions sont solutions de 
        \begin{equation*}
            \left\lbrace
                \begin{aligned}
                    \vrot \vrot \vect{U}(\rtp) - k^2 \vect{U}(\rtp) &= 0
                    \\
                    \vdiv \vect{U}(\rtp) &= 0
                \end{aligned}
            \right.
        \end{equation*}
    \end{prop}
    \begin{proof}
        Voir \cite[Théorème.~5.3.1]{nedelec_acoustic_2001} et sa démonstration ou l'annexe \ref{sec:annex:harmoniques_spheriques}.
    \end{proof}

    On admet les propriétés suivantes
    \begin{prop}
        \label{prop:Mmn_Nmn_rot}
        \begin{align*}
            \vrot \Mmn[z_n](\rtp) &= k\Nmn[z_n](\rtp)
            \\
            \vrot \Nmn[z_n](\rtp) &= k\Mmn[z_n](\rtp)
        \end{align*}
    \end{prop}

    % Ces vecteurs harmoniques sphériques possèdent les propriétés suivantes

    % \begin{align}
    % \int_{S(0,R)} \vect{M_{m,n}^{z_n}} \cdot \conj{\vect{N_{p,q}^{z_n}}} ds &= 0
    % \\
    % \int_{S(0,R)} \vect{M_{m,n}^{z_n}} \cdot \conj{\vect{M_{p,q}^{z_n}}} ds &= \gamma_{m,n}R^2 \delta_{mp}\delta_{nq}
    % \\
    % \int_{S(0,R)} \vect{N_{m,n}^{z_n}} \cdot \conj{\vect{N_{p,q}^{z_n}}} ds &= \frac{\gamma_{m,n}}{k^2} \delta_{mp}\delta_{nq}
    % \end{align}

    \begin{defn}
        On définit les fonctions de \(\RR\times\RR\mapsto \CC^2\)
        \begin{align}
            \label{eq:defUmn_tgt}
            \Umn(\tp) &=
            \begin{bmatrix}
                \frac{im}{\sin\theta}Y_{mn}(\tp)
                \\
                - \ddr{\theta}{Y_{mn}}(\tp)
            \end{bmatrix}
            \\
            \label{eq:defNmn_tgt}
            \Umn^\perp(\tp) &=
            \begin{bmatrix}
                \ddr{\theta}{Y_{mn}}(\tp)
                \\
                \frac{im}{\sin\theta}Y_{mn}(\tp)
            \end{bmatrix}
        \end{align}
    \end{defn}

    \begin{prop}
        On remarque alors que les parties tangentielles des vecteurs harmoniques sphériques peuvent s'écrire:
        \begin{align*}
          \Mmn[z_n]_t(\rtp) &= z_n(k(r)r)\Umn(\tp)
          \\
          \Nmn[z_n]_t(\rtp) &= \frac{1}{k(r)r}\tilde{z_n}(k(r)r)\Umn^\perp(\tp)
        \end{align*}
    \end{prop}

    On a alors les propriétés supplémentaires
    \begin{prop}
        \label{prop:Mmn_Nmn_vect}
        \begin{align*}
          \vect{e_r} \pvect \Mmn[z_n]_t(\rtp) &= k(r)rz_n(k(r)r)\Umn^\perp(\tp)
          \\
          \vect{e_r} \pvect \Nmn[z_n]_t(\rtp) &= -\frac{1}{k(r)r}\tilde{z_n}(k(r)r)\Umn(\tp)
        \end{align*}
    \end{prop}

\section{Expressions exactes des matrices d'impédance et des matrice de réflexions pour une sphère}

    \begin{figure}[!hbt]
        \centering
        \tikzsetnextfilename{sphere_1_couche}
        \begin{tikzpicture}
              \tikzmath{
    \a = 80;
    \b = 100;
    \d = 0.5;
    \ri = 20;
    \re = \ri;
}

% Le conducteur
\tikzmath{
    \ri = \re;
    \re = \ri + 0.5*\d;
    \xa = cos(\a)*\re;
    \ya = sin(\a)*\re;
    \xb = cos(\b)*\ri;
    \yb = sin(\b)*\ri;
}

\coordinate (a) at (\xa,\ya);
\coordinate (b) at (\xb,\yb);

\fill [pattern=north east lines] (a) arc (\a:\b:\re) -- (b) arc (\b:\a:\ri) -- cycle;
\draw (a) arc (\a:\b:\re);
\draw (a) node [right] {$r_0$};


% Le repère
\coordinate (n) at ($(a)+(0.5,-1)$);
%
%
%\draw [->] (n) -- ++(0,1) node [at end, right] {$\v{\pr}$};
%\draw [->] (n) -- ++(1,0) node [at end, right] {$\v{\pt}$};
%
\draw (n) ++(0.2,0.2) circle(0.1cm) node [above=0.1cm] {\(\vect{e_\phi}\)};
\draw (n) ++(0.2,0.2) +(135:0.1cm) -- +(315:0.1cm);
\draw (n) ++(0.2,0.2) +(45:0.1cm) -- +(225:0.1cm);


% 1ere couche
\tikzmath{
    \ri = \re;
    \re = \ri + \d;
    \xa = cos(\a)*\re;
    \ya = sin(\a)*\re;
    \xb = cos(\b)*\ri;
    \yb = sin(\b)*\ri;
    \xc = cos(0.5*(\b+\a))*(\ri+0.5*\d);
    \yc = sin(0.5*(\b+\a))*(\ri+0.5*\d);
}

\coordinate (a) at (\xa,\ya);
\coordinate (b) at (\xb,\yb);
\coordinate (c) at (\xc,\yc);

\fill [lightgray] (a) arc (\a:\b:\re) -- (b) arc (\b:\a:\ri) -- cycle;
\draw (a) arc (\a:\b:\re);
\draw (c) node {$\nu,\eta,d$};

% Le vide
\tikzmath{
    \xc = cos(0.5*(\b+\a))*(\re);
    \yc = sin(0.5*(\b+\a))*(\re);
}

\draw (\xc,\yc) node [above] {vide};
        \end{tikzpicture}
    \end{figure}

    On exprime les équations de Maxwell dans le matériau en coordonnées sphériques.

    Les solutions des équations de Maxwell sont développables en série de Mie
    \begin{prop}
        Il existe des constantes \((a_{mn},c_{mn},d_{mn},d_{mn})\) constantes par morceaux en \(r\) telles que les champs sont
        \begin{multline*}
            \vE(\rtp) = \sum_{n\in\ZZ}\sum_{m\in\ZZ} a_{mn} \Mmn[j_n](\rtp) + b_{mn} \Nmn[j_n](\rtp)
            \\
            + c_{mn} \Mmn[h^{(2)}_n](\rtp) + d_{mn} \Nmn[h^{(2)}_n](\rtp)
        \end{multline*}
        \begin{multline*}
            \vH(\rtp) = \frac{i}{\eta(r)}\sum_{n\in\ZZ}\sum_{m\in\ZZ} a_{mn} \Nmn[j_n](\rtp) + b_{mn} \Mmn[j_n](\rtp)
            \\
            + c_{mn} \Nmn[h^{(2)}_n](\rtp) + d_{mn} \Mmn[h^{(2)}_n](\rtp)
        \end{multline*}
    \end{prop}
    \begin{proof}
        Admise. On renvoie à \cite{cheng_spectral_1993}.
    \end{proof}

  \subsection{Expressions des champs tangentiels dans chaque couche}


    \begin{defn}
      \label{def:sphere:je-jH-hE-hH}
      On définit les matrices \(\mj_{E}(r,n),\mh_{E}(r,n),\mj_{H}(r,n),\mh_{H}(r,n)\) où \(k,\eta\) sont constants par morceaux en \(r\)
      \begin{align*}
        \mj_{E}(r,n) &=
        \begin{bmatrix}
            j_n(k(r)r) & 0
            \\
            0 & \tilde{j_n}(k(r)r)
        \end{bmatrix}
        \\
        \mh_{E}(r,n) &=
        \begin{bmatrix}
            h^{(2)}_n(k(r)r) & 0
            \\
            0 & \tilde{h^{(2)}_n}(k(r)r)
        \end{bmatrix}
        \\
        \mj_{H}(r,n) &=\frac{i}{k(r)\eta(r)r}
        \begin{bmatrix}
                -\tilde{j_n}(k(r)r) & 0
                \\
                0 & (k(r)r)^2 j_n(k(r)r)
        \end{bmatrix}
        \\
        \mh_{H}(r,n) &=\frac{i}{k(r)\eta(r)r}
        \begin{bmatrix}
                -\tilde{h^{(2)}_n}(k(r)r) & 0
                \\
                0 & (k(r)r)^2 h^{(2)}_n(k(r)r)
        \end{bmatrix}
      \end{align*}
    \end{defn}

    En utilisant les fonctions définies précédemment, on a les expressions des coordonnées tangentielles de \(\vE,\vect{e_r}\times\vH\).
    \begin{prop}
        \label{prop:imp_fourier:sphere:Et}
     \begin{align*}
            \vE_t(\rtp) &= \sum_{n\in\ZZ}\sum_{m\in\ZZ}
            \begin{bmatrix}
              \Umn(\tp) & \Umn^\perp(\tp)
            \end{bmatrix}
            \left(
              \mj_E(r,n)
              \begin{bmatrix}
                  a_{mn}
                  \\
                  b_{mn}
              \end{bmatrix}
              +
              \mh_E(r,n)
              \begin{bmatrix}
                  c_{mn}
                  \\
                  d_{mn}
              \end{bmatrix}
            \right)
            \\
            \vJ(\rtp) &= \sum_{n\in\ZZ}\sum_{m\in\ZZ}
            \begin{bmatrix}
                \Umn(\tp) & \Umn^\perp(\tp)
            \end{bmatrix}
            \left(
                \mj_H(r,n)
                \begin{bmatrix}
                    a_{mn}
                    \\
                    b_{mn}
                \end{bmatrix}
                +
                \mh_H(r,n)
                \begin{bmatrix}
                    c_{mn}
                    \\
                    d_{mn}
                \end{bmatrix}
            \right)
        \end{align*}
    \end{prop}

    \begin{proof}
        On exprime les champs tangentiels
        \begin{multline*}
            \vE_t(\rtp) = \sum_{n\in\ZZ}\sum_{m\in\ZZ} a_{mn} j_n(k(r)r)\Umn(\tp) + b_{mn} \frac{\tilde{j_n}(k(r)r)}{k(r)r}\Umn^\perp(\tp)
            \\
            + c_{mn} h^{(2)}_n(k(r)r)\Umn(\tp) + d_{mn} \frac{\tilde{j_n}(k(r)r)}{k(r)r}\Umn^\perp(\tp)
        \end{multline*}
        \begin{multline*}
            \vJ(\rtp) = \frac{i}{\eta(r)}\sum_{n\in\ZZ}\sum_{m\in\ZZ} - a_{mn} \frac{\tilde{j_n}(k(r)r)}{k(r)r}\Umn(\tp) + b_{mn} k(r)r j_n(k(r)r) \Umn^\perp(\tp)
            \\
            -  \frac{\tilde{h^{(2)}_n}(k(r)r)}{k(r)r} c_{mn} \Umn(\tp) + k(r)r h^{(2)}_n(k(r)r) d_{mn} \Umn^\perp(\tp)
        \end{multline*}
        Puis en réorganisant les termes, on a la propriété
        \begin{multline*}
            \vE_t(\rtp) = \sum_{n\in\ZZ}\sum_{m\in\ZZ}
            \begin{bmatrix}
              \Umn(\tp) & \Umn^\perp(\tp)
            \end{bmatrix}
            \\
            \left(
              \begin{bmatrix}
                j_n(k(r)r) & 0
                \\              
                0 & \tilde{j_n}(k(r)r)
              \end{bmatrix}
              \begin{bmatrix}
                  a_{mn}
                  \\
                  b_{mn}
              \end{bmatrix}
              +
              \begin{bmatrix}
                h^{(2)}_n(k(r)r) & 0
                \\
                0 & \tilde{h^{(2)}_n}(k(r)r)
              \end{bmatrix}
              \begin{bmatrix}
                  c_{mn}
                  \\
                  d_{mn}
              \end{bmatrix}
            \right)
        \end{multline*}
        \begin{multline*}
            \vJ(\rtp) = \frac{i}{k(r)\eta(r)r}\sum_{n\in\ZZ}\sum_{m\in\ZZ}
            \begin{bmatrix}
                \Umn(\tp) & \Umn^\perp(\tp)
            \end{bmatrix}
            \\
            \left(
                \begin{bmatrix}
                    -\tilde{j_n}(k(r)r) & 0
                    \\
                    0 & (k(r)r)^2 j_n(k(r)r)
                \end{bmatrix}
                \begin{bmatrix}
                    a_{mn}
                    \\
                    b_{mn}
                \end{bmatrix}
                +
                \begin{bmatrix}
                    -\tilde{h^{(2)}_n}(k(r)r) & 0
                    \\                
                    0 & (k(r)r)^2 h^{(2)}_n(k(r)r)
                \end{bmatrix}
                \begin{bmatrix}
                    c_{mn}
                    \\
                    d_{mn}
                \end{bmatrix}
            \right)
        \end{multline*}
    \end{proof}

    On remarque que \(\eps,\mu\) sont constantes par morceaux.
    Sur une interface d'équation \(r=r_p\) entre deux matériaux, il y a donc un saut de valeurs pour ces matrices: \(\lim_{\delta\rightarrow 0 } \mj_E(r_p+ \delta,n) - \mj_E(r_p - \delta,n) \not = 0\) .

    \begin{prop}
      \label{lem:sphere:imp:inv_matrices_JE-HE}
      Si \(\mu(r)\eps(r) \in \CC\backslash\RR\), alors les matrices \(\mj_E(r,n)\), \(\mj_H(r,n)\)  sont inversibles pour tout \(n\).
      % Sinon, pour \(n\) donné, il existe un nombre fini de \(k_z\) tels que ces matrices ne soient pas inversibles. 
    \end{prop}

    \begin{proof}
      Par définition des matrices,
      \begin{align*}
        \det(\mj_E(r,n)) &= j_n(k(r)r)\tilde{j_n}(k(r)r)
        \\
        \det(\mj_H(r,n)) &= -\frac{1}{\eta(r)^2}j_n(k(r)r)\tilde{j_n}
      \end{align*}

      D’après \cite[p.~370]{abramowitz_handbook_1964}, les zéros des fonctions de Bessel d'ordre réel \(\nu >-1\) sont tous réels, or les fonctions de Bessel sphériques d'ordre \(n\) sont des fonctions de Bessel d'ordre \(n+1/2\) donc si \(k(r) \in\CC\backslash\RR\) alors ces matrices sont inversibles.
    \end{proof}


    Pour les matrices \(\mh_E(r,n)\), \(\mh_H(r,n)\), on a
    \begin{align*}
        \det(\mh_E(r,n)) &= h^{(2)}_n(k(r)r)\tilde{h^{(2)}_n}(k(r)r)
        \\
        \det(\mh_H(r,n)) &= -\frac{1}{\eta(r)^2}h^{(2)}_n(k(r)r)\tilde{h^{(2)}_n}
     \end{align*}
    Et on ne peut rien conclure, car les zéros peuvent être complexes (voir \cite{sandstrom_note_2007}). On va donc supposer ces déterminants non nuls.


    %%%%%%%%%%%%%%%%%%%%%%%%%%%%%%%%%%%%%%%%%%%%%%%%%%%%%%%%%%%%%%%%%%%%%%%%%%%%%%%%%%%%%%%%%%%%%%%%%%%%%%%%
    %%%%%%%%%%%%%%%%%%%%%%%%%%%%%%%%%%%%%%%%%%%%%%%%%%%%%%%%%%%%%%%%%%%%%%%%%%%%%%%%%%%%%%%%%%%%%%%%%%%%%%%%
    %%%%%%%%%%%%%%%%%%%%%%%%%%%%%%%%%%%%%%%%%%%%%%%%%%%%%%%%%%%%%%%%%%%%%%%%%%%%%%%%%%%%%%%%%%%%%%%%%%%%%%%%


  \subsection{Expression de la matrice d'impédance pour une couche}

    Soit \(r_1 = r_0 + d\). On se place dans le matériau donc \(r_0 \le r < r_1\).

    \begin{prop}
      Si on suppose que la matrice \(\mh_{H}(r_1^-,n) - \mj_{H}(r_1^-,n)\mj_{E}(r_0^+,n)^{-1}\mh_{E}(r_0^+,n)\) est inversible pour tout \(n\), alors on a 
      \begin{equation*}
        \hat \vE_t(r_1^-,n) = \hat \mZ(n) \left(\vect{e_r}\pvect \hat \vH_t(r_1^-,n)\right)
      \end{equation*}
      où
      \begin{multline*}
        \hat \mZ(n) =
        \left(\mh_{E}(r_1^-,n)\mh_{E}(r_0^+,n)^{-1} - \mj_{E}(r_1^-,n)\mj_{E}(r_0^+,n)^{-1}\right)\\
        \left(\mh_{H}(r_1^-,n)\mh_{E}(r_0^+,n)^{-1} - \mj_{H}(r_1^-,n)\mj_{E}(r_0^+,n)^{-1}\right)^{-1}
      \end{multline*}
    \end{prop}

    \begin{proof}
      On se place dans la matériau: \(r_0 \le r \le r_ 1 \).

      On applique la condition limite du conducteur parfait \(\hat \vE(r_0,n) = 0\) dans \eqref{prop:imp_fourier:sphere:Et}. Comme les harmoniques sphériques forment un base de \(L^2(S(r_0))\), chaque terme doit être nul.
      \begin{equation*}
        \mj_{E}(r_0,n)
        \begin{bmatrix}
          a_{nm} \\
          b_{nm}
        \end{bmatrix}
        =-\mh_{E}(r_0,n)
        \begin{bmatrix}
          c_{nm} \\
          d_{nm}
        \end{bmatrix}
      \end{equation*}

      On peut donc exprimer les composantes tangentielles%, on omet les dépendances en \((n)\).
      \begin{align*}
        \hat \vE_t(r_1,n) &=
        \left(\mh_{E}(r_1,n) - \mj_{E}(r_1,n)\mj_{E}(r_0,n)^{-1}\mh_{E}(r_0,n)\right)
        \begin{bmatrix}
          c_{nm} \\
          d_{nm}
        \end{bmatrix}
        \\
        \vect{e_r}\pvect \hat \vH_t(r_1,n) &=
        \left(\mh_{H}(r_1,n) - \mj_{H}(r_1,n)\mj_{E}(r_0,n)^{-1}\mh_{E}(r_0,n) \right)
        \begin{bmatrix}
          c_{nm} \\
          d_{nm}
        \end{bmatrix}
      \end{align*}

      On suppose donc \(\mh_{H}(r_1,n) - \mj_{H}(r_1,n)\mj_{E}(r_0,n)^{-1}\mh_{E}(r_0,n)\) inversible, la matrice \(\hat\mZ(n)\) telle que \(\hat\vE_t(r_1,n) = \hat\mZ(n) (\vect{e_r}\pvect\vH(r_1,n))\) est alors

      \begin{multline*}
        \hat \mZ =
        \left(\mh_{E}(r_1,n) - \mj_{E}(r_1,n)\mj_{E}(r_0,n)^{-1}\mh_{E}(r_0,n)\right)
        \\
        \left(\mh_{H}(r_1,n) - \mj_{H}(r_1,n)\mj_{E}(r_0,n)^{-1}\mh_{E}(r_0,n)\right)^{-1}.
      \end{multline*}
      Comme on a supposé la matrice \(\mh_E(r_0,n)\) inversible, on peut factoriser par cette matrice à droite le numérateur et le dénominateur d'où la proposition.

    \end{proof}


  \subsection{Expression de la matrice d'impédance pour plusieurs couches}

    \begin{figure}[!hbt]
      \centering
      \tikzsetnextfilename{sphere_n_couches}
      \begin{tikzpicture}
        \tikzmath{
    \a = 83;
    \b = 97;
    \d = 0.5;
    \ri = 30;
    \re = \ri;
}

% Le conducteur
\tikzmath{
    \ri = \re;
    \re = \ri + 0.5*\d;
    \xa = cos(\a)*\re;
    \ya = sin(\a)*\re;
    \xb = cos(\b)*\ri;
    \yb = sin(\b)*\ri;
}

\coordinate (a) at (\xa,\ya);
\coordinate (b) at (\xb,\yb);

\fill [pattern=north east lines] (a) arc (\a:\b:\re) -- (b) arc (\b:\a:\ri) -- cycle;
\draw (a) arc (\a:\b:\re);
\draw (a) node [right] {$r_0$};

% Le repère
\coordinate (n) at ($(a)+(0.5,-1)$);
%
%
%\draw [->] (n) -- ++(0,1) node [at end, right] {$\v{\pr}$};
%\draw [->] (n) -- ++(1,0) node [at end, right] {$\v{\pt}$};
%
\draw (n) ++(0.2,0.2) circle(0.1cm) node [above=0.1cm] {$\vect{e_\phi}$};
\draw (n) ++(0.2,0.2) +(135:0.1cm) -- +(315:0.1cm);
\draw (n) ++(0.2,0.2) +(45:0.1cm) -- +(225:0.1cm);

% 1 ere couche

\tikzmath{
    \ri = \re;
    \re = \ri + \d;
    \xa = cos(\a)*\re;
    \ya = sin(\a)*\re;
    \xb = cos(\b)*\ri;
    \yb = sin(\b)*\ri;
    \xc = cos(0.5*(\b+\a))*(\ri+0.5*\d);
    \yc = sin(0.5*(\b+\a))*(\ri+0.5*\d);
}

\coordinate (a) at (\xa,\ya);
\coordinate (b) at (\xb,\yb);
\coordinate (c) at (\xc,\yc);

\fill [lightgray] (a) arc (\a:\b:\re) -- (b) arc (\b:\a:\ri) -- cycle;
\draw (a) arc (\a:\b:\re);
\draw (c) node {$\nu_1,\eta_1,d_1$};


% Des couches

\tikzmath{
    \ri = \re;
    \re = \ri + 2*\d;
    \xa = cos(\a)*\re;
    \ya = sin(\a)*\re;
    \xb = cos(\b)*\ri;
    \yb = sin(\b)*\ri;
    \xc = cos(0.5*(\b+\a))*(\ri+0.5*\d);
    \yc = sin(0.5*(\b+\a))*(\ri+0.5*\d);
}

\coordinate (a) at (\xa,\ya);
\coordinate (b) at (\xb,\yb);
\coordinate (c) at (\xc,\yc);

\fill [lightgray]    (a) arc (\a:\b:\re) -- (b) arc (\b:\a:\ri) -- cycle;
\fill [pattern=dots] (a) arc (\a:\b:\re) -- (b) arc (\b:\a:\ri) -- cycle;
\draw (a) arc (\a:\b:\re);

% n eme couche

\tikzmath{
    \ri = \re;
    \re = \ri + \d;
    \xa = cos(\a)*\re;
    \ya = sin(\a)*\re;
    \xb = cos(\b)*\ri;
    \yb = sin(\b)*\ri;
    \xc = cos(0.5*(\b+\a))*(\ri+0.5*\d);
    \yc = sin(0.5*(\b+\a))*(\ri+0.5*\d);
}

\coordinate (a) at (\xa,\ya);
\coordinate (b) at (\xb,\yb);
\coordinate (c) at (\xc,\yc);

\fill [lightgray] (a) arc (\a:\b:\re) -- (b) arc (\b:\a:\ri) -- cycle;
\draw (a) arc (\a:\b:\re);
\draw (c) node {$\nu_p,\eta_p,d_p$};

% Le vide
\tikzmath{
    \xc = cos(0.5*(\b+\a))*(\re);
    \yc = sin(0.5*(\b+\a))*(\re);
}

\draw (\xc,\yc) node [above] {vide};


      \end{tikzpicture}
    \end{figure}

    Soit \(r_p\) le rayon de l'interface \(p\), \(r_p = r_0 +\sum_{i=1}^{p} d_{i}\). On dit que l'on se trouve dans la couche \(p\) si \(r_{p-1} \le r \le r_p \).

    \begin{defn}
      \label{def:sphere:matrices_MJ-MH}
      On définit les fonctions de \([r_{p-1}, r_p[\times \NN \times \mathcal{M}_{2}(\CC) \rightarrow \mathcal{M}_{2}(\CC)\)
      \begin{align*}
        \mM_{\mj}(r,n,\mZ) &= \mj_{E}(r,n) -  \mZ \mj_{H}(r,n)
        \\
        \mM_{\mh}(r,n,\mZ) &= \mh_{E}(r,n) -  \mZ \mh_{H}(r,n)
      \end{align*}
    \end{defn}

    \begin{defn}
      \label{def:sphere:reflexion:impedance}
      On définit la fonction de \([r_{p-1}, r_p[\times \NN \times \mathcal{M}_{2}(\CC) \rightarrow \mathcal{M}_{2}(\CC)\)
      \begin{align*}
        \hat\mR(r,n,\mZ) &= -\mM_{\mh}(r,n,\mZ)^{-1}\mM_{\mj}(r,n,\mZ)
      \end{align*}
    \end{defn}
    A priori, pour \(r,n\) donné, \(\hat\mR(r,n,\mZ)\) n'est pas défini pour toute matrice \(\mZ\).

    On prolonge ces définitions aux autres couches.

    \begin{defn}%[Fonction de transfert]{}~
      \label{def:sphere:transfert:impedance}

      On définit \(\mT_p\) la fonction de \([r_{p-1}, r_p[^2\times\NN\times\RR\times\mathcal{M}_2(\CC)\rightarrow \mathcal{M}_{2}(\CC)\)
      \begin{multline*}
        \mT_p(r,r',n,\mA) = \\
          \left(\mj_{E}(r,n)\mM_{\mj}(r',n,\mA)^{-1} - \mh_{E}(r,n)\mM_{\mh}(r',n,\mA)^{-1}\right) 
          \\
          \left(\mj_{H}(r,n)\mM_{\mj}(r',n,\mA)^{-1} - \mh_{H}(r,n)\mM_{\mh}(r',n,\mA)^{-1}\right)^{-1}
      \end{multline*}
    \end{defn}
    A priori, pour \(r,r',n\) donné, \(\mT_p(r,r',n,\mA)\) n'est pas défini pour toute matrice \(\mA\).

    \begin{prop}%[Théorème de transfert]~
      \label{prop:sphere:transfert:impedance}

      Soient \(\hat\vE_t,\hat\vH_t\) tels que \(\vE_t(r_{p}^-,n) = \hat\mZ_{p}(n)(\vn \pvect \hat\vH(r_{p}^-,n)\).

      Si les matrices suivantes sont inversibles
      \begin{align*}
        \mM_{\mj}(r_p^-,n,\hat\mZ_{p}(n)) && \mM_{\mh}(r_p^-,n,\hat\mZ_{p}(n))
      \end{align*}
      \begin{align*}
        \mj_{H}(r_{p-1}^+,n)\mM_{\mj}(r_{p}^-,n,\hat\mZ_{p}(n))^{-1} - \mh_{H}(r_{p-1}^+,n)\mM_{\mh}(r_{p}^-,n,\hat\mZ_{p}(n))^{-1}
      \end{align*}

      alors \(\hat\vE_t(r_{p-1}^+,n) = \mT_p(r_{p-1}^+,r_{p}^-,n,\hat\mZ_{p}(n))(\vn \pvect \hat\vH(r_{p-1}^+,n))\).

      Une condition d'impédance sur le bord supérieur d'une couche détermine la condition limite sur le bord inférieur.
    \end{prop}


    \begin{proof}
      On se situe dans la couche \(p\) (\(r_{p-1}\le r\le r_p\))
      \begin{multline*}
        \mj_{E}(r_{p},n)
        \begin{bmatrix}
          a_{nm} \\
          b_{nm}
        \end{bmatrix}
        +
        \mh_{E}(r_{p},n)
        \begin{bmatrix}
          c_{nm} \\
          d_{nm}
        \end{bmatrix}
        =
        \\
        \hat \mZ_{p}(n)
        \left(
          \mj_{H}(r_{p},n)
          \begin{bmatrix}
            a_{nm} \\
            b_{nm}
          \end{bmatrix}
          +
          \mh_{H}(r_{p},n)
          \begin{bmatrix}
            c_{nm} \\
            d_{nm}
          \end{bmatrix}
        \right)
      \end{multline*}

      Ce qui revient à 
      \begin{equation*}
        \mM_{\mj}(r_{p},n,\hat\mZ_p(n))
        \begin{bmatrix}
          a_{nm} \\
          b_{nm}
        \end{bmatrix}
        =
        -\mM_{\mh}(r_{p},n,\hat\mZ_p(n))
        \begin{bmatrix}
          c_{nm} \\
          d_{nm}
        \end{bmatrix}
      \end{equation*}

      On suppose que les matrices \(\mM_{\mj}(r_p,n,\hat\mZ_p(n)), \mM_{\mh}(r_p,n,\hat\mZ_p(n))\) sont inversibles pour donc
      \begin{equation*}
        \begin{bmatrix}
          c_{nm} \\
          d_{nm}
        \end{bmatrix}
        =
        \mR(r_{p},n,\hat\mZ_p(n))
        \begin{bmatrix}
          a_{nm} \\
          b_{nm}
        \end{bmatrix}
      \end{equation*}

      On injecte ce qui précède en \(r = r_{p-1}\)
      \begin{align*}
        \hat{\vE}_t(r_{p-1},n) &= 
        \left(\mh_{E}(r_{p-1},n)\mR(r_{p},n,\hat{\mZ}_p(n)) + \mj_{E}(r_{p-1},n)\right)
        \begin{bmatrix}
          a_{nm} \\
          b_{nm}
        \end{bmatrix}
        \\
        \vect{e_r}\times\hat{\vH}_t(r_{p-1},n) &=
        \left(\mh_{H}(r_{p-1},n)\mR(r_{p},n,\hat{\mZ}_p(n)) + \mj_{H}(r_{p-1},n))\right)
        \begin{bmatrix}
          a_{nm} \\
          b_{nm}
        \end{bmatrix}
      \end{align*}

      On suppose alors que cette dernière est inversible pour tout \((n)\).

      On obtient
      \begin{multline*}
        \hat{\vE}_t(r_{p-1},n) =
        \left(\mj_{E}(r_{p-1},n) + \mh_{E}(r_{p-1},n)\mR(r_{p},n,\hat{\mZ}_p(n))\right) \\
        \left(\mj_{H}(r_{p-1},n) + \mh_{H}(r_{p-1},n)\mR(r_{p},n,\hat{\mZ}_p(n))\right)^{-1}
        \vect{e_r}\times\hat{\vH}_t(r_{p-1},n)
      \end{multline*}

      Comme on a supposé l'inversibilité des deux matrices \(\mM_j\), \(\mM_h\) alors on peut factoriser à droite le numérateur et le dénominateur et on a la propriété.
    \end{proof}

    \begin{prop}%[Théorème de relèvement]~
      \label{prop:sphere:relevement:impedance}

      Soient \(\hat\vE_t,\hat\vH_t\) tels que \(\vE_t(r_{p-1}^+,n) = \hat\mZ_{p-1}(n)(\vn \pvect \hat\vH(r_{p-1}^+,n)\).

      Si les matrices suivantes sont inversibles
      \begin{align*}
        \mM_{\mj}(r_{p-1}^+,n,\hat\mZ_{p-1}(n)), && \mM_{\mh}(r_{p-1}^+,n,\hat\mZ_{p-1}(n)),
      \end{align*}
      \begin{align*}
        \mh_{H}(r_{p}^-,n)\mM_{\mh}(r_{p-1}^+,n,\hat\mZ_{p-1}(n))^{-1} - \mj_{H}(r_{p}^-,n)\mM_{\mj}(r_{p-1}^+,n,\hat\mZ_{p-1}(n))^{-1},
      \end{align*}

      alors \(\hat\vE_t(r_{p}^-,n) = \mT_p(r_p^-,r_{p-1}^+,n,\hat\mZ_{p-1}(n))(\vn \pvect \hat\vH(r_{p}^-,n))\).

      Une condition d'impédance sur le bord inférieur d'une couche détermine la condition limite sur le bord supérieur.
    \end{prop}

    \begin{proof}
      Même méthodologie que pour la proposition \ref{prop:sphere:transfert:impedance}.
    \end{proof}

    \begin{prop}%[Corollaire aux théorèmes de transfert et de relèvement.]{}~
      \label{prop:sphere:synthese:impedance}{}~

      Soient \(\hat\vE_t,\hat\vH_t\) tels que 
      \begin{align*}
      \vE_t(r_{p-1}^+,n) &= \hat\mZ_{p-1}(n)(\vn \pvect \hat\vH(r_{p-1}^+,n))
      \\
      \vE_t(r_{p}^-,n) &= \hat\mZ_{p}(n)(\vn \pvect \hat\vH(r_{p}^-,n))
      \end{align*}

      Si les matrices suivantes sont inversibles
      \begin{align*}
        \mM_{\mj}(r_p^-,n,\hat\mZ_{p}(n)), && \mM_{\mj}(r_{p-1}^+,n,\hat\mZ_{p-1}(n)),
        \\
        \mM_{\mh}(r_p^-,n,\hat\mZ_{p}(n)), && \mM_{\mh}(r_{p-1}^+,n,\hat\mZ_{p-1}(n)),
      \end{align*}
      \begin{align*}
        \mj_{H}(r_{p}^-,n)\mM_{\mj}(r_{p-1}^+,n,\hat\mZ_{p-1}(n))^{-1} - \mh_{H}(r_{p}^-,n)\mM_{\mh}(r_{p-1}^+,n,\hat\mZ_{p-1}(n))^{-1},
        \\
        \mj_{H}(r_{p-1}^+,n)\mM_{\mj}(r_{p}^-,n,\hat\mZ_{p}(n))^{-1} - \mh_{H}(r_{p-1}^+,n)\mM_{\mh}(r_{p}^-,n,\hat\mZ_{p}(n))^{-1},
      \end{align*}

      Alors 
      \begin{align*}
        \hat\mZ_{p-1}(n) &= \mT_p(r_{p-1}^+,r_{p}^-,n,\hat\mZ_{p}(n))
        \\
        \hat\mZ_{p}(n) &= \mT_p(r_{p}^-,r_{p-1}^+,n,\hat\mZ_{p-1}(n))
      \end{align*}

    \end{prop}

    On peut donc déterminer itérativement les matrices d'impédance. Dans notre cadre d'étude, la présence d'un conducteur parfait sur l'interface \(r=r_0^+\) implique \(\hat\mZ_{0}(n) = 0\).

  %%%%%%%%%%%%%%%%%%%%%%%%%%%%%%%%%%%%%%%%%%%%%%%%%%%%%%%%%%%%%%%%%%%%%%%%%%%%%%%%%%%%%%%%%%%%%%%%%%%%%%%%
  %%%%%%%%%%%%%%%%%%%%%%%%%%%%%%%%%%%%%%%%%%%%%%%%%%%%%%%%%%%%%%%%%%%%%%%%%%%%%%%%%%%%%%%%%%%%%%%%%%%%%%%%
  %%%%%%%%%%%%%%%%%%%%%%%%%%%%%%%%%%%%%%%%%%%%%%%%%%%%%%%%%%%%%%%%%%%%%%%%%%%%%%%%%%%%%%%%%%%%%%%%%%%%%%%%

  \subsection{Expression des coefficients de la série de Fourier}

    On se place à l'interface \(p\) donc \(r_{p-1} \le r \le r_{p+1} \).

    \begin{defn}
      \label{def:sphere:matrices_NE-NH}
      On définit les fonctions de \(\RR\times \NN \times \mathcal{M}_{2}(\CC) \rightarrow \mathcal{M}_{2}(\CC)\)
      \begin{align*}
        \mN_{E}(r,n,\mA) &= \mj_{E}(r,n) + \mh_{E}(r,n)\mA
        \\
        \mN_{H}(r,n,\mA) &= \mj_{H}(r,n) + \mh_{H}(r,n)\mA
      \end{align*}
    \end{defn}

    \begin{defn}%[Fonction de transfert]{}~
      \label{def:sphere:transfert:reflexion}{}~

      On définit \(\mathfrak{T}_p\) la fonction de \([r_{p-1}, r_p]\times[r_p, r_{p+1}]\times\NN\times\RR\times\mathcal{M}_2(\CC)\rightarrow \mathcal{M}_{2}(\CC)\)
      \begin{multline*}
        \mathfrak{T}_p(r,r',n,\mA) = \\
          -\left(\mN_{E}(r',n,\mA)^{-1}\mh_{E}(r,n) - \mN_{H}(r',n,\mA)^{-1}\mh_{H}(r,n)\right)^{-1}
          \\
          \left(\mN_{E}(r',n,\mA)^{-1}\mj_{E}(r,n) - \mN_{H}(r',n,\mA)^{-1}\mj_{H}(r,n)\right)
      \end{multline*}
    \end{defn}
    A priori, pour \(r,r',n\) donné, \(\mathfrak{T}_p(r,r',n,\mA)\) n'est pas défini pour toute matrice \(\mA\).

    \begin{prop}%[Théorème de transfert]~
      \label{prop:sphere:transfert:reflexion}{}~

      On suppose qu'il existe \(\hat\mR_{p+1}(n)\) telle que 
      \begin{align*}
        \vE_t(r_{p}^+,n) &= \mN_{E}(r_p^+,n,\hat\mR_{p+1}(n))\vect{C}_{p+1}(n)
        \\
        \vect{e_r}\pvect\vH(r_{p}^+,n) &= \mN_{H}(r_p^+,n,\hat\mR_{p+1}(n))\vect{C}_{p+1}(n)
      \end{align*}

      Si les matrices suivantes sont inversibles
      \begin{align*}
        \mN_{E}(r_p^+,n,\hat\mR_{p+1}(n)), && \mN_{H}(r_p^+,n,\hat\mR_{p+1}(n)),
      \end{align*}
      \begin{align*}
        \mN_{E}(r_p^+,n,\hat\mR_{p+1}(n))^{-1}\mh_{E}(r_p^-,n) - \mN_{H}(r_p^+,n,\hat\mR_{p+1}(n))^{-1}\mh_{H}(r_p^-,n),
      \end{align*}
      alors
      \begin{align*}
        \vE_t(r_{p}^-,n) &= \mN_{E}(r_p^-,n,\mathfrak{T}_p(r_p^-,r_p^+,n,\hat\mR_{p+1}(n)))\vect{C}_{p}(n)
        \\
        \vE_t(r_{p}^-,n) &= \mN_{H}(r_p^-,n,\mathfrak{T}_p(r_p^-,r_p^+,n,\hat\mR_{p+1}(n)))\vect{C}_{p}(n)
      \end{align*}
    \end{prop}

    \begin{proof}
      De part et d'autre de \(r=r_p\), on a 
      \begin{align*}
        \vE_t(r_p^+,n) &= \mN_{E}(r_p^+,n,\hat\mR_{p+1}(n))\vect{C}_{1}^+(n)
        \\
        \vE_t(r_p^-,n) &= \mj_E(r_p^-,n)\vect{C}_{1}^-(n) + \mh_E(r_p^-,n)\vect{C}_{2}^-(n)
      \end{align*}
      \begin{align*}
        \vect{e_r}\pvect\vH(r_p^+,n) &= \mN_{H}(r_p^+,n,\hat\mR_{p+1}(n))\vect{C}_{1}^+(n)
        \\
        \vect{e_r}\pvect\vH(r_p^-,n) &= \mj_H(r_p^-,n)\vect{C}_{1}^-(n) + \mh_H(r_p^-,n)\vect{C}_{2}^-(n)
      \end{align*}
      Il y a continuité des champs au travers de l'interface donc
      \begin{align*}
        \mj_E(r_p^-,n)\vect{C}_{1}^-(n) + \mh_E(r_p^-,n)\vect{C}_{2}^-(n) &= \mN_{E}(r_p^+,n,\hat\mR_{p+1}(n))\vect{C}_{1}^+(n)
        \\
        \mj_H(r_p^-,n)\vect{C}_{1}^-(n) + \mh_H(r_p^-,n)\vect{C}_{2}^-(n) &= \mN_{H}(r_p^+,n,\hat\mR_{p+1}(n))\vect{C}_{1}^+(n)
      \end{align*}
      donc si on suppose les matrices \(\mN_E, \mN_H\) inversibles
      \begin{multline*}
        \mN_{E}(r_p^+,n,\hat\mR_{p+1}(n))^{-1}\left(\mj_E(r_p^-,n)\vect{C}_{1}^-(n) + \mh_E(r_p^-,n)\vect{C}_{2}^-(n)\right) =
        \\
        \mN_{H}(r_p^+,n,\hat\mR_{p+1}(n))^{-1}\left(\mj_H(r_p^-,n)\vect{C}_{1}^-(n) + \mh_H(r_p^-,n)\vect{C}_{2}^-(n)\right)
      \end{multline*}

      On factorise les termes
      \begin{multline*}
        \left(\mN_{E}(r_p^+,n,\hat\mR_{p+1}(n))^{-1}\mj_E(r_p^-,n) - \mN_{H}(r_p^+,n,\hat\mR_{p+1}(n))^{-1}\mj_H(r_p^-,n)\right)\vect{C}_{1}^-(n) =
        \\
        -\left(\mN_{E}(r_p^+,n,\hat\mR_{p+1}(n))^{-1}\mh_E(r_p^-,n) + \mN_{H}(r_p^+,n,\hat\mR_{p+1}(n))^{-1}\mh_H(r_p^-,n)\right)\vect{C}_{2}^-(n)
      \end{multline*}

      On suppose l'inversion de cette dernière et alors
      \begin{equation*}
        \vect{C}_{2}^-(n) = \mathfrak{T}_p(r_p^-,r_p^+,n,\hat\mR_{p+1}(n)) \vect{C}_{1}^-(n)
      \end{equation*}
    \end{proof}

    \begin{prop}%[Théorème de relévement]~
      \label{prop:sphere:relevement:reflexion}{}~

      On suppose qu'il existe \(\hat\mR_{p}(n)\) telle que 
      \begin{align*}
        \vE_t(r_{p}^-,n) &= \mN_{E}(r_p^-,n,\hat\mR_{p}(n))\vect{C}_{p}(n)
        \\
        \vect{e_r}\pvect\vH(r_{p}^-,n) &= \mN_{H}(r_p^-,n,\hat\mR_{p}(n))\vect{C}_{p}(n)
      \end{align*}

      Si les matrices suivantes sont inversibles
      \begin{align*}
        \mN_{E}(r_p^-,n,\mR_{p}(n)), && \mN_{H}(r_p^-,n,\hat\mR_{p}(n)),
      \end{align*}
      \begin{align*}
        \mN_{E}(r_p^-,n,\mR_{p}(n))^{-1}\mh_{E}(r_p^+,n) - \mN_{H}(r_p^-,n,\hat\mR_{p}(n))^{-1}\mh_{H}(r_p^+,n),
      \end{align*}
      alors
      \begin{align*}
        \vE_t(r_{p}^+,n) &= \mN_{E}(r_p^+,n,\mathfrak{T}_p(r_p^+,r_p^-,n,\hat\mR_{p}(n)))\vect{C}_{p+1}(n)
        \\
        \vE_t(r_{p}^+,n) &= \mN_{H}(r_p^+,n,\mathfrak{T}_p(r_p^+,r_p^-,n,\hat\mR_{p}(n)))\vect{C}_{p+1}(n)
      \end{align*}
    \end{prop}

    \begin{proof}
      Même méthodologie que pour la proposition \ref{prop:sphere:transfert:reflexion}.
    \end{proof}

    \begin{prop}%[Théorème de relévement]~
      \label{prop:sphere:synthese:reflexion}{}~

      On suppose qu'il existe \(\hat\mR_{p}(n)\) et \(\hat\mR_{p+1}(n)\) telles que 
      \begin{align*}
      &\left\lbrace\begin{aligned}
        \vE_t(r_{p}^-,n) &= \mN_{E}(r_p^-,n,\hat\mR_{p}(n))\vect{C}_{p}(n)
        \\
        \vect{e_r}\pvect\vH(r_{p}^-,n) &= \mN_{H}(r_p^-,n,\hat\mR_{p}(n))\vect{C}_{p}(n)
        \end{aligned}
      \right.
      \\
      &\left\lbrace\begin{aligned}
        \vE_t(r_{p}^+,n) &= \mN_{E}(r_p^+,n,\hat\mR_{p+1}(n))\vect{C}_{p+1}(n)
        \\
        \vect{e_r}\pvect\vH(r_{p}^+,n) &= \mN_{H}(r_p^+,n,\hat\mR_{p+1}(n))\vect{C}_{p+1}(n)
        \end{aligned}
      \right.      
      \end{align*}

      Si les matrices suivantes sont inversibles
      \begin{align*}
        \mN_{E}(r_p^-,n,\mR_{p}(n)), && \mN_{H}(r_p^-,n,\hat\mR_{p}(n)),
        \\
        \mN_{E}(r_p^+,n,\mR_{p+1}(n)), && \mN_{H}(r_p^+,n,\hat\mR_{p+1}(n)),
      \end{align*}
      \begin{align*}
        \mN_{E}(r_p^-,n,\mR_{p}(n))^{-1}\mh_{E}(r_p^+,n) - \mN_{H}(r_p^-,n,\hat\mR_{p}(n))^{-1}\mh_{H}(r_p^+,n),
        \\
        \mN_{E}(r_p^+,n,\mR_{p+1}(n))^{-1}\mh_{E}(r_p^-,n) - \mN_{H}(r_p^+,n,\hat\mR_{p+1}(n))^{-1}\mh_{H}(r_p^-,n),
      \end{align*}
      alors
      \begin{align*}
        \hat\mR_{p+1}(n) &= \mathfrak{T}_p(r_p^+,r_p^-,n,\hat\mR_{p}(n))
        \\
        \hat\mR_{p}(n) &= \mathfrak{T}_p(r_p^-,r_p^+,n,\hat\mR_{p+1}(n))
      \end{align*}
    \end{prop}

    On peut donc déterminer itérativement les matrices de réflexions. Dans notre cadre d'étude, la présence d'un conducteur parfait sur l'interface \(r=r_0^+\) implique \(\mR_{1}(n) = -\mh_E(r_0^+,n)^{-1}\mj_E(r_0^+,n)\).

  \subsection{Applications numériques}
    Comme pour le cylindre, on pose \((k_x,k_y) = (k_0 s, 0)\) pour le plan. On compare l'impédance du plan \(\hat{\mZ}(k_x,0)\) à l'impédance de la sphère \(\hat{\mZ}(n)\) quand \(n\) est de l'ordre de \(k_0r_1s\). On reprend la notation de \cite[p.~62]{hoppe_impedance_1995} qui défini \(k_t= n/r_1\).

    Pour une couche de matériau sans pertes, la matrice \(\hat\mZ\) est imaginaire pure, donc les parties réelles ne sont pas tracées.

    On trace dans la figure \ref{fig:imp_fourier:sphere:hoppe_p62:converge_rayon} les parties imaginaires des coefficients diagonaux de ces matrices.

    \begin{figure}[!hbt]
      \centering
      \tikzsetnextfilename{Z_HOPPE_62_sphere_converge.TM}
\begin{tikzpicture}[scale=1]
  \begin{axis}[
      title={Polarisation TM},
      ylabel={\(\Im(\hat{Z}(n));\Im(\hat{Z}(k_x,0))\)},
      xlabel={\(k_t \slash k_0 ; k_x \slash k_0\)},
      width=0.37\textwidth,
      xmin=0,
      xmax=1.5,
      legend pos=outer north east
    ]

    \addplot [black,dotted,mark=diamond] table [col sep=comma, x={s2}, y={Im(z_ex.22)}] {csv/HOPPE_62/HOPPE_62.z_ex.MODE_2_TYPE_S_+3.000E-02m.csv};

    %\addplot [black,dotted,mark=*] table [col sep=comma, x={s2}, y={Im(z_ex.22)}] {csv/HOPPE_62/HOPPE_62.z_ex.MODE_2_TYPE_S_+3.000E-01m.csv};

    %\addplot [black,dashed] table [col sep=comma, x={s2}, y={Im(z_ex.22)}] {csv/HOPPE_62/HOPPE_62.z_ex.MODE_2_TYPE_S_+3.000E+00m.csv};

    \addplot [black] table [col sep=comma, x={s1}, y={Im(z_ex.11)}] {csv/HOPPE_62/HOPPE_62.z_ex.MODE_2_TYPE_P.csv};
  \end{axis}
\end{tikzpicture}
\tikzsetnextfilename{Z_HOPPE_62_sphere_converge.TE}
\begin{tikzpicture}[scale=1]
  \begin{axis}[
      title={Polarisation TE},
      ylabel={},
      xlabel={\(k_t \slash k_0 ; k_x \slash k_0\)},
      width=0.37\textwidth,
      xmin=0,
      xmax=1.5,
      legend pos=outer north east
    ]

    \addplot [black,dotted,mark=diamond] table [col sep=comma, x={s2}, y={Im(z_ex.11)}] {csv/HOPPE_62/HOPPE_62.z_ex.MODE_2_TYPE_S_+3.000E-02m.csv};
    \addlegendentry{\(r_0=0.03m\)}

    %\addplot [black,dotted,mark=*] table [col sep=comma, x={s2}, y={Im(z_ex.11)}] {csv/HOPPE_62/HOPPE_62.z_ex.MODE_2_TYPE_S_+3.000E-01m.csv};
    %\addlegendentry{\(r_0=0.3m\)}

    %\addplot [black,dashed] table [col sep=comma, x={s2}, y={Im(z_ex.11)}] {csv/HOPPE_62/HOPPE_62.z_ex.MODE_2_TYPE_S_+3.000E+00m.csv};
    %\addlegendentry{\(r_0=3m\)}

    \addplot [black] table [col sep=comma, x={s1}, y={Im(z_ex.22)}] {csv/HOPPE_62/HOPPE_62.z_ex.MODE_2_TYPE_P.csv};
    \addlegendentry{plan}
  \end{axis}
\end{tikzpicture}
      \caption{\(\eps = 6, \mu = 1, d=0.0225\text{m}, f=1\text{GHz}\)}
      \label{fig:imp_fourier:sphere:hoppe_p62:converge_rayon}
    \end{figure}
     On voit donc que l'impédance de la sphère tend vers celle du plan quand \(r\) augmente.