\section{Calcul des coefficients de la CI3 par moindres carrés sur les coefficients de la série de Fourier}

  Soit \(\mM_j\) et \(\mM_h\) les fonctions introduites à la définition \ref{def:sphere:matrices_MJ-MH} et \(\hat\mR\) la fonction définie à la définition \ref{def:sphere:reflexion:impedance}.

  \begin{defn}%[]
    \label{def:sphere:minimisation:matrices_MR}
    On définit les fonctions \(\hat\mR_{ex}, \hat\mR_{CI3}\) de \(\NN\times \rightarrow \mathcal{M}_2(\CC)\) telles que
    \begin{align*}
      \hat\mR_{ex}(n) &= \hat\mR(n, \hat\mZ_{ex}(n)),
      \\
      \hat\mR_{CI3}(n) &= \hat\mR(n, \hat\mZ_{CI3}(n)),
    \end{align*}
    où \(\hat\mZ_{ex},\hat\mZ_{CI3}\) sont des fonctions définies à la proposition \ref{prop:sphere:synthese:impedance} et à l'équation \eqref{eq:sphere:hoibc:ci3}.
  \end{defn}

  \subsection{Expression de la fonctionnelle}

    On utilise les fonctions \(\mN_E, \mN_H\) introduites à la définition \ref{def:sphere:matrices_NE-NH}.

    \begin{defn}
      On définit \(\mA_0,\mA_1,\mA_2,\mA_2,\mB_1,\mB_2\) les fonctions de \(\NN \times \mathcal{M}_2(\CC) \rightarrow \ \mathcal{M}_2(\CC)\) telles que        
      \begin{align*}
        \mA_0(n,\mR) &= \mN_E(r_c^+,n,\mR),
        \\
        \mA_1(n,\mR) &= \hat{\mLD}(n)\mN_E(r_c^+,n,\mR),
        \\
        \mA_2(n,\mR) &= -\hat{\mLR}(n)\mN_E(r_c^+,n,\mR),
        \\
        \mB_1(n,\mR) &= \hat{\mLD}(n)\mN_H(r_c^+,n,\mR),
        \\
        \mB_2(n,\mR) &= -\hat{\mLR}(n)\mN_H(r_c^+,n,\mR).
      \end{align*}

      On définit \(\tilde{\mH}_{CI3}\) la fonction de \(\NN \times \mathcal{M}_2(\CC) \rightarrow \mathcal{M}_{4\times5}(\CC)\) telle que
      \begin{align*}
        & \tilde\mH_{CI3}(n,\mR) =  \\ &
        \begin{bmatrix}
          \mA_0(n,\mR)_{11} & \mA_1(n,\mR)_{11} & \mA_2(n,\mR)_{11} & \mB_1(n,\mR)_{11} & \mB_2(n,\mR)_{11}
          \\
          \mA_0(n,\mR)_{12} & \mA_1(n,\mR)_{12} & \mA_2(n,\mR)_{12} & \mB_1(n,\mR)_{12} & \mB_2(n,\mR)_{12}
          \\
          \mA_0(n,\mR)_{21} & \mA_1(n,\mR)_{21} & \mA_2(n,\mR)_{21} & \mB_1(n,\mR)_{21} & \mB_2(n,\mR)_{21}
          \\
          \mA_0(n,\mR)_{22} & \mA_1(n,\mR)_{22} & \mA_2(n,\mR)_{22} & \mB_1(n,\mR)_{22} & \mB_2(n,\mR)_{22}
        \end{bmatrix}.
      \end{align*}

      On définit \(\tilde{b}\) la fonction de \(\NN \times \mathcal{M}_2(\CC) \rightarrow \mathcal{M}_{4\times1}(\CC)\) telle que
      \begin{equation*}
        \tilde{b}(n,\mR) = -
        \begin{bmatrix}
          \mN_H(r_c^+,n,\mR)_{11}
          \\
          \mN_H(r_c^+,n,\mR)_{12}
          \\
          \mN_H(r_c^+,n,\mR)_{21}
          \\
          \mN_H(r_c^+,n,\mR)_{22}
        \end{bmatrix}.
      \end{equation*}
    \end{defn}

    \begin{prop}
      Soit \(X = (a_0,a_1,a_2,b_1,b_2)^t\), \((n)\) fixé et \(\hat\mR_{ex}\) la matrice définie en \ref{def:sphere:minimisation:matrices_MR}, alors
      \begin{align*}
        \argmin{X\in\CC^5} \norm{\hat\mR_{CI3}(n,X) - \hat\mR_{ex}(n)} =
        \argmin{X\in\CC^5} \norm{\tilde{\mH}_{CI3}(n,\hat\mR_{ex}(n))X - \tilde{b}(n,\hat\mR_{ex}(n))}.
      \end{align*}
    \end{prop}

    \begin{proof}
      C'est la même méthodologie que pour l'impédance.
      On rappelle de la section précédente
      \begin{align*}
        \hat{\mZ}_{CI3}(n) = \left(\mI + b_1 \hat{\mLD}(n) - b_2 \hat{\mLR}(n) \right)^{-1}
        \left(a_0 \mI + a_1 {\hat{\mLD}(n)} - a_2 {\hat{\mLR}(n)}\right).
      \end{align*}
      On pose \(\hat\mZ_D(n,X) = \mI + b_1 \hat{\mLD}(n) - b_2 \hat{\mLR}(n)\) et \(\hat\mZ_N(n,X) = a_0 \mI + a_1 {\hat{\mLD}(n)} - a_2 {\hat{\mLR}(n)}\) donc
      \begin{align*}
        &{\hspace{1em}}~ \argmin{X\in\CC^5} \norm{\hat\mR_{CI3}(n,X) - \hat\mR_{ex}(n)}
        \\
        & = \argmin{X\in\CC^5} \norm{ - \mM_h(r_s^+,n,\hat\mZ_{CI3})^{-1}\mM_j(r_s^+,n,\hat\mZ_{CI3})- \hat\mR_{ex}(n) },
        \\
        & = \argmin{X\in\CC^5} \norm{ - \mM_h(r_s^+,n,\hat\mZ_{CI3})^{-1}\left(\mM_j(r_s^+,n,\hat\mZ_{CI3}) +  \mM_h(r_s^+,n,\hat\mZ_{CI3})\hat\mR_{ex}(n)\right) },      
        \\ 
        & = \argmin{X\in\CC^5} \norm{\mM_j(r_s^+,n,\hat\mZ_{CI3}) +\mM_h(r_s^+,n,\hat\mZ_{CI3})\hat\mR_{ex}(n)}.
        \intertext{D'après la définition \ref{def:sphere:matrices_MJ-MH} des fonctions \(\mM_j, \mM_h\),}
        & = \argmin{X\in\CC^5} \left\lVert \left(\mJ_E(r_s^+,n)-\hat\mZ_{CI3}(n)\mJ_H(r_s^+,n)\right) \right.
        \\
        & \qquad \qquad \quad + \left.\left(\mH_E(r_s^+,n)-\hat\mZ_{CI3}(n)\mH_H(r_s^+,n)\right)\hat\mR_{ex}(n) \right\lVert.
        \intertext{D'après la définition de \(\hat\mZ_{CI3}\),}        
        & = \argmin{X\in\CC^5} \left\lVert \hat\mZ_D(n)^{-1}\left(\hat\mZ_D(n)\mJ_E(r_s^+,n)-\hat\mZ_N(n)\mJ_H(r_s^+,n)\right) \right.
        \\
        & \qquad \qquad \quad + \left.\hat\mZ_D(n)^{-1}\left(\hat\mZ_D(n)\mH_E(r_s^+,n)-\hat\mZ_N(n)\mH_H(r_s^+,n)\right)\hat\mR_{ex}(n) \right\lVert,
        \\
        & = \argmin{X\in\CC^5} \left\lVert \left(\hat\mZ_D(n)\mJ_E(r_s^+,n)-\hat\mZ_N(n)\mJ_H(r_s^+,n)\right) \right.
        \\
        & \qquad \qquad \quad + \left.\left(\hat\mZ_D(n)\mH_E(r_s^+,n)-\hat\mZ_N(n)\mH_H(r_s^+,n)\right)\hat\mR_{ex}(n) \right\lVert.
        \intertext{D'après la définition \ref{def:sphere:matrices_NE-NH} des fonctions \(\mN_E, \mN_H\),}        
        & = \argmin{X\in\CC^5} \norm{\hat\mZ_N(n)\mN_E(r_s^+,n,\hat\mR_{ex}(n)) + \hat\mZ_D(n)\mN_H(r_s^+,n,\hat\mR_{ex}(n))},
      \end{align*}
      et on conclut d'après la définition des fonctions \(\hat\mZ_D, \hat\mZ_N\).
    \end{proof}

    On tronque la série de Mie à \(N_{n}\) termes.
    \begin{defn}
      On définit \(\tilde{\mA}_{CI3}\) la matrice de \(\mathcal{M}_{4N_{n}\times5}(\CC)\) telle que
      \begin{equation*}
        \tilde{\mA}_{CI3} = 
        \begin{bmatrix}
          \tilde\mH_{CI3}(n_1,\hat\mR_{ex}(n_1))
          \\
          \vdots
          \\
          \tilde\mH_{CI3}(n_i,\hat\mR_{ex}(n_i))
          \\
          \vdots
          \\
          \tilde\mH_{CI3}(n_{N_n},\hat\mR_{ex}(n_{N_n}))
        \end{bmatrix}.
      \end{equation*}
      On définit \(\tilde{g}\) le vecteur colonne \(\CC^{4N_{n}}\) telle que
      \begin{equation*}
        \tilde{g} = 
        \begin{bmatrix}
          \tilde{b}(n_1,\hat\mR_{ex}(n_1))
          \\
          \vdots
          \\
          \tilde{b}(n_i,\hat\mR_{ex}(n_i))
          \\
          \vdots
          \\
          \tilde{b}(n_{N_n},\hat\mR_{ex}(n_{N_n}))
        \end{bmatrix}.
      \end{equation*}
    \end{defn}

    On peut alors calculer les coefficients de la CI3

    \begin{defn}
      On définit la fonctionnelle \(J_R\)
      \begin{equation*}
        J_R(X) = \norm{\tilde{\mA}_{CI3}X - \tilde{g}}.
      \end{equation*}
    \end{defn}

    \begin{thm}[Minimisation sans contraintes pour la CI3]

      Les coefficients de la CIOE sont solutions du problème

      Trouver \(X^* \in \CC^5\) tel que
      \begin{equation*}
        X^* = \argmin{X\in \CC^5} J_R(X).
      \end{equation*}
    \end{thm}

    \begin{prop}
      Si \(\conj{\tilde{\mA}_{CI3}^t}\tilde{\mA}_{CI3}\) est inversible, alors
      \begin{equation*}
        X^* = (\conj{\tilde{\mA}_{CI3}^t}\tilde{\mA}_{CI3})^{-1}\conj{\tilde{\mA}_{CI3}^t}\tilde{g}.
      \end{equation*}
    \end{prop}
    \begin{proof}
      Même méthode que pour la proposition \ref{prop:sphere:minimisation:minimum_sans_contraintes} sur l'impédance.
    \end{proof}

    Nous n'avons pas réussi à démontrer que cette matrice était définie pour tout empilement et toute incidence.

    \begin{thm}[Minimisation avec contraintes pour la CI3]

      Soit \(\CSU[3]{CI3}\) le sous-espace de \(\CC^5\) issu de la définition \ref{def:csu:ci3-3}, alors les coefficients de la CIOE respectant les CSU sont solutions du problème

      Trouver \(X^* \in \CC^5\) tel que
      \begin{equation*}
        X^* = \argmin{X\in \CSU[3]{CI3}} J_R(X).
      \end{equation*}
    \end{thm}
