\section{Approximation de l'opérateur d'impédance pour un cylindre infini}

  \subsection{Expression de la condition d'impédance d'ordre élevée}

    On reprend les méthodes et résultats de la partie précédente.
    Soit \(R_{ext}\) le rayon de la couche extérieure.
    Le changement de système de coordonnées modifie l'expression des opérateurs \(\LD, \LR\).

    \begin{equation}
      \hat{\LD}(n,k_z) = -
      \begin{bmatrix}
        \left(\frac{n}{R_{ext}}\right)^2 & \frac{n}{R_{ext}} k_z
        \\
        \frac{n}{R_{ext}} k_z & k_z^2
      \end{bmatrix}
    \end{equation}

    \begin{equation}
      \hat{\LR}(n,k_z) =
      \begin{bmatrix}
        k_z^2 & -\frac{n}{R_{ext}} k_z
        \\
        -\frac{n}{R_{ext}} k_z & \left(\frac{n}{R_{ext}}\right)^2
      \end{bmatrix}
    \end{equation}

    On peut donc redéfinir \(\hat{\mZ}_{IBC}\) l’opérateur matriciel associé à la condition d'impédance.

    \begin{multline}
        \hat{\mZ}_{IBC}(n,k_z) = \left(I + b_1 \hat{\LD}(n,k_z) - b_2 \hat{\LR}(n,k_z) \right)^{-1}\\
        \left(a_0 I + a_1 \hat{\LD}(n,k_z) - a_2 \hat{\LR}(n,k_z)\right)
    \end{multline}

    Les coefficients s'obtiennent de la même manière par minimisation.
    De plus, là encore il faut au moins considérer \((n\slash R_{ext})^2 + k_z^2\) supérieur à \(k_0^2\) pour prendre en compte des ondes planes homogènes. Dans le cas d'une incidence normale au cylindre (\(k_z = 0 \)), cela revient à prendre \(n\) au moins jusque \(k_0 R_{ext}\).

  \subsection{Résultats numériques}

  La figure \ref{fig:imp_fourier:plan:hoppe:62:hoibc} permet de vérifier les résultats de \cite[p.~62]{hoppe_impedance_1995} pour une couche de matériau sans perte.

 \begin{figure}[!hbt]
          \centering
          \begin{tikzpicture}[scale=1]
              \begin{axis}[
                      title={Polarisation TM},
                      ylabel={\(\Im(\hat{\mZ}(k_t r_{1},0)\)},
                      xlabel={\(k_t\slash k_0\)},
                      width=0.4\textwidth,
                      xmin=0,
                      xmax=2.5,
                      mark repeat=20,
                      legend pos=outer north east
                  ]
                  \addplot [black,mark=square*] table [col sep=comma, x={s1}, y={Im(z_ex.tm)}] {tikz/csv/impedance/HOPPE_62/HOPPE_62.z_ex.C_+3.000E-02.csv};

                  \addplot [blue,mark=x] table [col sep=comma, x={s1}, y={Im(z_ibc0.tm)}] {tikz/csv/impedance/HOPPE_62/HOPPE_62.z_ibc.IBC_ibc0_TYPE_C_+3.000E-02_SUC_F.csv};

                  \addplot [red,mark=diamond*] table [col sep=comma, x={s1}, y={Im(z_ibc3.tm)}] {tikz/csv/impedance/HOPPE_62/HOPPE_62.z_ibc.IBC_ibc3_TYPE_C_+3.000E-02_SUC_F.csv};
              \end{axis}
          \end{tikzpicture}
          \begin{tikzpicture}[scale=1]
              \begin{axis}[
                      title={Polarisation TE},
                      ylabel={},
                      xlabel={\(k_t\slash k_0\)},
                      width=0.4\textwidth,
                      xmin=0,
                      xmax=2.5,
                      mark repeat=20,
                      legend pos=outer north east
                  ]
                  \addplot [black,mark=square*] table [col sep=comma, x={s1}, y={Im(z_ex.te)}] {tikz/csv/impedance/HOPPE_62/HOPPE_62.z_ex.C_+3.000E-02.csv};
                  \addlegendentry{Exact};

                  \addplot [blue,mark=x] table [col sep=comma, x={s1}, y={Im(z_ibc0.te)},color=] {tikz/csv/impedance/HOPPE_62/HOPPE_62.z_ibc.IBC_ibc0_TYPE_C_+3.000E-02_SUC_F.csv};
                  \addlegendentry{CI0};

                  \addplot [red,mark=diamond*] table [col sep=comma, x={s1}, y={Im(z_ibc3.te)}] {tikz/csv/impedance/HOPPE_62/HOPPE_62.z_ibc.IBC_ibc3_TYPE_C_+3.000E-02_SUC_F.csv};
                  \addlegendentry{CI3};
              \end{axis}
          \end{tikzpicture}
          \caption[CIOE sur empilement de Hoppe & Rahmat-Samii p.~62]{\(\eps = 6, \mu = 1, d=0.0225\text{m}, f=1\text{GHz}, r_0=0.03\text{m}\)}
          \label{fig:imp_fourier:plan:hoppe:62:hoibc}
      \end{figure}


    On remarque que le CIOE CI3 si performante dans l'approximation plan infini ne donnent pas de bons résultats dans l’approximation cylindre infini. 
    En effet, le symbole de l'opérateur n'est pas un multiple de l'identité pour \(n=k_z=0\) au contraire du plan ( pour \(k_x=k_y=0\) ). 
    Or la CIOE est multiple de l'identité pour ce couple. 
    On subit donc cette erreur dans les résultats. 

    Une CIOE plus intéressante, que l'on nomme CI6 et qui est inspirée de \cite[p.~60]{hoppe_impedance_1995}, serait:

    \begin{equation}
      \left(\oI + c_1\LD -c_2\LR\right)\vE_t = \left(\diag{a_1}{a_2} + b_1\LD - b_2 \LR \right)\vJ
    \end{equation}

     \begin{figure}[!hbt]
          \centering
          \begin{tikzpicture}[scale=1]
              \begin{axis}[
                      title={Polarisation TM},
                      ylabel={\(\Im(\hat{\mZ}(k_t r_{1},0)\)},
                      xlabel={\(k_t\slash k_0\)},
                      width=0.4\textwidth,
                      xmin=0,
                      xmax=2.5,
                      mark repeat=20,
                      legend pos=outer north east
                  ]
                  \addplot [black,mark=square*] table [col sep=comma, x={s1}, y={Im(z_ex.tm)}] {tikz/csv/impedance/HOPPE_62/HOPPE_62.z_ex.C_+3.000E-02.csv};

                  \addplot [blue,mark=x] table [col sep=comma, x={s1}, y={Im(z_ibc0.tm)}] {tikz/csv/impedance/HOPPE_62/HOPPE_62.z_ibc.IBC_ibc0_TYPE_C_+3.000E-02_SUC_F.csv};

                  \addplot [red,mark=diamond*] table [col sep=comma, x={s1}, y={Im(z_ibc3.tm)}] {tikz/csv/impedance/HOPPE_62/HOPPE_62.z_ibc.IBC_ibc3_TYPE_C_+3.000E-02_SUC_F.csv};

                  \addplot [violet,mark=triangle*] table [col sep=comma, x={s1}, y={Im(z_ibc6.tm)}] {tikz/csv/impedance/HOPPE_62/HOPPE_62.z_ibc.IBC_ibc6_TYPE_C_+3.000E-02_SUC_F.csv};
              \end{axis}
          \end{tikzpicture}
          \begin{tikzpicture}[scale=1]
              \begin{axis}[
                      title={Polarisation TE},
                      ylabel={},
                      xlabel={\(k_t\slash k_0\)},
                      width=0.4\textwidth,
                      xmin=0,
                      xmax=2.5,
                      mark repeat=20,
                      legend pos=outer north east
                  ]
                  \addplot [black,mark=square*] table [col sep=comma, x={s1}, y={Im(z_ex.te)}] {tikz/csv/impedance/HOPPE_62/HOPPE_62.z_ex.C_+3.000E-02.csv};
                  \addlegendentry{Exact};

                  \addplot [blue,mark=x] table [col sep=comma, x={s1}, y={Im(z_ibc0.te)},color=] {tikz/csv/impedance/HOPPE_62/HOPPE_62.z_ibc.IBC_ibc0_TYPE_C_+3.000E-02_SUC_F.csv};
                  \addlegendentry{CI0};

                  \addplot [red,mark=diamond*] table [col sep=comma, x={s1}, y={Im(z_ibc3.te)}] {tikz/csv/impedance/HOPPE_62/HOPPE_62.z_ibc.IBC_ibc3_TYPE_C_+3.000E-02_SUC_F.csv};
                  \addlegendentry{CI3};

                  \addplot [violet,mark=triangle*] table [col sep=comma, x={s1}, y={Im(z_ibc6.te)}] {tikz/csv/impedance/HOPPE_62/HOPPE_62.z_ibc.IBC_ibc6_TYPE_C_+3.000E-02_SUC_F.csv};
                  \addlegendentry{CI6};
              \end{axis}
          \end{tikzpicture}
          \caption[CIOE sur empilement de Hoppe & Rahmat-Samii p.~62]{\(\eps = 6, \mu = 1, d=0.0225\text{m}, f=1\text{GHz}, r_0=0.03\text{m}\)}
          \label{fig:imp_fourier:plan:hoppe:62:hoibc:ibc6}
      \end{figure}

      Cependant cette CIOE ne sera pas retenue car son implémentation dans le code équation intégrale nécessite une modification de ce dernier. On présente néanmoins dans la figure \ref{fig:imp_fourier:plan:hoppe:62:hoibc:ibc6} sa performance vis à vis de la CI3 sur le cylindre.