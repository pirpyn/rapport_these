\thispagestyle{empty}
\begin{center}
\Large
\textbf{\doctitlefr}
\end{center}
\section*{Résumé}
Nous considérons la diffraction électromagnétique d'un objet modélisé par une condition d'impédance (CI). L'originalité de cette thèse est l'établissement de conditions suffisantes d'unicité (CSU) pour calculer les coefficients de cette condition limite et leur mise en œuvre dans un code équation intégrale. Cette modélisation est validée sur certains objets d'intérêt par comparaison avec des codes de référence.
\\

\textbf{Mots-clefs}


\dockeywordsfr

\hrulefill
\begin{center}
\Large
\textbf{\doctitleeng}
\end{center}
\section*{Abstract}
We consider the electromagnetic scattering from an object modelled by an impedance boundary condition (IBC). The originality of this thesis is the exhibition of sufficient uniqueness conditions (SUC) to compute the IBC coefficients and their implementation in a integral equation code. The approximated solution is then validated on some objects of interest.
\\

\textbf{Keywords}


\dockeywordseng
