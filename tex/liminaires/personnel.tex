\chapterstar{Remerciements}

Je souhaite remercier les rapporteurs \mme{} Hélène \textsc{Barucq} et M{} Xavier \textsc{Antoine} pour les remarques et conseils bienveillants dans la rédaction de ce manuscrit.
Je remercie aussi les membres du jury, \mme{} Marion \textsc{Darbas}, \mme{} Laurence \textsc{Halpern}, \mme{} Bérangère \textsc{Delourme}, M{} Paul \textsc{Soudais} pour leurs questions pendant la soutenance, elles ont amenés de belles discutions.
Enfin je remercie mes deux directeurs de thèse.

Bruno, et j'en profite aussi pour remercier à travers toi les membres du CEA sans les nommer, car tel est l'usage, je te remercie de la patience dont tu as fait preuve à l'égard de la production de ce manuscrit, je n'ai pas toujours voulu suivre le chemin que tu avais déjà tracé pour me l'approprier.
J'ai eu la chance de pouvoir assister grâce au CEA à des conférences passionnantes et visiter le LMJ (Laser MégaJoule) et je recommande à des doctorants qui me liraient d'en profiter, ne serait-ce que par curiosité scientifique.

Olivier, je te remercie, et le mot est trop faible, pour tout ce que tu as apporté à cette thèse. 
Depuis les cours d'ingénieur de la MACS, j'ai apprécié ta pédagogie, ton enthousiasme, voire ton exubérance. Il fallait suivre ton rythme ( "Zébulon", premier cours de 2013, et je ne permettrai pas qu'il tombe dans l'oubli), jusqu'à traverser l'Atlantique où, après de studieuses journées, nous profitions du charme de Montréal.
 Ça à beaucoup compter. Merci.

~{}\\

Beaucoup de personnes ont contribué, sans le savoir à cette thèse. À vous.

Thomas, Jeanne, Clémence, Édouard, pour ces incroyables soirées au R4ndom, chez nous et chez vous. Tunak Tunak ou Kazoo ? 
Alex, Isa, Thibault, Benjamin, pour le plaisir des jeux de société au parc Bordelais. Vive le Hannabi, maudite soit ma mémoire.
Antho, Ben, Jean, Ptich, Nico, Dadou parce que la compagnie de la Belette vaincra, car c'était quand même une sacrée demi-finale et attend, attend ... attend !
Guigui, Noémie, Alexandre, Corentin, quel heureux hasard que cette promo 2010 de la prépa Camille Guérin. En attendant un prochain Freemusic ou nouvel An.


Gentien et Rémi qui m'ont embrigadé dans la section foot, je vous dois quelques courbatures. Thomas, mon correspondant corse LaTeX et Matlab. Paul et Julien, avec qui j'ai pu enfin discuter ondes au milieu de toutes ces couches limites.
Mickaël et Manon que j'ai réussi à embrigader dans la section foot du CEA, pour les pique-niques au bord du bassin.

~{}\\

À Florence, Yann, Sarah et Solène, merci pour l'accueil chaleureux que vous m'avez accordé.

À mes parents, mes plus grands supporters, merci pour tout. À Ambre et Gabin, je vous souhaite le meilleur dans la vie étudiante et active qui s'ouvre à vous. J'ai ptet pas de flow, mais vous resterez "les petits".

À Floriane, tout simplement.

\href{https://www.wolframalpha.com/input/?i=parametric+plot+x+\%3D+16+sin\%5E3+t\%2C+y+\%3D+13+cos+t+-+5+cos\%282+t\%29+-+2+cos+\%28+3+t+\%29+-cos\%284t\%29}{\(x(t) = 16 \sin^3(t), y(t) = 13 \cos(t) - 5 \cos(2 t) - 2 \cos ( 3 t ) - \cos(4t), \forall t \in [0,2\pi]\)}

\sectionstar{Informations}

Ce manuscrit a été réalisé avec \(\LaTeXe\) dont les sources sont disponibles à l'adresse \url{https://github.com/pirpyn/thesis} et compilent avec l'outil \href{https://mg.readthedocs.io/latexmk.html}{latexmk}. 
Cette version est celle du 17/12/2020.

Un code de calcul numérique des coefficients de CIOE en C++, inspiré par celui qui appartient au CEA et écrit en Fortran pendant la thèse, est disponible à l'adresse \url{https://github.com/pirpyn/choibc}.


