\section{CSU pour la CIOE de \cite{aubakirov_electromagnetic_2014}}

  Soit la CIOE énoncé dans \cite{soudais_3d_2017} que l'on nommera CI3 :
  \begin{equation}
    \label{eq:unicite:ci3:ci3}
    ( 1 + b_1 L_D - b_2 L_R)\vE_t = (a_0 + a_1 L_D - a_2 L_R ) \vJ
  \end{equation}

  On rappelle les expressions des opérateurs \gls{ope-LD} et \gls{ope-LR} pour des vecteurs tangents \(\vect U \in (C^\infty(\Gamma))^3\), \( \vect V \in (C^\infty(\Gamma))^3\) :
  \begin{align*}
    \LD(\vect U) &= \tgrads \tdivs \vect U\\
    \LR(\vect V) &= \trots( \vn ( \vn \cdot \trots \vect V))\\
  \end{align*}
  \begin{prop}
    Soit \(\OO\) un domaine borné de \(\RR^3\) , de surface \(\Gamma\) fermée et régulière, où \(\vect n\) y est la normale unitaire
    sortante
    \begin{equation}
      \begin{matrix}
        \forall \vect U \in (C^\infty(\Gamma))^3 ,& \LR(\LD(\vect U)) = \LD(\LR(\vect U)) = 0
      \end{matrix}
    \end{equation}
  \end{prop}
  \begin{proof}

    Soient un vecteur \textbf{tangent} \(\vect U \in (C^\infty(\Gamma))^3\).

    Montrons que \(\LR\LD = 0\). D’après \cite[p.~1029, A3.42]{bladel_electromagnetic_2007}, \(\vn \cdot \trots\tgrads f = 0\)
    \begin{align*}
      \LR(\LD \vect U)  &= \trots \left(\vn \left(\vn \cdot \trots \left( \tgrads \left(\tdivs \vect U\right)\right)\right)\right) \\
      &= 0
    \end{align*}
    Montrons que \(\LD\LR = 0\). D’après \cite[p.~1029, A3.43]{bladel_electromagnetic_2007}, \(\tdivs \trots (X\vn) = 0\).
    \begin{align*}
      \LD(\LR \vect U) &= \tgrads \tdivs \trots (\vn (\vn \cdot \trots \vect U)) \\
      &= 0
    \end{align*}
  \end{proof}
  % Une relation importante qui découle des propriétés des opérateurs différentiels surfacique \secref{eq:op-LD-LR:prop:LDLR0} est :

  % \begin{equation}
  % \int_\Gamma L_D(\vect U) \cdot L_R(\vect V) ds = 0 , \forall \vect U, \vect V \in (H^1(\OO))^3
  % \end{equation}

  % Cette relation \Gamma'exprime sous forme forte par \(L_DL_R\equiv0\). Elle est là aussi symétrique entre les deux opérateurs.

\subsection{CSU de Stupfel}

  On prend \eqref{eq:unicite:ci3:ci3} et on l’intègre avec des produits scalaires judicieusement choisis.

  \begin{multline}
    \label{eq:unicite:ci3:csu3-1}
    \int_\Gamma \vJ\cdot\conj{\eqref{eq:unicite:ci3:ci3}}ds \Rightarrow
    \int_\Gamma \vJ \cdot \conj{\vE_t} ds  + \conj{b_1} \int_\Gamma \vJ\cdot L_D\conj{\vE_t} ds - \conj{b_2} \int_\Gamma \vJ L_R\conj{\vE_t} ds \\
    = \conj{a_0} \int_\Gamma |\vJ|^2ds - \conj{a_1} \int_\Gamma |\tdivs \vJ|^2 ds - \conj{a_2} \int_\Gamma |\vn \cdot \trots \vJ|^2 ds
  \end{multline}
  \begin{multline}
    \label{eq:unicite:ci3:csu3-2}
    \int_\Gamma \eqref{eq:unicite:ci3:ci3} \cdot \conj{\vE_t} ds \Rightarrow
    \int_\Gamma |\vE_t|^2 ds  - b_1 \int_\Gamma | \tdivs \vE |^2 ds - b_2 \int_\Gamma | \vn \cdot \trots \vE_t|^2 ds \\
    = a_0 \int_\Gamma \vJ\cdot \conj{\vE_t}ds + a_1 \int_\Gamma \conj{\vE_t} L_D \vJ ds - a_2 \int_\Gamma \conj{\vE_t} \cdot L_R \vJ ds
  \end{multline}
  \begin{multline}
    \label{eq:unicite:ci3:csu3-3}
    \int_\Gamma \vJ \cdot L_R ( \conj{\eqref{eq:unicite:ci3:ci3}} ) ds \Rightarrow
    \int_\Gamma \vJ \cdot L_R \conj{\vE_t} ds  - \conj{b_2} \int_\Gamma L_R \vJ \cdot L_R \conj{\vE_t} ds \\
    =  \conj{a_0} \int_\Gamma |\vn \cdot \trots \vJ|^2ds - \conj{a_2} \int_\Gamma | L_R \vJ|^2 ds
  \end{multline}
  \begin{multline}
    \label{eq:unicite:ci3:csu3-4}
    \int_\Gamma  L_R ( \eqref{eq:unicite:ci3:ci3} ) \cdot \conj{\vE_t} ds \Rightarrow
    \int_\Gamma | \vn \cdot \trots \vE_t |^2 ds  - \conj{b_2} \int_\Gamma | L_R \vE_t|^2 ds \\
    = a_0 \int_\Gamma \conj{\vE_t} L_R \vJ ds - a_2 \int_\Gamma L_R \conj{\vE_t} \cdot L_R \vJ ds
  \end{multline}
    \begin{multline}
    \label{eq:unicite:ci3:csu3-5}
    \int_\Gamma \vJ \cdot L_D ( \conj{\eqref{eq:unicite:ci3:ci3}} ) ds \Rightarrow
    \int_\Gamma \vJ \cdot L_D \conj{\vE_t} ds  + \conj{b_1} \int_\Gamma L_D \vJ \cdot L_D \conj{\vE_t} ds \\
    = - \conj{a_0} \int_\Gamma |\tdivs \vJ|^2ds + \conj{a_1} \int_\Gamma | L_D \vJ|^2 ds
  \end{multline}
  \begin{multline}
    \label{eq:unicite:ci3:csu3-6}
    \int_\Gamma  L_D ( \eqref{eq:unicite:ci3:ci3} ) \cdot \conj{\vE_t} ds \Rightarrow
    -\int_\Gamma | \tdivs \vE_t |^2 ds  + \conj{b_1} \int_\Gamma | L_D \vE_t|^2 ds \\
    = a_0 \int_\Gamma \conj{\vE_t} L_D \vJ ds + a_1 \int_\Gamma L_D \conj{\vE_t} \cdot L_D \vJ ds
  \end{multline}
  On pose alors les définitions suivantes :
  \begin{align*}
    X&:= \int_\Gamma \vJ \cdot \conj{\vE_t} ds\\
    Y_D&:= \int_\Gamma \vJ \cdot L_D \conj{\vE_t} ds
    &Y_R&:= \int_\Gamma \vJ \cdot L_R \conj{\vE_t} ds\\
    Z_D&:= \int_\Gamma L_D \vJ \cdot L_D \conj{\vE_t} ds
    &Z_R&:= \int_\Gamma L_R \vJ \cdot L_R \conj{\vE_t} ds
  \end{align*}

  Les équations \eqref{eq:unicite:ci3:csu3-1} à \eqref{eq:unicite:ci3:csu3-4} sont équivalentes au système \(M_R X_R = F_R\) où

  \begin{align*}
    M_R&:=
    \begin{bmatrix}
      1&\conj{b_1}&-\conj{b_2}&0\\
      a_0&a_1&-a_2&0\\
      0&0&1&-\conj{b_2}\\
      0&0&a_0&-a_2\\
    \end{bmatrix},\;
    X_R =
    \begin{bmatrix}
      X\\
      Y_D\\
      Y_R\\
      Z_R
    \end{bmatrix}\\
    F_R &=
    \begin{bmatrix}
      \conj{a_0} \int_\Gamma |\vJ|^2ds - \conj{a_1} \int_\Gamma |\tdivs \vJ|^2 ds - \conj{a_2} \int_\Gamma |\vn \cdot \trots \vJ|^2 ds \\
      \int_\Gamma |\vE_t|^2 ds  - b_1 \int_\Gamma | \tdivs \vE |^2 ds - b_2 \int_\Gamma | \vn \cdot \trots \vE_t|^2 ds \\
      \conj{a_0} \int_\Gamma |\vn \cdot \trots \vJ|^2ds - \conj{a_2} \int_\Gamma | L_R \vJ|^2 ds \\
      \int_\Gamma | \vn \cdot \trots \vE_t |^2 ds  - \conj{b_2} \int_\Gamma | L_R \vE_t|^2 ds
    \end{bmatrix}
  \end{align*}

  Tandis que les équations \eqref{eq:unicite:ci3:csu3-1},\eqref{eq:unicite:ci3:csu3-2},\eqref{eq:unicite:ci3:csu3-5},\eqref{eq:unicite:ci3:csu3-6} sont équivalentes au système \(M_D X_D= F_D\) où

  \begin{align*}
    M_D&:=
    \begin{bmatrix}
      1&-\conj{b_2}&\conj{b_1}&0\\
      a_0&-a_2&a_1&0\\
      0&0&1&\conj{b_1}\\
      0&0&a_0&a_1\\
    \end{bmatrix},\;
    X_D =
    \begin{bmatrix}
      X\\
      Y_R\\
      Y_D\\
      Z_D
    \end{bmatrix}\\
    F_D &=
    \begin{bmatrix}
      \conj{a_0} \int_\Gamma |\vJ|^2ds - \conj{a_1} \int_\Gamma |\tdivs \vJ|^2 ds - \conj{a_2} \int_\Gamma |\vn \cdot \trots \vJ|^2 ds \\
      \int_\Gamma |\vE_t|^2 ds  - b_1 \int_\Gamma | \tdivs \vE |^2 ds - b_2 \int_\Gamma | \vn \cdot \trots \vE_t|^2 ds \\
      -\conj{a_0} \int_\Gamma |\tdivs \vJ|^2ds + \conj{a_1} \int_\Gamma | L_R \vJ|^2 ds \\
      -\int_\Gamma | \tdivs \vE_t |^2 ds  + \conj{b_1} \int_\Gamma | L_R \vE_t|^2 ds
    \end{bmatrix},\;
  \end{align*}

  On note dans la suite \(\Delta_i = a_i-\conj{b_i}a_0\), \(i=1,2\). On suppose que ces système aient une unique solution. Alors on obtient la première condition suffisante:

  \begin{equation}
    \label{eq:unicite:ci3:csu3-cn-det}
    \Delta_1\Delta_2 \not = 0
  \end{equation}

  \begin{minipage}{0.49\textwidth}
    \textbf{Cas LR}:
    \begin{align}
      \label{eq:unicite:ci3:csu3r-j2}&\Re\left(a_0\conj{a_2}\Delta_2\right) \ge 0 \\
      \label{eq:unicite:ci3:csu3r-e2}&\Re\left(\frac{\conj{b_2}}{\Delta_2}\right) \le 0 \\
      \label{eq:unicite:ci3:csu3r-jdj}&\Re\left(\conj{a_0}a_1\left(\frac{\conj{b_1}}{\Delta_1}-\frac{\conj{b_2}}{\Delta_2}\right) + \frac{\conj{a_1}a_2}{\Delta_2} \right)\le 0\\
      \label{eq:unicite:ci3:csu3r-ede}&\Re\left(2\Re(b_1)\frac{\conj{b_2}}{\Delta_2}-\frac{\conj{b_1}^2}{\Delta_1}\right) \ge 0\\
      \label{eq:unicite:ci3:csu3r-jrj}&\Re\left(|a_2|^2\Delta_2\right) \le 0 \\
      \label{eq:unicite:ci3:csu3r-ere}&\Re\left(|b_2|^2\Delta_2\right) \ge 0 \\
      \label{eq:unicite:ci3:csu3r-rj2}&\Re\left(|a_1|^2\left(\frac{\conj{b_1}}{\Delta_1}-\frac{\conj{b_2}}{\Delta_2}\right)\right)\ge 0\\
      \label{eq:unicite:ci3:csu3r-re2}&\Re\left(|b_1|^2\left(\frac{\conj{b_1}}{\Delta_1}-\frac{\conj{b_2}}{\Delta_2}\right)\right)\le 0
    \end{align}
    Les conditions \eqref{eq:unicite:ci3:csu3r-jrj} et \eqref{eq:unicite:ci3:csu3r-ere} impliquent :
    \begin{equation}
      \Re\left(\Delta_2\right) = 0\\\
    \end{equation}
    Les conditions \eqref{eq:unicite:ci3:csu3r-rj2} et \eqref{eq:unicite:ci3:csu3r-re2} impliquent :
    \begin{equation}
      \Re\left(\frac{\conj{b_1}}{\Delta_1}-\frac{\conj{b_2}}{\Delta_2}\right) = 0\\\
    \end{equation}
  \end{minipage}
  \begin{minipage}{0.49\textwidth}
    \textbf{Cas LD}:
    \begin{align}
      \label{eq:unicite:ci3:csu3d-j2}&\Re\left(a_0\conj{a_1}\Delta_1\right) \ge 0 \\
      \label{eq:unicite:ci3:csu3d-e2}&\Re\left(\frac{\conj{b_1}}{\Delta_1}\right) \le 0 \\
      \label{eq:unicite:ci3:csu3d-jrj}&\Re\left(\conj{a_0}a_2\left(\frac{\conj{b_2}}{\Delta_2}-\frac{\conj{b_2}}{\Delta_2}\right) + \frac{\conj{a_2}a_1}{\Delta_1} \right)\le 0\\
      \label{eq:unicite:ci3:csu3d-ere}&\Re\left(2\Re(b_2)\frac{\conj{b_1}}{\Delta_1}-\frac{\conj{b_2}^2}{\Delta_2}\right) \ge 0\\
      \label{eq:unicite:ci3:csu3d-jdj}&\Re\left(|a_1|^2\Delta_1\right) \le 0 \\
      \label{eq:unicite:ci3:csu3d-ede}&\Re\left(|b_1|^2\Delta_1\right) \ge 0 \\
      \label{eq:unicite:ci3:csu3d-dj2}&\Re\left(|a_2|^2\left(\frac{\conj{b_2}}{\Delta_2}-\frac{\conj{b_1}}{\Delta_1}\right)\right)\ge 0\\
      \label{eq:unicite:ci3:csu3d-de2}&\Re\left(|b_2|^2\left(\frac{\conj{b_2}}{\Delta_2}-\frac{\conj{b_1}}{\Delta_1}\right)\right)\le 0
    \end{align}
    Les conditions \eqref{eq:unicite:ci3:csu3d-jdj} et \eqref{eq:unicite:ci3:csu3d-ede} impliquent :
    \begin{equation}
      \Re\left(\Delta_1\right) = 0\\\
    \end{equation}
    Les conditions \eqref{eq:unicite:ci3:csu3d-dj2} et \eqref{eq:unicite:ci3:csu3d-de2} impliquent :
    \begin{equation}
      \Re\left(\frac{\conj{b_1}}{\Delta_1}-\frac{\conj{b_2}}{\Delta_2}\right) = 0\\\
    \end{equation}
  \end{minipage}

  %Pour le système \(M_D X_D = F_D\), les conditions sont identiques à une permutation des indices 1 et 2 près.
  Ces CSU sont très contraignantes et ne permettent pas de retrouver des CSU des CIOE d'ordres inférieurs lorsque l'on annule les coefficients \(b_1, b_2\).


\subsection{CSU de Payen}
  On remarque que les inconnus \((Y_R,Z_R)\) (resp. \((Y_D,Z_R)\)) sont déterminées par les équations \eqref{eq:unicite:ci3:csu3-3} et \eqref{eq:unicite:ci3:csu3-4} (resp. \eqref{eq:unicite:ci3:csu3-5} et \eqref{eq:unicite:ci3:csu3-6})

  On déduit donc que si \(\Delta_1 \not = 0\) et \(\Delta_2 \not = 0\) alors

  \begin{align}
    Y_R &= \frac{1}{\Delta_2}\left(a_2\left[\conj{a_0}\int_\Gamma \vJ\cdot\LR\conj{\vJ} - \conj{a_2}||\LR J||^2\right]  -\conj{b_2}\left[\int_\Gamma \conj{\vE}\LR{\vE} - b_2 ||\LR \vE ||^2\right]\right) \\
    Y_D &= \frac{1}{\Delta_1}\left(a_1\left[\conj{a_0}\int_\Gamma \vJ\cdot\LD\conj{\vJ} + \conj{a_1}||\LD J||^2\right]  -\conj{b_1}\left[\int_\Gamma \conj{\vE}\LD{\vE} + b_1 ||\LD \vE ||^2\right]\right)
  \end{align}

  Il reste alors à utiliser l'équation \eqref{eq:unicite:ci3:csu3-1} pour obtenir
  \begin{equation}
    X = -\conj{b_1} Y_D + \conj{b_2} Y_R + \conj{a_0} || \vJ ||^2 + \conj{a_1} \int_\Gamma \vJ \LD \conj{J} - \conj{a_2} \int_\Gamma \vJ \LR \conj{J}
  \end{equation}

  \begin{multline}
    X = \conj{a_0} || \vJ ||^2 - \conj{a_1} || \vdiv \vJ ||^2 - \conj{a_2} || \vn \times \vrot \vJ ||^2\\
    + \frac{\conj{b_2}}{\Delta_2}\left(a_2\left[\conj{a_0}||\vn \times \vrot \vJ||^2 - \conj{a_2}||\LR J||^2\right]  -\conj{b_2}\left[||\vn\times\vrot\vE||^2 - b_2 ||\LR \vE ||^2\right]\right) \\
    - \frac{\conj{b_1}}{\Delta_1}\left(a_1\left[-\conj{a_0}||\vdiv\vJ||^2 + \conj{a_1}||\LD J||^2\right]  -\conj{b_1}\left[-||\vdiv\vE||^2 + b_1 ||\LD \vE ||^2\right]\right)
  \end{multline}

  Et on obtiens des CSU suivantes :

  \begin{align}
    \Re(a_0)\ge 0 \\
    \Re(a_1 - \frac{\conj{b_1a_0}a_1}{\Delta_1}) \le 0 \\
    \Re(a_2 - \frac{\conj{b_2a_0}a_2}{\Delta_2}) \le 0 \\
    \Re(b_1\Delta_1) = 0 \\
    \Re(b_2\Delta_2) = 0 \\
    \Im(b_1\Delta_1)\Im(b_1)\ge 0\\
    \Im(b_2\Delta_2)\Im(b_2)\ge 0
  \end{align}

\subsection{CSU de Lafitte-Stupfel}

  On rappelle l'expression de la CIOE

  \begin{equation}
    \left(\oI + b_1 \LD - b_2 \LR \right) \vE_t = \left(a_0\oI + a_1 \LD - a_2 \LR \right) \vJ
  \end{equation}

  Par définition de la CIOE, on a

  \begin{align}
    X &= \int_\Gamma \left(a_0\oI + a_1 \LD - a_2 \LR \right)^{-1}\left(\oI + b_1 \LD - b_2 \LR \right) \vE_t\cdot \conj{\vE_t}
  \end{align}

  \begin{multline}
    X = \int_\Gamma \left(a_0\oI + a_1 \LD - a_2 \LR \right)^{-1}
    \\
    + b_1 \left(a_0\oI + a_1 \LD - a_2 \LR \right)^{-1}\LD
    \\
    \left.
    - b_2 \left(a_0\oI + a_1 \LD - a_2 \LR \right)^{-1}\LR \right) \vE_t\cdot \conj{\vE_t}
  \end{multline}

  \begin{multline}
    X = \int_\Gamma \left(a_0\oI + a_1 \LD - a_2 \LR \right)^{-1}
    \\
    + b_1 \left(a_0\oI + a_1 \LD - a_2 \LR \right)^{-1}\LD
    \\
    \left.
    - b_2 \left(a_0\oI + a_1 \LD - a_2 \LR \right)^{-1}\LR \right) \vE_t\cdot \conj{\vE_t}
  \end{multline}

  On note que
  \begin{align}
    \LD & = \frac{a_0 + a_1 \LD - a_0}{a_1}
    \\
    \LR & = -\frac{a_0 - a_2 \LD - a_0}{a_2}
  \end{align}

  On pose
  \begin{equation}
    z = \left(1 - \frac{b_1a_0}{a_1} - \frac{b_2a_0}{a_2}\right)
  \end{equation}

  On déduit de ce qui précède que

  \begin{multline}
    X = \int_\Gamma z\left(a_0\oI + a_1 \LD - a_2 \LR \right)^{-1}
    \\
    + \frac{b_1}{a_1} \left(a_0\oI + a_1 \LD - a_2 \LR \right)^{-1}\left(a_0+a_1\LD\right)
    \\
    - \frac{b_2}{a_2} \left(a_0\oI + a_1 \LD - a_2 \LR \right)^{-1}\left(a_0-a_2\LR\right) \vE_t\cdot \conj{\vE_t}
  \end{multline}

  On définit

  \newcommand{\vD}{\vect{D}}
  \newcommand{\vF}{\vect{F}}

  \begin{align}
    \vD & = \left(a_0 \oI + a_1 \LD - a_2\LR \right)^{-1} \vE_t
    \\
    \vF_1 & = \left(\oI - a_2 \left( a_0 + a_1\LD\right)^{-1}\LR\right)^{-1} \vE_t
    \\
    \vF_2 & = \left(\oI + a_1 \left( a_0 - a_2\LR\right)^{-1}\LD\right)^{-1} \vE_t
  \end{align}

  Alors immédiatement, on a 

  \begin{multline}
    X = \int_\Gamma z \vD \cdot \left(\conj{a_0} \oI + \conj{a_1} \LD - \conj{a_2}\LR\right)\conj{\vD}
    \\
    + \frac{b_1}{a_1} \left(\oI - \conj{a_2} \left( \conj{a_0} + \conj{a_1}\LD\right)^{-1}\LR\right)\conj{\vF_1}\cdot\vF_1
    \\
    + \frac{b_2}{a_2} \left(\oI + \conj{a_1} \left( \conj{a_0} - \conj{a_2}\LR\right)^{-1}\LD\right)\conj{\vF_2}\cdot\vF_2
  \end{multline}

  Finalement posons

  \newcommand{\vG}{\vect{G}}

  \begin{align}
    \vG_1 & = \left(\conj{a_0} \oI + \conj{a_1} \LD \right)^{-1}\LR \conj{\vF_1}
    \\
    \vG_2 & = \left(\conj{a_0} \oI - \conj{a_2} \LR \right)^{-1}\LD \conj{\vF_2}
  \end{align}

  Puisque \(\LD\) (resp. \(\LR\)) commute avec lui-même, on a les égalités suivantes

  \begin{align}
    \LD\left(\conj{a_0} \oI + \conj{a_1} \LD \right)&=\left(\conj{a_0} \oI + \conj{a_1} \LD \right)\LD
    \\
    \LR\left(\conj{a_0} \oI - \conj{a_2} \LR \right)&=\left(\conj{a_0} \oI - \conj{a_2} \LR \right)\LR
  \end{align}

  Or puisque par définition \(\LD\LR=\LR\LD=0\), on a

  \begin{align}
    \LD\LR\conj{\vF_1} &= \LD\left(\conj{a_0} \oI + \conj{a_1} \LD \right)\vG_1
    \\
    0 & =\left(\conj{a_0} \oI + \conj{a_1} \LD \right)\LD\vG_1
  \end{align}

  Si l'on suppose alors que \(\Re(a_0) \ge 0 \) et \(\Re(a_1) \le 0\) (resp. \(\Re(a_2)\le0\)), alors \(\left(\conj{a_0} \oI + \conj{a_1} \LD \right)\) (resp. \(\left(\conj{a_0} \oI - \conj{a_2} \LR \right)\)) est injectif et donc on déduit que 

  \begin{align}
    \LD\vG_1 = 0
    \\
    \LR\vG_2 = 0
  \end{align}

  Or par définition \(\LR\conj{\vF_1} = \left(\conj{a_0} \oI + \conj{a_1} \LD \right)\vG_1\) (resp. \(\LR\conj{\vF_2} = \left(\conj{a_0} \oI - \conj{a_2} \LR \right)\vG_2\)) donc
  \begin{align}
    \LR\conj{\vF_1} &= \conj{a_0}\vG_1
    \\
    \LD\conj{\vF_2} &= \conj{a_0}\vG_2    
  \end{align}

  On réinjecte ce résultat dans la définition de \(X\)

  \begin{multline}
    X = \int_\Gamma z \vD \cdot \left(\conj{a_0} \oI + \conj{a_1} \LD - \conj{a_2}\LR\right)\conj{\vD}
    \\
    + \frac{b_1}{a_1} ||\vF_1||^2 - \frac{b_1\conj{a_2}}{a_1\conj{a_0}} \LR\conj{\vF_1}\cdot\vF_1
    \\
    + \frac{b_2}{a_2} ||\vF_2||^2 + \frac{b_2\conj{a_1}}{a_2\conj{a_0}} \LD\conj{\vF_2}\cdot\vF_2
  \end{multline}

  Les CSU sont alors

  \begin{minipage}{0.5\textwidth}
  \begin{align}
    \Re\left(\conj{a_0}z\right) \ge 0
    \\
    \Re\left(\conj{a_1}z\right) \le 0
    \\
    \Re\left(\conj{a_2}z\right) \le 0
    \\
    \Re\left(\frac{b_1}{a_1}\right) \ge 0
    \\
    \Re\left(\frac{b_2}{a_2}\right) \ge 0
  \end{align}
  \end{minipage}
  \begin{minipage}{0.49\textwidth}
  \begin{align}
    \Re\left(a_0\right) \ge 0
    \\
    \Re\left(a_1\right) \le 0
    \\
    \Re\left(a_2\right) \le 0
    \\
    \Re\left(\frac{b_1\conj{a_2}}{a_1\conj{a_0}}\right) \le 0
    \\
    \Re\left(\frac{b_2\conj{a_1}}{a_2\conj{a_0}}\right) \le 0
  \end{align}
  \end{minipage}