\section{Solutions du problème de Maxwell dans la sphère décomposées sur les harmoniques sphérique.}\label{sec:maxwell_harmonique}

Nous combinons donc les résultats des sections \ref{sec:sol_maxwell} et \ref{sec:helmholtz_scal} pour donner les solutions.

On se place à l'intérieur de la sphère: nous allons utiliser les solutions de Helmholtz définies à l'aide des fonctions de Bessel \(z_n(kr) = \gls{mat-jn}(kr)\).


\begin{TODO}
  Continuer à développer ces séries pour retomber sur la série de Mie
\end{TODO}

\subsection{Solutions du problème de Maxwell sur la sphère}
On déduit de  \eqref{eq:sol_maxwell:pot_hertz} et \eqref{eq:helm:sol_helmoltz_scal_somme}, que tout champs \((\vE,\vH)\) solutions du problème de Maxwell \eqref{eq:pres_th:intro:maxwell}, où \(\OO = B(0,R)\) se décomposent ainsi:

\(\exists (a_{m,n},b_{m,n}) \in (\CC)_{m,n}\times (\CC)_{m,n}\).
\begin{align*}
  \vE(\rtp) &= \sum\limits_{n=0}^\infty\sum\limits_{m=-n}^n\left( a_{m,n}   M_{m,n}(\rtp) - i \eta b_{m,n} N_{m,n}(\rtp)\right)\\
  \vH(\rtp) &= \sum\limits_{n=0}^\infty\sum\limits_{m=-n}^n\left( b_{m,n}   M_{m,n}(\rtp) + i {\eta}^{-1} a_{m,n} N_{m,n}(\rtp)\right)
\end{align*}

Soit \(\gls{mat-tild}(x) := \ddr{x}{\left(xu(x)\right)}\). Les vecteurs \(\gls{phy-Mmn} ,\gls{phy-Nmn}\) sont définis tels que:
\begin{align}
  M_{m,n}(\rtp) &:= \vrot \vect r \Psi_{m,n} \\
  &= z_n(kr)
  \begin{bmatrix}
    0 \\ \frac{im}{\sin\theta}Y_{m,n}\\
    - \ddr{\theta}{Y_{m,n}}
  \end{bmatrix}
\end{align}
\begin{align}
  N_{m,n}(\rtp) &:= \frac{\vrot M_{m,n}}{k} \\
  &= \frac{1}{kr}\begin{bmatrix}
    \frac{z_n(kr)}{\sin\theta}\ddr{\theta}{}\left(\sin\theta\ddr{\theta}{Y_{m,n}}\right)\\
    \tilde z_n(kr)\ddr{\theta}{Y_{m,n}}\\
    -\tilde z_n(kr)\frac{m^2}{\sin\theta}Y_{m,n}\\
  \end{bmatrix}
\end{align}


\subsection{Solution en tant qu'onde plane.}

Soit une onde plane incidente : \(\vE_i = E_0 e^{ikz}\vect e_x\). En coordonnée sphérique on a \(\vE = E_0e^{ikr\cos\theta}\left(\sin\theta\cos\phi\vect e_r + \cos\theta\cos\phi\vect e_\theta - \sin\phi \vect e \phi\right)\). On veut déterminer alors les coefficients \(a_{m,n}\), \(b_{m,n}\).

On remarque que \(\int_S e^{im\phi} \cos\phi \, ds = 0 = \int_S e^{im\phi} \sin\phi \,ds\; \forall m \not \in \left\lbrace -1,1 \right\rbrace\) donc \(\int_S M_{m,n} \cdot \vE = 0\). De façon analogue \(\int_S N_{m,n} \cdot \vE = 0.\)

On en déduit donc que les champs incidents se réécrivent

\begin{align*}
    \vE_i &= \sum\limits_{n=0}^\infty\left[
    a_{1,n}  M_{1,n} - i \eta b_{1,n} N_{1,n}
    + a_{-1,n}  M_{-1,n} - i \eta b_{-1,n} N_{-1,n}
    \right]\\
    \vH_i &= \sum\limits_{n=0}^\infty\left[
    b_{1,n}  M_{1,n} + i {\eta}^{-1} a_{1,n} N_{1,n}
    + b_{-1,n}  M_{-1,n} + i {\eta}^{-1} a_{-1,n} N_{-1,n}
    \right]
\end{align*}

% \iffalse

% Ceci étant vrai à l'intérieur de la sphère, raccorder avec les solutions à l'extérieur permet de déterminer les coefficients \(a_{m,n}\) et \(b_{m,n}\) \cite[eq.~(13.3.90) p.~1882]{morse_methods_1953}.

% \subsection{Prise en compte des conditions limite}
% Nous allons utiliser les condition limite à l'infini afin de déterminer les champs \((\E,\H)\) finaux.


% \subsection{Raccord notation Bruno}
% Dans les papiers de Bruno\cite{marceaux_high-order_2000} on a \(\vect \Phi_E = M_{m,n}, \rot \vect \Phi_E = k_0 N_{m,n}\)\\ \textbf{Remarque}: Il manque le coefficient \(\gamma_n^m\) dans \(\Psi\), cf \cite[p24, Th 2.4.4]{nedelec_acoustic_2001}.
% \subsection{De Helmholtz à Maxwell}

% Par définition, la divergence des champs est nulle. En représentant ainsi les potentiels comme issus d'un rotationnel, dans \eqref{eq:maxwell_potentiel}, les propriétés suivantes sont déduites, et l'on voit bien apparaître Helmholtz.
% \begin{equation}
%   \label{eq:pot_phi}
%   \left\lbrace
%     \begin{matrix}
%       \ddiv \vect \Phi_{X} &=& 0 \\
%       -\lapl  \vect \Phi_X - k^2 \vect \Phi_X &=& 0
%     \end{matrix}
%   \right.
%   \Rightarrow  \exists \vect \Phi_X,
%   \left\lbrace
%     \begin{matrix}
%       \vect \Phi_X &=& \rot \vect \Phi_X \\
%       \rot \left( \lapl \vect \Phi_X + k^2 \vect \Phi_X \right) &=& 0 \\
%     \end{matrix}
%   \right.
% \end{equation}
% %Nous allons maintenant montrer que l'on peut identifier l'un des \(\Phi_X\) à 0, et que cela correspond à choix de jauge, qui physiquement traduit le fait que l'on impose une onde radiale à l'infini, puisque de rotationnel nul.

% % \begin{equation}
% % \label{eq:maxwell_potentiel}
% % \left\lbrace \begin{matrix}
% % \E &=& \rot \left( \Phi_E + \frac{\vect \Phi_H}{iw\eps} \right) \\
% % \H &=& \vect \Phi_H - \frac{\rot \rot \Phi_E}{iw\mu} \\
% % \end{matrix}\right.
% % \end{equation}

% \subsection{Choix de jauge}

% On va exhiber les deux familles de solutions, liées au choix de la jauge.
% % Explicitons \(\vect \Phi_E = \rot \vect \Phi_E \) :
% % \[
% % \left\lbrace \begin{matrix}
% % \E &=& \rot \left( \Phi_E + \frac{\vect \Phi_H}{iw\eps} \right) \\
% % \H &=& \vect \Phi_H + \frac{\Delta \Phi_E}{iw\mu} \\
% % \end{matrix}\right.
% % \]

% \begin{prop}[Jauge \(\vect \Phi_H=0\) ]
% Si \(\vect \Phi_E\) est solution de \(\lapl \vect \Phi_E + k^2 \vect \Phi_E = 0\) alors,
% \[
%   \left\lbrace
%     \begin{matrix}
%       \E &=& \rot \Phi_E \\
%       \H &=& -iw\eps \Phi_E \\
%     \end{matrix}
%   \right.
% \]
% sont solutions du problème de Maxwell harmonique.
% \end{prop}

% On reprend la même méthode pour \(\vect \Phi_E\)

% \begin{prop}[Jauge \(\vect \Phi_E=0\) ]
% Si \(\vect \Phi_H\) est solution de \(\lapl \vect \Phi_H + k^2 \vect \Phi_H = 0\) alors,
% \[
% \left\lbrace
%   \begin{matrix}
%     \E &=& iw\mu \Phi_H \\
%     \H &=& \rot \Phi_H \\
%   \end{matrix}
% \right.
% \]
% sont solutions du problème de Maxwell harmonique.
% \end{prop}
% \fi