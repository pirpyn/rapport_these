\section{CSU pour les CIOE de \cite{stupfel_sufficient_2011}}
  La plupart des résultats de cette section ont déjà été montrés dans \cite{stupfel_sufficient_2011}.

  % On impose sur \(\Gamma\)  la relation \(\vE_t = Z \vJ\), où \(Z\) est un opérateur explicité ci-après.

  Pour tous \(\forall \vu, \vv \in (\mathcal C^\infty(\Gamma))\) des fonctions vecteurs complexes tangentes à la surface \(\Gamma\), on définit \(\LL\) un opérateur antisymétrique négatif, i.e :
  \begin{align*}
    \int_\Gamma \vu\cdot \LL(\conj{\vv}) &= \int_\Gamma \conj{\vv}\cdot \LL(\vu)\\
    \int_\Gamma \vu\cdot \LL(\conj{\vu}) &\le 0
  \end{align*}

  %%%%%%%%%%%%%%%%%%%%%%%%%%%%%%%%%%%%%%%%%%%%%%%%%%
  \subsection{Cas de la condition d'ordre 0}
    Utilisons la condition d’impédance d'ordre 0, valable sur la surface \(\Gamma\):

    Soit \(a_0 \in \CC\) tel que
    \[
      \vE_t = a_0 \vJ
    \]

    On a alors
    \begin{equation*}
    X = \conj{a_0}||\vJ||_{L_2(\Gamma)}^2
    \end{equation*}

    De \eqref{eq:unicite:form_var:cgu}, on déduit une \gls{acr-csu}:
    \begin{equation}
    \Re\left(a_0\right) \ge 0
    \end{equation}

  %%%%%%%%%%%%%%%%%%%%%%%%%%%%%%%%%%%%%%%%%%%%%%%%
  \subsection{Cas de la condition d'ordre 01}
    Utilisons la condition d’impédance d'ordre 01, valable sur la surface \(\Gamma\):

    Soit \((a_0, a_1) \in \CC\times\CC\) tel que

    \[
      \vE_t = (a_0 +a_1 \LL)\vJ
    \]
    On pose:
    \begin{align*}
      F&:= || \vJ|| ^2 \ge 0  & G&:= -\int_\Gamma \vJ\cdot \LL\conj{\vJ} ds \ge 0
    \end{align*}

    On a alors
    \begin{equation*}
      X = \conj{a_0}F - \conj{a_1}G
    \end{equation*}

    De \eqref{eq:unicite:form_var:cgu}, on déduit des CSU suivantes:
    \begin{align}
      \Re\left(a_0\right) \ge 0\\
      \Re\left(a_1\right) \le 0
    \end{align}
  %%%%%%%%%%%%%%%%%%%%%%%%%%%%%%%%%%%%%%%%%%%%%%%%
  \subsection{Cas de la condition d'ordre 1}

    Utilisons la condition d’impédance d'ordre 1, valable sur la surface \(\Gamma\):

    Soit \((a_0, a_1,b) \in \CC^3\) tel que
    \[
      \vE_t = (1 + b \LL)^{-1} (a_0 + a_1 \LL) \vJ
    \]

    \paragraph{Première méthode (\cite{stupfel_sufficient_2011})}~

      On utilise l'identité \((a_1-a_0\conj{b}) = (a_1(1+\conj{b}\LL) - \conj{b}(a_0+a_1\LL))\):

      \begin{align*}
        (a_1-a_0\conj{b})X &= \int_\Gamma \left(a_1(1+\conj{b}\LL) \vJ\right)\cdot\conj{\vE_t} - \left(\conj{b}(a_0+a_1 \LL)\vJ\right)\cdot\conj{\vE_t} ds\\
        &= \int_\Gamma \left(a_1(1+\conj{b}\LL) \conj{\vE_t}\right)\cdot\vJ ds - \int_\Gamma \left(\conj{b}(a_0+a_1 \LL)\vJ\right)\cdot\conj{\vE_t} ds\\
        &= \int_\Gamma \left(a_1(\conj{a_0}+\conj{a_1}\LL) \conj{\vJ}\right)\cdot\vJ ds  - \int_\Gamma \left(\conj{b}(1+b \LL)\vE_t\right)\cdot\conj{\vE_t} ds\\
        &= a_1\conj{a_0} ||\vJ||^2 + |a_1|^2 \int_\Gamma \vJ \LL \conj{\vJ} ds - \conj{b} ||\vE_t||^2 - |b|^2 \int_\Gamma \vE_t \LL \conj{\vE_t} ds
      \end{align*}

      On note \(\Delta = a_1 -a_0\conj{b}\), \(F = -\int_\Gamma \vJ \LL \conj{\vJ} ds \ge 0 \), \(G = -\int_\Gamma \vE_t \LL \conj{\vE_t} ds \ge 0 \) .

      Si on décompose les parties réelles et imaginaires de cette expression, on a
      \begin{align*}
        \Re(\Delta)\Re(X) - \Im(\Delta)\Im(X) &= \Re(a_1\conj{a_0}) ||\vJ||^2 - \Re(\conj{b})||\vE_t||^2 -|a_1|^2 F + |b|^2 G \\
        \Im(\Delta)\Re(X) + \Re(\Delta)\Im(X) &= \Im(a_1\conj{a_0}) ||\vJ||^2 - \Im(\conj{b})||\vE_t||^2
      \end{align*}
      La première relation nous empêche de conclure sur le signe de \(\Re(X)\) car il y des signes différents entre les deux derniers termes, sauf si nous imposons \(\Re( \Delta)= 0\) auquel cas, nous pouvons conclure grâce à la deuxième relation. Les CSU sont alors

      \begin{align}
        &\Re(\Delta) &= 0\\
        &\Im(\Delta)\Im(b) &\ge 0\\
        &\Im(\Delta)\Im(a_1\conj{a_0})&\ge 0
      \end{align}

    \paragraph{Deuxième méthode}
      ~
      \subparagraph{Cas \(a_1\not=0\)}
        ~

        En supposant \(a_1 \not=0\), on utilise l'identité \((a_0 + a_1 \LL)^{-1}(1 + b \LL)  = \frac{b}{a_1} I_d + \left(1-b\frac{a_0}{a_1}\right)(a_0+a_1 \LL)^{-1}\):
        \[
          X = \int_\Gamma \left(\left(\frac{b}{a_1} I_d + \left(1-b\frac{a_0}{a_1}\right)(a_0+a_1 \LL)^{-1}\right)\vE_t\right) \cdot \conj{\vE_t} ds
        \]

        On pose:
        \begin{align*}
          \vect D &:= (a_0 + a_1 \LL)^{-1}\vE_t & F&:= || \vect D || ^2 \ge 0  \\
          G&:= -\int_\Gamma \vect D \cdot \LL\conj{\v{D}} ds \ge 0 & H &:= || \vE_t || ^2 \ge 0
        \end{align*}
        Comme \(\conj{E_t} = (\conj{a_0} + \conj{a_1}\LL)D\) alors \(\ds\int_\Gamma (a_0 +a_1 \LL) ^{-1}\vE_t\cdot \conj{\vE_t} ds = \conj{a_0} F - \conj{a_1} G\) et l'on peut alors écrire

        \begin{equation}
          \label{eq:unicite:form_var:decomp_cgu_ci1_a1}
          X = \frac{b}{a_1}H   + \left(1-b\frac{a_0}{a_1}\right)\left(\conj{a_0} F - \conj{a_1} G\right)
        \end{equation}
        De \eqref{eq:unicite:form_var:cgu}, on déduit des CSU suivantes:
        \begin{align}
          \Re\left(\frac{b}{a_1}\right) \ge 0 \\
          \Re\left(a_0\right) \ge 0 \\
          \Re\left(\left(a_1-a_0 b\right)\frac{\conj{a_1}}{a_1}\right) \le 0
        \end{align}

      \subparagraph{Cas \(a_1=0\)}
        ~
        \[
          X = \int_\Gamma \left( \frac{1}{a_0}\left(1+b\LL\right)\vE_t\right) \cdot \conj{\vE_t} ds
        \]

        On pose:
        \begin{align*}
          F&:= \int_\Gamma | \vE_t | ^2 ds \ge 0 & G &:= -\int_\Gamma \vE_t \cdot \LL\conj{\vE_t} ds \ge 0
        \end{align*}

        On a alors
        \begin{equation}
          \label{eq:unicite:form_var:decomp_cgu_ci1_a1_nul}
          X = \frac{1}{a_0}F - \frac{b}{a_0}G
        \end{equation}

        De \eqref{eq:unicite:form_var:cgu}, on déduit des CSU suivantes:
        \begin{align}
          a_0 \not= 0\\
          \Re\left(a_0\right) \ge 0\\
          \Re\left(b\conj{a_0}\right) \le 0
        \end{align}

\subsection{Autres CSU}
  Méthode \cite{stupfel_implementation_2015}

  On cherche à résoudre le problème suivante:
  \[
    \begin{bmatrix}
      1 & \conj{b} \\
      a_0 & a_1
    \end{bmatrix}
    \begin{bmatrix}
      \int_\Gamma \vJ \cdot \conj{\vE_t} \\
      \int_\Gamma \vJ\cdot \LL\conj{\vE_t}
    \end{bmatrix}
    =
    \begin{bmatrix}
      \conj{a_0}||\vJ||_2^2 + \conj{a_1}\int_\Gamma \vJ \cdot \LL\conj{\vJ} \\
      ||\vE_t||_2^2 + b\int_\Gamma \conj{\vE_t} \cdot \LL\vE_t
    \end{bmatrix}
  \]

  Si la matrice est inversible, on pose \(\Delta = a_1 - a_0\conj{b}\) et alors
  \[
  \begin{bmatrix}
    \int_\Gamma \vJ \cdot \conj{\vE_t} \\
    \int_\Gamma \vJ\cdot \LL\conj{\vE_t}
  \end{bmatrix}
  =\frac{1}{\Delta}
  \begin{bmatrix}
    a_1 & -\conj{b} \\
    -a_0 & 1
  \end{bmatrix}
  \begin{bmatrix}
    \conj{a_0}||\vJ||_2^2 + \conj{a_1}\int_\Gamma \vJ \cdot \LL\conj{\vJ} \\
    ||\vE_t||_2^2 + b\int_\Gamma \conj{\vE_t} \cdot \LL\vE_t
  \end{bmatrix}
  \]
  Donc
  \begin{align*}
    X &=  \frac{a_1\conj{a_0}}{\Delta}||\vJ^2||_2^2 - \frac{\conj{b}}{\Delta}||\vE_t||_2^2 \\
    &~+\frac{|a_1|^2}{\Delta}\int_\Gamma\vJ\cdot \LL \conj{\vJ} - \frac{|b|^2}{\Delta}\int_\Gamma\conj{\vE_t}\cdot \LL\vE
  \end{align*}
  Les CSU sont alors

    \begin{align}
    \Re\left(a_0\conj{a_1}\Delta\right) &\ge 0\\
    \Re\left(b\Delta\right) &\le 0\\
    \Re\left(|a_1|^2\Delta\right) &\le 0\\
    \Re\left(|b|^2\Delta\right) &\ge 0
  \end{align}
  Si la matrice n'est pas inversible, alors on cherche à résoudre
  \[
    \begin{bmatrix}
      1 & \conj{b} \\
      a_0 & a_0\conj{b}
    \end{bmatrix}
    \begin{bmatrix}
      \int_\Gamma \vJ \cdot \conj{\vE_t} \\
      \int_\Gamma \vJ \cdot \LL\conj{\vE_t}
    \end{bmatrix}
    =
    \begin{bmatrix}
      \conj{a_0}||\vJ||_2^2 + \conj{a_1}\int_\Gamma \vJ \cdot \LL\conj{\vJ} \\
      ||\vE_t||_2^2 + b\int_\Gamma \conj{\vE_t} \cdot \LL\vE_t
    \end{bmatrix}
  \]

  Le noyau de la matrice est alors \(\Vect{\begin{bmatrix}\conj{b}\\-1\end{bmatrix}}\) dont l'orthogonal est  \(\Vect{\begin{bmatrix}1\\\conj{b}\end{bmatrix}}\).
  Pour tout \(\int_\Gamma \vJ\cdot \LL\conj{\vE_t} = \conj{b} \int_\Gamma \vJ \cdot \conj{\vE_t} \), on a unicité des solutions. On déduit alors que

  \[
    (1 + \conj{b}^2) X = \conj{a_0} ||\vJ||_2^2 + \conj{a_1}\int_\Gamma \vJ \cdot \LL\conj{\vJ}
  \]

  Les CSU sont alors

  \begin{align}
    \Re\left(\frac{\conj{a_0}}{1 + \conj{b}^2}\right) &\ge 0 \\
    \Re\left(\frac{\conj{a_1}}{1 + \conj{b}^2}\right) &\le 0
  \end{align}

  % \begin{proof}
  % Les 3 CSU originales sont

  % \(\hfill
  % \bullet \Re\left(\frac{b}{a_1}\right) \ge 0 \hfill
  % \bullet \Re\left(\left(1-a_0\frac{b}{a_1}\right)\conj{a_0}\right) \ge 0 \hfill
  % \bullet \Re\left(\left(1-a_0\frac{b}{a_1}\right)\conj{a_1}\right) \le 0
  % \hfill\)

  % On va séparer la deuxième CSU et faire apparaître la première CSU dont on va utiliser le résultat pour simplifier le résultat.
  % \begin{align*}
  % \Re\left(\left(1-a_0\frac{b}{a_1}\right)\conj{a_0}\right)& \ge 0 \\
  % \Re\left(\conj{a_0}\right) &\ge \Re\left(|a_0|^2\frac{b}{a_1}\right)\\
  % \Re\left(a_0\right) &\ge 0
  % \end{align*}
  % \end{proof}
  %%%%%%%%%%%%%%%%%%%%%%%%%%%%%%%%%%%%%%%%%%%%%%%%%%%%%%%%%%%%%%%%%%%%%%%%%%%%%%%%%%%%%%%%%%%%%%%%%%%%%%%%%
