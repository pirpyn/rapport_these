\section{Condition suffisante d'unicité générale}

  %Considérons le problème énoncé dans \cite{stupfel_sufficient_2011} (convention \(e^{i\omega t}\)).
  Soit une \(\Gamma\) une surface régulière fermée dans \(\RR^3\).
  À l'extérieur de cette surface, on définit \(k_0 \in \RR_+^*\), le nombre d'onde dans le vide et \(\eta_0 \in \RR^*\) l'impédance du vide.
  À l'intérieur, on définit les constantes relatives \(\eps,\mu \in \CC\), invariantes par translation.

  \begin{tcolorbox}
    \centering
    La dépendance temporelle est en \(e^{i\w t}\), et l'on identifie \(\vH \equiv \eta _0 \vH\).
  \end{tcolorbox}

  Soit \((\vE,\vH)\) dans \((\Hrot(\OO)) \times (\Hrot(\OO))\).

  Alors \((\vE,\vH)\) sont solutions de Maxwell si
  \begin{align}
    \label{eq:unicite:form_var:maxwell-E}
    \trot \vE + ik_0 \mu \vH &= 0 && \text{dans \(\OO\)}
    \\\label{eq:unicite:form_var:maxwell-H}
    \trot \vH - ik_0\eps \vE &= 0 && \text{dans \(\OO\)}
    % \\\label{eq:unicite:form_var:TR}
    % \Tr(\vE_t) &= - \vn_{Y_R} \pvect \vH && \text{sur \(\Gamma(0,R)\)}
  \end{align}
  % Où \(\Tr\) est l'opérateur de capacité \cite[p.~200]{nedelec_acoustic_2001}, \(\vn_{Y_R}\) la normale unitaire sortante à \(\Gamma(0,R)\).\\

  Les relations \eqref{eq:unicite:form_var:maxwell-E} et \eqref{eq:unicite:form_var:maxwell-H} permettent grâce aux formules de Green d'obtenir la forme variationnelle suivante :
  Trouver \(\vE \in \left(\Hrot(\OO)\right)\), \(\forall \vect \phi \in \left(\Hrot(\OO)\right)\)
  \[
    a(\vE,\vect\phi) = 0
  \]
  où a est une forme sesquilinéaire telle que, soit \(\vn\) la normale unitaire sortante à \(\Gamma\).
  \begin{align*}
    a(\vE,\vect\phi) &:=  \frac{1}{-ik_0\mu} \int_\OO \trot \vE \cdot \trot \conj{\vect{\phi}} dx + -ik_0\eps\int_\OO\vE\cdot\conj{\vect{\phi}} dx
     %+ \int_{Y_R} \conj{ \vect \phi } \cdot \Tr(\vE_t)ds
     - \int_\Gamma \left(\vn \pvect \frac{\trot \vE}{-ik_0\mu}\right) \cdot \conj{\vect \phi} ds \\
   \end{align*}

  \begin{defn}[Coercivité]
    Une forme sesquilinéaire \(a(\vu,\vv)\) est coercive dans \(\Hrot(\OO)\) si \(\exists \alpha > 0\) tel que
    \[
      |\Re(a(\vu,\vu))| \ge \alpha ||\vu||_{\Hrot(\OO)}^2 = \alpha \left( || \trot \vu ||_{L^2}^2 + || \vu ||_{L^2}^2\right) \, \forall \vu \in \Hrot(\OO)
    \]
   \end{defn}

  %La forme bilinéaire \(a\) est coercive \Gamma'il existe une constante réel positive \(\mathcal{C}\) telle que \(|a(\vE,\vE)|^2 \ge \mathcal{C}  || \vE ||_{\Hrot}^4 = \mathcal{C}\left( || \trot \vE ||_{L^2}^2  + || \vE ||_{L^2}^2 \right)^2 \).

  %Supposons \(\eps, \mu\) constants et
  Posons les définitions suivantes:
  \begin{align*}
    % X&:= \int_\OO | \trot \vE | ^2 dx  =|| \trot \vE ||_{L^2}^2
    % \\B&:= \int_\OO | \vE | ^2 dx  = || \vE ||_{L^2}^2
    %\\CC&:= \int_{Y_R} \conj{E_t}\cdot T_R \vE_t ds
    \vJ &:=  \vn \pvect \frac{\trot \vE}{-ik_0\mu} && \text{sur \(\Gamma\)}
    \\
    X&:= \int_\Gamma \vJ \cdot \conj{\vE_t} ds
  \end{align*}
  La partie réelle de la forme bilinéaire \(a\) s'écrit donc
  \begin{equation}
    \label{eq:unicite:form_var:decomp_form_bilin_1}
    |\Re(a(\vE,\vE))| = \left|\frac{\Im(\mu)}{k_0} || \trot \vE ||_{L^2}^2  + k_0 \Im(\eps) || \vE ||_{L^2}^2
    %+ \Re(C)
    - \Re(X)\right|
  \end{equation}

  \begin{hyp}[Hypothèses de coercivité en convention \(i\omega t\)]\label{hyp:unicite:form_var:hyp_coercivite}
    ~{}

    \begin{enumerate}
      \item \(\Im(\eps)\) et \(\Im(\mu)\) sont de même signe.
      \item \(\Im(\eps)\) et \(\Re(X)\) sont de signes opposés\footnote{En convention \(-i\omega t\), il faut que le signe soit le même, et donc \(\Re(X) \le 0\).}.
      %\item \(\Re(C)\) et \(\Re(X)\) sont de même signe.
    \end{enumerate}
  \end{hyp}

\subsection{Cas des matériaux avec pertes}

  Si l'hypothèse \ref{hyp:unicite:form_var:hyp_coercivite} est vérifiée alors \(a\) est coercive ( \(\alpha = \min(-\Im(\mu)k_0^{-1},-\Im(\eps)k_0)\)) donc il y a unicité des solutions du problème de Maxwell avec conditions d'impédance.

  Comme en convention \(e^{i\omega t}\), le signe de \(\Im(\eps)\) et \(\Im(\mu)\) est négatif
  %, sachant que d'après \cite[p.~97]{nedelec_acoustic_2001} \(\Re(C)\ge 0\)
  alors l'unicité est assurée par la
  \begin{defn}[\gls{acr-cgu}]~\\
    \begin{equation}\label{eq:unicite:form_var:cgu}
      \Re(X) = \Re\left(\int_\Gamma \vJ \cdot \conj{\vE_t} ds\right) \ge 0
    \end{equation}
  \end{defn}

\subsection{Cas des matériaux sans pertes}

  Si \(\Im(\mu) = \Im(\eps) = 0\), le résultat précédent n'est plus valable car \(\alpha = 0\).

  \begin{REF}
      Fredholm
  \end{REF}
