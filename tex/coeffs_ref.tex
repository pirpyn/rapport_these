\section{Opérateur d'impédance dans le cas du plan infini}\label{sec:coeffs_ref}

On cherche à résoudre le problème de Maxwell, avec condition de rayonnement à l'infini. Si cette dernière permet d'avoir des solutions qui soient rapidement décroissante à l'infini, alors on peut définir les transformées de Fourier des champs électromagnétiques \cite[p.~146]{yosida_functional_1995}. Grâce à l'égalité de Parseval-Plancherel \cite[p.~153]{yosida_functional_1995}, on peut alors travailler sur les transformées de Fourier, ce qui permet de passer des opérateurs différentielles à des opérateurs multiplicatifs.

\TODO{Simplifier l'analyse spectrale. Ex: se baser sur \cite[p.~28]{hoppe_impedance_1995}, mais reformuler avec les espaces $L^2$,etc}


Soit un système de coordonnées $(\x,\y,\z)$ tel que le demi-espace des $\z>0$ soit le vide, où une onde est de célérité \gls{phy-c}, de fréquence $\gls{phy-f}$ et de nombre d'onde $\kO=\frac{2\pi f}{c}$  et d'impédance scalaire $\etaO = \sqrt{\frac{\muO}{\epsO}}$ ( de plus $\kO=2\pi f\sqrt{\muO\epsO}$ ).

On définit en tous point la permittivité diélectrique relative $\gls{phy-eps}_p$ et la perméabilité magnétique relative $\gls{phy-mu}_p$, qui sont invariantes par translation dans le matériau.
Ces constantes adimensionnées peuvent être complexes\footnote{Dans le vide $\eps_0=\mu_0=1.$}.

Le demi-espace $\z<-d$ restant est assimilé à un \gls{acr-cp}.

On utilisera les notations  suivantes dans la suite de cette partie: le vide sera indiqué comme milieu d'indice $0$ , le matériau comme milieu d'indice $1$.
\renewcommand{\x}{{x}}
\renewcommand{\y}{{y}}
\renewcommand{\z}{{z}}
\renewcommand{\peps}{{\eps_1}}
\renewcommand{\pmu}{{\mu_1}}
\begin{figure}[h!]
\centering
\begin{tikzpicture}
\coordinate (l) at (-2,0);
\coordinate (r) at (2,0);
\coordinate (m) at ($(r)!0.5!(l)$);

\fill [lightgray] (l) rectangle (2,1) ;
\foreach \t in {-2.2,-2.1,...,2} {
\draw plot[domain=0:0.3] (\t + \x, -\x, 0);
}
\coordinate (ll) at (-2.2,-0.3);
\fill [white] (ll) rectangle (l) ;
\coordinate (rr) at (2,-0.31);
\fill [white] (rr) rectangle (2.3,0) ;


\coordinate (n) at (-4,0);

\coordinate (lt) at (-2,1);
\coordinate (rt) at (2,1);
\coordinate (mt) at ($(rt)!0.5!(lt)$);

\draw (l) -- (r) node [at end,right] {$z = -d$};
\draw (lt)  -- (rt) node [at end,right] {$z = 0$};

\draw (lt) node [above right] {$\eps_0,\mu_0$};
\draw ($(l)!0.5!(lt)$) node [right] {$\eps_1,\mu_1$};

\draw [->] (n) -- ++(0,1) node [at end, right] {$z$};
\draw [->] (n) -- ++(1,0) node [at end, right] {$x$};

\draw (n) circle(0.1cm) node [below=0.1cm] {$y$};
\draw (n) +(135:0.1cm) -- +(315:0.1cm);
\draw (n) +(45:0.1cm) -- +(225:0.1cm);

%\draw [->>,thick] (lt) ++ (1,1) -- (mt) ;


\end{tikzpicture}
\caption{schéma du plan infini}
\end{figure}\label{fig:tikz:plan}

\subsection{Objectif}
On cherche à obtenir l'opérateur différentiel \gls{ope-imp}, tel que l'on ait la relation,

\begin{equation}
\label{eq:coeff_ref:intro:ci}
-\vn \pvect \vn \pvect \vE(\x,\y,0) = \mathcal{Z}( \vn \pvect \vH)(\x,\y,0)
\end{equation}

%Le premier cas d'étude est celui du plan infini. Il permet de résoudre facilement le problème de Maxwell \eqref{eq:maxwell} et de déterminer l'expression de la la condition d'impédance (\gls{acr-ci}).
%%%% SECTION HYPOTHESES
\begin{hyp}[Hypothèses simplificatrices]{}~\\
\begin{enumerate}
    \item On suppose que les solutions s'écriront comme $c e^{i(k_\x\x +k_\y\y+k_\z\z)}, c\in\CC$.
    \item Comme le plan est infini et quitte à redéfinir les axes cartésiens, on peut supposer que l'onde se propage suivant le vecteur $\v k_p = (k_{\x,p},0,k_{\z,p})$ : pas de dépendance en $\y$ et $k_\y$.
\end{enumerate}
\end{hyp}

On note $\eta_p = \sqrt{\frac{\mu_p}{\eps_p}}$ et $\nu_p = \sqrt{\mu_p \eps_p}$.
On a donc $k_p = \kO\nu_p$\footnote{Évidemment $k_0 = \kO$} et $k_{z,p} = \sqrt{k_p^2 - k_{x,p}^2}$.

\begin{hyp}[Principe de superposition]{}~\\
La solution générale est une combinaison linéaire de deux problèmes: 
\begin{enumerate}
    \item \gls{acr-te} : $\vE = (0,E_y,0)$.
    \item \gls{acr-tm} : $\vH = (0,H_y,0)$. 
\end{enumerate}
\end{hyp}

\subsection{Solutions}

\begin{tcolorbox}
\centering
La dépendance temporelle est en $e^{i \omega t}$, et l'on identifie $\vH \equiv \etaO \vH$.
\end{tcolorbox}


Soit $(\vE,\vH)$ dans $(\Hrot(\OO))^3 \times (\Hrot(\OO))^3$. 

Alors $(\vE,\vH)$ sont solutions de Maxwell si 
\begin{align}
\label{eq:coeffs_ref:maxwell-E}
\trot \vE + i k_p \eta_p \vH &= 0 && \text{dans $\OO$}\\
\label{eq:coeffs_ref:maxwell-H}
\trot \vH - i \frac{k_p}{\eta_p} \vE &= 0 && \text{dans $\OO$}
\end{align}

Les équations de Maxwell donnent dans chacun des domaines (indicé par le $p = \lbrace0,1\rbrace$) et pour chacune des polarisations:

\Gls{acr-te}: $ \vE(\x,\z) = (0,E_y,0), \vH(\x,\y) = (H_x,0,H_z)$. 
\[
\left\lbrace 
\begin{matrix}
-\dd_{\z} E_\y + i k_p \eta_p H_\x & = & 0 \\
\dd_{\x} E_\y + i k_p \eta_p H_\z &= & 0 \\
\dd_{\z} H_\x- \dd_{\x} H_\z - i\frac{k_p}{\eta_p} E_\y & = & 0\\
\end{matrix}
\right.
\Rightarrow
\left\lbrace
\begin{matrix}
\dd_{\z\z}^2 E_\y + \dd_{\x\x}^2 E_\y + k_p^2 E_\y = 0\\
H_\x = \frac{\dd_{\z} E_\y}{i k_p \eta_p }\\
H_\y =  -\frac{\dd_{\x} E_\z}{i k_p \eta_p }
\end{matrix}
\right.
\]

\Gls{acr-tm} : $ \vE(\x,\z) = (E_\x,0,E_\z), \vH(\x,\z) = (0,H_\y,0)$. 
\[
\left\lbrace 
\begin{matrix}
\dd_{\z} E_\x - \dd_{\x} E_\z + i k_p \eta_p  H_\y &= & 0 \\
- \dd_{\z} H_\y - i\frac{k_p}{\eta_p} E_\x & = & 0\\
\dd_{\x} H_\y -i\frac{k_p}{\eta_p} E_\z &=&0\\
\end{matrix}
\right.
\Rightarrow
\left\lbrace
\begin{matrix}
\dd_{\z\z}^2 H_\y + \dd_{\x\x}^2 H_\y + k_p^2 H_\y = 0\\
E_\x = -\eta_p\frac{\dd_{\z} H_\y}{i k_p}\\
E_\y = \eta_p\frac{\dd_{\x} H_\y}{i k_p}
\end{matrix}
\right.
\]

On obtient alors une équation de Helmholtz à résoudre respectivement pour $E_\y$ et $H_\y$.

Dans la suite, nous allons donc considérer comme inconnu $U$, solution du problème suivant : 

\begin{equation}
\dd_{\z\z}^2 U + \dd_{\x\x}^2 U + k_p^2 U = 0 \quad \text { dans $\OO\cup\OO^c$}
\end{equation}

% \begin{hyp}[Hypothèses simplificatrices 2]{Passage dans le domaine de Fourier}\\
%   Par définition d'une onde plane $U(x_1,x_2,\z)$, on a la propriété suivante sur sa transformée de Fourier $ U(k_{1,i},k_{2,i},k_{3,i})$:
% \[
%  \dd_j { U}(\v \kO) = ik_{j,i}  U(\v \kO)
% \]
% \end{hyp}

% 
% \iffalse
% \subsection{Onde plane et transformée de Fourier}
% Une onde plane harmonique (dépendance $e^{i\wt}$) est de type $U(x_1,x_2,\z) = U_0e^{i(k_1x_1+k_2x_2+k_3\z)}$, dont on remarque que . 
% 
% Or une propriété de la transformée de Fourier $ U$ de $U$ est ${ \dd_j U}(\xi) = i\xi_j  U$. 
% Donc on peut faire l'analogie entre l'hypothèse d'une onde plane et la transformée de Fourier de $U$,
%  avec l'analogie $\xi_j \leftrightarrow  k_j$.
% \fi
% %%%% SECTION Z
% \subsection{Impédance de surface pour une couche de matériau}
% Comme le plan est infini, on peut prendre les transformées de Fourier des 2 équations trouvées précédemment afin que les opérateur différentiels issus des équations de Maxwell deviennent des coefficients multiplicatifs. On peut alors rechercher une CI tel que $\mathcal{Z} = Z\operatorname{Id}$ et on note $ \kO = \w^2 \eps_p \mu_p$.
% 
% Rappellons que dans le domaine de Fourier, ${ \dd_j U}(\xi) = i\xi_j  U$ %(on rappelle que $k_{\z,1} \leftrightarrow k_1$)

%%% OLD FOURIER

% Comme $(\E,\H)$ sont dans $L_2(\OO)$, ils admettent une transformée de Fourier $( \E,  \H)$ dans $L^2$ et l'on a

% \[
%   U(\x,\y,\z) = \frac{1}{2\pi} \int_{\RR^3}  U (k_\x,k_\y,k_\z) e^{i(k_\x\x +k_\y\y+k_\z\z)} dk_\x dk_\y dk_\z
% \]

% Le problème que l'on veut résoudre est maintenant le suivant:
% \begin{equation}
%   \dd_{\z\z}^2  U + k_{\z,p}^2  U = 0 \quad \text { dans $\OO\cup\OO^c$}
% \end{equation}

% Et dont les solutions de l'équation ci-dessus sont

% \[
% \hfill U_p(k_z\z) = A_p e^{-ik_{\z,p} \z}  + B_p e^{ik_{\z,p} \z}
% \]

% On suivra comme convention qu'en polarisation TE, $ U =  E_\y$ et en polarisation TM, $ U =  H_\y$, l'autre champ découlant des équations de Maxwell.

Grâce à l'hypothèse sur la forme des solutions

\[
\dd_{\x} U(\x,\z) = ik_\x U(\x,\z)
\]

Le problème que l'on veut résoudre est maintenant le suivant:
\begin{equation}
\dd_{\z\z}^2 U + k_{\z,p}^2 U = 0 \quad \text { dans $\OO\cup\OO^c$}
\end{equation}

Et dont les solutions de l'équation ci-dessus sont

\[
\hfill U(\z) = A_p e^{-ik_{\z,p} \z}  + B_p e^{ik_{\z,p} \z}
\]

On suivra comme convention qu'en polarisation TE, $ U = E_\y$ et en polarisation TM, $ U = H_\y$, l'autre champ découlant des équations de Maxwell.


Les conditions limites sur le matériau vont permettre de déterminer les valeurs des constantes $A_p, B_p$.
\begin{itemize}
    \item 
    Conducteur parfait (\gls{acr-cp})\footcite[p.~217]{morse_methods_1953} : $\vE_t(-d) = 0  $
    \item 
    Continuité des champs tangents: $
    \left\lbrace 
    \begin{matrix}
    [ \vE_t]_{\z=0} &=& 0 \\
    [ \vH_t]_{\z=0} &=& 0 
    \end{matrix}
    \right.$
\end{itemize}

%%%  TE
Nous allons alors calculer les champs tangents, utiliser les conditions limites puis la condition d'impédance pour obtenir le multiplicateur de Fourier associé à l'opérateur d'impédance.

\TODO{Trouver quelque chose de mieux que le tableau. Arrêter TE, TM => cas général 3D ?}

\begin{center}
\begin{tabular}{| C | C | C |}
\hline
& \text{\textbf{TE}} & \text{\textbf{TM}} \\
\hline\hline

\vE_t & \left(A_p e^{-ik_{\z,p} \z}  + B_p e^{ik_{\z,p} \z}\right) \v e_2 &  \frac{k_{\z,p}\eta_p}{k_p}\left( A_p e^{-ik_{\z,p} \z} - B_p e^{ik_{\z,p}z}\right) \v e_1\\
\hline

\vH_t &\frac{k_{\z,p}}{k_p\eta_p} \left(-A_p e^{-ik_{\z,p} \z}  + B_p e^{ik_{\z,p} \z}\right) \v e_1 & \left(A_p e^{-ik_{\z,p} \z} + B_p e^{ik_{\z,p} \z}\right) \v e_2\\
\hline

\vE_t(-d) = 0 & A_1 = -B_1e^{-ik_{\z,1}2d} &  A_1 = B_1e^{-ik_{\z,1}2d}\\
\hline

\multirow{2}{*}{$\vE_t(0)=Z\vn\times \vH(0)$} & B_1\left(1-e^{-ik_{\z,1}2d}\right) & B_1\left( 1 - e^{-ik_{\z,1}2d} \right)\frac{k_{\z,1}\eta_1}{k_1}  \\
& =  Z\frac{k_{\z,1}}{k_1\eta_1}B_1\left(1+e^{-ik_{\z,1}2d}\right) & = Z\left(e^{-ik_{\z,1}2d} + 1\right)B_1\\
\hline
\hline
Z & \frac{k_1\eta_1}{k_{\z,1}}i\tan(k_{\z,1}d) & \frac{k_{\z,1}\eta_1}{k_1}i\tan(k_{\z,1}d) \\
\hline
\end{tabular}
\end{center}

\subsection{Matrice diagonale d'impédance dans le domaine de Fourier}

L'opérateur $Z$ est la matrice $diag(Z_{TE},Z_{TM})$:
\begin{align}
\notag Z&=\dfrac{i\eta_1}{k_1^2}\frac{k_1\tan(k_{\z,1}d)}{k_{\z,1}}(k_1^2 I - L_R ) \\
&=\dfrac{i\eta_1}{k_1^2}P(k_1^2 I - L_R )\label{eq:coeff_ref:mat_z:imped}
\end{align}

où $L_R = diag(0,k_{\x,1}^2)$. C'est la forme que l'on retrouve dans \cite{marceaux_high-order_2000}.

%%%% SECTION REFLEXION
\subsection{Coefficients de réflexions}
Le rapport $R=B_0/A_0$ est appelé coefficient de réflexion. On cherche à l'exprimer en fonction de l'impédance de surface $Z$. On réutilise les équations de continuités: 

\begin{center}
\begin{tabular}{| C | C | C |}
\hline
& \text{\textbf{TE}} & \text{\textbf{TM}} \\
\hline\hline

1+R & \frac{B_1(1-e^{-ik_{\z,1}2d})}{A_0}& \frac{B_1(1+e^{-ik_{\z,1}2d})}{A_0} \\ 
\hline

-1+R & \frac{k_{\z,1}\mu_0 B_1(1+e^{-ik_{\z,1}2d})}{k_{\z,0} \mu_1 A_0} & \frac{k_{\z,1}\eps_0 B_1(1-e^{-ik_{\z,1}2d})}{k_{\z,0} \eps_1A_0}\\
\hline
\multirow{2}{*}{$\frac{1+R}{-1+R}$}  & -\frac{k_{\z,0} \mu_1}{k_{\z,1} \mu_0}i\tan(k_{\z,1} d) & -\frac{k_{\z,0} \eps_1}{k_{\z,1}\eps_0}\frac{1}{i\tan(k_{\z,1}d)}\\
& =\frac{k_{\z,0}Z_{TE}}{k_0\eta_0} & =\frac{k_{\z,0}\eta_0}{k_0Z_{TM}}\\
\hline
\hline
R & \frac{k_{z,0}Z_{TE}-1}{k_{z,0}Z_{TE}+1} & \frac{k_{z,0}-Z_{TM}}{k_{z,0}+Z_{TM}} \\
\hline
\end{tabular}
\end{center}

\subsection{Ondes évanescentes}

On note $s=\frac{k_{x,p}}{k_p}$. Si $s>1$ alors $k_{z,p} = k_p\sqrt{\nu_p^2-s^2}$ devient complexe\footnote{Sinon on peut noter $s=\sin(\theta_{inc})$}.
La partie imaginaire de ce complexe crée une exponentielle réelle dans la solution générale. On obtient donc des ondes exponentiellement décroissante dans la direction des $\z$ croissant. Ces ondes sont appelées évanescentes.

%%% SECTION MULTICOUCHE
\subsection{Cas d'un empilement de matériau}

Un empilement peut-être résolue itérativement: étant donnée une condition d'impédance en $\z$, on va déduire la condition d'impédance en $\z+d$.

En effet, on peut toujours se ramener dans le cas du plan infini avec une couche, c'est à dire avec la condition d'impédance connue en $\z=-d$ et celle que l'on cherche en $\z=0$

On suppose que l'on ait une condition d'impédance en $\z=-d$
\[
\vE_t=Z_d \, \vn \pvect  \vH
\]


\begin{align*}
\gamma_{TE} := \frac{k_{\z,1}}{k_1} \frac{Z_d}{\eta_1} \qquad
\gamma_{TM} := \frac{k_{\z,1}}{k_1} \frac{\eta_1}{Z_d}  \qquad
\alpha(\gamma) = \frac{1-\gamma}{1+\gamma}
\end{align*}

\begin{center}
\begin{tabular}{| C | C | C |}
\hline
& \text{\textbf{TE}} & \text{\textbf{TM}} \\
\hline\hline

\vE_t & \left(A_p e^{-ik_{\z,p} \z}  + B_p e^{ik_{\z,p} \z}\right) \v e_2 &  \frac{k_{\z,p}\eta_p}{k_p}\left( A_p e^{-ik_{\z,p} \z} - B_p e^{ik_{\z,p}z}\right) \v e_1\\
\hline

\vH_t &\frac{k_{\z,p}}{k_p\eta_p} \left(-A_p e^{-ik_{\z,p} \z}  + B_p e^{ik_{\z,p} \z}\right) \v e_1 & \left(A_p e^{-ik_{\z,p} \z} + B_p e^{ik_{\z,p} \z}\right) \v e_2\\
\hline

\vE_t(-d) = Z_d \vn\times\vH(-d) & A_1 = -\alpha(\gamma_{TE}) B_1e^{-ik_{\z,1}2d} &  \\
\hline

\multirow{2}{*}{$\vE_t(0)=Z\vn\times \vH(0)$} & B_1\left(1-\alpha(\gamma_{TE})e^{-ik_{\z,1}2d}\right) &   \\
& =  Z\frac{k_{\z,1}}{k_1\eta_1}B_1\left(1+\alpha(\gamma_{TE})e^{-ik_{\z,1}2d}\right) & \\
\hline
\hline
Z & \frac{k_1\eta_1}{k_{\z,1}}\tanh\left(ik_{\z,1}d-\frac{\log(\alpha(\gamma_{TE}))}{2}\right) &  \\
\hline
\end{tabular}
\end{center}
\[
P = \frac{ik_1}{k_{\z,1}}diag\left(\frac{1-\alpha(\gamma_{TE}) e^{ik_{\z,1}2d}}{1+\alpha(\gamma_{TE}) e^{ik_{\z,1}2d}},\frac{1-\alpha(\gamma_{TM}) e^{ik_{\z,1}2d}}{1+\alpha(\gamma_{TM}) e^{ik_{\z,1}2d}}\right)
\]
% {
%   \color{blue}Valeur particulière en $Z_d = -\gamma$. À étudier si l'occasion se présente. 
% }
Alors, la condition d'impédance en $\z=0$ est
\begin{align}
\label{eq:coeff_ref:mult_couch:imped}
Z = i \frac{\eta_1}{k_1^2}P\left(k_1^2 I-\LR\right)
\end{align}
où $\LR$ est défini de la même façon qu'en \eqref{eq:coeff_ref:mat_z:imped}
