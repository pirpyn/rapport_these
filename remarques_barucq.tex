\documentclass{article}

\title{Remarques Hélène Barucq}
%\author{}
\date{}

\newcommand{\rep}[1]{\\\textbf{#1}\\}

\begin{document}

\maketitle

\section{Plus de références}
\begin{cite}
    Ce chapitre (1) aurait mérité à mon avis un peu plus de liant car, comme son titre l’indique, il contient des outils mathématiques fondamentaux dont l’alternative de Fredholm sans laquelle tous les résultats mathématiques de la thèse seraient partiels ou même inutilisables.
\end{cite}


\section{Par 2.4.2}
\begin{cite}
    Le paragraphe 2.4.2 est le passage que j’ai préféré même s’il est assez mal rédigé
\end{cite}

\section{Différence CI3 poget vs aubakirov}
\begin{cite}
    Il (chapitre 2) manque par certains endroits de clarté.
    Par exemple, j’aurais aimé comprendre pourquoi il est intéressant de considérer les conditions de Marceaux et Stupfel à la place de celles de Stupfel et Poget.
    Ces dernières sont plus récentes, on
    aurait pu s’attendre à ce qu’elles soient plus performantes.
    Sauf erreur de ma part, il n’y a aucune discussion sur ce sujet dans la thèse.
    Je ne serais pas surprise que les conditions de Marceaux et Stupfel représentent des matériaux plus complexes (des empilements de matériaux?) vu le nombre élevé de paramètres.
    Il serait d’ailleurs intéressant d’avoir sous forme de propriété l’énoncé des conditions d’impédance.
\end{cite}

\section{CI6 pas implémentée}
\begin{cite}
    J’aurais bien aimé avoir plus d’explications à ce sujet car il me semble que les conditions CI3 et CI6 ont la même structure.
\end{cite}

\section{Resultats nums}
\begin{cite}
    Les résultats numériques sont convaincants, on peut cependant regretter qu’ils ne soient pas très nombreux.
\end{cite}

\section{Ouverture}
\begin{cite}
    L’approche proposée pour construire les conditions d’impédance est très originale et mériterait d’être analysée plus en profondeur afin de définir un cadre mathématique
plus rigoureux.
    Notamment, les questions ouvertes posées par le cas où la perméabilité et la permittivité sont complexes décrivent les contours de beaux problèmes mathématiques.
\end{cite}

\end{document}
