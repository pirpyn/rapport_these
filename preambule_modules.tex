%\usepackage{cmbright} % fichier + lisible sur ordi, - lisible sur papier
\usepackage[utf8]{inputenc} %gestion des accents par le compilateurs
\usepackage[cyr]{aeguill}
\usepackage[T1]{fontenc} %gestion des accents par l'afficheurs et la césure
\usepackage[french]{babel} % traduction des packages

\usepackage{lmodern} % fontes tailles variables
% substitution des petites capitales grasses manquantes
\rmfamily
\DeclareFontShape{T1}{lmr}{b}{sc}{<->ssub*cmr/bx/sc}{}
\DeclareFontShape{T1}{lmr}{bx}{sc}{<->ssub*cmr/bx/sc}{}

\usepackage{tabularx} % Permet d'utiliser l'environnement tabularx
\usepackage{graphicx} % gestion de figure, dessins
\usepackage[dvipsnames]{xcolor} % charge des couleurs

\usepackage{xspace}
\usepackage[%
% paperwidth=270.0mm,%  A supprimer si plus besoin de todonotes
headheight=14pt,%
top    = 2.5cm,%
bottom = 2.5cm,%
%left   = 2cm,%
%right  = 2cm,%
]{geometry} % feuille a4 de taille 21.0 x 29.7 

% suite de paquets mathematiques
\usepackage{amsthm} % gestion des théoremes
\usepackage{amsfonts} % gestions des polices mathématques
\usepackage{amsmath}

\usepackage[autostyle=true]{csquotes}
\usepackage[%
  language=english,%
  sorting=ynt,%
  backend=biber,%
  style=alphabetic,%
  hyperref=true,%
  giveninits=true, % initialles pour prénoms
  isbn=false,%
  url=false,%
  doi=false,%
  backref=true,%
  backrefstyle=three% si cité en page 1,2,3, ecrire 1-3, 
  ]{biblatex}
  \renewbibmacro{in:}{} 
  \DeclareFieldFormat[book,report]{title}{\mkbibquote{#1\isdot}}
  \DefineBibliographyStrings{french}{%
    bibliography = {Références},
  }
\usepackage[%
  linkcolor=blue!70!black,%
  colorlinks=true,%
  citecolor=purple%
  ]{hyperref} % gestion des liens hypertexts et reférences

\usepackage{excludeonly}

\usepackage[section]{placeins} % Place un FloatBarrier à chaque nouvelle section
\usepackage{epigraph}
\usepackage[francais,nohints]{minitoc}   % Mini table des matières, en français
\setcounter{minitocdepth}{2} % Mini-toc détaillées (sections/sous-sections)
\usepackage[Lenny]{fncychap}

%\usepackage{multicol} % texte sur plusieurs colonnes
%\usepackage{wrapfig} % permet de gérer les flotants dans les multiples colonnes

\usepackage[%
acronym,%
nomain,%
sanitizesort=false,%
style=super4col%
]{glossaries} %voir glossaire.tex

\setcounter{tocdepth}{1} % Sommaires n'inclus que les niveau 1 à 2 : section, sous-section
\setlength\parindent{0pt} % pas d'alinéa à chaque paragraphe

\usepackage{tikz} % Permet de tracer des figures et schémas
\usetikzlibrary{calc} % permet de réaliser des calculs dans tikz

% \setlength{\oddsidemargin}{25mm}
% \setlength{\evensidemargin}{25mm}
% \setlength{\textwidth}{170mm} 
% \reversemarginpar
% \usepackage{todonotes} % permet d'ajouter des notes en marges

\usepackage{fancyhdr}      % Entête et pieds de page. Doit être placé APRES geometry
  \pagestyle{fancy}    % Indique que le style de la page sera justement fancy
  \fancyfoot[CE,CO]{}
  \fancyfoot[LE,RO]{\thepage}
  \fancyhead[RE,LO]{}
  \fancyhead[RO]{\tiny{\rightmark}}
  \fancyhead[LE]{\tiny{\leftmark}}
%\bibliographystyle{apalike}

\usepackage{array}   % for \newcolumntype macro
\usepackage{multirow}
\newcolumntype{L}{>{$}l<{$}} % math-mode version of column type
\newcolumntype{C}{>{$}c<{$}}
\newcolumntype{R}{>{$}r<{$}}

\usepackage{tcolorbox}