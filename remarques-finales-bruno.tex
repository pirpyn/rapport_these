\documentclass{article}

\usepackage{latexsym,amsmath,amssymb,amscd,amsfonts,amsthm}
\newtheorem{proposition}{proposition}
\newtheorem{remark}{remark}
\newcommand{\bs}{\left\{}
\newcommand{\es}{\right.}
\newcommand{\ba}{\begin{array}}
\newcommand{\ea}{\end{array}}
\newcommand{\be}{\begin{equation}}
\newcommand{\ee}{\end{equation}}
\newcommand{\R}{\mathbb{R}}
\def\RR{{\rm I\hspace{-0.50ex}R} }
\title{Mes remarques et demandes d'ajouts finaux}
%\author{}
\date{}
\usepackage{xcolor}
\newcommand{\rep}[1]{\\\textbf{\color{green!50!black}#1}\\}
\newcommand{\que}[1]{\\\textbf{\color{red!50!black}#1}\\}

\begin{document}

\maketitle

-- Notations physiques.  J: \(A/m^2\). C'est  V/m, comme H au-dessus
\rep{Fait}

-- p 10. Remplacer "Des CIOE similaires, mais plus performantes ont été proposées dans Marceaux et al. 2000 ; Aubakirov 2014 ; Soudais 2017" par  "Une CIOE similaire, mais plus performante a été proposée dans Marceaux et al. 2000 et reprise dans  Aubakirov 2014 ;      Soudais 2017".
\rep{fait. 10eme page du pdf, mais page 2 numéroté}

-- haut p18. Remplacer "... elle est néanmoins une bonne approximation pour des incidences proches de l’incidence oblique ((kx, ky) = (0, 0)) mais pour des matériaux dont l’indice relatif..." par "elle est néanmoins une bonne approximation quelque soit l'incidence pour des matériaux dont l’indice relatif..." 
\rep{Fait, idem page 10 numéroté}

-- bas p17. Remettre la déf de LR du §1.3.2
\que{La page ne correspond pas à la convention que tu as pris jusque ici. choix possible: p 17(25), p22(30) }

-- p40. Supprimer "Supposons l’opérateur a0I + a1LD - a2LR inversible". Il est injectif car Re(a0,a1,a2) ont les bons signes.
\que{La page ne correspont pas, j'imagine que tu parles de la page 33 (41 du pdf). C'est modifié suite aux remarques d'Olivier, regarde si ça te convient.}

-- p63. Fig 3.2. la fig de droite est en trop?
\rep{Elle permet de zoomer. Je te l'accorde, ça n'ajoute pas grand chose, mais ça permet de voir une différence entre TE et TM}

-- p75, haut de page. J'ai minimisé sous contraintes avec mon matlab (je n'ai pas utilisé ton matlab ou ton programme fortran) et je n'ai aucun problème pour trouver \(b1=b2=s_z^2\) A CONDITION de prendre suffisamment de points de discrétisation pour s: il faut mailler finement en incidence. Donc, je remplacerai " Si l’on utilise une méthode basée sur le gradient de type Newton, ce que nous avons fait, on comprend pourquoi la méthode numérique échoue à calculer des coefficients" par "ce qui peut poser un problème si l'on utilise un  gradient de type Newton"
\que{Justement, ce n'est pas "ce qui peut poser problème" mais "ça m'a posé des problèmes". Donc je veux que ce soit visible. Certes je n'ai pas peut être pas assez maillé. Je vais ajouter ça.}

--haut P82. Remplacer " De plus ces géométries sont moins documentées et nous n’avons pas trouvé de littérature traitant de CIOE avec CSU sur des géométries courbes. Nous présentons dans cette partie le cas d’un cylindre infini, en reprenant largement
les résultats du plan" par "Nous considérons ici le cas d’un cylindre infini déjà traité par [Hoppe, rahmat-samii] en reprenant largement les résultats du plan".
\rep{Fait}

-- p93, fig.4.1. On en a parlé plusieurs fois, et le lien entre kt/k0 et kx/k0 n'est toujours pas explicité. Soit tu laisses comme ça, soit tu mets: lorsque la courbure est suffisamment faible, on sait par ailleurs que l'équivalent pour le cercle du  paramètre s défini sur le plan est n/(k0 r0 ).
\rep{J'ai rajouté la notion de courbure faible dans le texte: "On compare l'impédance du plan \(\hat{Z}(k_x,0)\) à l'impédance du cylindre \(\hat{Z}(n,0)\) quand \(n\) est de l'ordre de \(k_0r_1s\) pour de faibles courbures"}

-- p 100. . Or la matrice d’impédance exacte \(Z(0, 0)\) est diagonale --> Or la matrice d’impédance exacte \(Z(0, 0)\) n'est pas multiple de l'identité
\rep{fait}

-- p132. Supprimer somme sur j dans déf de LDij
\rep{fait. Merci.}

-- p134, \$6.8.3. il y a une coquille ds les deux formules en dessous de " dont on déduit à l’aide de la CIOE que": remplacer b1,b2 ds le 1er terme par a1,a2
\que{Faux. (1 + b1 LD - b2 LR)Et = (a0 +a1 LD -a2 LR) donc J = 1/a0((1+b1LD-b2LR)Et - (a1LD -a2 LR)J). Donc c'est bien b1 b2}.

-- p135, proposition 6.24. j'ai +b2 LD au lieu de -b2 LD dans la sous matrice ligne 2, colonne 2 de la matrice finale. A vérifier.
\rep{Les calculs sont dans \$6.8.3, et en effet il faut remplacer b1 LR - b2 LD par b2 LD - b1 LR. L'erreur se situe au paragraphe 'en remarquant que n vect phi ...'}

-- Plus loin, remplacer  " Nous n’avons pas pu bénéficier des solveurs haute-performance du CEA, optimisés pour les systèmes symétriques. Les résultats numériques qui suivent sont donc issus d’un...." par " Les résultats numériques qui suivent sont issus d’un....".
      raison: Un solveur HP pour matrice non symétrique est maintenant disponible. De toutes façons, il te sera facile de dire à l'oral qu'introduire la CIOE ds un code industriel haute performance est un très gros boulot, de + réservé à des spécialistes.
\rep{Fait}

--p137.  Remplacer " Cependant, cette CIOE n’était pas du tout performante même si l’on ne vérifiait pas les CSU" par "Cependant, cette CIOE n’était pas vraiment performante ". Faut tout de même pas en rajouter...
\rep{Fait}

Courage, tu es presque arrivé!



 \end{document}