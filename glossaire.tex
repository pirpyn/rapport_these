\renewcommand{\glstextformat}[1]{\textcolor{black}{#1}}
\renewcommand{\glossarysection}[2][1]{%
  \def\theglstoctitle{#2}%
  \par\noindent
  {\section*{\theglstoctitle}}
  %{\addcontentsline{toc}{section}{\theglstoctitle}}%
}

% create :
% \newglossary{glossary}{glossaryls}{glossarylo}{text}

\newglossary{acr}{acrls}{acrlo}{Acronymes}{
\newglossary{mat}{matls}{matlo}{Notations Mathématiques}{
\newglossary{phy}{phyls}{phylo}{Notations Physiques}{
\newglossary{ope}{opels}{opelo}{Opérateurs}{

\makeglossaries

% create : 
% \newglossaryentry{glossary-shortlabel}{
%  type={glossary},
%  sort={sort},
%  name={\ensuremath{text}},
%  description={description},
}
% use :
% \gls{glossary-shortlabel}

\newglossaryentry{mat-om}{
  type={mat},
  sort={om},
  name={\ensuremath{\O}},
  description={domaine,objet},
}
\newglossaryentry{mat-ga}{
  type={mat},
  sort={ga},
  name={\ensuremath{\Gamma}},
  description={frontière de $\Omega$},
}
\newglossaryentry{mat-vx}{
  type={mat},
  sort={vx},
  name={\ensuremath{\v x}},
  description={point de $\O$},
}
\newglossaryentry{mat-vn}{
  type={mat},
  sort={vn},
  name={\ensuremath{\v n}},
  description={normale unitaire sortante de $\O$},
}
\newglossaryentry{mat-boulexr}{
  type={mat},
  sort={boulexr},
  name={\ensuremath{B(\v x,R)}},
  description={boule de centre $ x$ et de rayon R},
}
\newglossaryentry{mat-spherexr}{
  type={mat},
  sort={spherexr},
  name={\ensuremath{S(\v x,R)}},
  description={sphère de centre $ x$ et de rayon R},
}
\newglossaryentry{mat-hi}{
  type={mat},
  sort={hi},
  name={\ensuremath{H^i(\O)}},
  description={espace de Sobolev d'ordre $i$},
}
\newglossaryentry{mat-xcart}{
  type={mat},
  sort={xcart},
  name={\ensuremath{(x_1,x_2,x_3)}},
  description={coordonnées cartésiennes de $x$},
}
\newglossaryentry{mat-xsphe}{
  type={mat},
  sort={xsphe},
  name={\ensuremath{(\rtp)}},
  description={coordonnées sphérique de $x$},
}
\newglossaryentry{mat-pn}{
  type={mat},
  sort={pn},
  name={\ensuremath{p_n}},
  description={polynôme de Legendre de degré $n$},
}
\newglossaryentry{mat-pmn}{
  type={mat},
  sort={pmn},
  name={\ensuremath{\P^m_n}},
  description={fonction de Legendre d'ordre $m$ et de degré $n$},
}
\newglossaryentry{mat-jn}{
  type={mat},
  sort={bjn},
  name={\ensuremath{j_n}},
  description={fonction de Bessel du 1er ordre de degré $n$},
}
\newglossaryentry{mat-hn}{
  type={mat},
  sort={bhn},
  name={\ensuremath{h_n}},
  description={fonction de Hankel du 2eme type de degré $n$},
}
\newglossaryentry{mat-ymn}{
  type={mat},
  sort={bymn},
  name={\ensuremath{Y_{m,n}}},
  description={harmonique sphérique d'ordre $m$ et de degré $n$},
}
\newglossaryentry{mat-tild}{
  type={mat},
  sort={tild},
  name={\ensuremath{\tilde u}},
  description={$:= \dr{t}(tu(t))$},
}
\newglossaryentry{mat-four}{
  type={mat},
  sort={four},
  name={\ensuremath{\hat u}},
  description={transformée de Fourrier de $u$},
}
\newglossaryentry{mat-matimp}{
  type={mat},
  sort={matimp},
  name={\ensuremath{Z}},
  description={matrice d'impédance},
}
\newglossaryentry{mat-conj}{
  type={mat},
  sort={cconj},
  name={\ensuremath{z^*}},
  description={complexe conjugué de $z$},
}
\newglossaryentry{mat-real}{
  type={mat},
  sort={creal},
  name={\ensuremath{z'}},
  description={partie réelle},
}
\newglossaryentry{mat-imag}{
  type={mat},
  sort={cimag},
  name={\ensuremath{z''}},
  description={partie imaginaire},
}
\newglossaryentry{mat-sum}{
  type={mat},
  sort={sum},
  name={\ensuremath{\ov x}},
  description={$:= \sum_n x_n$. Somme des termes d'une suite.},
}
\newglossaryentry{mat-gmn}{
  type={mat},
  sort={gmn},
  name={\ensuremath{\gamma_{m,n}}},
  description={$:=\frac{4\pi}{2n+1}n(n+1)\frac{(n+m)!}{(n-m)!}$},
}

%%%%%%%%%%%%%%%%%%%%%%%%%%%%%%%%%%%%%%%%%%%%%%%%%%%%%%%%%%%%%%%%%%%%%%%%%%%%%%%%%%%%%%%%%%%%%%%%%%%%%%%%%
%%%%%%%%%%%%%%%%%%%%%%%%%%%%%%%%%%%%%%%%%%%%%%%%%%%%%%%%%%%%%%%%%%%%%%%%%%%%%%%%%%%%%%%%%%%%%%%%%%%%%%%%%
%%%%%%%%%%%%%%%%%%%%%%%%%%%%%%%%%%%%%%%%%%%%%%%%%%%%%%%%%%%%%%%%%%%%%%%%%%%%%%%%%%%%%%%%%%%%%%%%%%%%%%%%%


\newglossaryentry{ope-pv}{
  type={ope},
  sort={pv},
  name={\ensuremath{\cdot \pvect \cdot }},
  description={produit vectoriel},
}
\newglossaryentry{ope-grad}{
  type={ope},
  sort={grad},
  name={\ensuremath{\grad}},
  description={gradient},
}
\newglossaryentry{ope-grads}{
  type={ope},
  sort={grads},
  name={\ensuremath{\grads}},
  description={$\grads u:= \grad u - \n ( \n \cdot \grad u ) $. gradient surfacique},
}
\newglossaryentry{ope-div}{
  type={ope},
  sort={div},
  name={\ensuremath{\div}},
  description={divergent},
}
\newglossaryentry{ope-divs}{
  type={ope},
  sort={divs},
  name={\ensuremath{\divs}},
  description={divergent surfacique},
}
\newglossaryentry{ope-rot}{
  type={ope},
  sort={rot},
  name={\ensuremath{\rot}},
  description={rotationnel},
}
\newglossaryentry{ope-rots}{
  type={ope},
  sort={rots},
  name={\ensuremath{\rots}},
  description={rotationnel surfacique},
}
\newglossaryentry{ope-laps}{
  type={ope},
  sort={laps},
  name={\ensuremath{\Delta}},
  description={laplacien},
}
\newglossaryentry{ope-hess}{
  type={ope},
  sort={hess},
  name={\ensuremath{\Hess}},
  description={hessienne},
}
\newglossaryentry{ope-LD}{
  type={ope},
  sort={hodge-LD},
  name={\ensuremath{L_D}},
  description={$ \int_\Gamma \LD( u)\cdot  v := - \int_S \divs  u \divs  v $},
}
\newglossaryentry{ope-LR}{
  type={ope},
  sort={hodge-LR},
  name={\ensuremath{L_R}},
  description={$ \int_\gamma \LR( u)\cdot  v :=  \int_S \left( \n \cdot \rots  u \right)\left(\n \cdot \rots  v \right)$},
}
\newglossaryentry{ope-L}{
  type={ope},
  sort={hodge-L},
  name={\ensuremath{L}},
  description={$:=L_D - L_R$},
}
\newglossaryentry{ope-real}{
  type={ope},
  sort={creal},
  name={\ensuremath{\Re}},
  description={partie réelle de $z$},
}
\newglossaryentry{ope-imag}{
  type={ope},
  sort={cimag},
  name={\ensuremath{\Im}},
  description={partie imaginaire de $z$},
}
\newglossaryentry{ope-trans}{
  type={ope},
  sort={trans},
  name={\ensuremath{A^t}},
  description={transposée de A},
}
\newglossaryentry{ope-imp}{
  type={ope},
  sort={imp},
  name={\ensuremath{\mathcal Z}},
  description={opérateur d'impédance},
}
\newglossaryentry{ope-saut}{
  type=ope,
  sort=saut,
  name={\ensurmath{[\hat U]_{s}},
  description={$:= U(s+) - U_(s-)$ saut de U à travers la surface s.},
}

%%%%%%%%%%%%%%%%%%%%%%%%%%%%%%%%%%%%%%%%%%%%%%%%%%%%%%%%%%%%%%%%%%%%%%%%%%%%%%%%%%%%%%%%%%%%%%%%%%%%%%%%%
%%%%%%%%%%%%%%%%%%%%%%%%%%%%%%%%%%%%%%%%%%%%%%%%%%%%%%%%%%%%%%%%%%%%%%%%%%%%%%%%%%%%%%%%%%%%%%%%%%%%%%%%%
%%%%%%%%%%%%%%%%%%%%%%%%%%%%%%%%%%%%%%%%%%%%%%%%%%%%%%%%%%%%%%%%%%%%%%%%%%%%%%%%%%%%%%%%%%%%%%%%%%%%%%%%%


\newglossaryentry{phy-e}
  type={phy},
  sort={E},
  name={\ensuremath{\E}},
  description={champ électrique},
}
\newglossaryentry{phy-h}{
  type={phy},
  sort={H},
  name={\ensuremath{\H}},
  description={champ magnétique},
}
\newglossaryentry{phy-f}{
  type={phy},
  sort={f},
  name={\ensuremath{f}},
  description={fréquence $[Hz]$},
}
\newglossaryentry{phy-c}{
  type={phy},
  sort={c},
  name={\ensuremath{c}},
  description={célérité de la lumière $[m.s^{-1}]$},
}
\newglossaryentry{phy-d}{
  type={phy},
  sort={d},
  name={\ensuremath{d}},
  description={épaisseur d'une couche de matériau $[m]$},
}
\newglossaryentry{phy-theta}{
  type={phy},
  sort={theta},
  name={\ensuremath{\theta}},
  description={angle d'observation ou d'incidence de l'onde.},
}
\newglossaryentry{phy-w}{
  type={phy},
  sort={w},
  name={\ensuremath{w}},
  description={pulsation d'une onde $[s^{-1}]$},
}
\newglossaryentry{phy-mu}{
  type={phy},
  sort={mu},
  name={\ensuremath{\mu}},
  description={permittivité magnétique},
}
\newglossaryentry{phy-eps}{
  type={phy},
  sort={eps},
  name={\ensuremath{\eps}},
  description={permittivité diélectrique},
}
\newglossaryentry{phy-eta}{
  type={phy},
  sort={eta},
  name={\ensuremath{\eta}},
  description={impédance diélectrique [ohm]},
}
\newglossaryentry{phy-nu}{
  type={phy},
  sort={nu},
  name={\ensuremath{\nu}},
  description={indice du matériau},
}
\newglossaryentry{phy-k}{
  type={phy},
  sort={k},
  name={\ensuremath{k}},
  description={nombre d'onde},
}
\newglossaryentry{phy-Mmn}{
  type={phy},
  sort={Mmn},
  name={\ensuremath{M_{m,n}}},
  description={harmonique sphérique d'ordre $m$ et de degré $n$},
}
\newglossaryentry{phy-Nmn}{
  type={phy},
  sort={Nmn},
  name={\ensuremath{N_{m,n}}},
  description={rotationnel d'une harmonique sphérique d'ordre $m$ et de degré $n$},
}
\newglossaryentry{phy-coeff}{
  type={phy},
  sort={coeff},
  name={\ensuremath{c_0,c_1,c_2}},
  description={coefficients d'approximation de la CI.},
}

%%%%%%%%%%%%%%%%%%%%%%%%%%%%%%%%%%%%%%%%%%%%%%%%%%%%%%%%%%%%%%%%%%%%%%%%%%%%%%%%%%%%%%%%%%%%%%%%%%%%%%%%%
%%%%%%%%%%%%%%%%%%%%%%%%%%%%%%%%%%%%%%%%%%%%%%%%%%%%%%%%%%%%%%%%%%%%%%%%%%%%%%%%%%%%%%%%%%%%%%%%%%%%%%%%%
%%%%%%%%%%%%%%%%%%%%%%%%%%%%%%%%%%%%%%%%%%%%%%%%%%%%%%%%%%%%%%%%%%%%%%%%%%%%%%%%%%%%%%%%%%%%%%%%%%%%%%%%%

\newglossaryentry{acr-csu}{
  type={acr},
  name={CSU},
  description={condition suffisante d'unicité},
  plural={CSU},
  descriptionplural={conditions suffisantes d'unicité},
  first={\glsentrydesc{acr-csu} (\glsentrytext{acr-csu})},
  firstplural={\glsentrydescplural{acr-csu} (\glsentryplural{acr-csu})},
}
\newglossaryentry{acr-cgu}{
  type={acr},
  name={CSUG},
  description={condition suffisante d'unicité générale},
  first={\glsentrydesc{acr-cgu} (\glsentrytext{acr-cgu})},
}
\newglossaryentry{acr-ci}{
  type={acr},
  name={CI},
  description={condition d'impédance},
  plural={CI},
  descriptionplural={conditions d'impédance},
  first={\glsentrydesc{acr-ci} (\glsentrytext{acr-ci})},
  firstplural={\glsentrydescplural{acr-ci} (\glsentryplural{acr-ci})},
}
\newglossaryentry{acr-cioe}{
  type={acr},
  name={CIOE},
  description={condition d'impédance d'ordre élevé},
  first={\glsentrydesc{acr-cioe} (\glsentrytext{acr-cioe})},
  plural={CIOE},
  descriptionplural={conditions d'impédance d'ordre élevé},
  firstplural={\glsentrydescplural{acr-cioe} (\glsentryplural{acr-cioe})},
}
\newglossaryentry{acr-pec}{
  type={acr},
  name={PEC},
  description={perfect electric conductor},
  plural={PEC},
  descriptionplural={perfect electric conductors},
  first={\glsentrydesc{acr-pec} (\glsentrytext{acr-pec})},
  firstplural={\glsentrydescplural{acr-pec} (\glsentryplural{acr-pec})},
}
\newglossaryentry{acr-cp}{
  type={acr},
  name={CP},
  description={conducteur parfait},
  plural={CP},
  descriptionplural={conducteurs parfait},
  first={\glsentrydesc{acr-cp} (\glsentrytext{acr-cp})},
  firstplural={\glsentrydescplural{acr-cp} (\glsentryplural{acr-cp})},
}
\newglossaryentry{acr-cep}{
  type={acr},
  name={CEP},
  description={conducteur électriquement parfait},
  plural={CEP},
  descriptionplural={conducteurs électriquement parfait},
  first={\glsentrydesc{acr-cep} (\glsentrytext{acr-cep})},
  firstplural={\glsentrydescplural{acr-cep} (\glsentryplural{acr-cep})},
}
\newglossaryentry{acr-cmp}{
  type={acr},
  name={CMP},
  description={conducteur magnétiquement parfait},
  plural={CMP},
  descriptionplural={conducteurs magnétiquement parfait},
  first={\glsentrydesc{acr-cmp} (\glsentrytext{acr-cmp})},
  firstplural={\glsentrydescplural{acr-cmp} (\glsentryplural{acr-cmp})},
}
\newglossaryentry{acr-te}{
  type={acr},
  name={TE},
  description={polarisation transverse électrique},
  plural={TE},
  descriptionplural={polarisations transverse électrique},
  first={\glsentrydesc{acr-te} (\glsentrytext{acr-te})},
  firstplural={\glsentrydescplural{acr-te} (\glsentryplural{acr-te})},
}
\newglossaryentry{acr-tm}{
  type={acr},
  name={TM},
  description={polarisation transverse magnétique},
  plural={TM},
  descriptionplural={polarisation transverse électrique},
  first={\glsentrydesc{acr-tm} (\glsentrytext{acr-tm})},
  firstplural={\glsentrydescplural{acr-tm} (\glsentryplural{acr-tm})},
}
\newglossaryentry{acr-cr}{
  type={acr},
  name={CR},
  description={coefficient de réflexion},
  plural={CR},
  descriptionplural={coefficients de réflexion},
  first={\glsentrydesc{acr-cr} (\glsentrytext{acr-cr})},
  firstplural={\glsentrydescplural{acr-cr} (\glsentryplural{acr-cr})},
}