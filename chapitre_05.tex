\chapter{Contexte mathématique global}

\label{sec:context_math}
\minitoc
\newpage

On s’intéresse à la propagation des ondes électromagnétiques à l'extérieur d'un objet fermé borné, convexe et régulier, sans sources. 
Les champs associés sont solutions des équations de Maxwell harmoniques, la convention dans toute la thèse étant en \(e^{i\w t}\). 

Soit \(\gls{mat-ga}=\partial\overline{\gls{mat-om}}\) la frontière de \(\OO\), on décompose donc tout l'espace entre l'intérieur et l'extérieur, soit 
\[
\RR^3 = \overline{\OO}\cup{\OO}^c.
\]
On définit en tout point de \(\Gamma\), supposée régulière, \(\gls{mat-vn}\) la normale unitaire sortante à \(\OO\).
Dans \(\OO^c\) on définit \(\gls{phy-k0}=\gls{phy-w} / \gls{phy-c} \in \RR_+^*\), le nombre d'onde dans le vide, où \gls{phy-c} 
est la célérité d'une onde dans le vide.
Dans \(\OO\), on a \(k(\gls{mat-vx})=k_0\sqrt{\gls{phy-eps}(\vx)\gls{phy-mu}(\vx)}\).

% Soit \((\vE,\vH)\) dans \((\Hrot(\OO^c)) \times (\Hrot(\OO^c))\) tels que:

Nous utilisons les équations de Maxwell-Helmholtz où les inconnues sont le champ électrique \gls{phy-e} et l'induction magnétique \gls{phy-h2}, elles se déduisent du système d'équations de Maxwell (voir annexe \ref{sec:annex:maxwell_equation}). 

Soit \((\gls{phy-e},\gls{phy-h2})\) dans \((\mathcal{C}_0^\infty(\RR^3)\cap\Hrot(\RR^3))^3\) tels que
\begin{align}
  \left\lbrace
  \begin{matrix}
    \vrot \vE(\vx) + i k(\vx) \vH(\vx) &= 0
    \\
    \vrot \vH(\vx) - i k(\vx) \vE(\vx) &= 0
  \end{matrix}
  \right. && \text{dans \(\OO\cup\OO^c\).}
  \label{eq:unicite:maxwell_ext}
\end{align}

De plus, les conditions limites considérées sont les conditions de Silver-Müller à l'infini (voir \cite[eq (5.2.24), p.~183]{nedelec_acoustic_2001}) et les conditions de saut sur \(\Gamma\).

\section{Équations de Maxwell à l'intérieur et à l'extérieur de l'objet}

  \subsection{Étude de l'unicité des solutions du problème intérieur}

    \begin{thm}
      Si \(\w^2\eps\mu\) n'est pas une valeur propre du Laplacien vectoriel, l'unique solution \(\vE,\vH\) de \eqref{eq:unicite:maxwell_int} avec condition limite de Dirichlet \(n\wedge \vE_{|\Gamma} = 0\),  est \((0,0)\).
      Si \(\w^2\eps\mu\) est une valeur propre du Laplacien vectoriel, \eqref{eq:unicite:maxwell_int} avec condition limite de Dirichlet \(\vE_{|\Gamma} = 0\) a pour solution tous les vecteurs propres de l'opérateur \(A\) introduit par \cite[eq.~(133), p.~58]{cessenat_mathematical_1996}.
    \end{thm}
    La preuve s'appuie notamment sur l'alternative de Fredholm, voir \cite[Théorème~8, p.~111]{cessenat_mathematical_1996}.

  \subsection{Étude de l'unicité des solutions du problème extérieur: Lemme de Rellich}

    On définit en tout point de \(\gls{mat-ga}\) la trace tangentielle de \(\gls{phy-h2}\) que l'on note \(\gls{phy-J} = \gls{mat-vn}_\Gamma \gls{ope-pv} \vH\).
    \begin{prop}[Une \glsentrydesc{acr-csu}]~\\
      Si l'on suppose que
      \begin{equation}
        \label{prop:cgu}
        \Re\left(\int_\Gamma \vJ(\vx) \cdot \conj{\vE_t}(\vx) \dd{\Gamma(\vx)}\right) \ge 0,
      \end{equation}
      alors le système \eqref{eq:unicite:maxwell_ext} (sans source) muni de la condition aux limites \eqref{eq:unicite:TR} admet \(\vE=\vH=0\) comme unique solution.
    \end{prop}
    
    \begin{proof}
      La démonstration s'appuie sur le lemme de Rellich, énoncé dans \cite[p.~74]{cessenat_mathematical_1996} et nous renvoyons à la proposition \ref{eq:unicite:form_var:cgu} du prochain chapitre pour le détail des opérations.

      % \begin{lemme}[Lemme de Rellich]
      %   Soit \(\OO^c\) un domaine connexe, complément d'un domaine borné, et soit \(u\) satisfaisant
      %   \begin{subequations}
      %     \begin{align}
      %       \Delta u + k^2 u = 0 & &\text{dans \(\OO^c\)},
      %       \\
      %       \lim_{R\rightarrow\infty}\int_{S_R} |u(\vx)|^2 \dd{S_R}(\vx) = 0,
      %     \end{align}
      %   \end{subequations}
      %   alors \(u=0\) dans \(\OO^c\).
      % \end{lemme}
      % \begin{proof}
      %   Voir \cite[p.~74]{cessenat_mathematical_1996}.
      % \end{proof}

      % On définit les quantités suivantes
      % \begin{align}
      %   X &= \int_\Gamma \vJ(\vx) \cdot \conj{\vE_t}(\vx)\dd{\Gamma(\vx)},
      %   \label{eq:unicite:x}
      %   \\
      %   C &= \int_{S_R} \Tr(\vE_t)(\vx)  \cdot \conj{\vE_t}(\vx)\dd{S_R(\vx)}.
      % \end{align}
      
      % \begin{align*}
      %   \intertext{De}
      %   a(\vE,\vE) &= \frac{1}{ik_0} \int_{\OO^c_R} \norm{\trot \vE(\vx)}^2 \dd{x} + ik_0\int_{\OO^c_R}\norm{\vE(\vx)}^2 \dd{x}
      %     + X + C,
      %   \intertext{on déduit}
      %   \Re(a(\vE,\vE)) & = 0,
      %   \intertext{ce qui se réecrit}
      %   \Re(C) + \Re(X) & = 0.
      %   \intertext{Comme d'après \cite[Théorème~5.3.5, p.~200]{nedelec_acoustic_2001} \(\Re(C) \ge 0\), on en déduit} 
      %   \Re(X) & \le 0.
      % \end{align*}
      % L'hypothèse \eqref{eq:unicite:form_var:cgu} \(\Re(X) \ge 0\) impose alors \(\Re(X)=0\) et donc
      % \begin{align*} 
      %   \Re(C) &= 0,
      %   \intertext{donc d'après le théorème 5.3.5 de Nédélec}
      %   \vE_t &= 0 && \text{sur \(S_R\)},
      %   \intertext{et d'après le lemme de Rellich}
      %   \vE &= 0 && \text{dans \(\OO^c_R\)}.
      % \end{align*}
      
      % On en déduit \(\vH = 0 \).
    
      % On conclut donc que si on suppose \eqref{eq:unicite:form_var:cgu}, alors en l'absence de sources le couple \((\vE,\vH)=(0,0)\) est l'unique solution, ce qui démontre l'unicité par linéarité.
    \end{proof}      

\section{Équations de Maxwell à l'extérieur avec condition limite de Calderón}
  L'opérateur de Calderón \cite[Def~4, p.~108]{cessenat_mathematical_1996}, lie les traces tangentielles à la surface extérieure de l'objet des champs solutions du problème intérieur et extérieur.
  
  À un couple de domaines ouverts \((\Omega, \Omega^c)\) et à deux problèmes de type Maxwell \eqref{eq:unicite:maxwell_ext} dans chacun des ouverts, est associé un unique opérateur de Calderón.

  En imposant sur le bord extérieur de l'objet une condition limite utilisant cet opérateur, alors les solutions \((\vE,\vH)\) de ce nouveau problème posé uniquement à l'extérieur sont les mêmes que celles du problème intérieur et extérieur.

  % \begin{REM}
  %   % Je cherche la ref dans le Cessenat, mais si tu l'as, je prend.
    
  %   Il me semble que tu peux t'appuyer pour Helmholtz sur le Théorème 5 page 106: unicité de la solution du problème extérieur lorsque \(u_0\) est donné dans \(\Sobolev[1/2](\Gamma)\).
  %   Ou le théorème 6 page 107 pour Maxwell avec \(m_0\) donné dans \(\Sobolev[-1/2](\Gamma)\).
  %   L'opérateur de Calderón extérieur est alors donné dans la définition 4 page 109.
    
  %   Il s'agit alors de comparer les solutions des problèmes (P) et (P'), l'un est le problème de Maxwell dans \(\RR^3\) avec condition de radiation à l'infini et l'appartenance à H(rot) des champs, l'autre est le problème de Maxwell dans \(\Omega\) avec opérateur de Calderón extérieur sur le bord.
  %   J'impose dans (P) et dans (P') la donnée de \(n\wedge E=m_0\) sur \(\Gamma\). (si c'est la restriction à \(\Gamma\), saut nul).
  %   Le problème (P)  a une unique solution.
  %   La restriction de cette solution à \(\Omega^c\) peut donc être étudiée.
    
    
  %   Le problème de Maxwell extérieur admet une unique solution avec \(m_0\) comme donnée. On appelle \((E^*, H^*)\) cette solution , et on déduit \(n\wedge H\vert_{\Gamma_+}= C_e(n\wedge E)_{\Gamma_+})\).
  %   Traitons maintenant le problème (P') et imposons maintenant dans \(\Omega\) Maxwell avec \((m_0, m_1=C_e(m_0))\) sur \(\Gamma_-\).
  %   Ce problème a une unique solution \((E_*, H_*)\).
  %   Si on étudie maintenant \(E=E^*\) dans \(\Omega^c\), \(E=E_*\) dans \(\Omega\), et même chose pour \(H\), les champs ainsi construits sont des élément de \(\Hrot\) (par condition de saut sur \(n\wedge E\) et sur \(n\wedge H\)).
  %   Ces éléments vérifient Maxwell dans chaque ouvert, satisfont à la condition de rayonnement à l'infini, et par la formule des sauts au sens des distributions vérifient Maxwell partout (exactement comme \(\vert x\vert\) vérifie \((u')^2=1\) partout).
  %   De plus la trace de \(n\wedge E\) sur \(\Gamma\) est \(m_0\). C'est donc l'unique solution de (P).  
  % \end{REM}
  % \begin{REM}
  %   Une autre remarque est que le Théorème 3 page 86 de Cessenat te permet de résoudre séparément le problème intérieur et le problème extérieur, en considérant chaque fois \((E,H)=(0,0)\) à l'intérieur et en mettant la masse de Dirac surfacique comme terme source, et réciproquement.
  % \end{REM}

  L'unicité des solutions est assurée par le lemme de Rellich, avec la même condition suffisante d'unicité de la proposition \ref{prop:cgu}, que l'on peut aussi exprimer à l'aide de l'opérateur de Calderón.

  \subsection{Expression de l'opérateur de Calderón pour un plan infini en tant que multiplicateur de Fourier}

    Pour une couche d'épaisseur \(d\) de matériau diélectrique de constante relative \(\eps,\mu\) et donc de nombre d'onde complexe \(k = k_0\sqrt{\eps\mu}\) et d'impédance relative \(\eta=\sqrt{{\mu}/{\eps}}\), l'opérateur de Calderón s'exprime en Fourier (nous choisissons la notation \gls{mat-four} pour désigner la transformée de \(u\)) comme un multiplicateur matriciel en \((k_x,k_y)\), supposant \(\sqrt{k^2-k_x^2-k_y^2}\not=0\),

    \begin{align*}
      \gls{mat-matimp}(k_x,k_y) &= i\eta\frac{\tan\left(d\sqrt{k^2-k_x^2-k_y^2}\right)}{k \sqrt{k^2-k_x^2-k_y^2}}
      \begin{bmatrix}
        k^2-k_x^2  & -k_xk_y
        \\
        -k_xk_y & k^2-k_y^2
      \end{bmatrix}.
    \end{align*}

    Notons ici que nous généralisons l'emploi du nom "opérateur de Calderón" dans ce cas, même si la couche de matériau diélectrique n'est pas un ouvert borné. Nous adopterons cette convention dans toute cette thèse. 
    
    La démonstration de ce calcul, ainsi que son extension à un nombre arbitraire de couches est l'objet du chapitre \ref{sec:plan}.

    On remarque que l'opérateur de Calderón n'est pas un opérateur aux dérivées partielles puisque son multiplicateur de Fourier fait intervenir le complexe \(\sqrt{k^2-k_x^2 - k_y^2}\), qui n'est pas un polynôme en $(k_x,k_y)$ et donc, par transformée inverse de Fourier ne peut s'exprimer avec les opérateurs \(\partial/\partial x, \partial / \partial y\).
    Par contre, on peut approcher cette dernière par un polynôme en \((k_x,k_y)\), voire par une fraction de polynômes, et les CIOE sont issues de cette approximation.
    Par exemple si on approche ce multiplicateur de Fourier par le quotient \(\frac{Q(k_x,k_y)}{R(k_x,k_y)}\), l'égalité \({\hat v}(k_x,k_y)=\frac{Q(k_x,k_y)}{R(k_x,k_y)}{\hat u}(k_x,k_y)\) se réécrit (même s'il faut des résultats supplémentaires pour que la formulation qui suit lui soit équivalente) \(R(k_x,k_y){\hat v}(k_x,k_y)=Q(k_x,k_y){\hat u}(k_x,k_y)\), qui se transforme en une égalité entre $v$ et $u$ utilisant des opérateurs différentiels.
    
    Cette thèse ne fera cependant qu'utiliser des CIOE déjà écrites.
    Nous renvoyons par exemple à \cite{senior_approximate_1995} pour ce problème.

  \subsection{Expression de l'opérateur de Calderón pour un cylindre infini}

    Soit \(\gls{mat-jn}\) la fonction de Bessel de première espèce et \(\gls{mat-hn}\) la fonction de Hankel de deuxième espèce, et \(J_n', H_n^{(2)}{}'\) leur dérivée.

    Pour un cylindre de rayon \(r_0\) recouvert d'une couche d'épaisseur \(d\) (on note \(r_1=r_0 + d\)) de matériau diélectrique de constante relative \(\eps,\mu\) et donc de nombre d'onde complexe \(k = k_0\sqrt{\eps\mu}\) et d'impédance relative \(\eta=\sqrt{{\mu}/{\eps}}\), l'opérateur de Calderón s'exprime en tant que multiplicateur de Fourier matriciel en \((n,k_z)\) où \(n\) est l'indice d'un coefficient de série de Bessel ( qui est une série de Fourier sur \(\theta\) ) et \(k_z\) est la variable de Fourier associée à la transformée de Fourier partielle en \(z\).

    La démonstration des formules qui suivent, ainsi que leur extension à un nombre arbitraire de couches est l'objet du chapitre \ref{sec:cylindre}.

    \subsubsection{Expression de l'opérateur de Calderón quand \(k_z=0\)}
      Le cas de l'incidence perpendiculaire à l'axe du cylindre simplifie grandement les calculs, car cette matrice est alors diagonale:

      \begin{align*}
        (\hat\mZ_n(0))_{11} &= -i\eta\frac{
          (H^{(2)}_n{}')(kr_1)J_n'(kr_0)-J_n{}'(kr_1)(H^{(2)}_n{})'(kr_0)
        }{
          H^{(2)}_n(kr_1)J_n'(kr_0)-J_n(kr_1)(H^{(2)}_n{})'(kr_0)
        },
        \\
        (\hat\mZ_n(0))_{12} &= 0,
        \\
        (\hat\mZ_n(0))_{21} &= 0,
        \\
        (\hat\mZ_n(0))_{22} &=i\eta\frac{
            H^{(2)}_n(kr_1)J_n(kr_0)-J_n(kr_1)H^{(2)}_n(kr_0)
          }{
            (H^{(2)}_n{})'(kr_1)J_n(kr_0)-J_n'(kr_1)H^{(2)}_n(kr_0)
          }.
      \end{align*}

    \subsubsection{Expression de l'opérateur de Calderón pour \(k_z\) quelconque}

      On pose \(k_3 = \sqrt{k^2 - k_z^2}\) que l'on suppose non nul et tel que \(k_z^2 < \Re(k^2)\).
      Ainsi la partie imaginaire de \(k_3\) reste petite devant sa partie réelle.
      Dans le cadre de cette thèse, nous avons considéré \(k_z < k_0\).

      L'opérateur \(\hat\mZ_n(k_z)\) est alors

      \newcommand{\Sni}{\frac{
        (H^{(2)}_n{})'(k_3r_1)J_n'(k_3r_0)-J_n{}'(k_3r_1)(H^{(2)}_n{})'(k_3r_0)
      }{
        H^{(2)}_n(k_3r_1)J_n'(k_3r_0)-J_n(k_3r_1)H^{(2)}_n{}'(k_3r_0)
      }}

      \newcommand{\Tni}{\frac{
        H^{(2)}_n(k_3r_1)J_n(k_3r_0)-J_n(k_3r_1)H^{(2)}_n(k_3r_0)
      }{
        H^{(2)}_n{}'(k_3r_1)J_n(k_3r_0)-J_n'(k_3r_1)H^{(2)}_n(k_3r_0)
      }}

      \begin{align*}
        \begin{split}
          \hat\mZ_n(k_z)_{11} ={}& - i\eta\frac{k}{k_3}\Sni
          \\
          {}& + i\eta\frac{n^2k_z^2}{k^2k_3^2r_1^2}\Tni,
        \end{split}
        \\
        \hat\mZ_n(k_z)_{12} ={}& -i\eta\frac{nk_z}{kk_3r_1}\Tni,
        \\
        \hat\mZ_n(k_z)_{21} ={}& -i\eta\frac{nk_z}{kk_3r_1}\Tni,
        \\
        \hat\mZ_n(k_z)_{22} ={}& i\eta\frac{k_3}{k}\Tni.
      \end{align*}

      On remarque que ces valeurs ne changent pas si on prend n'importe quel couple de solutions linéairement indépendantes de l'équation de Bessel.

      Notons aussi que nous avons parlé d'opérateur de Calderón, même si le cylindre infini n'est pas borné.

  \subsection{Expression de l'opérateur de Calderón pour une sphère}

    Soit \(j_n\) la fonction de Bessel sphérique de première espèce et \(h_n^{(2)}\) la fonction de Hankel sphérique de deuxième type.

    On utilise la notation \gls{mat-tild} pour désigner les dérivées de ces fonctions telles que \(\tilde{f}(z) = \ddp{z}{(zf(z))}\).

    Pour une sphère de rayon \(r_0\) recouverte d'une couche d'épaisseur \(d\) (on pose \(r_1=r_0 + d\)) de matériau diélectrique de constantes relatives \(\eps,\mu\) et donc de nombre d'onde complexe \(k = k_0\sqrt{\eps\mu}\) et d'impédance relative \(\eta=\sqrt{{\mu}/{\eps}}\), l'opérateur de Calderón s'exprime en tant que multiplicateur  matriciel de Fourier sur les termes en série de Mie.

    \begin{align*}
      (\hat \mZ_n)_{11} &=  i\eta kr_1 
        \frac{
          h^{(2)}_n(kr_1)j_n(kr_0)-j_n(kr_1)h^{(2)}_n(kr_0)
        }{
          \tilde{h^{(2)}_n}(kr_1)j_n(kr_0)-\tilde{j_n}(kr_1)h^{(2)}_n(kr_0)
        },
      \\
      (\hat \mZ_n)_{12} &= 0,
      \\
      (\hat \mZ_n)_{21} &= 0,
      \\
      (\hat \mZ_n)_{22} &= i \frac{\eta}{kr_1}
        \frac{
          \tilde{h^{(2)}_n}(kr_1)\tilde{j_n}(kr_0)-\tilde{j_n}(kr_1)\tilde{h^{(2)}_n}(kr_0)
        }{
          {h^{(2)}_n}(kr_1)\tilde{j_n}(kr_0)-{j_n}(kr_1)\tilde{h^{(2)}_n}(kr_0)
        }.
    \end{align*}

    La démonstration de ce calcul, ainsi que son extension à un nombre arbitraire de couches est l'objet du chapitre \ref{sec:sphere}.

\section[Problème extérieur et condition limite de Calderón]{Équations de Maxwell à l'extérieur de l'objet avec une approximation de l'opérateur de Calderón sur son bord}

  Nous cherchons à résoudre le problème de Maxwell extérieur avec condition de Silver-Müller à l'infini et une \gls{acr-cl} sur le bord de l'objet que nous choisirons.
  Cette CL est alors à choisir judicieusement si l'on espère que les solutions de ce nouveau problème approchent bien celui du problème extérieur avec opérateur de Calderón.

  \subsection{Approximation constante de l'opérateur de Calderón: la condition de Leontovich}
        
    Définie par \cite{leontovich_investigations_1948}, cette condition d'impédance consiste à approcher l'opérateur de Calderón par un opérateur constant.
    \[
      \hat\vE_t = a_0 \hat \vJ
    \]
    Très facile à mettre en œuvre numériquement, elle est néanmoins une bonne approximation quelque soit l'incidence pour des matériaux dont l'indice relatif \(|\sqrt{\eps\mu}|\) est grand devant l'unité (\cite[par.~3, p.421-422]{senior_impedance_1960}).

    Des choix possibles pour \(a_0\) sont alors
    \begin{itemize}
      \item plan: \(a_0 = \hat\mZ(0,0)_{11}\), pour ce couple \((k_x,k_y)\), l'opérateur est toujours multiple de l'identité.
      \item cylindre: \(a_0 = \hat\mZ_0(0)_{11}\), \(a_0 = \hat\mZ_0(0)_{22}\) ou une combinaison linéaire des deux. À noter que la courbure est prise en compte.
      \item sphère: \(a_0 = \hat\mZ_0,{}_{11}\), pour le 1\ier coefficient de la série de Mie, l'opérateur est toujours multiple de l'identité. Là aussi, on prend en compte les courbures.
    \end{itemize}

  \subsection{Approximation de Taylor et Padé: les CIOE}

    Comme dit précédemment, les CIOE approchent l'opérateur de Calderón par des opérateurs différentiels surfaciques.
    Dans le cadre de cette thèse, les opérateurs que nous utiliserons sont

    \begin{align*}
      \gls{ope-LD} &= \tgrads \tdivs,
      \\
      \gls{ope-LR} &= \tvrots \trots,
      \\
      \gls{ope-L} &= \gls{ope-LD} - \gls{ope-LR}.
    \end{align*}

    Leurs multiplicateurs de Fourier matriciels associés pour le plan infini sont
    \begin{align*}
      \hat\mLD(k_x,k_y) &= \begin{bmatrix}-k_x^2 & -k_xk_y \\ -k_xk_y & -k_y^2\end{bmatrix}, &
      \hat\mLR(k_x,k_y) &= \begin{bmatrix}k_y^2 & -k_xk_y \\ -k_xk_y & k_x^2\end{bmatrix}.
    \end{align*}
    Soit \(r_e\) le rayon de la surface extérieure de l'objet.
      
    Les multiplicateurs de Fourier matriciels associés au cylindre infini sont, 
    \begin{align*}
      \hat\mLD_n(k_z) &= \begin{bmatrix}-(n/r_e)^2 & -k_zn/r_e \\ -k_zn/r_e &-k_z^2\end{bmatrix},
      &
      \hat\mLR_n(k_z) &= \begin{bmatrix}k_z^2 & -k_zn/r_e \\ -k_zn/r_e & (n/r_e)^2\end{bmatrix},
    \end{align*}

    et ceux de la sphère sont
    \begin{align*}
      \hat\mLD_n &= \begin{bmatrix}0 & 0 \\ 0 &-n(n+1)/ r_e^2\end{bmatrix},
      &
      \hat\mLR_n &= \begin{bmatrix}n(n+1)/ r_e^2 &0 \\ 0 & 0\end{bmatrix}.
    \end{align*}

    Le nom des CIOE est local à cette thèse et ne reflète que l'ordre dans lequel elles ont été utilisées, notamment par \cite{stupfel_sufficient_2011}, puis ici.
    D'autres ouvrages les désignent autrement, voir \cite{hoppe_higher_1994,senior_approximate_1995,aubakirov_electromagnetic_2014}.
      
    Nous désignons la condition de Leontovich par CI0.

  \subsection{Taylor à l'ordre 1: la CI01}

    On approche l'égalité caractérisant l'opérateur de Calderón par
    \[
      \hat\vE_t = ( a_0\oI + a_1 \LL ) \hat \vJ
    \]

  \subsection{Padé à l'ordre 1:1: la CI1}

    On approche l'égalité caractérisant l'opérateur de Calderón par
    \[
      ( \oI + b_1 \LL )\hat\vE_t = ( a_0\oI + a_1 \LL ) \hat \vJ
    \]

  \subsection{Padé à l'ordre 1:1, découplé: la CI3}

    On approche l'égalité caractérisant l'opérateur de Calderón par
    \[
      ( \oI + b_1 \LD -b_2 \LR )\hat\vE_t = ( a_0\oI + a_1 \LD -a_2 \LR) \hat \vJ
    \]

  \subsection{Taylor à l'ordre 1, découplé: la CI4}

    On approche l'égalité caractérisant l'opérateur de Calderón par
    \[
      \hat\vE_t = ( a_0\oI + a_1 \LD - a_2 \LR ) \hat \vJ
    \]

\section{Étude de l'unicité des solutions du problème extérieur avec CIOE sur le bord}

  Il est bien entendu que pour chaque \gls{acr-cioe}, le problème avec une onde incidente conduit \emph{s'il y a unicité}, à une unique solution diffractée dont on espère qu'elle est une approximation de la solution exacte du problème diffracté avec opérateur de Calderón.

  {~}\\{~}
  Tout le problème est de choisir les paramètres complexes des CIOE, obtenus grâce à des approximations issues de critères physiques (matériaux, courbure et fréquence) mais ne dépendant pas de la solution du problème diffracté calculée avec l'opérateur de Calderón. 

  Il apparaît donc comme incorrect de considérer des paramètres qui ne permettraient pas d'obtenir une unique solution.
  Plus exactement, comme on ne connaît pas de condition nécessaire et suffisante d'existence et d'unicité, on se contentera de \emph{garantir} l'unicité de la solution dans des cas plus restreints, ceux de conditions \emph{suffisantes} d'unicité.

  S'il n'y a pas unicité, d'après l'alternative de Fredholm pour un problème avec source, on peut ne pas avoir existence.
  Dans le cas d'une source nulle, l'existence est assurée puisque la solution nulle vérifie toutes les conditions et équations du problème.

  {~}\\{~}
  Nous allons donc permettre d'assurer l'unicité de la solution du problème sans source (c'est-à-dire déterminer des \glspl{acr-csu} qui piloteront le choix des paramètres), mais il est extrêmement important de comprendre que grâce à cela, nous pourrons \emph{garantir} l'unicité de la solution du problème avec source.

  {~}\\{~}
  Les démonstrations des résultats de cette partie sont l'objet du chapitre \ref{sec:csu}, ainsi les \glspl{acr-csu} des autres CIOE.
  Afin de garantir la proposition \ref{prop:cgu}, les conditions suivantes permettent de choisir les paramètres des CIOE correspondantes.

  \subsection{Condition suffisante d'unicité (CSU) de la CI0}
    \begin{align*}
      \Re(a_0) \ge 0.
    \end{align*}
  
  \subsection{CSU de la CI1}
    \begin{align*}
      \Re(a_0) \ge 0,&& \Re(a_1) \le 0.
    \end{align*}

  \subsection{CSU de la CI4}
    \begin{align*}
      \Re(a_0) \ge 0,&& \Re(a_1) \le 0,&& \Re(a_2) \le 0.
    \end{align*}
