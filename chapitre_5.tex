\chapter{Contribution des CIOE dans une formulation équation intégrale}
\label{sec:equation_integrale}
\minitoc
\newpage
\sectionstar{Introduction}
Nous pouvons donc aborder la dernière partie de notre problème pour évaluer la pertinence de notre démarche. En effet, c'est en calculant les champs électromagnétiques tangentiels sur la surface de notre objet et en les comparants avec des résultats de référence.

Les équations intégrales sont des méthodes bien éprouvées pour résoudre les équations de Maxwell, tel que l'a démontré \cite{nedelec_acoustic_2001}. Ce chapitre rappelle quelques résultats connus et montre la contribution des \gls{acr-cioe}.

\section{Espaces fonctionnels}
  Les équations intégrales sont résolues grâce à la méthode des éléments finis de frontière. Nous rappelons le théorème de trace le plus important pour résoudre les équations de Maxwell par équations intégrales.

  \begin{defn}
    Pour toute fonction \(\vect{U}(\vx) \in \Hrot(\OO_h), \vrot \vect{U}(\vx) \in L^2(\OO)\).

    Pour toute fonction \(\vect{U}(\vx) \in \Hrot^{-1/2}(\Gamma), \vrots \vect{U}(\vx) \in L^2(\Gamma)\).

    Pour toute fonction \(\vect{U}(\vx) \in \Hdiv^{-1/2}(\Gamma), \vdivs \vect{U}(\vx) \in L^2(\Gamma)\).
  \end{defn}
  \begin{thm}[Théorème de trace de \(\Hrot\)]
    L'opérateur \(\gamma_t\) qui pour tout  \(\vect{v}\in\Hrot(\OO)\) associe ses composantes tangentielles \(\vect{v}_t\) sur \(\Gamma\) est continu et surjectif de \(\Hrot(\OO)\) vers \(\Hrot^{-1/2}(\Gamma)\) et l'opérateur \(\gamma_t'\) qui pour tout  \(\vect{v}\in\Hrot{\OO}\) associe \(\vn \pvect \vect{v}\) sur \(\Gamma\) est continu et surjectif de \(\Hrot(\OO)\) vers \(\Hdiv^{-1/2}(\Gamma)\).
  \end{thm}
  \begin{proof}
    Voir la démonstration de \cite[Théorème~5.4.2]{nedelec_acoustic_2001}, page 209.
  \end{proof}

\section{Équations intégrales}

    La démonstration de ces formules dépasse le cadre de cette thèse et nous renvoyons à \cite[\textsection~5.6]{nedelec_mixed_1980}.

    Soit \(k_0\) le nombre d'onde dans le vide. Soit \(G\) la fonction de Green définie en tout point extérieur à \(\Gamma\) telle que
    \begin{equation*}
      G(\vz) = \frac{e^{-ik_0|\vz|}}{4\pi|\vz|}.
    \end{equation*}
    On définit \(g(\vx,\vy) := G(\vx - \vy)\).

    Soit \(\vE^{inc}_t,\vH^{inc}_t\) le champ incident tangentiel à \(\Gamma\). On définit sur la surface \(\Gamma\), les traces tangentielles des champs \(\vJ = \vn \pvect \vH\), \(\vK = \vn \pvect \vE\).

    En utilisant le théorème de représentation intégrale (voir \cite[Théorème 5.5.1]{nedelec_acoustic_2001}), on obtient les équations intégrales suivantes
    \subsection{EFIE}

      \begin{prop}
        \label{eq:form_int:EFIE}
        Soit \(\Gamma\) une surface régulière.
        \(\vJ,\vK\) sont solutions des équations intégrales en champ électrique (\gls{acr-efie}) s'il existe \((\vJ,\vK) \in \Hdiv^{-1/2}(\Gamma)^2\) tels que
        \begin{multline*}
          \vE^{inc}_t(\vx) =
            \frac{\vE_t(\vx)}{2}
              - \int_{\Gamma} \vgrad g(\vx - \vy) \pvect \vK(\vy) \dd{\Gamma}(\vy) \\
            + \frac{i}{k}\vgrad \int_{\Gamma}  g(\vx - \vy)\vdiv\vJ(\vy)(\vx) \dd{\Gamma}(\vy)
              +  ik\int_{\Gamma} g(\vx - \vy)\vJ(\vy) \dd{\Gamma}(\vy).
        \end{multline*}
      \end{prop}


    \subsection{MFIE}
      \begin{prop}
        \label{eq:form_int:MFIE}
        Soit \(\Gamma\) une surface régulière.
        \(\vJ,\vK\) sont solutions des équations intégrales en champ magnétique (\gls{acr-mfie}) s'il existe \((\vJ,\vK) \in \Hdiv^{-1/2}(\Gamma)^2\) tels que
        \begin{multline*}
          \vH^{inc}(\vx) =
          \frac{\vH_t(\vx)}{2}
            - \int_{\Gamma} \vgrad g(\vx - \vy) \pvect \vJ(\vy) \dd{\Gamma}(\vy) \\
          - \frac{i}{k} \vgrad \int_{\Gamma}  g(\vx - \vy)\vdiv\vK(\vy)\dd{\Gamma}(\vy)
            - ik \int_{\Gamma} g(\vx - \vy)\vK(\vy)\dd{\Gamma}(\vy).
        \end{multline*}
      \end{prop}

\section{Discrétisation de la surface de l'objet}

  La méthode numérique utilisée au CEA  est basée sur l'article de \cite{medgyesi-mitschang_integral_1985}. Nous rappelons cette méthode en conservant les notations de \cite{stupfel_implementation_2015}.\\

  Soit \(\Gamma_h\) une triangulation de \(\Gamma\) en \(N_e\) éléments triangulaires. Soit \(N=\frac{3}{2}N_e\) le nombre d'arête du maillage. On désigne par \(\vn\) la normale unitaire sortante du maillage, non défini aux nœuds et arêtes du maillage.

  Soit \(T\) un triangle défini par un nœud \(N_1\) et deux vecteurs \(\vect{e_1}\) et \(\vect{e_2}\). On rappelle la relation entre l'aire du triangle et ces vecteurs
  \begin{equation*}
    2|T| = |\vect{e_1}\pvect\vect{e_2}|
  \end{equation*}

  % Les commandes qui suivent servent pour les prochains schémas
  \newcommand{\ncouche}{6}
  \newcommand{\setnodes}[6]{
      \renewcommand{\xa}{#1}
      \renewcommand{\ya}{#2}
      \renewcommand{\xb}{#3}
      \renewcommand{\yb}{#4}
      \renewcommand{\xc}{#5}
      \renewcommand{\yc}{#6}
  }
  \newcommand{\xa}{0.0}
  \newcommand{\ya}{0.0}
  \newcommand{\xb}{3.0}
  \newcommand{\yb}{0.0}
  \newcommand{\xc}{0.0}
  \newcommand{\yc}{3.0}
    \begin{center}
      \tikzsetnextfilename{fonbasloc_triangle_loc}
      \begin{tikzpicture}[scale=1]
        %%% Triangle de gauche

%%% Triangle de gauche

\coordinate (n1) at (\xa,\ya);
\coordinate (n2) at (\xc,\yc);
\coordinate (n3) at (\xb,\yb);

\coordinate (e1) at ($(n2)-(n3)$);
\coordinate (e2) at ($(n3)-(n1)$);
\coordinate (e3) at ($(n1)-(n2)$);


\draw (n1) -- (n2) -- (n3) -- cycle;


%\draw [-Latex] (n3) -- (n2) node [midway,above right] {\(\vect{e_1}\)};
\draw [-Latex] (n1) -- (n3) node [midway,below] {\(\vect{e_1}\)};
\draw [-Latex] (n1) -- (n2) node [midway,left] {\(\vect{e_2}\)};

\coordinate (pt) at  ($1/3*(n1)+1/3*(n2)+1/3*(n3)$);

\draw (pt) node {\(T\)};

\newcommand{\nutau}[3]{
  % la normale nu
  \path let \p1=(#1) in coordinate (nu) at (\y1,-\x1);

  \tikzmath
  {
      \ra = sqrt((\xb-\xc)^2 + (\yb-\yc)^2);
  }

  \draw [-Latex] (#2) -- ($(#2) + 1/\ra*(nu)$) node [midway,below,sloped] {\(\vect{\nu_{#3}}\)};

  % % la normale tau_j
  \path let \p1=(nu) in coordinate (tau) at (-\y1,\x1);
  \draw [-Latex] (#2) -- ($(#2) + 1/\ra*(tau)$) node [midway,above,sloped] {\(\vect{\tau_{#3}}\)};
}

% \nutau{e1}{$1/3*(n2)+2/3*(n3)$}{1}
% \nutau{e2}{n1}{2}
% \nutau{e3}{n2}{3}

% La normale au triangle.
\coordinate (pn) at  ($3/4*(n1)+1/8*(n2)+1/8*(n3)$);
\draw[thick] (pn) circle(0.1) node [right=0.6] {\(\vn\)};
\fill (pn) circle(0.03);


\draw (n1) node [below left] {\(N_1\)};
%\draw (n2) node [above left] {\(N_2\)};
%\draw (n3) node [below right] {\(N_3\)};

      \end{tikzpicture}
    \end{center}
    \captionof{figure}{Triangles, arêtes et nœuds définis par l'arête \(\uj\)}
    \label{fig:form_int:fon_base:tri}

  On se place dans le repère local au triangle, on fait donc le changement de coordonnées suivant:
  \begin{align*}
    \RR^3 &\rightarrow [0,1]^2 \\
    (x_1,x_2,x_3) & \mapsto (\xi_1,\xi_2),
  \end{align*}
    tel que \(\vx =\vect{N_1}+ \xi_1 \vect{e_1} + \xi_2 \vect{e_2}\).

  \subsection[Fonctions de Raviart-Thomas phi Hdiv-conforme]{Fonctions de Raviart-Thomas \(\vect{\phi}\) \(\Hdiv\)-conforme}

    Ces fonctions ont été introduites par \cite{raviart_mixed_1977}. Nous présentons ici les fonctions Raviart-Thomas de degré 0, c'est à dire telles que les degrés de liberté soient au milieu de chaque arête (voir \cite[eq.~(3.10)]{raviart_mixed_1977}).

    \begin{minipage}{\textwidth}
      \begin{minipage}{0.329\textwidth}
          % Les commandes qui suivent servent pour les 4 prochains schémas
          \setnodes{0}{0}{3}{0}{0}{3}
          \begin{center}
            \tikzsetnextfilename{fonbasloc_phi_1}
            \begin{tikzpicture}[scale=1]
              %%% Triangle de gauche

\tikzmath
{
    \ra = sqrt((\xb-\xc)^2 + (\yb-\yc)^2);
    \rb = sqrt((\xc-\xa)^2 + (\yc-\ya)^2);
    \rc = sqrt((\xa-\xb)^2 + (\ya-\yb)^2);
    \aire = 0.25*sqrt((\ra+\rb-\rc)*(\ra-\rb+\rc)*(-\ra+\rb+\rc)*(\ra+\rb+\rc));
}

\coordinate (na) at (\xa,\ya);
\coordinate (nb) at (\xb,\yb);
\coordinate (nc) at (\xc,\yc);

\coordinate (ec) at ($(nb)-(na)$);
\coordinate (ea) at ($(nc)-(nb)$);

\draw (na) -- (nc) -- (nb) -- cycle;

\foreach \couche in {1,...,\ncouche}
{
    \tikzmath
    {
        \a = \couche/\ncouche;
    }
    \foreach \n in {0,...,\couche}
    {
        \tikzmath
        {
            \b = \n/\couche;
        }
        \coordinate (phi) at ($\a*(ec)+\a*\b*(ea)$);
        \coordinate (x) at ($(na)+(phi)$);
        % \draw (x) node {\(x_\n^\couche\)};
        \draw [->] (x) -- ($(x)+1/\aire*(phi)$);
    }
}

%%% Triangle de droite

\tikzmath
{
    \rd = sqrt((\xb-\xc)^2 + (\yb-\yc)^2);
    \rb = sqrt((\xc-\xd)^2 + (\yc-\yd)^2);
    \rc = sqrt((\xd-\xb)^2 + (\yd-\yb)^2);
    \aire = 0.25*sqrt((\rd+\rb-\rc)*(\rd-\rb+\rc)*(-\rd+\rb+\rc)*(\rd+\rb+\rc));
}

\coordinate (nd) at (\xd,\yd);

\coordinate (ed) at ($(nc)-(nb)$);
\coordinate (ec) at ($(nb)-(nd)$);

\draw (nb) -- (nd) -- (nc) -- cycle;

\foreach \couche in {1,...,\ncouche}
{
    \tikzmath
    {
        \a = \couche/\ncouche;
    }
    \foreach \n in {0,...,\couche}
    {
        \tikzmath
        {
            \b = \n/\couche;
        }
        \coordinate (phi) at ($\a*(ec)+\a*\b*(ed)$);
        \coordinate (x) at ($(nd)+(phi)$);
        % \draw (x) node {\(x_\n^\couche\)};
        \draw [->] (x) -- ($(x)-1/\aire*(phi)$);
    }
}

            \end{tikzpicture}
          \end{center}
          \begin{equation*}
            \vect{\phi_1}=\frac{\xi_1}{2|T|}\vect{e_1} + \frac{\xi_2}{{2|T|}}\vect{e_2},
          \end{equation*}
      \end{minipage}
      \begin{minipage}{0.329\textwidth}
          \setnodes{3}{0}{0}{3}{0}{0}
          \begin{center}
            \tikzsetnextfilename{fonbasloc_phi_2}
            \begin{tikzpicture}[scale=1]
              %%% Triangle de gauche

\tikzmath
{
    \ra = sqrt((\xb-\xc)^2 + (\yb-\yc)^2);
    \rb = sqrt((\xc-\xa)^2 + (\yc-\ya)^2);
    \rc = sqrt((\xa-\xb)^2 + (\ya-\yb)^2);
    \aire = 0.25*sqrt((\ra+\rb-\rc)*(\ra-\rb+\rc)*(-\ra+\rb+\rc)*(\ra+\rb+\rc));
}

\coordinate (na) at (\xa,\ya);
\coordinate (nb) at (\xb,\yb);
\coordinate (nc) at (\xc,\yc);

\coordinate (ec) at ($(nb)-(na)$);
\coordinate (ea) at ($(nc)-(nb)$);

\draw (na) -- (nc) -- (nb) -- cycle;

\foreach \couche in {1,...,\ncouche}
{
    \tikzmath
    {
        \a = \couche/\ncouche;
    }
    \foreach \n in {0,...,\couche}
    {
        \tikzmath
        {
            \b = \n/\couche;
        }
        \coordinate (phi) at ($\a*(ec)+\a*\b*(ea)$);
        \coordinate (x) at ($(na)+(phi)$);
        % \draw (x) node {\(x_\n^\couche\)};
        \draw [->] (x) -- ($(x)+1/\aire*(phi)$);
    }
}

%%% Triangle de droite

\tikzmath
{
    \rd = sqrt((\xb-\xc)^2 + (\yb-\yc)^2);
    \rb = sqrt((\xc-\xd)^2 + (\yc-\yd)^2);
    \rc = sqrt((\xd-\xb)^2 + (\yd-\yb)^2);
    \aire = 0.25*sqrt((\rd+\rb-\rc)*(\rd-\rb+\rc)*(-\rd+\rb+\rc)*(\rd+\rb+\rc));
}

\coordinate (nd) at (\xd,\yd);

\coordinate (ed) at ($(nc)-(nb)$);
\coordinate (ec) at ($(nb)-(nd)$);

\draw (nb) -- (nd) -- (nc) -- cycle;

\foreach \couche in {1,...,\ncouche}
{
    \tikzmath
    {
        \a = \couche/\ncouche;
    }
    \foreach \n in {0,...,\couche}
    {
        \tikzmath
        {
            \b = \n/\couche;
        }
        \coordinate (phi) at ($\a*(ec)+\a*\b*(ed)$);
        \coordinate (x) at ($(nd)+(phi)$);
        % \draw (x) node {\(x_\n^\couche\)};
        \draw [->] (x) -- ($(x)-1/\aire*(phi)$);
    }
}

            \end{tikzpicture}
             \begin{equation*}
              \vect{\phi_2}=\frac{\xi_1-1}{2|T|}\vect{e_1} + \frac{\xi_2}{2|T|}\vect{e_2},
            \end{equation*}
          \end{center}
      \end{minipage}
      \begin{minipage}{0.329\textwidth}
          \setnodes{0}{3}{0}{0}{3}{0}
          \begin{center}
            \tikzsetnextfilename{fonbasloc_phi_3}            
            \begin{tikzpicture}[scale=1]
              %%% Triangle de gauche

\tikzmath
{
    \ra = sqrt((\xb-\xc)^2 + (\yb-\yc)^2);
    \rb = sqrt((\xc-\xa)^2 + (\yc-\ya)^2);
    \rc = sqrt((\xa-\xb)^2 + (\ya-\yb)^2);
    \aire = 0.25*sqrt((\ra+\rb-\rc)*(\ra-\rb+\rc)*(-\ra+\rb+\rc)*(\ra+\rb+\rc));
}

\coordinate (na) at (\xa,\ya);
\coordinate (nb) at (\xb,\yb);
\coordinate (nc) at (\xc,\yc);

\coordinate (ec) at ($(nb)-(na)$);
\coordinate (ea) at ($(nc)-(nb)$);

\draw (na) -- (nc) -- (nb) -- cycle;

\foreach \couche in {1,...,\ncouche}
{
    \tikzmath
    {
        \a = \couche/\ncouche;
    }
    \foreach \n in {0,...,\couche}
    {
        \tikzmath
        {
            \b = \n/\couche;
        }
        \coordinate (phi) at ($\a*(ec)+\a*\b*(ea)$);
        \coordinate (x) at ($(na)+(phi)$);
        % \draw (x) node {\(x_\n^\couche\)};
        \draw [->] (x) -- ($(x)+1/\aire*(phi)$);
    }
}

%%% Triangle de droite

\tikzmath
{
    \rd = sqrt((\xb-\xc)^2 + (\yb-\yc)^2);
    \rb = sqrt((\xc-\xd)^2 + (\yc-\yd)^2);
    \rc = sqrt((\xd-\xb)^2 + (\yd-\yb)^2);
    \aire = 0.25*sqrt((\rd+\rb-\rc)*(\rd-\rb+\rc)*(-\rd+\rb+\rc)*(\rd+\rb+\rc));
}

\coordinate (nd) at (\xd,\yd);

\coordinate (ed) at ($(nc)-(nb)$);
\coordinate (ec) at ($(nb)-(nd)$);

\draw (nb) -- (nd) -- (nc) -- cycle;

\foreach \couche in {1,...,\ncouche}
{
    \tikzmath
    {
        \a = \couche/\ncouche;
    }
    \foreach \n in {0,...,\couche}
    {
        \tikzmath
        {
            \b = \n/\couche;
        }
        \coordinate (phi) at ($\a*(ec)+\a*\b*(ed)$);
        \coordinate (x) at ($(nd)+(phi)$);
        % \draw (x) node {\(x_\n^\couche\)};
        \draw [->] (x) -- ($(x)-1/\aire*(phi)$);
    }
}

            \end{tikzpicture}
            \begin{equation*}
              \vect{\phi_3}=\frac{\xi_1}{2|T|}\vect{e_1} + \frac{\xi_2-1}{2|T|}\vect{e_2}.
            \end{equation*}
          \end{center}
      \end{minipage}
      \captionof{figure}{Fonctions \(\phij\) locales}
      \label{fig:form_int:fon_base:phi}
    \end{minipage}

    \begin{prop}
      Les fonctions \(\phij\) sont dans \(\Hdiv(\Gamma_h)\).
    \end{prop}
    \begin{proof}
      Concernant la relation de compatibilité \(\Hdiv\), nous renvoyons à la propriété~\ref{prop:annex:hdiv_hrot:hdiv} en annexe.
    \end{proof}

    \begin{defn}
      Pour tout \(\vect{U} \in \Hdiv(\Gamma_h)\), alors
      \begin{equation*}
        \vect{U} \in \Vect{\vect{\phi_1},\ldots,\vect{\phi_N}} \Leftrightarrow \exists \comp{u} = (u_1,\cdots,u_N)^t \in \RR^N, \vect{U}(\vx) = \sum_{j=1}^N u_j \phij(\vx).
      \end{equation*}
    \end{defn}
%%%%%%%%%%%%%%%%%%%%%%%%%%%%%%%%%%%%%%%%%%%%%%%%%%%%%%%%%%%%%%%%%%%%%%%%%%%%%%%%%%%%%%%%%%%%%%%%%%%%%%%%
%%%%%%%%%%%%%%%%%%%%%%%%%%%%%%%%%%%%%%%%%%%%%%%%%%%%%%%%%%%%%%%%%%%%%%%%%%%%%%%%%%%%%%%%%%%%%%%%%%%%%%%%
%%%%%%%%%%%%%%%%%%%%%%%%%%%%%%%%%%%%%%%%%%%%%%%%%%%%%%%%%%%%%%%%%%%%%%%%%%%%%%%%%%%%%%%%%%%%%%%%%%%%%%%%

\section{Fonctions de projections}


  \subsection[Fonctions de Nédélec p Hrot-conforme]{Fonctions de Nédelec \(\vect{p}\) \(\Hrot\)-conforme}

    Une première famille est celle des \(\pj=-\vn\pvect\phij\), où \(\vn\) est la normale extérieure au maillage. 

    \begin{minipage}{\textwidth}
      \begin{minipage}{0.329\textwidth}
          % Les commandes qui suivent servent pour les 4 prochains schémas
          \setnodes{0}{0}{3}{0}{0}{3}
          \begin{center}
            \tikzsetnextfilename{fonbasloc_p_1}
            \begin{tikzpicture}[scale=1]
              %%% Triangle de gauche
%%% Triangle de gauche

\coordinate (n1) at (\xa,\ya);
\coordinate (n2) at (\xc,\yc);
\coordinate (n3) at (\xb,\yb);

\coordinate (e1) at ($(n2)-(n3)$);
\coordinate (e2) at ($(n3)-(n1)$);
\coordinate (e3) at ($(n1)-(n2)$);


\draw (n1) -- (n2) -- (n3) -- cycle;


\tikzmath
{
    \ra = sqrt((\xb-\xc)^2 + (\yb-\yc)^2);
    \rb = sqrt((\xc-\xa)^2 + (\yc-\ya)^2);
    \rc = sqrt((\xa-\xb)^2 + (\ya-\yb)^2);
    \aire = 0.25*sqrt((\ra+\rb-\rc)*(\ra-\rb+\rc)*(-\ra+\rb+\rc)*(\ra+\rb+\rc));
}

\foreach \couche in {1,...,\ncouche}
{
    \tikzmath
    {
        \a = \couche/\ncouche;
    }
    \foreach \n in {0,...,\couche}
    {
        \tikzmath
        {
            \b = \n/\couche;
        }
        \coordinate (phi) at ($\a*(ec)+\a*\b*(ea)$);
        \coordinate (x) at ($(na)+(phi)$);
        \path let \p1=(phi) in coordinate (p) at (\y1,-\x1);
        % \draw (x) node {\(x_\n^\couche\)};
        \draw [-Latex] (x) -- ($(x)+1/\aire*(p)$);
    }
}

%%% Triangle de droite
%%% Triangle de droite
\coordinate (na) at (\xa,\ya);
\coordinate (nb) at (\xb,\yb);
\coordinate (nc) at (\xc,\yc);
\coordinate (nd) at (\xd,\yd);

\coordinate (ed) at ($(nc)-(nb)$);
\coordinate (ec) at ($(nb)-(nd)$);

\draw (nb) -- (nd) -- (nc) -- cycle;




\tikzmath
{
    \rd = sqrt((\xb-\xc)^2 + (\yb-\yc)^2);
    \rb = sqrt((\xc-\xd)^2 + (\yc-\yd)^2);
    \rc = sqrt((\xd-\xb)^2 + (\yd-\yb)^2);
    \aire = 0.25*sqrt((\rd+\rb-\rc)*(\rd-\rb+\rc)*(-\rd+\rb+\rc)*(\rd+\rb+\rc));
}

\foreach \couche in {1,...,\ncouche}
{
    \tikzmath
    {
        \a = \couche/\ncouche;
    }
    \foreach \n in {0,...,\couche}
    {
        \tikzmath
        {
            \b = \n/\couche;
        }
        \coordinate (phi) at ($\a*(ec)+\a*\b*(ed)$);
        \coordinate (x) at ($(nd)+(phi)$);
        \path let \p1=(phi) in coordinate (p) at (\y1,-\x1);
        % \draw (x) node {\(x_\n^\couche\)};
        \draw [-Latex] (x) -- ($(x)-1/\aire*(p)$);
    }
}

            \end{tikzpicture}
          \end{center}
          \begin{equation*}
            \vect{p_1}=\frac{\xi_2}{2|T|}\vect{e_1}-\frac{\xi_1}{2|T|}\vect{e_2},
          \end{equation*}
      \end{minipage}
      \begin{minipage}{0.329\textwidth}
          \setnodes{3}{0}{0}{3}{0}{0}
          \begin{center}
            \tikzsetnextfilename{fonbasloc_p_2}
            \begin{tikzpicture}[scale=1]
              %%% Triangle de gauche
%%% Triangle de gauche

\coordinate (n1) at (\xa,\ya);
\coordinate (n2) at (\xc,\yc);
\coordinate (n3) at (\xb,\yb);

\coordinate (e1) at ($(n2)-(n3)$);
\coordinate (e2) at ($(n3)-(n1)$);
\coordinate (e3) at ($(n1)-(n2)$);


\draw (n1) -- (n2) -- (n3) -- cycle;


\tikzmath
{
    \ra = sqrt((\xb-\xc)^2 + (\yb-\yc)^2);
    \rb = sqrt((\xc-\xa)^2 + (\yc-\ya)^2);
    \rc = sqrt((\xa-\xb)^2 + (\ya-\yb)^2);
    \aire = 0.25*sqrt((\ra+\rb-\rc)*(\ra-\rb+\rc)*(-\ra+\rb+\rc)*(\ra+\rb+\rc));
}

\foreach \couche in {1,...,\ncouche}
{
    \tikzmath
    {
        \a = \couche/\ncouche;
    }
    \foreach \n in {0,...,\couche}
    {
        \tikzmath
        {
            \b = \n/\couche;
        }
        \coordinate (phi) at ($\a*(ec)+\a*\b*(ea)$);
        \coordinate (x) at ($(na)+(phi)$);
        \path let \p1=(phi) in coordinate (p) at (\y1,-\x1);
        % \draw (x) node {\(x_\n^\couche\)};
        \draw [-Latex] (x) -- ($(x)+1/\aire*(p)$);
    }
}

%%% Triangle de droite
%%% Triangle de droite
\coordinate (na) at (\xa,\ya);
\coordinate (nb) at (\xb,\yb);
\coordinate (nc) at (\xc,\yc);
\coordinate (nd) at (\xd,\yd);

\coordinate (ed) at ($(nc)-(nb)$);
\coordinate (ec) at ($(nb)-(nd)$);

\draw (nb) -- (nd) -- (nc) -- cycle;




\tikzmath
{
    \rd = sqrt((\xb-\xc)^2 + (\yb-\yc)^2);
    \rb = sqrt((\xc-\xd)^2 + (\yc-\yd)^2);
    \rc = sqrt((\xd-\xb)^2 + (\yd-\yb)^2);
    \aire = 0.25*sqrt((\rd+\rb-\rc)*(\rd-\rb+\rc)*(-\rd+\rb+\rc)*(\rd+\rb+\rc));
}

\foreach \couche in {1,...,\ncouche}
{
    \tikzmath
    {
        \a = \couche/\ncouche;
    }
    \foreach \n in {0,...,\couche}
    {
        \tikzmath
        {
            \b = \n/\couche;
        }
        \coordinate (phi) at ($\a*(ec)+\a*\b*(ed)$);
        \coordinate (x) at ($(nd)+(phi)$);
        \path let \p1=(phi) in coordinate (p) at (\y1,-\x1);
        % \draw (x) node {\(x_\n^\couche\)};
        \draw [-Latex] (x) -- ($(x)-1/\aire*(p)$);
    }
}

            \end{tikzpicture}
             \begin{equation*}
              \vect{p_2}= \frac{\xi_2}{2|T|}\vect{e_1} + \frac{1-\xi_1}{2|T|}\vect{e_2},
            \end{equation*}
          \end{center}
      \end{minipage}
      \begin{minipage}{0.329\textwidth}
          \setnodes{0}{3}{0}{0}{3}{0}
          \begin{center}
            \tikzsetnextfilename{fonbasloc_p_3}
            \begin{tikzpicture}[scale=1]
              %%% Triangle de gauche
%%% Triangle de gauche

\coordinate (n1) at (\xa,\ya);
\coordinate (n2) at (\xc,\yc);
\coordinate (n3) at (\xb,\yb);

\coordinate (e1) at ($(n2)-(n3)$);
\coordinate (e2) at ($(n3)-(n1)$);
\coordinate (e3) at ($(n1)-(n2)$);


\draw (n1) -- (n2) -- (n3) -- cycle;


\tikzmath
{
    \ra = sqrt((\xb-\xc)^2 + (\yb-\yc)^2);
    \rb = sqrt((\xc-\xa)^2 + (\yc-\ya)^2);
    \rc = sqrt((\xa-\xb)^2 + (\ya-\yb)^2);
    \aire = 0.25*sqrt((\ra+\rb-\rc)*(\ra-\rb+\rc)*(-\ra+\rb+\rc)*(\ra+\rb+\rc));
}

\foreach \couche in {1,...,\ncouche}
{
    \tikzmath
    {
        \a = \couche/\ncouche;
    }
    \foreach \n in {0,...,\couche}
    {
        \tikzmath
        {
            \b = \n/\couche;
        }
        \coordinate (phi) at ($\a*(ec)+\a*\b*(ea)$);
        \coordinate (x) at ($(na)+(phi)$);
        \path let \p1=(phi) in coordinate (p) at (\y1,-\x1);
        % \draw (x) node {\(x_\n^\couche\)};
        \draw [-Latex] (x) -- ($(x)+1/\aire*(p)$);
    }
}

%%% Triangle de droite
%%% Triangle de droite
\coordinate (na) at (\xa,\ya);
\coordinate (nb) at (\xb,\yb);
\coordinate (nc) at (\xc,\yc);
\coordinate (nd) at (\xd,\yd);

\coordinate (ed) at ($(nc)-(nb)$);
\coordinate (ec) at ($(nb)-(nd)$);

\draw (nb) -- (nd) -- (nc) -- cycle;




\tikzmath
{
    \rd = sqrt((\xb-\xc)^2 + (\yb-\yc)^2);
    \rb = sqrt((\xc-\xd)^2 + (\yc-\yd)^2);
    \rc = sqrt((\xd-\xb)^2 + (\yd-\yb)^2);
    \aire = 0.25*sqrt((\rd+\rb-\rc)*(\rd-\rb+\rc)*(-\rd+\rb+\rc)*(\rd+\rb+\rc));
}

\foreach \couche in {1,...,\ncouche}
{
    \tikzmath
    {
        \a = \couche/\ncouche;
    }
    \foreach \n in {0,...,\couche}
    {
        \tikzmath
        {
            \b = \n/\couche;
        }
        \coordinate (phi) at ($\a*(ec)+\a*\b*(ed)$);
        \coordinate (x) at ($(nd)+(phi)$);
        \path let \p1=(phi) in coordinate (p) at (\y1,-\x1);
        % \draw (x) node {\(x_\n^\couche\)};
        \draw [-Latex] (x) -- ($(x)-1/\aire*(p)$);
    }
}

            \end{tikzpicture}
            \begin{equation*}
              \vect{p_3}= \frac{\xi_2-1}{2|T|}\vect{e_1} - \frac{\xi_1}{2|T|}\vect{e_2}.
            \end{equation*}
          \end{center}
      \end{minipage}
      \captionof{figure}{Fonctions \(\pj\) locales}
      \label{fig:form_int:fon_base:p}
    \end{minipage}

    \begin{prop}
      Les fonctions \(\pj\) sont dans \(\Hrot(\Gamma_h)\).
    \end{prop}
    \begin{proof}
      Nous renvoyons au travail de \cite{nedelec_mixed_1980} (voir Annexe prop.~\ref{prop:annex:hdiv_hrot:hrot}).
    \end{proof}


    \begin{defn}
      Pour tout \(\vect{V} \in \Hrot(\Gamma_h)\), alors
      \begin{equation*}
        \vect{V} \in \Vect{\vect{p_1},\ldots,\vect{p_N}} \Leftrightarrow \exists \comp{v} = (v_1,\cdots,v_N)^t \in \RR^N, \vect{V}(\vx) = \sum_{j=1}^N v_j \pj(\vx).
      \end{equation*}
    \end{defn}

  \subsection[Fonctions de Bendali q]{Fonctions de Bendali \(\vect{q}\)}

    Ces fonctions ont été introduites par \cite[eq.~28]{bendali_boundary-element_1999}. Elles sont linéairement indépendantes. Nous modifions ces fonctions en les divisant par 2 fois l'aire du triangle. Ainsi, la matrice de passage (définie plus tard) sera plus simple.


    \begin{minipage}{\textwidth}
      \begin{minipage}{0.329\textwidth}
          % Les commandes qui suivent servent pour les 4 prochains schémas
          \setnodes{0}{0}{3}{0}{0}{3}
          \begin{center}
            \tikzsetnextfilename{fonbasloc_q_1}
            \begin{tikzpicture}[scale=1]
              %%% Triangle de gauche
%%% Triangle de gauche

\coordinate (n1) at (\xa,\ya);
\coordinate (n2) at (\xc,\yc);
\coordinate (n3) at (\xb,\yb);

\coordinate (e1) at ($(n2)-(n3)$);
\coordinate (e2) at ($(n3)-(n1)$);
\coordinate (e3) at ($(n1)-(n2)$);


\draw (n1) -- (n2) -- (n3) -- cycle;


\tikzmath
{
    \ra = sqrt((\xb-\xc)^2 + (\yb-\yc)^2);
    \rb = sqrt((\xc-\xa)^2 + (\yc-\ya)^2);
    \rc = sqrt((\xa-\xb)^2 + (\ya-\yb)^2);
    \aire = 0.25*sqrt((\ra+\rb-\rc)*(\ra-\rb+\rc)*(-\ra+\rb+\rc)*(\ra+\rb+\rc));
}

% normalement il n'y pas de facteur normaliseur devant les q,
% mais sinon le schéma est trop grand

\coordinate (q) at (e1);
\coordinate (phi) at (0,0);
\coordinate (x) at ($(n1)+(phi)$);
\draw [-Latex] (x) -- ($(x)+1/\aire*(q)$);


\foreach \couche in {2,...,\ncouche}
{
    \tikzmath
    {
        \a = (\couche-1)/(\ncouche-1);
        \c = 1 - 2*\a;
    }
    \coordinate (q) at ($\c*(e1)$);
    \foreach \n in {1,...,\couche}
    {
        \tikzmath
        {
            \b = (\n-1)/(\couche-1);
        }
        \coordinate (phi) at ($\a*(e2)+\a*\b*(e1)$);
        \coordinate (x) at ($(n1)+(phi)$);
        \draw [-Latex] (x) -- ($(x)+1/\aire*(q)$);
    }
}

            \end{tikzpicture}
          \end{center}
          \begin{equation*}
            \vect{q_1}=\frac{1-2(\xi_1+\xi_2)}{2|T|}(\vect{e_2}-\vect{e_1}),
          \end{equation*}
      \end{minipage}
      \begin{minipage}{0.329\textwidth}
          \setnodes{3}{0}{0}{3}{0}{0}
          \begin{center}
            \tikzsetnextfilename{fonbasloc_q_2}
            \begin{tikzpicture}[scale=1]
              %%% Triangle de gauche
%%% Triangle de gauche

\coordinate (n1) at (\xa,\ya);
\coordinate (n2) at (\xc,\yc);
\coordinate (n3) at (\xb,\yb);

\coordinate (e1) at ($(n2)-(n3)$);
\coordinate (e2) at ($(n3)-(n1)$);
\coordinate (e3) at ($(n1)-(n2)$);


\draw (n1) -- (n2) -- (n3) -- cycle;


\tikzmath
{
    \ra = sqrt((\xb-\xc)^2 + (\yb-\yc)^2);
    \rb = sqrt((\xc-\xa)^2 + (\yc-\ya)^2);
    \rc = sqrt((\xa-\xb)^2 + (\ya-\yb)^2);
    \aire = 0.25*sqrt((\ra+\rb-\rc)*(\ra-\rb+\rc)*(-\ra+\rb+\rc)*(\ra+\rb+\rc));
}

% normalement il n'y pas de facteur normaliseur devant les q,
% mais sinon le schéma est trop grand

\coordinate (q) at (e1);
\coordinate (phi) at (0,0);
\coordinate (x) at ($(n1)+(phi)$);
\draw [-Latex] (x) -- ($(x)+1/\aire*(q)$);


\foreach \couche in {2,...,\ncouche}
{
    \tikzmath
    {
        \a = (\couche-1)/(\ncouche-1);
        \c = 1 - 2*\a;
    }
    \coordinate (q) at ($\c*(e1)$);
    \foreach \n in {1,...,\couche}
    {
        \tikzmath
        {
            \b = (\n-1)/(\couche-1);
        }
        \coordinate (phi) at ($\a*(e2)+\a*\b*(e1)$);
        \coordinate (x) at ($(n1)+(phi)$);
        \draw [-Latex] (x) -- ($(x)+1/\aire*(q)$);
    }
}

            \end{tikzpicture}
             \begin{equation*}
              \vect{q_2}=\frac{1-2\xi_1}{2|T|}\vect{e_2},
            \end{equation*}
          \end{center}
      \end{minipage}
      \begin{minipage}{0.329\textwidth}
          \setnodes{0}{3}{0}{0}{3}{0}
          \begin{center}
            \tikzsetnextfilename{fonbasloc_q_3}
            \begin{tikzpicture}[scale=1]
              %%% Triangle de gauche
%%% Triangle de gauche

\coordinate (n1) at (\xa,\ya);
\coordinate (n2) at (\xc,\yc);
\coordinate (n3) at (\xb,\yb);

\coordinate (e1) at ($(n2)-(n3)$);
\coordinate (e2) at ($(n3)-(n1)$);
\coordinate (e3) at ($(n1)-(n2)$);


\draw (n1) -- (n2) -- (n3) -- cycle;


\tikzmath
{
    \ra = sqrt((\xb-\xc)^2 + (\yb-\yc)^2);
    \rb = sqrt((\xc-\xa)^2 + (\yc-\ya)^2);
    \rc = sqrt((\xa-\xb)^2 + (\ya-\yb)^2);
    \aire = 0.25*sqrt((\ra+\rb-\rc)*(\ra-\rb+\rc)*(-\ra+\rb+\rc)*(\ra+\rb+\rc));
}

% normalement il n'y pas de facteur normaliseur devant les q,
% mais sinon le schéma est trop grand

\coordinate (q) at (e1);
\coordinate (phi) at (0,0);
\coordinate (x) at ($(n1)+(phi)$);
\draw [-Latex] (x) -- ($(x)+1/\aire*(q)$);


\foreach \couche in {2,...,\ncouche}
{
    \tikzmath
    {
        \a = (\couche-1)/(\ncouche-1);
        \c = 1 - 2*\a;
    }
    \coordinate (q) at ($\c*(e1)$);
    \foreach \n in {1,...,\couche}
    {
        \tikzmath
        {
            \b = (\n-1)/(\couche-1);
        }
        \coordinate (phi) at ($\a*(e2)+\a*\b*(e1)$);
        \coordinate (x) at ($(n1)+(phi)$);
        \draw [-Latex] (x) -- ($(x)+1/\aire*(q)$);
    }
}

            \end{tikzpicture}
            \begin{equation*}
              \vect{q_3}=\frac{2\xi_2-1}{2|T|}\vect{e_1}.
            \end{equation*}
          \end{center}
      \end{minipage}
      \captionof{figure}{Fonctions \(\qj\) locales}
      \label{fig:form_int:fon_base:q}
    \end{minipage}

    \begin{prop}
      Les fonctions \(\qj\) ne sont pas dans \(\Hrot\).
    \end{prop}
    \begin{proof}
      La composante tangentielle à l'arête le long d'une arête n'est pas nulle.
    \end{proof}

    \begin{prop}
      Les fonctions \(\qj\) forment une base de \(L^2(\Gamma_h)\).
    \end{prop}

    \begin{defn}
      Pour tout \(\vect{W} \in L^2(\Gamma_h)\), alors
      \begin{equation*}
        \vect{W} \in \Vect{\vect{q_1},\ldots,\vect{q_N}} \Leftrightarrow \exists \comp{w} = (w_1,\cdots,w_N)^t \in \RR^N, \vect{W}(\vx) = \sum_{j=1}^N w_j \qj(\vx).
      \end{equation*}
    \end{defn}

%%%%%%%%%%%%%%%%%%%%%%%%%%%%%%%%%%%%%%%%%%%%%%%%%%%%%%%%%%%%%%%%%%%%%%%%%%%%%%%%%%%%%%%%%%%%%%%%%%%%%%%%
%%%%%%%%%%%%%%%%%%%%%%%%%%%%%%%%%%%%%%%%%%%%%%%%%%%%%%%%%%%%%%%%%%%%%%%%%%%%%%%%%%%%%%%%%%%%%%%%%%%%%%%%
%%%%%%%%%%%%%%%%%%%%%%%%%%%%%%%%%%%%%%%%%%%%%%%%%%%%%%%%%%%%%%%%%%%%%%%%%%%%%%%%%%%%%%%%%%%%%%%%%%%%%%%%

\section{Matrices de projections}

  Nous allons souvent réaliser le produit scalaire entre ces fonctions. On note donc
  \begin{defn}
    Soient \(f \in \Vect{\vect{f_1},\ldots,\vect{f_N}}\) et \(g \in \Vect{\vect{g_1},\ldots,\vect{g_N}}\), on définit la matrice \(\mG{f}{g}\) telle que
    \begin{equation*}
      \mG{f}{g}_{ij} = \int_{\Gamma_h} \vect{f_i}(\vx)\cdot \vect{g_j}(\vx) \dd{\Gamma}(\vx).
    \end{equation*}
  \end{defn}

  \begin{prop}
    \begin{equation*}
      \mG{\phi}{\phi} = \mG{p}{p}
    \end{equation*}
  \end{prop}
  \begin{proof}
    C'est une conséquence immédiate du produit triangulaire \(a\cdot (b \pvect c) = c\cdot (a \pvect b)\) et que les fonctions de bases soient tangentes à la surface.
    \[
      \begin{aligned}
        \pj \cdot \pii &= (\vn \pvect \phij) \cdot ( \vn \pvect \phii),
        \\
        &= \phii \cdot ( (\vn \pvect \phij) \pvect \vn),
        \\
        &= \phii \cdot \phij.
      \end{aligned}
    \]
  \end{proof}

  \begin{defn}
    On désigne par \(\mM\) la matrice \(\mG{\phi}{\phi} = \mG{p}{p}\).
  \end{defn}

  On rappelle les décompositions sur ces familles de fonctions.
  \begin{align*}
    \vect{U}(\vx) &= \sum_j^N  u_j \phij(\vx),
    &
    \vect{V}(\vx) &= \sum_j^N  v_j \pj(\vx),
    &
    \vect{W}(\vx) &= \sum_j^N  w_j \qj(\vx).
  \end{align*}

  \begin{prop}
    Pour toute fonction \(\vect{W} \in L^2(\Gamma_h)\cap\Vect{(\qj)_j}\),

    il existe \(\vect{U} \in L^2(\Gamma_h) \cap \Vect{(\phij)_j} \) telle que \(\) avec
    \begin{equation*}
      \comp{u} = \mG{\phi}{q}\comp{w}.
    \end{equation*}
  \end{prop}

  \begin{prop}
    Pour toute fonction \(\vect{W} \in L^2(\Gamma_h)\cap\Vect{(\qj)_j}\),

    il existe \(\vect{V} \in L^2(\Gamma_h) \cap \Vect{(\pj)_j} \) telle que
    \begin{equation*}
      \comp{v} = \mG{p}{q}\comp{w}.
    \end{equation*}
  \end{prop}

  \begin{proof}
    Soit  \(\vect{W}(\vx) = \sum_{j=1}^N w_j \qj(\vx) \in L^2(\Gamma_h)\).

    On définit alors deux éléments de \(L^2(\Gamma_h)\) :
    \begin{align*}
      \vect{U}(\vx) &= \sum_{j=1}^N u_j \phij(\vx) = \sum_{j=1}^N \int_{\Gamma_h}\vect{W}(\vy)\cdot\phij(\vy)\dd{\Gamma}(\vy) \phij(\vx),
      \\
      \vect{V}(\vx) &= \sum_{j=1}^N v_j \pj(\vx) = \sum_{j=1}^N \int_{\Gamma_h}\vect{W}(\vy)\cdot\pj(\vy)\dd{\Gamma}(\vy) \pj(\vx).
    \end{align*}

    Les deux propriétés se déduisent en injectant \(\vect{W}\) dans \(\vect{U}\) et \(\vect{V}\).
  \end{proof}



%%%%%%%%%%%%%%%%%%%%%%%%%%%%%%%%%%%%%%%%%%%%%%%%%%%%%%%%%%%%%%%%%%%%%%%%%%%%%%%%%%%%%%%%%%%%%%%%%%%%%%%%
%%%%%%%%%%%%%%%%%%%%%%%%%%%%%%%%%%%%%%%%%%%%%%%%%%%%%%%%%%%%%%%%%%%%%%%%%%%%%%%%%%%%%%%%%%%%%%%%%%%%%%%%
%%%%%%%%%%%%%%%%%%%%%%%%%%%%%%%%%%%%%%%%%%%%%%%%%%%%%%%%%%%%%%%%%%%%%%%%%%%%%%%%%%%%%%%%%%%%%%%%%%%%%%%%

\section{Matrice de changement d'espace}

  Soit \(\vect{U}(\vx) = \sum_{i=1}^N u_j \phij(\vx) \in \Hdiv(\Gamma_h)\).
  Soit \(\vect{V}(\vx) = \sum_{i=1}^N v_j \pj(\vx) \in \Hrot(\Gamma_h)\)  

  Pour utiliser la CIOE \hyperlink{ci3}{CI3}, nous devons appliquer les opérateurs \gls{ope-LD} et \gls{ope-LR} sur un même objet. Or l'espace fonctionnel usuel \(\Hdiv\) ne permet pas d'appliquer \(\LR\). 

  On veut déterminer une relation pour passer de \(\Hdiv\) à \(\Hrot\).

  \begin{defn}On définit \(\mPP\) la matrice
  \label{def:eq_int:matrice_changment_mP}
    \begin{equation*}
      \mPP = (\mG{q}{p})^{-1}\mG{q}{\phi}.
    \end{equation*}
  \end{defn}

  \begin{prop}La projection de \(\Hdiv\) vers \(\Hrot\) est
    \begin{equation*}
      \Pi(\phii) = \sum_{j=1}^N \mPP_{ji}\pj.
    \end{equation*}
  \end{prop}

  \begin{proof}
    Soit \(\vect{W}(\vx) = \sum_{i=1}^N w_j \qj \in L^2(\Gamma_h)\).

    On définit
    \begin{align*}
      \vect{U}= \sum_{i=1}^N u_j \phij(\vx) && \text{où~} u_i = \int_{\Gamma_h} \vect{W}(\vx)\cdot\phii(\vx),
      \\
      \vect{V} = \sum_{i=1}^N v_j \pj(\vx) && \text{où~} v_i = \int_{\Gamma_h} \vect{W}(\vx)\cdot\pii(\vx).
    \end{align*}

    On a donc
    \begin{align*}
      \comp{u}=\mG{\phi}{q}\comp{w},
      &&
      \comp{v}=\mG{p}{q}\comp{w}.
    \end{align*}

    D'après \cite[annexe B]{stupfel_implementation_2015} la matrice \(\mG{p}{q}=1/3\mI\) donc elle est inversible et \(\comp{u} = \mG{\phi}{q} (\mG{p}{q})^{-1} \comp{v}\). On peut identifier \(\comp{u}= \mPP^{t}\comp{v}\).

  \end{proof}

    %On retrouve le résultat de \cite[annexe B]{stupfel_implementation_2015}.%, à la différence que dans cet article, un élément de \(\Hdiv\) et un élément de \(\Hrot\) sont projetés dans \(L^2\) et cet article conclu que si les projetés sont identiques alors les deux éléments initiaux représentent le même objet.

%%%%%%%%%%%%%%%%%%%%%%%%%%%%%%%%%%%%%%%%%%%%%%%%%%%%%%%%%%%%%%%%%%%%%%%%%%%%%%%%%%%%%%%%%%%%%%%%%%%%%%%%
%%%%%%%%%%%%%%%%%%%%%%%%%%%%%%%%%%%%%%%%%%%%%%%%%%%%%%%%%%%%%%%%%%%%%%%%%%%%%%%%%%%%%%%%%%%%%%%%%%%%%%%%
%%%%%%%%%%%%%%%%%%%%%%%%%%%%%%%%%%%%%%%%%%%%%%%%%%%%%%%%%%%%%%%%%%%%%%%%%%%%%%%%%%%%%%%%%%%%%%%%%%%%%%%%


\section{Forme variationnelle des équations intégrales}

  Nous ne redémontrerons pas l'obtention des équations intégrales. Nous renvoyons à \cite[Section~5.5, p.~234]{nedelec_acoustic_2001}.

  Nous rappelons notamment le théorème de trace (Theorem~5.4.2, p.~209) qui est déterminant dans les choix de discrétisations. Ce dernier stipule que si \(\vE,\vH \in \Sobolev[rot]{(\Omega)}\) alors \(\vn \pvect \vH,\vn \pvect \vE \in \Sobolev[div]^{-1/2}{(\Gamma)}\). 

  Les équations intégrales s'énoncent alors:

  Soit \(g\) la fonction de Green. Étant donné \((k,\vE^{inc},\vH^{inc})\), on cherche \(\vJ = \vn \pvect \vH, \vK = \vn \pvect \vE \in \Sobolev[div]^{-1/2}{(\Gamma)}\)  tels que pour tout \(\vect{\phi} \in \Sobolev[div]^{-1/2}(\Gamma)\),

  l'EFIE s'exprime telle que
  \begin{multline}
    \label{eq:form_int:EFIE:var}
    \int_{\Gamma} \vE^{inc}_t(\vx) \cdot \vect{\phi}(\vx) \dd{\Gamma}(\vx) =
      \frac{1}{2} \int_{\Gamma} \vE_t(\vx) \cdot \vect{\phi}(\vx) \dd{\Gamma}(\vx) \\
        - \int_{\Gamma} \left( \int_{\Gamma} \vgrad g(\vx - \vy) \pvect \vK(\vy) \dd{\Gamma}(\vy) \right) \cdot \vect{\phi}(\vx) \dd{\Gamma}(\vx) \\
      - \frac{i}{k} \int_{\Gamma} \left( \int_{\Gamma}  g(\vx - \vy)\vdivs\vJ(\vy) \dd{\Gamma}(\vy) \right)\vdivs \vect{\phi}(\vx) \dd{\Gamma}(\vx) \\
        +  ik\int_{\Gamma} \left( \int_{\Gamma} g(\vx - \vy)\vJ(\vy) \dd{\Gamma}(\vy) \right) \cdot \vect{\phi}(\vx) \dd{\Gamma}(\vx),
  \end{multline}
  et la MFIE s'exprime telle que
  \begin{multline}
    \label{eq:form_int:MFIE:var}
    \int_{\Gamma} \vH^{inc}_t(\vx) \cdot \vect{\phi}(\vx) \dd{\Gamma}(\vx) =
      \frac{1}{2} \int_{\Gamma} \vH_t(\vx) \cdot \vect{\phi}(\vx) \dd{\Gamma}(\vx) \\
        - \int_{\Gamma} \left( \int_{\Gamma} \vgrad g(\vx - \vy) \pvect \vJ(\vy) \dd{\Gamma}(\vy) \right) \cdot \vect{\phi}(\vx) \dd{\Gamma}(\vx) \\
      + \frac{i}{k} \int_{\Gamma} \left( \int_{\Gamma}  g(\vx - \vy)\vdivs\vK(\vy) \dd{\Gamma}(\vy) \right)\vdivs \vect{\phi}(\vx) \dd{\Gamma}(\vx) \\
        -  ik\int_{\Gamma} \left( \int_{\Gamma} g(\vx - \vy)\vK(\vy) \dd{\Gamma}(\vy) \right) \cdot \vect{\phi}(\vx) \dd{\Gamma}(\vx).
  \end{multline}

  Ceci est l'expression continue de la formulation variationnelle des équations intégrales. Dans la suite nous approcherons cette forme variationnelle à l'aide des familles de fonction présentées précédemment. De plus, dans un abus de notation, nous omettrons le \(-1/2\) dans l'espace \(\Sobolev[div]^{-1/2}(\Gamma)\equiv\Sobolev[div](\Gamma)\).

%%%%%%%%%%%%%%%%%%%%%%%%%%%%%%%%%%%%%%%%%%%%%%%%%%%%%%%%%%%%%%%%%%%%%%%%%%%%%%%%%%%%%%%%%%%%%%%%%%%%%%%%
%%%%%%%%%%%%%%%%%%%%%%%%%%%%%%%%%%%%%%%%%%%%%%%%%%%%%%%%%%%%%%%%%%%%%%%%%%%%%%%%%%%%%%%%%%%%%%%%%%%%%%%%
%%%%%%%%%%%%%%%%%%%%%%%%%%%%%%%%%%%%%%%%%%%%%%%%%%%%%%%%%%%%%%%%%%%%%%%%%%%%%%%%%%%%%%%%%%%%%%%%%%%%%%%%

\section{Contribution de la CIOE dans la discrétisation de la forme variationnelle}

  On prend comme fonction test \(\phii\), la fonction de base associée à l'arête \(i\) qui appartient à \(\Hdiv(\Gamma_h)\).
  On décompose \(\vJ,\vK \in \Hdiv(\Gamma)\) sur les fonctions de base \(\phij \in \Hdiv(\Gamma_h)\).
    \begin{align*}
      \vJ & \simeq \sum_{j=1}^N I_j \phij,
      &
      \vK & \simeq \sum_{j=1}^N k_j \phij.
    \end{align*}

  Il reste alors à trouver une bonne approximations de \(\vE_t\) et \(\vH_t\) pour exprimer \( \frac{1}{2} \int_{\Gamma_h} \vE_t(\vx) \cdot \phii(\vx) \dd{\Gamma}(\vx)\) et \(\frac{1}{2} \int_{\Gamma_h} \vH_t(\vx) \cdot \phii(\vx) \dd{\Gamma}(\vx)\) en fonctions des inconnues \(\vJ, \vK\). Selon la CIOE employée, cette décomposition ne sera pas la même. Notre but est alors de trouver une formulation qui se comporte bien quand on déduit d'une CIOE un autre plus simple. C'est-à-dire que la formulation intégrale déterminée avec la CI3 doit être équivalente à la formulation intégrale déterminée avec la CI0 quand les coefficients \(a_1,a_2,b_1,b_2\) sont nuls. En effet, à cause des opérateurs différentiels, nous ne représenterons pas \(\vE_t\) et \(\vH_t\) à l'aide d'une seule famille, mais potentiellement plusieurs familles. Nous préciserons lesquels dans la suite.

  \subsection[Discrétisation des opérateurs LD, LR]{Discrétisation des opérateurs \(\LD,\LR\)}
    Les CIOE font intervenir les opérateurs différentiels \gls{ope-LD}, \gls{ope-LR}. On introduit donc les matrices \(\mLD,\mLR\) correspondantes.

    Comme les \(\phij\) sont dans \(\Hdiv\), de l'opérateur \gls{ope-LD} on définit la matrice \(\mLD\)
    \begin{defn}[Discrétisation de l'opérateur \(\LD\)]
      On définit la matrice \(\mLD\) telle que
      \begin{align*}
          \mLD_{ij} = - \sum_{j=1}^N \int_{\Gamma_h} \vdivs \phij(\vx) \vdivs \phii(\vx) \dd{\Gamma}(\vx).
      \end{align*}
    \end{defn}

    Par définition on a \(\mLD_{ij} = \int_{\Gamma_h} \LD(\phij)(\vx)\cdot\phii(\vx)\dd{\Gamma}(\vx)\).

    De plus, par définition des fonctions \(\phij\), on ne peut pas définir \( \int_{\Gamma_h} \LR(\phii)(\vx) \cdot \phij(\vx) \dd{\Gamma}(\vx)\) puisqu'elles ne sont pas dans \(\Hrot\).

    On projette alors les fonctions \(\phij\) sur les fonctions \(\pj\) grâce aux \(\qj\).
    \begin{defn}[Discrétisation de l'opérateur \(\LR\)]
      Soit \(\mPP\) la matrice introduite à la définition \ref{def:eq_int:matrice_changment_mP}. Alors on définit la matrice
      \begin{align*}
        \mLR & = -\mPP^t\mLD\mPP.
      \end{align*}
    \end{defn}
    \begin{prop}
      Cette matrice approche l'action de \(\LR\) sur les \(\phij\).
    \end{prop}
    \begin{proof}
      Cette propriété est démontrée par \cite[eq.~(17)]{stupfel_implementation_2015}. On projette les fonctions \(\phij\) sur les fonctions \(\pj\).
      \begin{align*}
        \int_{\Gamma_h} \phii(\vx) \cdot \LR(\phij)(\vx) \dd{\Gamma}(\vx) &\simeq \int_{\Gamma_h} \Pi(\phii)(\vx) \cdot \LR(\Pi(\phij)) (\vx) \dd{\Gamma}(\vx),
        \\
        &=\sum_{n=1}^{N} \sum_{m=1}^{N} \mPP_{ni} \mPP_{mj} \int_{\Gamma_h}\pn(\vx)\cdot\LR(\ppm)) \dd{\Gamma}(\vx),
        \\
        &=-\sum_{n=1}^{N} \sum_{m=1}^{N} \mPP_{ni} \mPP_{mj} \int_{\Gamma_h}\phin(\vx)\cdot\LD(\phim)) \dd{\Gamma}(\vx),
        \\
        &=-\sum_{n=1}^{N} \sum_{m=1}^{N} \mPP_{ni} \mPP_{mj} \mLD_{nm}.
      \end{align*}
    \end{proof}

  \subsection{Contribution de la CIOE à l'EFIE}

    On déduit de la CIOE que 
    \begin{align*}
        \frac{1}{2} \int_{\Gamma_h} \vE_t(\vx) \cdot \phii(\vx) \dd{\Gamma}(\vx)
        &= \frac{1}{2} \int_{\Gamma_h}  \left( \left(\oI + b_1 \LD - b_2 \LR\right) \vJ (\vx)\right.
        \\
        & \hspace{5em} \left.  - \left( b_1 \LD - b_2 \LR\right) \vE_t (\vx) \right) \cdot \phii(\vx) \dd{\Gamma}(\vx).
        \intertext{Nous rappelons de \cite[Théorème~5.4.2, p.~209]{nedelec_acoustic_2001} que \((\vE, \vH) \in \Sobolev[rot]{(\OO)}\) donc  \((\vK=\vn \pvect \vE, \vJ=\vn \pvect \vH) \in \Sobolev[div]{(\Gamma)}\) et \((\vE_t, \vH_t) \in \Sobolev[rot]{(\Gamma)}\). Cependant, nous avons choisis (voir \cite[eq.~10, p.~1660]{stupfel_implementation_2015}) d'approcher \(\vE_t\) dans \(\Sobolev[div]{(\Gamma_h)}\) pour le décomposer sur les fonctions de Raviart-Thomas. Ainsi}
        & \simeq \frac{1}{2} \sum_{j=1}^N \int_{\Gamma_h} \left( \left(\oI + b_1 \LD - b_2 \LR\right) I_j \phij (\vx)\right.
        \\
        & \hspace{5em} \left.  - \left( b_1 \LD - b_2 \LR\right) e_j \phij (\vx) \right) \cdot \phii(\vx) \dd{\Gamma}(\vx),
        \intertext{ce qui s'exprime matriciellement}
        & = \frac{1}{2} \left( \left(\mG{\phi}{\phi} + b_1 \mLD - b_2 \mLR\right) \comp{I}  - \left( b_1 \mLD - b_2 \mLR\right) \comp{e}\right).
    \end{align*}

    Or comme \(\vK = \vn \pvect \vE\), alors
    \begin{align*}
      \sum_{j=1}^N e_j \vn \pvect \phij & \simeq \sum_{j=1}^N k_j \phij,
      \intertext{ et on alors la relation entre les composantes de l'approximation de ces deux vecteurs}
      \comp{e} \simeq \mPP \comp{k}.
    \end{align*}
    Donc la contribution des CIOE à l'EFIE est
    \begin{equation*}
      \begin{aligned}
        \frac{1}{2} \int_{\Gamma_h} \vE_t(\vx) \cdot \phii(\vx) \dd{\Gamma}(\vx)
        & \simeq \frac{1}{2} \left( \left(\mM + b_1 \mLD - b_2 \mLR\right) \comp{I}  - \left( b_1 \mLD - b_2 \mLR\right) \mPP \comp{m}\right).
      \end{aligned}
    \end{equation*}

  \subsection{Contribution de la CIOE à la MFIE}

    On commence par remarquer que
    \begin{align*}
      \frac{1}{2}\int_{\Gamma_h} \vH_t(\vx) \cdot \phii(\vx) \dd{\Gamma}(\vx)
      &= \frac{1}{2}\int_{\Gamma_h} \vJ(\vx) \cdot (\vn \pvect \phii)(\vx) \dd{\Gamma}(\vx),
      \intertext{ dont on déduit à l'aide de la CIOE que }
      &= \frac{1}{2a_0}\int_{\Gamma_h} \left( \left(\oI + b_1 \LD - b_2 \LR\right) \vE_t (\vx) \right.
      \\
      & \hspace{5em} \left.- \left(a_1 \LD - a_2 \LR\right) \vJ (\vx) \right) \cdot (\vn \pvect \phii)(\vx) \dd{\Gamma}(\vx).
      \\
      \intertext{Nous remplaçons \(\vE_t\) par \(-\vn \pvect \vK\) ce qui va simplifier la discrétisation}
      &= \frac{1}{2a_0}\int_{\Gamma_h} \left( -\left(\oI + b_1 \LD - b_2 \LR\right) \vn \pvect \vK (\vx) \right.
      \\
      & \hspace{5em} \left.- \left(a_1 \LD - a_2 \LR\right) \vJ (\vx) \right) \cdot (\vn \pvect \phii)(\vx) \dd{\Gamma}(\vx),
      \intertext{nous approchons \(\vK\) dans \(\Sobolev[div]{(\Gamma_h)}\) mais nous avons choisis d'approcher \(\vJ\) dans \(\Sobolev[rot]{(\Gamma_h)}\) au lieu de \(\Sobolev[div]{(\Gamma_h)}\)}
      & \simeq \frac{1}{2a_0}\sum_{i=1}^N \int_{\Gamma_h} \left( -\left(\oI + b_1 \LD - b_2 \LR\right) k_j \vn \pvect   \phij (\vx) \right.
      \\
      & \hspace{5em} \left. - \left(a_1 \LD - a_2 \LR\right) I_j' \vn \pvect \phij (\vx) \right) \cdot (\vn \pvect \phii)(\vx) \dd{\Gamma}(\vx).
      \intertext{En remarquant que \(\vn \pvect \phi_i \cdot \vn \pvect \phi_j = \phi_i \cdot \phi_j\) et \(\int_{\Gamma} \LR(\vn \pvect \vect{\phi})\cdot (\vn \pvect \vect{\psi}) = - \int_{\Gamma} \LD(\vect{\phi})\cdot \vect{\psi}\), on déduit}
      & \simeq \frac{1}{2a_0}\sum_{i=1}^N \int_{\Gamma_h} \left( -\left(\oI + b_1 \LR - b_2 \LD\right) k_j \phij (\vx) \right.
      \\
      & \hspace{5em} \left. - \left(a_1 \LR - a_2 \LD\right) I_j' \phij (\vx) \right) \cdot \phii(\vx) \dd{\Gamma}(\vx),
      \intertext{ce qui s'exprime matriciellement}
      & = \frac{1}{2a_0} \left( -\left(\mG{\phi}{\phi} + b_1 \mLR - b_2 \mLD\right) \comp{k}  - \left( a_1 \mLR - a_2 \mLD\right) \comp{I'}\right),
      \\
      & \simeq -\frac{1}{2a_0} \left( \left(\mM + b_1 \mLR - b_2 \mLD\right) \comp{k}  + \left( a_1 \mLR - a_2 \mLD\right) \mPP \comp{I} \right).
    \end{align*}
    Donc la contribution des CIOE à la MFIE est
    \begin{equation*}
      \begin{aligned}
        \frac{1}{2}\int_{\Gamma_h} \vH_t(\vx) \cdot \phii(\vx) \dd{\Gamma}(\vx)
        &\simeq -\frac{1}{2a_0} \left( \left(\mM + b_1 \mLR - b_2 \mLD\right) \comp{k}  + \left( a_1 \mLR - a_2 \mLD\right) \mPP \comp{I} \right).
      \end{aligned}
    \end{equation*}

%%%%%%%%%%%%%%%%%%%%%%%%%%%%%%%%%%%%%%%%%%%%%%%%%%%%%%%%%%%%%%%%%%%%%%%%%%%%%%%%%%%%%%%%%%%%%%%%%%%%%%%%
%%%%%%%%%%%%%%%%%%%%%%%%%%%%%%%%%%%%%%%%%%%%%%%%%%%%%%%%%%%%%%%%%%%%%%%%%%%%%%%%%%%%%%%%%%%%%%%%%%%%%%%%
%%%%%%%%%%%%%%%%%%%%%%%%%%%%%%%%%%%%%%%%%%%%%%%%%%%%%%%%%%%%%%%%%%%%%%%%%%%%%%%%%%%%%%%%%%%%%%%%%%%%%%%%

\section{Forme finale du système linéaire}

  \begin{defn}
    On définit les matrices \(\mS\), \(\mSt\) telles que

    \begin{align*}
      S_{pq}
        &= ik \int_{\Gamma_h} \left( \int_{\Gamma_h} g(\vx - \vy) \phip(\vy) \dd{\Gamma}(\vy) \right) \cdot \phiq (\vx)\dd{\Gamma}(\vx)\\
        \notag&~+ \frac{i}{k}\int_{\Gamma_h} \left( \int_{\Gamma_h} g(\vx - \vy) \vdiv \phip (\vy) \dd{\Gamma}(\vy) \right) \vdiv \phiq(\vx) \dd{\Gamma}(\vx),
      \\
      \tilde{S}_{pq}
        &= \int_{\Gamma_h} \left( \int_{\Gamma_h} \vgrad g(\vx - \vy) \pvect \phip(\vy) \dd{\Gamma}(\vy) \right) \cdot \phiq(\vx) \dd{\Gamma}(\vx).
    \end{align*}
  \end{defn}

  \begin{prop}
    Les matrices \(\mS\) et \(\mSt\) sont symétriques.
  \end{prop}
  \begin{proof}
    C'est une conséquence immédiate des propriétés de l'intégrale et de la parité de la fonction de Green.
  \end{proof}

  Le second membre est obtenu en projetant le champ incident
    \begin{align*}
      b_i^E &= \int_{\Gamma_h} \vE_t^{inc}(\vx) \cdot \phii(\vx)\dd{\Gamma}(\vx), \\
      b_i^H &= \int_{\Gamma_h} \vH_t^{inc}(\vx) \cdot \phii(\vx)\dd{\Gamma}(\vx).
    \end{align*}

  \begin{prop}
    \label{prop:form_int:ci3}
    On cherche les vecteurs complexes \(\comp{I},\comp{k}\) solution de
    \begin{equation*}
      \begin{bmatrix}
        \mS +  \frac{1}{2} \left(a_0 \mM + a_1 \mLD - a_2 \mLR\right) & \mSt + \frac{1}{2}\left(b_1 \mLD - b_2 \mLR \right)\mPP \\
        \mSt + \frac{1}{2a_0}\left( a_2 \mLD - a_1 \mLR \right)\mPP & -\mS - \frac{1}{2a_0}\left(\mM + b_1 \mLR - b_2 \mLD \right)
      \end{bmatrix}
      \begin{bmatrix}
        \comp{I}\\
        \comp{k}
      \end{bmatrix}
      =
      \begin{bmatrix}
        \comp{b^E}\\
        \comp{b^H}
      \end{bmatrix}.
    \end{equation*}

    Ces vecteurs sont les composantes dans la famille des \(\phij\) de l'approximation de \(\vJ,\vK\), les inconnues de l'EFIE et de la MFIE.
  \end{prop}

  On remarque que le système n'est pas symétrique, sauf si \(b_2=a_2/a_0\) et \(b_1=a_1/a_0\) et que la contribution de la CIOE n'introduit que des matrices creuses. 
  C'est une amélioration par rapport aux travaux initiaux de \cite{stupfel_implementation_2015} où l'on introduisait une matrice dense dans le bloc inférieur droit. 
  Par contre, dans cet article la matrice était symétrique. Nous n'avons pas pu bénéficier des solveurs haute-performance du CEA, optimisés pour les systèmes symétriques. 
  Les résultats numériques qui suivent sont donc issus d'un code maquette où le système linéaire est codé tel quel, ce qui impose des limites évidentes en coût mémoire. 
\section{Résultats numériques}

\begin{TODO}
  courbes de waves
\end{TODO}

courbes de SER EQ int + CIOE vs code axis
