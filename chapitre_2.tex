\chapter[Unicité des solutions de Maxwell]{Conditions d'unicité des solutions du problème de Maxwell-Helmholtz}
\label{sec:csu}
\minitoc
\newpage
\sectionstar{Introduction}
Nous présentons ici la contribution principale de la thèse : l'établissement de \glsentrydescplural{acr-csu} des solutions des équations de Maxwell.
Nous étendons les travaux entamés par \cite{stupfel_sufficient_2011} à une nouvelle \glsentrydesc{acr-cioe}, la \hyperlink{ci3}{CI3}.
L'établissement de CSU pour cette CIOE est intéressant, car comme nous l'observerons dans les chapitres suivants, elle est bien plus performante que les CIOE de \cite{stupfel_sufficient_2011}.
\section{Une condition suffisante assurant l'unicité des solutions du problème de Maxwell extérieur}
  On s’intéresse à la propagation des ondes électromagnétiques à l'extérieur d'un objet fermé borné, sans source.
  Les champs associés sont solutions des équations de Maxwell harmoniques, la convention dans toute la thèse étant en \(e^{i\w t}\). 

  % \begin{REM}
  %   Remplacer partout "l'unicité des solutions" par "l'unique solution est 0".
  % \end{REM}

  Puisque le problème est sans source, nous allons déterminer une condition suffisante qui assure que seuls les champs nuls sont solutions de ce problème.

  % \begin{REM}
  %     Régularité de \(\Gamma\) ? Attention, comment définir l'extérieur sans convexité ( bouteille de Klein ) ?
  % \end{REM}
  % \begin{REP}
  %     Je n'ai étudié que des objets convexes, mais la régularité était au pire \(C^1\) : cône-sphère.
  % \end{REP}

  Soit \(\gls{mat-ga}=\partial\overline{\gls{mat-om}}\) la frontière de \(\OO\) que l'on suppose régulière (au moins \(\mathcal{C}^2\)), on décompose donc tout l'espace entre l'intérieur et l'extérieur, soit 
  \[
    \RR^3 = \overline{\OO}\cup{\OO}^c.
  \]
  On définit en tout point de \(\Gamma\), supposée régulière, \(\gls{mat-vn}\) la normale unitaire sortante à \(\OO\).
  Dans \(\OO^c\) on définit \(\gls{phy-k0}=\gls{phy-w} / \gls{phy-c} \in \RR_+^*\), le nombre d'onde dans le vide, où \gls{phy-c} 
  est la célérité d'une onde dans le vide.
  On s’intéresse à la propagation des ondes électromagnétiques à l'extérieur d'un objet fermé borné, sans source.

  Les champs associés sont solutions des équations de Maxwell harmoniques, la convention dans toute la thèse étant en \(e^{i\w t}\). 

  % Soit \((\vE,\vH)\) dans \((\Hrot(\OO^c)) \times (\Hrot(\OO^c))\) tels que:
  Nous utilisons les équations de Maxwell-Helmholtz (Helmholtz car utilisent le nombre d'onde \(k_0\) et non la pulsation \(\w\)) où les inconnues sont le champ électrique \gls{phy-e} et l'excitation magnétique \gls{phy-h2} dans le vide, elles se déduisent du système d'équations de Maxwell (voir annexe \ref{sec:annex:maxwell_equation}). 

  Soit \((\gls{phy-e},\gls{phy-h2})\) dans \((\mathcal{C}_0^\infty(\OO^c)\cap\Hrot(\OO^c))^2\) tels que
  \begin{align}
  \left\lbrace
    \begin{matrix}
      \vrot \vE + i k_0 \vH &= 0
      \\
      \vrot \vH - i k_0 \vE &= 0
    \end{matrix}
    \right. && \text{dans \(\OO^c\).}
    \label{eq:unicite:maxwell_ext_2}
  \end{align}

  %  En l'absence de source, le rotationnel d'un des champs a la même régularité que le champs lui-même, donc ces champs sont \(\mathcal{C}^\infty\).

  Pour étudier ce problème numériquement, on se ramène à un ouvert borné de la manière suivante : soit \(R\in\RR_+^*\) tel que le domaine \(\overline{\Omega}\) soit inclus dans \(B_R\), la boule de rayon \(R\).
  On cherche alors les solutions de \eqref{eq:unicite:maxwell_ext_2} dans \(\OO^c_R=\OO^c\cap B_R\) avec \(\partial\OO^c_R = \Gamma\cup S_R\), \(S_R\) étant la sphère de rayon \(R\) où l'on écrit la condition de radiation sortante
  \begin{equation}
    \label{eq:unicite:TR}
    \begin{aligned}
      \Tr(\vE_t) = - \vn_{B_R} \pvect \vH && \text{sur \(S_R\).}
    \end{aligned}
  \end{equation}
  \(\Tr\) est l'opérateur de capacité introduit par \cite[p.~200]{nedelec_acoustic_2001}, \(\vn_{B_R}=\frac{1}{R}\vect{OM}\) la normale unitaire sortante à \(B_R\).

  % Quand il est bien posé, le problème \eqref{eq:unicite:maxwell_ext_2} sans source muni de la condition \eqref{eq:unicite:TR} est équivalent à
  % \begin{REM}
  % Quand le problème est bien posé
  % \end{REM}
  \begin{prop}[Formulation variationnelle du problème de Maxwell extérieur sans source]
    Le champ électrique \(\vE\), solution du problème \eqref{eq:unicite:maxwell_ext_2} sans source muni de la condition \eqref{eq:unicite:TR}, est solution du problème

    Soit \(V = \lbrace \vect{u} \in (\mathcal{C}_0^\infty(\OO^c_R)\cap\Hrot(\OO^c_R) \rbrace\).

    Trouver \(\vE \in V\), tel que
    \begin{align*}
      a(\vE,\vect\phi) = 0 && \forall \vect \phi \in V,
    \end{align*}

    où \(a\) est la forme sesquilinéaire donnée par
    \begin{equation*}
      \begin{aligned}
      a(\vE,\vect\phi) &:=  \frac{1}{ik_0} \int_{\OO^c_R} \trot \vE(\vx) \cdot \trot \conj{\vect{\phi}}(\vx) \dd{x} + ik_0\int_{\OO^c_R}\vE(\vx)\cdot\conj{\vect{\phi}}(\vx) \dd{x}
        \\ 
        & \quad + \int_{S_R} \conj{ \vect \phi }(\vx) \cdot \Tr(\vE_t)(\vx) \dd{S_R}(\vx) - \int_\Gamma \left(\vn \pvect \frac{\trot \vE(\vx)}{ik_0}\right) \cdot \conj{\vect \phi} (\vx) \dd{\Gamma}(\vx).
      \end{aligned}
    \end{equation*}
  \end{prop}
  % \begin{REM}
  %     Ici, tu viens seulement de démontrer \(\Rightarrow\). Tu n'as pas démontré l'équivalence. A faire ? Ou supprimer l'équivalence.
  % \end{REM}
  % \begin{REP}
  %   Mis une ref vers Cessenat pour la réciproque.
  % \end{REP}
  \begin{proof} Sens direct.
    Nous omettons les dépendances en \(\vx\).
    Partons de la première équation de Maxwell
    \begin{align*}
      0 & = \frac{\trot \vE}{ik_0} + \vH.
      \intertext{En appliquant cette égalité à \(\trot\vect{\phi}\) pour \(\vect{\phi} \in V\), on déduit la forme variationnelle }
      0 & = \int_{\OO^c_R} \frac{\trot \vE}{ik_0} \cdot \trot \conj{\vect{\phi}}\dd{x} + \int_{\OO^c_R} \vH \cdot \trot\conj{\vect{\phi}}\dd{x}, && \forall \vect{\phi} \in V.
      \intertext{Utilisons la formule de Green du rotationnel (voir \cite[eq.~(A1.32)]{bladel_electromagnetic_2007}) où \(W\) est un ouvert bornée de frontière \(S\).
      \[
        \int_W \vect{u}\cdot \trot{\vect{v}}\dd{x} = \int_W \trot\vect{u}\cdot \vect{v}\dd{x} + \int_{\partial W} \left( \vect{v}\pvect\vect{u} \right)\cdot \vn_{S} \dd{S(x)},
      \]
      où \(\vn_{S}\) est la normale unitaire sortante de \(W\).
      On rappelle que \(\vn_\Gamma\) est la normale sortante de  \(\OO\) et \(\vn_{B_R}\) la normale sortante de \(B_R\).}
      0 & = \int_{\OO^c_R} \frac{\trot \vE}{ik_0} \cdot \trot\vect{\phi}\dd{x} +\int_{\OO^c_R} \trot\vH \cdot \conj{\vect{\phi}}\dd{x}
      \\
      & \qquad \qquad+ \int_{S_R} \left( \conj{\vect \phi} \pvect \vH\right)  \cdot \vn_{B_R}\dd{S_R(x)} - \int_\Gamma \left( \conj{\vect \phi} \pvect \vH\right)  \cdot \vn_\Gamma \dd{\Gamma(x)}, && \forall \vect{\phi} \in V.
      \intertext{On utilise  la deuxième équation de Maxwell, la condition de rayonnement et on permute les termes de l'intégrale sur \(\Gamma\)}
      0 & =\int_{\OO^c_R} \frac{\trot \vE}{ik_0}\cdot \trot \conj{\vect{\phi}}\dd{x}  +  \int_{\OO^c_R} ik_0 \vE \cdot \conj{\vect{\phi}}\dd{x}
      \\
      & \qquad \qquad + \int_{S_R} \Tr(\vE_t)  \cdot \conj{\vect{\phi}}\dd{S_R(x)} + \int_\Gamma \left(\vn_\Gamma \pvect \vH \right) \cdot \conj{\vect \phi}\dd{\Gamma(x)},
      && \forall \vect{\phi} \in V.
    \end{align*}

    Réciproque: voir \cite[p.~121, section~5, "SCATTERING PROBLEMS BY A DIELECTRIC OBSTACLE"]{cessenat_mathematical_1996}.
\begin{REM}
  quelle motivation pour faire le sens direct et négliger la réciproque ? car l'un n'est pas plus original que l'autre :)
\end{REM}
\begin{REP}
  Je ne veux pas m'embarquer la dedans, mais prévoir pour la soutenance.
\end{REP}
  \end{proof}

  Le problème n'a pas de source et l'unique solution du problème \eqref{sec:annex:maxwell_equation} si ce dernier est bien posé est \(\vE=\vH=0\).

  On définit en tout point de \(\Gamma\) la trace tangentielle de \(\vH\) que l'on note \(\vJ = \vn_\Gamma \pvect \vH\).
  \begin{prop}[Une \glsentrydesc{acr-csu}]~\\
    Si l'on suppose que
    \begin{equation}
      \label{eq:unicite:form_var:cgu}
      \Re\left(\int_\Gamma \vJ(\vx) \cdot \conj{\vE_t}(\vx) \dd{\Gamma(\vx)}\right) \ge 0,
    \end{equation}
    alors le système \eqref{eq:unicite:maxwell_ext_2} muni de la condition aux limites \eqref{eq:unicite:TR} admet \(\vE=\vH=0\) comme unique solution.
  \end{prop}

\begin{proof}
  Pour démontrer cela, on utilise le lemme de Rellich, énoncé dans \cite[p.~74]{cessenat_mathematical_1996}:
  \begin{lemme}[Lemme de Rellich]
    Soit \(\OO^c\) un domaine connexe, complément d'un domaine borné, et soit \(u\) satisfaisant
    \begin{subequations}
      \begin{align}
        \Delta u + k^2 u = 0 & &\text{dans \(\OO^c\)},
        \\
        \lim_{R\rightarrow\infty}\int_{S_R} |u(\vx)|^2 \dd{S_R}(\vx) = 0,
      \end{align}
    \end{subequations}
    alors \(u=0\) dans \(\OO^c\).
  \end{lemme}
  \begin{proof}
    Voir \cite[p.~74]{cessenat_mathematical_1996}.
  \end{proof}

  On définit les quantités suivantes
  \begin{align}
    X &= \int_\Gamma \vJ(\vx) \cdot \conj{\vE_t}(\vx)\dd{\Gamma(\vx)},
    \label{eq:unicite:x}
    \\
    C &= \int_{S_R} \Tr(\vE_t)(\vx)  \cdot \conj{\vE_t}(\vx)\dd{S_R(\vx)}.
  \end{align}

  \begin{align*}
    \intertext{De}
    a(\vE,\vE) &= \frac{1}{ik_0} \int_{\OO^c_R} \norm{\trot \vE(\vx)}^2 \dd{x} + ik_0\int_{\OO^c_R}\norm{\vE(\vx)}^2 \dd{x}
     + X + C*
     \\
    &= 0,
    \intertext{Donc}
    \Re(a(\vE,\vE)) & = 0,
    \intertext{ce qui se réecrit}
    \Re(C) + \Re(X) & = 0.
    \intertext{Comme d'après \cite[Théorème~5.3.5, p.~200]{nedelec_acoustic_2001} \(\Re(C) \ge 0\), on en déduit} 
    \Re(X) & \le 0.
  \end{align*}
  L'hypothèse \eqref{eq:unicite:form_var:cgu} \(\Re(X) \ge 0\) impose alors \(\Re(X)=0\) et donc
  \begin{align*} 
    \Re(C) &= 0,
    \intertext{donc d'après le théorème 5.3.5 de Nédélec}
    \vE_t &= 0 && \text{sur \(S_R\)},
    \intertext{et d'après le lemme de Rellich}
    \vE &= 0 && \text{dans \(\OO^c_R\)}.
  \end{align*}

  On en déduit \(\vH = 0 \).

  On conclut donc que si on suppose \eqref{eq:unicite:form_var:cgu}, alors en l'absence de sources le couple \((\vE,\vH)=(0,0)\) est l'unique solution, ce qui démontre l'unicité par linéarité.
\end{proof}

\section[Des CSU pour les CIOE de Stupfel et Poget 2011]{Des conditions suffisante pour les CIOE de \cite{stupfel_sufficient_2011}}

  Nous définissons les \glspl{acr-cioe} comme une condition limite liant \(\vE_t\) et \(\vn\pvect\vH\) sur \(\Gamma\). L'existence des CIOE est en dehors du cadre de cette thèse, nous ne ferons donc qu'utiliser des CIOE existantes.

  Grâce à ces CIOE, nous allons établir des conditions suffisante qui impliquent la \gls{acr-cgu} \eqref{eq:unicite:form_var:cgu}. Par leur nature suffisante, il n'y pas un unique jeu de CSU pour une CIOE donnée. Une difficulté est d'être capable de juger si ce jeu est satisfaisant, et si ce n'est pas le cas, être capable de proposer un autre jeu.

  Les CIOE de \cite{stupfel_sufficient_2011} font intervenir l'opérateur de Hodge \(\mathcal{L}\), commençons par rappeler son expression et quelques propriétés.

  \begin{defn}
    % Pour tout \(\vu \in (\mathcal{C}^\infty(\Gamma))^2\)
    \begin{equation}
      \label{eq:operator:L}
        \fonction{\LL}{(\mathcal{C}^\infty(\Gamma))^2}{(\mathcal{C}^\infty(\Gamma))^2}%
          {\vu}{\tgrads{\tdivs \vu} - \tvrots{\tvrots \vu}}
    \end{equation}
  \end{defn}

  \begin{prop}
    Par définition, l’opérateur hermitien \(\LL\) est symétrique négatif.

    Pour tous \(\vu,\vv \in (\mathcal C^\infty(\Gamma))^2\)
    \begin{align}
      \int_\Gamma \vu\cdot \LL(\conj{\vv}) &= \int_\Gamma \conj{\vv}\cdot \LL(\vu)
      \\
      \int_\Gamma \vu\cdot \LL(\conj{\vu}) &= -\norm{\vgrads{\vu}} \le 0
      \label{eq:hodge:negatif}
    \end{align}
  \end{prop}

  On rappelle que l'on veut trouver des conditions permettant de garantir \eqref{eq:unicite:form_var:cgu} soit \(\Re(X)\ge0\) où \(X = \int_\Gamma \vJ \cdot \conj{\vE_t}\).

  %%%%%%%%%%%%%%%%%%%%%%%%%%%%%%%%%%%%%%%%%%%%%%%%%%
  \subsection{CSU de la CI0}
    Considérons la condition d’impédance de Leontovich, la \hyperlink{ci0}{CI0} caractérisé par
    \begin{align}
      \label{eq:unicite:ci0}
      \vE_t = a_0 \vJ && \forall a_0 \in \CC
    \end{align}

    \begin{defn}
      \label{def:csu:ci0}
      On définit le sous-espace fermé de \(\CC\)
      \begin{equation*}
        \CSU{CI0} = \lbrace a_0 \in \CC; \Re(a_0) \ge 0 \rbrace
      \end{equation*}
    \end{defn}

    \begin{prop}[Une CSU pour la CI0]
      \label{prop:csu:ci0}
      Il suffit que
      \begin{equation*}
        a_0 \in \CSU{CI0}
      \end{equation*}
      pour que \(\Re(X)\ge 0\), ce qui entraîne l'unicité.
    \end{prop}
    \begin{proof}
      On a \( X = \conj{a_0}\norm{\vJ}^2\) donc \(\Re(X) = \Re(a_0)\norm{\vJ}^2 \)
    \end{proof}
  %%%%%%%%%%%%%%%%%%%%%%%%%%%%%%%%%%%%%%%%%%%%%%%%
  \subsection{CSU de la CI01}
    Considérons la condition d’impédance \hyperlink{ci01}{CI01}:
    \begin{align}
      \label{eq:unicite:ci01}
      \vE_t = (a_0\oI + a_1 \LL)\vJ && \forall (a_0, a_1) \in \CC^2
    \end{align}

    \begin{defn}
      \label{def:csu:ci01}
      On définit le sous-espace fermé de \(\CC^2\)
      \begin{equation*}
        \CSU{CI01} = \left\lbrace (a_0,a_1) \in \CC^2; \begin{matrix}
        \Re\left(a_0\right) \ge 0
        \\
        \Re\left(a_1\right) \le 0
        \end{matrix}
        \right\rbrace
      \end{equation*}
    \end{defn}

    \begin{prop}[Une CSU pour la CI01]
      \label{prop:csu:ci01}
      Il suffit que
      \begin{equation*}
        (a_0,a_1) \in \CSU{CI01}
      \end{equation*}
      pour que \(\Re(X)\ge 0\), ce qui entraîne l'unicité.
    \end{prop}
    \begin{proof}
      On a
      \begin{align*}
        X & = \conj{a_0} \norm{\vJ} ^2 - \conj{a_1} \norm{\vgrads{\vJ}}^2
        \intertext{donc}
        \Re(X) & = \Re{(a_0)} \norm{\vJ} ^2 - \Re{(a_1)}\norm{\vgrads{\vJ}}^2
      \end{align*}
      Si on suppose \((a_0,a_1) \in \CSU{CI01}\), tous les termes sont positifs.
    \end{proof}

  %%%%%%%%%%%%%%%%%%%%%%%%%%%%%%%%%%%%%%%%%%%%%%%%
  \subsection{CSU de la CI1}

    Considérons la condition d’impédance \hyperlink{ci1}{CI1}:
    \begin{align}
    \label{eq:unicite:ci1}
      (\oI + b \LL) \vE_t = (a_0\oI + a_1 \LL) \vJ && \forall (a_0, a_1,b) \in \CC^3
    \end{align}

    Pour cette CIOE, nous démontrons qu'il existe plusieurs CSU.

    %\subsubsection{CSU de \cite{stupfel_sufficient_2011}}

    On définit 
    \begin{equation}
      \label{eq:unicite:delta}
      \begin{matrix}
        \Delta: & \CC^3 &\rightarrow& \CC
        \\
        & (a_0,a_1,b) & \mapsto & a_1 - a_0\conj{b}
      \end{matrix}
    \end{equation}
    Par abus de notation, on omet les variables \((a_0,a_1,b)\)
    \begin{equation}
       \Delta(a_0,a_1,b) \equiv \Delta
    \end{equation}

    On rappelle la CSU montrée dans \cite{stupfel_sufficient_2011}.

    \begin{defn}
      \label{def:csu:ci1-1}

      On définit le sous-espace ouvert de \(\CC^3\)
      \begin{equation*}
        \CSU[1]{CI1} = \left\lbrace 
        \begin{matrix}
        (a_0,a_1,b) \in \CC^3
        \\
        \Re(\Delta) = 0
        \\
        \Im(\Delta) \not = 0
        \\
        \Im(\Delta)\Im(b) \ge 0
        \\
        \Im(\Delta )\Im(a_1\conj{a_0})\ge 0
        \end{matrix}
        \right\rbrace
      \end{equation*}
    \end{defn}

    \begin{prop}[Une première CSU pour la CI1]
      \label{prop:csu:ci1-1}
      Il suffit que
      \begin{equation*}
        (a_0,a_1,b) \in \CSU[1]{CI1}
      \end{equation*}
      pour que \(\Re(X)\ge 0\), ce qui entraîne l'unicité.
    \end{prop}

    \begin{proof}
      On utilise l'identité \((a_1-a_0\conj{b})\oI = (a_1(\oI +\conj{b}\LL) - \conj{b}(a_0\oI + a_1\LL))\):
      \begin{align*}
        (a_1-a_0\conj{b})X &= \int_\Gamma \left(a_1(\oI +\conj{b}\LL) \vJ\right)\cdot\conj{\vE_t} - \left(\conj{b}(a_0\oI + a_1 \LL)\vJ\right)\cdot\conj{\vE_t}
        \intertext{Comme l'opérateur \(\LL\) est symétrique}
        (a_1-a_0\conj{b})X &= \int_\Gamma \left(a_1(\oI +\conj{b}\LL) \conj{\vE_t}\right)\cdot\vJ - \int_\Gamma \left(\conj{b}(a_0\oI + a_1 \LL)\vJ\right)\cdot\conj{\vE_t}
        \intertext{Par définition de la CI1}
        (a_1-a_0\conj{b})X &= \int_\Gamma \left(a_1(\conj{a_0}+\conj{a_1}\LL) \conj{\vJ}\right)\cdot\vJ - \int_\Gamma \left(\conj{b}(\oI +b \LL)\vE_t\right)\cdot\conj{\vE_t} \\
        (a_1-a_0\conj{b})X &= a_1\conj{a_0} \norm{ \vJ }^2 - |a_1|^2 \norm{\vgrads{\vJ}}^2 - \conj{b} \norm{ \vE_t }^2 + |b|^2\norm{\vgrads{\vE_t}}^2
      \end{align*}

      % On pose 
      % \begin{align*}
      %   F &= -\int_\Gamma \vJ \LL \conj{\vJ} \ge 0
      %   \\
      %   G &= -\int_\Gamma \vE_t \LL \conj{\vE_t} \ge 0
      % \end{align*}

      Explicitons la partie imaginaire de \( (a_1-a_0\conj{b})X\),
      \begin{align*}
        % \Re(\Delta)\Re(X) - \Im(\Delta)\Im(X) &= \Re(a_1\conj{a_0}) \norm{\vJ}^2 - \Re(\conj{b})\norm{\vE_t}^2 -|a_1|^2 F + |b|^2 G \\
        \Im(\Delta)\Re(X) + \Re(\Delta)\Im(X) &= \Im(a_1\conj{a_0}) \norm{\vJ}^2 - \Im(\conj{b})\norm{\vE_t}^2
        \intertext{En supposant \(\Re(\Delta)= 0\) et \(\Im(\Delta) \not = 0\) nous pouvons conclure car}
        \Im(\Delta)^2\Re(X) &= \Im(\Delta)\Im(a_1\conj{a_0}) \norm{\vJ}^2 - \Im(\Delta)\Im(\conj{b})\norm{\vE_t}^2
      \end{align*}
      Dans le cas où \(\Im(\Delta)\not=0\), il suffit d'imposer que tous ces termes soient positifs pour que \(\Re(X)\) le soit.
    \end{proof}

    On remarque que
    \begin{align}
      \CSU[1]{CI1} &\subset \CSU{CI01}\times\CC
      \\
      \CSU[1]{CI1}\cap(\CC^2 \times \lbrace0\rbrace) &\subsetneq (\CSU{CI01}\times\lbrace0\rbrace)
      \intertext{C'est insatisfaisant, car pour des fonctions infiniment régulières, la CIOE CI1 avec \(b=0\) est équivalente à la CI01.
      Plus précisément, soit \(S = \lbrace (a_0,a_1) \in \CC^2; \Re(a_1)=0 \rbrace\), on a}
      \CSU[1]{CI1}\cap(\CC^2 \times \lbrace0\rbrace) &= ((\CSU{CI01}\cap S) \times\lbrace0\rbrace) 
    \end{align}

    \begin{lemme}
      Soit \(z\in \CC\) et \(a\) alors la forme bilinéaire
      \begin{equation*}
        a(\vect{u},\vect{v}) = \int_\Gamma(\oI + z\LL)\vect{u}\cdot\conj{\vect{v}}
      \end{equation*}
      Si \(z\in \CC\backslash \RR_+^*\) alors \(a\) est coercive.
    \end{lemme}
    \begin{proof}
      \begin{align*}
        |a(\vect{u},\vect{u})|^2 &= \left(\norm{\vu}^2-\Re(z)\norm{\vgrads{\vu}}^2\right)^2 + \left(\Im(z)\norm{\vgrads{\vu}}^2\right)^2
        \\
        &= \begin{bmatrix}
          \norm{\vu}^2
          &
          \norm{\vgrads{\vu}}^2
        \end{bmatrix}
        \begin{bmatrix}
          1 & - \Re(z)
          \\
          -\Re(z) & |z|^2
        \end{bmatrix}
        \begin{bmatrix}
          \norm{\vu}^2
          \\
          \norm{\vgrads{\vu}}^2
        \end{bmatrix}
        \\
      \end{align*}
      Or la matrice n'est inversible que si \(\Im(z)^2\not=0\). Dans ce cas, on a la coercivité.

      Si \(\Im(z)=0\) alors
      \begin{align*}
        a(\vect{u},\vect{u}) &= \norm{\vu}^2-\Re(z)\norm{\vgrads{\vu}}^2
        \\
        &\ge \min(1,-\Re(z))\left(\norm{\vu}^2+\norm{\vgrads{\vu}}^2\right)
      \end{align*}
      Il suffit que \(-\Re(z) \ge 0 \) pour avoir la coercivité
    \end{proof}

    \begin{defn}
      \label{def:csu:ci1-2}

      On définit le sous-espace fermé de \(\CC^3\)
      \begin{equation*}
        \CSU[2]{CI1} = \left\lbrace 
        \begin{matrix}
        (a_0,a_1,b) \in \CC^3
        \\
        \Re(b) \le 0
        \\
        \Re\left(a_0\right) \ge 0
        \\
        \Re\left(b\conj{a_0}+\conj{a_1}\right) \le 0
        \\
        \Re\left(b\conj{a_1}\right) \ge 0
        \end{matrix}
        \right\rbrace
      \end{equation*}
    \end{defn}

    \begin{prop}[Une deuxième CSU pour la CI1]
      \label{prop:csu:ci1-2}
      Il suffit que
      \begin{equation*}
        (a_0,a_1,b) \in \CSU[2]{CI1}
      \end{equation*}
      pour que \(\Re(X)\ge 0\), ce qui entraîne l'unicité.
    \end{prop}

    \begin{proof}
      On suppose \(\Re(b)\le 0\) donc l'opérateur \(\oI + b\LL\) est bijectif. Il existe \(\vect{D}\) tel que
      \begin{align*}
        \vJ &= (\oI + b \LL)\vect{D}
      \end{align*}
      Donc 
      \begin{align*}
        (\oI + b \LL)\vE_t &= (a_0\oI + a_1\LL)\vJ
        \\
        (\oI + b \LL)\vE_t &= (a_0\oI + a_1\LL)(\oI + b \LL)\vect{D}
        \\
        0 &= (\oI + b \LL)(\vE_t -  (a_0\oI + a_1\LL)\vect{D})
        \intertext{L'opérateur est injectif donc}
        \vE_t &= (a_0\oI + a_1\LL)\vect{D}
        \\
        \int_\Gamma \vJ \cdot \conj{\vE_t} &= \int_\Gamma \vJ \cdot (\conj{a_0}\oI + \conj{a_1}\LL)\conj{\vect{D}}
        \\
        X &= \int_\Gamma (\oI + b \LL)\vect{D} \cdot (\conj{a_0}\oI + \conj{a_1}\LL)\conj{\vect{D}}
        \\
        X &= \conj{a_0}\norm{\vect{D}}^2 - (b\conj{a_0}+\conj{a_1})\norm{\vgrads{\vect{D}}}^2 + b\conj{a_1} \norm{\LL\vect{D}}^2
      \end{align*}
    \end{proof}

    On remarque que
    \begin{align}
      \CSU[2]{CI1} & \subset \CSU{CI01}\times\lbrace0\rbrace
      \\
      \CSU[2]{CI1}\cap(\CC^2 \times \lbrace0\rbrace) &= (\CSU{CI01}\times\lbrace0\rbrace)
    \end{align}

    Pour des fonctions infiniment régulières, la CIOE CI1 avec \(b=0\) est équivalente à la CI01 et la \CSU[2]{CI1} est donc meilleure que la  \CSU[1]{CI1}.
\section[Des CSU pour la CIOE de Marceaux et Stupfel 2000]{Des conditions suffisantes pour la CIOE de \cite{marceaux_high-order_2000}}

  Soit \(V=(\mathcal{C}^\infty(\Gamma))^3\) l'ensemble des champs tangents à \(\Gamma\) exprimé dans le repère cartésien.
  Exprimé dans le repère local à la surface, les éléments de cette ensemble n'ont pas de composantes normale.

  \begin{defn}
    On définit les opérateurs \gls{ope-LD} et \gls{ope-LR} tels que %sont introduits par \cite[eq.~4]{stupfel_implementation_2015} et s'expriment pour tous vecteurs complexes régulier tangents à \(\Gamma\) 

    % \begin{REM}
    %   Vivent-ils sur les vecteurs tangents ?
    % \end{REM}
    % \begin{REP}
    %   En commentaire dans le \TeX : [les opérateurs] introduits par \cite[eq.~4]{stupfel_implementation_2015} et s'expriment pour tous vecteurs complexes régulier tangents à \(\Gamma\).
    % \end{REP}

    \begin{equation*}
      \begin{aligned}
        &\fonction{\LD}{V}{V}%
          {\vu}{\vgrads{\vdivs \vu},}
        \\
        &\fonction{\LR}{V}{V}%
          {\vu}{\vrots{\vrots \vu}.}
      \end{aligned}
    \end{equation*}
  \end{defn}

  \begin{prop}
    L'opérateur \(\LD\) est hermitien symétrique négatif et \(\LR\) est hermitien symétrique positif.
  \end{prop}

  % \begin{REM}
  %   Si j'ai bien compris, \(\vu\) peut avoir des composantes tangentes mais pas le résultat ?
  % \end{REM}
  % \begin{REP}
  %   Je ne comprends pas cette remarque.
  % \end{REP}

  \begin{proof}
    Pour tout \(\vu \in V\),
    \begin{align*}
      \int_\Gamma \LD(\vu)\cdot \conj{\vu} &= -\norm{\vdivs \vu}^2,
      \\
      \int_\Gamma \LR(\vu)\cdot \conj{\vu} &=\norm{ \vrots \vu}^2.
    \end{align*}
  \end{proof}

  % \begin{REM}
  %   L'ensemble des champs tangents \(C^\infty\) ?
  % \end{REM}
  % \begin{REP}
  %   Oui ?
  % \end{REP}
  % \begin{REM}
  %   Est-ce que \(\vu\) est un vecteur tangent au bord ?
  %   Si oui, c'est un vecteur de \(C^\infty(\Gamma)^2\).
  %   Dis-le précisément s'il y a une composante normale ( je sais qu'on en a parlé à Montréal )
  % \end{REM}
  % \begin{REP}
  %   Oui.
  %   Quand je met un vecteur à 3 composantes, c'est que la dernière est nulle en pratique.
  % \end{REP}
  \begin{prop}
    \label{prop:unicite:injectif:opérateur:LD}
    Soit \(\vu \in V\).

    Soit \(\mathcal{P}_D\) l'opérateur tel que \(\mathcal{P}_D\vu = (a_0\oI + a_1 \LD)\vu\).

    Si \(\Re(a_0)\ge 0\) et \(\Re(a_1)\le 0\) alors l'opérateur \(\mathcal{P}_D\) est injectif sur \(V\).
  \end{prop}
  \begin{prop}
    \label{prop:unicite:injectif:opérateur:LR}
    Soit \(\vu \in V\).

    Soit \(\mathcal{P}_R\) l'opérateur tel que \(\mathcal{P}_R\vu = (a_0\oI - a_2 \LR)\vu\).

    Si \(\Re(a_0)\ge 0\) et \(\Re(a_2)\le 0\), alors l'opérateur \(\mathcal{P}_R\) est injectif sur \(V\).
  \end{prop}
  \begin{prop}
    \label{prop:unicite:injectif:opérateur:LD-LR}
    Soit \(\vu \in V\).
    
    Soit \(\mathcal{P}\) l'opérateur tel que \(\mathcal{P}\vu = (a_0\oI + a_1 \LD - a_2\LR)\vu\).

    Si \(\Re(a_0)\ge 0\), \(\Re(a_1)\le 0\) et \(\Re(a_2)\le 0\), alors l'opérateur \(\mathcal{P}\) est injectif sur \(V\).
  \end{prop}
  \begin{proof}
   Identique à la démonstration de la propriété \ref{prop:unicite:injectif:opérateur:L}.
  \end{proof}

  \begin{prop}
    Soit \(\OO\) un domaine borné de \(\RR^3\), de surface \(\Gamma\) fermée et régulière, de normale unitaire sortante \(\vect n\) et \(\vect{u} \in (\mathcal{C}^\infty(\Gamma))^3\), alors
    \begin{equation*}
        \LR(\LD(\vu)) = \LD(\LR(\vu)) = 0.
    \end{equation*}
  \end{prop}

  \begin{proof}
    Soit (\(x_1,x_2\)) un système de coordonnées locales sur \(\Gamma\).

    Soit un vecteur tangent défini en tout point de la surface. On définit une base locale \(\vect{u_1},\vect{u_2},\vect{n}\) où \((\vect{u_1},\vect{u_2})\) sont tangents à \(\Gamma\) et \(\vn\) est le vecteur normal unitaire sortant à \(\Gamma\).
    \\
    % \[
    %   \vect{u} = 
    %   \begin{bmatrix}
    %     U_1(x_1,x_2)
    %     \\
    %     U_2(x_1,x_2)
    %     \\
    %     0
    %   \end{bmatrix}.
    % \]

    % D'après \cite[p.~1028, A3.22]{bladel_electromagnetic_2007}, le rotationnel surfacique d'un vecteur tangent est un vecteur normal à la surface
    % \[
    %   \vrots{\vect{u}} =\vn \left(\vn\cdot\trot{\vect{u}}\right)
    % \]

    Montrons que \(\LR\LD = 0\).

    D’après \cite[propriété A3.42, p.~1029]{bladel_electromagnetic_2007}, 
    % on sait que l'application de \(\vrots \vgrads\) sur une fonction scalaire renvoi un vecteur tangent à la surface. Or par définition, le rotationnel surfacique d'un
    soit \(f(x_1,x_2)\) une fonction régulière de \(\Gamma\) dans \(\CC\), alors \(\vn \cdot \vrots(\vgrads f(x_1,x_2)) = 0\).

    \begin{align*}
      \LR(\LD \vu) &= \vrots \left(\left(\vn \cdot \vrots \left( \vgrads \left(\vdivs \vu\right)\right)\right)\vn \right),
      \\
      &= \vrots \left(0\vn \right), 
      \\
      &= 0.
    \end{align*}

    Montrons que \(\LD\LR = 0\).

    D’après \cite[p.~1029, A3.43]{bladel_electromagnetic_2007}, \(\vdivs \vrots (f\vn) = 0\).
    \begin{align*}
      \LD(\LR \vu) &= \vgrads \vdivs \vrots (\vn (\vn \cdot \vrots \vu)),
      \\
      &= \vgrads 0,
      \\
      &= 0.
    \end{align*}
  \end{proof}

\subsection{CSU pour la CI4}
  Soit la CIOE que l'on nomme \hyperlink{ci4}{CI4} :
  \begin{equation}
    \label{eq:unicite:ci4}
    \vE_t = (a_0 + a_1 \LD - a_2 \LR ) \vJ.
  \end{equation}

  \begin{defn}
    \label{def:csu:ci4}

    On définit le sous-espace fermé de \(\CC^3\)
    \begin{equation*}
      \CSU{CI4} = \left\lbrace 
      \begin{aligned}
      &(a_0,a_1,a_2) \in \CC^3,
      \\
      & \begin{aligned}
        \Re(a_0) &\ge 0,
        \\
        \Re(a_1) &\le 0,
        \\
        \Re(a_2) &\le 0,
        \end{aligned}
      \end{aligned}
      \right\rbrace.
    \end{equation*}
  \end{defn}

 \begin{prop}[Une CSU pour la CI4]
    \label{prop:csu:ci4}
    On a 
    \begin{equation*}
      (a_0,a_1,a_2) \in \CSU{CI4} \Rightarrow \Re(X)\ge 0,
    \end{equation*}
    ce qui entraîne l'unicité de la solution du problème \{\eqref{eq:unicite:probleme_sans_ci},\eqref{eq:unicite:ci4}\}.
  \end{prop}
  % \begin{REM}
  %   (\eqref{eq:unicite:probleme_sans_ci}, CI4) ou (\eqref{eq:unicite:probleme_sans_ci},\eqref{eq:unicite:ci4}).
  % \end{REM}
  \begin{proof}
    Par définition de \(X\) \eqref{eq:unicite:x}, on a
    \begin{align*}
      X &= \conj{a_0}\norm{\vJ}^2 + \conj{a_1}\norm{\vdivs{\vJ}}^2 - \conj{a_2}\norm{ \vrots{\vJ}}^2,
      \intertext{donc}
      \Re(X) &= \Re(\conj{a_0})\norm{\vJ}^2 + \Re(\conj{a_1})\norm{\vdivs{\vJ}}^2 - \Re(\conj{a_2})\norm{ \vrots{\vJ}}^2.
    \end{align*}
  \end{proof}

  Soit \(S = \left\lbrace (a_0,a_1,a_2) \in \CC^3 ; a_1 = a_2 \right\rbrace \). On remarque que
  \begin{align}
    \CSU{CI4}\cap S &= (\CSU{CI01}\times\CC)\cap S. 
  \end{align}
  Dans le cas \(a_1=a_2\), (\eqref{eq:unicite:probleme_sans_ci},\eqref{eq:unicite:ci01}) et (\eqref{eq:unicite:probleme_sans_ci},\eqref{eq:unicite:ci4}) sont le même problème, et les CSU correspondantes sont les mêmes.
  % \begin{REM}
  %   Mal exprimé.
  %   Dans le cas \(a_1=a_2\), (\eqref{eq:unicite:probleme_sans_ci},\eqref{eq:unicite:ci01}) et (\eqref{eq:unicite:probleme_sans_ci},\eqref{eq:unicite:ci4}) sont le même problème, et les CSU correspondantes sont les mêmes.
  % \end{REM}
  % \begin{REP}
  %   Fait
  % \end{REP}
\subsection{CSU pour la CI3}

  Soit la CIOE énoncée dans \cite{marceaux_high-order_2000} que l'on nomme \hyperlink{ci3}{CI3} :
  \begin{equation}
    \label{eq:unicite:ci3:ci3}
    ( \oI + b_1 \LD - b_2 \LR)\vE_t = (a_0\oI + a_1 \LD - a_2 \LR ) \vJ.
  \end{equation}

  Par abus de notation, on omet les variables de la fonction \(\Delta\) \eqref{eq:unicite:delta} :
  \begin{align*}
     \Delta(a_0,a_1,b_1) &\equiv \Delta_1 = a_1 - a_0\conj{b_1},
     \\
     \Delta(a_0,a_2,b_2) &\equiv \Delta_2 = a_2 - a_0\conj{b_2}.
  \end{align*}

  \begin{defn}
    \label{def:csu:ci3-0}

    On définit le sous-espace de \(\CC^5\)

    \begin{equation*}
      \CSU[0]{CI3} = \left\lbrace 
      \begin{aligned}
      &(a_0,a_1,a_2,b_1,b_2) \in \CC^5,
      \\
      &\begin{aligned}
        &\Delta_1 &\not = 0,
        \\
        &\Delta_2 &\not = 0,
        \\
        &\Re\left(a_0\conj{a_1}\Delta_1\right) &\ge 0,
        \\
        &\Re\left(\frac{\conj{b_1}}{\Delta_1}\right) &\le 0,
        \\
        &\Re\left(\conj{a_0}a_2\left(\frac{\conj{b_1}}{\Delta_1}-\frac{\conj{b_2}}{\Delta_2}\right) + \frac{\conj{a_2}a_1}{\Delta_1} \right)&\le 0,
        \\
        &\Re\left(2\Re(b_2)\frac{\conj{b_1}}{\Delta_1}-\frac{\conj{b_2}^2}{\Delta_2}\right) &\ge 0,
        \\
        &\Re\left(a_0\conj{a_2}\Delta_2\right) &\ge 0,
        \\
        &\Re\left(\frac{\conj{b_2}}{\Delta_2}\right) &\le 0,
        \\
        &\Re\left(\conj{a_0}a_1\left(\frac{\conj{b_1}}{\Delta_1}-\frac{\conj{b_2}}{\Delta_2}\right) + \frac{\conj{a_1}a_2}{\Delta_2} \right)&\le 0,
        \\
        &\Re\left(2\Re(b_1)\frac{\conj{b_2}}{\Delta_2}-\frac{\conj{b_1}^2}{\Delta_1}\right) &\ge 0,
        \\
        &\Re\left(\Delta_1\right) &= 0,
        \\
        &\Re\left(\Delta_2\right) &= 0,
        \\
        &\Re\left(\frac{\conj{b_2}}{\Delta_2}-\frac{\conj{b_1}}{\Delta_1}\right) &= 0,
        \end{aligned}
      \end{aligned}
      \right\rbrace.
    \end{equation*}
  \end{defn}
  
 \begin{prop}[Une première CSU pour la CI3]
    \label{prop:csu:ci3-0}
    On a 
    \begin{equation*}
      (a_0,a_1,a_2,b_1,b_2) \in \CSU[0]{CI3} \Rightarrow \Re(X)\ge 0,
    \end{equation*}
    ce qui entraîne l'unicité de la solution du problème \{\eqref{eq:unicite:probleme_sans_ci},\eqref{eq:unicite:ci3:ci3}\}.
  \end{prop}
  % \begin{REM}
  %   (\eqref{eq:unicite:probleme_sans_ci}, CI3) ou (\eqref{eq:unicite:probleme_sans_ci},\eqref{eq:unicite:ci3:ci3}).
  % \end{REM}
  % \begin{REP}
  %   Fait
  % \end{REP}
  \begin{proof}
    Soit \(\LL_3\) l'opérateur
    \begin{equation}
      \label{eq:unicite:ci3:L3}
      \fonction{\LL_3}{(\mathcal{C}^\infty(\Gamma))^3 \times (\mathcal{C}^\infty(\Gamma))^3}{(\mathcal{C}^\infty(\Gamma))^3}
      {(\vE_t,\vJ)}{( \oI + b_1 \LD - b_2 \LR) \vE_t - (a_0\oI + a_1 \LD - a_2 \LR ) \vJ.}
    \end{equation}

    De la condition limite sur \(\Gamma\), on déduit par produit scalaire \(L^2\) avec un certain nombre de vecteurs bien choisis, des égalités qui sont nécessairement vérifiées.
    On fait cependant remarquer que ces égalités ne  permettent  pas de revenir  à la CL.

    AInsi on utilise l'identité \(\int_\Gamma \conj{\LL_3(\vE_t,\vJ)}\cdot \vJ \dd{\Gamma}(\vx) = 0\) qui découle de la CI3, et dont on déduit
    % \begin{REM}
    %   Là où ça devient plus compliqué, tu pourrais dire:
    %   De la CL, on déduit par produit scalaire \(L^2\) avec un certain nombre de vecteurs bien choisis, des égalités qui sont nécessairement vérifiées. On fait cependant remarquer que ces égalités ne  permettent  pas de revenir  à la CL. 
    % \end{REM}
    \begin{multline}
      \label{eq:unicite:ci3:csu3-1}
      \int_\Gamma \vJ \cdot \conj{\vE_t}\dd{\Gamma}(\vx)   + \conj{b_1} \int_\Gamma \vJ\cdot \LD\conj{\vE_t}\dd{\Gamma}(\vx)  - \conj{b_2} \int_\Gamma \vJ \LR\conj{\vE_t}\dd{\Gamma}(\vx)  \\
      = \conj{a_0} \norm{\vJ}^2 - \conj{a_1} \norm{\vdivs \vJ}^2  - \conj{a_2} \norm{\vrots \vJ}^2. 
    \end{multline}
    On utilise de même \(\int_\Gamma {\LL_3(\vE_t,\vJ)}\cdot \conj{\vE_t} = 0\),
    \begin{multline}
      \label{eq:unicite:ci3:csu3-2}
      \norm{\vE_t}^2   - b_1 \norm{ \vdivs \vE }^2  - b_2 \norm{\vrots \vE_t}^2\dd{\Gamma}(\vx)  \\
      = a_0 \int_\Gamma \vJ\cdot \conj{\vE_t}\dd{\Gamma}(\vx) + a_1 \int_\Gamma \conj{\vE_t} \LD \vJ\dd{\Gamma}(\vx)  - a_2 \int_\Gamma \conj{\vE_t} \cdot \LR \vJ \dd{\Gamma}(\vx).
    \end{multline}
    On utilise \(\int_\Gamma \conj{\LL_3(\vE_t,\vJ)}\cdot\LR\vJ\dd{\Gamma}(\vx) = 0\),
    \begin{equation}
      \label{eq:unicite:ci3:csu3-3}
      \int_\Gamma \vJ \cdot \LR \conj{\vE_t}\dd{\Gamma}(\vx)   - \conj{b_2} \int_\Gamma \LR \vJ \cdot \LR \conj{\vE_t}\dd{\Gamma}(\vx)
      =  \conj{a_0} \norm{ \vrots \vJ}^2ds - \conj{a_2} \norm{ \LR \vJ}^2. 
    \end{equation}
    On utilise \(\int_\Gamma {\LL_3(\vE_t,\vJ)}\cdot\LR\conj{\vE_t}\dd{\Gamma}(\vx)=0\),
    \begin{equation}
      \label{eq:unicite:ci3:csu3-4}
      \norm{ \vrots \vE_t }^2   - {b_2} \norm{ \LR \vE_t}^2 
      = a_0 \int_\Gamma \conj{\vE_t} \cdot \LR \vJ \dd{\Gamma}(\vx) - a_2 \int_\Gamma \LR \conj{\vE_t} \cdot \LR \vJ \dd{\Gamma}(\vx).
    \end{equation}
    On utilise \(\int_\Gamma \conj{\LL_3(\vE_t,\vJ)}\cdot\LD\vJ\dd{\Gamma}(\vx)=0\),
    \begin{equation}
      \label{eq:unicite:ci3:csu3-5}
      \int_\Gamma \vJ \cdot \LD \conj{\vE_t}\dd{\Gamma}(\vx)   + \conj{b_1} \int_\Gamma \LD \vJ \cdot \LD \conj{\vE_t}\dd{\Gamma}(\vx)
      = - \conj{a_0} \norm{\vdivs \vJ}^2 + \conj{a_1} \norm{ \LD \vJ}^2. 
    \end{equation}
    On utilise \(\int_\Gamma {\LL_3(\vE_t,\vJ)}\cdot\LD\conj{\vE_t}\dd{\Gamma}(\vx)=0\),
    \begin{equation}
      \label{eq:unicite:ci3:csu3-6}
      -\norm{ \vdivs \vE_t }^2   + {b_1} \norm{ \LD \vE_t}^2
      = a_0 \int_\Gamma \conj{\vE_t} \cdot \LD \vJ\dd{\Gamma}(\vx)  + a_1 \int_\Gamma \LD \conj{\vE_t} \cdot \LD \vJ \dd{\Gamma}(\vx).
    \end{equation}

    On note
    \begin{align*}
      Y_D &= \int_\Gamma \vJ \cdot \LD \conj{\vE_t}\dd{\Gamma}(\vx),  &
      Y_R &= \int_\Gamma \vJ \cdot \LR \conj{\vE_t} \dd{\Gamma}(\vx),
      \\
      Z_D &= \int_\Gamma \LD \vJ \cdot \LD \conj{\vE_t}\dd{\Gamma}(\vx),  &
      Z_R &= \int_\Gamma \LR \vJ \cdot \LR \conj{\vE_t} \dd{\Gamma}(\vx).
    \end{align*}

    Les 4 égalités \eqref{eq:unicite:ci3:csu3-1} à \eqref{eq:unicite:ci3:csu3-4} sont équivalentes au système

    \begin{align*}
      \mM_R X_R &= F_R,
      \\
      \begin{bmatrix}
        1 & \conj{b_1} & -\conj{b_2} & 0
        \\
        a_0 & a_1 & -a_2 & 0
        \\
        0 & 0 & 1 & -\conj{b_2}
        \\
        0 & 0 & a_0 & -a_2
        \\
      \end{bmatrix}
      \begin{bmatrix}
        X\\
        Y_D\\
        Y_R\\
        Z_R
      \end{bmatrix}
      &=
      \begin{bmatrix}
        \conj{a_0} \norm{\vJ}^2 - \conj{a_1} \norm{\vdivs \vJ}^2 - \conj{a_2} \norm{\vrots \vJ}^2
        \\
        \norm{\vE_t}^2  - b_1 \norm{\vdivs \vE}^2  - b_2 \norm{\vrots \vE_t}^2
        \\
        \conj{a_0} \norm{\vrots \vJ}^2 - \conj{a_2} \norm{\LR \vJ}^2
        \\
        \norm{\vrots \vE_t}^2 - {b_2} \norm{\LR \vE_t}^2
      \end{bmatrix},
    \end{align*}
    
    et les 4 égalités \eqref{eq:unicite:ci3:csu3-1},\eqref{eq:unicite:ci3:csu3-2},\eqref{eq:unicite:ci3:csu3-5},\eqref{eq:unicite:ci3:csu3-6} sont équivalentes au système

    \begin{align*}
      \mM_D  X_D & =  F_D,
      \\
      \begin{bmatrix}
        1 & -\conj{b_2} & \conj{b_1} & 0
        \\
        a_0 & -a_2 & a_1 & 0
        \\
        0 & 0 & 1 & \conj{b_1}
        \\
        0 & 0 & a_0 & a_1
      \end{bmatrix}
      \begin{bmatrix}
        X
        \\
        Y_R
        \\
        Y_D
        \\
        Z_D
      \end{bmatrix}
      & =
      \begin{bmatrix}
        \conj{a_0} \norm{\vJ}^2 - \conj{a_1} \norm{\vdivs \vJ}^2 - \conj{a_2} \norm{\vrots \vJ}^2
        \\
        \norm{\vE_t}^2   - b_1 \norm{\vdivs \vE }^2  - b_2 \norm{\vrots \vE_t}^2
        \\
        -\conj{a_0} \norm{\vdivs \vJ}^2ds + \conj{a_1} \norm{\LR \vJ}^2
        \\
        -\norm{\vdivs \vE_t}^2   + {b_1} \norm{\LR \vE_t}^2 
      \end{bmatrix}.
    \end{align*}

    Le déterminant de ces matrices triangulaires par blocs est \(\Delta_1\Delta_2\). Dans le cas où
    \begin{equation}
      \label{eq:unicite:ci3:csu3-cn-det}
      \Delta_1\Delta_2 \not = 0,
    \end{equation}
    elles sont inversibles et on a alors leurs inverses
    % \begin{REM}
    %   Verification Mathematica ?
    % \end{REM}
    % \begin{REP}
    % \(\mM_R^{-1}\) \href{https://www.wolframalpha.com/input/?i=Simplify%5B%28a1-a0*Conjugate%5Bb1%5D%29*%28a2-a0*Conjugate%5Bb2%5D%29*Inverse+%7B++%7B1%2C+Conjugate%5Bb1%5D%2C+-Conjugate%5Bb2%5D%2C+0%7D%2C++%7Ba0%2C+a1%2C+-a2%2C+0%7D%2C++%7B0%2C+0%2C+1%2C+-Conjugate%5Bb2%5D%7D%2C++%7B0%2C+0%2C+a0%2C+-a2%7D++%7D%5D}{lien web}.

    % \(\mM_D^{-1}\) \href{https://www.wolframalpha.com/input/?i=Simplify%5B%28a1-a0*Conjugate%5Bb1%5D%29*%28a2-a0*Conjugate%5Bb2%5D%29*Inverse+%7B++%7B1%2C-Conjugate%5Bb2%5D%2C+Conjugate%5Bb1%5D%2C+0%7D%2C++%7Ba0%2C+-a2%2C+a1%2C+0%7D%2C++%7B0%2C+0%2C+1%2C+Conjugate%5Bb1%5D%7D%2C++%7B0%2C+0%2C+a0%2C+a1%7D++%7D%5D}{lien web}
    % \end{REP}
    \begin{align*}
      \mM_R^{-1} & =\frac{1}{\Delta_1\Delta_2}
      \begin{bmatrix}
        a_1 \Delta_2 & -\conj{b_1}\Delta_2 & a_2(a_1\conj{b_2}-a_2\conj{b_1}) & -\conj{b_2}(a_1\conj{b_2}-a_2\conj{b_1})
        \\
        -a_0 \Delta_2 & \Delta_2 & a_2\Delta_2 & -\conj{b_2}\Delta_2
        \\
        0 & 0 & a_2\Delta_1 & -\conj{b_2}\Delta_1
        \\
        0 & 0 & a_0\Delta_1 & -\Delta_1
      \end{bmatrix},
      \\
      \mM_D^{-1} & =\frac{1}{\Delta_1\Delta_2}
      \begin{bmatrix}
        a_2 \Delta_1 & -\conj{b_2}\Delta_1 & a_1(a_1\conj{b_2}-a_2\conj{b_1}) & -\conj{b_1}(a_1\conj{b_2}-a_2\conj{b_1})
        \\
        a_0 \Delta_1 & -\Delta_1 & a_1\Delta_1 & -\conj{b_1}\Delta_1
        \\
        0 & 0 & a_1\Delta_2 & -\conj{b_1}\Delta_2
        \\
        0 & 0 & -a_0\Delta_2 & \Delta_2
      \end{bmatrix}.
    \end{align*}
    
    On déduit alors les vecteurs \(X_D\) et \(X_R\) en fonction des normes présentes dans \(F_D\) et \(F_R\), et donc pour chacun de ces vecteurs, on a une expression de \(X\), \(X = (\mM_R^{-1}F_R)_1\) et \(X = (\mM_D^{-1}F_D)_1\).

    Cette dernière quantité est une combinaison linéaire de toutes les normes rencontrées, donc \(X = A_{D/R} \norm{\vJ}^2 + B_{D/R} \norm{\vdivs \vJ}^2 + \ldots \)
    Une condition suffisante pour que la partie réelle de \(X\) soit positive est d'avoir les parties réelles des constantes \(A_{D/R},B_{D/R} \ldots\) positives.
    % Une condition suffisante sur la positivité de la partie réelle de \(X\) est que toutes les parties réelles de ces  constantes \(A_{D/R},B_{D/R} \ldots\) soient positives.

    On pose \(\Theta=\frac{\conj{b_1}}{\Delta_1}-\frac{\conj{b_2}}{\Delta_2}=-\frac{(a_1\conj{b_2}-a_2\conj{b_1})}{\Delta_1\Delta_2}\).

    % \begin{minipage}{0.99\textwidth}
      {En résolvant le système \(\mM_RX_R=F_R\), on obtient},

      \begin{multline*}
        X =
        \frac{a_1}{\Delta_1}
        \left(
          \conj{a_0} \norm{\vJ}^2 - \conj{a_1} \norm{\vdivs \vJ}^2 - \conj{a_2} \norm{\vrots \vJ}^2
        \right)
        \\
        -\frac{\conj{b_1}}{\Delta_1}  
        \left(
          \norm{\vE_t}^2  - b_1 \norm{\vdivs \vE}^2  - b_2 \norm{\vrots \vE_t}^2
        \right)
         \\
        + a_2\frac{(a_1\conj{b_2}-a_2\conj{b_1})}{\Delta_1\Delta_2}
        \left(
          \conj{a_0} \norm{\vrots \vJ}^2 - \conj{a_2} \norm{\LR \vJ}^2
        \right)
        \\
        -\conj{b_2}\frac{(a_1\conj{b_2}-a_2\conj{b_1})}{\Delta_1\Delta_2}
        \left(
          \norm{\vrots \vE_t}^2 - {b_2} \norm{\LR \vE_t}^2
        \right).
      \end{multline*}
      
      Les conditions suivantes garantissent que \(\Re(X)\ge 0\).
      \begin{align}
        \label{eq:unicite:ci3:csu3d-j2}&\Re\left(\frac{\conj{a_0}{a_1}}{\Delta_1}\right) \ge 0,
        \\
        \label{eq:unicite:ci3:csu3d-e2}&\Re\left(\frac{\conj{b_1}}{\Delta_1}\right) \le 0,
        \\
        \label{eq:unicite:ci3:csu3d-jrj}&\Re\left(\conj{a_0}a_2\Theta + \frac{\conj{a_2}a_1}{\Delta_1} \right)\le 0,
        \\
        \label{eq:unicite:ci3:csu3d-ere}&\Re\left(\conj{b_2}\Theta + \frac{\conj{b_1}b_2}{\Delta_1}\right) \ge 0,
        \\
        \label{eq:unicite:ci3:csu3d-jdj}&\Re\left(|a_1|^2\Delta_1\right) \le 0,
        \\
        \label{eq:unicite:ci3:csu3d-ede}&\Re\left(|b_1|^2\Delta_1\right) \ge 0,
        \\
        \label{eq:unicite:ci3:csu3d-dj2}&\Re\left(|a_2|^2\Theta\right)\ge 0,
        \\
        \label{eq:unicite:ci3:csu3d-de2}&\Re\left(|b_2|^2\Theta\right)\le 0.
      \end{align}
      L'on remarque \eqref{eq:unicite:ci3:csu3d-jdj} \& \eqref{eq:unicite:ci3:csu3d-ede} impliquent
      \begin{equation}
        \label{eq:unicite:ci3:csu3d-jde}
        \Re\left(\Delta_1\right) = 0,
      \end{equation}
      et \eqref{eq:unicite:ci3:csu3r-rj2} \& \eqref{eq:unicite:ci3:csu3r-re2} impliquent
      \begin{equation}
        \label{eq:unicite:ci3:csu3-rje2}
        \Re\left(\Theta\right) = 0.
      \end{equation}
    % \end{minipage}

    % \begin{minipage}{0.99\textwidth}
      {Tandis qu'avec le système \(\mM_DX_D=F_D\)}, on a

      \begin{multline*}
        X =
        \frac{a_2}{\Delta_2}
        \left(
        \conj{a_0} \norm{\vJ}^2 - \conj{a_1} \norm{\vdivs \vJ}^2 - \conj{a_2} \norm{\vrots \vJ}^2 
        \right)
        \\
        -\frac{\conj{b_2}}{\Delta_2}
        \left(
        \norm{\vE_t}^2   - b_1 \norm{\vdivs \vE }^2  - b_2 \norm{\vrots \vE_t}^2
        \right)
        \\
        + a_1\frac{(a_1\conj{b_2}-a_2\conj{b_1})}{\Delta_1\Delta_2}
        \left(
        -\conj{a_0} \norm{\vdivs \vJ}^2 + \conj{a_1} \norm{\LR \vJ}^2
        \right)
        \\
        -\conj{b_1}\frac{(a_1\conj{b_2}-a_2\conj{b_1})}{\Delta_1\Delta_2}
        \left(
        -\norm{\vdivs \vE_t}^2   + {b_1} \norm{\LR \vE_t}^2
        \right).
      \end{multline*}
 
      Les conditions suivantes garantissent que \(\Re(X)\ge 0\).

      \begin{align}
        \label{eq:unicite:ci3:csu3r-j2}&\Re\left(\frac{\conj{a_0}{a_2}}{\Delta_2}\right) \ge 0,
        \\
        \label{eq:unicite:ci3:csu3r-e2}&\Re\left(\frac{\conj{b_2}}{\Delta_2}\right) \le 0,
        \\
        \label{eq:unicite:ci3:csu3r-jdj}&\Re\left(\conj{a_0}a_1\Theta - \frac{\conj{a_1}a_2}{\Delta_2} \right)\le 0,
        \\
        \label{eq:unicite:ci3:csu3r-ede}&\Re\left(-\conj{b_1}\Theta + \frac{\conj{b_2}b_1}{\Delta_2} \right) \ge 0,
        \\
        \label{eq:unicite:ci3:csu3r-jrj}&\Re\left(|a_2|^2\Delta_2\right) \le 0,
        \\
        \label{eq:unicite:ci3:csu3r-ere}&\Re\left(|b_2|^2\Delta_2\right) \ge 0,
        \\
        \label{eq:unicite:ci3:csu3r-rj2}&\Re\left(|a_1|^2\Theta\right)\ge 0,
        \\
        \label{eq:unicite:ci3:csu3r-re2}&\Re\left(|b_1|^2\Theta\right)\le 0,
      \end{align}
      L'on remarque \eqref{eq:unicite:ci3:csu3r-jrj} \& \eqref{eq:unicite:ci3:csu3r-ere} impliquent
      \begin{equation}
        \label{eq:unicite:ci3:csu3r-jre}
        \Re\left(\Delta_2\right) = 0,
      \end{equation}
      et \eqref{eq:unicite:ci3:csu3d-dj2} \& \eqref{eq:unicite:ci3:csu3d-de2} impliquent
      \begin{equation}
        \label{eq:unicite:ci3:csu3-dje2}
        \Re\left(\Theta\right) = 0.
      \end{equation}
    % \end{minipage}

    Pour conclure, on remarque que l'on peut utiliser 3 CSU
    \begin{itemize}
      \item les relations \eqref{eq:unicite:ci3:csu3-cn-det}, \eqref{eq:unicite:ci3:csu3d-j2} à \eqref{eq:unicite:ci3:csu3d-ere}, \eqref{eq:unicite:ci3:csu3d-jde}, \eqref{eq:unicite:ci3:csu3-dje2}
      \item les relations \eqref{eq:unicite:ci3:csu3-cn-det}, \eqref{eq:unicite:ci3:csu3r-j2} à \eqref{eq:unicite:ci3:csu3r-ede}, \eqref{eq:unicite:ci3:csu3r-jre}, \eqref{eq:unicite:ci3:csu3-rje2}
      \item l'union de ces deux ensembles qui est la CSU de la proposition.
    \end{itemize}

    \end{proof}
  \begin{defn}
    \label{def:csu:ci3-1}

    On définit le sous-espace de \(\CC^5\)

    \begin{equation*}
      \CSU[1]{CI3} = \left\lbrace 
      \begin{aligned}
      &(a_0,a_1,a_2,b_1,b_2) \in \CC^5,
      \\
      &\begin{aligned}
        &\Delta_1 & \not= 0,
        \\
        &\Delta_2 & \not= 0,
        \\
        &\Re\left(a_0\conj{a_1}\Delta_1\right) &\ge 0,
        \\
        &\Re\left(a_0\conj{a_2}\Delta_2\right) &\ge 0,
        \\
        &\Re(\Delta_1) &= 0,
        \\
        &\Re(\Delta_2) &= 0,
        \\
        &\Im(b_1)\Im(\Delta_2) - \Im(b_2)\Im(\Delta_1) &= 0,
        \\
        &\Re(b_1\Delta_1) &\le 0.
        \end{aligned}
      \end{aligned}
      \right\rbrace.
    \end{equation*}
  \end{defn}
  
 \begin{prop}[CSU de Lafitte : réduction du nombre de relations dans la CSU précédente]
    \label{prop:csu:ci3-1}
    On a 
    \begin{equation*}
      (a_0,a_1,a_2,b_1,b_2) \in \CSU[1]{CI3} \Rightarrow \Re(X)\ge 0,
    \end{equation*}
    ce qui entraîne l'unicité de la solution du problème \{\eqref{eq:unicite:probleme_sans_ci},\eqref{eq:unicite:ci3:ci3}\}.
  \end{prop}
  % \begin{REM}
  %   (\eqref{eq:unicite:probleme_sans_ci}, CI3) ou (\eqref{eq:unicite:probleme_sans_ci},\eqref{eq:unicite:ci3:ci3}).
  % \end{REM}
  % \begin{REP}
  %   Fait
  % \end{REP}
  \begin{proof}
    Utilisant cette dernière CSU, O.~Lafitte a éliminé les redondances et a obtenu une forme plus compacte.
    \begin{align}
    \label{eq:unicite:ci3:csu3rd-D1}\Re(\Delta_1) &= 0 & &\Leftrightarrow & \Re(a_1) - \Re(a_0\conj{b_1}) &= 0,
    \\
    \label{eq:unicite:ci3:csu3rd-D2}\Re(\Delta_2) &= 0 & &\Leftrightarrow & \Re(a_2) - \Re(a_0\conj{b_2}) &= 0,
    \\
    \label{eq:unicite:ci3:csu3rd-T}\Re(\Theta) &= 0 & &\Leftrightarrow & \Im(b_1)\Im(\Delta_2) - \Im(b_2)\Im(\Delta_1) &= 0.
    \intertext{Les conditions \eqref{eq:unicite:ci3:csu3d-e2} et \eqref{eq:unicite:ci3:csu3r-e2} sont alors redondantes et deviennent}
    \label{eq:unicite:ci3:csu3rd-bD} \Im(b_1)\Im(\Delta_1) &\ge 0 & &\Leftrightarrow & \Im(b_2)\Im(\Delta_2) &\ge 0.
    \end{align}
    Utilisant \eqref{eq:unicite:ci3:csu3rd-T}, l'inégalité \eqref{eq:unicite:ci3:csu3d-ere} se simplifie trivialement car
    \begin{align*}
      \conj{b_2}\Theta + \frac{\conj{b_1}b_2}{\Delta_1} & = \conj{b_2}\Theta + \frac{\conj{b_1}b_2}{\Delta_1} + \frac{\conj{b_2}b_2}{\Delta_2} - \frac{\conj{b_2}b_2}{\Delta_2},
      \\
      &=\left(\conj{b_2} + b_2\right)\Theta + \frac{b_2\conj{b_2}}{\Delta_2},
      \\
      &=2\Re(b_2)\Theta + \frac{|b_2|^2}{\Delta_2},
    \end{align*}
    et ce dernier est imaginaire pur d'après \eqref{eq:unicite:ci3:csu3rd-D2} \& \eqref{eq:unicite:ci3:csu3rd-T}.
    De même, pour l'inégalité \eqref{eq:unicite:ci3:csu3r-ede}.
    % \begin{align*}
    %   -\conj{b_1}\Theta + \frac{\conj{b_2}b_1}{\Delta_2} & = -\conj{b_1}\Theta + \frac{\conj{b_2}b_1}{\Delta_2} - \frac{\conj{b_1}b_1}{\Delta_1} + \frac{\conj{b_1}b_1}{\Delta_1},
    %   \\
    %   &=-\left(\conj{b_1} + b_1\right)\Theta + \frac{b_1\conj{b_1}}{\Delta_1},
    %   \\
    %   &=-2\Re(b_1)\Theta + \frac{|b_1|^2}{\Delta_1},
    % \end{align*}
    % qui est aussi imaginaire pur d'après \eqref{eq:unicite:ci3:csu3rd-D2} \& \eqref{eq:unicite:ci3:csu3rd-T}.

    Enfin l'inégalité \eqref{eq:unicite:ci3:csu3d-jrj} se simplifie aussi trivialement si l'on injecte la définition de \(\Delta_1\) et \(\Delta_2\) car
    \begin{align*}
    \conj{a_0}a_2\Theta + \frac{\conj{a_2}a_1}{\Delta_1} & = \conj{a_0}a_2\Theta + \conj{a_2}\frac{\Delta_1 + a_0\conj{b_1}}{\Delta_1},
    \\
    &=\conj{a_0}a_2\Theta + \conj{a_2} + a_0\conj{a_2}\frac{\conj{b_1}}{\Delta_1} - a_0\conj{a_2}\frac{\conj{b_2}}{\Delta_2} + a_0\conj{a_2}\frac{\conj{b_2}}{\Delta_2},
    \\
    &=\left(\conj{a_0}a_2 + a_0\conj{a_2}\right)\Theta + \conj{a_2} + a_0\conj{a_2}\frac{\conj{b_2}}{\Delta_2},
    \\
    &=\left(\conj{a_0}a_2 + a_0\conj{a_2}\right)\Theta + \conj{a_2}\frac{\Delta_2 + a_0\conj{b_2}}{\Delta_2},
    \\
    &=2\Re(\conj{a_0}a_2)\Theta + \frac{|a_2|^2}{\Delta_2},
    \end{align*}
    qui est imaginaire pur.
    De même pour l'inégalité \eqref{eq:unicite:ci3:csu3r-jdj}.

    Ainsi des 13 relations initiales, seules 8 sont conservées
    \begin{align*}
      \eqref{eq:unicite:ci3:csu3-cn-det} && \Delta_1 & \not= 0 && \Leftrightarrow & \Im(\Delta_1) &\not = 0,
      \\
      \eqref{eq:unicite:ci3:csu3-cn-det} && \Delta_2 & \not= 0 && \Leftrightarrow & \Im(\Delta_2) &\not = 0,
      \\
      \eqref{eq:unicite:ci3:csu3d-j2} && \Re\left(a_0\conj{a_1}\Delta_1\right) &\ge 0 & &\Leftrightarrow & \Im(a_0\conj{a_1})\Im(\Delta_1) &\le 0 ,
      \\
      \eqref{eq:unicite:ci3:csu3r-j2} && \Re\left(a_0\conj{a_2}\Delta_2\right) &\ge 0 & &\Leftrightarrow & \Im(a_0\conj{a_2})\Im(\Delta_2) &\le 0,
      \\
      \eqref{eq:unicite:ci3:csu3rd-D1} && \Re(\Delta_1) &= 0 & &\Leftrightarrow & \Re(a_1) - \Re(a_0\conj{b_1}) &= 0,
      \\
      \eqref{eq:unicite:ci3:csu3rd-D2} && \Re(\Delta_2) &= 0 & &\Leftrightarrow & \Re(a_2) - \Re(a_0\conj{b_2}) &= 0,
      \\
      \eqref{eq:unicite:ci3:csu3rd-T} && \Re(\Theta) &= 0 & &\Leftrightarrow & \Im(b_1)\Im(\Delta_2) - \Im(b_2)\Im(\Delta_1) &= 0,
      \\
      \eqref{eq:unicite:ci3:csu3rd-bD} && \Re(b_1\Delta_1) &\le 0 & &\Leftrightarrow & \Im(b_2)\Im(\Delta_2) &\ge 0.
    \end{align*}

\end{proof}


  Soit \(S = \left\lbrace (a_0,a_1,a_2,b_1,b_2) \in \CC^5 ; a_1 = a_2 ; b_1 = b_2 = 0 \right\rbrace \). On remarque que
  \begin{align}
    \CSU[1]{CI3} & \subset \CSU{CI4}\times\CC^2,
    \\
    \CSU[1]{CI3}\cap S & \subsetneq (\CSU{CI4}\times\CC^2)\cap S.
  \end{align}

  \begin{defn}
    \label{def:csu:ci3-2}

    On définit le sous-espace de \(\CC^5\)
    \begin{equation*}
      \CSU[2]{CI3} = \left\lbrace 
      \begin{aligned}
      &(a_0,a_1,a_2,b_1,b_2) \in \CC^5,
      \\
      &\begin{aligned}
        \Delta_1 &\not = 0,
        \\
        \Delta_2 &\not = 0,
        \\
        \Re\left(a_0\right)&\ge 0,
        \\
        \Re\left(a_1 - \frac{\conj{b_1a_0}a_1}{\Delta_1}\right) &\le 0,
        \\
        \Re\left(a_2 - \frac{\conj{b_2a_0}a_2}{\Delta_2}\right) &\le 0,
        \\
        \Re\left(b_1\Delta_1\right) &= 0,
        \\
        \Re\left(b_2\Delta_2\right) &= 0,
        \\
        \Im\left(b_1\Delta_1\right)\Im(b_1)&\ge 0,
        \\
        \Im\left(b_2\Delta_2\right)\Im(b_2)&\ge 0,
        \end{aligned}
      \end{aligned}
      \right\rbrace.
    \end{equation*}
  \end{defn}
  % \begin{REM}
  %   Et là tu peux mettre le passage à la limite que je t'ai envoyé par mail.
  % \end{REM}
  % \begin{REP}
  %   Je l'ai perdu...
  % \end{REP}

 \begin{prop}[Une deuxième CSU pour la CI3]
    \label{prop:csu:ci3-2}
    \begin{equation*}
      (a_0,a_1,a_2,b_1,b_2) \in \CSU[2]{CI3} \Rightarrow \Re(X) \ge 0,
    \end{equation*}
    ce qui entraîne l'unicité de la solution du problème de Maxwell extérieur avec CI3.
  \end{prop}

  \begin{proof}
    En se basant sur la démonstration de la \CSU[1]{CI3}, on remarque que l'on peut déterminer les quantités \((Y_R,Z_R)\) (resp. \((Y_D,Z_R)\)) uniquement en utilisant les équations \eqref{eq:unicite:ci3:csu3-3} et \eqref{eq:unicite:ci3:csu3-4} (resp. \eqref{eq:unicite:ci3:csu3-5} et \eqref{eq:unicite:ci3:csu3-6}).

    On déduit donc que si \(\Delta_1 \not = 0\) et \(\Delta_2 \not = 0\), alors

    \begin{align*}
      Y_R &= \frac{1}{\Delta_2}\left(a_2\left(\conj{a_0}\int_\Gamma \vJ\cdot\LR\conj{\vJ} - \conj{a_2}\norm{\LR \vJ}^2\right)  -\conj{b_2}\left(\int_\Gamma \conj{\vE}\cdot\LR{\vE} - b_2 \norm{\LR \vE}^2\right)\right), \\
      Y_D &= \frac{1}{\Delta_1}\left(a_1\left(\conj{a_0}\int_\Gamma \vJ\cdot\LD\conj{\vJ} + \conj{a_1}\norm{\LD \vJ}^2\right)  -\conj{b_1}\left(\int_\Gamma \conj{\vE}\cdot\LD{\vE} + b_1 \norm{\LD \vE}^2\right)\right).
    \end{align*}

    Utilisons \eqref{eq:unicite:ci3:csu3-1}, on obtient
    \begin{equation*}
      X = -\conj{b_1} Y_D + \conj{b_2} Y_R + \conj{a_0} \norm{\vJ}^2 + \conj{a_1} \int_\Gamma \vJ \cdot \LD \conj{\vJ} - \conj{a_2} \int_\Gamma \vJ \cdot \LR \conj{\vJ}.
    \end{equation*}

    On développe les résultats de la démonstration précédente,
    \begin{multline*}
      X = \conj{a_0} \norm{\vJ }^2 - \conj{a_1} \norm{\vdivs \vJ }^2 - \conj{a_2} \norm{\vrots \vJ }^2
      \\
      + \frac{\conj{b_2}}{\Delta_2}\left(a_2\left(\conj{a_0}\norm{\vrots \vJ}^2 - \conj{a_2}\norm{\LR J}^2\right)  -\conj{b_2}\left(\norm{\vrots\vE}^2 - b_2 \norm{\LR \vE }^2\right)\right)
      \\
      - \frac{\conj{b_1}}{\Delta_1}\left(a_1\left(-\conj{a_0}\norm{\vdivs\vJ}^2 + \conj{a_1}\norm{\LD J}^2\right)  -\conj{b_1}\left(-\norm{\vdivs\vE}^2 + b_1 \norm{\LD \vE }^2\right)\right).
    \end{multline*}

    On factorise les termes en \(\norm{\vJ}^2\), \(\norm{\vdivs\vJ}^2\),  \(\norm{\vrots\vJ}^2\)
    \begin{multline*}
      X = \conj{a_0} \norm{\vJ }^2 - \left(\conj{a_1} - \frac{\conj{b_1a_0}a_1}{\Delta_1}\right) \norm{\vdivs \vJ}^2 - \left(\conj{a_2} - \frac{\conj{b_2a_0}a_2}{\Delta_2}\right) \norm{\vrots \vJ }^2
      \\
      + \frac{\conj{b_2}}{\Delta_2}\left( - |a_2|^2\norm{\LR \vJ}^2  - \conj{b_2}\left(\norm{\vrots\vE}^2 - b_2 \norm{\LR \vE }^2\right)\right) 
      \\
      - \frac{\conj{b_1}}{\Delta_1}\left( |a_1|^2\norm{\LD \vJ}^2  - \conj{b_1}\left(-\norm{\vdivs\vE}^2 + b_1 \norm{\LD \vE }^2\right)\right).
    \end{multline*}

    On développe l'expression précédente
    \begin{multline*}
      X = \conj{a_0} \norm{\vJ }^2 - \left(\conj{a_1} - \frac{\conj{b_1a_0}a_1}{\Delta_1}\right) \norm{\vdivs \vJ }^2 - \left(\conj{a_2} - \frac{\conj{b_2a_0}a_2}{\Delta_2}\right) \norm{\vrots \vJ }^2
      \\
      - \frac{\conj{b_2}|a_2|^2}{\Delta_2}\norm{\LR \vJ}^2  -  \frac{\conj{b_2}^2}{\Delta_2}\norm{\vrots\vE}^2 +  \frac{\conj{b_2}|b_2|^2}{\Delta_2} \norm{\LR \vE }^2
      \\
      - \frac{\conj{b_1}|a_1|^2}{\Delta_1}\norm{\LD \vJ}^2  - \frac{\conj{b_1}^2}{\Delta_1}\norm{\vdivs\vE}^2 + \frac{\conj{b_1}|b_1|^2}{\Delta_1} \norm{\LD \vE }^2.
    \end{multline*}

    Comme condition suffisante, on impose à la partie réelle de chaque terme d'être positive, ce qui s'écrit
    \begin{align*}
      \Re\left(a_0\right)\ge 0, && 
      \\
      \Re\left(\conj{a_1} - \frac{\conj{b_1a_0}a_1}{\Delta_1}\right) \le 0, && \Re\left(\frac{\conj{b_1}^2}{\Delta_1}\right) \le 0,
      \\
      \Re\left(\conj{a_2} - \frac{\conj{b_2a_0}a_2}{\Delta_2}\right) \le 0, && \Re\left(\frac{\conj{b_2}^2}{\Delta_2}\right) \le 0,
      \\
      \Re\left(\frac{|a_1|^2\conj{b_1}}{\Delta_1}\right) \le 0, && \Re\left(\frac{|b_1|^2\conj{b_1}}{\Delta_1}\right) \ge 0,
      \\
      \Re\left(\frac{|a_2|^2\conj{b_2}}{\Delta_2}\right) \le 0, && \Re\left(\frac{|b_2|^2\conj{b_2}}{\Delta_2}\right) \ge 0.
    \end{align*}


    On remarque alors que les conditions des deux dernières lignes se combinent et imposent aux parties réelles de \(b_1\Delta_1\) et \(b_2\Delta_2\) d'être nulles.
  \end{proof}

  Soit \(S = \left\lbrace (a_0,a_1,a_2,b_1,b_2) \in \CC^5 ; a_1 = a_2 ; b_1 = b_2 = 0 \right\rbrace \). On remarque que
  \begin{align}
    \CSU[2]{CI3} & \subset \CSU{CI4}\times\CC^2,
    \\
    \CSU[2]{CI3}\cap S & \subsetneq (\CSU{CI4}\times\CC^2)\cap S. 
  \end{align}

    On définit 
    \begin{equation}
      \label{eq:fonction:z-ci3}
      \fonction{z}{\CC\times\CC^*\times\CC^*\times \CC \times \CC}{\CC}%
        {(a_0,a_1,a_2,b_1,b_2)}{1 - \frac{b_1a_0}{a_1} - \frac{b_2a_0}{a_2}.}
    \end{equation}
    Par abus de notation, on omet les variables \( (a_0,a_1,a_2,b_1,b_2)\)
    \begin{equation}
       z(a_0,a_1,a_2,b_1,b_2) \equiv z.
    \end{equation}
  \begin{defn}
    \label{def:csu:ci3-3}

    On définit le sous-espace  de \(\CC^5\)
    \begin{equation*}
      \CSU[3]{CI3} = \left\lbrace
      \begin{aligned}
        &(a_0,a_1,a_2,b_1,b_2) \in \CC^5,
        \\
        &\begin{aligned}
          &a_1 &\not = 0,
          \\
          &a_2 &\not = 0,
          \\
          &\Re\left(\conj{a_0}z\right) &\ge 0,
          \\
          &\Re\left(\conj{a_1}z\right) &\le 0,
          \\
          &\Re\left(\conj{a_2}z\right) &\le 0,
          \\
          &\Re\left(\frac{b_1}{a_1}\right) &\ge 0,
          \\
          &\Re\left(\frac{b_2}{a_2}\right) &\ge 0,
          \\
          &\Re\left(a_0\right) &\ge 0,
          \\
          &\Re\left(a_1\right) &\le 0,
          \\
          &\Re\left(a_2\right) &\le 0,
          \\
          &\Re\left(\frac{b_1\conj{a_2}}{a_1\conj{a_0}}\right) &\le 0,
          \\
          &\Re\left(\frac{b_2\conj{a_1}}{a_2\conj{a_0}}\right) &\le 0,
        \end{aligned}
      \end{aligned}
      \right\rbrace.
    \end{equation*}
  \end{defn}

  
  \begin{prop}[Une troisième CSU pour la CI3]
    \label{prop:csu:ci3-3}
    \begin{equation*}
      (a_0,a_1,a_2,b_1,b_2) \in \CSU[3]{CI3} \Rightarrow \Re(X) \ge 0,
    \end{equation*}
    ce qui entraîne l'unicité du problème \{\eqref{eq:unicite:probleme_sans_ci},\eqref{eq:unicite:ci3:ci3}\}.
  \end{prop}

  \begin{proof}
    Supposons l'opérateur \(a_0\oI + a_1 \LD - a_2 \LR\) inversible.
    % \begin{REM}
    %   Sur \(\Sobolev[1]\) ? Précisez les espaces. Il faut préciser les espaces, de plus il y a un autre problème. Prenons P l'opérateur
    %   \(Pf=f''+af\). \(Pf=g\) a une infinité de solutions: si on en a une, alors en ajoutant  \(f_0\) telle que \(f_0''+a_f0 =0\), c'est à dire \(f_0=Ae^{\sqrt{-a}t}+Be^{-\sqrt{-a}t}\) alors comment ????????-tu \(f_0\)
    % \end{REM}
    % \begin{REP}
    %   On a vu ça ensemble à Montréal: je ne sais pas dans quel espace me placer.
    % \end{REP}
   Alors,

    \begin{align*}
      X &= \int_\Gamma \left(a_0\oI + a_1 \LD - a_2 \LR \right)^{-1}\left( \oI + b_1 \LD - b_2 \LR \right) \vE_t\cdot \conj{\vE_t.}
    \end{align*}

    On développe chaque terme

    \begin{multline*}
      X = \int_\Gamma \left(\left(a_0\oI + a_1 \LD - a_2 \LR \right)^{-1}
      \right.
      \\
      + b_1 \left(a_0\oI + a_1 \LD - a_2 \LR \right)^{-1}\LD
      \\
      \left.
      - b_2 \left(a_0\oI + a_1 \LD - a_2 \LR \right)^{-1}\LR \right) \vE_t\cdot \conj{\vE_t.}
    \end{multline*}

    On suppose \(a_1\not=0\) et \(a_2\not=0\), alors
    \begin{align*}
      \LD & = \frac{1}{a_1}\left(a_0\oI + a_1 \LD - a_0\oI\right),
      \\
      \LR & = -\frac{1}{a_2}\left(a_0\oI - a_2 \LR - a_0\oI\right).
    \end{align*}


    On déduit de ce qui précède que

    \begin{multline*}
      X = \int_\Gamma \left(1-\frac{b_1}{a_1} -\frac{b_2}{a_2}\right)\left(a_0\oI + a_1 \LD - a_2 \LR \right)^{-1}
      \\
      + \frac{b_1}{a_1} \left(a_0\oI + a_1 \LD - a_2 \LR \right)^{-1}\left(a_0\oI + a_1\LD\right)
      \\
      + \frac{b_2}{a_2} \left(a_0\oI + a_1 \LD - a_2 \LR \right)^{-1}\left(a_0\oI - a_2\LR\right) \vE_t\cdot \conj{\vE_t.}
    \end{multline*}

    On définit

    \newcommand{\vV}{\vect{V}}
    \newcommand{\vW}{\vect{W}}

    \begin{align*}
      z &= 1-\frac{b_1}{a_1}a_0 -\frac{b_2}{a_2}a_0,
      \\
      \vV & = \left(a_0\oI  + a_1 \LD - a_2\LR \right)^{-1} \vE_t,
      \\
      \vW_1 & = \left( \oI - a_2 \left( a_0\oI + a_1\LD\right)^{-1}\LR\right)^{-1} \vE_t,
      \\
      \vW_2 & = \left( \oI + a_1 \left( a_0\oI - a_2\LR\right)^{-1}\LD\right)^{-1} \vE_t.
    \end{align*}

    On remarque que
    \begin{align*}
      \left( \oI - a_2 \left( a_0\oI + a_1\LD\right)^{-1}\LR\right)\vW_1 &= \vE_t,
      \\
      \left( a_0\oI + a_1 \LD \right)\left( \oI - a_2 \left( a_0\oI + a_1\LD\right)^{-1}\LR\right)\vW_1&= \left( a_0\oI + a_1 \LD \right)\vE_t,
      \\
      \left( a_0\oI + a_1 \LD - a_2 \LR\right)\vW_1 &= \left( a_0\oI + a_1 \LD \right)\vE_t,
    \end{align*}
    \begin{align*}
      \left( \oI + a_1 \left( a_0\oI - a_2\LR\right)^{-1}\LD\right)\vW_2 &= \vE_t,
      \\
      \left( a_0\oI - a_2 \LR \right)\left( \oI + a_1 \left( a_0\oI - a_2\LR\right)^{-1}\LD\right)\vW_2&= \left( a_0\oI - a_2 \LR \right)\vE_t,
      \\
      \left( a_0\oI + a_1 \LR - a_2 \LR\right)\vW_2 &= \left( a_0\oI - a_2 \LR \right)\vE_t.
    \end{align*}

    On déduit

    \begin{multline*}
      X = \int_\Gamma z \vV \cdot \left(\conj{a_0}\oI  + \conj{a_1} \LD - \conj{a_2}\LR\right)\conj{\vV}
      \\
      + \frac{b_1}{a_1} \left( \oI - \conj{a_2} \left( \conj{a_0}\oI + \conj{a_1}\LD\right)^{-1}\LR\right)\conj{\vW_1}\cdot\vW_1
      \\
      + \frac{b_2}{a_2} \left( \oI + \conj{a_1} \left( \conj{a_0}\oI - \conj{a_2}\LR\right)^{-1}\LD\right)\conj{\vW_2}\cdot\vW_2.
    \end{multline*}

    Notons

    \newcommand{\vR}{\vect{R}}

    \begin{align*}
      \vR_1 & = \left(\conj{a_0}\oI  + \conj{a_1} \LD \right)^{-1}\LR \conj{\vW_1},
      \\
      \vR_2 & = \left(\conj{a_0}\oI  - \conj{a_2} \LR \right)^{-1}\LD \conj{\vW_2}.
    \end{align*}

    Donc 
    \begin{equation*}
      X = \int_\Gamma z \vV \cdot \left(\conj{a_0}\oI  + \conj{a_1} \LD - \conj{a_2}\LR\right)\conj{\vV} + \frac{b_1}{a_1} \norm{\vW_1} +\frac{b_2}{a_2} \norm{\vW_2} - \frac{b_1}{a_1}\conj{a_2}\vR_1\cdot\vW_1 + \frac{b_2}{a_2}\conj{a_1}\vR_2\cdot\vW_2.
    \end{equation*}

    Comme
    \begin{align*}
      \LD\left(\conj{a_0}\oI  + \conj{a_1} \LD \right)&=\left(\conj{a_0}\oI  + \conj{a_1} \LD \right)\LD,
      \\
      \LR\left(\conj{a_0}\oI  - \conj{a_2} \LR \right)&=\left(\conj{a_0}\oI  - \conj{a_2} \LR \right)\LR,
    \end{align*}

    et que \(\LD\LR=\LR\LD=0\), on trouve

    \begin{equation*}
      \begin{aligned}
        \LD\LR\conj{\vW_1} &= \LD\left(\conj{a_0}\oI  + \conj{a_1} \LD \right)\vR_1,
        \\
        0 & =\left(\conj{a_0}\oI  + \conj{a_1} \LD \right)\LD\vR_1,
      \end{aligned}
    \end{equation*}
    \begin{equation*}
      \begin{aligned}
        \LR\LD\conj{\vW_2} &= \LR\left(\conj{a_0}\oI  - \conj{a_2} \LR \right)\vR_2,
        \\
        0 & =\left(\conj{a_0}\oI  - \conj{a_2} \LR \right)\LR\vR_2.
      \end{aligned}
    \end{equation*}

    % Pour conclure, il faut utiliser le résultat suivant
    % \begin{prop}[Injectivité]
    %   On suppose que
    %   \begin{align*}
    %     \Re(a_0)\ge0 && \Re(a_1) \le 0 && \Re(a_2) \le 0
    %   \end{align*}
    %   alors \(a_0\oI + a_1 \LD\)  et \(a_0\oI - a_2 \LR\) sont injectif
    % \end{prop}

    % \begin{proof}
    %   Par définition, \(a_0\oI + a_1 \LD\) est injectif si pour \(\vect{U}\) vecteur régulier tangent  à  \(\Gamma\)
    %   \begin{align*}
    %     \int_\Gamma \left(a_0\oI + a_1 \LD\right)\vect{U}\cdot\conj{\vect{U}} &= 0 \Rightarrow \vect{U} = 0
    %     \intertext{Or par définition de l'opérateur \(\LD\)}
    %     \int_\Gamma \left(a_0\oI + a_1 \LD\right)\vect{U}\cdot\conj{\vect{U}} &=a_0\norm{\vect{U}}^2 - a_1\norm{\vdivs{\vect{U}}}^2
    %   \end{align*}
      
    %   Prenons la partie réelle de cette égalité et utilisons les hypothèses sur les coefficients. Alors le membre de droite ne contient que des termes positifs, donc tous ces termes sont nuls, donc \(\norm{\vect{U}} = 0\) et \(\norm{\vdivs\vect{U}} = 0\) donc \(\vect{U} = 0\).

    %   Le même raisonnement est valable pour \(a_0\oI - a_2 \LR\).
    % \end{proof}

    % On utilise les propositions \ref{prop:unicite:injectif:opérateur:LD},\ref{prop:unicite:injectif:opérateur:LR},\ref{prop:unicite:injectif:opérateur:L}. On suppose donc \(\Re(a_0) \ge 0 \),\(\Re(a_1) \le 0\) et \(\Re(a_2)\le0\)), donc \(\left(\conj{a_0}  + \conj{a_1} \LD \right)\) et \(\left(\conj{a_0}  - \conj{a_2} \LR \right)\) sont injectifs et donc on déduit que

    % \begin{align*}
    %   \left(\conj{a_0}\oI  + \conj{a_1} \LD \right)\LD\vR_1 = 0 &\Rightarrow \LD\vR_1 = 0,
    %   \\
    %   \left(\conj{a_0}\oI  - \conj{a_2} \LR \right)\LR\vR_2 = 0 &\Rightarrow \LR\vR_2 = 0
    % \end{align*}

    Donc on déduit que \(\LD\vR_1 = 0\), \(\LR\vR_2 = 0\).

    Or par définition \(\LR\conj{\vW_1} = \left(\conj{a_0}  + \conj{a_1} \LD \right)\vR_1\) et \(\LD\conj{\vW_2} = \left(\conj{a_0}  - \conj{a_2} \LR \right)\vR_2\), donc
    \begin{align*}
      \LR\conj{\vW_1} = \conj{a_0}\vR_1, && \LD\conj{\vW_2} = \conj{a_0}\vR_2,
      \intertext{donc}
      \vR_1 = \frac{1}{\conj{a_0}}\LR\conj{\vW_1}, && \vR_2 = \frac{1}{\conj{a_0}}\LD\conj{\vW_2}.
    \end{align*}

    On réinjecte ce résultat dans la définition de \(X\).

    \begin{multline*}
      X = \int_\Gamma z \vV \cdot \left(\conj{a_0}\oI  + \conj{a_1} \LD - \conj{a_2}\LR\right)\conj{\vV}
      \\
      + \frac{b_1}{a_1} \norm{\vW_1}^2 - \frac{b_1\conj{a_2}}{a_1\conj{a_0}}\int_\Gamma \LR\conj{\vW_1}\cdot\vW_1
      \\
      + \frac{b_2}{a_2} \norm{\vW_2}^2 + \frac{b_2\conj{a_1}}{a_2\conj{a_0}}\int_\Gamma \LD\conj{\vW_2}\cdot\vW_2.
    \end{multline*}

    On impose la positivité de la partie réelle de chaque terme pour avoir \(\Re(X)\ge 0\).
  \end{proof}

  Soit \(S = \left\lbrace (a_0,a_1,a_2,b_1,b_2) \in \CC^5 ; a_1 = a_2 ; b_1 = b_2 = 0 \right\rbrace \). On remarque que
  \begin{align}
    \CSU[3]{CI3} & \subset \CSU{CI4}\times\CC^2,
    \\ 
    \CSU[3]{CI3}\cap S & = (\CSU{CI4}\times\CC^2)\cap S. 
  \end{align}


\section{Une condition nécessaire et suffisante pour l'unicité des solutions du problème intérieur}
  \subsection{Cas général: alternative de Fredholm}

    Pour exprimer l'opérateur d'impédance, nous devons exprimer les champs dans l'objet. Il apparaît nécessaire que ces derniers soient donc bien définis.

    Nous exhibons une condition nécessaire et suffisante d'unicité des solutions du problème de Maxwell-Helmholtz  \eqref{eq:unicite:maxwell_int} dans \(\OO\) est un objet multicouche autour d'un \gls{acr-cep} c'est-a-dire tel que \(\partial \OO = \Gamma \cup \Gamma_0\) où \(\Gamma\) est la surface extérieure et \(\Gamma_0\) la surface intérieure. Le problème de Maxwell harmonique en \(e^{i\w t}\) s'exprime

    \begin{align}
    \left\lbrace
      \begin{matrix}
        \vrot \vE(\vx) + i \w\mu(\vx) \vH(\vx) &= 0
        \\
        \vrot \vH(\vx) - i \w\eps(\vx) \vE(\vx) &= 0
      \end{matrix}
      \right. && \text{dans \(\OO\).}
      \label{eq:unicite:maxwell_int}
    \end{align}

    Nous rappelons le résultat général de \cite[Théorème~8, p.~111]{cessenat_mathematical_1996}.
    \begin{thm}[Alternative de Fredholm]
      Les champs \(\vE,\vH\) solutions de \eqref{eq:unicite:maxwell_int} avec conditions aux limites de Dirichlet \(\vE_{|\Gamma} = 0\), \(\vE_{|\Gamma_0} = 0\) sont soient uniques si \(\w^2\eps\mu\) n'est pas une valeur propre du Laplacien vectoriel sinon ces champs sont engendrés par les vecteurs propres.
    \end{thm}
    Nous renvoyons à l'ouvrage cité précédemment pour la démonstration, en dehors du cadre de cette thèse. Ce que nous tirons de ce résultat est que pour le problème intérieur, nous pouvons avoir des résonances, c'est-à-dire des solutions non nulles du problème sans sources donc non-uniques.

  \subsection{Cas particuliers de géométries particulières}
    Cette thèse s'attache à déterminer les coefficients des \glspl{acr-cioe} des sections précédentes. Nous devons être capables d'exprimer l'opérateur d'impédance exact pour l'approcher par les CIOE. Il faut donc que les champs soient déterminés de façon unique pour exprimer cette opérateur. Nous ne pouvons nous satisfaire de l'alternative de Fredholm et devons exprimer une condition plus pratique pour déterminer s'il y a unicité. Nous allons alors approcher l'objet par une forme simple mais avec les mêmes caractéristiques pour l'empilement. Ainsi nous étudierons 3 approximations
    \begin{enumerate}
      \item Un objet plan infini 
      \item Un objet cylindrique
      \item Un objet sphérique
    \end{enumerate}
   
    L'alternative de Fredholm n'étant pas utilisable en pratique, nous allons simplifier la géométrie du problème pour aboutir à une condition nécessaire d'unicité.

    L'objet est représenté schématiquement:
    \begin{figure}[h!btp]
        \centering
        \tikzsetnextfilename{plan_n_couches}            
        \begin{tikzpicture}
            \tikzmath{
    \largeur = 6;
    \hauteur = 0.5;
    \milieu = 1.3;
    \xC = \largeur;
    \xA = 0;
}

%% 1ere couche
\tikzmath{
    \yC = \hauteur;
    \yA = 0;
}

\coordinate (A) at (\xA,\yA);
\coordinate (B) at (\xA,\yC);
\coordinate (C) at (\xC,\yC);

\draw ($(B)!0.5!(C)$) node [above] {vide};


\fill [lightgray] (A) rectangle (C);
\draw ($(A)!0.5!(C)$) node {$\eps_n,\mu_n,d_n$};
\draw (B) -- (C) node [right] {$e_3 = 0$};

%% Des couches
\tikzmath{
    \yC = \yC - \hauteur;
    \yA = \yA - \milieu*\hauteur;
}

\coordinate (A) at (\xA,\yA);
\coordinate (B) at (\xA,\yC);
\coordinate (C) at (\xC,\yC);

\fill [lightgray]    (A) rectangle (C);
\fill [pattern=dots] (A) rectangle (C);
\draw (B) -- (C);

%% N ieme couche
\tikzmath{
    \yC = \yC - \milieu*\hauteur;
    \yA = \yA - \hauteur;
}

\coordinate (A) at (\xA,\yA);
\coordinate (B) at (\xA,\yC);
\coordinate (C) at (\xC,\yC);
\fill [lightgray] (A) rectangle (C);
\draw ($(A)!0.5!(C)$) node {$\eps_1,\mu_1,d_1$};
\draw (B) -- (C);

%% Le repère
\tikzmath{
    \xD = \xC + 0.5;
}

\coordinate (n) at (\xD,\yA);
\draw [->] (n) -- ++(1,0) node [at end, right] {$\v{e_1}$};
\draw [->] (n) -- ++(0,1) node [at end, right] {$\v{e_3}$};

\draw (n) circle(0.1cm) node [below=0.1cm] {$\v{e_2}$};
\draw (n) +(135:0.1cm) -- +(315:0.1cm);
\draw (n) +(45:0.1cm) -- +(225:0.1cm);

%% Le conducteur
\tikzmath{
    \yC = \yC - \hauteur;
    \yA = \yA - 0.5*\hauteur;
}

\coordinate (A) at (\xA,\yA);
\coordinate (B) at (\xA,\yC);
\coordinate (C) at (\xC,\yC);
\draw (B) -- (C);

\fill [pattern=north east lines] (A) rectangle (C);



        \end{tikzpicture}
    \end{figure}

    Les constantes relatives \(\eps,\mu\) (ou \(k,\eta\)) sont constantes par morceaux en fonction de \(z\).
    \begin{prop}
      \label{prop:unicite:interieur:postulat:multi-couche}
      Le problème Maxwell-Helmholtz \eqref{eq:unicite:maxwell_int} avec conditions aux limites de Dirichlet \(\vE_{|z=z_n} = 0\), \(\vE_{|z=z_0} = 0\) admet une unique solution si est seulement si le déterminant d'un système linéaire est non-nul.
    \end{prop}

    \begin{proof}
      Nous ne traiterons dans cette thèse que l'exemple de deux couches de matériaux.
      Nous supposons qu'il existe un champ \(\vE\) solution du problème de Maxwell-Helmholtz \eqref{eq:unicite:maxwell_int} avec conditions limite de Dirichlet sur les bords du domaine. La démonstration de l'existence de ces champs est l'objet de la partie suivante.

      \newcommand{\kk}{\tilde{k}}

      On pose \(\kk_1 = \sqrt{k_1^2 - k_x^2 - k_y^2}\),  \(\kk_2 = \sqrt{k_2^2 - k_x^2 - k_y^2}\). On a formellement,
      \begin{align*}
        \left\lbrace
        \begin{aligned}
          \hat{E}_x &= \sin(\kk_2(z-z_1))a_2 +  \cos(\kk_2(z-z_1))b
          \\
          \hat{E}_y &= \sin(\kk_2(z-z_1))c_2 +  \cos(\kk_2(z-z_1))d
        \end{aligned}
        \right. && z_1 < z < z_2,
        \\
        \left\lbrace
        \begin{aligned}
          \hat{E}_x &= \sin(\kk_1(z-z_1))a_1 +  \cos(\kk_1(z-z_1))b
          \\
          \hat{E}_y &= \sin(\kk_1(z-z_1))c_1 +  \cos(\kk_2(z-z_1))d
        \end{aligned}
        \right. && z_0 < z < z_1.
      \end{align*}
      Ces relations ne sont valables que pour des constantes \(a_2,c_2,a_1,c_1,b,d\) telles que \((\hat{E}_x,\hat{E}_y)\) soient dans \(\mathcal{S}'(\RR^3)\).
      La composante en z de \(\vE\) se déduit des deux autres car sa divergence est nulle.

      On déduit le champ \(\vH\) alors \(\vH(\vx) = \frac{-\vrot{\vE}(\vx)}{i k(\vx)\eta(\vx)}\). On note \(\mLR\) la matrice \(
      \begin{bmatrix}
          k_y^2 & k_xk_y
          \\
          k_xk_y & k_x^2
      \end{bmatrix}
      \).


      Le champ s'exprime matriciellement
      \begin{align*}
        \hat{\vH} &= \frac{\sin(\kk_2(z-z_1))}{-i k_2\kk_2 \eta_2}(\kk_2^2 \mI - \mLR)\begin{bmatrix}d \\ b\end{bmatrix} - \frac{\cos(\kk_2(z-z_1))}{-i k_2\kk_2 \eta_2}(\kk_2^2 \mI - \mLR)\begin{bmatrix}c_2 \\ a_2\end{bmatrix} && \text{pour } z_1 < z < z_2,
        \\
        \hat{\vH} &= \frac{\sin(\kk_1(z-z_1))}{-i k_1\kk_1 \eta_1}(\kk_1^2 \mI - \mLR)\begin{bmatrix}d \\ b\end{bmatrix} - \frac{\cos(\kk_1(z-z_1))}{-i k_1\kk_1 \eta_1}(\kk_1^2 \mI - \mLR)\begin{bmatrix}c_1 \\ a_1\end{bmatrix} && \text{pour } z_0 < z < z_1.
      \end{align*}

      Pour que le problème soit bien posé, il faut et il suffit que les constantes complexes \(a_2,c_2,b,d,a_1,c_1\) soient déterminées de façon unique quand on exprime les conditions limites. Ce sont des conditions limites de Dirichlet en \(z=z_0\) et \(z=z_2\) pour \(\vE\) et de sauts nuls en \(z=z_1\) pour \(\vH\) (la condition de saut nul pour \(\vE\) a déjà été utilisé dans l'expression des champs).

      On a unicité si dans le cas de conditions de Dirichlet homogènes, l'unique solution est \(\vE=0\). Ces conditions s'expriment
      \begin{equation*}
        \label{eq:unicite:interieur:2couches:cldirichlet}
        \left\lbrace
        \begin{aligned}
          a_2\sin{(\kk_2(z_2-z_1))} + b\cos{(\kk_2(z_2-z_1))} &= 0
          \\
          c_2\sin{(\kk_2(z_2-z_1))} + d\cos{(\kk_2(z_2-z_1))} &= 0
          \\
          a_1\sin{(\kk_1(z_0-z_1))} + b\cos{(\kk_1(z_0-z_1))} &= 0
          \\
          c_1\sin{(\kk_1(z_0-z_1))} + d\cos{(\kk_1(z_0-z_1))} &= 0
        \end{aligned}
        \right.
      \end{equation*}

      C'est un système à 4 équations pour 6 inconnues, donc à deux degrés de liberté. Il existe donc \(A,B\) tels que
      \begin{equation*}
        \left\lbrace
        \begin{aligned}
        a_2 &= A\cos{(\kk_2(z_2-z_1))}\sin{(\kk_1(z_0-z_1))}
        \\
        b &= -A\sin{(\kk_2(z_2-z_1))}\sin{(\kk_1(z_0-z_1))}
        \\
        a_1 &= A\sin{(\kk_2(z_2-z_1))}\cos{(\kk_1(z_0-z_1))}
        \\
        c_2 &= B\cos{(\kk_2(z_2-z_1))}\sin{(\kk_1(z_0-z_1))}
        \\
        d &= -B\sin{(\kk_2(z_2-z_1))}\sin{(\kk_1(z_0-z_1))}
        \\
        c_1 &= B\sin{(\kk_2(z_2-z_1))}\cos{(\kk_1(z_0-z_1))}
        \end{aligned}
        \right.
      \end{equation*}
      soient solutions du système précédent. 

      La dernière condition de saut nul en \(z=z_1\) pour \(\hat\vH_t\) s'exprime alors
      \begin{align*}
         \cos(\kk_2(z_2-z_1)) \sin(\kk_1(z_0-z_1))\frac{i}{k_2\kk_2\eta_2}(\kk_2^2 \mI - \mLR)\begin{bmatrix}B\\A\end{bmatrix}
        \\= 
         \sin(\kk_2(z_2-z_1)) \cos(\kk_1(z_0-z_1))\frac{i}{k_1\kk_1\eta_1}(\kk_1^2 \mI - \mLR)\begin{bmatrix}B\\A\end{bmatrix}
      \end{align*}

      On voit alors apparaître le système final dont le déterminant est
      \begin{multline*}
        -\det{}\left(
        \frac{\cos(\kk_2(z_2-z_1)) \sin(\kk_1(z_0-z_1))}{k_2\kk_2\eta_2}(\kk_2^2 \mI - \mLR)
        \right.
        \\
        \left.-
         \frac{\sin(\kk_2(z_2-z_1)) \cos(\kk_1(z_0-z_1))}{k_1\kk_1\eta_1}(\kk_1^2 \mI - \mLR)
         \right)
      \end{multline*}

      \newcommand{\alp}{\frac{\cos(\kk_2(z_2-z_1)) \sin(\kk_1(z_0-z_1))}{k_2\kk_2\eta_2}}
      \newcommand{\bet}{\frac{\sin(\kk_2(z_2-z_1)) \cos(\kk_1(z_0-z_1))}{k_1\kk_1\eta_1}}

      \begin{multline*}
        =-\left(\alp \kk_2^2 - \bet \kk_1^2\right)
        \\
        \left(\alp(\kk_2^2-k_y^2)-\bet(\kk_1^2-k_x^2)\right)
      \end{multline*}


      Si le déterminant de ce système n'est pas nul, alors il existe des champs non nuls solutions du problème homogène, et alors il n'y a pas unicité des solutions.

    \end{proof}


\sectionstar{Conclusion}
Nous avons réussi à fournir pour plusieurs CIOE des CSU qui permettent de garantir l'unicité des solutions du problème de Maxwell.
Nous avons vu que ces conditions étaient non-linéaires, s'exprimaient comme des inégalités, des égalités, mais aussi par différences (ex \(z\not=0\)).
Par contre, elles sont indépendantes de la géométrie et de la fréquence.
De plus, nous sommes en mesure de calculer l'expression des champs et de l'opérateur d'impédance grâce au calcul en amont d'un déterminant.
Grâce à cette condition, le problème est bien posé.